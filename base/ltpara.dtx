% \iffalse meta-comment
%
%% File: ltpara.dtx (C) Copyright 2020-2021
%       Frank Mittelbach, LaTeX Team
%
% It may be distributed and/or modified under the conditions of the
% LaTeX Project Public License (LPPL), either version 1.3c of this
% license or (at your option) any later version.  The latest version
% of this license is in the file
%
%    https://www.latex-project.org/lppl.txt
%
%
%%% From File: ltpara.dtx
%
%    \begin{macrocode}
\def\ltparaversion{v1.0k}
\def\ltparadate{2022/05/13}
%    \end{macrocode}
%<*driver>
\documentclass{l3doc}

%\usepackage{ltpara}

% Fixing footnotes in  functions and variables: this should be in l3doc!

\newcommand\fixfootnote[2]{\footnotemark
  \AddToHookNext{env/#1/after}{\footnotetext{#2}}}
\AddToHook{env/function/begin}{\def\footnote{\fixfootnote{function}}}
\AddToHook{env/variable/begin}{\def\footnote{\fixfootnote{variable}}}

\EnableCrossrefs
\CodelineIndex
\begin{document}
  \DocInput{ltpara.dtx}
\end{document}
%</driver>
%
% \fi
%
% \providecommand\hook[1]{\texttt{#1}}
% \providecommand\env[1]{\texttt{#1}}
%
%
%
% \title{The \texttt{ltpara.dtx} code\thanks{This file has version
%    \ltparaversion\ dated \ltparadate, \copyright\ \LaTeX\
%    Project.}}
% \author{Frank Mittelbach}
%
% \maketitle
%
%
% \begin{abstract}
%    This code defines four special kernel hooks to support paragraph
%    tagging as well as four public hooks which can be occasionally
%    useful.
% \end{abstract}
%
% \tableofcontents
%
%
% \section{Introduction}
%
%
%    The building of paragraphs in the \TeX{} engine(s) has a number
%    of peculiarities that makes it on one hand fairly flexible but on
%    the other hand somewhat awkward to control or reliably to extend.
%    Thus to better understand the code below we start with a brief introduction
%    of the mechanism; for more details refer to the
%    \TeX{}book~\cite[chap.~14]{texbook} (for the full truth you may
%    even have to study the program code).
%
% \subsection{The default processing done by the engine}
%
%    \TeX{} automatically starts building a paragraph when it is
%    currently in vertical mode and encounters anything that can only
%    live in horizontal mode. Most often this is a character, but
%    there are also many commands that can be used only in horizontal mode.
%    If any of them is encountered, \TeX{} will immediately back up
%    (i.e., the character or command is read later again), adds a
%    \cs{parskip} glue to the current vertical list unless the list is
%    empty, switches to
%    horizontal mode, starts its special \enquote{start of paragraph
%    processing} and only then rereads the character or command that
%    caused the mode change.\footnote{Already not quite true: the
%    command \cs{noindent} starts the paragraph but influences the
%    special processing by suppressing the paragraph indentation box
%    normally inserted by it.}
%
%    This \enquote{start of paragraph
%    processing} first adds an empty box at the start of the
%    horizontal list of width \cs{parindent} (which represents the
%    paragraph indentation) unless the paragraph was started with
%    \cs{noindent} in which case no such box is
%    added\footnote{That's a bit different from placing a zero-sized
%    box!}. It then reads and processes all tokens stored in the
%    special engine token register \cs{everypar}. After that it reads
%    and processes whatever has caused the paragraph to start.
%
%    Thus out of the box, \TeX{} offers the possibility to put some
%    special code into \cs{everypar} to gain control at (more or less)
%    the start of the paragraph. For example, in LaTeX{} and a number
%    of packages,  special code like the following is sometimes used:
%\begin{verbatim}
%  \everypar{{\setbox\z@\lastbox}\everypar{} ...}
%\end{verbatim}
%    This removes the paragraph indentation box again  (that was already
%    placed by \TeX), then resets \cs{everypar} so that it doesn't do
%    anything on the next paragraph start and then does whatever it
%    wants to do, e.g., in an \cs{item} of a list it will typeset the label in
%    front of the paragraph text.
%    However, there is only one such \cs{everypar} token register and
%    if different packages and/or the kernel all attempt to add their
%    own code here, coordination is very difficult if not impossible.
%
%    The process when the paragraph ends has different mechanisms and interfaces.
%    A paragraph ends when the engine  primitive \cs{par} is called
%    while \TeX{} is in unrestricted horizontal mode, i.e., is
%    building a paragraph. At other times this primitive does nothing
%    or generates as an error depending on the mode \TeX{} is in,
%    e.g., the \cs{par} in
%    \verb=\hbox{a\par b}= is ignored, but \verb=$a\par b$= would complain.
%
%    If this primitive ends the paragraph it does some special
%    \enquote{end of horizontal list} processing, then calls \TeX{}'s paragraph
%    builder; this breaks the horizontal list into lines and then these
%    lines are added as boxes to the enclosing vertical list and
%    \TeX{} returns to vertical mode.
%
%    This \cs{par} command can be given explicitly, but there are also
%    situations in which \TeX{} is generating it on the fly. Most
%    often this happens when \TeX\ encounters a blank line which is
%    automatically changed to a \cs{par} command which is then
%    executed. The other possibility is that \TeX{} encounters a
%    command which is incompatible with horizontal processing, e.g.,
%    \cs{vskip} (a request for adding vertical space). In such cases it
%    silently backs up, and inserts a \cs{par} in the hope that this
%    gets it out of horizontal mode and makes the vertical command
%    acceptable.

%    The important point to note here is that \TeX{} really inserts
%    the command with the name \cs{par}, which can be redefined. 
%    Thus, it may not have
%    its original \enquote{primitive} meaning and therefore may not end the
%    horizontal list and call the paragraph builder. This approach
%    offers some flexibility but also allows you  to easily produce a
%    \TeX{} document that loops forever, for example, the simple line
%\begin{verbatim}
%   A \let\par\relax \vskip
%\end{verbatim}
%    will start a horizontal list at \texttt{A}, redefines \cs{par},
%    then sees \cs{vskip} and inserts \cs{par} to end the
%    paragraph. But this now only runs \cs{relax} so nothing changes
%    and \cs{vskip} is read again, issues a \cs{par} which \ldots. In
%    short, it only takes a plain \TeX{} document with five tokens to run
%    forever (since no memory is consumed and therefore eventually
%    exhausted).
%
%    There is no way other than changing \cs{par} to gain control at
%    the end of a paragraph, i.e., there is no token list like
%    \cs{everypar} that is inserted.  Hence the only way to change the
%    default behavior is to modify the action that \cs{par} executes,
%    with similar issues as outlined before: different processes need
%    to ensure that they do not overwrite their modifications or
%    worse, think that the \cs{par} in front of them is the engine
%    primitive while in fact it has already been changed by other
%    code.
%
%    To make matters slightly worse there are a few places where
%    \TeX{} handles the situation differently (most likely for speed
%    reasons back when computers were much slower). If \TeX{} finds
%    itself in unrestricted horizontal mode at the end of building a
%    vertical box (for an \cs{insert}, \cs{vadjust} or 
%    executing the output routine code), it will finish the horizontal
%    list not by issuing a \cs{par} command (which would be consistent
%    with all other places) but by simply executing the primitive meaning
%    of \cs{par}, regardless of the actual definition that \cs{par} 
%    has at the time.
%
%    Thus, if you have carefully crafted a redefined \cs{par} to execute
%    some special actions at the end of a paragraph and you write
%    something like
%\begin{verbatim}
%   \vbox{Some paragraph ... text.}
%\end{verbatim}
%    you will find that your code does not get run for the last paragraph
%    in that box. \LaTeX{} avoids this problem, by making sure that
%    its boxes (such as \cs{parbox} or the \env{minipage}
%    environment, etc.) all internally add an explicit \cs{par} at the
%    end so that such code is run and \TeX{} finds itself in vertical
%    mode already without the need to start up the paragraph builder
%    internally. But, of course, this only works for boxes under direct
%    control of the \LaTeX{} kernel; if some package uses low-level
%    \cs{vbox}es without adding this precaution the \TeX{}
%    optimization kicks in and no special \cs{par} code is executed.
%
%    And there is another optimization that is painful: if a paragraph
%    is interrupted by a mathematical display, e.g., \verb=\[...\]= in
%    \LaTeX{} or \verb=$$...$$= in plain \TeX{}, then \TeX{} will
%    resume horizontal mode afterward, i.e., it will start to build 
%    a new horizontal
%    list without inserting an indentation box or \cs{everypar} at
%    that point.  However, if that list immediately ends with an
%    explicit or implicit \cs{par} then \TeX{} will simply throw away
%    this \enquote{null} paragraph and not do its usual \enquote{end
%    of horizontal list} processing, so this special case also needs to be
%    accounted for when introducing any extended processing.
%
%
%
% \section{The new mechanism implemented for \LaTeX{}}
%
%    To improve the situation (and also to support automatic tagging
%    of PDF documents) we now offer public as well as private hooks at
%    the start and end of the paragraph processing. The public hooks
%    can be used by packages (or by the user in the preamble or
%    within the document) and using the hook mechanisms it is possible
%    to reorder or arrange code from different packages in such a way that
%    these can safely coexist.
%
%    To make that happen we have to make use of the basic
%    functionality that is offered by \TeX{}, e.g., we install
%    special code inside \cs{everypar} to provide hooks at the
%    beginning and we redefine \cs{par} to do some special processing
%    when appropriate to install hooks at the end of the paragraph.
%
%    In order to make this work, we have to ensure that package use of
%    \cs{everypar} is not overwriting our code. This is done through a
%    trick: we basically hide the real \cs{everypar} from the packages
%    and offer them a new token register (with the same name). So if
%    they install their own code it doesn't overwrite ours. Our code
%    then inserts the new \cs{everypar} at the right place inside the
%    process so that it looks as if it was the primitive
%    \cs{everypar}.\footnote{Ideally, \cs{everypar} wouldn't be used
%    at all by packages and instead they would simply write their code
%    into the hooks now offered by the kernel. However, while this is
%    the longterm goal and clearly an improvement (because then the
%    packages do no longer need to worry about getting their code
%    overwritten or needing to account for already existing code in
%    \cs{everypar}), this will not happen overnight. For that reason
%    support for this legacy method is retained.}
%
%    At the end of the paragraph it would be great if we could use a
%    similar trick. However, due to the fact that \TeX{} inserts the
%    token \cs{par} (that doesn't have a defined meaning) we can't hide
%    \enquote{the real thing\textsuperscript{TM}} and offer the
%    package an indistinguishable alternate.
%
%    Fortunately, \LaTeX{} has already redefined \cs{par} for its own
%    purposes. As a result there aren't many packages that attempt to
%    change \cs{par}, because without a lot of extra care that would
%    fail miserably. But the bottom line is that, if you load a package that
%    alters \cs{par} then the end of paragraph hooks are most likely
%    not executing while that redefinition is
%    active.\footnote{Similarly to the \cs{everypar} situation, the
%    remedy is that such packages stop doing this and instead add
%    their alterations into the paragraph hooks now provided.}
%
%
%
% \subsection{The provided hooks}
%
%
%  \begin{variable}{para/before,
%                   para/begin,
%                   para/end,
%                   para/after
%                 }
%    The following four public hooks are defined and executed for
%    each paragraph:
%    \begin{description}
%
%    \item[\hook{para/before}]
%
%      This hook is executed after the kernel hook
%      \cs{@kernel@before@para@before} (discussed below) in vertical
%      mode immediately after \TeX{} has contributed \cs{parskip} to
%      the vertical list and before the actual paragraph processing in
%      horizontal mode starts.
%
%      This hook should either not produce any typeset material or add
%      only vertical material. If it starts a paragraph an error is
%      generated. The reason is that we are in the starting process of
%      processing a paragraph and so this would lead to endless
%      recursion.\footnote{One could allow it but only if the newly
%      started paragraph is processed without any hooks. Furthermore
%      correct spacing would be a bit of a nightmare so for now this
%      is forbidden.}
%
%    \end{description}
%  \end{variable}
%
%    \vspace{-\bigskipamount}
%
%    \begin{description}
%
%    \item[\hook{para/begin}]
%
%      This hook is executed after the kernel hook
%      \cs{@kernel@before@para@begin} (discussed below) in horizontal
%      mode immediately before the indentation box is placed (if there
%      is any, i.e., if the paragraph hasn't been started with
%      \cs{noindent}).
%
%      The indentation box to be typeset is available to the hook as
%      \cs{IndentBox} and its automatic placement (after the hook is
%      executed) can be prevented through \cs{OmitIndent}.
%      More precisely \cs{OmitIndent} voids the box.
%
%      The indentation box is then typeset directly
%      after the hook execution by something equivalent to
%      \cs{box}\cs{IndentBox} followed by the current content of
%      the token register \cs{everypar} that it is available to the
%      kernel or to packages (that run some legacy code).
%
%      One has to be careful not to add any code to the hook that
%      starts its own paragraph (e.g., by adding a \cs{parbox} or a
%      \cs{marginpar} inside) because that would call the hook inside
%      again (as a new paragraph is started there) and thus lead to an
%      endless recursion ending only after exhausting the available
%      memory. This can only be done by making sure that is not
%      executed for the inner paragraphs (or at least not recursively
%      forever).
%
%
%    \item[\hook{para/end}]
%
%      This hook is executed at the end of a paragraph when \TeX{} is
%      ready to return to vertical mode and after it has removed the
%      last horizontal glue (but not any kerns) placed on the horizontal
%      list. The code is still executed in horizontal mode so it is
%      possible to add further horizontal material at this point, but
%      it should not alter the mode (even a temporary exit from
%      horizontal mode would create chaos---any attempt will cause an
%      error message)! After the hook has ended the kernel hook
%      \cs{@kernel@after@para@end} is executed and then \TeX{} returns to
%      vertical mode.
%
%      The hook is offered as public hook, but because of the
%      requirement to stay within horizontal mode one needs to be
%      careful in what is placed into the hook.\footnote{Maybe we
%      should guard against that, but it would be rather tricky to
%      implement as mode changes can happen across group boundaries so
%      one would need to keep a private stack just for that. Well,
%      something to ponder.}
%
%      This hook is implemented as a reversed hook.
%
%    \item[\hook{para/after}]
%
%      This hook is executed directly after \TeX{} has returned to
%      vertical mode and after any material that migrated out of the
%      horizontal list (e.g., from a \cs{vadjust}) has processed.
%
%      This hook should either not produce any typeset material or add
%      only vertical material.
%      However, for this hook starting a new paragraph is not a
%      disaster so that it isn't prevented.
%
%      This hook is implemented as a reversed hook.
%
%      Once that hook code has been processed the kernel hook
%      \cs{@kernel@after@para@after} is executed as the final action
%      of the paragraph processing.
%
%    \end{description}
%
%  \begin{variable}{\@kernel@before@para@before,
%                   \@kernel@after@para@after,
%                    \@kernel@before@para@begin,
%                   \@kernel@after@para@end,
%                  }
%    As already mentioned above there are also four kernel hooks that
%    are executed at the start and end of the processing.
%    \begin{description}
%
%    \item[\cs{@kernel@before@para@before}]
%       For future extensions, not currently used by the kernel.
%
%
%    \item[\cs{@kernel@after@para@after}]
%       For future extensions, not currently used by the kernel.
%
%
%    \item[\cs{@kernel@before@para@begin}]
%
%      Used by the kernel to implement tagging. This hook is executed
%      at the very beginning of a paragraph after \TeX{} has switched to
%      horizontal mode but before any indentation box got added or any
%      \cs{everypar} was run.
%
%      It should not generate typeset material that could alter the
%      position.  Note that it should never leave hmode, otherwise you
%      will end with a loop! We could guard against this, but since it
%      is an internal kernel hook that shouldn't be touched this isn't
%      checked.
%    \end{description}
%  \end{variable}
%
%    \vspace{-\bigskipamount}
%
%    \begin{description}
%
%    \item[\cs{@kernel@after@para@end}]
%
%      Used by the kernel to implement tagging. It is executed
%      directly after the public \hook{para/end} hook. After it there
%      is a quick check that we are still in horizontal mode, i.e.,
%      that the public hook has not mistakenly ended horizontal mode
%      prematurely (this is an incomplete check just testing the mode
%      and could perhaps be improved (at the cost of speed)).
%
%    \end{description}
%
%
% \subsection{Altered and newly provided commands}
%
% \begin{function}{\par,\endgraf,\para_end:}
%    An explicit request for ending a paragraph is provided in plain
%    \TeX{} under the name \cs{endgraf}, which simply uses the
%    primitive meaning (regardless of what \cs{par} may have as its
%    current definition). In \LaTeX{} \cs{endgraf} (with that behavior)
%    was originally also available.
%
%    With the new paragraph handling in \LaTeX{}, ending a paragraph
%    means a bit more than just calling the engine's paragraph
%    builder: the process also has to add any hook code for the end of
%    a paragraph. Thus
%    \cs{endgraf} was changed to provide this additional functionality
%    (along with \cs{par} remaining subject to its current meaning).
%
%    The \pkg{expl3} name for this functionality is \cs{para_end:}.
% \end{function}
%
% \begin{quote}
%    \textbf{Note:} \em The next two commands are still under
%    discussion and may slightly change their semantics (as described
%    in the document) and/or their names between now and the 2021
%    Spring release!
% \end{quote}
%
% \begin{function}{\OmitIndent,\para_omit_indent:}
%    Inside the \hook{para/begin} hook one can use this command to
%    suppress the indentation box at the start of the
%    paragraph. (Technically it is possible to use this command
%    outside the hook as well, but this should not be relied upon.)
%    The box itself remains available for use.
%
%    The \pkg{expl3} name for the function is \cs{para_omit_indent:}.
% \end{function}
%
% \begin{variable}{\IndentBox,\g_para_indent_box}
%    The box register holding the indentation box for the paragraph is
%    available for inspection (or changes) inside hooks. It remains
%    available even if the \cs{OmitIndent} command was
%    used; in that case it  will just not be automatically placed.
%
%    The \pkg{expl3} name for the box register is \cs{g_para_indent_box}.
% \end{variable}
%
%
% \begin{function}{\RawIndent,\para_raw_indent:,
%                  \RawNoindent,\para_raw_noindent:,
%                  \RawParEnd,\para_raw_end:}
% \begin{syntax}
%   \cs{RawIndent}   \textit{hmode material} \cs{RawParEnd}
%   \cs{RawNoindent} \textit{hmode material} \cs{RawParEnd}
% \end{syntax}
%
%    The commands \cs{RawIndent} and \cs{RawNoindent} are not meant
%    for normal paragraph building (where the result is a textual
%    paragraph in the traditional meaning of the word), but for
%    special cases where \TeX{}'s low-level algorithm is used to
%    achieve special effects, but where the result is not a
%    \enquote{paragraph}.
%
%    They are called \enquote{raw}, because they bypass \LaTeX{}'s
%    hook mechanism for paragraphs and simply invoke the low-level
%    \TeX{} algorithm.  I.e., they are like the original \TeX{}
%    primitives \cs{indent} and \cs{noindent} (that is they execute no
%    hooks other than \cs{everypar}) except that they can only be used
%    in vertical mode and generate an error if found elsewhere.
%
%    To avoid issues a paragraph started by them should always be
%    ended by \cs{RawParEnd}\footnote{Technical note for those who
%    know their \textit{\TeX book\/}: the \cs{RawParEnd} command
%    invokes the original \TeX{} engine definition of \cs{par} that
%    (soley) triggers the paragraph builder in \TeX{} when found
%    inside unrestricted horizontal mode and does nothing in other
%    processing modes.}
%    and not by \cs{par} (or a blank line), because the latter will execute
%    hooks which then have no counterpart at the beginning of the
%    paragraph. It is the responsibility of the programmer to make
%    sure that they are properly paired. This also means that one
%    should not put arbitrary user content between these commands if
%    that content could contain stray \cs{par}s.
%
%    The \pkg{expl3} names for the functions are
%    \cs{para_raw_indent:}, \cs{para_raw_indent:} and
%    \cs{para_raw_end:}.
% \end{function}
%
%
%
% \subsection{Examples}
%
%    None of the examples in this section are meant for real use as
%    they are far too simpleminded  but they should give some ideas of
%    what could be possible if a bit more care is applied.
%
% \subsubsection{Testing the mechanism}
%
%    The idea is to output for each paragraph encountered some
%    information: a paragraph sequence number, a level number in roman
%    numerals, the environment in which this paragraph appears, and
%    the line number where the start or end of the paragraph is, e.g.,
%    something like
%\begin{verbatim}
%  PARA: 1-i start (document env. on input line 38)
%  PARA: 1-i end   (document env. on input line 38)
%  PARA: 2-i start (document env. on input line 40)
%  PARA: 3-ii start (minipage env. on input line 40)
%  PARA: 3-ii end   (minipage env. on input line 40)
%  PARA: 2-i end   (document env. on input line 41)
%\end{verbatim}
%    As you can see paragraph 2 starts on line 40 and ends on 41 and
%    inside a minipage started paragraph 3 (start and end on line 40).
%    If you run this on some document you will find that \LaTeX{}
%    considers more things \enquote{a paragraph} than you have
%    probably thought.
%
%    This was generated by the following hook code:
%\begin{verbatim}
%  \newcounter{paracnt}         % sequence counter
%  \newcounter{paralevel}       % level counter
%\end{verbatim}
%
%    To support paragraph nesting we need to maintain a stack of the
%    sequence numbers. This is most easily done using \pkg{expl3}
%    functions, so we switch over. This is not a very general
%    implementation, just enough for what we need and a bit of
%    \LaTeXe{} thrown in as well. When popping, the result gets stored
%    in \cs{paracntvalue} and the \cs{ERROR}  should never happen
%    because it means we have tried to pop from an empty stack.
%\begin{verbatim}
%  \ExplSyntaxOn
%  \seq_new:N \g_para_seq
%  \cs_new:Npn \ParaPush
%    {\seq_gpush:No  \g_para_seq {\the\value{paracnt}}}
%  \cs_new:Npn \ParaPop  {\seq_gpop:NNF  \g_para_seq \paracntvalue \ERROR }
%  \ExplSyntaxOff
%\end{verbatim}
%    At the start of the paragraph increment both sequence counter and
%    level and also save the then current sequence number on our stack.
%\begin{verbatim}
%  \AddToHook{para/begin}{%
%    \stepcounter{paracnt}\stepcounter{paralevel}%
%    \ParaPush
%\end{verbatim}
%    To display the sequence number we \cs{typeout} the current
%    sequence and level number. The command \cs{@currenvir} gives us
%    the current environment and \cs{on@line} produces a space and
%    the current input line number.
%\begin{verbatim}
%    \typeout{PARA: \arabic{paracnt}-\roman{paralevel} start
%      (\@currenvir\space env.\on@line)}%
%\end{verbatim}
%    We also typeset the sequence number as a tiny red number in a box
%    that takes up no horizontal space. This helps us seeing where
%    \LaTeX{} sees the start and end of the paragraphs in the
%    document.
%\begin{verbatim}
%    \llap{\color{red}\tiny\arabic{paracnt}\ }%
%  }
%\end{verbatim}
%
%   At the end of the paragraph we display sequence number and
%    level again. The level counter has the correct value but we need
%    to retrieve the right sequence value by popping it off the stack
%    after which it is available in \cs{paracntvalue} the way we have
%    set this up above.
%\begin{verbatim}
%  \AddToHook{para/end}{%
%    \ParaPop
%    \typeout{PARA: \paracntvalue-\roman{paralevel} end \space\space
%      (\@currenvir\space env.\on@line)}%
%\end{verbatim}
%    We also typeset again a tiny red number with that value, this
%    time sticking out to the right.\footnote{Note that this can alter
%    the document pagination, because a paragraph ending in a display
%    (e.g., an equation) will get an extra line---in that case our tiny number
%    has an effect even though it doesn't take up any space, because
%    it paragraph is no longer empty and thus isn't dropped!}
%    We also decrement the level counter since our level has finished.
%\begin{verbatim}
%    \rlap{\color{red}\tiny\ \paracntvalue}%
%    \addtocounter{paralevel}{-1}%
%  }
%  \makeatother
%\end{verbatim}
%
%
%
% \subsubsection{Mark the first paragraph of each \env{itemize}}
%
%    The code for this is rather simple. We supply some code that is
%    executed only once inside a hook at the start of
%    each \env{itemize}. We explicitly change the color back and
%    forth so that we don't introduce grouping around the paragraph.
%\begin{verbatim}
%  \AddToHook{env/itemize/begin}{%
%    \AddToHookNext{para/begin}{\color{blue}}%
%    \AddToHookNext{para/end}{\color{black}}%
%  }
%\end{verbatim}
%    As a result the first paragraph of each \env{itemize} will appear
%    in blue.
%
%
%
% \subsection{Some technical notes}
%
%    The code tries hard to be transparent for package code, but of
%    course any change means that there is a potential for breaking
%    other code. So in section we collect a few cases that may be of
%    importance if low-level code is dealing with paragraphs that are
%    now behaving slightly differently. The notes are from issues we
%    observed and will probably grow over time.
%
% \subsubsection{Glue items between paragraphs (found with \pkg{fancypar})}
%
%    In the past \LaTeX{} placed two glue items between two
%    consecutive paragraphs, e.g.,
%\begin{verbatim}
%   text1 \par text2 \par
%\end{verbatim}
%   would show something like
%\begin{verbatim}
% \glue(\parskip) 0.0 plus 1.0
% \glue(\baselineskip) 5.16669
%\end{verbatim}
%   but now there is anothe \cs{parskip} glue (that is always 0pt):
%\begin{verbatim}
% \glue(\parskip) 0.0 plus 1.0
% \glue(\parskip) 0.0
% \glue(\baselineskip) 5.16669
%\end{verbatim}
%    The reason is that we generate a \enquote{fake} paragraph to
%    gain control and safely add the early hooks, but this generates
%    an additional glue item. That item doesn't contribute anything
%    vertically but if somebody writes code that unravels a constructed
%    list using \cs{lastbox}, \cs{unskip} and \cs{unpenalty} then the
%    code has to remove one additional glue item  or else it will fail.

%
% ^^A \subsubsection{}
%
%
%
% ^^A \subsubsection{}
%
%
%
%
%
%
% \MaybeStop{\setlength\IndexMin{200pt}  \PrintIndex  }
%
%
% \section{The Implementation}
%
%    \begin{macrocode}
%<@@=para>
%    \end{macrocode}
%
% \changes{v1.0g}{2021/05/24}{Use \cs{msg_...} instead of \cs{__kernel_msg...}}
%
%
%    \begin{macrocode}
%<*2ekernel|latexrelease>
\ExplSyntaxOn
%<latexrelease>\NewModuleRelease{2021/06/01}{ltpara}
%<latexrelease>                 {Paragraph~handling~and~hooks}
%    \end{macrocode}
%
%
%
% \subsection{Providing hooks for paragraphs}
%
%
%  \begin{macro}{para/before,para/after,para/begin,para/end}
%    The public hooks. They are implemented as a paired set of hooks.
%    \begin{macrocode}
\hook_new_pair:nn{para/before}{para/after}
\hook_new_pair:nn{para/begin}{para/end}
%    \end{macrocode}
%  \end{macro}
%
%
%
%  \begin{macro}{\@kernel@before@para@before,
%                \@kernel@after@para@after,
%                \@kernel@before@para@begin,
%                \@kernel@after@para@end}
%    The corresponding kernel hooks (for tagging and future extensions).
%    \begin{macrocode}
\let \@kernel@before@para@before \@empty
\let \@kernel@before@para@begin  \@empty
\let \@kernel@after@para@end     \@empty
\let \@kernel@after@para@after   \@empty
%    \end{macrocode}
%  \end{macro}
%
%
%
%
%
%  \begin{macro}{\g_@@_standard_everypar_tl}
%    Whenever \TeX{} starts a paragraph it inserts first an
%    indentation box and then executes the tokens stored in
%    \cs{tex_everypar:D} (known to \LaTeX{} as \cs{everypar}). We
%    alter this behavior slightly here, so that hooks are added
%    into the right place. Otherwise the process change remains
%    transparent to any legacy code for this space.
%
%    We keep the standard code to be used by \cs{tex_everypar:D} in a
%    separate token list because we have to switch back and forth
%    for error recovery and so altering \cs{tex_everypar:D}  all the
%    time should be a tiny bit faster.
%    \begin{macrocode}
\tl_new:N \g_@@_standard_everypar_tl
%    \end{macrocode}
%    Here is now its definition:
%    \begin{macrocode}
\tl_gset:Nn \g_@@_standard_everypar_tl {
%    \end{macrocode}
%    First we remove the indentation box and store it in
%    \cs{g_para_indent_box}. If there was none because the paragraph
%    was started by \cs{noindent} the box register will be void.
%    \begin{macrocode}
  \box_gset_to_last:N \g_para_indent_box
%    \end{macrocode}
%
%    This will make the newly started horizontal list empty, so if we
%    stop it now and return to vertical mode it will be dropped by
%    \TeX{}. We do that but inside a group so that any \cs{parshape}
%    settings will not get lost as we need them for later.
%    \begin{macrocode}
  \group_begin:
    \tex_par:D
  \group_end:
%    \end{macrocode}
%    We then change \cs{tex_everypar:D} to generate an error so that
%    we can detect and report if the \hook{para/before} hook illegally
%    changed out of vmode.
%    \begin{macrocode}
  \tex_everypar:D { \msg_error:nnnn { hooks }{ para-mode }{before}{vertical} }
  \@kernel@before@para@before
  \hook_use:n {para/before}
%    \end{macrocode}
%    Assuming the hooks have been well behaved it is time to return to
%    horizontal mode and start the paragraph in earnest. We already
%    have the indentation box saved away so we now have to restart the
%    paragraph with an empty \cs{tex_everypar:D} and with
%    \cs{tex_noindent:D}. And we need to make sure not to get another
%    \cs{parskip} or rather (since we can't prevent that) that it is
%    of zero size.
%    \begin{macrocode}
  \group_begin:
    \tex_everypar:D {}
    \skip_zero:N \tex_parskip:D
    \tex_noindent:D
  \group_end:
%    \end{macrocode}
%    That brings us back to the start of the horizontal list but we
%    need to change \cs{tex_everypar:D} back to its normal content in
%    case there are nested paragraphs coming up.
%    \begin{macrocode}
  \tex_everypar:D{\g_@@_standard_everypar_tl}
%    \end{macrocode}
%
%    This is followed by executing the kernel and the public hook. The
%    kernel hook is there to enable tagging.
%    \begin{macrocode}
  \@kernel@before@para@begin
  \hook_use:n {para/begin}
%    \end{macrocode}
%    If we aren't in horizontal mode any longer the hooks above misbehaved.
%    \begin{macrocode}
  \if_mode_horizontal: \else:
    \msg_error:nnnn { hooks }{ para-mode }{begin}{vertical} \fi:
%    \end{macrocode}
%    Finally we reinsert the indentation box (unless suppressed) and
%    then call \cs{everypar} the way legacy \LaTeX\ code expects it.
%
%    However, adding the public \cs{everypar} is a bit tricky (see below) so
%    we add that later, and indirectly.
%    \begin{macrocode}
  \@@_handle_indent:
% \the \everypar           % <--- done differently below
}
%    \end{macrocode}
%  \end{macro}
%
%  \begin{macro}[int]{\tex_everypar:D}
%    \cs{tex_everypar:D} then only has to execute
%    \cs{g_@@_standard_everypar_tl} by default.
%    \begin{macrocode}
\tex_everypar:D{\g_@@_standard_everypar_tl}
%    \end{macrocode}
%  \end{macro}
%
%
%
%  \begin{macro}[int]{\everypar}
%
%    Tokens inserted at the beginning of the paragraph are placed into
%    \cs{everypar} inside legacy \LaTeX{} code, e.g., by the list
%    environments or by headings to handle \cs{clubpenalty}, etc. Now
%    this isn't any longer the primitive but simply a toks register
%    used in the code above but to legacy \LaTeX{} code that is
%    transparent.
%
%    There is, however, a problem: a handful packages use exactly the
%    same trick and replace the primitive with a token register and
%    call the token register inside the renamed primitive. That is
%    they assume that \cs{everypar} is the primitive and that it will
%    still be called at the start of the paragraph even if renamed.
%
%    But if we have already replaced it by a token register then all
%    they do is to give that token register a new name. Thus our code
%    in \cs{tex_everypar:D} would call \cs{everypar} (which is now their
%    token register) and the code that they added ends up in our
%    token register which is then never used at all. A bit mind
%    boggling I guess.
%
%    So what we have to do is not to call the token register
%    \cs{everypar} by its name inside \cs{tex_everypar:D} but by using
%    its actual register number.
%    \begin{macrocode}
\newtoks \everypar
%    \end{macrocode}
%
%    After we have allocated a new toks register with the name
%    \cs{everypar} the actual register number is available (briefly)
%    inside \cs{allocationnumber}. So instead of \cs{the}\cs{everypar}
%    we have to put \cs{the}\cs{toks}\meta{allocated number} at the end of
%    \cs{tex_everypar:D}.
%
%    So what remains doing is to append a few tokens to the token list
%    \cs{g_@@_standard_everypar_tl} which we do now. We use \texttt{x}
%    expansion here to get the value of \cs{allocationnumber} in, all
%    the other tokens should not be expanded at this point.

%    One important point here is to terminate the register allocation
%    number with a real space. This space will get swallowed up when
%    the number is read. Anything else, such as \cs{scan_stop:} would
%    remain in the input and that would mean that it would interfere
%    with \cs{everypar} code that attempts to scan ahead to see how
%    the paragraph text starts.
%    \begin{macrocode}
\tl_gput_right:Nx \g_@@_standard_everypar_tl {
    \exp_not:N \the
    \exp_not:N \toks
    \the \allocationnumber
    \c_space_tl
}
%    \end{macrocode}
%  \end{macro}
%
%
%  \begin{macro}{\g_para_indent_box}
%    For managing the indentation we need to provide a public
%    accessible box register
%    \begin{macrocode}
\box_new:N \g_para_indent_box
%    \end{macrocode}
%  \end{macro}

%  \begin{macro}{\@@_handle_indent:}
%    Adding (typesetting) the indent box is straight forward.
%    If it was emptied before it does nothing.
%    \begin{macrocode}
\cs_new:Npn \@@_handle_indent: {
  \box_use_drop:N \g_para_indent_box
}
%    \end{macrocode}
%     The declaration \cs{para_omit_indent:} (or
%    \cs{OmitIndent}) changes that to do nothing.
%    \begin{macrocode}
\cs_new:Npn \para_omit_indent: {
  \box_gclear:N \g_para_indent_box
}
%    \end{macrocode}
%  \end{macro}
%
%
%  \begin{macro}{\IndentBox,\OmitIndent}
%    The \LaTeXe{} names for the indentation box and for suppressing it
%    for use in the \hook{para/begin} hook.
%    \begin{macrocode}
\cs_set_eq:NN \IndentBox \g_para_indent_box
\cs_set_eq:NN \OmitIndent \para_omit_indent:
%    \end{macrocode}
%  \end{macro}
%
%
%
%
%
%
%  \begin{macro}{\para_end:}
%    Adding hooks to the end of a paragraph is similar but here we
%    need to alter the command that is used by \TeX{} to end horizontal
%    mode and return to vertical mode, i.e., \cs{par}.
%
%    This is a bit more complicated as this command can appear anywhere
%    either explicitly or implicitly added by \TeX{} in certain
%    situations:
%    \begin{itemize}
%    \item
%      when using \cs{par} in the code or the document
%    \item
%      when using a blank line (which is converted to \cs{par})
%    \item
%      when \TeX{} finds any commands incompatible with horizontal
%      mode it issues a \cs{par} and then rereads the command.
%    \end{itemize}
%
%    Unfortunately, \TeX{} has some (these days) unnecessary
%    optimizations: if a \cs{vbox} ends and \TeX{} is still in
%    horizontal mode it simply exercises the paragraph builder instead
%    of issuing a \cs{par}. It is therefore necessary for \LaTeX{} to
%    ensure that this case doesn't happen and all boxes internally
%    have a \cs{par} command at their end.
%
%    This \cs{par} may or may not run the ``par primitive'' (which is
%    always available as \cs{tex_par:D} in \pkg{expl3}); it is
%    permissible to have a changed meaning and it is in fact changed
%    by \LaTeX{} in various ways at various points inside
%    \texttt{latex.ltx}.  For this \LaTeXe{} code has the following
%    conventions: \cs{@@@@par} and \cs{endgraf} both refer to the
%    default meaning (in the past this was the initex primitive) while
%    \cs{par} is the current meaning which maybe does something else.
%
%
%    We are now going to change this default meaning to instead run
%    \cs{para_end:}, which ultimately executes the initex
%    primitive but additionally adds our hooks when appropriate.
%    This way the change is again transparent to the legacy \LaTeXe{}
%    code.
%
%    In most cases \cs{para_end:} should behave exactly like the
%    primitive and we achieve this by simply expanding it to the
%    primitive which is available to us as \cs{tex_par:D}. This way we
%    don't have to care about whether \TeX{} just does nothing (e.g.,
%    if in vertical mode already) or generates an error, etc.
%    \begin{macrocode}
\cs_new_protected:Npn \para_end: {
%    \end{macrocode}
%
%    CCC Maybe needs more explanation.
%    TEMP NOTE: What should happen if in outer hmode with an empty hlist? 
%
%    The only case we care about is when we are in horizontal mode
%    (i.e., doing typesetting) and not also in inner mode (i.e.,
%    making paragraphs and not building an \cs{hbox}.
%\begin{verbatim}
%  \bool_lazy_and:nnT
%       { \mode_if_horizontal_p: }
%       { \bool_not_p:n { \mode_if_inner_p: } }
%       { ...
%\end{verbatim}
%    Since this is executed for each and every paragraph in a document
%    we try to stay as fast as possible, so we do not use the
%    above construct but two conditionals instead. Using low-level
%    \cs{if_mode...} conditions would be even faster but has the
%    danger to conflict with conditionals in the user hooks.
%
%    If \cs{para_end:} is executed while \TeX{} is currently doing a
%    low-level assignment the test for horizontal mode may get
%    executed as part of the assignment. That is normally not an issue
%    but we just found one case where it is:
%\begin{verbatim}
%   \afterassignment\lst@vskip\@tempskipa \z@ \par
%\end{verbatim}
%    If \TeX{} is in hmode while that assignment happens then the
%    \cs{par} is seen in hmode because in the above case the
%    assignment may not be finished (one should have used \cs{z@skip}) and
%    the \cs{lst@vskip} will get inserted into the middle of the
%    conditional. The \cs{lst@vskip} then changes to vmode and you get
%    a surprising error about the \texttt{para/end} hook having
%    changed modes even if you don't have any hook code(!): it is
%    the inserted \cs{lst@vskip} that is actually causing the change of 
%    mode.This is what happened
%    when the output routines got started while a \texttt{lstlisting}
%    environment (that redefines \cs{vskip} in this way) was
%    active. This is really faulty coding, but we try to be proactive
%    and guard the conditional so that any scanning is first stopped, thus:
%    \begin{macrocode}
  \scan_stop:
%    \end{macrocode}
%
%    \begin{macrocode}
  \mode_if_horizontal:TF {
    \mode_if_inner:F {
%    \end{macrocode}
%    In that case the action of the primitive would be to remove the
%    last glue (but no kerns) from the horizontal list (constructed to form
%    a paragraph) and then to append a penalty of 10000 and the
%    \cs{parfillskip}; it then passes the whole list to the
%    paragraph builder, which breaks it into lines and \TeX{} then
%    returns to vertical mode.
%
%    What we want to do is to add this hook code at the end of
%    the horizontal list before any of the above happens. 
%    If there was a glue item at the end of the list then
%    it should get removed before the hook code gets added so we have
%    to arrange for this removal.  
%
%    As in other simular cases, it maybe best to add here 
%    a \cs{nobreak} in case the hook itself adds glue and thus 
%    creates a non-explicit and unwanted potential breakpoont.  
%    On the other hand (as has been argued) the code in the hook 
%    should perhaps have the responsibility for adding such a 
%    guard penalty in this casse.  
%    This needs further analysis and decisions (as in emails). 
%
%    In either case, good documentation of these hooks is essential, 
%    covering what the hook may or should provide and all  
%    such related considerations convernimg the content.
%
%    There is not much point in checking if there was really a glue
%    item at the end of the horizontal list, instead we simply try to
%    remove one using \cs{tex_unskip:D}: if there wasn't one this will
%    do nothing.
%    \begin{macrocode}
         \tex_unskip:D
%    \end{macrocode}
%    We then execute the public hook (which may add some final typeset
%    material) followed by the kernel hook that we need for adding tagging
%    support. None of this is supposed to change the mode---at the
%    moment we make only a very simple test for this, more devious
%      changes go unnoticed, but too bad as they will then probably
%      backfire badly.
%    \begin{macrocode}
         \hook_use:n{para/end}
         \@kernel@after@para@end
         \mode_if_horizontal:TF {
%    \end{macrocode}
%    The final action (before getting to the point where
%    \cs{tex_par:D} is called) is to add an extra glue item so that the
%    primitive is prevented from removing intended glue
%    (if there was some). If we don't do this and the
%    horizontal list ends in several glue items we would end up removing 
%    two glue items instead of just the last one, which would be wrong.  
%    We use glue (rather than a kern) as that will be removed by the 
%    primitive.
%
%    There is however one other \TeX{} optimization that hurts: in a
%    sequence like this \verb=$$ ... $$ \par= (with \cs{par} being the primitive)
%    \TeX{} will be in
%    horizontal mode after the display, ready to receive further
%    paragraph text, but since the \cs{par} follows immediately there
%    is a ``null'' paragraph at the end and \TeX{} simply throws that
%    away. 
%    The space between \verb=$$= and \cs{par} got already
%    dropped during the display processing so the \cs{par} is not
%    removing any space and appending \cs{parfillskip}, instead it
%    simply goes silently to vmode. 
%
%    Now if we would have added something (to
%    prevent glue removal) that would look to \TeX{} like material
%    after the display and so we would end up with an empty paragraph
%    just containing a penalty and \cs{parfillskip}.
%
%    We therefore check if the current hlist does end in glue
%    (\cs{tex_lastnodetype:D} has the value \texttt{11}) and
%    if so we add a zero-length guard skip which will be removed by the
%    following \cs{tex_par:D}.
% \changes{v1.0i}{2021/09/18}{Use skip rather than kern as guard.}
%    \begin{macrocode}
           \if_int_compare:w 11 = \tex_lastnodetype:D
             \tex_hskip:D \c_zero_dim
           \fi:
%    \end{macrocode}
%    To run the \hook{para/after} hook we first end the
%    paragraph. This means that the \cs{tex_par:D} at the very end is
%    unnecessary but executing it there unnecessarily is better than
%    having code that tests for all the different mode possibilities.
%    \begin{macrocode}
           \tex_par:D
           \hook_use:n{para/after}
           \@kernel@after@para@after
         }
%    \end{macrocode}
%    If we were not horizontal mode (the F case from above) 
%    then the earlier hook
%    \hook{para/end} must have been at fault, so we report that.
%    \begin{macrocode}
         { \msg_error:nnnn { hooks }{ para-mode }{end}{horizontal} }
%    \end{macrocode}
%    Finally close out the nested conditionals.
%    \begin{macrocode}
    }
  }
%    \end{macrocode}
%     And then we can use the primitive to truly end the paragraph.
%    \begin{macrocode}
  \tex_par:D
}
%    \end{macrocode}
%  \end{macro}
%
%
%
%
%  \begin{macro}{\para_raw_indent:,
%                \para_raw_noindent:,
%                \para_raw_end:}
%
%    The commands \cs{para_raw_indent:} and \cs{para_raw_noindent:}
%    are like the primitives \cs{indent} and \cs{noindent} except that
%    they can only be used in vertical mode.
%
%    To avoid issues a paragraph started by them should always be
%    ended by \cs{para_raw_end:} and not by \cs{para_end:} or
%    \cs{par} as the latter will execute hooks which then have no
%    counterpart at the beginning of the paragraph. It is the
%    responsibility of the programmer to make sure that they are
%    properly paired.
%    \begin{macrocode}
\cs_new:Npn \para_raw_indent: {
  \mode_if_vertical:TF
       {
         \tex_everypar:D {
           \box_gset_to_last:N \g_para_indent_box
           \tex_everypar:D { \g_@@_standard_everypar_tl }
           \@@_handle_indent:
           \the\everypar }
       }
       { \msg_error:nn { latex2e }{ raw-para } }
  \tex_indent:D
}
%    \end{macrocode}
%
%    \begin{macrocode}
\cs_new:Npn \para_raw_noindent: {
  \mode_if_vertical:TF
       {
         \tex_everypar:D {
           \tex_everypar:D { \g_@@_standard_everypar_tl }
           \the\everypar }
       }
       { \msg_error:nn { latex2e }{ raw-para } }
  \tex_noindent:D
}
%    \end{macrocode}
%
%    \begin{macrocode}
\cs_new_eq:NN \para_raw_end: \tex_par:D
%    \end{macrocode}
%  \end{macro}
%
%
%  \begin{macro}{\RawIndent,\RawNoIndent,\RawParEnd}
%    The \LaTeXe{} names for starting and ending a paragraph without adding any hooks.
%    \begin{macrocode}
\cs_set_eq:NN \RawIndent   \para_raw_indent:
\cs_set_eq:NN \RawNoindent \para_raw_noindent:
\cs_set_eq:NN \RawParEnd   \para_raw_end:
%    \end{macrocode}
%  \end{macro}
%
%
%    This ends the \texttt{para} module code.
%    \begin{macrocode}
%<@@=>
%    \end{macrocode}
%
%  \begin{macro}{\par,\endgraf}
%  \begin{macro}[int]{\@@par}
%
%    Having the new default definition for \cs{par} we also have to
%    set it up so that it gets used. This involves three commands:
%    \cs{par}, \cs{@@par} (to which \LaTeX{} resets \cs{par}
%    occasionally) and \cs{endgraf}, which is another name for the
%    ``default'' action of \cs{par}.
%
%    \begin{macrocode}
\cs_set_eq:NN \par     \para_end:
\cs_set_eq:NN \@@par   \para_end:
\cs_set_eq:NN \endgraf \para_end:
%    \end{macrocode}
%  \end{macro}
%  \end{macro}
%
%    While this is not integrated properly into the format we have to
%    redo the \cs{everypar} setting from the kernel, otherwise that
%    gets lost (as it happens before that file is loaded).
%    \begin{macrocode}
\everypar{\@nodocument} %% To get an error if text appears before the
%    \end{macrocode}
%
%
%   \subsection{The error messages}
%
%    This one is used when we detect that some hook code has changed
%    the mode where it shouldn't, e.g., by starting or ending a
%    paragraph.
%    The first argument is the hook name second the mode
%    it should have stayed in but didn't.
%
%    \begin{macrocode}
\msg_new:nnnn { hooks } { para-mode }
  {
    Illegal~mode~ change~ in~ hook~ 'para/#1'.\\
    Hook~ code~ did~ not~ remain~ in~ #2~ mode.
  }
  {
    Paragraph~ hooks~ cannot~ change~ the~ TeX~ mode~ without~ causing~
    endless~ recursion.~ The~ hook~ code~ in~ 'para/#1'~ needs~ to~ stay~
    in~ #2~ mode,~ but~ it~ didn't.~ Examine~ the~ hook~
    code~ with~ \iow_char:N \\ShowHook~ to~ find~ the~ issue.
  }
%    \end{macrocode}
%
%   \changes{v1.0i}{2021/08/27}{Internal message name changes}
%    And here is one used in the \enquote{raw} commands when they are
%    used outside of vertical mode.
%    \begin{macrocode}
\msg_new:nnnn { latex2e } { raw-para }
  {
     Not~ in~ vertical~ mode.
  }
  {
    Starting~ a~ paragraph~ with~ \iow_char:N \\RawIndent~ or~
    \iow_char:N \\RawNoindent \\
    (or~ \iow_char:N \\para_raw_indent:~ or~
    \iow_char:N \\para_raw_noindent:)~ is~ only~ allowed \\
    if~ LaTeX~ is~ in~ vertical~ mode.
  }
%    \end{macrocode}


%    \begin{macrocode}
%
%<latexrelease>\IncludeInRelease{0000/00/00}%
%<latexrelease>                 {ltpara}{Undo~hooks~for~paragraphs}
%<latexrelease>
%<latexrelease>\let \OmitIndent  \@undefined
%<latexrelease>\let \IndentBox   \@undefined
%<latexrelease>\let \RawIndent   \@undefined
%<latexrelease>\let \RawNoindent \@undefined
%<latexrelease>\let \RawParEnd   \@undefined
%<latexrelease>
%<latexrelease>\cs_set_eq:NN \par     \tex_par:D
%<latexrelease>\cs_set_eq:NN \@@par   \tex_par:D
%<latexrelease>\cs_set_eq:NN \endgraf \tex_par:D
%<latexrelease>
%    \end{macrocode}
%    We also need to clean up the primitive ``everypar'' as that
%    should no longer execute any code by default. And, of course,
%    make \cs{everypar} become the primitive again.
% \changes{v1.0k}{2021/10/19}{Remove content from \cs{tex_everypar:D}
%    on rollback}
%    \begin{macrocode}
%<latexrelease>\tex_everypar:D {}
%<latexrelease>\cs_set_eq:NN \everypar \tex_everypar:D
%<latexrelease>
%<latexrelease>\EndModuleRelease
\ExplSyntaxOff
%</2ekernel|latexrelease>
%    \end{macrocode}
%
%
%
%
%
%%%%%%%%%%%%%%%%%%%%%%%%%%%%%%%%%%%%%%%%%%%%%
\endinput
%%%%%%%%%%%%%%%%%%%%%%%%%%%%%%%%%%%%%%%%%%%%%
%
