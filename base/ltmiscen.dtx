% \iffalse meta-comment
%
% Copyright (C) 1993-2020
% The LaTeX3 Project and any individual authors listed elsewhere
% in this file.
%
% This file is part of the LaTeX base system.
% -------------------------------------------
%
% It may be distributed and/or modified under the
% conditions of the LaTeX Project Public License, either version 1.3c
% of this license or (at your option) any later version.
% The latest version of this license is in
%    https://www.latex-project.org/lppl.txt
% and version 1.3c or later is part of all distributions of LaTeX
% version 2008 or later.
%
% This file has the LPPL maintenance status "maintained".
%
% The list of all files belonging to the LaTeX base distribution is
% given in the file `manifest.txt'. See also `legal.txt' for additional
% information.
%
% The list of derived (unpacked) files belonging to the distribution
% and covered by LPPL is defined by the unpacking scripts (with
% extension .ins) which are part of the distribution.
%
% \fi
% \iffalse
%%% From File: ltmiscen.dtx
%
%<*driver>
% \fi
\ProvidesFile{ltmiscen.dtx}
             [2020/08/15 v1.1v LaTeX Kernel (Misc. Environments)]
% \iffalse
\documentclass{ltxdoc}
\GetFileInfo{ltmiscen.dtx}
\title{\filename}
\date{\filedate}
 \author{%
  Johannes Braams\and
  David Carlisle\and
  Alan Jeffrey\and
  Leslie Lamport\and
  Frank Mittelbach\and
  Chris Rowley\and
  Rainer Sch\"opf}

\begin{document}
 \MaintainedByLaTeXTeam{latex}
 \maketitle
 \DocInput{\filename}
\end{document}
%</driver>
% \fi
%
%
% \changes{v1.0c}{1994/03/28}{Improve Documentation}
% \changes{v1.0d}{1994/03/29}{Remove counter macros to ltcntlen}
% \changes{v1.0g}{1994/05/02}{Changed 91 to 1991 and moved some bits}
% \changes{v1.0i}{1994/05/05}{Removed braces from ifnextchar and
% ifstar arguments}
% \changes{v1.0m}{1994/05/20}{Use new warning commands}
% \changes{v1.0n}{1994/05/21}{Use new error commands}
% \changes{v1.0s}{1994/10/14}{Move math to other file}
% \changes{v1.0v}{1994/11/17}
%         {\cs{@tempa} to \cs{reserved@a}}
% \changes{v1.0x}{1995/04/22}{Removed extra def of \cs{@gobble}}
% \changes{v1.0z}{1995/07/13}{Improve Documentation}
% \changes{v1.1c}{1996/04/22}{Improve Documentation}
% \changes{v1.1d}{1996/06/03}{Move setting of verbatim font and
%         \cs{@noligs}.}
% \changes{v1.1g}{1998/08/17}{(RmS) Minor documentation fixes.}
% \changes{v1.1p}{2019/08/27}{Make various commands robust}
%
% \section{Miscellaneous Environments}
%  This section implements the basic environment mechanism, and also
% a few specific environments including |document|, The math
% environments and related commands, the `flushing' environments,
% (|center|, |flushleft|, |flushright|), and |verbatim|.
%
% \StopEventually{}
%
%    \begin{macrocode}
%<*2ekernel>
\message{environments,}
%    \end{macrocode}
%
% \subsection{Environments}
%
%  |\begin{foo}| and |\end{foo}| are used to delimit environment |foo|.
%
%  |\begin{foo}| starts a group and calls |\foo| if it is defined,
%  otherwise it does nothing.
%
% |\end{foo}| checks to see that it matches the
%  corresponding |\begin| and if so, it calls |\endfoo| and does an
%  |\endgroup|.  Otherwise, |\end{foo}| does nothing.
%
%  If |\end{foo}| needs to ignore blanks after it, then |\endfoo| should
%  globally set the |@ignore| switch true with |\@ignoretrue|
%  (this will automatically be global).
%
%
%  NOTE: |\@@end| is defined to be the |\end| command of \TeX82.
%
%  |\enddocument| is the user's command for ending the manuscript file.
%
%  |\stop| is a panic button --- to end \TeX\ in the middle.
%
% \begin{oldcomments}
% \enddocument ==
%   BEGIN
%    \@checkend{document}   %% checks for unmatched \begin
%    \clearpage
%    \begingroup
%      if @filesw = true
%        then  close file @mainaux
%              if G@refundefined = true
%               then LaTeX Warning: 'There are undefined references.' fi
%              if @multiplelabels = true
%                then LaTeX Warning:
%                    'One or more label(s) multiply defined.'
%                else
%                \@setckpt {ARG1}{ARG2} == null
%                \newlabel{LABEL}{VAL} ==
%                    BEGIN
%                      \reserved@a == VAL
%                      if def(\reserved@a) = def(\r@LABEL)
%                        else @tempswa := true          fi
%                    END
%                \bibcite{LABEL}{VAL} == null
%                    BEGIN
%                      \reserved@a == VAL
%                      if def(\reserved@a) = def(\g@LABEL)
%                        else @tempswa := true          fi
%                    END
%                @tempswa := false
%                make @ a letter
%                \input \jobname.AUX
%                if @tempswa = true
%                  then LaTeX Warning: 'Label may have changed.
%                                  Rerun to get cross-references right.'
%       fi     fi     fi
%    \endgroup
%    finish up
%   END
%
%  \@writefile{EXT}{ENTRY} ==
%      if tf@EXT undefined
%        else \write\tf@EXT{ENTRY}
%      fi
% \end{oldcomments}
%
% \begin{macro}{\@currenvir}
%    The name of the current environment.  Initialized to
%    \texttt{document} to so that |\end{document}| works correctly.
%    \begin{macrocode}
\def\@currenvir{document}
%    \end{macrocode}
% \end{macro}
%
% \begin{macro}{\if@ignore}
% \begin{macro}{\@ignoretrue}
% \begin{macro}{\@ignorefalse}
% \changes{v1.1e}{1996/07/26}{put \cs{global} into definition}
%    \begin{macrocode}
\def\@ignorefalse{\global\let\if@ignore\iffalse}
\def\@ignoretrue {\global\let\if@ignore\iftrue}
\@ignorefalse
%    \end{macrocode}
% \end{macro}
% \end{macro}
% \end{macro}
%
%
% \begin{macro}{\ignorespacesafterend}
% \changes{v1.1e}{1996/07/26}{user level macro added}
%    \begin{macrocode}
\let\ignorespacesafterend\@ignoretrue
%    \end{macrocode}
% \end{macro}
%
%  \begin{environment}{document}
%  \begin{macro}{\enddocument}
% \changes{LaTeX2.09}{1993/08/03}
%         {Changed redefinition of \cs{global} to redefinition
%               of \cs{@setckpt}.}
% \changes{LaTeX2.09}{1993/09/08}
%         {Added warning in case of undefined references.}%
% \changes{v0.9e}{1993/12/09}{Hook added}
%    \begin{macrocode}
%</2ekernel>
%<*2ekernel|latexrelease>
%<latexrelease>\IncludeInRelease{2020/10/01}%
%<latexrelease>                 {\enddocument}{Use Hooks}%
\def\enddocument{%
%    \end{macrocode}
%    The |\end{document}| hook is executed first. If necessary it can
%    contain a |\clearpage| to output dangling floats first. In this
%    position it can also contain something like |\end{foo}| so that
%    the whole document effectively starts and ends with some special
%    environment. However, this must be used with care, eg if two
%    applications would use this without knowledge of each other the
%    order of the environments will be wrong after all.
%    |\AtEndDocument| is redefined
%    at this point so that and such commands that get into the hook do
%    not chase their tail\ldots
% \changes{v1.0y}{1995/04/27}{\cs{@checkend} moved after hook}
% \changes{v1.0z}{1995/07/13}{Set \cs{@setckpt} to \cs{@gobbletwo}
%                    instead of defining it by hand}
% \changes{v1.1i}{2000/05/19}
%            {Reset \cs{AtEndDocument} for latex/3060}
%    \begin{macrocode}
   \UseOneTimeHook{enddocument}%
   \@kernel@after@enddocument
%    \end{macrocode}
%    
%    \begin{macrocode}
   \@checkend{document}%
   \clearpage
   \UseOneTimeHook{enddocument/afterlastpage}%
   \@kernel@after@enddocument@afterlastpage
   \begingroup
     \if@filesw
       \immediate\closeout\@mainaux
       \let\@setckpt\@gobbletwo
       \let\@newl@bel\@testdef
%    \end{macrocode}
% \changes{v1.0z}{1995/07/13}{Shorten redefinition of \cs{bibcite} and
%          \cs{newlabel}}
%    The previous line is equiv to setting
%\begin{verbatim}
%       \def\newlabel{\@testdef r}%
%       \def\bibcite{\@testdef b}%
%\end{verbatim}
% \changes{v1.1k}{2010/08/17}{Use braces around \cs{input} arg (pr/4124)}
% \changes{v1.1l}{2010/08/17}{Change of plan: use \cs{@@input} instead
%                             (pr/4124)}
%    We use |\@@input| to load the \texttt{.aux} file, so that it doesn't
%    show up in the list of files produced by |\listfiles|.
%    \begin{macrocode}
       \@tempswafalse
       \makeatletter \@@input\jobname.aux
     \fi
     \UseHook{enddocument/afteraux}%
%    \end{macrocode}
%    Next hook is expect to contain only code for writing info
%    messages on the terminal.
%    \begin{macrocode}
     \UseOneTimeHook{enddocument/info}%
   \endgroup
   \UseOneTimeHook{enddocument/end}%
   \deadcycles\z@\@@end}
%    \end{macrocode}
%
%    The public hooks used in \cs{enddocument}:
%    \begin{macrocode}
\NewHook{enddocument}
\NewHook{enddocument/afterlastpage}
\NewHook{enddocument/afteraux}
\NewHook{enddocument/info}
\NewHook{enddocument/end}
%    \end{macrocode}
%
%    This is one of the few places where we already add data and rules
%    to a hook already in the kernel.
% \changes{v1.0w}{1994/11/30}{(DPC) Use \cs{@dofilelist}}
%    \begin{macrocode}
\AddToHook{enddocument/info}[kernel/filelist]{\@dofilelist}
\AddToHook{enddocument/info}[kernel/warnings]{\@enddocument@kernel@warnings}
\DeclareHookRule{enddocument/info}{kernel/filelist}{before}{kernel/warnings}
%    \end{macrocode}
%  \end{macro}
%  \end{environment}
%
%
%  \begin{macro}{\@kernel@after@enddocument,
%                \@kernel@after@enddocument@afterlastpage}
%    
%    The two kernel hooks above are used by the shipout code.   
%    \begin{macrocode}
\let\@kernel@after@enddocument\@empty
\let\@kernel@after@enddocument@afterlastpage\@empty
%    \end{macrocode}
%  \end{macro}
%
%  \begin{macro}{\@enddocument@kernel@warnings}
%    
%    \begin{macrocode}
\def\@enddocument@kernel@warnings{%
%    \end{macrocode}
%    First we check for font size substitution bigger than
%    |\fontsubfuzz|. The |\relax| is necessary because this is a macro
%    not a register.
% \changes{v1.0w}{1994/11/30}
%         {(DPC) Do warnings even for \cs{nofiles}}
%    \begin{macrocode}
   \ifdim \font@submax >\fontsubfuzz\relax
%    \end{macrocode}
%    In case you wonder about the |\@gobbletwo| inside the message
%    below, this is a horrible hack to remove the tokens |\on@line.|
%    that are added by |\@font@warning| at the end.
% \changes{v1.1j}{2000/07/11}{Fix typo in warning}
%    \begin{macrocode}
     \@font@warning{Size substitutions with differences\MessageBreak
                up to \font@submax\space have occurred.\@gobbletwo}%
   \fi
%    \end{macrocode}
%    The macro |\@defaultsubs| is initially |\relax| but gets redefined
%    to produce
%    a warning if there have been some default font substitutions.
% \changes{v1.0z}{1995/07/13}{Use \cs{@defaultsubs} instead of switch}
%    \begin{macrocode}
   \@defaultsubs
%    \end{macrocode}
%    The macro |\@refundefined| is initially |\relax| but gets redefined
%    to produce a warning if there are undefined refs.
% \changes{v1.1b}{1995/10/24}{Use \cs{@refundefined} instead of switch}
%    \begin{macrocode}
   \@refundefined
%    \end{macrocode}
%    If a label is defined more than once, |\@tempswa| will always be
%    true and thus produce a ``Label(s) may \ldots'' warning. But
%    since a rerun will not solve that problem (unless one uses a
%    package like \texttt{varioref} that generates labels on the fly),
%    we suppress this message.
% \changes{v1.0e}{1994/04/20}{Changed logic for producing
%                             warning messages}
% \changes{v1.1b}{1995/10/24}{Changed logic for producing
%                             warning messages and removed switch}
%    \begin{macrocode}
   \if@filesw
     \ifx \@multiplelabels \relax
       \if@tempswa
         \@latex@warning@no@line{Label(s) may have changed.
             Rerun to get cross-references right}%
       \fi
     \else
       \@multiplelabels
     \fi
     \ifx \@extra@page@added \relax
         \@latex@warning@no@line{Temporary extra page added at the end.
             Rerun to get it removed}%
     \fi
%    \end{macrocode}
%    We could think of adding a warning that nothing can be corrected
%    while \cs{nofiles} is in force. In the past the warnings related
%    to the \texttt{aux} file are simply suppressed in this case. 
%    \begin{macrocode}
   \fi
}
%    \end{macrocode}
%  \end{macro}
%
%    \begin{macrocode}
%</2ekernel|latexrelease>
%<latexrelease>\EndIncludeInRelease
%<latexrelease>\IncludeInRelease{0000/00/00}%
%<latexrelease>                 {\enddocument}{Use Hooks}%
%<latexrelease>
%<latexrelease>\def\enddocument{%
%<latexrelease>   \let\AtEndDocument\@firstofone
%<latexrelease>   \@enddocumenthook
%<latexrelease>   \@checkend{document}%
%<latexrelease>   \clearpage
%<latexrelease>   \begingroup
%<latexrelease>     \if@filesw
%<latexrelease>       \immediate\closeout\@mainaux
%<latexrelease>       \let\@setckpt\@gobbletwo
%<latexrelease>       \let\@newl@bel\@testdef
%<latexrelease>       \@tempswafalse
%<latexrelease>       \makeatletter \@@input\jobname.aux
%<latexrelease>     \fi
%<latexrelease>     \@dofilelist
%<latexrelease>     \ifdim \font@submax >\fontsubfuzz\relax
%<latexrelease>       \@font@warning{Size substitutions with differences\MessageBreak
%<latexrelease>                  up to \font@submax\space have occurred.\@gobbletwo}%
%<latexrelease>     \fi
%<latexrelease>     \@defaultsubs
%<latexrelease>     \@refundefined
%<latexrelease>     \if@filesw
%<latexrelease>       \ifx \@multiplelabels \relax
%<latexrelease>         \if@tempswa
%<latexrelease>           \@latex@warning@no@line{Label(s) may have changed.
%<latexrelease>               Rerun to get cross-references right}%
%<latexrelease>         \fi
%<latexrelease>       \else
%<latexrelease>         \@multiplelabels
%<latexrelease>       \fi
%<latexrelease>     \fi
%<latexrelease>   \endgroup
%<latexrelease>   \deadcycles\z@\@@end}
%<latexrelease>
%<latexrelease>\let\@kernel@after@enddocument\@undefined
%<latexrelease>\let\@kernel@after@enddocument@afterlastpage\@undefined
%<latexrelease>\let\@enddocument@kernel@warnings\@undefined
%<latexrelease>
%<latexrelease>\EndIncludeInRelease
%<*2ekernel>
%    \end{macrocode}
%
%
%
% \begin{macro}{\@testdef}
%    \begin{macrocode}
\def\@testdef #1#2#3{%
  \def\reserved@a{#3}\expandafter \ifx \csname #1@#2\endcsname
 \reserved@a  \else \@tempswatrue \fi}
%    \end{macrocode}
%  \end{macro}
%
%
% Reading data from auxiliary files (like \texttt{.toc} normally
% happens in vertical mode and it therefore doesn't matter if line
% endings are converted to spaces by \TeX{} during that process.
%
% However, especially the \texttt{.toc} file might be read in L-R mode
% (in cases the \cs{tableofcontents} attempts to put, say a list of
% sub-sections as a paragraph. In that case the newlines after a line
% like
% \begin{verbatim}
% \contentsline {subsubsection}{\numberline {1.1.1}A C-head}{2}
% \end{verbatim}
% might result in spurious spaces (e.g., when that level is not
% included).
%
% That could be fixed by reading in the file using
% \cs{endlinechar}\texttt{=-1} but that has the danger that it drops
% some valid endlines that should be converted to spaces (for example
% when the user edited the TOC and then used \cs{nofiles} to preserve
% it.
%
% So the approach taken instead is this:
% \begin{itemize}
% \item \cs{addcontentsline} adds the command
%    \cs{protected@file@percent} to the end of the second argument of
%    \cs{@writefile} that is written to the \texttt{.aux}. As the name
%    indicates this is a protected macro so it doesn't change if it is
%    written out.
%
% \item When the \texttt{.aux} is read back in at the end of the run,
%    \cs{@writefile} is executed and writes its second argument
%    unmodified to the file with the extension given by its first
%    argument. Or rather that was how it was in the past.
%
% \item Instead we change \cs{@writefile} slightly: basically it looks
%    at the second argument and if the last token in there is
%    \cs{protected@file@percent} then it is replaced by a percent
%    character and that is then written out. If not (for example, if
%    the data came from a user issued \cs{addtocontents}, or from some
%    package that uses \cs{@writefile} for writing its own files) then
%    the command behaves exactly as before.
% \end{itemize}
%
%
%  \begin{macro}{\protected@file@percent}
%    Dummy cs to be replaced by a percent sign inside
%    \cs{@writefile}. If it survives (when used incorrectly) it will
%    expand to nothing in a typsetting context.
% \changes{v1.1n}{2018/09/26}{Sometimes mask the endline char when
%    writing to files (github/73)}
%    \begin{macrocode}
%</2ekernel>
%<*2ekernel|latexrelease>
%<latexrelease>\IncludeInRelease{2018/12/01}%
%<latexrelease>                 {\protected@file@percent}{Mask line endings}%
\protected\def\protected@file@percent{}
%    \end{macrocode}
%  \end{macro}


%  \begin{macro}{\add@percent@to@temptokena}
%    Helper function which is used to inspect a sequence of tokens
%    (the second argument of \cs{@writefile} and it the last token is
%    \cs{protected@file@percent} it will replace it by a harmless
%    percent. The result is saved in \cs{@temptokena} for later use.
% \changes{v1.1n}{2018/09/26}{Sometimes mask the endline char when
%    writing to files (github/73)}
%
%    \begin{macrocode}
\catcode`\^^A=9
\long\gdef\add@percent@to@temptokena
    #1\protected@file@percent#2\add@percent@to@temptokena
%    \end{macrocode}
%    When we call this macro in |\@writefile| we stick in |\@empty| at
%   the beginning, so that in case the tokenlist consists of a single brace
%   group the braces aren't stripped. The |\expandafter| then expands
%   this extra token away again.
% \changes{v1.1q}{2019/10/25}{Allow unbalanced conditionals in \texttt{\#1} (gh/202)}
% \changes{v1.1r}{2019/11/10}{fix to special comment catcodes (gh/202)}
%    \begin{macrocode}
    {\expandafter\ifx\expandafter X\detokenize{#2}X\expandafter\dont@add@percent@to@temptokena\else
             \expandafter\do@add@percent@to@temptokena\fi{#1}}
%    \end{macrocode}
%
%    \begin{macrocode}
\long\def\dont@add@percent@to@temptokena#1{%
  \@temptokena\expandafter{#1}}
%    \end{macrocode}
%    \texttt{latexrelease} will read this code in high-speed
%    mode in certain situations. During that it will only look for
%    \cs{if} tests but not actually execute the \cs{catcode} change
%    above. As a result it will drop anything after the |%| character
%    in the definition. Therefore the |\fi| needs to be on the next
%    line and we need locally another comment character to avoid
%    getting spaces into the definition---a weird problem :-)
%
%    \begin{macrocode}
\begingroup
\catcode`\%=12
\catcode`\^^A=14
\long\gdef\do@add@percent@to@temptokena#1{\@temptokena\expandafter{#1%^^A
%    \end{macrocode}
%    Can't be on the same line as the |%| --- see above.
%    \begin{macrocode}
  }}
\endgroup
%    \end{macrocode}
%  \end{macro}
%
% \begin{macro}{\@writefile}
% \changes{v1.0l}{1994/05/20}{Added correct setting of \cs{protect}.}
% \changes{v1.0t}{1994/11/04}{Removed setting of \cs{protect}. ASAJ.}
% \changes{v1.0z}{1995/07/13}{Added missing percent and use \cs{relax}
%  in the THEN case}
% \changes{v1.1n}{2018/09/26}{Sometimes mask the endline char when
%    writing to files (github/73)}
%    \begin{macrocode}
\long\def\@writefile#1#2{%
  \@ifundefined{tf@#1}\relax
    {%
%    \end{macrocode}
%    If we write to the file we first prepare |#2| using
%    \cs{add@percent@to@temptokena} and then write the token register
%    out.
%    \begin{macrocode}
      \add@percent@to@temptokena
        \@empty#2\protected@file@percent
        \add@percent@to@temptokena
     \immediate\write\csname tf@#1\endcsname{\the\@temptokena}%
    }%
}
%    \end{macrocode}
%
%    \begin{macrocode}
%</2ekernel|latexrelease>
%<latexrelease>\EndIncludeInRelease
%<latexrelease>\IncludeInRelease{0000/00/00}%
%<latexrelease>                 {\protected@file@percent}{Mask line endings}%
%<latexrelease>\let\protected@file@percent\@undefined
%<latexrelease>\let\add@percent@to@temptokena\@undefined
%<latexrelease>\let\do@add@percent@to@temptokena\@undefined
%<latexrelease>\let\dont@add@percent@to@temptokena\@undefined
%<latexrelease>\long\def\@writefile#1#2{%
%<latexrelease>  \@ifundefined{tf@#1}\relax
%<latexrelease>    {\@temptokena{#2}%
%<latexrelease>     \immediate\write\csname tf@#1\endcsname{\the\@temptokena}%
%<latexrelease>    }%
%<latexrelease>}
%<latexrelease>\EndIncludeInRelease
%<*2ekernel>
%    \end{macrocode}
%  \end{macro}
%
% \begin{macro}{\stop}
%    \begin{macrocode}
\def\stop{\clearpage\deadcycles\z@\let\par\@@par\@@end}
%    \end{macrocode}
%  \end{macro}
%
%
% \begin{oldcomments}
%
%    \begin{macrocode}
\everypar{\@nodocument} %% To get an error if text appears before the
\nullfont               %% \begin{document}
%    \end{macrocode}
%
% \begin, \end, and \@checkend changed so \end{document} will catch
% an unmatched \begin.  Changed 24 May 89 as suggested by
% Frank Mittelbach and Rainer Sch\"opf.
%
% \begin{NAME} ==
%  BEGIN
%    IF \NAME undefined  THEN  \reserved@a == BEGIN report error END
%                        ELSE  \reserved@a ==
%                                   (\@currenvir :=L NAME) \NAME
%    FI
%    @ignore :=G F      %% Added 30 Nov 88
%    \begingroup
%    \@endpe := F
%    \@currenvir :=L NAME
%    \NAME
%  END
%
% \end{NAME} ==
%  BEGIN
%   \endNAME
%   \@checkend{NAME}
%   \endgroup
%   IF @endpe = T              %% @endpe set True by \@endparenv
%     THEN \@doendpe           %% \@doendpe redefines \par and \everypar
%                              %% to suppress paragraph indentation in
%   FI                         %% immediately following text
%   IF @ignore = T
%     THEN @ignore :=G F
%          \ignorespaces
%   FI
%  END
%
% \@checkend{NAME} ==
%  BEGIN
%   IF \@currenvir = NAME
%     ELSE \@badend{NAME}
%   FI
%  END
%
% \end{oldcomments}
%
%
% \begin{macro}{\begin}
% \changes{LaTeX2.09}{1992/03/18}{Changed \cs{@ignoretrue} to
%               \cs{@ignorefalse} (as documented)}
% \changes{LaTeX2.09}{1992/08/24}{Added code to \cs{begin} to
%      remember line number. Used by \cs{@badend} to display
%      position of non-matching \cs{begin}.}
% \changes{v1.1e}{1996/07/26}{remove \cs{global} before \cs{@ignore...}}
% \changes{v1.1p}{2019/08/27}{Make command robust}
%    \begin{macrocode}
%</2ekernel>
%<*2ekernel|latexrelease>
%<latexrelease>\IncludeInRelease{2020/10/01}%
%<latexrelease>                 {\begin}{Use hook system}%
\DeclareRobustCommand*\begin[1]{%
  \UseHook{env/#1/before}%
  \@ifundefined{#1}%
    {\def\reserved@a{\@latex@error{Environment #1 undefined}\@eha}}%
    {\def\reserved@a{\def\@currenvir{#1}%
        \edef\@currenvline{\on@line}%
        \@execute@begin@hook{#1}%
        \csname #1\endcsname}}%
  \@ignorefalse
  \begingroup\@endpefalse\reserved@a}
%    \end{macrocode}
%
%    Before the \cs{document} code is executed we have to first undo
%    the \cs{endgroup} as there should be none for this environment to
%    avoid that changes on top-level unnecessarily go to \TeX's
%    savestack, and we have to initialize all hooks in the hook system.
%    So we need to test for this environment name. But once it has be
%    found all this testing is no longer needed and so we redefine
%    \cs{@execute@begin@hook} to simply use the hook.
%    \begin{macrocode}
\def\@execute@begin@hook #1{
  \expandafter\ifx\csname #1\endcsname\document
    \endgroup
    \gdef\@execute@begin@hook##1{\UseHook{env/##1/begin}}
    \@expl@@initialize@all@
  \fi
%    \end{macrocode}
%    If this is an environment before \verb=\begin{document}= we just
%    run the hook so this can be outside the test.
%    \begin{macrocode}
  \UseHook{env/#1/begin}
}
%    \end{macrocode}
%
%    The top level definition for \cs{end}. for an explanation see
%    below (this is the same as the 2019 version where it was
%    introduced, but for rollback we have to repeat it). 
%    \begin{macrocode}
\edef\end
  {\unexpanded{%
     \romannumeral
       \ifx\protect\@typeset@protect
       \expandafter       %1
         \expandafter        %2
       \expandafter       %1
           \expandafter         %3 expands the \csname inside \end<space>
       \expandafter       %1
         \expandafter        %2  expands \end<space>
       \expandafter       %1     expands the \else
           \z@
       \else
         \expandafter\z@\expandafter\protect
       \fi
   }%
   \expandafter\noexpand\csname end \endcsname
  }
%    \end{macrocode}
%    Version that adds hooks (so different from the 2019 version). It
%    fixes tlb3722 but the change should perhaps be made in
%    \texttt{tabularx} instead.
%    \begin{macrocode}
\@namedef{end }#1{%
  \romannumeral
    \IfHookEmptyTF{env/#1/end}%
        {\expandafter\z@}%
        {\z@\UseHook{env/#1/end}}%
    \csname end#1\endcsname\@checkend{#1}%
    \expandafter\endgroup\if@endpe\@doendpe\fi
    \UseHook{env/#1/after}%
    \if@ignore\@ignorefalse\ignorespaces\fi
}
%    \end{macrocode}
%    Version without the fix for tlb3722 for the record:
%    \begin{macrocode}
%\@namedef{end }#1{%
%  \UseHook{env/#1/end}%
%  \csname end#1\endcsname\@checkend{#1}%
%  \expandafter\endgroup\if@endpe\@doendpe\fi
%  \UseHook{env/#1/after}%
%  \if@ignore\@ignorefalse\ignorespaces\fi}%
%    \end{macrocode}
%    
%    \begin{macrocode}
%</2ekernel|latexrelease>
%<latexrelease>\EndIncludeInRelease
%    \end{macrocode}
%    
%    \begin{macrocode}
%<latexrelease>\IncludeInRelease{2019/10/01}%
%<latexrelease>                 {\begin}{Making \begin/\end robust}%
%<latexrelease>\DeclareRobustCommand\begin[1]{%
%<latexrelease>  \@ifundefined{#1}%
%<latexrelease>    {\def\reserved@a{\@latex@error{Environment #1 undefined}\@eha}}%
%<latexrelease>    {\def\reserved@a{\def\@currenvir{#1}%
%<latexrelease>     \edef\@currenvline{\on@line}%
%<latexrelease>     \csname #1\endcsname}}%
%<latexrelease>  \@ignorefalse
%<latexrelease>  \begingroup\@endpefalse\reserved@a}
%    \end{macrocode}
%    A version that doesn't start out with \cs{relax} when in
%    typesetting mode would be the following, but since \cs{begin}
%    issues a \cs{begingroup} it wouldn't help much with respect to
%    allowing things like \cs{noalign} or \cs{multicolumn} inside.
%    \begin{macrocode}
%\edef\begin
%  {\unexpanded{%
%       \ifx\protect\@typeset@protect
%         \expandafter\@gobble
%       \fi
%       \protect
%     }%
%   \expandafter\noexpand\csname begin \endcsname
%  }
%\@namedef{begin }#1{%
%  \@ifundefined{#1}%
%    {\def\reserved@a{\@latex@error{Environment #1 undefined}\@eha}}%
%    {\def\reserved@a{\def\@currenvir{#1}%
%     \edef\@currenvline{\on@line}%
%     \csname #1\endcsname}}%
%  \@ignorefalse
%  \begingroup\@endpefalse\reserved@a}
%    \end{macrocode}
%
%  \begin{macro}{\end}
%    While \cs{begin} was made robust simply by using
%    \cs{DeclareRobustCommand} we need to be a bit more subtle with
%    \cs{end} as there are packages out there that try to look into
%    the top-level contents of \verb=\end{foo}= (that is at the expansion
%    of \verb=\endfoo=) to see if it contains certain macros. This is
%    done by hitting \verb=\end{foo}=  with three \cs{expandafter}s,
%    the first to get
%\begin{verbatim}
%  \csname endfoo\endcsname       \@checkend{foo}% etc.
%\end{verbatim}
%    the second to expand the \cs{csname}, i.e., to get to
%\begin{verbatim}
%  \endfoo                        \@checkend{foo}% etc.
%\end{verbatim}
%    and the third to finally get to the top-level content of
%    \verb=\endfoo=, i.e.
%\begin{verbatim}
%  <top-level content of \endfoo> \@checkend{foo}% etc.
%\end{verbatim}
%    Therefore a robust replacement should produce the same results
%    after three expansions (there first is obviously different).
%
%    Basically the definition of \cs{end} should either produce
%    \verb*=\protect\end = (when not doing typesetting) or it should
%    produce \verb*=\end = (without the \cs{protect}) when doing typesetting.
%    Furthermore, it should (when in typesetting mode) show exactly the
%    same result as
%    \verb*=\end = (which is the original fragile definition
%    of \cs{end}) when you expand either of them twice, i.e.,
%\begin{verbatim}
%  \endfoo                        \@checkend{foo}% etc.
%\end{verbatim}
%    That is achieved with the code below (which is worth studying
%    carefully).
%
%    There is some trickery involved here: in particular we use
%    \cs{romannumeral} to change a single expansion into three
%    successive expansions in one go. That primitive expands until it
%    has scanned a number (0 in this case, so it doesn't produce any
%    output) and so it allows us to place arbitrary many
%    \cs{expandafter}s inside that are all going to be executed when
%    \cs{romannumeral} is hit by a single \cs{expandafter}.
%
% \changes{v1.1e}{1996/07/26}{remove \cs{global} before \cs{@ignore...}}
% \changes{v1.1p}{2019/08/27}{Make command robust}
%
%    \begin{macrocode}
%<latexrelease>\edef\end
%<latexrelease>  {\unexpanded{%
%<latexrelease>     \romannumeral
%<latexrelease>       \ifx\protect\@typeset@protect
%<latexrelease>       \expandafter       %1
%<latexrelease>         \expandafter        %2
%<latexrelease>       \expandafter       %1
%<latexrelease>           \expandafter         %3 expands the \csname inside \end<space>
%<latexrelease>       \expandafter       %1
%<latexrelease>         \expandafter        %2  expands \end<space>
%<latexrelease>       \expandafter       %1     expands the \else
%<latexrelease>           \z@
%<latexrelease>       \else
%<latexrelease>         \expandafter\z@\expandafter\protect
%<latexrelease>       \fi
%<latexrelease>   }%
%<latexrelease>   \expandafter\noexpand\csname end \endcsname
%<latexrelease>  }
%    \end{macrocode}
%    And here is the original definition of \cs{end} the way it was in
%    \LaTeX{} for several decades now hidden in \verb*=\end =.
%    \begin{macrocode}
%<latexrelease>\@namedef{end }#1{%
%<latexrelease>  \csname end#1\endcsname\@checkend{#1}%
%<latexrelease>  \expandafter\endgroup\if@endpe\@doendpe\fi
%<latexrelease>  \if@ignore\@ignorefalse\ignorespaces\fi}
%<latexrelease>\EndIncludeInRelease
%    \end{macrocode}
%    An here the rollback in case that is ever needed.
%    \begin{macrocode}
%<latexrelease>\IncludeInRelease{0000/00/00}%
%<latexrelease>                 {\begin}{Making \begin/\end robust}%
%<latexrelease>\def\begin#1{%
%<latexrelease>  \@ifundefined{#1}%
%<latexrelease>    {\def\reserved@a{\@latex@error{Environment #1 undefined}\@eha}}%
%<latexrelease>    {\def\reserved@a{\def\@currenvir{#1}%
%<latexrelease>     \edef\@currenvline{\on@line}%
%<latexrelease>     \csname #1\endcsname}}%
%<latexrelease>  \@ignorefalse
%<latexrelease>  \begingroup\@endpefalse\reserved@a}
%<latexrelease>\def\end#1{%
%<latexrelease>  \csname end#1\endcsname\@checkend{#1}%
%<latexrelease>  \expandafter\endgroup\if@endpe\@doendpe\fi
%<latexrelease>  \if@ignore\@ignorefalse\ignorespaces\fi}
%<latexrelease>
%<latexrelease>\EndIncludeInRelease
%<*2ekernel>
%    \end{macrocode}
%  \end{macro}
%  \end{macro}
%
%
%
%
%
%  \begin{macro}{\@checkend}
%    \begin{macrocode}
\def\@checkend#1{\def\reserved@a{#1}\ifx
      \reserved@a\@currenvir \else\@badend{#1}\fi}
%    \end{macrocode}
%  \end{macro}
%
%  \begin{macro}{\@currenvline}
%    We do need a default value for |\@currenvline| on top-level since
%    the document environment cancels the brace group. This means that
%    a mismatch with |\begin|\allowbreak|{document}| will not produce
%    a line number. Thus the outer default must be |\@empty| or we
%    will end up with two spaces.
% \changes{v1.0q}{1994/05/24}{Use \cs{@empty} as outer default}
%    \begin{macrocode}
\let\@currenvline\@empty
%    \end{macrocode}
%  \end{macro}
%
%
%
%  \begin{macro}{\AtBeginEnvironment,\AtEndEnvironment,
%                \BeforeBeginEnvironment,\AfterEndEnvironment}
%    
%    We provide 4 high-level hook interfaces directly, the others only when
%    etoolbox is loaded
%    \begin{macrocode}
%</2ekernel>
%<*2ekernel|latexrelease>
%<latexrelease>\IncludeInRelease{2020/10/01}%
%<latexrelease>                 {\AtBeginEnvironment}{Hooks for environments}%
%    \end{macrocode}
%
%    \begin{macrocode}
\newcommand\AtBeginEnvironment[2][.]    {\AddToHook{env/#2/begin}[#1]}
\newcommand\AtEndEnvironment[2][.]      {\AddToHook{env/#2/end}[#1]}
\newcommand\BeforeBeginEnvironment[2][.]{\AddToHook{env/#2/before}[#1]}
\newcommand\AfterEndEnvironment[2][.]   {\AddToHook{env/#2/after}[#1]}
%    \end{macrocode}
%
%    \begin{macrocode}
%</2ekernel|latexrelease>
%<latexrelease>\EndIncludeInRelease
%    \end{macrocode}
%
%    \begin{macrocode}
%<latexrelease>\IncludeInRelease{0000/00/00}%
%<latexrelease>                 {\AtBeginEnvironment}{Hooks for environments}%
%<latexrelease>
%<latexrelease>\let\AtBeginEnvironment\@undefined
%<latexrelease>\let\AtEndEnvironment\@undefined
%<latexrelease>\let\BeforeBeginEnvironment\@undefined
%<latexrelease>\let\AfterEndEnvironment\@undefined
%<latexrelease>
%<latexrelease>\EndIncludeInRelease
%<*2ekernel>
%    \end{macrocode}
%  \end{macro}
%    
%
%
% \subsection{Center, Flushright, Flushleft}
%
%    \begin{macrocode}
\message{center,}
%    \end{macrocode}
%
% \begin{oldcomments}
%
% \center, \flushright and \flushleft set
%   \rightskip = 0pt or \@flushglue (as appropriate)
%   \leftskip  = 0pt or \@flushglue (as appropriate)
%   \parindent = 0pt
%   \parfillskip   = 0pt. (except \flushleft)
%   \\         == \par \vskip -\parskip
%   \\[LENGTH] == \\ \vskip LENGTH
%   \\*        == \par \penalty 10000 \vskip -\parskip
%   \\*[LEN]   == \\* \vskip LENGTH
%
% They invoke the trivlist environment to handle vertical spacing before
% and after them.
%
% \centering, \raggedright and \raggedleft are the declaration analogs
% of the above.
%
% \raggedright has a more universal effect, however.  It sets
% \@rightskip := flushglue.  Every environment, like the list
% environments,
% that set \rightskip to its 'normal' value set it to \@rightskip
%
% \end{oldcomments}
%
% \begin{macro}{\@centercr}
% \changes{v1.0h}{1994/05/03}{\cs{@badcrerr} replaced by \cs{@nolnerr}}
% \changes{v1.0z}{1995/07/13}{Use \cs{nobreak}}
% \changes{v1.1s}{2019/11/02}{Make \cs{@centercr} robust (gh/203)}
%    \begin{macrocode}
%</2ekernel>
%<*2ekernel|latexrelease>
%<latexrelease>\IncludeInRelease{2020/02/02}%
%<latexrelease>                 {\@centercr}{Make robust}%
\protected\def\@centercr{\ifhmode \unskip\else \@nolnerr\fi
       \par\@ifstar{\nobreak\@xcentercr}\@xcentercr}
%</2ekernel|latexrelease>
%    \end{macrocode}
%
%    \begin{macrocode}
%<latexrelease>\EndIncludeInRelease
%<latexrelease>\IncludeInRelease{0000/00/00}%
%<latexrelease>                 {\@centercr}{Make robust}%
%<latexrelease>
%<latexrelease>\def\@centercr{\ifhmode \unskip\else \@nolnerr\fi
%<latexrelease>       \par\@ifstar{\nobreak\@xcentercr}\@xcentercr}
%<latexrelease>
%<latexrelease>\EndIncludeInRelease
%<*2ekernel>
%    \end{macrocode}
% \end{macro}
%
% \begin{macro}{\@xcentercr}
%    \begin{macrocode}
\def\@xcentercr{\addvspace{-\parskip}\@ifnextchar
    [\@icentercr\ignorespaces}
%    \end{macrocode}
% \end{macro}
%
% \begin{macro}{\@icentercr}
% \changes{v1.1t}{2020/04/21}{Support calc syntax (gh/152)}
%    \begin{macrocode}
%</2ekernel>
%<*2ekernel|latexrelease>
%<latexrelease>\IncludeInRelease{2020/10/01}%
%<latexrelease>                 {\@icentercr}{centering, etc support calc}%
\def\@icentercr[#1]{\@vspace@calcify{#1}\ignorespaces}
%</2ekernel|latexrelease>
%<latexrelease>\EndIncludeInRelease
%    \end{macrocode}
%    
%    \begin{macrocode}
%<latexrelease>\IncludeInRelease{0000/00/00}%
%<latexrelease>                 {\@icentercr}{centering, etc support calc}%
%<latexrelease>
%<latexrelease>\def\@icentercr[#1]{\vskip #1\ignorespaces}
%<latexrelease>\EndIncludeInRelease
%<*2ekernel>
%    \end{macrocode}
% \end{macro}
%
%
% \begin{environment}{center}
% \changes{v0.9h}{1993/12/13}{Removed optional argument of \cs{item}}
% \changes{v1.0u}{1994/11/12}{Changed end macro to \cs{def}: safer and
% consistent}
%    We use |\relax| to prevent |\item| scanning too far.
%    \begin{macrocode}
\def\center{\trivlist \centering\item\relax}
%    \end{macrocode}
%
%    \begin{macrocode}
\def\endcenter{\endtrivlist}
%    \end{macrocode}
% \end{environment}
%
%
%    \begin{macrocode}
%</2ekernel>
%<*2ekernel|latexrelease>
%<latexrelease>\IncludeInRelease{2020/10/01}%
%<latexrelease>                 {\centering}{Set finaldhypendemerits}%
%    \end{macrocode}
%
% \begin{macro}{\centering}
% \changes{v1.1u}{2020/05/31}{Added \cs{finalhyphendemerits} setting (gh/247)}
%    \begin{macrocode}
\DeclareRobustCommand\centering{%
  \let\\\@centercr
  \rightskip\@flushglue\leftskip\@flushglue
  \finalhyphendemerits=\z@
  \parindent\z@\parfillskip\z@skip}
%    \end{macrocode}
% \end{macro}
%
% \begin{macro}{\raggedright}
% \changes{v1.1u}{2020/05/31}{Added \cs{finalhyphendemerits} setting (gh/247)}
%    \begin{macrocode}
\DeclareRobustCommand\raggedright{%
  \let\\\@centercr\@rightskip\@flushglue \rightskip\@rightskip
  \finalhyphendemerits=\z@
  \leftskip\z@skip
  \parindent\z@}
%    \end{macrocode}
% \end{macro}
%
% \begin{macro}{\raggedleft}
% \changes{v1.1u}{2020/05/31}{Added \cs{finalhyphendemerits} setting (gh/247)}
%    \begin{macrocode}
\DeclareRobustCommand\raggedleft{%
  \let\\\@centercr
  \rightskip\z@skip\leftskip\@flushglue
  \finalhyphendemerits=\z@
  \parindent\z@\parfillskip\z@skip}
%    \end{macrocode}
% \end{macro}
%
%    \begin{macrocode}
%</2ekernel|latexrelease>
%<latexrelease>\EndIncludeInRelease
%<latexrelease>\IncludeInRelease{2019/10/01}%
%<latexrelease>                 {\centering}{Make commands robust}%
%<latexrelease>
%<latexrelease>\DeclareRobustCommand\centering{%
%<latexrelease>  \let\\\@centercr
%<latexrelease>  \rightskip\@flushglue\leftskip\@flushglue
%<latexrelease>  \parindent\z@\parfillskip\z@skip}
%<latexrelease>\DeclareRobustCommand\raggedright{%
%<latexrelease>  \let\\\@centercr\@rightskip\@flushglue \rightskip\@rightskip
%<latexrelease>  \leftskip\z@skip
%<latexrelease>  \parindent\z@}
%<latexrelease>\DeclareRobustCommand\raggedleft{%
%<latexrelease>  \let\\\@centercr
%<latexrelease>  \rightskip\z@skip\leftskip\@flushglue
%<latexrelease>  \parindent\z@\parfillskip\z@skip}
%<latexrelease>\EndIncludeInRelease
%<latexrelease>
%<latexrelease>\IncludeInRelease{0000/00/00}%
%<latexrelease>                 {\centering}{Make commands robust}%
%<latexrelease>
%<latexrelease>\kernel@make@fragile\centering
%<latexrelease>\kernel@make@fragile\raggedright
%<latexrelease>\kernel@make@fragile\raggedleft
%<latexrelease>
%<latexrelease>\EndIncludeInRelease
%<*2ekernel>
%    \end{macrocode}
%
%
%
% \begin{macro}{\@rightskip}
%    \begin{macrocode}
\newskip\@rightskip \@rightskip \z@skip
%    \end{macrocode}
% \end{macro}
%
% \begin{environment}{flushleft}
% \changes{v0.9h}{1993/12/13}{Removed optional argument of \cs{item}}
% \changes{v1.0u}{1994/11/12}{Changed end macro to \cs{def}: safer and
% consistent}
%    We use |\relax| to prevent |\item| scanning too far.
%    \begin{macrocode}
\def\flushleft{\trivlist \raggedright\item\relax}
%    \end{macrocode}
%
%    \begin{macrocode}
\def\endflushleft{\endtrivlist}
%    \end{macrocode}
% \end{environment}
%
%
% \begin{environment}{flushright}
% \changes{v0.9h}{1993/12/13}{Removed optional argument of \cs{item}}
% \changes{v1.0u}{1994/11/12}{Changed end macro to \cs{def}: safer and
% consistent}
%    We use |\relax| to prevent |\item| scanning too far.
%    \begin{macrocode}
\def\flushright{\trivlist \raggedleft\item\relax}
%    \end{macrocode}
%
%    \begin{macrocode}
\def\endflushright{\endtrivlist}
%    \end{macrocode}
% \end{environment}
%
%
% \subsection{Verbatim}
%
%    \begin{macrocode}
\message{verbatim,}
%    \end{macrocode}
%
%  The verbatim environment uses the fixed-width |\ttfamily| font, turns
%  blanks into spaces, starts a new line for each carriage return (or
%  sequence of consecutive carriage returns), and interprets
%  \emph{every} character literally.
%  I.e., all special characters |\, {, $|, etc.
%   are |\catcode|'d to 'other'.
%
% The command |\verb| produces in-line verbatim text, where the argument
% is delimited by any pair of characters.  E.g., |\verb #...#| takes
%  `|...|' as its argument, and sets it verbatim in |\ttfamily| font.
%
%  The *-variants of these commands are the same, except that spaces
%  print as the \TeX{}book's space character instead of as blank spaces.
%
%  \begin{macro}{\@vobeyspaces}
%    \begin{macrocode}
{\catcode`\ =\active%
\gdef\@vobeyspaces{\catcode`\ \active\let \@xobeysp}}
%    \end{macrocode}
% \end{macro}
%
%  \begin{macro}{\@xobeysp}
% \changes{v1.0z}{1995/07/13}{Use \cs{nobreak}}
% \changes{v1.1f}{1996/09/28}{Moved to ltspace.dtx}
% \end{macro}
%
%
%  \begin{macro}{\@xverbatim}
%  \begin{macro}{\@sxverbatim}
%    \begin{macrocode}
\begingroup \catcode `|=0 \catcode `[= 1
\catcode`]=2 \catcode `\{=12 \catcode `\}=12
\catcode`\\=12 |gdef|@xverbatim#1\end{verbatim}[#1|end[verbatim]]
|gdef|@sxverbatim#1\end{verbatim*}[#1|end[verbatim*]]
|endgroup
%    \end{macrocode}
% \end{macro}
% \end{macro}
%
%  \begin{macro}{\@verbatim}
% \changes{LaTeX2.09}{1991/07/24}{Added \cs{penalty}\cs{interlinepenalty}
%               to definition of \cs{par} so that \cs{samepage} works}
% \changes{v0.9h}{1993/12/13}{Removed optional argument of \cs{item}}
%    Real start of verbatim environment
%    We use |\relax| to prevent |\item| scanning too far.
% \changes{v0.9p}{1994/01/18}
%         {Add \cs{global}\cs{@inlabelfalse}}
% \changes{v1.0b}{1994/03/16}
%         {Remove \cs{global}\cs{@inlabelfalse} again.}
%    \begin{macrocode}
%</2ekernel>
%<*2ekernel|latexrelease>
%<latexrelease>\IncludeInRelease{2017-04-15}{\@verbatim}%
%<latexrelease>                 {Disable hyphenation in verbatim}%
\def\@verbatim{\trivlist \item\relax
  \if@minipage\else\vskip\parskip\fi
  \leftskip\@totalleftmargin\rightskip\z@skip
  \parindent\z@\parfillskip\@flushglue\parskip\z@skip
%    \end{macrocode}
% \changes{LaTeX2.09}{1991/08/26}{\cs{@@par} added}
%    Added |\@@par| to clear possible |\parshape| definition
%    from a surrounding list (the verbatim guru says).
% \changes{v0.9p}{1994/01/18}
%         {Only add \cs{penalty} if in hmode}
% Switch language when in vertical mode.
%    \begin{macrocode}
  \@@par
%    \end{macrocode}
% Set |\language| here to suppress hyphenation. Done this way rather
% than setting |\hyphenchar| as that is a global setting.
%    \begin{macrocode}
  \language\l@nohyphenation
  \@tempswafalse
  \def\par{%
    \if@tempswa
%    \end{macrocode}
%    A |\leavevmode| added: needed if, for example, a blank verbatim
%    line is the first thing in a list item (wow!).
% \changes{v1.0f}{1994/04/29}{\cs{leavevmode} added}
%    \begin{macrocode}
      \leavevmode \null \@@par\penalty\interlinepenalty
    \else
      \@tempswatrue
      \ifhmode\@@par\penalty\interlinepenalty\fi
    \fi}%
%    \end{macrocode}
%    To allow customization we hide the font used in a separate macro.
%  \changes{v0.9a}{1993/11/21}{use \cs{verbatim@font} instead of \cs{tt}}
%  \changes{v0.9h}{1993/12/13}{Readded \cs{@noligs}}
%  \changes{v1.1d}{1996/06/03}{Exchanged the following two code lines
%           so that \cs{dospecials} cannot reset the category code
%           of characters handled by \cs{@noligs}.}
%  \changes{v1.1h}{2000/01/07}{Disable hyphenation even if the font allows it.}
%  \changes{v1.1m}{2017-03-09}{Use \cs{language} not \cs{hyphenchar}}
%    \begin{macrocode}
  \let\do\@makeother \dospecials
  \obeylines \verbatim@font \@noligs
%    \end{macrocode}
%    To avoid a breakpoint after the labels box, we remove the penalty
%    put there by the list macros: another use of |\unpenalty|!
% \changes{v1.0f}{1994/04/29}{Change to \cs{everypar} added}
%    \begin{macrocode}
  \everypar \expandafter{\the\everypar \unpenalty}%
}
%</2ekernel|latexrelease>
%<latexrelease>\EndIncludeInRelease
%<latexrelease>\IncludeInRelease{0000-00-00}{\@verbatim}%
%<latexrelease>                 {Disable hyphenation in verbatim}%
%<latexrelease>\def\@verbatim{\trivlist \item\relax
%<latexrelease>  \if@minipage\else\vskip\parskip\fi
%<latexrelease>  \leftskip\@totalleftmargin\rightskip\z@skip
%<latexrelease>  \parindent\z@\parfillskip\@flushglue\parskip\z@skip
%<latexrelease>  \@@par
%<latexrelease>  \@tempswafalse
%<latexrelease>  \def\par{%
%<latexrelease>    \if@tempswa
%<latexrelease>      \leavevmode \null \@@par\penalty\interlinepenalty
%<latexrelease>    \else
%<latexrelease>      \@tempswatrue
%<latexrelease>      \ifhmode\@@par\penalty\interlinepenalty\fi
%<latexrelease>    \fi}%
%<latexrelease>  \let\do\@makeother \dospecials
%<latexrelease>  \obeylines \verbatim@font \@noligs
%<latexrelease>  \hyphenchar\font\m@ne
%<latexrelease>  \everypar \expandafter{\the\everypar \unpenalty}%
%<latexrelease>}
%<latexrelease>\EndIncludeInRelease
%<*2ekernel>
%    \end{macrocode}
%  \end{macro}
%
%  \begin{environment}{verbatim}
%  \begin{macro}{\verbatim}
%  \begin{macro}{\endverbatim}
%    (RmS 93/09/19) Protected against `missing item' error message
%               triggered by empty verbatim environment.
%    \begin{macrocode}
\def\verbatim{\@verbatim \frenchspacing\@vobeyspaces \@xverbatim}
\def\endverbatim{\if@newlist \leavevmode\fi\endtrivlist}
%    \end{macrocode}
%  \end{macro}
%  \end{macro}
%  \end{environment}
%
%  \begin{macro}{\verbatim@font}
% \changes{v0.9a}{1993/11/21}{Macro added}
%    Macro to select the font  used for verbatim typesetting.
%    It also does other work if necessary for the font used.
% \changes{v0.9s}{1994/01/21}{Removed unnecessary category code
%                            hackery.}
%    \begin{macrocode}
\def\verbatim@font{\normalfont\ttfamily}
%    \end{macrocode}
%  \end{macro}
%
%
%    \begin{macrocode}
%</2ekernel>
%<*2ekernel|latexrelease>
%<latexrelease>\IncludeInRelease{2018/12/01}%
%<latexrelease>                 {\verbvisiblespace}{Setup visible space for \verb}%
%    \end{macrocode}
%
%
%  \begin{macro}{\asciispace}
%    The character in slot 32, in typewriter fonts (historically) a
%    visible space but in other fonts a real space or something else
% \changes{v1.1o}{2018/10/11}
%     {Provide visible space in \cs{verb*} also for XeTeX and LuaTeX (github/69)}
%    \begin{macrocode}
\DeclareRobustCommand\asciispace{\char 32 }
%    \end{macrocode}
%  \end{macro}
%
%  \begin{macro}{\verbvisiblespace}
%    This defines how to get a visible space in
%    |\verb*| and friends. In classic \TeX{} this is just the
%    slot 32, but in TU encoded fonts we switch fonts and take the
%    character from cmtt.
% \changes{v1.1o}{2018/10/11}
%     {Provide visible space in \cs{verb*} also for XeTeX and LuaTeX (github/69)}
% \changes{v1.1o}{2018/10/11}
%     {Provide \cs{verbvisiblespace} such that it is usable in normal text (github/70)}
%    \begin{macrocode}
\ifx\Umathcode\@undefined
  \let\verbvisiblespace\asciispace                                % Pdftex version
\else
  \DeclareRobustCommand\verbvisiblespace
           {\leavevmode{\usefont{OT1}{cmtt}{m}{n}\asciispace}}    % xetex/luatex version
\fi
%    \end{macrocode}
%  \end{macro}
%
%

%  \begin{macro}{\@setupverbvisiblespace}
%
%    In pdf\TeX{} a catcode 12 space will produce the character in
%   slot 32 which is assumed to be a visible space character (in a
%   typewriter font in OT1 or T1 encoding). In Xe\TeX{} or Lua\TeX{} a
%   font in TU encoding is normally used and that has a real space in
%   this slot. So what we do in this case is this: we check the
%   definition of
%   |\verbvisiblespace| and if it is |\asciispace| we assume that the
%   char32 can be used (e.g., in pdf\TeX{}). We then redefine
%   |\@xobeysp| so that after running |\@vobeyspaces| we get
%   characters from slot 32 for each active space.
%
% \changes{v1.1o}{2018/10/11}
%     {Provide visible space in \cs{verb*} also for XeTeX and LuaTeX (github/69)}
%    \begin{macrocode}
\def\@setupverbvisiblespace{%
  \ifx\verbvisiblespace\asciispace
    \let\@xobeysp\asciispace
  \else
%    \end{macrocode} Otherwise we measure the width of a character in
%   the mon-spaced current font and place a
%   |\verbvisiblespace| into a box of the right width which we are then
%   using as the character for a space. By default this will be the space
%   character from OT1 cmtt but by changing
%   |\verbvisiblespace| one could use, for example, the |\textvisiblespace|
%   of the current typewriter font.
%    \begin{macrocode}
    \setbox\z@\hbox{x}%
    \setbox\@verbvisiblespacebox\hbox to\wd\z@{\hss\verbvisiblespace\hss}%
    \def\@xobeysp{\leavevmode\copy\@verbvisiblespacebox}%
  \fi
}
%    \end{macrocode}
%  \end{macro}
%

%  \begin{macro}{\@verbvisiblespacebox}
%    The box to hold the visible space character if it isn't in slot
%    32 in the current typewriter font.
% \changes{v1.1o}{2018/10/11}
%     {Provide visible space in \cs{verb*} also for XeTeX and LuaTeX (github/69)}
%    \begin{macrocode}
\newbox\@verbvisiblespacebox
%    \end{macrocode}
%  \end{macro}
%
%^^A  \@sverb was here
%
%  \begin{environment}{verbatim*}
%    For \texttt{verbatim*} we also set up the correct visible space
%    character definition and then run |\@vobeyspaces|. As this code
%    is not called as part of the normal verbatim environment (the
%    method is done the other way around this time) we don't have to
%    check if space is already active---it shouldn't be.
% \changes{v1.1o}{2018/10/11}
%     {Provide visible space in \cs{verb*} also for XeTeX and LuaTeX (github/69)}
%    \begin{macrocode}
\@namedef{verbatim*}{\@verbatim
  \@setupverbvisiblespace
  \frenchspacing\@vobeyspaces\@sxverbatim}
\expandafter\let\csname endverbatim*\endcsname =\endverbatim
%    \end{macrocode}
%  \end{environment}
%
%
%
%    \begin{macrocode}
%</2ekernel|latexrelease>
%<latexrelease>\EndIncludeInRelease
%<latexrelease>\IncludeInRelease{0000/00/00}%
%<latexrelease>                 {\verbvisiblespace}{Setup visible space for \verb}%
%<latexrelease>
%<latexrelease>\@namedef{verbatim*}{\@verbatim\@sxverbatim}
%<latexrelease>
%<latexrelease>\let\asciispace            \@undefined
%<latexrelease>\let\verbvisiblespace      \@undefined
%<latexrelease>\let\@setupverbvisiblespace\@undefined
%<latexrelease>\let\@verbvisiblespacebox  \@undefined
%<latexrelease>\EndIncludeInRelease
%<*2ekernel>
%    \end{macrocode}
%
%
%
% \begin{macro}{\@sverb}
% \begin{macro}{\@@sverb}
% \changes{v1.0j}{1994/05/10}{Slight change in error message text.}
% Definitions of |\@sverb| and |\@verb| changed so |\verb+ foo+|
% does not lose leading blanks when it comes at the beginning of a line.
% Change made 24 May 89. Suggested by Frank Mittelbach and Rainer
% Sch\"opf.
%    \begin{macrocode}
%</2ekernel>
%<*2ekernel|latexrelease>
%<latexrelease>\IncludeInRelease{2020/10/01}%
%<latexrelease>                 {\@sverb}{Drop spaces before \verb delimiter}%
%    \end{macrocode}
%    
%    If the users types \verb=\verb !~! foo= then surprisingly we would
%    get the space as the delimiter and thus
%    ``\verb=!~!foo='' in the output.
%    To avoid this scenario we check if  \verb=#1= has the character
%    code of a space, if so we recurse otherwise we call \cs{@@sverb}
%    (which is the original definition of \cs{@sverb}.
% \changes{v1.1u}{2020/04/22}{Drop spaces before \cs{verb} delimiter (gh/327)}
%    \begin{macrocode}
\def\@sverb#1{\if\noexpand#1 \expandafter\@sverb\else\@@sverb{#1}\fi}
%    \end{macrocode}
%    
%    \begin{macrocode}
\def\@@sverb#1{%
  \catcode`#1\active
  \lccode`\~`#1%
  \gdef\verb@balance@group{\verb@egroup
     \@latex@error{\noexpand\verb illegal in command argument}\@ehc}%
  \aftergroup\verb@balance@group
  \lowercase{\let~\verb@egroup}%
%    \end{macrocode}
%    If |\@sverb| is called from |\@verb| then space is already active
%    and supposed to produce a real space. In this case we do
%    nothing. Otherwise we run |\@setupverbvisiblespace| to setup the
%    right visible space char and afterwards |\@vobeyspaces| to make
%    it the definition for the active space character.
% \changes{v1.1o}{2018/10/11}
%     {Provide visible space in \cs{verb*} also for XeTeX and LuaTeX (github/69)}
%    \begin{macrocode}
  \ifnum\catcode`\ =\active
  \else  \@setupverbvisiblespace \@vobeyspaces \fi
}
%    \end{macrocode}
%    
%    \begin{macrocode}
%</2ekernel|latexrelease>
%<latexrelease>\EndIncludeInRelease
%<latexrelease>\IncludeInRelease{2018/12/01}%
%<latexrelease>                 {\@sverb}{Setup visible space for \verb}%
%<latexrelease>
%<latexrelease>\def\@sverb#1{%
%<latexrelease>  \catcode`#1\active
%<latexrelease>  \lccode`\~`#1%
%<latexrelease>  \gdef\verb@balance@group{\verb@egroup
%<latexrelease>     \@latex@error{\noexpand\verb illegal in command argument}\@ehc}%
%<latexrelease>  \aftergroup\verb@balance@group
%<latexrelease>  \lowercase{\let~\verb@egroup}%
%<latexrelease>  \ifnum\catcode`\ =\active
%<latexrelease>  \else  \@setupverbvisiblespace \@vobeyspaces \fi
%<latexrelease>}
%<latexrelease>\let\@@sverb\@undefined
%<latexrelease>\EndIncludeInRelease
%<latexrelease>
%<latexrelease>\IncludeInRelease{0000/00/00}%
%<latexrelease>                 {\@sverb}{Setup visible space for \verb}%
%<latexrelease>\def\@sverb#1{%
%<latexrelease>  \catcode`#1\active
%<latexrelease>  \lccode`\~`#1%
%<latexrelease>  \gdef\verb@balance@group{\verb@egroup
%<latexrelease>     \@latex@error{\noexpand\verb illegal in command argument}\@ehc}%
%<latexrelease>  \aftergroup\verb@balance@group
%<latexrelease>  \lowercase{\let~\verb@egroup}}%
%<latexrelease>
%<latexrelease>\EndIncludeInRelease
%<*2ekernel>
%    \end{macrocode}
%  \end{macro}
%  \end{macro}
%
%
%  \begin{macro}{\@makeother}
%    \begin{macrocode}
\def\@makeother#1{\catcode`#112\relax}
%    \end{macrocode}
%  \end{macro}
%
% \begin{macro}{\verb@balance@group}
% \changes{LaTeX2.09}{1993/09/07}
%     {(RmS) Changed definition of \cs{verb} so that it detects a
%              missing second delimiter.}
%    \begin{macrocode}
\let\verb@balance@group\@empty
%    \end{macrocode}
%  \end{macro}
%
% \begin{macro}{\verb@egroup}
%    \begin{macrocode}
\def\verb@egroup{\global\let\verb@balance@group\@empty\egroup}
%    \end{macrocode}
%  \end{macro}
%
% \begin{macro}{\verb@eol@error}
%    \begin{macrocode}
\begingroup
  \obeylines%
  \gdef\verb@eol@error{\obeylines%
    \def^^M{\verb@egroup\@latex@error{%
            \noexpand\verb ended by end of line}\@ehc}}%
\endgroup
%    \end{macrocode}
% \end{macro}
%
%  \begin{macro}{\verb}
% \changes{LaTeX2.09}{1992/08/24}
%         {Changed \cs{verb} and \cs{@sverb} to work correctly
%            in math mode}
% \changes{v0.9a}{1993/11/21}{Use \cs{verbatim@font} instead of
%                             \cs{tt}.}
% \changes{v1.1a}{1995/09/19}{Put \cs{@noligs} after
%                    \cs{verbatim@font} where it belongs.}
%    Typesetting a small piece verbatim.
%  \changes{v1.1d}{1996/06/03}{Put setting of verbatim font after
%           \cs{dospecials}
%           so that \cs{dospecials} cannot reset the category code
%           of characters handled by \cs{@noligs}.}
%  \changes{v1.1m}{2017/03/09}
%         {Use \cs{language} to stop hyphenation}
%    \begin{macrocode}
%</2ekernel>
%<*2ekernel|latexrelease>
%<latexrelease>\IncludeInRelease{2017-04-15}{\verb}%
%<latexrelease>                 {Disable hyphenation in verb}%
\def\verb{\relax\ifmmode\hbox\else\leavevmode\null\fi
  \bgroup
    \verb@eol@error \let\do\@makeother \dospecials
    \verbatim@font\@noligs
%    \end{macrocode}
% Set |\language| here to suppress hyphenation. Done this way rather
% than setting |\hyphenchar| as that is a global setting.
%    \begin{macrocode}
    \language\l@nohyphenation
    \@ifstar\@sverb\@verb}
%</2ekernel|latexrelease>
%<latexrelease>\EndIncludeInRelease
%<latexrelease>\IncludeInRelease{0000-00-00}{\verb}%
%<latexrelease>                 {Disable hyphenation in verb}%
%<latexrelease>\def\verb{\relax\ifmmode\hbox\else\leavevmode\null\fi
%<latexrelease>  \bgroup
%<latexrelease>    \verb@eol@error \let\do\@makeother \dospecials
%<latexrelease>    \verbatim@font\@noligs
%<latexrelease>    \@ifstar\@sverb\@verb}
%<latexrelease>\EndIncludeInRelease
%<*2ekernel>
%    \end{macrocode}
%  \end{macro}
%
%
%
% \begin{macro}{\@verb}
%    \begin{macrocode}
\def\@verb{\@vobeyspaces \frenchspacing \@sverb}
%    \end{macrocode}
% \end{macro}
%
%  \begin{macro}{\verbatim@nolig@list}
% \changes{LaTeX2.09}{1993/09/03}
%         {Replaced \cs{@noligs} by extensible list}
%    \begin{macrocode}
\def\verbatim@nolig@list{\do\`\do\<\do\>\do\,\do\'\do\-}
%    \end{macrocode}
%  \end{macro}
%
%  \begin{macro}{\do@noligs}
%    \begin{macrocode}
\def\do@noligs#1{%
  \catcode`#1\active
  \begingroup
     \lccode`\~`#1\relax
     \lowercase{\endgroup\def~{\leavevmode\kern\z@\char`#1}}}
%    \end{macrocode}
%  \end{macro}
%
%  \begin{macro}{\@noligs}
%    To stay compatible with packages that use |\@noligs| we keep it.
% \changes{v0.9h}{1993/12/13}{Readded \cs{@noligs}}
%    \begin{macrocode}
\def\@noligs{\let\do\do@noligs \verbatim@nolig@list}
%    \end{macrocode}
%  \end{macro}
%
% \changes{v0.9i}{1993/12/16}{\cs{literal} added}
% \changes{v1.0r}{1994/05/26/16}{\cs{literal} removed}
%    \begin{macrocode}
%</2ekernel>
%    \end{macrocode}
%
% \Finale
%
