% \iffalse meta-comment
%
% Copyright (C) 1993-2023
% The LaTeX Project and any individual authors listed elsewhere
% in this file.
%
% This file is part of the LaTeX base system.
% -------------------------------------------
%
% It may be distributed and/or modified under the
% conditions of the LaTeX Project Public License, either version 1.3c
% of this license or (at your option) any later version.
% The latest version of this license is in
%    https://www.latex-project.org/lppl.txt
% and version 1.3c or later is part of all distributions of LaTeX
% version 2008 or later.
%
% This file has the LPPL maintenance status "maintained".
%
% The list of all files belonging to the LaTeX base distribution is
% given in the file `manifest.txt'. See also `legal.txt' for additional
% information.
%
% The list of derived (unpacked) files belonging to the distribution
% and covered by LPPL is defined by the unpacking scripts (with
% extension .ins) which are part of the distribution.
%
% \fi
%
%
% \iffalse
%%% From File: ltdirchk.dtx
%
%<*unstripped>
\ifx\documentclass\undefined\let\next\relax\else\def\next{%
%</unstripped>
%<*driver>
% \fi
% \fi
\ProvidesFile{ltdirchk.dtx}
             [2024/02/11 v1.3a LaTeX Kernel (System Dependent Parts)]
% \iffalse
% \iffalse
\documentclass{ltxdoc}
\GetFileInfo{ltdirchk.dtx}
\title{\filename\thanks{%
 this document also includes the source for
 texsys.cfg and ltxcheck.tex}}
\author{David Carlisle}
\date{\filedate}
\begin{document}
 \MaintainedByLaTeXTeam{latex}
 \maketitle
 \DocInput{\filename}
\end{document}
%</driver>
%<*unstripped>
}\fi\next
%</unstripped>
% \fi
%
% \changes{v0.2j}{1994/02/25}
%         {Remove need for drv file}
% \changes{v0.2k}{1994/03/01}
%         {Add unstripped module,
%          so that dircheck.dtx may be used with initex}
% \changes{v1.0a}{1994/03/08}
%         {Reorganize driver module into `new style'}
% \changes{v1.0b}{1994/03/12}
%         {Change name from dircheck.dtx}
% \changes{v1.0b}{1994/03/12}
%         {Minor edits to the typeouts in ltxcheck}
% \changes{v1.0d}{1994/03/28}
%         {Improve documentation}
% \changes{v1.0i}{1994/11/03}
%         {Generate an error if latex.ltx not used with clean initex}
% \changes{v1.0j}{1994/11/17}
%         {\cs{@tempa} to \cs{reserved@a}}
% \changes{v1.0r}{1996/06/13}
%         {documentation improvements mainly from internal/2174}
% \changes{v1.0v}{1997/06/16}
%         {documentation improvements mainly from internal/2520}
% \changes{v1.0w}{1998/08/17}{(RmS) Documentation improvements.}
%
%
% \section{\LaTeX\ System Dependent Initializations}
%
% \changes{v0.2g}{1994/01/21}
%         {Improve documentation, reorganize docstrip module}
%
% This file implements the semi-automatic determination of various
% system dependent parts of the initialization. The actual definitions
% may be placed in a file |texsys.cfg|. Thus for operating systems for
% which the tests here do not result in acceptable settings, a `hand
% written' |texsys.cfg| may be produced.
%
% The macros that must be defined are:
%
% \DescribeMacro{\@currdir}
% |\@currdir|\meta{filename}\meta{space} should expand to a form of the
% filename that uniquely refers to the `current directory' if this is
% possible. (The expansion should also end with a space.) on UNIX,
% this is |\def\@currdir{./}|. For more exotic operating systems you may
% want to make |\@currdir| a macro with arguments delimited by |.|
% and/or \meta{space}. If the operating system has no concept of
% directory structure, this macro should be defined to be empty.
%
%
% \DescribeMacro{\input@path}
% If the primitive |\openin| searches the same directories as the
% primitive |\input|, then it is possible to tell (using |\ifeof|)
% whether a file exists before trying to input it. For systems like
% this, |\input@path| should be left undefined.
%
% If |\openin| does not `follow' |\input| then |\input@path| must be
% defined to be a list of directories to search for input files. The
% format for each directory is as for |\@currdir|, normally just a
% prefix is required, but it may be a macro with space-delimited
% argument. That is, if \meta{dir} is an entry in the input path,
% \TeX\ will try to load the expansion of
%   \meta{dir}\meta{filename}\meta{space}
%
% So either \meta{dir} should be defined as a macro with argument
% delimited by space, or it should just expand to a directory name,
% including the final directory separator, so that it may be
% concatenated with the \meta{filename}. This means that for UNIX-like
% syntax, each \meta{dir} should end with a slash, |/|.
%
% |\input@path| should expand to a list of such directories, each in a
% |{}| group.
%
% \begin{sloppypar}
% \DescribeMacro{\filename@parse}
% After a call of the form: |\filename@parse{|\meta{filename}|}|, the
% three macros |\filename@area|, |\filename@base| and |\filename@ext| should
% be defined to be the `area' (or directory), basename and
% extension respectively. If there was no extension specified in
% \meta{filename}, |\filename@ext| should be |\let| to |\relax| (so this
% case may be tested with |\@ifundefined{filename@ext}| and, perhaps a
% default extension substituted).
% \end{sloppypar}
%
% Normally one would not need to define this macro in |texsys.cfg| as
% the automatic tests can supply parsers that work with UNIX and VMS and
% Macintosh syntax, as well as a basic parser that will cover many other
% cases. However some operating systems may need a `hand produced'
% parser in which case it should be defined in this file.
%
% The UNIX parser also works for most MSDOS \TeX\ versions.
% Currently if the UNIX, VMS or Macintosh parser is not used,
% |\filename@parse| is  defined to always return an empty area, and to
% split the argument into  basename and extension at the first `|.|'
% that occurs in the name.
% Parsers for other formats may be defined in |texsys.cfg|,
% in which case they will be used in preference to the default
% definitions.
%
% \DescribeMacro{\@TeXversion}
% \changes{v1.0f}{1994/05/23}{Document \cs{@TeXversion}}
% |\@TeXversion| is now set automatically by the initialization tests
% in this file. You should not need to set it in |texsys.cfg|, however
% the following documentation is left for information. \LaTeX\ does
% not set this variable exactly, the automatic tests set it to:\\
%  |2| for any version, $v$,  $v < 3.0$\\
%  |3| for any version, $v$,  $3.0 \leq v \leq 3.14$\\
%  \meta{undefined} otherwise.\\
% However these values are accurate enough for \LaTeX\ to take
% appropriate action for these old \TeX{}s.
%
% If your \TeX\ is older than version 3.141, then you should define
% |\@TeXversion| (using |\def|) to be the version number. If you do not
% do this\footnote
%       {Actually if your \TeX\ is really old, version~2, \LaTeX\ can
%       detect this, and sets \cs{@TeXversion} to~2 if it is not set in
%       the \texttt{cfg} file.}
%, \LaTeX\ will not work around a bug in old \TeX\ versions, and
% so error messages will appear in a very strange format, with |^^J|
% appearing instead of line breaks:
% \begingroup
%  ^^A FMi next line doesn't work with hacked class
%  ^^A\catcode`\==\active \def=#1#2{}\hfuzz\maxdimen
%\begin{verbatim}
%! LaTeX Error: \rubbish undefined.^^J^^JSee the LaTeX manual or LaTeX=
% Companion
% for explanation.^^JType  H <return>  for immediate help.
% ...
%
%l.3 \renewcommand{\rubbish}
%                           {}
%?
%\end{verbatim}
% \endgroup\noindent
% However if you put |\def\@TeXversion{3.14}| in \texttt{texsys.cfg}
% the following format will be used:
%\begin{verbatim}
%! LaTeX Error: \rubbish undefined.
%
%See the LaTeX manual or LaTeX Companion for explanation.
%Type  H <return>  for immediate help.
%! .
% ...
%
%l.3 \renewcommand{\rubbish}
%                           {}
%?
%\end{verbatim}
% Note that this has an extra line |! .| which does not appear in
% error messages that use the default settings with a current version of
% \TeX, but this should not cause any confusion we hope.
%
% \MaybeStop{}
%
% \section{Initialization}
% As this file is read at a very early stage, some definitions that
% are normally considered to be part of the format must be made here.
%
% Most such definitions are repeated later in the ``right'' place,
% usually (but not always) with different implementations. To be able
% to spot this more easily if you look into the file
% \texttt{latex.ltx} (which is stripped of comments) we add some
% comment lines to that effect that survive the stripping process by
% \texttt{docstrip}.
% \changes{v1.3a}{2021/12/11}{Add comment lines into latex.ltx to
%     indicate temp definitions that are later overwritten (gh/725)}
%    \begin{macrocode}
%<*dircheck>
%% ---- START temporary definitions for bootstrapping; later overwritten ----
%</dircheck>
%    \end{macrocode}
%
% \subsection{INITEX}
% \changes{v0.2i}{1994/01/25}
%         {Protect against looping on \cs{@@input} and \cs{@@end}.}
%    \begin{macrocode}
%<*dircheck>
%<*initex>
%<initex>\ifnum\catcode`\{=1
%<initex>  \errmessage
%<initex>  {LaTeX must be made using an initex with no format preloaded}
%<initex>\fi
\catcode`\{=1
\catcode`\}=2
%    \end{macrocode}
% \changes{v1.1}{2015/01/03}{Enable extra primitives when Lua\TeX{} is used}
% \changes{v1.2}{2015/08/23}{Do not use luatex prefix}
% \changes{v1.2a}{2015/10/02}{Allow backing out of unprefixed names}
%    If Lua\TeX{} is in use the extensions and other new primitives
%    have to be activated: this is done as early as possible. Older
%    versions of Lua\TeX{} do not hide the primitives: a version check is
%    not needed as the version itself will be missing in the case where
%    action is needed!
%    \begin{macrocode}
\ifx\directlua\undefined
\else
  \ifx\luatexversion\undefined
%    \end{macrocode}
% Enable e-TeX/pdfTeX/Umath primitives with their natural names
%    \begin{macrocode}
    \directlua{tex.enableprimitives("",%
                 tex.extraprimitives('etex', 'pdftex', 'umath'))}
%    \end{macrocode}
%
% In current formats enable primitives with unprefixed names.
% the \textsf{latexrelease} guards allow the primitives to be
% defined with a |\luatex| prefix if older formats are specified.
%    \begin{macrocode}
%</initex>
%</dircheck>
%<*initex,latexrelease>
%<latexrelease>\ifx\directlua\undefined\else
%<latexrelease>\IncludeInRelease{2015/10/01}{\luatexluafunction}
%<latexrelease>                             {LuaTeX (prefixed names)}%
    \directlua{tex.enableprimitives("",%
                 tex.extraprimitives("omega", "aleph", "luatex"))}
%<latexrelease>\EndIncludeInRelease
%<latexrelease>\IncludeInRelease{0000/00/00}{\luatexluafunction}
%<latexrelease>                             {LuaTeX (prefixed names)}%
%<latexrelease>\directlua{
%<latexrelease>  tex.enableprimitives(
%<latexrelease>    "luatex",
%<latexrelease>    tex.extraprimitives("core","omega", "aleph", "luatex")
%<latexrelease>  )
%<latexrelease>  local i
%<latexrelease>  local t = { }
%<latexrelease>  for _,i in pairs(tex.extraprimitives("luatex")) do
%<latexrelease>    if not string.match(i,"^U") then
%<latexrelease>      if not string.match(i, "^luatex") then
%<latexrelease>        table.insert(t,i)
%<latexrelease>      end
%<latexrelease>    else
%<latexrelease>      if string.match(i,"^Uchar$") then
%<latexrelease>        table.insert(t,i)
%<latexrelease>      end
%<latexrelease>    end
%<latexrelease>  end
%<latexrelease>  for _,i in pairs(t) do
%<latexrelease>    tex.print(
%<latexrelease>      "\noexpand\\let\noexpand\\" .. i
%<latexrelease>        .. "\noexpand\\undefined"
%<latexrelease>    )
%<latexrelease>  end
%<latexrelease>}
%<latexrelease>\EndIncludeInRelease
%<latexrelease>\fi
%</initex,latexrelease>
%<*dircheck>
%<*initex>
%    \end{macrocode}
%
%    \begin{macrocode}
  \fi
\fi
%    \end{macrocode}
%
% \changes{v1.2b}{2016/10/15}{Require e\TeX{}}
% A test can now be made for e\TeX{}.
%    \begin{macrocode}
%<initex>\ifx\eTeXversion\undefined
%<initex>  \errmessage
%<initex>    {LaTeX requires e-TeX}
%<initex>  \expandafter\endinput
%<initex>\fi
%    \end{macrocode}
%
% That distraction over, back to the basics of a format.
%    \begin{macrocode}
\catcode`\#=6
\catcode`\^=7
\chardef\active=13
\catcode`\@=11
\countdef\count@=255
\let\bgroup={ \let\egroup=}
\ifx\@@input\@undefined\let\@@input\input\fi
\ifx\@@end\@undefined\let\@@end\end\fi
\chardef\@inputcheck0
\chardef\sixt@@n=16
\newlinechar`\^^J
\def\typeout{\immediate\write17}
\def\dospecials{\do\ \do\\\do\{\do\}\do\$\do\&%
  \do\#\do\^\do\_\do\%\do\~}
\def\@makeother#1{\catcode`#1=12\relax}
\def\space{ }
\def\@tempswafalse{\let\if@tempswa\iffalse}
\def\@tempswatrue{\let\if@tempswa\iftrue}
\let\if@tempswa\iffalse
\def\loop#1\repeat{\def\iterate{#1\relax\expandafter\iterate\fi}%
  \iterate \let\iterate\relax}
\let\repeat\fi
%</initex>
%    \end{macrocode}
%
% \subsection{Some bits of 2e}
%    \begin{macrocode}
%<*2ekernel>
\def\two@digits#1{\ifnum#1<10 0\fi\number#1}
\long\def\@firstoftwo#1#2{#1}
\long\def\@secondoftwo#1#2{#2}
%    \end{macrocode}
% \changes{v1.0e}{1994/05/11}
%         {Add \cs{ProvidesFile} as used in fd files.}
% \changes{v1.0l}{1995/10/17}
%         {Modify initex version of \cs{ProvidesFile}}
% \changes{v1.0n}{1995/11/01}
%         {Initialise \cs{@addtofilelist} to \cs{@gobble}}
% This is a special version of |\ProvidesFile| for initex use.
% \changes{v1.0x}{2001/05/25}{Explicitly set catcode of
%                              \cs{endlinechar} to 10 (pr/3334)}
% \changes{v1.0y}{2001/06/04}{But only if it is a char (pr/3334)}
%    \begin{macrocode}
\def\ProvidesFile#1{%
  \begingroup
    \catcode`\ 10 %
    \ifnum \endlinechar<256 %
      \ifnum \endlinechar>\m@ne
        \catcode\endlinechar 10 %
      \fi
    \fi
    \@makeother\/%
    \@ifnextchar[{\@providesfile{#1}}{\@providesfile{#1}[]}}
\def\@providesfile#1[#2]{%
    \wlog{File: #1 #2}%
    \@addtofilelist{ #2}%
    \endgroup}
\long\def\@addtofilelist#1{}
%    \end{macrocode}
%
%    \begin{macrocode}
\def\@empty{}
\catcode`\%=12
\def\@percentchar{%}
\catcode`\%=14
\let\@currdir\@undefined
\let\input@path\@undefined
\let\filename@parse\@undefined
%    \end{macrocode}
% \begin{macro}{\strip@prefix}
% \changes{v0.2a}{1993/12/13}
%         {modified, name changed from \cs{stripmeaning}.}
% \changes{v0.2e}{1994/01/19}
%         {name changed from \cs{strip@meaning}, to match NFSS.}
%    \begin{macrocode}
\def\strip@prefix#1>{}
%</2ekernel>
%    \end{macrocode}
% \end{macro}
%
% \section{texsys.cfg}
% As mentioned above, any site specific definitions required to describe
% the filename handling must be entered into a file |texsys.cfg|. If
% |texsys.cfg| can not be located by |\openin|, we write a default
% version out. The default version only contains comments, so we do not
% actually input the file in that case. The automatic tests later will,
% hopefully, correctly define the required macros.
%
% The tricky code below checks to see if |texsys.cfg| exists. If it does
% not, all the text in this file between START and END is copied
% verbatim to a new file |texsys,cfg|. If |texsys.cfg| is found, then it
% is simply input. This is only done when this file is being used
% unstripped.
%    \begin{macrocode}
%<*docstrip>
\openin15=texsys.cfg
\ifeof15
\typeout{** Writing a default texsys.cfg}
\immediate\openout15=texsys.cfg
\begingroup
\catcode`\^^M\active%
\let^^M\par%
\def\reserved@a#1^^M{%
 \def\reserved@b{#1}%
 \ifx\reserved@b\reserved@c\endgroup\else%
     \immediate\write15{#1}%
     \expandafter\reserved@a\fi}%
\def\reserved@d#1START^^M{\let\do\@makeother\dospecials\reserved@a}%
\catcode`\%=12
\def\reserved@c{%END}
\reserved@d
%    \end{macrocode}
%START
% \subsection{texsys.cfg}
%
% This file contains the site specific definitions of the four macros\\
% |\@currdir|, |\input@path|, |\filename@parse| and |\@TeXversion|.
%
% As distributed it only contains comments, however this `empty'
% file will work on many systems because of the automatic tests built
% into |ltdirchk.dtx|. You \emph{are} allowed to edit this file to add
% definitions of these macros appropriate to your system.
%
%
% The macros that must be defined are:
%
% \DescribeMacro{\@currdir}
% |\@currdir|\meta{filename}\meta{space} should expand to a form of the
% filename that uniquely refers to the `current directory' if this is
% possible. (The expansion should also end with a space.) on UNIX,
% this is |\def\@currdir{./}|. For more exotic operating systems you may
% want to make |\@currdir| a macro with arguments delimited by |.|
% and/or \meta{space}. If the operating system has no concept of
% directory structure, this macro should be defined to be empty.
%
%
% \DescribeMacro{\input@path}
% If the primitive |\openin| searches the same directories as the
% primitive |\input|, then it is possible to tell (using |\ifeof|)
% whether a file exists before trying to input it. For systems like
% this, |\input@path| should be left undefined.
%
% If |\openin| does not `follow' |\input| then |\input@path| must be
% defined to be a list of directories to search for input files. The
% format for each directory is as for |\@currdir|, normally just a
% prefix is required, but it may be a macro with space-delimited
% argument. That is, if \meta{dir} is an entry in the input path,
% \TeX will try to load the expansion of
%
%   \meta{dir}\meta{filename}\meta{space}
%
% So either \meta{dir} should be defined as a macro with argument
% delimited by space, or it should just expand to a directory name,
% including the final directory separator, so that it may be
% concatenated with the \meta{filename}. This means that for UNIX-like
% syntax, each \meta{dir} should end with a slash, |/|. One exception to
% this rule is that the input path should \emph{always} contain the
% empty directory |{}| as this will allow `full pathnames' to be used,
% and the  `current directory' to be searched.
%
% |\input@path| should expand to a list of such directories, each in a
% |{}| group.
%
%
% \begin{sloppypar}
% \DescribeMacro{\filename@parse}
% After a call of the form: |\filename@parse{|\meta{filename}|}|, the
% three macros |\filename@area|, |\filename@base|, |\filename@ext| should
% be defined to be the `area' (or directory), basename and
% extension respectively. If there was no extension specified in
% \meta{filename}, |\filename@ext| should be |\let| to |\relax| (so this
% case may be tested with |\@ifundefined{filename@ext}| and, perhaps a
% default extension substituted).
% \end{sloppypar}
%
% Normally one would not need to define this macro in |texsys.cfg| as
% the automatic tests can supply parsers that work with UNIX and VMS
% syntax, as well as a basic parser that will cover many other cases.
% However some operating systems may need a `hand produced' parser
% in which case it should be defined in this file.
%
% The UNIX parser also works for most MSDOS \TeX\ versions.
% Currently if the UNIX or VMS parser is not used, |\filename@parse| is
% defined to always return an empty area, and to split the argument into
% basename and extension at the first `|.|' that occurs in the name.
% Parsers for other formats may be defined in |texsys.cfg|,
% in which case they will be used in preference to the default
% definitions.
%
%
% \DescribeMacro{\@TeXversion}
% You should not need to set this macro in |texsys.cfg|. \LaTeX\
% tests to set this automatically. See the comments in the opening
% section of \texttt{ltdirchk.dtx}.
%
%
% The following sections give examples of definitions which might
% work on various systems. These are currently mainly untested as I only
% have access to a few systems, all of which do not need this file as
% the automatic tests work. All the code is commented out.
%
% \subsection{UNIX (web2c)}
% This implementation does make |\openin| and |\input| look in the same
% places. Acceptable settings are made by |ltdirchk.dtx|, and so this
% file may be empty. The definitions below are therefore just for
% information.
%    \begin{macrocode}
%\def\@currdir{./}
%\let\input@path\@undefined
%    \end{macrocode}
%
% \subsection{UNIX (other)}
% Apparently some commercial UNIX implementations have different paths
% for |\openin| and |\input|. For these one could use definitions like
% the following (with whatever directories are used at your site):
% note that the directory names should end with |/|.
%    \begin{macrocode}
% \def\@currdir{./}
% \def\input@path{%
%   {/usr/local/lib/tex/inputs/distrib/}%
%   {/usr/local/lib/tex/inputs/contrib/}%
%   {/usr/local/lib/tex/inputs/local/}%
% }
%    \end{macrocode}
%
% \subsection{MSDOS (emtex)}
% This implementation does make |\openin| and |\input| look in the same
% places. Acceptable settings are made by |ltdirchk.dtx|, and so this
% file may be empty. The definitions below are therefore just for
% information.
%    \begin{macrocode}
% \def\@currdir{./}
% \let\input@path\@undefined
%    \end{macrocode}
%
% \subsection{MSDOS (other)}
% Some PC implementations have different paths
% for |\openin| and |\input|. For these one could use definitions like
% the following (with whatever directories are used at your site):
% note that the directory names should end with |/|.
% This assumes the implementation uses UNIX style |/| as the directory
% separator.
%    \begin{macrocode}
% \def\@currdir{./}
% \def\input@path{%
%   {c:/tex/inputs/distrib/}%
%   {c:/tex/inputs/contrib/}%
%   {c:/tex/inputs/local/}%
% }
%    \end{macrocode}
%
% \subsection{VMS (DECUS \TeX, PD VMS 3.6)}
% This implementation does make |\openin| and |\input| look in the same
% places. Acceptable settings are made by |ltdirchk.dtx|, and so this
% file may be empty. The definitions below are therefore just for
% information.
%    \begin{macrocode}
% \def\@currdir{[]}
% \let\input@path\@undefined
%    \end{macrocode}
%
% \subsection{VMS (???)}
% Some VMS implementations have different paths
% for |\openin| and |\input|. For these one could use definitions like
% the following:
%    \begin{macrocode}
% \def\@currdir{[]}
% \def\input@path{%
%   {tex_inputs:}%
%   {SOMEDISK:[SOME.TEX.DIRECTORY]}%
% }
%    \end{macrocode}
%
% \subsection{MACINTOSH (OzTeX 1.6)}
% This implementation does make |\openin| and |\input| look in the same
% places. Acceptable settings are made by |ltdirchk.dtx|, and so this
% file may be empty. The definitions below are therefore just for
% information.
%    \begin{macrocode}
% \def\@currdir{:}
% \let\input@path\@undefined
%    \end{macrocode}
%
% \subsection{MACINTOSH (other)}
% Some Macintosh implementations have different paths
% for |\openin| and |\input|. For these one could use definitions like
% the following (with whatever folders are used on your machine):
% note that the directory names should end with |:|, and they should
% contain {\em no\/} spaces.
%    \begin{macrocode}
% \def\@currdir{:}
% \def\input@path{%
%   {Hard-Disk:Applications:TeX:TeX-inputs:}%
%   {Hard-Disk:Applications:TeX:My-inputs:}%
% }
%    \end{macrocode}
%
% \subsection{FAKE EXAMPLE}
%  This example is for an operating system that has filenames of the
%  form |<area>name|  For maximum compatibility with macro sets,
%  you want |name.ext| to be mapped to |<ext>name|.
%  and |<area>name.ext| to be mapped to |<area.ext>name|.
%  |\input| does this mapping automatically, but |\openin| does not, and
%  does not look in the same places as |\input|.
%  |<>name| is the desired `current directory' syntax.
%
% the following code would possibly work:
%    \begin{macrocode}
% \def\@dir#1#2 {%
%   \@d@r{#1}#2..\@nil}
% \def\@d@r#1#2.#3.#4\@nil{%
%   <\ifx\@dir#1\@dir\else#1\ifx\@dir#3\@dir\else.\fi\fi#3>#2 }
%
% \def\@currdir{\@dir{}}
% \def\input@path{%
%   {\@dir{area.one}}%
%   {\@dir{area.two}}%
% }
%    \end{macrocode}
%END
%    \begin{macrocode}
\immediate\closeout15
%    \end{macrocode}
% If |texsys.cfg| did exist, then input it.
%    \begin{macrocode}
\else
\typeout{** Using the existing texsys.cfg}
\closein15
\input texsys.cfg
\fi
%</docstrip>
%    \end{macrocode}
%
% If the stripped version of this file is being used (in latex2e.ltx)
% then texsys.cfg should be there, so just input it.
%    \begin{macrocode}
%<dircheck>\input texsys.cfg
%    \end{macrocode}
%
% \changes{v0.2f}{1994/01/20}
%         {\cs{@copytexsys} and the texsys.new file removed}
%
% \section{Setting \texttt{\cs{@currdir}}}
%
% \begin{macro}{\@currdir}
% \begin{macro}{\IfFileExists}
% \changes{v0.2e}{1994/01/19}
%         {name changed from \cs{test}}
% This is a local definition of |\IfFileExists|. It tries to relocate
% |texsxys.aux|. If it succeeds, then the |\@currdir| syntax has been
% determined. If all the tests fail then |\@currdir| will be set to
% |\@empty|, and |ltxcheck| will warn of this when it checks the format.
%    \begin{macrocode}
\begingroup
\count@\time
\divide\count@ 60
\count2=-\count@
\multiply\count2 60
\advance\count2 \time
%    \end{macrocode}
%
% \begin{macro}{\today}
% The current date and time stamp.
% \changes{v0.2g}{1994/01/21}
%         {Name changed from \cs{stamp}, to save memory}
%    \begin{macrocode}
\edef\today{%
  \the\year/\two@digits{\the\month}/\two@digits{\the\day}:%
    \two@digits{\the\count@}:\two@digits{\the\count2}}
%    \end{macrocode}
% \end{macro}
%
% Create a file |texsys.aux| (hopefully in the current directory),
% then try to locate it again.
%    \begin{macrocode}
\immediate\openout15=texsys.aux
\immediate\write15{\today^^J}
\immediate\closeout15 %
%    \end{macrocode}
%
% |#1| is the file to try, |#2| is what to do on success, |#3| on
% failure. Note that this definition is overwritten later on again!
%    \begin{macrocode}
\def\IfFileExists#1#2#3{%
  \openin\@inputcheck#1 %
  \ifeof\@inputcheck
     #3\relax
  \else
    \read\@inputcheck to \reserved@a
    \ifx\reserved@a\today
      \typeout{#1 found}#2\relax
    \else
      \typeout{BAD: old file \reserved@a (should be \today)}%
      #3\relax
    \fi
  \fi
  \closein\@inputcheck}
%    \end{macrocode}
%
%    \begin{macrocode}
\endlinechar=-1
%    \end{macrocode}
%
% If |\@currdir| has not been pre-defined in |texsys.cfg| then test for
%  UNIX, VMS and  Oz-\TeX-Mac. syntax.
% \changes{v0.2h}{1994/01/24}
%         {Stop testing once texsys.aux has been found}
%    \begin{macrocode}
\ifx\@currdir\@undefined
  \IfFileExists{./texsys.aux}{\gdef\@currdir{./}}%
    {\IfFileExists{[]texsys.aux}{\gdef\@currdir{[]}}%
      {\IfFileExists{:texsys.aux}{\gdef\@currdir{:}}{}}}
%    \end{macrocode}
% If it is still undefined at this point, all the above tests failed.
% Earlier versions interactively prompted for a definition at this
% point, but it seems impossible to reliably obtain information from
% users at this point in the installation. This version of the file
% produces a format with no user-interaction. Later if the format is not
% suitable for the system, |texsys.cfg| may be edited and the format
% re-made.
% \changes{v0.2a}{1993/12/13}
%         {Removed interactive prompting for current directory syntax}
% \changes{v0.2f}{1994/01/20}
%         {\cs{@copytexsys} removed}
% \changes{v1.0u}{1996/12/06}
%         {*** removed from various messages for GNU Make.
%         internal/2338}
%    \begin{macrocode}
  \ifx\@currdir\@undefined
    \global\let\@currdir\@empty
    \typeout{^^J^^J%
      !! No syntax for the current directory could be found^^J%
      }%
  \fi
%    \end{macrocode}
% Otherwise |\@currdir| was defined in |texsys.cfg|. In this case check
% that the syntax specified works on this system. (In case a complete
% \LaTeX\  system has been copied from  one system to another.) If the
% test fails, give up. The installer should remove or correct the
% offending |texsys.cfg| and try again.
%    \begin{macrocode}
\else
  \IfFileExists{\@currdir texsys.aux}{}{%
    \edef\reserved@a{\errhelp{%
      texsys.cfg specifies the current directory syntax to be^^J%
      \meaning\@currdir^^J%
      but this does not work on this system.^^J%
      Remove texsys.cfg and restart.}}\reserved@a
    \errmessage{Bad texsys.cfg file: \noexpand\@currdir}\@@end}
%    \end{macrocode}
% The version of |\@currdir| in |texsys.cfg| looks OK.
%    \begin{macrocode}
\fi
%    \end{macrocode}
% \changes{v0.2d}{1994/01/14}
%         {Close the texsys.aux output stream}
%    \begin{macrocode}
\immediate\closeout15 %
\endgroup
%    \end{macrocode}
%
%    \begin{macrocode}
\typeout{^^J^^J%
         \noexpand\@currdir set to:
           \expandafter\strip@prefix\meaning\@currdir.^^J%
         }
%    \end{macrocode}
% \end{macro}
% \end{macro}
%
%
% \changes{v0.2a}{1993/12/13}
%         {on the `docstrip' pass, do not check openin path}
%
% Stop here if the file is being used unstripped.
%    \begin{macrocode}
%<*docstrip>
\relax\endinput
%</docstrip>
%    \end{macrocode}
%
% \section{Setting \texttt{\cs{input@path}}}
%
% Earlier versions of this file attempted to automatically test whether
% |\input@path| was required, and interactively prompt for a path if
% necessary. This was not found to be very reliable The first-time
% installer of \LaTeXe\ can not be expected to have enough information
% to supply the correct information to the prompts. Now the interaction
% is omitted. After the format is made the installer can attempt to run
% the test document |ltxcheck.tex| through \LaTeXe. This will check,
% among other things, whether |texsys.cfg| will need to be edited and
% the format remade.
%
% \begin{macro}{\input@path}
% Now set up the |\input@path|.
%
% |\input@path| should either be undefined, or a list of directories as
% described in the introduction.
% \changes{v0.2e}{1994/01/19}
%         {No longer check that an empty group is in the path}
%    \begin{macrocode}
  \typeout{^^J%
    Assuming \noexpand\openin and \noexpand\input^^J%
    \ifx\input@path\@undefined
%    \end{macrocode}
% |\input@path| has not been pre-defined.
%    \begin{macrocode}
      have the same search path.^^J%
    \else
%    \end{macrocode}
% |\input@path| has  been defined in |texsys.cfg|.
%    \begin{macrocode}
      have different  search paths.^^J%
      LaTeX will use the path specified by \noexpand\input@path:^^J%
    \fi
    }
%    \end{macrocode}
% \end{macro}
%
% \section{Filename Parsing}
%
% \begin{macro}{\filename@parse}
% Split a filename into its components.
% \changes{v0.2g}{1994/01/21}
%         {Minor changes, and add Mac version (:)}
%    \begin{macrocode}
\ifx\filename@parse\@undefined
  \def\reserved@a{./}\ifx\@currdir\reserved@a
%    \end{macrocode}
% |\filename@parse| was not specified in |texsys.cfg|, but |\@currdir|
% looks like UNIX\ldots
%    \begin{macrocode}
    \typeout{^^JDefining UNIX/DOS style filename parser.^^J}
    \def\filename@parse#1{%
      \let\filename@area\@empty
      \expandafter\filename@path#1/\\}
%    \end{macrocode}
%
% Search for the last |/|.
%    \begin{macrocode}
    \def\filename@path#1/#2\\{%
      \ifx\\#2\\%
         \def\reserved@a{\filename@simple#1.\\}%
      \else
         \edef\filename@area{\filename@area#1/}%
         \def\reserved@a{\filename@path#2\\}%
      \fi
      \reserved@a}
%    \end{macrocode}
%
%    \begin{macrocode}
  \else\def\reserved@a{[]}\ifx\@currdir\reserved@a
%    \end{macrocode}
% |\filename@parse| was not specified in |texsys.cfg|, but |\@currdir|
% looks like VMS\ldots
%    \begin{macrocode}
    \typeout{^^JDefining VMS style filename parser.^^J}
    \def\filename@parse#1{%
      \let\filename@area\@empty
      \expandafter\filename@path#1]\\}
%    \end{macrocode}
%
% Search for the last |]|.
%    \begin{macrocode}
    \def\filename@path#1]#2\\{%
      \ifx\\#2\\%
         \def\reserved@a{\filename@simple#1.\\}%
      \else
         \edef\filename@area{\filename@area#1]}%
         \def\reserved@a{\filename@path#2\\}%
      \fi
      \reserved@a}
%    \end{macrocode}
%
%    \begin{macrocode}
  \else\def\reserved@a{:}\ifx\@currdir\reserved@a
%    \end{macrocode}
% |\filename@parse| was not specified in |texsys.cfg|, but |\@currdir|
% looks like Macintosh\ldots
%    \begin{macrocode}
    \typeout{^^JDefining Mac style filename parser.^^J}
    \def\filename@parse#1{%
      \let\filename@area\@empty
      \expandafter\filename@path#1:\\}
%    \end{macrocode}
%
% Search for the last |:|.
% \changes{v1.0g}{1994/05/25}
%         {Mac parser had " typo for :}
%    \begin{macrocode}
    \def\filename@path#1:#2\\{%
      \ifx\\#2\\%
         \def\reserved@a{\filename@simple#1.\\}%
      \else
         \edef\filename@area{\filename@area#1:}%
         \def\reserved@a{\filename@path#2\\}%
      \fi
      \reserved@a}
%    \end{macrocode}
%
%    \begin{macrocode}
  \else
%    \end{macrocode}
% |\filename@parse| was not specified in |texsys.cfg|.
% So just make a simple parser that always sets |\filename@area| to
% empty.
%    \begin{macrocode}
    \typeout{^^JDefining generic filename parser.^^J}
    \def\filename@parse#1{%
      \let\filename@area\@empty
      \expandafter\filename@simple#1.\\}
  \fi\fi\fi
%    \end{macrocode}
%
% |\filename@simple| is used by all three versions.
% Finally we can split off the extension.
% \changes{v1.3a}{2019/11/01}
%         {take last . not first}
%    \begin{macrocode}
%</dircheck>
%<*dircheck,latexrelease>
%<latexrelease>\IncludeInRelease{2019/10/01}{\filename@simple}
%<latexrelease>                             {Final dot for extension}%
\def\filename@simple#1.#2\\{%
  \ifx\\#2\\%
    \let\filename@ext\relax
    \edef\filename@base{#1}%
  \else
    \filename@dots{#1}#2\\%
  \fi}
%    \end{macrocode}
%
%    \begin{macrocode}
\def\filename@dots#1#2.#3\\{%
  \ifx\\#3\\%
    \def\filename@ext{#2}%
    \edef\filename@base{#1}%
  \else
    \filename@dots{#1.#2}#3\\%
  \fi}
%<latexrelease>\EndIncludeInRelease
%<latexrelease>\IncludeInRelease{0000/00/00}{\filename@simple}
%<latexrelease>                             {Final dot for extension}%
%<latexrelease>  \def\filename@simple#1.#2\\{%
%<latexrelease>    \ifx\\#2\\%
%<latexrelease>       \let\filename@ext\relax
%<latexrelease>    \else
%<latexrelease>       \edef\filename@ext{\filename@dot#2\\}%
%<latexrelease>    \fi
%<latexrelease>    \edef\filename@base{#1}}
%<latexrelease>\EndIncludeInRelease
%</dircheck,latexrelease>
%<*dircheck>
%    \end{macrocode}
%
% Remove a final dot, added earlier.
%    \begin{macrocode}
  \def\filename@dot#1.\\{#1}
%    \end{macrocode}
%
%    \begin{macrocode}
\else
%    \end{macrocode}
% Otherwise, |\filename@parse| was specified in |texsys.cfg|.
%    \begin{macrocode}
  \typeout{^^J^^J%
    \noexpand\filename@parse was defined in texsys.cfg:^^J%
    \expandafter\strip@prefix\meaning\filename@parse.^^J%
    }
\fi
%    \end{macrocode}
% \end{macro}
%
% \section{\TeX\ Versions}
%
% \begin{macro}{\@TeXversion}
% \TeX\ versions older than 3.141 require |\@TeXversion| to be
% set. This can be determined automatically due to a trick suggested
% by Bernd Raichle. (Actually this will not always get the correct
% version number, eg \TeX3.14 would be detected as \TeX3, but \LaTeX\
% only needs to take account of \TeX's older than 3, or between 3 and
% 3.14.
% \changes{v1.0h}{1994/10/11}
%         {Check for TeX3.14}
%    \begin{macrocode}
\ifx\@TeXversion\@undefined
  \ifx\@undefined\inputlineno
    \def\@TeXversion{2}
  \else
   {\catcode`\^^J=\active
     \def\reserved@a#1#2\@@{\if#1\string^3\fi}
     \edef\reserved@a{\expandafter\reserved@a\string^^J\@@}
     \ifx\reserved@a\@empty\else\gdef\@TeXversion{3}\fi}
  \fi
\fi
%    \end{macrocode}
% \end{macro}
%
%    \begin{macrocode}
%% ---- END temporary definitions for bootstrapping ----
%</dircheck>
%    \end{macrocode}
%
% \section{ltxcheck.tex}
% After the format has been made, and article.cls moved with the
% other files to the `standard input directory' as specified in
% |install.txt|, the format may be checked by running the file
% |ltxcheck.tex|.
% \changes{v0.2f}{1994/01/20}
%         {Modify all of ltxcheck}
% \changes{v1.0h}{1994/10/11}
%         {Modify all of ltxcheck again}
% \changes{v1.0t}{1996/09/25}
%         {Move ltxcheck to separate file}
%
%
%
% \Finale
%
