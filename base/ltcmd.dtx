% \iffalse meta-comment
%
% Copyright (C) 1999 Frank Mittelbach, Chris Rowley, David Carlisle
% Copyright (C) 2004-2008 Frank Mittelbach, The LaTeX3 Project
% Copyright (C) 2009-2021
% The LaTeX Project and any individual authors listed elsewhere
% in this file.
%
% This file is part of the LaTeX base system.
% -------------------------------------------
%
% It may be distributed and/or modified under the
% conditions of the LaTeX Project Public License, either version 1.3c
% of this license or (at your option) any later version.
% The latest version of this license is in
%    https://www.latex-project.org/lppl.txt
% and version 1.3c or later is part of all distributions of LaTeX
% version 2008 or later.
%
% This file has the LPPL maintenance status "maintained".
%
% The list of all files belonging to the LaTeX base distribution is
% given in the file `manifest.txt'. See also `legal.txt' for additional
% information.
%
% The list of derived (unpacked) files belonging to the distribution
% and covered by LPPL is defined by the unpacking scripts (with
% extension .ins) which are part of the distribution.
%
% \fi
%
% \iffalse
%
%%% From File: ltcmd.dtx
%
%    \begin{macrocode}
\def\ltcmdversion{v1.0f}
\def\ltcmddate{2021-06-04}
%    \end{macrocode}
%
%<*driver>
% \fi
\ProvidesFile{ltcmd.dtx}
  [\ltcmddate\space \ltcmdversion\space
   LaTeX Kernel (Document commands)]
% \iffalse
\documentclass{l3doc}
\GetFileInfo{ltcmd.dtx}
\title{\filename}
\date{\filedate}
\author{Frank Mittelbach, Chris Rowley, David Carlisle, \LaTeX{} Project Team}
\begin{document}
  \maketitle
  \DocInput{ltcmd.dtx}
\end{document}
%</driver>
% \fi
%
% \section{Creating document commands}
%
% \changes{v1.0a}{2020/11/20}{Initial version derived from \texttt{xparse.dtx}}
%
% Document commands should be created using the tools provided by this module:
% \cs{NewDocumentCommand}, etc.\@, in almost all cases. This allows clean
% separation of document-level syntax from code-level interfaces. Users have
% a need to create new document commands, and as such a significant amount of
% documentation for \pkg{ltcmd} is provided as part of \texttt{usrguide3}. Here,
% additional material aimed at programmers is provided
%
% \StopEventually{}
%
%    \begin{macrocode}
%<@@=cmd>
%    \end{macrocode}
%
%    \begin{macrocode}
%<*2ekernel>
\message{document commands,}
%</2ekernel>
%    \end{macrocode}
%
% \changes{v1.0b}{2021/03/18}{Use \cs{NewModuleRelease}.}
% \changes{v1.0e}{2021/05/24}{Use \cs{msg_...} instead of \cs{__kernel_msg...}}
%
%
% \pkg{ltcmd} code contains an |^^@| character, which usually has
% catcode~15, so \cs{IncludeInRelease} will break when this code is
% being skipped, so we'll save the catcode of |^^@| to restore later:
%    \begin{macrocode}
%<*2ekernel|latexrelease>
%<latexrelease>\edef\@latexrelease@catcode@null{\the\catcode`\^^@ }
%<latexrelease>\catcode`\^^@=12
\ExplSyntaxOn
%<latexrelease>\NewModuleRelease{2020/10/01}{ltcmd}
%<latexrelease>                 {Document~command~parser}%
%    \end{macrocode}
%
% \subsection{Variables and constants}
%
% \begin{variable}{\l_@@_arg_spec_tl}
%   Holds the argument specification after normalization of shorthands.
%    \begin{macrocode}
\tl_new:N \l_@@_arg_spec_tl
%    \end{macrocode}
% \end{variable}
%
% \begin{variable}{\l_@@_args_tl}
%   Token list variable for grabbed arguments.
%    \begin{macrocode}
\tl_new:N \l_@@_args_tl
%    \end{macrocode}
% \end{variable}
%
% \begin{variable}{\l_@@_args_i_tl, \l_@@_args_ii_tl}
%   Hold the modified arguments when dealing with default values or
%   processors.
%    \begin{macrocode}
\tl_new:N \l_@@_args_i_tl
\tl_new:N \l_@@_args_ii_tl
%    \end{macrocode}
% \end{variable}
%
% \begin{variable}{\l_@@_current_arg_int}
%   The number of the current argument being set up: this is used to
%   make sure there are at most 9 arguments, then for creating the
%   expandable auxiliary functions and knowing how many arguments the
%   code function should take.
%    \begin{macrocode}
\int_new:N \l_@@_current_arg_int
%    \end{macrocode}
% \end{variable}
%
% \begin{variable}{\l_@@_defaults_bool, \l_@@_defaults_tl}
%   The boolean indicates whether there are any argument with default
%   value other than |-NoValue-|; the token list holds the code to
%   determine these default values in terms of other arguments.
%    \begin{macrocode}
\bool_new:N \l_@@_defaults_bool
\tl_new:N \l_@@_defaults_tl
%    \end{macrocode}
% \end{variable}
%
% \begin{variable}{\l_@@_environment_bool}
%   Generating environments uses the same mechanism as generating functions.
%   However, full processing of arguments is always needed for environments,
%   and so the function-generating code needs to know this.
%   This variable is also used at run time to give correct error messages.
%    \begin{macrocode}
\bool_new:N \l_@@_environment_bool
%    \end{macrocode}
% \end{variable}
%
% \begin{variable}{\l_@@_environment_str}
%   Name of the environment, used at definition time and at run time.
%    \begin{macrocode}
\str_new:N \l_@@_environment_str
%    \end{macrocode}
% \end{variable}
%
% \begin{variable}{\l_@@_expandable_bool}
%   Used to indicate if an expandable command is begin generated, as this
%   affects both the acceptable argument types and how they are implemented.
%    \begin{macrocode}
\bool_new:N \l_@@_expandable_bool
%    \end{macrocode}
% \end{variable}
%
% \begin{variable}{\l_@@_expandable_aux_name_tl}
%   Used to create pretty-printing names for the auxiliaries: although the
%   immediate definition does not vary, the full expansion does and so it
%   does not count as a constant.
%    \begin{macrocode}
\tl_new:N \l_@@_expandable_aux_name_tl
\tl_set:Nn \l_@@_expandable_aux_name_tl
  {
    \l_@@_function_tl \c_space_tl
    ( arg~ \int_use:N \l_@@_current_arg_int )
  }
%    \end{macrocode}
% \end{variable}
%
% \begin{variable}{\g_@@_grabber_int}
%   Used (in exceptional cases) to get unique names for grabbers used by
%   expandable commands.
%    \begin{macrocode}
\int_new:N \g_@@_grabber_int
%    \end{macrocode}
% \end{variable}
%
% \begin{variable}{\l_@@_fn_tl}
%   For passing the pre-formed name of the auxiliary to be used as the
%   parsing function.
%    \begin{macrocode}
\tl_new:N \l_@@_fn_tl
%    \end{macrocode}
% \end{variable}
%
% \begin{variable}{\l_@@_fn_code_tl}
%   For passing the pre-formed name of the auxiliary that contains the
%   actual code.
%    \begin{macrocode}
\tl_new:N \l_@@_fn_code_tl
%    \end{macrocode}
% \end{variable}
%
% \begin{variable}{\l_@@_function_tl}
%   Holds the control sequence name of the function currently being
%   defined: used to avoid passing this as an argument and to avoid repeated
%   use of \cs{cs_to_str:N}.
%    \begin{macrocode}
\tl_new:N \l_@@_function_tl
%    \end{macrocode}
% \end{variable}
%
% \begin{variable}{\l_@@_grab_expandably_bool}
%   When defining a non-expandable command, indicates whether the
%   arguments can all safely be grabbed by expandable grabbers.  This is
%   to support abuses of \pkg{xparse} that use protected functions
%   inside csname constructions.
%    \begin{macrocode}
\bool_new:N \l_@@_grab_expandably_bool
%    \end{macrocode}
% \end{variable}
%
% \begin{variable}{\l_@@_obey_spaces_bool}
%   For trailing optionals.
%    \begin{macrocode}
\bool_new:N \l_@@_obey_spaces_bool
%    \end{macrocode}
% \end{variable}
%
% \begin{variable}{\l_@@_last_delimiters_tl}
%   Holds the delimiters (first tokens) of all optional arguments since
%   the previous mandatory argument, to warn about cases where it would
%   be impossible to omit optional arguments completely because the
%   following mandatory argument has the same delimiter as one of the
%   optional arguments.
%    \begin{macrocode}
\tl_new:N \l_@@_last_delimiters_tl
%    \end{macrocode}
% \end{variable}
%
% \begin{variable}{\l_@@_long_bool}
%   Used to indicate that an argument is long, on a per-argument basis.
%    \begin{macrocode}
\bool_new:N \l_@@_long_bool
%    \end{macrocode}
% \end{variable}
%
% \begin{variable}{\l_@@_m_args_int}
%   The number of \texttt{m} arguments: if this is the same as the total
%   number of arguments, then a short-cut can be taken in the creation of
%   the grabber code.
%    \begin{macrocode}
\int_new:N \l_@@_m_args_int
%    \end{macrocode}
% \end{variable}
%
% \begin{variable}{\l_@@_prefixed_bool}
%   When preparing the signature of non-expandable commands, indicates
%   that the current argument is affected by a processor or by |+|
%   (namely is long).
%    \begin{macrocode}
\bool_new:N \l_@@_prefixed_bool
%    \end{macrocode}
% \end{variable}
%
% \begin{variable}{\l_@@_process_all_tl, \l_@@_process_one_tl, \l_@@_process_some_bool}
%   When preparing the signature, the processors that will be applied to
%   a given argument are collected in \cs{l_@@_process_one_tl}, while
%   \cs{l_@@_process_all_tl} contains processors for all arguments.  The
%   boolean indicates whether there are any processors (to bypass the
%   whole endeavour otherwise).
%    \begin{macrocode}
\tl_new:N \l_@@_process_all_tl
\tl_new:N \l_@@_process_one_tl
\bool_new:N \l_@@_process_some_bool
%    \end{macrocode}
% \end{variable}
%
% \begin{variable}{\l_@@_saved_args_tl}
%   Stores \cs{l_@@_args_tl} to deal with space-trimming of
%   \texttt{b}-type arguments.
%    \begin{macrocode}
\tl_new:N \l_@@_saved_args_tl
%    \end{macrocode}
% \end{variable}
%
% \begin{variable}{\l_@@_signature_tl}
%   Used when constructing the signature (code for argument grabbing) to
%   hold what will become the implementation of the main function.
%   When arguments are grabbed (at point of use of the command/environment),
%   it also stores the code for grabbing the remaining arguments.
%    \begin{macrocode}
\tl_new:N \l_@@_signature_tl
%    \end{macrocode}
% \end{variable}
%
% \begin{variable}
%   {\l_@@_some_obey_spaces_bool, \l_@@_some_long_bool, \l_@@_some_short_bool}
%   These flags are set while normalizing the argument specification.
%   The \texttt{obey_spaces} one is used to detect when |!| is used on
%   an argument that is not a trailing optional argument.
%   The other two are used to check whether all short arguments appear
%   before long arguments: this is needed to grab arguments expandably.
%   As soon as the first long argument is seen (other than
%   \texttt{t}-type, whose long status is ignored) the
%   \texttt{some_long} flag is set.  The \texttt{some_short} flag is
%   used for expandable commands, to know whether to define a short
%   auxiliary too.
%    \begin{macrocode}
\bool_new:N \l_@@_some_obey_spaces_bool
\bool_new:N \l_@@_some_long_bool
\bool_new:N \l_@@_some_short_bool
%    \end{macrocode}
% \end{variable}
%
% \begin{variable}{\l_@@_tmp_prop, \l_@@_tmpa_tl, \l_@@_tmpb_tl}
% \begin{macro}{\@@_tmp:w}
%   Scratch space.
%    \begin{macrocode}
\prop_new:N \l_@@_tmp_prop
\tl_new:N \l_@@_tmpa_tl
\tl_new:N \l_@@_tmpb_tl
\cs_new_eq:NN \@@_tmp:w ?
%    \end{macrocode}
% \end{macro}
% \end{variable}
%
%   With \pkg{xparse}, information about commands being (re)defined was
%   switched off by default, unless the |log-declarations| package
%   option was used, so here we'll switch that off as well.
%    \begin{macrocode}
\msg_redirect_module:nnn { cmd } { info } { none }
%    \end{macrocode}
%
%   Also add \pkg{cmd} to the \pkg{LaTeX} messages.
%    \begin{macrocode}
\prop_gput:Nnn \g_msg_module_type_prop { cmd } { LaTeX }
%    \end{macrocode}
%
% \subsection{Declaring commands and environments}
%
% \begin{macro}{\@@_declare_cmd:Nnn, \@@_declare_expandable_cmd:Nnn}
% \begin{macro}{\@@_declare_cmd_aux:Nnn}
% \begin{macro}
%   {\@@_declare_cmd_internal:Nnnn, \@@_declare_cmd_internal:cnxn}
%   The main functions for creating commands set the appropriate flag then
%   use the same internal code to do the definition.
%    \begin{macrocode}
\cs_new_protected:Npn \@@_declare_cmd:Nnn
  {
    \bool_set_false:N \l_@@_expandable_bool
    \@@_declare_cmd_aux:Nnn
  }
\cs_new_protected:Npn \@@_declare_expandable_cmd:Nnn
  {
    \bool_set_true:N \l_@@_expandable_bool
    \@@_declare_cmd_aux:Nnn
  }
%    \end{macrocode}
%  The first stage is to log information, both for the user in the log and
%  for programmatic use in a property list of all declared commands.
%    \begin{macrocode}
\cs_new_protected:Npn \@@_declare_cmd_aux:Nnn #1#2#3
  {
    \cs_if_exist:NTF #1
      {
        \msg_info:nnxx { cmd } { redefine-command }
          { \token_to_str:N #1 } { \tl_to_str:n {#2} }
      }
      {
        \bool_lazy_or:nnT
          { \cs_if_exist_p:c { \cs_to_str:N #1 ~ code } }
          { \cs_if_exist_p:c { \cs_to_str:N #1 ~ defaults } }
          {
            \msg_warning:nnx { cmd } { unsupported-let }
              { \token_to_str:N #1 }
          }
        \msg_info:nnxx { cmd } { define-command }
          { \token_to_str:N #1 } { \tl_to_str:n {#2} }
      }
    \bool_set_false:N \l_@@_environment_bool
    \@@_declare_cmd_internal:Nnnn #1 {#2} {#3} { }
  }
%    \end{macrocode}
%   At definition time, the variable \cs{l_@@_fn_tl} is only used for error messages.
%   The real business of defining a document command starts with setting up
%   the appropriate name, then normalizing the argument specification to get rid of
%   shorthands.
%    \begin{macrocode}
\cs_new_protected:Npn \@@_declare_cmd_internal:Nnnn #1#2#3#4
  {
    \tl_set:Nx \l_@@_function_tl { \cs_to_str:N #1 }
    \tl_set:Nx \l_@@_fn_tl
      { \exp_not:c { \l_@@_function_tl \c_space_tl } }
    \@@_normalize_arg_spec:n {#2}
    \exp_args:No \@@_prepare_signature:n \l_@@_arg_spec_tl
    \@@_declare_cmd_code:Nnn #1 {#2} {#3}
    #4
    \@@_break_point:n {#2}
  }
\cs_generate_variant:Nn \@@_declare_cmd_internal:Nnnn { cnx }
%    \end{macrocode}
% \end{macro}
% \end{macro}
% \end{macro}
%
% \begin{macro}{\@@_break_point:n}
%   A marker used to escape from creating a definition if necessary.
%    \begin{macrocode}
\cs_new_eq:NN \@@_break_point:n \use_none:n
%    \end{macrocode}
% \end{macro}
%
% \begin{macro}{\@@_declare_cmd_code:Nnn}
% \begin{macro}
%   {\@@_declare_cmd_code_aux:Nnn, \@@_declare_cmd_code_expandable:Nnn}
%   The appropriate auxiliary is called.
%    \begin{macrocode}
\cs_new_protected:Npn \@@_declare_cmd_code:Nnn
  {
    \bool_if:NTF \l_@@_grab_expandably_bool
      { \@@_declare_cmd_code_expandable:Nnn }
      { \@@_declare_cmd_code_aux:Nnn }
   }
%    \end{macrocode}
%   Standard functions call \cs{@@_start:nNNnnn}, which receives the
%   argument specification, an auxiliary used for
%   grabbing arguments, an auxiliary containing the code, and then the
%   signature, default arguments, and processors.
%    \begin{macrocode}
\cs_new_protected:Npn \@@_declare_cmd_code_aux:Nnn #1#2#3
  {
    \cs_generate_from_arg_count:cNnn
      { \l_@@_function_tl \c_space_tl code }
      \cs_set_protected:Npn \l_@@_current_arg_int {#3}
    \cs_set_protected_nopar:Npx #1
      {
        \bool_if:NTF \l_@@_environment_bool
          {
            \@@_start_env:nnnnn { \exp_not:n {#2} }
              { \l_@@_environment_str }
          }
          {
            \@@_start:nNNnnn { \exp_not:n {#2} }
              \exp_not:c { \l_@@_function_tl \c_space_tl }
              \exp_not:c { \l_@@_function_tl \c_space_tl code }
          }
          { \exp_not:o \l_@@_signature_tl }
          {
            \bool_if:NT \l_@@_defaults_bool
              { \exp_not:o \l_@@_defaults_tl }
          }
          {
            \bool_if:NT \l_@@_process_some_bool
              { \exp_not:o \l_@@_process_all_tl }
          }
      }
  }
%    \end{macrocode}
%   Expandable functions and functions whose arguments can be grabbed
%   expandably call \cs{@@_start_expandable:nNNNNn}, which receives the
%   argument specification, four auxiliaries (two for grabbing arguments, one for
%   the code, and one for default arguments), and finally the signature.
%   Non-expandable functions that take this branch should nevertheless
%   be protected, as well as their \texttt{code} function.  They will
%   only be expanded in contexts such as constructing a csname.
%   The two grabbers (named after the function with one or two spaces)
%   are needed when there are both short and long arguments; otherwise
%   the same grabber is included twice in the definition.  If all
%   arguments are long or all are short the (only) grabber is defined
%   correspondingly to be long/short.  Otherwise two grabbers are
%   defined, one long, one short.
%    \begin{macrocode}
\cs_new_protected:Npn \@@_declare_cmd_code_expandable:Nnn #1#2#3
  {
    \exp_args:Ncc \cs_generate_from_arg_count:NNnn
      { \l_@@_function_tl \c_space_tl code }
      { cs_set \bool_if:NF \l_@@_expandable_bool { _protected } :Npn }
      \l_@@_current_arg_int {#3}
    \bool_if:NT \l_@@_defaults_bool
      {
        \use:x
          {
            \cs_generate_from_arg_count:cNnn
              { \l_@@_function_tl \c_space_tl defaults }
              \cs_set:Npn \l_@@_current_arg_int
              { \exp_not:o \l_@@_defaults_tl }
          }
      }
    \bool_if:NTF \l_@@_expandable_bool
      { \cs_set_nopar:Npx } { \cs_set_protected_nopar:Npx } #1
      {
        \exp_not:N \@@_start_expandable:nNNNNn
          { \exp_not:n {#2} }
          \exp_not:c { \l_@@_function_tl \c_space_tl }
          \exp_not:c
            {
              \l_@@_function_tl \c_space_tl
              \bool_if:NT \l_@@_some_short_bool
                { \bool_if:NT \l_@@_some_long_bool { \c_space_tl } }
            }
          \exp_not:c { \l_@@_function_tl \c_space_tl code }
          \bool_if:NTF \l_@@_defaults_bool
            { \exp_not:c { \l_@@_function_tl \c_space_tl defaults } }
            { ? }
          { \exp_not:o \l_@@_signature_tl }
      }
    \bool_if:NTF \l_@@_some_long_bool
      {
        \bool_if:NT \l_@@_some_short_bool
          {
            \cs_set_nopar:cpx { \l_@@_function_tl \c_space_tl \c_space_tl }
              ##1##2 { ##1 {##2} }
          }
        \cs_set:cpx
      }
      { \cs_set_nopar:cpx }
          { \l_@@_function_tl \c_space_tl } ##1##2 { ##1 {##2} }
  }
%    \end{macrocode}
% \end{macro}
% \end{macro}
%
% \begin{macro}{\@@_declare_env:nnnn}
% \begin{macro}{\@@_declare_env_internal:nnnn}
%   The lead-off to creating an environment is much the same as that for
%   creating a command: issue the appropriate message, store the argument
%   specification then hand off to an internal function.
%    \begin{macrocode}
\cs_new_protected:Npn \@@_declare_env:nnnn #1#2
  {
    \str_set:Nx \l_@@_environment_str {#1}
    \str_set:Nx \l_@@_environment_str
      { \tl_trim_spaces:o { \l_@@_environment_str } }
    \cs_if_exist:cTF { \l_@@_environment_str }
      {
        \msg_info:nnxx { cmd } { redefine-environment }
          { \l_@@_environment_str } { \tl_to_str:n {#2} }
      }
      {
        \msg_info:nnxx { cmd } { define-environment }
          { \l_@@_environment_str } { \tl_to_str:n {#2} }
      }
    \bool_set_false:N \l_@@_expandable_bool
    \bool_set_true:N \l_@@_environment_bool
    \exp_args:NV \@@_declare_env_internal:nnnn
      \l_@@_environment_str {#2}
  }
%    \end{macrocode}
%   Creating a document environment requires a few more steps than creating
%   a single command. In order to pass the arguments of the command to the
%   end of the function, it is necessary to store the grabbed arguments.
%   To do that, the function used at the end of the environment has to be
%   redefined to contain the appropriate information. To minimize the amount
%   of expansion at point of use, the code here is expanded now as well as
%   when used.
%   The last argument of \cs{@@_declare_cmd_internal:Nnnn} is only run
%   if the definition succeeded.  In package mode this ensures that the
%   original definition of the environment is not changed if the
%   definition fails for any reason.  This also avoids an error when
%   defining the \verb*|end aux | function when the user asks for more
%   than $9$ arguments.
%    \begin{macrocode}
\cs_new_protected:Npn \@@_declare_env_internal:nnnn #1#2#3#4
  {
    \@@_declare_cmd_internal:cnxn { environment~ #1 } {#2}
      {
        \cs_set_nopar:Npx \exp_not:c { environment~ #1 ~end~aux }
          {
            \exp_not:N \exp_not:N \exp_not:c { environment~ #1~end~aux~ }
            \exp_not:n { \exp_not:o \l_@@_args_tl }
          }
        \exp_not:n {#3}
      }
      {
        \cs_set_nopar:cpx { environment~ #1 ~end }
          { \exp_not:c { environment~ #1 ~end~aux } }
        \cs_generate_from_arg_count:cNnn
          { environment~ #1 ~end~aux~ } \cs_set:Npn
          \l_@@_current_arg_int {#4}
        \cs_set_eq:cc {#1}       { environment~ #1 }
        \cs_set_eq:cc { end #1 } { environment~ #1 ~end }
      }
  }
%    \end{macrocode}
% \end{macro}
% \end{macro}
%
% \subsection{Structure of \pkg{xparse} commands}
%
% \begin{macro}{\@@_start_env:nnnnn, \@@_start:nNNnnn}
%   For error messages that occur during run-time when getting arguments
%   of environments it is necessary to keep track of the environment
%   name.  We begin non-expandable commands with a token equal to
%   \cs{scan_stop:}, whose name gives a reasonable error message if the
%   command is used inside a csname and protects against
%   \texttt{f}-expansion.  This is useless for environments since
%   \cs{begin} is already not expandable.  Both the command and
%   environment codes start with \cs{group_align_safe_begin:}, then
%   \cs{@@_run_code:} (used by both) does \cs{group_align_safe_end:}, so
%   that delimited arguments may be grabbed in alignments if they
%   contain and alignment tab token (see latex3/latex3/issues/839).
%    \begin{macrocode}
\cs_new_protected:Npn \@@_start_env:nnnnn #1#2
  {
    \conditionally@traceoff
    \group_align_safe_begin:
    \str_set:Nn \l_@@_environment_str {#2}
    \bool_set_true:N \l_@@_environment_bool
    \@@_start_aux:ccnnnn
      { environment~ \l_@@_environment_str \c_space_tl }
      { environment~ \l_@@_environment_str \c_space_tl code }
      {#1}
  }
\cs_new_protected:Npx \@@_start:nNNnnn #1#2#3
  {
    \exp_not:c { xparse~function~is~not~expandable }
    \exp_not:N \conditionally@traceoff
    \exp_not:N \group_align_safe_begin:
    \exp_not:n { \bool_set_false:N \l_@@_environment_bool }
    \exp_not:N \@@_start_aux:NNnnnn
    #2 #3 {#1}
  }
%    \end{macrocode}
% \end{macro}
%
% \begin{macro}{\@@_start_aux:NNnnnn, \@@_start_aux:ccnnnn}
%   This sets up a few variables to minimize the boilerplate code
%   included in all \pkg{xparse}-defined commands.  It then runs the
%   grabbers~|#4|.  Again, the argument specification |#1| is only for
%   diagnostics.
%    \begin{macrocode}
\cs_new_protected:Npn \@@_start_aux:NNnnnn #1#2#3#4#5#6
  {
    \tl_clear:N \l_@@_args_tl
    \tl_set:Nn \l_@@_fn_tl {#1}
    \tl_set:Nn \l_@@_fn_code_tl {#2}
    \tl_set:Nn \l_@@_defaults_tl {#5}
    \tl_set:Nn \l_@@_process_all_tl {#6}
    #4 \@@_run_code:
  }
\cs_generate_variant:Nn \@@_start_aux:NNnnnn { cc }
%    \end{macrocode}
% \end{macro}
%
% \begin{macro}{\@@_run_code:}
%   After arguments are grabbed, this function is responsible for
%   inserting default values, running processors, and finally doing
%   \cs{group_align_safe_end:} as promised, and running the code.
%    \begin{macrocode}
\cs_new_protected:Npn \@@_run_code:
  {
    \tl_if_empty:NF \l_@@_defaults_tl { \@@_defaults: }
    \tl_if_empty:NF \l_@@_process_all_tl { \@@_args_process: }
    \group_align_safe_end:
    \conditionally@traceon
    \exp_after:wN \l_@@_fn_code_tl \l_@@_args_tl
  }
%    \end{macrocode}
% \end{macro}
%
% \begin{macro}{\@@_defaults:}
% \begin{macro}{\@@_defaults_def:, \@@_defaults_def:nn, \@@_defaults_def:nnn}
% \begin{macro}{\@@_defaults_aux:,  \@@_defaults_error:w}
%   First construct \cs{@@_tmp:w} (see below) that will receive
%   the arguments found so far and determine default values for any
%   missing argument.  Then call it repeatedly until the set of
%   arguments stabilizes.  Since that could lead to an infinite loop we
%   only call it up to nine times, the maximal number needed for
%   stabilization if there is a chain of arguments that depend on each
%   other.  If that fails to stabilize raise an error.
%    \begin{macrocode}
\cs_new_protected:Npn \@@_defaults:
  {
    \@@_defaults_def:
    \tl_set_eq:NN \l_@@_args_i_tl \l_@@_args_tl
    \@@_defaults_aux: \@@_defaults_aux: \@@_defaults_aux:
    \@@_defaults_aux: \@@_defaults_aux: \@@_defaults_aux:
    \@@_defaults_aux: \@@_defaults_aux: \@@_defaults_aux:
    \@@_defaults_error:w
    \q_recursion_stop
    \tl_set_eq:NN \l_@@_args_tl \l_@@_args_i_tl
  }
\cs_new_protected:Npn \@@_defaults_aux:
  {
    \tl_set:Nx \l_@@_args_ii_tl
      { \exp_after:wN \@@_tmp:w \l_@@_args_i_tl }
    \tl_if_eq:NNT \l_@@_args_ii_tl \l_@@_args_i_tl
      { \use_none_delimit_by_q_recursion_stop:w }
    \tl_set_eq:NN \l_@@_args_i_tl \l_@@_args_ii_tl
  }
\cs_new_protected:Npn \@@_defaults_error:w \q_recursion_stop
  {
    \msg_error:nnx { cmd } { loop-in-defaults }
      { \@@_environment_or_command: }
  }
%    \end{macrocode}
%   To construct \cs{@@_tmp:w}, first go through the arguments
%   found and the corresponding defaults, building a token list with
%   |{#|\meta{arg number}|}| for arguments found in the input (whose
%   default will not be used) and otherwise
%   |{|\cs{exp_not:n}\Arg{default}|}| for arguments whose default will
%   be used.
%    \begin{macrocode}
\cs_new_protected:Npn \@@_defaults_def:
  {
    \tl_clear:N \l_@@_tmpa_tl
    \int_zero:N \l_@@_current_arg_int
    \@@_tl_mapthread_function:NNN \l_@@_args_tl \l_@@_defaults_tl
      \@@_defaults_def:nn
    \cs_generate_from_arg_count:NNVo \@@_tmp:w \cs_set:Npn
      \l_@@_current_arg_int \l_@@_tmpa_tl
  }
\cs_generate_variant:Nn \cs_generate_from_arg_count:NNnn { NNVo }
\cs_new_protected:Npn \@@_defaults_def:nn
  {
    \int_incr:N \l_@@_current_arg_int
    \exp_args:NV \@@_defaults_def:nnn \l_@@_current_arg_int
  }
\cs_new_protected:Npn \@@_defaults_def:nnn #1#2#3
  {
    \tl_put_right:Nx \l_@@_tmpa_tl
      {
        {
          \exp_not:N \exp_not:n
            {
              \tl_if_novalue:nTF {#2}
                { \exp_not:o {#3} }
                { \exp_not:n { ## #1 } }
            }
        }
      }
  }
%    \end{macrocode}
% \end{macro}
% \end{macro}
% \end{macro}
%
% \begin{macro}{\@@_args_process:}
% \begin{macro}{\@@_args_process_loop:nn, \@@_args_process_aux:n}
%   Loop through arguments (stored in \cs{l_@@_args_tl}) and the
%   corresponding processors (in \cs{l_@@_process_all_tl})
%   simultaneously, apply all processors for each argument and store the
%   result back into \cs{l_@@_args_tl}.  To allow processors to depend
%   on other arguments, for every processor define a temporary auxiliary
%   that receives all arguments \cs{l_@@_args_tl}.
%    \begin{macrocode}
\cs_new_protected:Npn \@@_args_process:
  {
    \tl_clear:N \l_@@_args_ii_tl
    \@@_tl_mapthread_function:NNN
      \l_@@_args_tl
      \l_@@_process_all_tl
      \@@_args_process_loop:nn
    \tl_set_eq:NN \l_@@_args_tl \l_@@_args_ii_tl
  }
\cs_new_protected:Npn \@@_args_process_loop:nn #1#2
  {
    \tl_set:Nn \ProcessedArgument {#1}
    \tl_if_novalue:nF {#1}
      { \tl_map_function:nN {#2} \@@_args_process_aux:n }
    \tl_put_right:No \l_@@_args_ii_tl
      { \exp_after:wN { \ProcessedArgument } }
  }
\cs_new_protected:Npn \@@_args_process_aux:n #1
  {
    \cs_generate_from_arg_count:NNnn \@@_tmp:w \cs_set:Npn
      { \tl_count:N \l_@@_args_tl } {#1}
    \exp_args:NNNo \exp_after:wN \@@_tmp:w \l_@@_args_tl
      { \ProcessedArgument }
  }
%    \end{macrocode}
% \end{macro}
% \end{macro}
%
% \begin{macro}[EXP]{\@@_start_expandable:nNNNNn}
%   This is called for all expandable commands.  |#6| is the signature,
%   responsible for grabbing arguments.  |#5| is used to determine
%   default values (or is |?| if there are none).  |#4| is the code to run.
%   |#2|~and~|#3| are functions (named after the command) that grab a single
%   argument in the input stream (|#3|~is~short).  The argument specification |#1| is
%   only used by diagnostic functions.  Same as for the non-expandable
%   version, this starts with \cs{group_align_safe_begin:}, which
%   expands to nothing, so may be safely used in an expandable context.
%    \begin{macrocode}
\cs_new:Npn \@@_start_expandable:nNNNNn #1#2#3#4#5#6
  {
    \group_align_safe_begin:
    #6 \@@_end_expandable:NNw #5 #4 \q_@@ #2#3
  }
%    \end{macrocode}
% \end{macro}
%
% \begin{macro}[EXP]{\@@_end_expandable:NNw}
% \begin{macro}[EXP]{\@@_end_expandable_aux:w}
% \begin{macro}[EXP]{\@@_end_expandable_aux:nNNNN}
% \begin{macro}[EXP]{\@@_end_expandable_defaults:nnnNNn}
% \begin{macro}[EXP]{\@@_end_expandable_defaults:nnw}
% \begin{macro}[EXP]{\@@_end_expandable_defaults:nw}
%   Followed by a function |#1| to determine default values (or |?| if
%   there are no defaults), the code
%   |#2|, arguments that have been grabbed, then \cs{q_@@} and two generic
%   grabbers.  The idea to find default values is similar to the
%   non-expandable case but we cannot define an auxiliary function, so
%   at every step in the loop we need to go through all arguments
%   searching for which ones started out as |-NoValue-| and replacing
%   these by the newly computed values.  In fact we need to keep track
%   of three versions of all arguments: the original version, the
%   previous version with default values, and the currently built
%   version (first argument of \cs{@@_end_expandable_defaults:nnnNNn}).
%    \begin{macrocode}
\cs_new:Npn \@@_end_expandable:NNw #1#2
  { \@@_end_expandable_aux:w #1#2 \prg_do_nothing: }
\cs_new:Npn \@@_end_expandable_aux:w #1#2#3 \q_@@
  { \exp_args:No \@@_end_expandable_aux:nNNNN {#3} #1 #2 }
\cs_new:Npn \@@_end_expandable_aux:nNNNN #1#2#3#4#5
  {
    \token_if_eq_charcode:NNT ? #2 { \exp_after:wN \use_iv:nnnn }
    \@@_end_expandable_defaults:nnnNNn {#1} { } {#1} #2#3
      { } { } { } { } { } { } { } { } { } { }
      {
        \msg_expandable_error:nnf { cmd } { loop-in-defaults }
          { \exp_args:Nf \tl_trim_spaces:n { \token_to_str:N #4 } }
        \use_iv:nnnn
      }
    \q_stop
  }
\cs_new:Npn \@@_end_expandable_defaults:nnnNNn #1#2#3#4#5#6
  {
    #6
    \str_if_eq:nnTF {#1} {#2}
      { \use_i_delimit_by_q_stop:nw { \group_align_safe_end: #5 #1 } }
      {
        \exp_args:No \@@_tl_mapthread_function:nnN
          { #4 #1 } {#3}
          \@@_end_expandable_defaults:nnw
        \@@_end_expandable_defaults:nnnNNn { } {#1} {#3} #4 #5
      }
  }
\cs_new:Npn \@@_end_expandable_defaults:nnw #1#2
  {
    \tl_if_novalue:nTF {#2}
      { \exp_args:No \@@_end_expandable_defaults:nw {#1} }
      { \@@_end_expandable_defaults:nw {#2} }
  }
\cs_new:Npn \@@_end_expandable_defaults:nw
    #1#2 \@@_end_expandable_defaults:nnnNNn #3
  { #2 \@@_end_expandable_defaults:nnnNNn { #3 {#1} } }
%    \end{macrocode}
% \end{macro}
% \end{macro}
% \end{macro}
% \end{macro}
% \end{macro}
% \end{macro}
%
% \subsection{Normalizing the argument specifications}
%
% The goal here is to expand aliases and check that the argument
% specification is valid before the main parsing run.  If it is not
% valid the entire set up is abandoned to avoid any strange internal
% errors.  A function is provided for each argument type that will grab
% any extra data items and call the loop function after performing the
% following checks and tasks.
% \begin{itemize}
%   \item Check that each argument has the correct number of data items
%     associated with it, and that where a single character is required,
%     one has actually been supplied.
%   \item Check that processors and the markers~|+| and~|!| are followed
%     by an argument for which they make sense, and are not redundant.
%   \item Check the absence of forbidden types for expandable commands,
%     namely \texttt{G}/\texttt{v} always, and \texttt{l}/\texttt{u}
%     after optional arguments (\pkg{xparse} may have inserted braces
%     due to a failed search for an optional argument).
%   \item Check that no optional argument is followed by a mandatory
%     argument with the same delimiter, as otherwise the optional
%     argument could never be omitted.
%   \item Keep track in \cs{l_@@_some_long_bool} and
%     \cs{l_@@_some_short_bool} of whether the command has some
%     long/short arguments.
%   \item Keep track in \cs{l_@@_grab_expandably_bool} of whether all
%     arguments are \texttt{m}/\texttt{l}/\texttt{u} type and short
%     arguments appear before long ones, in which case they can be
%     grabbed expandably just as safely as they could be grabbed
%     nonexpandably.  Regardless of that, arguments of expandable
%     commands will be grabbed expandably and arguments of environments
%     will not (because the list of arguments built by non-expandable
%     grabbing is used to pass them to the end-environment code).
% \end{itemize}
% Further checks happen at the end of the loop:
% \begin{itemize}
% \item that there are at most $9$ arguments;
% \item that an expandable command does not end with an optional
%   argument (this case is detected by using the fact that
%   \cs{l_@@_last_delimiters_tl} is cleared by every mandatory argument
%   and filled by every optional argument).
% \end{itemize}
%
% \begin{macro}{\@@_normalize_arg_spec:n}
% \begin{macro}{\@@_normalize_arg_spec_loop:n}
%   Loop through the argument specification, calling an auxiliary
%   specific to each argument type.  If any argument is unknown stop the
%   definition.
%    \begin{macrocode}
\cs_new_protected:Npn \@@_normalize_arg_spec:n #1
  {
    \int_zero:N \l_@@_current_arg_int
    \tl_clear:N \l_@@_last_delimiters_tl
    \tl_clear:N \l_@@_arg_spec_tl
    \bool_set_true:N \l_@@_grab_expandably_bool
    \bool_set_false:N \l_@@_obey_spaces_bool
    \bool_set_false:N \l_@@_long_bool
    \bool_set_false:N \l_@@_some_obey_spaces_bool
    \bool_set_false:N \l_@@_some_long_bool
    \bool_set_false:N \l_@@_some_short_bool
    \@@_normalize_arg_spec_loop:n #1
      \q_recursion_tail \q_recursion_tail \q_recursion_tail \q_recursion_stop
    \int_compare:nNnT \l_@@_current_arg_int > 9
      {
        \msg_error:nnxx { cmd } { too-many-arguments }
          { \@@_environment_or_command: } { \tl_to_str:n {#1} }
        \@@_bad_def:wn
      }
    \bool_if:NT \l_@@_expandable_bool
      {
        \tl_if_empty:NF \l_@@_last_delimiters_tl
          {
            \msg_error:nnxx { cmd } { expandable-ending-optional }
              { \iow_char:N \\ \l_@@_function_tl } { \tl_to_str:n {#1} }
            \@@_bad_def:wn
          }
      }
    \bool_if:NT \l_@@_expandable_bool
      { \bool_set_true:N \l_@@_grab_expandably_bool }
    \bool_if:NT \l_@@_environment_bool
      { \bool_set_false:N \l_@@_grab_expandably_bool }
  }
\cs_new_protected:Npn \@@_normalize_arg_spec_loop:n #1
  {
    \quark_if_recursion_tail_stop:n {#1}
    \int_incr:N \l_@@_current_arg_int
    \cs_if_exist_use:cF { @@_normalize_type_ \tl_to_str:n {#1} :w }
      {
        \bool_lazy_any:nTF
          {
            { \str_if_eq_p:nn {#1} { G } }
            { \str_if_eq_p:nn {#1} { g } }
            { \str_if_eq_p:nn {#1} { l } }
            { \str_if_eq_p:nn {#1} { u } }
          }
          {
            \msg_error:nnxx { cmd } { xparse-argument-type }
              { \@@_environment_or_command: } { \tl_to_str:n {#1} }
          }
          {
            \msg_error:nnxx { cmd } { unknown-argument-type }
              { \@@_environment_or_command: } { \tl_to_str:n {#1} }
          }
        \@@_bad_def:wn
      }
  }
%    \end{macrocode}
% \end{macro}
% \end{macro}
%
% \begin{macro}
%   {
%     \@@_normalize_type_d:w,
%     \@@_normalize_type_e:w,
%     \@@_normalize_type_o:w,
%     \@@_normalize_type_O:w,
%     \@@_normalize_type_r:w,
%     \@@_normalize_type_s:w,
%   }
%   These argument types are aliases of more general ones, for example
%   with the default argument |-NoValue-|.  To easily insert that marker
%   expanded in the definitions we call \cs{@@_tmp:w} with the argument
%   |-NoValue-|.  For argument types that need additional data, check
%   that the data is present (not \cs{q_recursion_tail}) before
%   proceeding.
%    \begin{macrocode}
\cs_set_protected:Npn \@@_tmp:w #1
  {
    \cs_new_protected:Npn \@@_normalize_type_d:w ##1##2
      {
        \quark_if_recursion_tail_stop_do:nn {##2} { \@@_bad_arg_spec:wn }
        \@@_normalize_type_D:w {##1} {##2} {#1}
      }
    \cs_new_protected:Npn \@@_normalize_type_e:w ##1
      {
        \quark_if_recursion_tail_stop_do:nn {##1} { \@@_bad_arg_spec:wn }
        \@@_normalize_type_E:w {##1} { }
      }
    \cs_new_protected:Npn \@@_normalize_type_o:w
      { \@@_normalize_type_D:w [ ] {#1} }
    \cs_new_protected:Npn \@@_normalize_type_O:w
      { \@@_normalize_type_D:w [ ] }
    \cs_new_protected:Npn \@@_normalize_type_r:w ##1##2
      {
        \quark_if_recursion_tail_stop_do:nn {##2} { \@@_bad_arg_spec:wn }
        \@@_normalize_type_R:w {##1} {##2} {#1}
      }
    \cs_new_protected:Npn \@@_normalize_type_s:w
      { \@@_normalize_type_t:w * }
  }
\exp_args:No \@@_tmp:w { \c_novalue_tl }
%    \end{macrocode}
% \end{macro}
%
% \begin{macro}
%   {
%     \@@_normalize_type_>:w,
%     \@@_normalize_type_+:w,
%     \@@_normalize_type_!:w,
%   }
%   Check that these prefixes have arguments, namely that the next token
%   is not \cs{q_recursion_tail}, and remember to leave it after the
%   looping macro.  Processors are forbidden in expandable commands.
%   If all is good, store the prefix in the cleaned up
%   \cs{l_@@_arg_spec_tl}, and decrement the argument number as prefixes
%   do not correspond to arguments.
%    \begin{macrocode}
\cs_new_protected:cpn { @@_normalize_type_>:w } #1#2
  {
    \quark_if_recursion_tail_stop_do:nn {#2} { \@@_bad_arg_spec:wn }
    \bool_if:NT \l_@@_expandable_bool
      {
        \msg_error:nnxx { cmd } { processor-in-expandable }
          { \iow_char:N \\ \l_@@_function_tl } { \tl_to_str:n {#1} }
        \@@_bad_def:wn
      }
    \tl_put_right:Nx \l_@@_arg_spec_tl { > { \tl_trim_spaces:n {#1} } }
    \int_decr:N \l_@@_current_arg_int
    \bool_set_false:N \l_@@_grab_expandably_bool
    \@@_normalize_arg_spec_loop:n {#2}
  }
\cs_new_protected:cpn { @@_normalize_type_+:w } #1
  {
    \quark_if_recursion_tail_stop_do:nn {#1} { \@@_bad_arg_spec:wn }
    \bool_if:NT \l_@@_long_bool
      {
        \msg_error:nnxx { cmd } { two-markers }
          { \@@_environment_or_command: } { + }
        \@@_bad_def:wn
      }
    \bool_set_true:N \l_@@_long_bool
    \int_decr:N \l_@@_current_arg_int
    \@@_normalize_arg_spec_loop:n {#1}
  }
\cs_new_protected:cpn { @@_normalize_type_!:w } #1
  {
    \quark_if_recursion_tail_stop_do:nn {#1} { \@@_bad_arg_spec:wn }
    \bool_if:NT \l_@@_obey_spaces_bool
      {
        \msg_error:nnxx { cmd } { two-markers }
          { \@@_environment_or_command: } { ! }
        \@@_bad_def:wn
      }
    \bool_set_true:N \l_@@_obey_spaces_bool
    \bool_set_true:N \l_@@_some_obey_spaces_bool
    \int_decr:N \l_@@_current_arg_int
    \@@_normalize_arg_spec_loop:n {#1}
  }
%    \end{macrocode}
% \end{macro}
%
% \begin{macro}
%   {
%     \@@_normalize_type_D:w,
%     \@@_normalize_type_E:w,
%     \@@_normalize_type_t:w,
%   }
% \begin{macro}{\@@_normalize_E_unique_check:w}
%   Optional argument types.  Check that all required data is present
%   (and consists of single characters if applicable) and check for
%   forbidden types for expandable commands.  For \texttt{E}-type
%   require that there is at least one embellishment, that each one is a
%   single character, and that there aren't more optional arguments than
%   embellishments; also remember that each embellishment counts as one
%   argument for \cs{l_@@_current_arg_int}.  Then in each case
%   store the data in \cs{l_@@_arg_spec_tl}, and
%   for later checks store in \cs{l_@@_last_delimiters_tl} the tokens
%   whose presence determines whether there is an optional argument (for
%   braces store |{}|, seen later as an empty delimiter).
%    \begin{macrocode}
\cs_new_protected:Npn \@@_normalize_type_D:w #1#2#3
  {
    \quark_if_recursion_tail_stop_do:nn {#3} { \@@_bad_arg_spec:wn }
    \@@_single_token_check:n {#1} \@@_allowed_token_check:N #1
    \@@_single_token_check:n {#2}
    \@@_add_arg_spec:n { D #1 #2 {#3} }
    \tl_put_right:Nn \l_@@_last_delimiters_tl {#1}
    \bool_set_false:N \l_@@_grab_expandably_bool
    \@@_normalize_arg_spec_loop:n
  }
\cs_new_protected:Npn \@@_normalize_type_E:w #1#2
  {
    \quark_if_recursion_tail_stop_do:nn {#2} { \@@_bad_arg_spec:wn }
    \tl_if_blank:nT {#1} { \@@_bad_arg_spec:wn }
    \tl_map_function:nN {#1} \@@_single_token_check:n
    \tl_map_function:nN {#1} \@@_allowed_token_check:N
    \@@_normalize_E_unique_check:w #1 \q_nil \q_stop
    \int_compare:nNnT { \tl_count:n {#2} } > { \tl_count:n {#1} }
      { \@@_bad_arg_spec:wn }
    \@@_add_arg_spec:n { E {#1} {#2} }
    \tl_put_right:Nn \l_@@_last_delimiters_tl {#1}
    \bool_set_false:N \l_@@_grab_expandably_bool
    \int_add:Nn \l_@@_current_arg_int { \tl_count:n {#1} - 1 }
    \@@_normalize_arg_spec_loop:n
  }
\cs_new_protected:Npn \@@_normalize_E_unique_check:w #1#2 \q_stop
  {
    \quark_if_nil:NF #1
      {
        \tl_if_in:nnT {#2} {#1} { \@@_bad_arg_spec:wn }
        \@@_normalize_E_unique_check:w #2 \q_stop
      }
  }
\cs_new_protected:Npn \@@_normalize_type_t:w #1
  {
    \quark_if_recursion_tail_stop_do:Nn #1 { \@@_bad_arg_spec:wn }
    \@@_single_token_check:n {#1} \@@_allowed_token_check:N #1
    \tl_put_right:Nx \l_@@_arg_spec_tl
      {
        \bool_if:NT \l_@@_obey_spaces_bool { ! }
        t \exp_not:n {#1}
      }
    \tl_put_right:Nn \l_@@_last_delimiters_tl {#1}
    \bool_set_false:N \l_@@_grab_expandably_bool
    \bool_set_false:N \l_@@_obey_spaces_bool
    \bool_set_false:N \l_@@_long_bool
    \@@_normalize_arg_spec_loop:n
  }
%    \end{macrocode}
% \end{macro}
% \end{macro}
%
% \begin{macro}
%   {
%     \@@_normalize_type_m:w,
%     \@@_normalize_type_R:w,
%     \@@_normalize_type_v:w
%   }
%   Mandatory arguments.  First check the required data is present,
%   consists of single characters where applicable, and that the argument
%   type is allowed for expandable commands if applicable.  For the
%   \texttt{m} and \texttt{R} argument types check that they do not
%   follow some optional argument with that delimiter as otherwise the
%   optional argument could not be omitted.  Then save data in
%   \cs{l_@@_arg_spec_tl}, count the mandatory argument, and empty the
%   list of last delimiters.
%    \begin{macrocode}
\cs_new_protected:Npn \@@_normalize_type_m:w
  {
    \@@_delimiter_check:nnn { } { m } { \iow_char:N \{ }
    \@@_add_arg_spec_mandatory:n { m }
    \@@_normalize_arg_spec_loop:n
  }
\cs_new_protected:Npn \@@_normalize_type_R:w #1#2#3
  {
    \quark_if_recursion_tail_stop_do:nn {#3} { \@@_bad_arg_spec:wn }
    \@@_single_token_check:n {#1} \@@_allowed_token_check:N #1
    \@@_single_token_check:n {#2}
    \@@_delimiter_check:nnn {#1} { R/r } { \tl_to_str:n {#1} }
    \bool_set_false:N \l_@@_grab_expandably_bool
    \@@_add_arg_spec_mandatory:n { R #1 #2 {#3} }
    \@@_normalize_arg_spec_loop:n
  }
\cs_new_protected:Npn \@@_normalize_type_v:w
  {
    \@@_normalize_check_gv:N v
    \@@_add_arg_spec_mandatory:n { v }
    \@@_normalize_arg_spec_loop:n
  }
%    \end{macrocode}
% \end{macro}
%
% \begin{macro}{\@@_normalize_type_b:w}
%   This argument type is not allowed for commands.  This is only
%   allowed at the end of the argument specification, hence we check
%   that |#1| is the end.
%    \begin{macrocode}
\cs_new_protected:Npn \@@_normalize_type_b:w #1
  {
    \bool_if:NF \l_@@_environment_bool
      {
        \msg_error:nnxx { cmd } { invalid-command-arg }
          { \@@_environment_or_command: } { b }
        \@@_bad_def:wn
      }
    \tl_clear:N \l_@@_last_delimiters_tl
    \@@_add_arg_spec:n { b }
    \quark_if_recursion_tail_stop:n {#1}
    \msg_error:nnxx { cmd } { arg-after-body }
      { \@@_environment_or_command: }
      { \tl_to_str:n {#1} }
    \@@_bad_def:wn
  }
%    \end{macrocode}
% \end{macro}
%
% \begin{macro}{\@@_single_token_check:n}
%   Checks that the argument is a single (non-space) token (possibly
%   surrounded by spaces), and aborts the definition otherwise.
%    \begin{macrocode}
\cs_new_protected:Npn \@@_single_token_check:n #1
  {
    \tl_trim_spaces_apply:nN {#1} \tl_if_single_token:nF
      {
        \msg_error:nnxx { cmd } { not-single-token }
          { \@@_environment_or_command: } { \tl_to_str:n {#1} }
        \@@_bad_def:wn
      }
  }
%    \end{macrocode}
% \end{macro}
%
% \begin{macro}{\@@_allowed_token_check:N}
%   Some tokens are now allowed as delimiters for some argument types,
%   notably implicit begin/end-group tokens (|\bgroup|/|\egroup|).
%   The major problem with these tokens is that for |\peek_...| functions,
%   a literal~|{|$_1$. is virtually indistinguishable from a |\bgroup| or
%   other token which was |\let| to a~|{|$_1$, and the same goes
%   for~|}|$_2$.  All other tokens can be easily distingushed from their
%   implicit counterparts by grabbing them and looking at the string
%   length (see \cs{@@_token_if_cs:NTF}), but for begin/end group tokens
%   that is not possible without the risk of mistakenly grabbing the
%   entire brace group (potentially leading to a~\texttt{!~Runaway argument}
%   error) or trying to grab a |}|$_2$, leading to
%   an~\verb|! Argument of \dots has an extra }| error.
%    \begin{macrocode}
\cs_new_protected:Npn \@@_allowed_token_check:N #1
  {
    \token_if_eq_meaning:NNTF #1 \c_group_begin_token
        { \use:n }
        {
          \token_if_eq_meaning:NNTF #1 \c_group_end_token
            { \use:n }
            { \use_none:n }
        }
      {
        \msg_error:nnxxx { cmd } { forbidden-implicit-group-token }
          { \@@_environment_or_command: } { \tl_to_str:n {#1} }
          {
            \token_if_eq_meaning:NNTF #1 \c_group_begin_token
              { begin } { end }
          }
        \@@_bad_def:wn
      }
  }
%    \end{macrocode}
% \end{macro}
%
% \begin{macro}{\@@_normalize_check_gv:N, \@@_normalize_check_lu:N}
%   Called for arguments that are always forbidden, or forbidden after
%   an optional argument, for expandable commands.
%    \begin{macrocode}
\cs_new_protected:Npn \@@_normalize_check_gv:N #1
  {
    \bool_if:NT \l_@@_expandable_bool
      {
        \msg_error:nnxx { cmd } { invalid-expandable-argument-type }
          { \iow_char:N \\ \l_@@_function_tl } { \tl_to_str:n {#1} }
        \@@_bad_def:wn
      }
    \bool_set_false:N \l_@@_grab_expandably_bool
  }
\cs_new_protected:Npn \@@_normalize_check_lu:N #1
  {
    \bool_if:NT \l_@@_expandable_bool
      {
        \tl_if_empty:NF \l_@@_last_delimiters_tl
          {
            \msg_error:nnxx { cmd } { invalid-after-optional-expandably }
              { \iow_char:N \\ \l_@@_function_tl } { \tl_to_str:n {#1} }
            \@@_bad_def:wn
          }
      }
  }
%    \end{macrocode}
% \end{macro}
%
% \begin{macro}{\@@_delimiter_check:nnn}
%   Called for \texttt{m} and \texttt{R} arguments.  Checks that the
%   leading token does not coincide with the token denoting the presence
%   of a previous optional argument.  Instead of dealing with braces for
%   the \texttt{m}-type we use an empty delimiter to denote that case.
%    \begin{macrocode}
\cs_new_protected:Npn \@@_delimiter_check:nnn #1#2#3
  {
    \tl_map_inline:Nn \l_@@_last_delimiters_tl
      {
        \tl_if_eq:nnT {##1} {#1}
          {
            \msg_warning:nnxx { cmd } { optional-mandatory }
              {#2} {#3}
          }
      }
  }
%    \end{macrocode}
% \end{macro}
%
% \begin{macro}{\@@_bad_arg_spec:wn, \@@_bad_def:wn}
%   If the argument specification is wrong, this provides an escape from
%   the entire definition process.
%    \begin{macrocode}
\cs_new_protected:Npn \@@_bad_arg_spec:wn #1 \@@_break_point:n #2
  {
    \msg_error:nnxx { cmd } { bad-arg-spec }
      { \@@_environment_or_command: } { \tl_to_str:n {#2} }
  }
\cs_new_protected:Npn \@@_bad_def:wn #1 \@@_break_point:n #2 { }
%    \end{macrocode}
% \end{macro}
%
% \begin{macro}{\@@_add_arg_spec:n, \@@_add_arg_spec_mandatory:n}
%   When adding an argument to the argument specification, set the
%   \texttt{some_long} or \texttt{some_short} booleans as appropriate
%   and clear the booleans keeping track of |+| and |!| markers.
%   Before that, test for a short argument following some long
%   arguments: this is forbidden for expandable commands and prevents
%   grabbing arguments expandably.
%
%   For mandatory arguments do some more work, in particular complain if
%   they were preceeded by~|!|.
%    \begin{macrocode}
\cs_new_protected:Npn \@@_add_arg_spec:n #1
  {
    \bool_lazy_and:nnT
      { ! \l_@@_long_bool }
      { \l_@@_some_long_bool }
      {
        \bool_if:NT \l_@@_expandable_bool
          {
            \msg_error:nnx { cmd } { inconsistent-long }
              { \iow_char:N \\ \l_@@_function_tl }
            \@@_bad_def:wn
          }
        \bool_set_false:N \l_@@_grab_expandably_bool
      }
    \bool_if:NTF \l_@@_long_bool
      { \bool_set_true:N \l_@@_some_long_bool }
      { \bool_set_true:N \l_@@_some_short_bool }
    \tl_put_right:Nx \l_@@_arg_spec_tl
      {
        \bool_if:NT \l_@@_long_bool { + }
        \bool_if:NT \l_@@_obey_spaces_bool { ! }
        \exp_not:n {#1}
      }
    \bool_set_false:N \l_@@_long_bool
    \bool_set_false:N \l_@@_obey_spaces_bool
  }
\cs_new_protected:Npn \@@_add_arg_spec_mandatory:n #1
  {
    \bool_if:NT \l_@@_some_obey_spaces_bool
      {
        \msg_error:nnxx { cmd } { non-trailing-obey-spaces }
          { \@@_environment_or_command: } { \tl_to_str:n {#1} }
        \@@_bad_def:wn
      }
    \tl_clear:N \l_@@_last_delimiters_tl
    \@@_add_arg_spec:n {#1}
  }
%    \end{macrocode}
% \end{macro}
%
% \subsection{Preparing the signature: general mechanism}
%
% \begin{macro}{\@@_prepare_signature:n}
% \begin{macro}{\@@_prepare_signature:N}
% \begin{macro}{\@@_prepare_signature_bypass:N}
%   Actually creating the signature uses the same loop approach as
%   normalizing the signature. There are first a number of variables which need
%   to be set to track what is going on. Many of these variables are unused
%   when defining expandable commands.
%    \begin{macrocode}
\cs_new_protected:Npn \@@_prepare_signature:n #1
  {
    \int_zero:N \l_@@_current_arg_int
    \bool_set_false:N \l_@@_long_bool
    \bool_set_false:N \l_@@_obey_spaces_bool
    \int_zero:N \l_@@_m_args_int
    \bool_set_false:N \l_@@_defaults_bool
    \tl_clear:N \l_@@_defaults_tl
    \tl_clear:N \l_@@_process_all_tl
    \tl_clear:N \l_@@_process_one_tl
    \bool_set_false:N \l_@@_process_some_bool
    \tl_clear:N \l_@@_signature_tl
    \@@_prepare_signature:N #1 \q_recursion_tail \q_recursion_stop
    \bool_if:NF \l_@@_expandable_bool { \@@_flush_m_args: }
  }
%    \end{macrocode}
%  The main looping function does not take an argument, but carries out the
%  reset on the processor boolean. This is split off from the rest of the
%  process so that when actually setting up processors the flag-reset can
%  be bypassed.
%
%  For each known argument type there is an appropriate function to actually
%  do the addition to the signature. These are separate for expandable and
%  standard functions, as the approaches are different.
%    \begin{macrocode}
\cs_new_protected:Npn \@@_prepare_signature:N
  {
    \bool_set_false:N \l_@@_prefixed_bool
    \@@_prepare_signature_bypass:N
  }
\cs_new_protected:Npn \@@_prepare_signature_bypass:N #1
  {
    \quark_if_recursion_tail_stop:N #1
    \use:c
      {
         @@_add
         \bool_if:NT \l_@@_grab_expandably_bool { _expandable }
         _type_  \token_to_str:N #1 :w
      }
  }
%    \end{macrocode}
% \end{macro}
% \end{macro}
% \end{macro}
%
% \subsection{Setting up a standard signature}
%
% Each argument-adding function appends to the signature a grabber (and
% for some types, the delimiters or default value), except the one for
% \texttt{m} arguments.  These are collected and added to the signature
% all at once by \cs{@@_flush_m_args:}, called for every other argument
% type.  All of the functions then call the loop function
% \cs{@@_prepare_signature:N}.  Default values of arguments are
% collected by \cs{@@_add_default:n} rather than being stored with the
% argument; this function and \cs{@@_add_default:} are also responsible
% for keeping track of \cs{l_@@_current_arg_int}.
%
% \begin{macro}{\@@_add_type_+:w}
%   Making the next argument long means setting the flag. The \texttt{m}
%   arguments are recorded here as
%   this has to be done for every case where there is then a long argument.
%    \begin{macrocode}
\cs_new_protected:cpn { @@_add_type_+:w }
  {
    \@@_flush_m_args:
    \bool_set_true:N \l_@@_long_bool
    \bool_set_true:N \l_@@_prefixed_bool
    \@@_prepare_signature_bypass:N
  }
%    \end{macrocode}
% \end{macro}
%
% \begin{macro}{\@@_add_type_!:w}
%   Much the same for controlling trailing optional arguments.
%    \begin{macrocode}
\cs_new_protected:cpn { @@_add_type_!:w }
  {
    \@@_flush_m_args:
    \bool_set_true:N \l_@@_obey_spaces_bool
    \bool_set_true:N \l_@@_prefixed_bool
    \@@_prepare_signature_bypass:N
  }
%    \end{macrocode}
% \end{macro}
%
% \begin{macro}{\@@_add_type_>:w}
%   When a processor is found, the processor code is stored.  It will be
%   used by \cs{@@_args_process:} once arguments are all found. Here too
%   the loop calls \cs{@@_prepare_signature_bypass:N} rather than
%   \cs{@@_prepare_signature:N} so that the flag is not reset.
%    \begin{macrocode}
\cs_new_protected:cpn { @@_add_type_>:w } #1
  {
    \@@_flush_m_args:
    \bool_set_true:N \l_@@_prefixed_bool
    \bool_set_true:N \l_@@_process_some_bool
    \tl_put_left:Nn \l_@@_process_one_tl { {#1} }
    \@@_prepare_signature_bypass:N
  }
%    \end{macrocode}
% \end{macro}
%
% \begin{macro}{\@@_add_type_b:w}
%    \begin{macrocode}
\cs_new_protected:Npn \@@_add_type_b:w
  {
    \@@_flush_m_args:
    \@@_add_default:
    \@@_add_grabber:N b
    \@@_prepare_signature:N
  }
%    \end{macrocode}
% \end{macro}
%
% \begin{macro}{\@@_add_type_D:w}
%    \begin{macrocode}
\cs_new_protected:Npn \@@_add_type_D:w #1#2#3
  {
    \@@_flush_m_args:
    \@@_add_default:n {#3}
    \@@_add_grabber:N D
    \tl_put_right:Nn \l_@@_signature_tl { #1 #2 }
    \@@_prepare_signature:N
  }
%    \end{macrocode}
% \end{macro}
%
% \begin{macro}{\@@_add_type_E:w}
%   The \texttt{E}-type argument needs a special handling of default
%   values.
%    \begin{macrocode}
\cs_new_protected:Npn \@@_add_type_E:w #1#2
  {
    \@@_flush_m_args:
    \@@_add_default_E:nn {#1} {#2}
    \@@_add_grabber:N E
    \tl_put_right:Nn \l_@@_signature_tl { {#1} }
    \@@_prepare_signature:N
  }
%    \end{macrocode}
% \end{macro}
%
% \begin{macro}{\@@_add_type_m:w}
%   The \texttt{m} type is special as short arguments which are not
%   post-processed are simply counted at this stage. Thus there is a check
%   to see if either of these cases apply. If so, a one-argument grabber
%   is added to the signature. On the other hand, if a standard short
%   argument is required it is simply counted at this stage, to be
%   added later using \cs{@@_flush_m_args:}.
%    \begin{macrocode}
\cs_new_protected:Npn \@@_add_type_m:w
  {
    \@@_add_default:
    \bool_if:NTF \l_@@_prefixed_bool
      { \@@_add_grabber:N m }
      { \int_incr:N \l_@@_m_args_int }
    \@@_prepare_signature:N
  }
%    \end{macrocode}
% \end{macro}
%
% \begin{macro}{\@@_add_type_R:w}
%   The \texttt{R}-type argument is very similar to the \texttt{D}-type.
%    \begin{macrocode}
\cs_new_protected:Npn \@@_add_type_R:w #1#2#3
  {
    \@@_flush_m_args:
    \@@_add_default:n {#3}
    \@@_add_grabber:N R
    \tl_put_right:Nn \l_@@_signature_tl { #1 #2 }
    \@@_prepare_signature:N
  }
%    \end{macrocode}
% \end{macro}
%
% \begin{macro}{\@@_add_type_t:w}
%   Setting up a \texttt{t} argument means collecting one token for the test,
%   and adding it along with the grabber to the signature.
%    \begin{macrocode}
\cs_new_protected:Npn \@@_add_type_t:w #1
  {
    \@@_flush_m_args:
    \@@_add_default:
    \@@_add_grabber:N t
    \tl_put_right:Nn \l_@@_signature_tl {#1}
    \@@_prepare_signature:N
  }
%    \end{macrocode}
% \end{macro}
%
% \begin{macro}{\@@_add_type_v:w}
%   At this stage, the \texttt{v} argument is identical to \texttt{l}
%   except that since the grabber may fail to read a verbatim argument
%   we need a default value.
%    \begin{macrocode}
\cs_new_protected:Npn \@@_add_type_v:w
  {
    \@@_flush_m_args:
    \exp_args:No \@@_add_default:n \c_novalue_tl
    \@@_add_grabber:N v
    \@@_prepare_signature:N
  }
%    \end{macrocode}
% \end{macro}
%
% \begin{macro}{\@@_flush_m_args:}
%   As \texttt{m} arguments are simply counted, there is a need to add
%   them to the token register in a block. As this function can only
%   be called if something other than \texttt{m} turns up, the flag can
%   be switched here.
%    \begin{macrocode}
\cs_new_protected:Npn \@@_flush_m_args:
  {
    \int_compare:nNnT \l_@@_m_args_int > 0
      {
        \tl_put_right:Nx \l_@@_signature_tl
          { \exp_not:c { @@_grab_m_ \int_use:N \l_@@_m_args_int :w } }
        \tl_put_right:Nx \l_@@_process_all_tl
          { \prg_replicate:nn { \l_@@_m_args_int } { { } } }
      }
    \int_zero:N \l_@@_m_args_int
  }
%    \end{macrocode}
% \end{macro}
%
% \begin{macro}{\@@_add_grabber:N}
%   To keep the various checks needed in one place, adding the grabber
%   to the signature is done here.  The only questions are whether the
%   grabber should be long or not, and whether to obey spaces.  The
%   \cs{l_@@_obey_spaces_bool} boolean can only be \texttt{true} for
%   trailing optional arguments.  In that case spaces will not be
%   ignored when looking for that optional argument.
%    \begin{macrocode}
\cs_new_protected:Npn \@@_add_grabber:N #1
  {
    \tl_put_right:Nx \l_@@_signature_tl
      {
        \exp_not:c
          {
            @@_grab_ #1
            \bool_if:NT \l_@@_long_bool { _long }
            \bool_if:NT \l_@@_obey_spaces_bool { _obey_spaces }
            :w
          }
      }
    \bool_set_false:N \l_@@_long_bool
    \bool_set_false:N \l_@@_obey_spaces_bool
    \tl_put_right:Nx \l_@@_process_all_tl
      { { \exp_not:o \l_@@_process_one_tl } }
    \tl_clear:N \l_@@_process_one_tl
  }
%    \end{macrocode}
% \end{macro}
%
% \begin{macro}{\@@_add_default:n, \@@_add_default:, \@@_add_default_E:nn}
%   Store the default value of an argument, or rather code that gives
%   that default value (it may involve other arguments).  This is
%   \cs{c_novalue_tl} for arguments with no actual default or with
%   default |-NoValue-|; and (in a brace group) \cs{prg_do_nothing:}
%   followed by a default value for others.  For \texttt{E}-type
%   arguments, pad the defaults |#2| with some \cs{c_novalue_tl}
%   until there are as many as embellishments~|#1|.  These functions are
%   also used when defining expandable commands.
%    \begin{macrocode}
\cs_new_protected:Npn \@@_add_default:n #1
  {
    \tl_if_novalue:nTF {#1}
      { \@@_add_default: }
      {
        \int_incr:N \l_@@_current_arg_int
        \bool_set_true:N \l_@@_defaults_bool
        \tl_put_right:Nn \l_@@_defaults_tl { { \prg_do_nothing: #1 } }
      }
  }
\cs_new_protected:Npn \@@_add_default:
  {
    \int_incr:N \l_@@_current_arg_int
    \tl_put_right:Nn \l_@@_defaults_tl { \c_novalue_tl }
  }
\cs_new_protected:Npn \@@_add_default_E:nn #1#2
  {
    \tl_map_function:nN {#2} \@@_add_default:n
    \prg_replicate:nn
      { \tl_count:n {#1} - \tl_count:n {#2} }
      { \@@_add_default: }
  }
%    \end{macrocode}
% \end{macro}
%
% \subsection{Setting up expandable types}
%
% The approach here is not dissimilar to that for standard types, but fewer types
% are supported. There is
% also a need to define the per-function auxiliaries: this is done here, while
% the general grabbers are dealt with later.
%
% \begin{macro}{\@@_add_expandable_type_+:w}
%   We have already checked that short arguments are before long
%   arguments, so \cs{l_@@_long_bool} only changes from \texttt{false}
%   to \texttt{true} once (and there is no need to reset it after each
%   argument).  Also knock back the argument count because |+| is not an
%   argument.  Continue the loop.
%    \begin{macrocode}
\cs_new_protected:cpn { @@_add_expandable_type_+:w }
  {
    \bool_set_true:N \l_@@_long_bool
    \@@_prepare_signature:N
  }
%    \end{macrocode}
% \end{macro}
%
% \begin{macro}{\@@_add_expandable_type_D:w}
% \begin{macro}{\@@_add_expandable_type_D_aux:NNNn}
% \begin{macro}{\@@_add_expandable_type_D_aux:NNN}
% \begin{macro}{\@@_add_expandable_type_D_aux:NN}
%   The set up for \texttt{D}-type arguments involves constructing a
%   rather complex auxiliary which is used
%   repeatedly when grabbing. There is an auxiliary here so that the
%   \texttt{R}-type can share code readily: |#1| is |D| or~|R|.
%   The |_aux:NN| auxiliary is needed if the two delimiting tokens are
%   identical: in contrast to the non-expandable route, the grabber here
%   has to act differently for this case.
%    \begin{macrocode}
\cs_new_protected:Npn \@@_add_expandable_type_D:w
  { \@@_add_expandable_type_D_aux:NNNn D }
\cs_new_protected:Npn \@@_add_expandable_type_D_aux:NNNn #1#2#3#4
  {
    \@@_add_default:n {#4}
    \tl_if_eq:nnTF {#2} {#3}
      { \@@_add_expandable_type_D_aux:NN #1 #2 }
      { \@@_add_expandable_type_D_aux:NNN #1 #2 #3 }
    \@@_prepare_signature:N
  }
\cs_new_protected:Npn \@@_add_expandable_type_D_aux:NNN #1#2#3
  {
    \bool_if:NTF \l_@@_long_bool
      { \cs_set:cpx }
      { \cs_set_nopar:cpx }
      { \l_@@_expandable_aux_name_tl } ##1 ##2 #2 ##3 \q_@@ ##4 #3
      { ##1 {##2} {##3} {##4} }
    \@@_add_expandable_grabber:nn {#1}
      {
        \exp_not:c  { \l_@@_expandable_aux_name_tl }
        \exp_not:n { #2 #3 }
      }
  }
\cs_new_protected:Npn \@@_add_expandable_type_D_aux:NN #1#2
  {
    \bool_if:NTF \l_@@_long_bool
      { \cs_set:cpx }
      { \cs_set_nopar:cpx }
      { \l_@@_expandable_aux_name_tl } ##1 #2 ##2 #2
      { ##1 {##2} }
    \@@_add_expandable_grabber:nn { #1_alt }
      {
        \exp_not:c  { \l_@@_expandable_aux_name_tl }
        \exp_not:n {#2}
      }
  }
%    \end{macrocode}
% \end{macro}
% \end{macro}
% \end{macro}
% \end{macro}
%
% \begin{macro}{\@@_add_expandable_type_E:w}
% \begin{macro}{\@@_add_expandable_type_E_aux:n}
%   For each embellishment, use \cs{@@_get_grabber:NN} to obtain an
%   auxiliary delimited by that token and store a pair constituted of
%   the auxiliary and the token in \cs{l_@@_tmpb_tl}, before appending
%   the whole set of these pairs to the signature, and an equal number
%   of |-NoValue-| markers (regardless of the default values of
%   arguments).  Set the current argument appropriately.
%    \begin{macrocode}
\cs_new_protected:Npn \@@_add_expandable_type_E:w #1#2
  {
    \@@_add_default_E:nn {#1} {#2}
    \tl_clear:N \l_@@_tmpb_tl
    \tl_map_function:nN {#1} \@@_add_expandable_type_E_aux:n
    \@@_add_expandable_grabber:nn
      { E \bool_if:NT \l_@@_long_bool { _long } }
      {
        { \exp_not:o \l_@@_tmpb_tl }
        {
          \prg_replicate:nn { \tl_count:n {#1} }
            { { \c_novalue_tl } }
        }
      }
    \@@_prepare_signature:N
  }
\cs_new_protected:Npn \@@_add_expandable_type_E_aux:n #1
  {
    \@@_get_grabber:NN #1 \l_@@_tmpa_tl
    \tl_put_right:Nx \l_@@_tmpb_tl
      { \exp_not:o \l_@@_tmpa_tl \exp_not:N #1 }
  }
%    \end{macrocode}
% \end{macro}
% \end{macro}
%
% \begin{macro}{\@@_add_expandable_type_m:w}
%   Unlike the standard case, when working expandably each argument is always
%   grabbed separately.
%    \begin{macrocode}
\cs_new_protected:Npn \@@_add_expandable_type_m:w
  {
    \@@_add_default:
    \@@_add_expandable_grabber:nn
      { m \bool_if:NT \l_@@_long_bool { _long } } { }
    \@@_prepare_signature:N
  }
%    \end{macrocode}
% \end{macro}
%
% \begin{macro}{\@@_add_expandable_type_R:w}
%   The \texttt{R}-type is very similar to the \texttt{D}-type
%   argument, and so the same internals are used.
%    \begin{macrocode}
\cs_new_protected:Npn \@@_add_expandable_type_R:w
  { \@@_add_expandable_type_D_aux:NNNn R }
%    \end{macrocode}
% \end{macro}
%
% \begin{macro}{\@@_add_expandable_type_t:w}
%   An auxiliary delimited by |#1| is built now.  It will be used to
%   test for the presence of that token.
%    \begin{macrocode}
\cs_new_protected:Npn \@@_add_expandable_type_t:w #1
  {
    \@@_add_default:
    \@@_get_grabber:NN #1 \l_@@_tmpa_tl
    \@@_add_expandable_grabber:nn { t }
      {
        \exp_not:o \l_@@_tmpa_tl
        \exp_not:N #1
      }
    \@@_prepare_signature:N
  }
%    \end{macrocode}
% \end{macro}
%
% \begin{macro}{\@@_add_expandable_grabber:nn}
%   This is called for all arguments to place the right grabber in the
%   signature.
%    \begin{macrocode}
\cs_new_protected:Npn \@@_add_expandable_grabber:nn #1#2
  {
    \tl_put_right:Nx \l_@@_signature_tl
      { \exp_not:c { @@_expandable_grab_ #1 :w } #2 }
  }
%    \end{macrocode}
% \end{macro}
%
% \begin{macro}{\@@_get_grabber:NN}
% \begin{macro}{\@@_get_grabber_auxi:NN}
% \begin{macro}{\@@_get_grabber_auxii:NN}
%   Given a token |#1|, defines an expandable function delimited by that
%   token and stores it in the token list~|#2|.  The function is named
%   after the token, unless that function name is already taken by some
%   other grabber (this can happen in the rare case where delimiters
%   with different category codes are used in the same document): in
%   that case use a global counter to get a unique name.  Since the
%   grabbers are not named after \pkg{xparse} commands they should not
%   be used to get material from the input stream.
%    \begin{macrocode}
\cs_new_protected:Npn \@@_get_grabber:NN #1#2
  {
    \cs_set:Npn \@@_tmp:w ##1 #1 {##1}
    \exp_args:Nc \@@_get_grabber_auxi:NN
      { @@_grabber_ \token_to_str:N #1 :w } #2
  }
\cs_new_protected:Npn \@@_get_grabber_auxi:NN #1#2
  {
    \cs_if_eq:NNTF \@@_tmp:w #1
      { \tl_set:Nn #2 {#1} }
      {
        \cs_if_exist:NTF #1
          {
            \int_gincr:N \g_@@_grabber_int
            \exp_args:Nc \@@_get_grabber_auxi:NN
              {
                @@_grabber_
                - \int_use:N \g_@@_grabber_int :w
              }
              #2
          }
          { \@@_get_grabber_auxii:NN #1 #2 }
      }
  }
\cs_new_protected:Npn \@@_get_grabber_auxii:NN #1#2
  {
    \cs_set_eq:NN #1 \@@_tmp:w
    \tl_set:Nn #2 {#1}
  }
%    \end{macrocode}
% \end{macro}
% \end{macro}
% \end{macro}
%
% \subsection{Grabbing arguments}
%
% All of the grabbers follow the same basic pattern. The initial
% function stores in \cs{l_@@_signature_tl} the code to grab further
% arguments, defines (the function in) \cs{l_@@_fn_tl} that will grab
% the argument, and calls it.
%
% Defining \cs{l_@@_fn_tl} means determining whether to use
% \cs{cs_set:Npn} or \cs{cs_set_nopar:Npn}, and for optional arguments
% whether to skip spaces. Once the argument is found, \cs{l_@@_fn_tl}
% calls \cs{@@_add_arg:n}, responsible for calling processors and
% grabbing further arguments.
%
% \begin{macro}
%   {
%     \@@_grab_b:w,
%     \@@_grab_b_long:w,
%     \@@_grab_b_obey_spaces:w,
%     \@@_grab_b_long_obey_spaces:w,
%     \@@_grab_b_aux:NNw,
%     \@@_grab_b_end:Nw
%   }
%   This uses the well-tested code of \texttt{D}-type arguments,
%   skipping the peeking step because the \texttt{b}-type argument is
%   always present, and adding a cleanup stage at the end by hijacking
%   the signature.  The clean-up consists of properly
%   dealing with \cs{l_@@_args_tl} and also putting back the \cs{end}
%   that served as an end-delimiter: this \cs{end} receives the
%   environment name as its argument and is run normally.  The
%   \texttt{D}-type code stores the argument found (body of the
%   environment) as a brace group in \cs{l_@@_args_tl} and depending on
%   the presence of a prefix~|!| we trim spaces or not before adding
%   this braced argument into the saved \cs{l_@@_args_tl}.
%   The strange \verb*|\begin | control sequence is there for display
%     purposes only: it has to look like |\begin| in the terminal but
%       not to delimited arguments.
%    \begin{macrocode}
\cs_new_protected:Npn \@@_grab_b:w
  { \@@_grab_b_aux:NNw \cs_set_protected_nopar:Npn \tl_trim_spaces:n }
\cs_new_protected:Npn \@@_grab_b_long:w
  { \@@_grab_b_aux:NNw \cs_set_protected:Npn \tl_trim_spaces:n }
\cs_new_protected:Npn \@@_grab_b_obey_spaces:w
  { \@@_grab_b_aux:NNw \cs_set_protected_nopar:Npn \exp_not:n }
\cs_new_protected:Npn \@@_grab_b_long_obey_spaces:w
  { \@@_grab_b_aux:NNw \cs_set_protected:Npn \exp_not:n }
\cs_new_protected:Npn \@@_grab_b_aux:NNw #1#2#3 \@@_run_code:
  {
    \@@_grab_D_aux:NNnN \begin \end {#3} #1
    \tl_put_left:Nn \l_@@_signature_tl { \@@_grab_b_end:Nw #2 }
    \tl_set_eq:NN \l_@@_saved_args_tl \l_@@_args_tl
    \tl_clear:N \l_@@_args_tl
    \exp_args:Nc \l_@@_fn_tl { begin ~ }
  }
\cs_new_protected:Npn \@@_grab_b_end:Nw #1#2 \@@_run_code:
  {
    \tl_set:Nx \l_@@_args_tl
      {
        \exp_not:V \l_@@_saved_args_tl
        { \exp_after:wN #1 \l_@@_args_tl }
      }
    #2
    \@@_run_code:
    \end
  }
%    \end{macrocode}
% \end{macro}
%
% \begin{macro}{\@@_grab_D:w}
% \begin{macro}{\@@_grab_D_long:w}
% \begin{macro}{\@@_grab_D_obey_spaces:w}
% \begin{macro}{\@@_grab_D_long_obey_spaces:w}
%   The generic delimited argument grabber. The auxiliary function does
%   a peek test before calling \cs{@@_grab_D_call:Nw}, so that the
%   optional nature of the argument works as expected.
%    \begin{macrocode}
\cs_new_protected:Npn \@@_grab_D:w #1#2#3 \@@_run_code:
  {
    \@@_grab_D_aux:NNnNN #1 #2 {#3} \cs_set_protected_nopar:Npn
      \@@_peek_nonspace_remove:NTF
  }
\cs_new_protected:Npn \@@_grab_D_long:w #1#2#3 \@@_run_code:
  {
    \@@_grab_D_aux:NNnNN #1 #2 {#3} \cs_set_protected:Npn
      \@@_peek_nonspace_remove:NTF
  }
\cs_new_protected:Npn \@@_grab_D_obey_spaces:w #1#2#3 \@@_run_code:
  {
    \@@_grab_D_aux:NNnNN #1 #2 {#3} \cs_set_protected_nopar:Npn
      \@@_peek_meaning_remove:NTF
  }
\cs_new_protected:Npn \@@_grab_D_long_obey_spaces:w #1#2#3 \@@_run_code:
  {
    \@@_grab_D_aux:NNnNN #1 #2 {#3} \cs_set_protected:Npn
      \@@_peek_meaning_remove:NTF
  }
%    \end{macrocode}
% \begin{macro}{\@@_grab_D_aux:NNnNN}
% \begin{macro}{\@@_grab_D_aux:NNnN}
%   This is a bit complicated. The idea is that, in order to check for
%   nested optional argument tokens (\texttt{[[...]]} and so on) the
%   argument needs to be grabbed without removing any braces at all. If
%   this is not done, then cases like |[{[}]| fail. So after testing for
%   an optional argument, it is collected piece-wise. Inserting a quark
%   prevents loss of braces, and there is then a test to see if there are
%   nested delimiters to handle.
%    \begin{macrocode}
\cs_new_protected:Npn \@@_grab_D_aux:NNnNN #1#2#3#4#5
  {
    \@@_grab_D_aux:NNnN #1#2 {#3} #4
    #5 #1
      { \@@_grab_D_call:Nw #1 }
      { \@@_add_arg:o \c_novalue_tl }
  }
%    \end{macrocode}
%   Inside the \enquote{standard} grabber, there is a test to see if the
%   grabbed argument is entirely enclosed by braces. There are a couple of
%   extra factors to allow for: the argument might be entirely empty, and
%   spaces at the start and end of the input must be retained around a brace
%   group. Also notice that a \emph{blank} argument might still contain
%   spaces.
%    \begin{macrocode}
\cs_new_protected:Npn \@@_grab_D_aux:NNnN #1#2#3#4
  {
    \tl_set:Nn \l_@@_signature_tl {#3}
    \exp_after:wN #4 \l_@@_fn_tl ##1 #2
      {
        \tl_if_in:nnTF {##1} {#1}
          { \@@_grab_D_nested:NNnN #1 #2 {##1} #4 }
          {
            \tl_if_blank:oTF { \use_none:n ##1 }
              { \@@_add_arg:o { \use_none:n ##1 } }
              {
                \str_if_eq:eeTF
                  { \exp_not:o { \use_none:n ##1 } }
                  { { \exp_not:o { \use_ii:nnn ##1 \q_nil } } }
                  { \@@_add_arg:o { \use_ii:nn ##1 } }
                  { \@@_add_arg:o { \use_none:n ##1 } }
              }
          }
      }
  }
%    \end{macrocode}
% \end{macro}
% \end{macro}
% \end{macro}
% \end{macro}
% \end{macro}
% \end{macro}
% \begin{macro}{\@@_grab_D_nested:NNnN}
% \begin{macro}{\@@_grab_D_nested:w}
% \begin{macro}{\l_@@_nesting_a_tl}
% \begin{macro}{\l_@@_nesting_b_tl}
% \begin{macro}{\q_@@}
%   Catching nested optional arguments means more work. The aim here is
%   to collect up each pair of optional tokens without \TeX{} helping out,
%   and without counting anything. The code above will already have
%   removed the leading opening token and a closing token, but the
%   wrong one. The aim is then to work through the material grabbed
%   so far and divide it up on each opening token, grabbing a closing
%   token to match (thus working in pairs). Once there are no opening
%   tokens, then there is a second check to see if there are any
%   opening tokens in the second part of the argument (for things
%   like |[][]|). Once everything has been found, the entire collected
%   material is added to the output as a single argument. The only tricky part
%   here is ensuring that any grabbing function that might run away is named
%   after the function currently being parsed and not after \pkg{xparse}. That
%   leads to some rather complex nesting! There is also a need to prevent the
%   loss of any braces, hence the insertion and removal of quarks along the
%   way.
%    \begin{macrocode}
\tl_new:N \l_@@_nesting_a_tl
\tl_new:N \l_@@_nesting_b_tl
\quark_new:N \q_@@
\cs_new_protected:Npn \@@_grab_D_nested:NNnN #1#2#3#4
  {
    \tl_clear:N \l_@@_nesting_a_tl
    \tl_clear:N \l_@@_nesting_b_tl
    \exp_after:wN #4 \l_@@_fn_tl ##1 #1 ##2 \q_@@ ##3 #2
      {
        \tl_put_right:No \l_@@_nesting_a_tl { \use_none:n ##1 #1 }
        \tl_put_right:No \l_@@_nesting_b_tl { \use_i:nn #2 ##3 }
        \tl_if_in:nnTF {##2} {#1}
          {
            \l_@@_fn_tl
              \q_nil ##2 \q_@@ \ERROR
          }
          {
            \tl_put_right:Nx \l_@@_nesting_a_tl
              { \@@_grab_D_nested:w \q_nil ##2 \q_stop }
            \tl_if_in:NnTF \l_@@_nesting_b_tl {#1}
              {
                \tl_set_eq:NN \l_@@_tmpa_tl \l_@@_nesting_b_tl
                \tl_clear:N \l_@@_nesting_b_tl
                \exp_after:wN \l_@@_fn_tl \exp_after:wN
                  \q_nil \l_@@_tmpa_tl \q_nil \q_@@ \ERROR
              }
              {
                \tl_put_right:No \l_@@_nesting_a_tl
                  \l_@@_nesting_b_tl
                \@@_add_arg:V \l_@@_nesting_a_tl
              }
          }
      }
    \l_@@_fn_tl #3 \q_nil \q_@@ \ERROR
  }
\cs_new:Npn \@@_grab_D_nested:w #1 \q_nil \q_stop
  { \exp_not:o { \use_none:n #1 } }
%    \end{macrocode}
% \end{macro}
% \end{macro}
% \end{macro}
% \end{macro}
% \end{macro}
% \begin{macro}{\@@_grab_D_call:Nw}
%   For \texttt{D} and \texttt{R}-type arguments, to avoid losing any
%   braces, a token needs to be inserted before the argument to be grabbed.
%   If the argument runs away because the closing token is missing then this
%   inserted token shows up in the terminal. Ideally, |#1| would therefore be
%   used directly, but that is no good as it will mess up the rest of the
%   grabber. Instead, a copy of |#1| with an altered category code is used,
%   as this will look right in the terminal but will not mess up the grabber.
%   The only issue then is that the category code of |#1| is unknown. So there
%   is a quick test to ensure that the inserted token can never be matched by
%   the grabber. (This assumes that the open and close delimiters are not the
%   same character with different category codes, but that really should not
%   happen in any sensible document-level syntax.)
%   An exception is when |#1| is a control sequence token, in which case the
%   character-token treatment is no good because if hit with \cs{token_to_str:N}
%   it would add sputios tokens to the argument. In this case a different
%   branch is taken. The token inserted is then the same \meta{csname} as |#1|,
%   but with a space appended, so that the grabber don't see it as another
%   of the same delimiter.
%    \begin{macrocode}
\cs_new_protected_nopar:Npn \@@_grab_D_call:Nw #1
  {
    \token_if_eq_catcode:NNTF + #1
      {
        \exp_after:wN \exp_after:wN \exp_after:wN
          \l_@@_fn_tl \char_generate:nn { `#1 } { 11 }
      }
      {
        \@@_token_if_cs:NTF #1
          {
            \exp_after:wN \l_@@_fn_tl
            \cs:w \cs_to_str:N #1 ~ \cs_end:
          }
          {
            \exp_after:wN \l_@@_fn_tl
            \token_to_str:N #1
          }
      }
  }
%    \end{macrocode}
% \end{macro}
%
% \begin{macro}
%   {
%     \@@_grab_E:w, \@@_grab_E_long:w,
%     \@@_grab_E_obey_spaces:w, \@@_grab_E_long_obey_spaces:w
%   }
% \begin{macro}{\@@_grab_E:nnNN}
% \begin{macro}{\@@_grab_E_loop:NnN}
% \begin{macro}{\@@_grab_E_finalise:}
%   Everything here needs to point to a loop.
%    \begin{macrocode}
\cs_new_protected:Npn \@@_grab_E:w #1#2 \@@_run_code:
  {
    \@@_grab_E:nnNN {#1} {#2}
      \cs_set_protected_nopar:Npn
      \@@_peek_nonspace_remove:NTF
  }
\cs_new_protected:Npn \@@_grab_E_long:w #1#2 \@@_run_code:
  {
    \@@_grab_E:nnNN {#1} {#2}
      \cs_set_protected:Npn
      \@@_peek_nonspace_remove:NTF
  }
\cs_new_protected:Npn \@@_grab_E_obey_spaces:w #1#2 \@@_run_code:
  {
    \@@_grab_E:nnNN {#1} {#2}
      \cs_set_protected_nopar:Npn
      \@@_peek_meaning_remove:NTF
  }
\cs_new_protected:Npn \@@_grab_E_long_obey_spaces:w #1#2 \@@_run_code:
  {
    \@@_grab_E:nnNN {#1} {#2}
      \cs_set_protected:Npn
      \@@_peek_meaning_remove:NTF
  }
%    \end{macrocode}
%   A loop is needed here to allow a random ordering of keys. These are
%   searched for one at a time, with any not found needing to be tracked:
%   they can appear later. The grabbed values are held in a property list
%   which is then turned into an ordered list to be passed back to the user.
%    \begin{macrocode}
\cs_new_protected:Npn \@@_grab_E:nnNN #1#2#3#4
  {
    \exp_after:wN #3 \l_@@_fn_tl ##1##2##3
      {
        \prop_put:Nnn \l_@@_tmp_prop {##1} {##3}
        \@@_grab_E_loop:NnN #4 { } ##2 \q_recursion_stop
      }
    \prop_clear:N \l_@@_tmp_prop
    \tl_set:Nn \l_@@_signature_tl {#2}
    \cs_set_protected:Npn \@@_grab_E_finalise:
      {
        \tl_map_inline:nn {#1}
          {
            \prop_get:NnNF \l_@@_tmp_prop {####1} \l_@@_tmpb_tl
              { \tl_set_eq:NN \l_@@_tmpb_tl \c_novalue_tl }
            \tl_put_right:Nx \l_@@_args_tl
              { { \exp_not:V \l_@@_tmpb_tl } }
          }
        \l_@@_signature_tl \@@_run_code:
      }
    \@@_grab_E_loop:NnN #4 { } #1 \q_recursion_tail \q_recursion_stop
  }
\cs_new_protected:Npn \@@_grab_E_loop:NnN #1#2#3#4 \q_recursion_stop
  {
    \cs_if_eq:NNTF #3 \q_recursion_tail
      { \@@_grab_E_finalise: }
      {
        #1 #3
          { \l_@@_fn_tl #3 {#2#4} }
          { \@@_grab_E_loop:NnN #1 {#2#3} #4 \q_recursion_stop }
      }
  }
\cs_new_protected:Npn \@@_grab_E_finalise: { }
%    \end{macrocode}
% \end{macro}
% \end{macro}
% \end{macro}
% \end{macro}
%
%
% \begin{macro}{\@@_grab_m:w}
% \begin{macro}{\@@_grab_m_long:w}
%   Collecting a single mandatory argument is quite easy.
%    \begin{macrocode}
\cs_new_protected:Npn \@@_grab_m:w #1 \@@_run_code:
  {
    \tl_set:Nn \l_@@_signature_tl {#1}
    \exp_after:wN \cs_set_protected_nopar:Npn \l_@@_fn_tl ##1
      { \@@_add_arg:n {##1} }
    \l_@@_fn_tl
  }
\cs_new_protected:Npn \@@_grab_m_long:w #1 \@@_run_code:
  {
    \tl_set:Nn \l_@@_signature_tl {#1}
    \exp_after:wN \cs_set_protected:Npn \l_@@_fn_tl ##1
      { \@@_add_arg:n {##1} }
    \l_@@_fn_tl
  }
%    \end{macrocode}
% \end{macro}
% \end{macro}
%
% \begin{macro}{\@@_grab_m_1:w}
% \begin{macro}{\@@_grab_m_2:w}
% \begin{macro}{\@@_grab_m_3:w}
% \begin{macro}{\@@_grab_m_4:w}
% \begin{macro}{\@@_grab_m_5:w}
% \begin{macro}{\@@_grab_m_6:w}
% \begin{macro}{\@@_grab_m_7:w}
% \begin{macro}{\@@_grab_m_8:w}
% \begin{macro}{\@@_grab_m_9:w}
% \begin{macro}{\@@_grab_m_aux:Nnnnnnnnn}
%   Grabbing 1--8 mandatory arguments is done by giving 8--1 known
%   arguments to a 9-argument function that stores them in
%   \cs{l_@@_args_tl}.  For simplicity, grabbing 9 mandatory arguments
%   is done by grabbing 5 then 4 arguments.
%    \begin{macrocode}
\cs_new_protected_nopar:Npn \@@_grab_m_aux:Nnnnnnnnn #1#2#3#4#5#6#7#8#9
  {
    \tl_put_right:No \l_@@_args_tl
      { #1 {#2} {#3} {#4} {#5} {#6} {#7} {#8} {#9} }
    \l_@@_signature_tl \@@_run_code:
  }
\cs_new_protected:cpn { @@_grab_m_1:w } #1 \@@_run_code:
  {
    \tl_set:Nn \l_@@_signature_tl {#1}
    \exp_after:wN \cs_set_eq:NN \l_@@_fn_tl \@@_grab_m_aux:Nnnnnnnnn
    \l_@@_fn_tl \use_none:nnnnnnn { } { } { } { } { } { } { }
  }
\cs_new_protected:cpn { @@_grab_m_2:w } #1 \@@_run_code:
  {
    \tl_set:Nn \l_@@_signature_tl {#1}
    \exp_after:wN \cs_set_eq:NN \l_@@_fn_tl \@@_grab_m_aux:Nnnnnnnnn
    \l_@@_fn_tl \use_none:nnnnnn { } { } { } { } { } { }
  }
\cs_new_protected:cpn { @@_grab_m_3:w } #1 \@@_run_code:
  {
    \tl_set:Nn \l_@@_signature_tl {#1}
    \exp_after:wN \cs_set_eq:NN \l_@@_fn_tl \@@_grab_m_aux:Nnnnnnnnn
    \l_@@_fn_tl \use_none:nnnnn { } { } { } { } { }
  }
\cs_new_protected:cpn { @@_grab_m_4:w } #1 \@@_run_code:
  {
    \tl_set:Nn \l_@@_signature_tl {#1}
    \exp_after:wN \cs_set_eq:NN \l_@@_fn_tl \@@_grab_m_aux:Nnnnnnnnn
    \l_@@_fn_tl \use_none:nnnn { } { } { } { }
  }
\cs_new_protected:cpn { @@_grab_m_5:w } #1 \@@_run_code:
  {
    \tl_set:Nn \l_@@_signature_tl {#1}
    \exp_after:wN \cs_set_eq:NN \l_@@_fn_tl \@@_grab_m_aux:Nnnnnnnnn
    \l_@@_fn_tl \use_none:nnn { } { } { }
  }
\cs_new_protected:cpn { @@_grab_m_6:w } #1 \@@_run_code:
  {
    \tl_set:Nn \l_@@_signature_tl {#1}
    \exp_after:wN \cs_set_eq:NN \l_@@_fn_tl \@@_grab_m_aux:Nnnnnnnnn
    \l_@@_fn_tl \use_none:nn { } { }
  }
\cs_new_protected:cpn { @@_grab_m_7:w } #1 \@@_run_code:
  {
    \tl_set:Nn \l_@@_signature_tl {#1}
    \exp_after:wN \cs_set_eq:NN \l_@@_fn_tl \@@_grab_m_aux:Nnnnnnnnn
    \l_@@_fn_tl \use_none:n { }
  }
\cs_new_protected:cpn { @@_grab_m_8:w } #1 \@@_run_code:
  {
    \tl_set:Nn \l_@@_signature_tl {#1}
    \exp_after:wN \cs_set_eq:NN \l_@@_fn_tl \@@_grab_m_aux:Nnnnnnnnn
    \l_@@_fn_tl \prg_do_nothing:
  }
\cs_new_protected:cpx { @@_grab_m_9:w }
  {
    \exp_not:c { @@_grab_m_5:w }
    \exp_not:c { @@_grab_m_4:w }
  }
%    \end{macrocode}
% \end{macro}
% \end{macro}
% \end{macro}
% \end{macro}
% \end{macro}
% \end{macro}
% \end{macro}
% \end{macro}
% \end{macro}
% \end{macro}
%
% \begin{macro}{\@@_grab_R:w, \@@_grab_R_long:w}
% \begin{macro}{\@@_grab_R_aux:NNnN}
%  The grabber for \texttt{R}-type arguments is basically the same as
%  that for \texttt{D}-type ones, but always skips spaces (as it is mandatory)
%  and has a hard-coded error message.
%    \begin{macrocode}
\cs_new_protected:Npn \@@_grab_R:w #1#2#3 \@@_run_code:
  { \@@_grab_R_aux:NNnN #1 #2 {#3} \cs_set_protected_nopar:Npn }
\cs_new_protected:Npn \@@_grab_R_long:w #1#2#3 \@@_run_code:
  { \@@_grab_R_aux:NNnN #1 #2 {#3} \cs_set_protected:Npn }
\cs_new_protected:Npn \@@_grab_R_aux:NNnN #1#2#3#4
  {
    \@@_grab_D_aux:NNnN #1 #2 {#3} #4
    \@@_peek_nonspace_remove:NTF #1
      { \@@_grab_D_call:Nw #1 }
      {
        \msg_error:nnxx { cmd } { missing-required }
          { \@@_environment_or_command: }
          { \token_to_str:N #1 }
        \@@_add_arg:o \c_novalue_tl
      }
  }
%    \end{macrocode}
% \end{macro}
% \end{macro}
%
% \begin{macro}{\@@_grab_t:w}
% \begin{macro}{\@@_grab_t_obey_spaces:w}
% \begin{macro}{\@@_grab_t_aux:NNw}
%   Dealing with a token is quite easy. Check the match, remove the
%   token if needed and add a flag to the output.
%    \begin{macrocode}
\cs_new_protected:Npn \@@_grab_t:w
  { \@@_grab_t_aux:NNw \@@_peek_nonspace_remove:NTF }
\cs_new_protected:Npn \@@_grab_t_obey_spaces:w
  { \@@_grab_t_aux:NNw \@@_peek_meaning_remove:NTF }
\cs_new_protected:Npn \@@_grab_t_aux:NNw #1#2#3 \@@_run_code:
  {
    \tl_set:Nn \l_@@_signature_tl {#3}
    \exp_after:wN \cs_set_protected:Npn \l_@@_fn_tl
      {
        #1 #2
          { \@@_add_arg:n { \BooleanTrue } }
          { \@@_add_arg:n { \BooleanFalse } }
      }
    \l_@@_fn_tl
  }
%    \end{macrocode}
% \end{macro}
% \end{macro}
% \end{macro}
%
% \begin{variable}{\l_@@_v_arg_tl}
%    \begin{macrocode}
\tl_new:N \l_@@_v_arg_tl
%    \end{macrocode}
% \end{variable}
%
% \begin{macro}{\@@_grab_v:w}
% \begin{macro}{\@@_grab_v_long:w}
% \begin{macro}{\@@_grab_v_aux:w}
% \begin{macro}{\@@_grab_v_group_end:}
%   The opening delimiter is the first non-space token, and is never
%   read verbatim.  This is required by consistency with the case where
%   the preceding argument was optional and absent: then \TeX{} has
%   already read and tokenized that token when looking for the optional
%   argument.  The first thing is thus to check is that this delimiter
%   is a character, and to distinguish the case of a left brace (in that
%   case, \cs{group_align_safe_end:} is needed to compensate for the
%   begin-group character that was just seen).  Then set verbatim
%   catcodes with \cs{@@_grab_v_aux_catcodes:}.
%
%   The group keep catcode changes local, and
%   \cs{group_align_safe_begin/end:} allow to use a character
%   with category code~$4$ (normally |&|) as the delimiter (all commands
%   do \cs{group_align_safe_begin/end:}, so there's no need to do that
%   again here).
%   It is ended by \cs{@@_grab_v_group_end:}, which smuggles
%   the collected argument out of the group.
%    \begin{macrocode}
\cs_new_protected:Npn \@@_grab_v:w
  {
    \bool_set_false:N \l_@@_long_bool
    \@@_grab_v_aux:w
  }
\cs_new_protected:Npn \@@_grab_v_long:w
  {
    \bool_set_true:N \l_@@_long_bool
    \@@_grab_v_aux:w
  }
\cs_new_protected:Npn \@@_grab_v_aux:w #1 \@@_run_code:
  {
    \tl_set:Nn \l_@@_signature_tl {#1}
    \group_begin:
      \tex_escapechar:D = 92 \scan_stop:
      \tl_clear:N \l_@@_v_arg_tl
      \peek_remove_spaces:n
        {
          \peek_meaning_remove:NTF \c_group_begin_token
            {
              \group_align_safe_end:
              \@@_grab_v_bgroup:
            }
            {
              \peek_N_type:TF
                { \@@_grab_v_aux_test:N }
                { \@@_grab_v_aux_abort:n { } }
            }
        }
  }
\cs_new_protected:Npn \@@_grab_v_group_end:
  {
        \exp_args:NNNo
      \group_end:
    \tl_set:Nn \l_@@_v_arg_tl { \l_@@_v_arg_tl }
  }
%    \end{macrocode}
% \end{macro}
% \end{macro}
% \end{macro}
% \end{macro}
%
% \begin{macro}{\@@_grab_v_aux_test:N}
% \begin{macro}
%   {
%     \@@_grab_v_aux_loop:N,
%     \@@_grab_v_aux_loop:NN,
%     \@@_grab_v_aux_loop_end:
%   }
%   Check that the opening delimiter is a character, setup category codes,
%   then start reading tokens one by one, keeping the delimiter as an argument.
%   If the verbatim was not nested, we will be grabbing one character
%   at each step. Unfortunately, it can happen that what follows the
%   verbatim argument is already tokenized. Thus, we check at each step
%   that the next token is indeed a \enquote{nice}
%   character, \emph{i.e.}, is not a character with
%   category code $1$ (begin-group), $2$ (end-group)
%   or $6$ (macro parameter), nor the space character,
%   with category code~$10$ and character code~$32$,
%   nor a control sequence.
%   The partially built argument is stored in \cs{l_@@_v_arg_tl}.
%   If we ever meet a token which we cannot grab (non-N-type),
%   or which is not a character according to
%   \cs{@@_grab_v_token_if_char:NTF}, then we bail out with
%   \cs{@@_grab_v_aux_abort:n}. Otherwise, we stop at the first
%   character matching the delimiter.
%    \begin{macrocode}
\cs_new_protected:Npn \@@_grab_v_aux_test:N #1
  {
    \@@_grab_v_token_if_char:NTF #1
      {
        \@@_grab_v_aux_put:N #1
        \@@_grab_v_aux_catcodes:
        \@@_grab_v_aux_loop:N #1
      }
      { \@@_grab_v_aux_abort:n {#1} #1 }
  }
\cs_new_protected:Npn \@@_grab_v_aux_loop:N #1
  {
    \peek_N_type:TF
      { \@@_grab_v_aux_loop:NN #1 }
      { \@@_grab_v_aux_abort:n { } }
  }
\cs_new_protected:Npn \@@_grab_v_aux_loop:NN #1#2
  {
    \@@_grab_v_token_if_char:NTF #2
      {
        \token_if_eq_charcode:NNTF #1 #2
          { \@@_grab_v_aux_loop_end: }
          {
            \@@_grab_v_aux_put:N #2
            \@@_grab_v_aux_loop:N #1
          }
      }
      { \@@_grab_v_aux_abort:n {#2} #2 }
  }
\cs_new_protected:Npn \@@_grab_v_aux_loop_end:
  {
    \@@_grab_v_group_end:
    \@@_add_arg:x { \tl_tail:N \l_@@_v_arg_tl }
  }
%    \end{macrocode}
% \end{macro}
% \end{macro}
%
% \begin{variable}{\l_@@_v_nesting_int}
%    \begin{macrocode}
\int_new:N \l_@@_v_nesting_int
%    \end{macrocode}
% \end{variable}
%
% \begin{macro}{\@@_grab_v_bgroup:}
% \begin{macro}{\@@_grab_v_bgroup_loop:}
% \begin{macro}{\@@_grab_v_bgroup_loop:N}
%   If the opening delimiter is a left brace, we keep track of
%   how many left and right braces were encountered so far in
%   \cs{l_@@_v_nesting_int} (the methods used for optional
%   arguments cannot apply here), and stop as soon as it reaches~$0$.
%
%   Some care was needed when removing the opening delimiter, which
%   has already been assigned category code~$1$: using
%   \cs{peek_meaning_remove:NTF} in the \cs{@@_grab_v_aux:w}
%   function would break within alignments. Instead, we first
%   convert that token to a string, and remove the result as a
%   normal undelimited argument.
%    \begin{macrocode}
\cs_new_protected:Npx \@@_grab_v_bgroup:
  {
    \exp_not:N \@@_grab_v_aux_catcodes:
    \exp_not:n { \int_set:Nn \l_@@_v_nesting_int { 1 } }
    \exp_not:N \@@_grab_v_aux_put:N \iow_char:N \{
    \exp_not:N \@@_grab_v_bgroup_loop:
  }
\cs_new_protected:Npn \@@_grab_v_bgroup_loop:
  {
    \peek_N_type:TF
      { \@@_grab_v_bgroup_loop:N }
      { \@@_grab_v_aux_abort:n { } }
  }
\cs_new_protected:Npn \@@_grab_v_bgroup_loop:N #1
  {
    \@@_grab_v_token_if_char:NTF #1
      {
        \token_if_eq_charcode:NNTF \c_group_end_token #1
          {
            \int_decr:N \l_@@_v_nesting_int
            \int_compare:nNnTF \l_@@_v_nesting_int > 0
              {
                \@@_grab_v_aux_put:N #1
                \@@_grab_v_bgroup_loop:
              }
              { \@@_grab_v_aux_loop_end: }
          }
          {
            \token_if_eq_charcode:NNT \c_group_begin_token #1
              { \int_incr:N \l_@@_v_nesting_int }
            \@@_grab_v_aux_put:N #1
            \@@_grab_v_bgroup_loop:
          }
      }
      { \@@_grab_v_aux_abort:n {#1} #1 }
  }
%    \end{macrocode}
% \end{macro}
% \end{macro}
% \end{macro}
%
% \begin{macro}{\@@_grab_v_aux_catcodes:}
% \begin{macro}{\@@_grab_v_aux_abort:n}
%   The approach for short verbatim arguments is to make the end-line
%   character a macro parameter character: this is forbidden by the
%   rest of the code. Then the error branch can check what caused the
%   bail out and give the appropriate error message.
%    \begin{macrocode}
\cs_new_protected:Npn \@@_grab_v_aux_catcodes:
  {
    \cs_set_eq:NN \do \char_set_catcode_other:N
    \dospecials
    \tex_endlinechar:D = `\^^M \scan_stop:
    \bool_if:NTF \l_@@_long_bool
      { \char_set_catcode_other:n { \tex_endlinechar:D } }
      { \char_set_catcode_parameter:n { \tex_endlinechar:D } }
  }
\cs_new_protected:Npn \@@_grab_v_aux_abort:n #1
  {
    \@@_grab_v_group_end:
    \exp_after:wN \exp_after:wN \exp_after:wN
      \peek_meaning_remove:NTF \char_generate:nn { \tex_endlinechar:D } { 6 }
      {
        \msg_error:nnxxx { cmd } { verbatim-newline }
          { \@@_environment_or_command: }
          { \tl_to_str:N \l_@@_v_arg_tl }
          { \tl_to_str:n {#1} }
        \@@_add_arg:o \c_novalue_tl
      }
      {
        \msg_error:nnxxx { cmd } { verbatim-tokenized }
          { \@@_environment_or_command: }
          { \tl_to_str:N \l_@@_v_arg_tl }
          { \tl_to_str:n {#1} }
        \@@_add_arg:o \c_novalue_tl
      }
  }
%    \end{macrocode}
% \end{macro}
% \end{macro}
%
% \begin{macro}{\@@_grab_v_aux_put:N}
%   Storing one token in the collected argument. Most tokens are
%   converted to category code $12$, with the exception of active
%   characters, and spaces (not sure what should be done for those).
%    \begin{macrocode}
\cs_new_protected:Npn \@@_grab_v_aux_put:N #1
  {
    \tl_put_right:Nx \l_@@_v_arg_tl
      {
        \token_if_active:NTF #1
          { \exp_not:N #1 } { \token_to_str:N #1 }
      }
  }
%    \end{macrocode}
% \end{macro}
%
% \begin{macro}{\@@_grab_v_token_if_char:NTF}
%   This function assumes that the escape character is printable.
%   Then the string representation of control sequences is at least
%   two characters, and \cs{str_tail:n} only removes the escape
%   character. Macro parameter characters are doubled by
%   \cs{tl_to_str:n}, and will also yield a non-empty result,
%   hence are not considered as characters.
%    \begin{macrocode}
\cs_new_protected:Npn \@@_grab_v_token_if_char:NTF #1
  { \str_if_eq:eeTF { } { \str_tail:n {#1} } }
%    \end{macrocode}
% \end{macro}
%
% \begin{macro}{\@@_add_arg:n, \@@_add_arg:V, \@@_add_arg:o, \@@_add_arg:x}
%   When an argument is found it is stored, then further arguments are
%   grabbed by calling \cs{l_@@_signature_tl}.
%    \begin{macrocode}
\cs_new_protected:Npn \@@_add_arg:n #1
  {
    \tl_put_right:Nn \l_@@_args_tl { {#1} }
    \l_@@_signature_tl \@@_run_code:
  }
\cs_generate_variant:Nn \@@_add_arg:n { V , o , x }
%    \end{macrocode}
% \end{macro}
%
% \subsection{Grabbing arguments expandably}
%
% \begin{macro}[EXP]{\@@_expandable_grab_D:w}
% \begin{macro}[EXP]{\@@_expandable_grab_D:NNNwNNn}
% \begin{macro}[EXP]{\@@_expandable_grab_D:NNNwNNnnn}
% \begin{macro}[EXP]{\@@_expandable_grab_D:Nw}
% \begin{macro}[EXP]{\@@_expandable_grab_D:nnNNNwNN}
%   The first step is to grab the first token or group. The generic grabbers
%   \cs{\meta{function}}\verb*| | and \cs{\meta{function}}\verb*| | are just after \cs{q_@@}, we go and find
%   them (and use the long one).
%    \begin{macrocode}
\cs_new:Npn \@@_expandable_grab_D:w #1 \q_@@ #2#3
  { #2 { \@@_expandable_grab_D:NNNwNNn #1 \q_@@ #2 #3 } }
%    \end{macrocode}
%   We then wish to test whether |#7|, which we just grabbed, is exactly |#2|.
%   A preliminary test is whether their string representations coincide, then
%   expand the only grabber function we have, |#1|, once: the two strings below
%   are equal if and only if |#7| matches |#2| exactly.\footnote{It is obvious
%   that if \texttt{\#7} matches \texttt{\#2} then the strings are equal. We
%   must check the converse. The right-hand-side of \cs{str_if_eq:onTF} does
%   not end with \texttt{\#3}, implying that the grabber function took
%   everything as its arguments. The first brace group can only be empty if
%   \texttt{\#7} starts with \texttt{\#2}, otherwise the brace group preceding
%   \texttt{\#7} would not vanish. The third brace group is empty, thus the
%   \cs{q_@@} that was used by our grabber \texttt{\#1} must be the one
%   that we inserted (not some token in \texttt{\#7}), hence the second brace
%   group contains the end of \texttt{\#7} followed by \texttt{\#2}. Since this
%   is \texttt{\#2} on the right-hand-side, and no brace can be lost there,
%   \texttt{\#7} must contain nothing else than its leading \texttt{\#2}.}
%   The preliminary test is needed as |#7| could validly contain
%   \tn{par} (because a later mandatory argument could be long) and our
%   grabber may be short.  If
%   |#7| does not match |#2|, then the optional argument is missing, we use the
%   default |-NoValue-|, and put back the argument |#7| in the input stream.
%
%   If it does match, then interesting things need to be done. We will grab the
%   argument piece by piece, with the following pattern:
%   \begin{quote}
%     \meta{grabber} \Arg{tokens} \\
%     ~~\cs{q_nil} \Arg{piece 1} \meta{piece 2} \cs{ERROR} \cs{q_@@}\\
%     ~~\cs{q_nil} \meta{input stream}
%   \end{quote}
%   The \meta{grabber} will find an opening delimiter in \meta{piece 2}, take
%   the \cs{q_@@} as a second delimiter, and find more material delimited
%   by the closing delimiter in the \meta{input stream}. We then move the part
%   before the opening delimiter from \meta{piece 2} to \meta{piece 1}, and the
%   material taken from the \meta{input stream} to the \meta{piece 2}. Thus,
%   the argument moves gradually from the \meta{input stream} to the
%   \meta{piece 2}, then to the \meta{piece 1} when we have made sure to find
%   all opening and closing delimiters. This two-step process ensures that
%   nesting works: the number of opening delimiters minus closing delimiters in
%   \meta{piece 1} is always equal to the number of closing delimiters in
%   \meta{piece 2}. We stop grabbing arguments once the \meta{piece 2} contains
%   no opening delimiter any more, hence the balance is reached, and the final
%   argument is \meta{piece 1} \meta{piece 2}.
%   The indirection via \cs{@@_tmp:w} allows to insert |-NoValue-| expanded.
%    \begin{macrocode}
\cs_set_protected:Npn \@@_tmp:w #1
  {
    \cs_new:Npn \@@_expandable_grab_D:NNNwNNn ##1##2##3##4 \q_@@ ##5##6##7
      {
        \str_if_eq:nnTF {##2} {##7}
          {
            \str_if_eq:onTF
              { ##1 { } { } ##7 ##2 \q_@@ ##3 }
              { { } {##2} { } }
          }
          { \use_ii:nn }
          {
            ##1
              { \@@_expandable_grab_D:NNNwNNnnn ##1##2##3##4 \q_@@ ##5##6 }
              \q_nil { } ##2 \ERROR \q_@@ \ERROR
          }
          { ##4 {#1} \q_@@ ##5 ##6 {##7} }
      }
  }
\exp_args:No \@@_tmp:w { \c_novalue_tl }
%    \end{macrocode}
%   At this stage, |#7| is \cs{q_nil} \Arg{piece 1} \meta{more for piece 1},
%   and we want to concatenate all that, removing \cs{q_nil}, and keeping the
%   opening delimiter |#2|. Simply use \cs{use_ii:nn}. Also, |#8| is
%   \meta{remainder of piece 2} \cs{ERROR}, and |#9| is \cs{ERROR} \meta{more
%   for piece 2}. We concatenate those, replacing the two \cs{ERROR} by the
%   closing delimiter |#3|.
%    \begin{macrocode}
\cs_new:Npn \@@_expandable_grab_D:NNNwNNnnn #1#2#3#4 \q_@@ #5#6#7#8#9
  {
    \exp_args:Nof \@@_expandable_grab_D:nnNNNwNN
      { \use_ii:nn #7 #2 }
      { \@@_expandable_grab_D:Nw #3 \exp_stop_f: #8 #9 }
    #1#2#3 #4 \q_@@ #5 #6
  }
\cs_new:Npn \@@_expandable_grab_D:Nw #1#2 \ERROR \ERROR { #2 #1 }
%    \end{macrocode}
%   Armed with our two new \meta{pieces}, we are ready to loop. However, we
%   must first see if \meta{piece 2} (here |#2|) contains any opening
%   delimiter |#4|. Again, we expand |#3|, this time removing its whole output
%   with \cs{use_none:nnn}. The test is similar to \cs{tl_if_in:nnTF}. The
%   token list is empty if and only if |#2| does not contain the opening
%   delimiter. In that case, we are done, and put the argument (from which we
%   remove a spurious pair of delimiters coming from how we started the loop).
%   Otherwise, we go back to looping with
%   \cs{@@_expandable_grab_D:NNNwNNnnn}. The code to deal with brace stripping
%   is much the same as for the non-expandable case.
%    \begin{macrocode}
\cs_new:Npn \@@_expandable_grab_D:nnNNNwNN #1#2#3#4#5#6 \q_@@ #7#8
  {
    \exp_args:No \tl_if_empty:oTF
      { #3 { \use_none:nnn } #2 \q_@@ #5 #4 \q_@@ #5 }
      {
        \tl_if_blank:oTF { \use_none:nn #1#2 }
          { \@@_put_arg_expandable:ow { \use_none:nn #1#2 } }
          {
            \str_if_eq:eeTF
              { \exp_not:o { \use_none:nn #1#2 } }
              { { \exp_not:o { \use_iii:nnnn #1#2 \q_nil } } }
              { \@@_put_arg_expandable:ow { \use_iii:nnn #1#2 } }
              { \@@_put_arg_expandable:ow { \use_none:nn #1#2 } }
          }
            #6 \q_@@ #7 #8
      }
      {
        #3
          { \@@_expandable_grab_D:NNNwNNnnn #3#4#5#6 \q_@@ #7 #8 }
          \q_nil {#1} #2 \ERROR \q_@@ \ERROR
      }
  }
%    \end{macrocode}
% \end{macro}
% \end{macro}
% \end{macro}
% \end{macro}
% \end{macro}
%
% \begin{macro}[EXP]{\@@_expandable_grab_D_alt:w}
% \begin{macro}[EXP]{\@@_expandable_grab_D_alt:NNwNNn}
% \begin{macro}[EXP]{\@@_expandable_grab_D_alt:Nwn}
%   When the delimiters are identical, nesting is not possible and a simplified
%   approach is used. The test concept here is the same as for the case where
%   the delimiters are different but there cannot be any nesting.
%    \begin{macrocode}
\cs_new:Npn \@@_expandable_grab_D_alt:w #1 \q_@@ #2#3
  { #2 { \@@_expandable_grab_D_alt:NNwNNn #1 \q_@@ #2 #3 } }
\cs_set_protected:Npn \@@_tmp:w #1
  {
    \cs_new:Npn \@@_expandable_grab_D_alt:NNwNNn ##1##2##3 \q_@@ ##4##5##6
      {
        \str_if_eq:nnTF {##6} {##2}
          {
            \str_if_eq:onTF
              { ##1 { } ##6 ##2 ##2 }
              { { } ##2 }
          }
          { \use_ii:nn }
          {
            ##1
              { \@@_expandable_grab_D_alt:NNwn ##4 ##5 ##3 \q_@@ }
              ##6 \ERROR
          }
          { ##3 {#1} \q_@@ ##4 ##5 {##6} }
      }
  }
\exp_args:No \@@_tmp:w { \c_novalue_tl }
\cs_new:Npn \@@_expandable_grab_D_alt:NNwn #1#2#3 \q_@@ #4
  {
    \tl_if_blank:oTF { \use_none:n #4 }
      { \@@_put_arg_expandable:ow { \use_none:n #4 } }
      {
        \str_if_eq:eeTF
          { \exp_not:o { \use_none:n #4 } }
          { { \exp_not:o { \use_ii:nnn #4 \q_nil } } }
          { \@@_put_arg_expandable:ow { \use_ii:nn #4 } }
          { \@@_put_arg_expandable:ow { \use_none:n #4 } }
      }
        #3 \q_@@ #1 #2
  }
%    \end{macrocode}
% \end{macro}
% \end{macro}
% \end{macro}
%
% \begin{macro}[EXP]{\@@_expandable_grab_E:w, \@@_expandable_grab_E_long:w}
% \begin{macro}[EXP]{\@@_expandable_grab_E_aux:w}
% \begin{macro}[EXP]{\@@_expandable_grab_E_test:nnw}
% \begin{macro}[EXP]{\@@_expandable_grab_E_loop:nnnNNw}
% \begin{macro}[EXP]{\@@_expandable_grab_E_find:w}
% \begin{macro}[EXP]{\@@_expandable_grab_E_find:nnw}
% \begin{macro}[EXP]{\@@_expandable_grab_E_end:nnw}
%   We keep track of long/short by placing the appropriate grabber as
%   the third token after \cs{q_@@}; it is eventually removed by the
%   \texttt{end:nnw} auxiliary.  The \texttt{aux:w} auxiliary will be
%   called repeatedly with two arguments: the set
%   of pairs \meta{parser} \meta{token}, and the set of arguments found
%   so far (initially all |{-NoValue-}|).  At each step, grab what
%   follows in the input stream then call the \texttt{loop:nnnNNw}
%   auxiliary to compare it with each possible embellishment in turn.
%   This auxiliary's |#1| is what was found in the input, |#2| collects
%   \meta{parser} \meta{token} pairs that did not match, |#3| collects
%   the corresponding arguments found previously, |#4| and |#5| is the
%   current pair, |#6| is the remaining pairs, |#7| is empty or two
%   \cs{q_nil}, and |#8| is the current argument.  If none of the pairs
%   matched (determined by \cs{quark_if_nil:NTF}) then call the
%   \texttt{end} auxiliary to stop looking for embellishments,
%   remembering to put what was grabbed in the input back where it
%   belongs, and storing the arguments found just before \cs{q_@@}.  If
%   the current argument |#8| is not |-NoValue-| or if the input |#1|
%   does not match |#5| (see \texttt{t}-type arguments below for a
%   similar \cs{str_if_eq:onTF} test) then carry on the loop.
%   Otherwise, we found a new embellishment: grab the corresponding
%   argument in the input using the \texttt{find:w} auxiliary.  To avoid
%   losing braces around that auxiliary's argument we include a
%   space, which will be eliminated in the next loop through
%   embellishments.
%    \begin{macrocode}
\cs_new:Npn \@@_expandable_grab_E:w #1 \q_@@ #2#3
  { \@@_expandable_grab_E_aux:w #1 \q_@@ #2 #3 #3 }
\cs_new:Npn \@@_expandable_grab_E_long:w #1 \q_@@ #2#3
  { \@@_expandable_grab_E_aux:w #1 \q_@@ #2 #3 #2 }
\cs_new:Npn \@@_expandable_grab_E_aux:w #1 \q_@@ #2#3#4
  { #2 { \@@_expandable_grab_E_test:nnw #1 \q_@@ #2 #3 #4 } }
\cs_new:Npn \@@_expandable_grab_E_test:nnw #1#2#3 \q_@@ #4#5#6#7
  {
    \@@_expandable_grab_E_loop:nnnNNw {#7} { } { }
      #1 \q_nil \q_nil \q_nil \q_mark #2 \q_nil
    #3 \q_@@ #4 #5 #6
  }
\cs_new:Npn \@@_expandable_grab_E_loop:nnnNNw
    #1#2#3#4#5#6 \q_nil #7 \q_mark #8
  {
    \quark_if_nil:NTF #4
      { \@@_expandable_grab_E_end:nnw {#1} {#3} }
      {
        \tl_if_novalue:nTF {#8}
          { \str_if_eq:onTF { #4 { } #1 #5 } {#5} }
          { \use_ii:nn }
            { \@@_expandable_grab_E_find:w { #2 #4 #5 #6 } {#3} ~ }
            {
              \@@_expandable_grab_E_loop:nnnNNw
                {#1} { #2 #4 #5 } { #3 {#8} }
                #6 \q_nil #7 \q_mark
            }
      }
  }
\cs_new:Npn \@@_expandable_grab_E_find:w #1 \q_@@ #2#3#4
  { #4 { \@@_expandable_grab_E_find:nnw #1 \q_@@ #2 #3 #4 } }
\cs_new:Npn \@@_expandable_grab_E_find:nnw #1#2#3 \q_nil #4 \q_@@ #5#6#7#8
  { \@@_expandable_grab_E_aux:w {#1} { #2 {#8} #3 } #4 \q_@@ #5 #6 #7 }
\cs_new:Npn \@@_expandable_grab_E_end:nnw #1#2#3 \q_@@ #4#5#6
  { #3 #2 \q_@@ #4 #5 {#1} }
%    \end{macrocode}
% \end{macro}
% \end{macro}
% \end{macro}
% \end{macro}
% \end{macro}
% \end{macro}
% \end{macro}
%
% \begin{macro}[EXP]{\@@_expandable_grab_m:w, \@@_expandable_grab_m_long:w}
% \begin{macro}[EXP]{\@@_expandable_grab_m_aux:wNn}
%   The mandatory case is easy: find the auxiliary after the \cs{q_@@},
%   and use it directly to grab the argument, then correctly position
%   the argument before \cs{q_@@}.
%    \begin{macrocode}
\cs_new:Npn \@@_expandable_grab_m:w #1 \q_@@ #2#3
  { #3 { \@@_expandable_grab_m_aux:wNn #1 \q_@@ #2 #3 } }
\cs_new:Npn \@@_expandable_grab_m_long:w #1 \q_@@ #2#3
  { #2 { \@@_expandable_grab_m_aux:wNn #1 \q_@@ #2 #3 } }
\cs_new:Npn \@@_expandable_grab_m_aux:wNn #1 \q_@@ #2#3#4
  { #1 {#4} \q_@@ #2 #3 }
%    \end{macrocode}
% \end{macro}
% \end{macro}
%
% \begin{macro}[EXP]{\@@_expandable_grab_R:w}
% \begin{macro}[EXP]{\@@_expandable_grab_R_aux:NNNwNNn}
%   Much the same as for the \texttt{D}-type argument, with only the lead-off
%   function varying.
%    \begin{macrocode}
\cs_new:Npn \@@_expandable_grab_R:w #1 \q_@@ #2#3
  { #2 { \@@_expandable_grab_R_aux:NNNwNNn #1 \q_@@ #2#3 } }
\cs_set_protected:Npn \@@_tmp:w #1
  {
    \cs_new:Npn \@@_expandable_grab_R_aux:NNNwNNn ##1##2##3##4 \q_@@ ##5##6##7
      {
        \str_if_eq:nnTF {##7} {##2}
          {
            \str_if_eq:onTF
              { ##1 { } { } ##7 ##2 \q_@@ ##3 }
              { { } {##2} { } }
          }
          { \use_ii:nn }
          {
            ##1
              { \@@_expandable_grab_D:NNNwNNnnn ##1##2##3##4 \q_@@ ##5##6 }
              \q_nil { } ##2 \ERROR \q_@@ \ERROR
          }
          {
            \msg_expandable_error:nnff { cmd } { missing-required }
              { \exp_args:Nf \tl_trim_spaces:n { \token_to_str:N ##5 } }
              { \tl_to_str:n {##2} }
            ##4 {#1} \q_@@ ##5 ##6 {##7}
          }
      }
  }
\exp_args:No \@@_tmp:w { \c_novalue_tl }
%    \end{macrocode}
% \end{macro}
% \end{macro}
%
% \begin{macro}[EXP]{\@@_expandable_grab_R_alt:w}
% \begin{macro}[EXP]{\@@_expandable_grab_R_alt_aux:NNwNNn}
%   When the delimiters are identical, nesting is not possible and a simplified
%   approach is used. The test concept here is the same as for the case where
%   the delimiters are different.
%    \begin{macrocode}
\cs_new:Npn \@@_expandable_grab_R_alt:w #1 \q_@@ #2#3
  { #2 { \@@_expandable_grab_R_alt_aux:NNwNNn #1 \q_@@ #2#3 } }
\cs_set_protected:Npn \@@_tmp:w #1
  {
    \cs_new:Npn \@@_expandable_grab_R_alt_aux:NNwNNn ##1##2##3 \q_@@ ##4##5##6
      {
        \str_if_eq:nnTF {##6} {##2}
          {
            \str_if_eq:onTF
              { ##1 { } ##6 ##2 ##2 }
              { { } ##2 }
          }
          { \use_ii:nn }
          {
            ##1
              { \@@_expandable_grab_D_alt:NNwn ##4 ##5 ##3 \q_@@ }
              ##6 \ERROR
          }
          {
            \msg_expandable_error:nnff { cmd } { missing-required }
              { \exp_args:Nf \tl_trim_spaces:n { \token_to_str:N ##4 } }
              { \tl_to_str:n {##2} }
            ##3 {#1} \q_@@ ##4 ##5 {##6}
          }
      }
  }
\exp_args:No \@@_tmp:w { \c_novalue_tl }
%    \end{macrocode}
% \end{macro}
% \end{macro}
%
% \begin{macro}[EXP]{\@@_expandable_grab_t:w}
% \begin{macro}[EXP]{\@@_expandable_grab_t_aux:NNwn}
%   As for a \texttt{D}-type argument, here we compare the grabbed tokens using
%   the only parser we have in order to work out if |#2| is exactly equal to
%   the output of the grabber.
%    \begin{macrocode}
\cs_new:Npn \@@_expandable_grab_t:w #1 \q_@@ #2#3
  { #2 { \@@_expandable_grab_t_aux:NNwn #1 \q_@@ #2 #3 } }
\cs_new:Npn \@@_expandable_grab_t_aux:NNwn #1#2#3 \q_@@ #4#5#6
  {
    \str_if_eq:onTF { #1 { } #6 #2 } {#2}
      { #3 { \BooleanTrue } \q_@@ #4 #5 }
      { #3 { \BooleanFalse } \q_@@ #4 #5 {#6} }
  }
%    \end{macrocode}
% \end{macro}
% \end{macro}
%
% \begin{macro}[EXP]
%   {\@@_put_arg_expandable:nw, \@@_put_arg_expandable:ow}
%   A useful helper, to store arguments when they are ready.
%    \begin{macrocode}
\cs_new:Npn \@@_put_arg_expandable:nw #1#2 \q_@@ { #2 {#1} \q_@@ }
\cs_generate_variant:Nn \@@_put_arg_expandable:nw { o }
%    \end{macrocode}
% \end{macro}
%
% \subsection{Argument processors}
%
% \begin{macro}{\@@_bool_reverse:N}
%   A simple reversal.
%    \begin{macrocode}
\cs_new_protected:Npn \@@_bool_reverse:N #1
  {
    \bool_if:NTF #1
      { \tl_set:Nn \ProcessedArgument { \c_false_bool } }
      { \tl_set:Nn \ProcessedArgument { \c_true_bool } }
  }
%    \end{macrocode}
% \end{macro}
%
% \begin{variable}{\l_@@_split_list_seq, \l_@@_split_list_tl}
% \begin{macro}{\@@_split_list:nn}
% \begin{macro}{\@@_split_list_multi:nn, \@@_split_list_multi:nV}
% \begin{macro}{\@@_split_list_single:Nn}
%   Splitting can take place either at a single token or at a longer
%   identifier. To deal with single active tokens, a two-part procedure is
%   needed.
%    \begin{macrocode}
\seq_new:N \l_@@_split_list_seq
\tl_new:N \l_@@_split_list_tl
\cs_new_protected:Npn \@@_split_list:nn #1#2
  {
    \tl_if_single:nTF {#1}
      {
        \token_if_cs:NTF #1
          { \@@_split_list_multi:nn {#1} {#2} }
          { \@@_split_list_single:Nn #1 {#2} }
      }
      { \@@_split_list_multi:nn {#1} {#2} }
  }
\cs_new_protected:Npn \@@_split_list_multi:nn #1#2
  {
    \seq_set_split:Nnn \l_@@_split_list_seq {#1} {#2}
    \tl_clear:N \ProcessedArgument
    \seq_map_inline:Nn \l_@@_split_list_seq
      { \tl_put_right:Nn \ProcessedArgument { {##1} } }
  }
\cs_generate_variant:Nn \@@_split_list_multi:nn { nV }
\group_begin:
\char_set_catcode_active:N \^^@
\cs_new_protected:Npn \@@_split_list_single:Nn #1#2
  {
    \tl_set:Nn \l_@@_split_list_tl {#2}
    \group_begin:
    \char_set_lccode:nn { `\^^@ } { `#1 }
    \tex_lowercase:D
      {
        \group_end:
        \tl_replace_all:Nnn \l_@@_split_list_tl { ^^@ }
      }   {#1}
     \@@_split_list_multi:nV {#1} \l_@@_split_list_tl
   }
\group_end:
%    \end{macrocode}
% \end{macro}
% \end{macro}
% \end{macro}
% \end{variable}
%
% \begin{macro}{\@@_split_argument:nnn}
% \begin{macro}{\@@_split_argument_aux:nnnn}
% \begin{macro}[EXP]{\@@_split_argument_aux:n}
% \begin{macro}[rEXP]{\@@_split_argument_aux:wn}
%   Splitting to a known number of items is a special version of splitting
%   a list, in which the limit is hard-coded and where there will always be
%   exactly the correct number of output items. An auxiliary function is
%   used to save on working out the token list length several times.
%    \begin{macrocode}
\cs_new_protected:Npn \@@_split_argument:nnn #1#2#3
  {
    \@@_split_list:nn {#2} {#3}
    \exp_args:Nf \@@_split_argument_aux:nnnn
      { \tl_count:N \ProcessedArgument }
      {#1} {#2} {#3}
  }
\cs_new_protected:Npn \@@_split_argument_aux:nnnn #1#2#3#4
  {
    \int_compare:nNnF {#1} = { #2 + 1 }
      {
        \int_compare:nNnTF {#1} > { #2 + 1 }
          {
            \tl_set:Nx \ProcessedArgument
              {
                \exp_last_unbraced:NnNo
                  \@@_split_argument_aux:n
                  { #2 + 1 }
                  \use_none_delimit_by_q_stop:w
                  \ProcessedArgument
                  \q_stop
              }
            \msg_error:nnxxx { cmd } { split-excess-tokens }
              { \tl_to_str:n {#3} } { \int_eval:n { #2 + 1 } }
              { \tl_to_str:n {#4} }
          }
          {
            \tl_put_right:Nx \ProcessedArgument
              {
                \prg_replicate:nn { #2 + 1 - (#1) }
                  { { \exp_not:V \c_novalue_tl } }
              }
          }
      }
  }
%    \end{macrocode}
%   Auxiliaries to leave exactly the correct number of arguments in
%   \cs{ProcessedArgument}.
%    \begin{macrocode}
\cs_new:Npn \@@_split_argument_aux:n #1
  { \prg_replicate:nn {#1} { \@@_split_argument_aux:wn } }
\cs_new:Npn \@@_split_argument_aux:wn #1 \use_none_delimit_by_q_stop:w #2
  {
    \exp_not:n { {#2} }
    #1
    \use_none_delimit_by_q_stop:w
  }
%    \end{macrocode}
% \end{macro}
% \end{macro}
% \end{macro}
% \end{macro}
%
% \begin{macro}{\@@_trim_spaces:n}
%   This one is almost trivial.
%    \begin{macrocode}
\cs_new_protected:Npn \@@_trim_spaces:n #1
  { \tl_set:Nx \ProcessedArgument { \tl_trim_spaces:n {#1} } }
%    \end{macrocode}
% \end{macro}
%
% \subsection{Access to the argument specification}
%
% \begin{macro}{\@@_get_arg_spec_error:N, \@@_get_arg_spec_error:n}
% \begin{macro}{\@@_get_arg_spec_error_aux:n}
%   Provide an informative error when trying to get the argument
%   specification of a non-\pkg{xparse} command or environment.
%    \begin{macrocode}
\cs_new_protected:Npn \@@_get_arg_spec_error:N #1
  {
    \bool_set_false:N \l_@@_environment_bool
    \tl_set:Nn \l_@@_fn_tl {#1}
    \@@_get_arg_spec_error_aux:n { \cs_if_exist:NTF #1 }
  }
\cs_new_protected:Npn \@@_get_arg_spec_error:n #1
  {
    \bool_set_true:N \l_@@_environment_bool
    \str_set:Nx \l_@@_environment_str {#1}
    \@@_get_arg_spec_error_aux:n
      { \cs_if_exist:cTF { \l_@@_environment_str } }
  }
\cs_new_protected:Npn \@@_get_arg_spec_error_aux:n #1
  {
    #1
      {
        \msg_error:nnx { cmd } { non-xparse }
          { \@@_environment_or_command: }
      }
      {
        \msg_error:nnx { cmd } { unknown }
          { \@@_environment_or_command: }
      }
  }
%    \end{macrocode}
% \end{macro}
% \end{macro}
%
% \begin{macro}{\@@_get_arg_spec:NTF}
%   If the command is not an \pkg{xparse} command, complain.  If it is,
%   its second \enquote{item} is the argument specification.
%    \begin{macrocode}
\cs_new_protected:Npn \@@_get_arg_spec:NTF #1#2#3
  {
    \__kernel_cmd_if_xparse:NTF #1
      {
        \tl_set:Nx \ArgumentSpecification { \tl_item:Nn #1 { 2 } }
        #2
      }
      {#3}
  }
%    \end{macrocode}
% \end{macro}
%
% Rolling forward from 2020-10-01 is tricky because the entire |ltcmd|
% module is new, but the user-level commands have the same name, so only
% these will clash.  To work around that, in |latexrelease| mode we will
% (temporarily) disable \cs{__kernel_chk_if_free_cs:N} for this final
% part of the code, then restore at the end.
%    \begin{macrocode}
%<latexrelease>\cs_new_eq:NN \@@_chk_if_free_cs:N \__kernel_chk_if_free_cs:N
%<latexrelease>\cs_gset_eq:NN \__kernel_chk_if_free_cs:N \use_none:n
%    \end{macrocode}
%
% \begin{variable}{\ArgumentSpecification}
%    \begin{macrocode}
\tl_new:N \ArgumentSpecification
%    \end{macrocode}
% \end{variable}
%
% \begin{macro}{\@@_get_arg_spec:N}
% \begin{macro}{\@@_get_arg_spec:n}
%   Recovering the argument specification is now trivial.
%    \begin{macrocode}
\cs_new_protected:Npn \@@_get_arg_spec:N #1
  {
    \@@_get_arg_spec:NTF #1 { }
      { \@@_get_arg_spec_error:N #1 }
  }
\cs_new_protected:Npn \@@_get_arg_spec:n #1
  {
    \exp_args:Nc \@@_get_arg_spec:NTF
      { environment~ \tl_to_str:n {#1} }
      { }
      { \@@_get_arg_spec_error:n {#1} }
  }
%    \end{macrocode}
% \end{macro}
% \end{macro}
%
% \begin{macro}{\@@_show_arg_spec:N}
% \begin{macro}{\@@_show_arg_spec:n}
%   Showing the argument specification simply means finding it and then
%   calling the \cs{tl_show:N} function.
%    \begin{macrocode}
\cs_new_protected:Npn \@@_show_arg_spec:N #1
  {
    \@@_get_arg_spec:NTF #1
      { \tl_show:N \ArgumentSpecification }
      { \@@_get_arg_spec_error:N #1 }
  }
\cs_new_protected:Npn \@@_show_arg_spec:n #1
  {
    \exp_args:Nc \@@_get_arg_spec:NTF
      { environment~ \tl_to_str:n {#1} }
      { \tl_show:N \ArgumentSpecification }
      { \@@_get_arg_spec_error:n {#1} }
  }
%    \end{macrocode}
% \end{macro}
% \end{macro}
%
% \subsection{Utilities}
%
% \begin{macro}{\@@_check_definable:nNT, \@@_check_definable_aux:nN}
%   Check that a token list is appropriate as a first argument of
%   \cs{NewDocumentCommand} and similar functions and otherwise
%   produce an error.  First trim whitespace to allow for spaces around
%   the actual command to be defined.  If the result has multiple
%   tokens, it is not a valid argument.  The single token is a control
%   sequence exactly if its string representation has more than one
%   character (using \cs{token_to_str:N} rather than \cs{tl_to_str:n}
%   to avoid problems with macro parameter characters, and setting
%   \cs{tex_escapechar:D} to prevent it from being non-printable).
%   Finally, check for an active character: this is done by lowercasing
%   the token to fix its character code (arbitrarily to that of~|?|)
%   and comparing the result to an active~|?|.  Both control sequences
%   and active characters are valid arguments, and non-active character
%   tokens are not.  In all cases, the group opened to keep assignments
%   local must be closed.
%    \begin{macrocode}
\cs_new_protected:Npn \@@_check_definable:nNT #1
  { \tl_trim_spaces_apply:nN {#1} \@@_check_definable_aux:nN }
\group_begin:
  \char_set_catcode_active:n { `? }
  \cs_new_protected:Npn \@@_check_definable_aux:nN #1#2
    {
      \group_begin:
      \tl_if_single_token:nTF {#1}
        {
          \int_set:Nn \tex_escapechar:D { 92 }
          \exp_args:Nx \tl_if_empty:nTF
            { \exp_args:No \str_tail:n { \token_to_str:N #1 } }
            {
              \exp_args:Nx \char_set_lccode:nn
                { ` \str_head:n {#1} } { `? }
              \tex_lowercase:D { \tl_if_eq:nnTF {#1} } { ? }
                { \group_end: \use_iii:nnn }
                { \group_end: \use_i:nnn }
            }
            { \group_end: \use_iii:nnn }
        }
        { \group_end: \use_ii:nnn }
      {
        \msg_error:nnxx { cmd } { not-definable }
          { \tl_to_str:n {#1} } { \token_to_str:N #2 }
      }
      {
        \msg_error:nnxx { cmd } { not-one-token }
          { \tl_to_str:n {#1} } { \token_to_str:N #2 }
      }
    }
\group_end:
%    \end{macrocode}
% \end{macro}
%
% \begin{macro}{\@@_token_if_cs:NTF}
%   Based on the definition of \cs{@@_check_definable_aux:nN} above, but
%   only checks for an actual control sequence (\emph{i.e.},
%   \cs[no-index]{\meta{anything}}). \cs{tex_escapechar:D} is
%   temporarily changed to a known value and then it checks if
%   |\string#1| contains more than one character: if it does, it's a
%   control sequence.  This test differs from \cs{token_if_cs:NTF} for
%   example in \verb|\token_if_cs:NTF \c_group_begin_token {T}{F}|,
%   where \cs{token_if_cs:NTF} returns false.
%    \begin{macrocode}
\cs_new_protected:Npn \@@_token_if_cs:NTF #1
  {
    \group_begin:
      \int_set:Nn \tex_escapechar:D { 92 }
      \exp_args:Nx \tl_if_empty:nTF
          { \exp_args:No \str_tail:n { \token_to_str:N #1 } }
        { \group_end: \use_ii:nn }
        { \group_end: \use_i:nn }
  }
%    \end{macrocode}
% \end{macro}
%
% \begin{macro}{\@@_tl_mapthread_function:NNN, \@@_tl_mapthread_function:nnN}
% \begin{macro}{\@@_tl_mapthread_loop:w}
%   Analogue of \cs{seq_mapthread_function:NNN} for token lists.
%    \begin{macrocode}
\cs_new:Npn \@@_tl_mapthread_function:NNN #1#2#3
  {
    \exp_after:wN \exp_after:wN
    \exp_after:wN \@@_tl_mapthread_loop:w
    \exp_after:wN \exp_after:wN
    \exp_after:wN #3
    \exp_after:wN #1
    \exp_after:wN \q_recursion_tail
    \exp_after:wN \q_mark
    #2
    \q_recursion_tail
    \q_recursion_stop
  }
\cs_new:Npn \@@_tl_mapthread_function:nnN #1#2#3
  {
    \@@_tl_mapthread_loop:w #3
      #1 \q_recursion_tail \q_mark
      #2 \q_recursion_tail \q_recursion_stop
  }
\cs_new:Npn \@@_tl_mapthread_loop:w #1#2#3 \q_mark #4
  {
    \quark_if_recursion_tail_stop:n {#2}
    \quark_if_recursion_tail_stop:n {#4}
    #1 {#2} {#4}
    \@@_tl_mapthread_loop:w #1#3 \q_mark
  }
%    \end{macrocode}
% \end{macro}
% \end{macro}
%
% \begin{macro}{\__kernel_cmd_if_xparse:NTF}
% \changes{v1.0d}{2021/04/19}{Renamed \cs{__cmd_cmd_if_xparse:NTF} to
%      \cs{__kernel_cmd_if_xparse:NTF} for cross-module usage}
% \begin{macro}{\@@_cmd_if_xparse_aux:N}
%
%    To determine whether the command is an \pkg{xparse} command check
%    that its |arg_spec| is empty (this also excludes non-macros) and
%    that its |replacement_spec| starts with either
%    \cs{@@_start:nNNnnn} (non-expandable command) or
%    \cs{@@_start_expandable:nNNNNn} (expandable command) or
%    \cs{@@_start_env:nnnnn} (environment).
%
%    This conditional is needed in several kernel modules and is
%    therefore has a kernel-internal name.
%    \begin{macrocode}
\cs_new_protected:Npn \__kernel_cmd_if_xparse:NTF #1
  {
    \exp_args:Nf \str_case_e:nnTF
      {
        \exp_args:Nf \tl_if_empty:nT { \cs_argument_spec:N #1 }
          {
            \exp_last_unbraced:Nf \@@_cmd_if_xparse_aux:w
              { \cs_replacement_spec:N #1 } ~ \q_stop
          }
      }
      {
        { \token_to_str:N \@@_start:nNNnnn } { }
        { \token_to_str:N \@@_start_expandable:nNNNNn } { }
        { \token_to_str:N \@@_start_env:nnnnn } { }
      }
  }
\cs_new:Npn \@@_cmd_if_xparse_aux:w #1 ~ #2 \q_stop {#1}
%    \end{macrocode}
% \end{macro}
% \end{macro}
%
% \begin{macro}{\@@_peek_nonspace:NTF, \@@_peek_nonspace_remove:NTF, \@@_peek_nonspace_aux:nNNTF}
%   Collect spaces in a loop, and put the collected spaces back in the
%   false branch of a call to \cs{peek_meaning:NTF} or
%   \cs{peek_meaning_remove:NTF}.
%    \begin{macrocode}
\cs_new_protected:Npn \@@_peek_nonspace:NTF
  { \@@_peek_nonspace_aux:nNNTF { } \@@_peek_meaning:NTF }
\cs_new_protected:Npn \@@_peek_nonspace_remove:NTF
  { \@@_peek_nonspace_aux:nNNTF { } \@@_peek_meaning_remove:NTF }
\cs_new_protected:Npn \@@_peek_nonspace_aux:nNNTF #1#2#3#4#5
  {
    \peek_meaning_remove:NTF \c_space_token
      { \@@_peek_nonspace_aux:nNNTF { #1 ~ } #2 #3 {#4} {#5} }
      { #2 #3 { #4 } { #5 #1 } }
  }
%    \end{macrocode}
% \end{macro}
%
% \begin{macro}{\@@_peek_meaning:NTF, \@@_peek_meaning_remove:NTF}
% \begin{macro}{\@@_peek_cs_check_equal:NNN, \@@_peek_meaning_aux:NNTF, \@@_peek_true_remove:NNw}
%   Peek ahead for a token with a given meaning.  In case the search
%   token is a control sequence, also check that the \meta{csname} is
%   the same as the control sequence peeked at.  This extra verification
%   is necessary when the command is delimited by control sequence tokens
%   (as opposed to character tokens), and we want the the exact same
%   control sequence to match.
%    \begin{macrocode}
\cs_new_protected:Npn \@@_peek_meaning:NTF
  { \@@_peek_meaning_aux:NNTF \c_false_bool }
\cs_new_protected:Npn \@@_peek_meaning_remove:NTF
  { \@@_peek_meaning_aux:NNTF \c_true_bool }
\cs_new_protected:Npn \@@_peek_meaning_aux:NNTF #1#2#3#4
  {
    \tl_set:Nn \l_@@_tmpa_tl {#3}
    \tl_set:Nn \l_@@_tmpb_tl {#4}
    \peek_meaning:NTF #2
      {
        \token_if_eq_meaning:NNTF #2 \c_group_begin_token
          { \@@_peek_true_remove:Nw #1 }
          {
            \@@_token_if_cs:NTF #2
              { \@@_peek_cs_check_equal:NNN #1 #2 }
              { \@@_peek_true_remove:Nw #1 }
          }
      }
      { \l_@@_tmpb_tl }
  }
\cs_new_protected:Npn \@@_peek_cs_check_equal:NNN #1#2#3
  {
    \str_if_eq:nnTF {#2} {#3}
      { \@@_peek_true_remove:Nw #1 }
      { \l_@@_tmpb_tl }
    #3
  }
\cs_new_protected:Npn \@@_peek_true_remove:Nw #1
  {
    \bool_if:NTF #1
      {
        \tex_afterassignment:D \l_@@_tmpa_tl
        \cs_set_eq:NN \@@_tmp:w
      }
      { \l_@@_tmpa_tl }
  }
%    \end{macrocode}
% \end{macro}
% \end{macro}
%
% \subsection{Messages}
%
% \begin{variable}{\c_@@_ignore_def_tl}
%    \begin{macrocode}
\tl_const:Nn \c_@@_ignore_def_tl
  { \\ \\ LaTeX~will~ignore~this~entire~definition. }
%    \end{macrocode}
% \end{variable}
%
% \begin{macro}{\@@_environment_or_command:}
%   Two texts used in several messages.
%    \begin{macrocode}
\cs_new:Npn \@@_environment_or_command:
  {
    \bool_if:NTF \l_@@_environment_bool
      { environment ~ ' \l_@@_environment_str ' }
      {
        command ~ '
        \exp_args:Nf \tl_trim_spaces:n
          { \exp_after:wN \token_to_str:N \l_@@_fn_tl }
        '
      }
  }
%    \end{macrocode}
% \end{macro}
%
% Some messages intended as errors when defining commands/environments.
%    \begin{macrocode}
\msg_new:nnnn { cmd } { arg-after-body }
  { Invalid~argument~type~'#2'~in~#1. }
  {
    The~'b'~argument~type~must~come~last~but~it~is~followed~
    by~'#2'~in~the~argument~specification.~This~is~not~allowed.
    \c_@@_ignore_def_tl
  }
\msg_new:nnnn { cmd } { bad-arg-spec }
  { Bad~argument~specification~'#2'~for~#1. }
  {
    The~argument~specification~provided~is~not~valid:~
    one~or~more~mandatory~parts~are~missing.
    \c_@@_ignore_def_tl
  }
\msg_new:nnnn { cmd } { command-already-defined }
  { Command~'#1'~already~defined! }
  {
    You~have~used~#2~
    with~a~command~that~already~has~a~definition. \\ \\
    The~existing~definition~of~'#1'~will~not~be~altered.
  }
\msg_new:nnnn { cmd } { command-undefined }
  { Command ~'#1'~undefined! }
  {
    You~have~used~#2~
    with~a~command~that~was~never~defined.
    \c_@@_ignore_def_tl
  }
\msg_new:nnnn { cmd } { environment-already-defined }
  { Environment~'#1'~already~defined! }
  {
    You~have~used~\NewDocumentEnvironment
    with~an~environment~that~already~has~a~definition. \\ \\
    The~existing~definition~of~'#1'~will~not~be~altered.
  }
\msg_new:nnnn { cmd } { environment-undefined }
  { Environment~'#1'~undefined! }
  {
    You~have~used~\RenewDocumentEnvironment
    with~an~environment~that~was~never~defined.
    \c_@@_ignore_def_tl
  }
\msg_new:nnnn { cmd } { expandable-ending-optional }
  { Bad~argument~specification~'#2'~for~#1. }
  {
    Expandable~commands~must~have~a~final~mandatory~argument~
    (or~no~arguments~at~all).~You~cannot~have~a~terminal~optional~
    argument~with~expandable~commands.
  }
\msg_new:nnnn { cmd } { inconsistent-long }
%  { Inconsistent~long~arguments~for~expandable~command~'#1'. }
  { Invalid~argument~prefix~'+'~in~command~'#1'. }
  {
    The~arguments~for~an~expandable~command~must~not~involve~short~
    arguments~after~long~arguments.~You~have~tried~to~mix~the~two~types~
    when~defining~'#1'.
  }
\msg_new:nnnn { cmd } { invalid-command-arg }
  { Invalid~argument~type~'#2'~in~#1. }
  {
    The~letter~'#2'~can~only~be~used~in~environment~argument~
    specifications,~but~not~for~commands.
    \\ \\
    LaTeX~will~ignore~the~entire~definition.
  }
\msg_new:nnnn { cmd } { invalid-expandable-argument-type }
  { Invalid~argument~type~'#2'~in~#1. }
  {
    The~letter~'#2'~specifies~an~argument~type~which~cannot~be~used~
    in~an~expandable~command.
    \c_@@_ignore_def_tl
  }
\msg_new:nnnn { cmd } { invalid-after-optional-expandably }
%  { Invalid~argument~type~'#2'~in~#1. }
  {
    Argument~type~'#2'~not~available~after~optional~argument~
    for~expandable~command~'#1'.
  }
  {
    The~letter~'#2'~specifies~an~argument~type~which~cannot~be~used~
    in~an~expandable~command~after~an~optional~argument.
    \c_@@_ignore_def_tl
  }
\msg_new:nnnn { cmd } { non-trailing-obey-spaces }
%  { Prefix~'!'~used~before~mandatory~argument~'#2'~of~#1. }
  { Invalid~argument~prefix~'!'~in~#1. }
  {
    The~prefix~'!'~is~only~allowed~for~trailing~optional~arguments.~
    You~tried~to~apply~it~to~'#2'.
    \c_@@_ignore_def_tl
  }
\msg_new:nnnn { cmd } { not-definable }
  { First~argument~of~'#2'~must~be~a~command. }
  {
    The~first~argument~of~'#2'~should~be~the~document~command~that~will~
    be~defined.~The~provided~argument~'#1'~is~a~character.~Perhaps~a~
    backslash~is~missing?
    \c_@@_ignore_def_tl
  }
\msg_new:nnnn { cmd } { not-one-token }
  { First~argument~of~'#2'~must~be~a~command. }
  {
    The~first~argument~of~'#2'~should~be~the~document~command~that~will~
    be~defined.~The~provided~argument~'#1'~contains~more~than~one~
    token.~Perhaps~a~backslash~is~missing?
    \c_@@_ignore_def_tl
  }
\msg_new:nnnn { cmd } { not-single-token }
  { Argument~delimiter~'#2'~invalid~in~#1. }
  {
    The~argument~specification~contains~
    \tl_if_empty:nTF{#2}{nothing}{'#2'}~
    in~a~place~
    where~a~single~token~is~required.
    \c_@@_ignore_def_tl
  }
\msg_new:nnnn { cmd } { forbidden-implicit-group-token }
  { Argument~delimiter~'#2'~invalid~in~#1. }
  {
    The~argument~specification~contains~the~implicit~
    #3-group~token~'#2'~which~is~not~allowed~as~an~argument~delimiter.
    \c_@@_ignore_def_tl
  }
\msg_new:nnnn { cmd } { processor-in-expandable }
%  { Argument~processor~'>{#2}'~cannot~be~used~for~the~expandable~command~'#1'. }
  { Invalid~argument~prefix~'>'~in~command~'#1'. }
  {
    The~argument~specification~for~'#1'~contains~the~processor~function~'>{#2}'.~
    This~is~only~supported~for~robust~commands,~but~not~for~expandable~ones.
    \c_@@_ignore_def_tl
  }
\msg_new:nnnn { cmd } { too-many-arguments }
  { Too~many~arguments~for~#1. }
  {
    The~argument~specification~'#2'~asks~for~more~than~9~arguments.~
    This~cannot~be~implemented.
    \c_@@_ignore_def_tl
  }
\msg_new:nnnn { cmd } { two-markers }
  { Invalid~argument~prefix~'#2'~in~#1. }
%  { Two~'#2'~apply~to~the~same~argument~in~argument~specification~of~#1. }
  {
    The~argument~specification~provided~for~#1~has~two~'#2'~markers~applied~
    to~the~same~argument;~one~is~redundant.
  }
\msg_new:nnnn { cmd } { unknown-argument-type }
  { Invalid~argument~type~'#2'~in~#1. }
  {
    The~letter~'#2'~does~not~specify~a~known~argument~type.
    \c_@@_ignore_def_tl
  }
\msg_new:nnnn { cmd } { xparse-argument-type }
  { Invalid~argument~type~'#2'~in~#1~(requires~xparse). }
  {
    The~letter~'#2'~specifies~a~known~but~deprecated~argument~type.~
    If~you~really~need~it~you~have~to~load~the~xparse~package.
    \c_@@_ignore_def_tl
  }
%    \end{macrocode}
%
% Errors when using commands/environments.  The \texttt{if-boolean}
% message is always used in expandable errors.  The
% \texttt{loop-in-defaults} and \texttt{missing-required} messages can
% be expandable or not expandable.
%    \begin{macrocode}
\msg_new:nnn { cmd } { if-boolean }
  { Invalid~use~\iow_char:N\\IfBooleanTF~{#1} }
\msg_new:nnnn { cmd } { loop-in-defaults }
  { Circular~dependency~in~defaults~of~#1. }
  {
    The~default~values~of~two~or~more~arguments~of~the~#1~
    depend~on~each~other~in~a~way~that~cannot~be~resolved.
  }
\msg_new:nnnn { cmd } { missing-required }
  { Missing~required~argument~for~#1. }
  {
    The~current~#1~expects~an~argument~starting~with~'#2'.~
    LaTeX~did~not~find~it,~and~will~insert~a~default~value~to~be~processed.
  }
\msg_new:nnnn { cmd } { non-xparse }
  { \str_uppercase:n #1~not~defined~using~xparse. }
  {
    You~have~asked~for~the~argument~specification~for~the~#1,~
    but~this~was~not~defined~using~xparse.
  }
\msg_new:nnnn { cmd } { split-excess-tokens }
  { Too~many~'#1'~tokens~when~trying~to~split~argument. }
  {
    LaTeX~was~asked~to~split~the~input~'#3'~
    at~each~occurrence~of~the~token~'#1',~up~to~a~maximum~of~#2~parts.~
    There~were~too~many~'#1'~tokens.
  }
\msg_new:nnnn { cmd } { unknown }
  { Unknown~document~#1. }
  {
    You~have~asked~for~the~argument~specification~for~the~#1,~
    but~it~is~not~defined.
  }
\msg_new:nnnn { cmd } { verbatim-newline }
  { Verbatim-like~#1~ended~by~end~of~line. }
  {
    The~verbatim~argument~of~the~#1~cannot~contain~more~than~one~line,~
    but~the~end~
    of~the~current~line~has~been~reached.~You~may~have~forgotten~the~
    closing~delimiter.
    \\ \\
    LaTeX~will~ignore~'#2'~and~you~may~get~some~additional~
    (low-level)~errors.
  }
\msg_new:nnnn { cmd } { verbatim-tokenized }
  { Verbatim-like~#1~illegal~in~argument. }
  {
    The~#1~takes~a~verbatim~argument~and~can~therefore~normally~
    not~appear~in~arguments~of~other~commands~or~environments.~
    LaTeX~found~an~illegal~token~ \tl_if_empty:nF {#3} { (#3)~ }
    after~'#2'~and~will~drop~everything~up~to~this~point.
    \\ \\
    Expect~further~(low-level)~errors.
  }
%    \end{macrocode}
%
% Intended more for information.
%    \begin{macrocode}
\msg_new:nnn { cmd } { define-command }
  {
    Defining~command~#1~
    with~sig.~'#2'~\msg_line_context:.
  }
\msg_new:nnn { cmd } { define-environment }
  {
    Defining~environment~'#1'~
    with~sig.~'#2'~\msg_line_context:.
  }
\msg_new:nnn { cmd } { redefine-command }
  {
    Redefining~command~#1~
    with~sig.~'#2'~\msg_line_context:.
  }
\msg_new:nnn { cmd } { redefine-environment }
  {
    Redefining~environment~'#1'~
    with~sig.~'#2'~\msg_line_context:.
  }
\msg_new:nnn { cmd } { optional-mandatory }
  {
    Since~the~mandatory~argument~'#1'~has~the~same~delimiter~'#2'~
    as~a~previous~optional~argument,~it~will~not~be~possible~to~
    omit~all~optional~arguments~when~calling~this~command.
  }
\msg_new:nnn { cmd } { unsupported-let }
  {
    The~command~'#1'~was~undefined~but~not~the~associated~commands~
    '#1~code'~and/or~'#1~defaults'.~Maybe~you~tried~using~
    \iow_char:N\\let.~This~may~lead~to~an~infinite~loop.
  }
%    \end{macrocode}
%
% \subsection{User functions}
%
% The user functions are more or less just the internal functions
% renamed.
%
% \begin{macro}{\BooleanFalse}
% \begin{macro}{\BooleanTrue}
%   Design-space names for the Boolean values.
%    \begin{macrocode}
\cs_new_eq:NN \BooleanFalse \c_false_bool
\cs_new_eq:NN \BooleanTrue  \c_true_bool
%    \end{macrocode}
% \end{macro}
% \end{macro}
%
% \begin{macro}{\NewDocumentCommand}
% \begin{macro}{\RenewDocumentCommand}
% \begin{macro}{\ProvideDocumentCommand}
% \begin{macro}{\DeclareDocumentCommand}
%   The user macros are pretty simple wrappers around the internal ones.
%   There is however a check that the first argument is a single token,
%   possibly surrounded by spaces (hence the strange \cs{use:nnn}), and
%   is definable.
%    \begin{macrocode}
\cs_new_protected:Npn \NewDocumentCommand #1#2#3
  {
    \@@_check_definable:nNT {#1} \NewDocumentCommand
      {
        \cs_if_exist:NTF #1
          {
            \msg_error:nnxx { cmd } { command-already-defined }
              { \use:nnn \token_to_str:N #1 { } }
              { \token_to_str:N \NewDocumentCommand }
          }
          { \@@_declare_cmd:Nnn #1 {#2} {#3} }
      }
  }
\cs_new_protected:Npn \RenewDocumentCommand #1#2#3
  {
    \@@_check_definable:nNT {#1} \RenewDocumentCommand
      {
        \cs_if_exist:NTF #1
          { \@@_declare_cmd:Nnn #1 {#2} {#3} }
          {
            \msg_error:nnxx { cmd } { command-undefined }
              { \use:nnn \token_to_str:N #1 { } }
              { \token_to_str:N \RenewDocumentCommand }
          }
      }
  }
\cs_new_protected:Npn \ProvideDocumentCommand #1#2#3
  {
    \@@_check_definable:nNT {#1} \ProvideDocumentCommand
      { \cs_if_exist:NF #1 { \@@_declare_cmd:Nnn #1 {#2} {#3} } }
 }
\cs_new_protected:Npn \DeclareDocumentCommand #1#2#3
  {
    \@@_check_definable:nNT {#1} \DeclareDocumentCommand
      { \@@_declare_cmd:Nnn #1 {#2} {#3} }
  }
%    \end{macrocode}
% \end{macro}
% \end{macro}
% \end{macro}
% \end{macro}
%
% \begin{macro}{\NewDocumentEnvironment}
% \begin{macro}{\RenewDocumentEnvironment}
% \begin{macro}{\ProvideDocumentEnvironment}
% \begin{macro}{\DeclareDocumentEnvironment}
%   Very similar for environments.
%    \begin{macrocode}
\cs_new_protected:Npn \NewDocumentEnvironment #1#2#3#4
  {
    \cs_if_exist:cTF {#1}
      { \msg_error:nnx { cmd } { environment-already-defined } {#1} }
      { \@@_declare_env:nnnn {#1} {#2} {#3} {#4} }
  }
\cs_new_protected:Npn \RenewDocumentEnvironment #1#2#3#4
  {
    \cs_if_exist:cTF {#1}
      { \@@_declare_env:nnnn {#1} {#2} {#3} {#4} }
      { \msg_error:nnx { cmd } { environment-undefined } {#1} }
  }
\cs_new_protected:Npn \ProvideDocumentEnvironment #1#2#3#4
  { \cs_if_exist:cF {#1} { \@@_declare_env:nnnn {#1} {#2} {#3} {#4} } }
\cs_new_protected:Npn \DeclareDocumentEnvironment #1#2#3#4
  { \@@_declare_env:nnnn {#1} {#2} {#3} {#4} }
%    \end{macrocode}
% \end{macro}
% \end{macro}
% \end{macro}
% \end{macro}
%
% \begin{macro}{\NewExpandableDocumentCommand}
% \begin{macro}{\RenewExpandableDocumentCommand}
% \begin{macro}{\ProvideExpandableDocumentCommand}
% \begin{macro}{\DeclareExpandableDocumentCommand}
%   The expandable versions are essentially the same as the basic
%   functions.  The strange \cs{use:nnn} is there in case |#1| is
%   surrounded with spaces, as can happen with usual document catcodes
%   in \cs{RenewExpandableDocumentCommand} |{| |\!| |}| \ldots{}
%    \begin{macrocode}
\cs_new_protected:Npn \NewExpandableDocumentCommand #1#2#3
  {
    \@@_check_definable:nNT {#1} \NewExpandableDocumentCommand
      {
        \cs_if_exist:NTF #1
          {
            \msg_error:nnxx { cmd } { command-already-defined }
              { \use:nnn \token_to_str:N #1 { } }
              { \token_to_str:N \NewExpandableDocumentCommand }
          }
          { \@@_declare_expandable_cmd:Nnn #1 {#2} {#3} }
      }
  }
\cs_new_protected:Npn \RenewExpandableDocumentCommand #1#2#3
  {
    \@@_check_definable:nNT {#1} \RenewExpandableDocumentCommand
      {
        \cs_if_exist:NTF #1
          { \@@_declare_expandable_cmd:Nnn #1 {#2} {#3} }
          {
            \msg_error:nnxx { cmd } { command-undefined }
              { \use:nnn \token_to_str:N #1 { } }
              { \token_to_str:N \RenewExpandableDocumentCommand }
          }
      }
  }
\cs_new_protected:Npn \ProvideExpandableDocumentCommand #1#2#3
  {
    \@@_check_definable:nNT {#1} \ProvideExpandableDocumentCommand
      {
        \cs_if_exist:NF #1
          { \@@_declare_expandable_cmd:Nnn #1 {#2} {#3} }
      }
 }
\cs_new_protected:Npn \DeclareExpandableDocumentCommand #1#2#3
  {
    \@@_check_definable:nNT {#1} \DeclareExpandableDocumentCommand
      { \@@_declare_expandable_cmd:Nnn #1 {#2} {#3} }
  }
%    \end{macrocode}
% \end{macro}
% \end{macro}
% \end{macro}
% \end{macro}
%
% \begin{macro}{\IfBooleanT, \IfBooleanF, \IfBooleanTF}
%   The logical \meta{true} and \meta{false} statements are just the
%   normal \cs{c_true_bool} and \cs{c_false_bool} so \cs{bool_if:NTF} is
%   almost enough.  However, this code-level function blows up badly
%   when passed invalid input.  We want \cs{IfBooleanTF} to accept a
%   single (non-space) token equal to \cs{c_true_bool} or
%   \cs{c_false_bool}, possibly surrounded by spaces.  If the input is
%   blank or multiple items, jump to the error and pick the false
%   branch.  If the input, ignoring spaces (we do this by omitting
%   braces in the \cs{tl_if_single_token:nF} test), is not a single
%   token then jump to the error as well.  It is then safe to compare
%   the token to the two booleans, picking the appropriate branch.  If
%   neither matches, we jump to the error as well.
%    \begin{macrocode}
\cs_new:Npn \IfBooleanTF #1
  {
    \tl_if_single:nF {#1}
      { \prg_break:n { \use:n } }
    \tl_if_single_token:nF #1
      { \prg_break:n { \use:n } }
    \token_if_eq_meaning:NNT #1 \c_true_bool
      { \prg_break:n { \use_ii:nnn } }
    \token_if_eq_meaning:NNT #1 \c_false_bool
      { \prg_break:n { \use_iii:nnn } }
    \prg_break:n { \use:n }
    \prg_break_point:
    {
      \msg_expandable_error:nnn { cmd } { if-boolean } {#1}
      \use_ii:nn
    }
  }
\cs_new:Npn \IfBooleanT #1#2 { \IfBooleanTF {#1} {#2} { } }
\cs_new:Npn \IfBooleanF #1 { \IfBooleanTF {#1} { } }
%    \end{macrocode}
% \end{macro}
%
% \begin{macro}{\IfNoValueT, \IfNoValueF, \IfNoValueTF}
%   Simple re-naming.
%    \begin{macrocode}
\cs_new_eq:NN \IfNoValueF  \tl_if_novalue:nF
\cs_new_eq:NN \IfNoValueT  \tl_if_novalue:nT
\cs_new_eq:NN \IfNoValueTF \tl_if_novalue:nTF
%    \end{macrocode}
% \end{macro}
% \begin{macro}{\IfValueT, \IfValueF, \IfValueTF}
%   Inverted logic.
%    \begin{macrocode}
\cs_new:Npn \IfValueF { \tl_if_novalue:nT }
\cs_new:Npn \IfValueT { \tl_if_novalue:nF }
\cs_new:Npn \IfValueTF #1#2#3 { \tl_if_novalue:nTF {#1} {#3} {#2} }
%    \end{macrocode}
% \end{macro}
%
% \begin{macro}{\ProcessedArgument}
%   Processed arguments are returned using this name, which is reserved
%   here although the definition will change.
%    \begin{macrocode}
\tl_new:N \ProcessedArgument
%    \end{macrocode}
% \end{macro}
%
% \begin{macro}{\ReverseBoolean, \SplitArgument, \SplitList, \TrimSpaces}
%   Simple copies.
%    \begin{macrocode}
\cs_new_eq:NN \ReverseBoolean \@@_bool_reverse:N
\cs_new_eq:NN \SplitArgument  \@@_split_argument:nnn
\cs_new_eq:NN \SplitList      \@@_split_list:nn
\cs_new_eq:NN \TrimSpaces     \@@_trim_spaces:n
%    \end{macrocode}
% \end{macro}
%
% \begin{macro}{\ProcessList}
%   To support \cs{SplitList}.
%    \begin{macrocode}
\cs_new_eq:NN \ProcessList \tl_map_function:nN
%    \end{macrocode}
% \end{macro}
%
% \begin{macro}{\GetDocumentCommandArgSpec}
% \begin{macro}{\GetDocumentEnvironmentArgSpec}
% \begin{macro}{\ShowDocumentCommandArgSpec}
% \begin{macro}{\ShowDocumentEnvironmentArgSpec}
%   More simple mappings, with a check that the argument is a single
%   control sequence or active character.
%    \begin{macrocode}
\cs_new_protected:Npn \GetDocumentCommandArgSpec #1
  {
    \@@_check_definable:nNT {#1} \GetDocumentCommandArgSpec
      { \@@_get_arg_spec:N #1 }
  }
\cs_new_eq:NN \GetDocumentEnvironmentArgSpec \@@_get_arg_spec:n
\cs_new_protected:Npn \ShowDocumentCommandArgSpec #1
  {
    \@@_check_definable:nNT {#1} \ShowDocumentCommandArgSpec
      { \@@_show_arg_spec:N #1 }
  }
\cs_new_eq:NN \ShowDocumentEnvironmentArgSpec \@@_show_arg_spec:n
%    \end{macrocode}
% \end{macro}
% \end{macro}
% \end{macro}
% \end{macro}
%
% Finally as promised, restore \cs{__kernel_chk_if_free_cs:N}:
%    \begin{macrocode}
%<latexrelease>\cs_gset_eq:NN \__kernel_chk_if_free_cs:N \@@_chk_if_free_cs:N
%<latexrelease>\cs_undefine:N \@@_chk_if_free_cs:N
%    \end{macrocode}
%
%    \begin{macrocode}
%<latexrelease>
%<latexrelease>\IncludeInRelease{0000/00/00}{ltcmd}%
%<latexrelease>                 {Document~command~parser}%
%<latexrelease>
%<latexrelease>\EndModuleRelease
\ExplSyntaxOff
%<latexrelease>\@ifundefined{ExplSyntaxOff}{}{\latexrelease@postexpl}
%<latexrelease>\catcode`\^^@=\@latexrelease@catcode@null\relax
%</2ekernel|latexrelease>
%    \end{macrocode}
%
% We need to stop DocStrip treating |@@| in a special way at this point.
%    \begin{macrocode}
%<@@=>
%    \end{macrocode}
%
% \Finale
