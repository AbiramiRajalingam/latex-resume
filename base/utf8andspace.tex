%% This is a patch of the LaTeX kernel to support UTF8 character in
%% all places where they can be supported by an 8bit engine such as
%% pdfTeX.
%%
%%
%% This should enable UTF8 not only in ordinary text (as
%% already provided by a recent LaTeX release but in addition
%% supports:
%%
%%  - utf8 characters in file names used by \input \includegraphics
%%    and the like --- this includes spaces and it is not longer necessary
%     to quote the file name in this case (not possible is the use of
%%    the " as part of a file name, this is restriction of the library
%%    the TeX engines use).
%%
%%  - use of all utf8 characters in labels
%%
%%  - in contrast to the utf8 characters that are used in typesetting
%%    it is not necessary that LaTeX has any knowledge how to render
%%    the character, e.g., without loading the textcomp package it is
%%    not possible to typeset € but even then you can have a file or a
%%    label with that character.
%%
%%
%% The plan is to integrate this patch (or a version of it) into the
%% kernel. Thus the current external version is intended to invite
%% tests with real documents beyond the test suite that we have
%% available at our disposal.
%%
%%
%% If you find any issues, please prepare a short example and submit
%% it as an issue at
%%
%%    https://github.com/latex3/latex2e/issues
%%
%% Thanks!


\makeatletter



% utf8
%
%
%  whenever we encounter a UTF8 char in non-typesetting  situation we make sure it
%  doesn't expand.

%-------------------------------------------------------------------------

% Approach
%
% The utf8 characters are seen by an 8-bit engine as a sequence of octets.
%
% We make each starting octet an active character.
%

%  - When typesetting we pick up the necessary number of additional
%    octets check if they form a command that LaTeX knows about
%    ( \csname u8:\string#1\string#2...\encsname ) and if so use that
%    for typesetting.  \string is needed as the octets may (all?) be
%    active and we want the literal values in the name.

%  - If the utf8 character is going to be part of a label then it is
%    essentially becoming part of some csname and with the
%    test \ifincsname we can find this out. If so we render the whole
%    sequence off octets harmless by using \string too when the
%    starting octet executes.
%

%  - Another possible case is that \protect has *not* the meaning
%    of \typeset@protect. In that case we may do a write or we may do
%    a \protected@edef or ...  In all such cases we want to keep the
%    sequence of octets unchanged, but we can't use \string since at
%    least in the case of \protect@edef the result may later be
%    typeset after all (in fact that is quite likely) and so at that
%    point the starting octet needs to be an active character again
%    (the others could be stringified). So for those cases we use \noexpand.
%

%  So the code for a start octet of a two byte sequence would there
%  look like this:

%
%
% \long\def\UTFviii@two@octets{%
%   \ifincsname
%     \expandafter
%     \UTF@twostring@octets
%   \else
%     \ifx\protect\@typeset@protect
%     \else
%       \expandafter\expandafter\expandafter
%      \UTF@twoharmless@octets
%     \fi
%   \fi
%   \UTFviii@two@octets@do
% }
% 

% \ifcsname is tested first because that can be true even if we are
%  otherwise doing typesetting. If this is the case use \string on the
%  whole octet sequence. \UTF@twostring@octets not only does this but
%  also gets rid of \UTFviii@two@octets@do in the input stream by
%  picking it up as a first argument and dropping it.
%
% If this is not the case and we are doing typesetting (i.e., \protect
%  is \typeset@protect) then execute \UTFviii@two@octets@do which
%  picks up all octets and typesets the character (or generates an
%  error if it doesn't know how to typeset it).
%
% If we are not doing typesetting then we run \UTFviii@two@octets@do
%  which is like \UTF@twostring@octets but uses \noexpand instead
%  of \string. This way the sequence is temporay frozen, eg would
%  display as is or stays put inside a \protected@edef but if the
%  result is later reused the starting octet is still active.
%
% The definitions for the other starting octets are the same except
% that they pick up more octets after them.



% In the original all starting octets would be defined as calling such
%  a \UTFviii@...@octets command followed by a \string version of the
%  octet itself (so that it can be used to form the character). We now
%  need to keep that octet active and so we have to do a slightly
%  different setup.
%
%
% So here is the new setup loop. Note that for error cases we can and
%  should of course use a \string version of the octet since there is
%  no point do extra work.

\begingroup
\catcode`\~13
\catcode`\"12
\def\UTFviii@loop{%
  \uccode`\~\count@
  \uppercase\expandafter{\UTFviii@tmp}%
  \advance\count@\@ne
  \ifnum\count@<\@tempcnta
  \expandafter\UTFviii@loop
  \fi}
    \def\UTFviii@tmp{\xdef~{\noexpand\UTFviii@undefined@err{:\string~}}}
    \count@"1
    \@tempcnta9
\UTFviii@loop
    \count@11
    \@tempcnta12
\UTFviii@loop
    \count@14
    \@tempcnta32
\UTFviii@loop
    \count@"80
    \@tempcnta"C2
    \def\UTFviii@tmp{\xdef~{\noexpand\UTFviii@invalid@err\string~}}
\UTFviii@loop
    \count@"C2
    \@tempcnta"E0
    \def\UTFviii@tmp{\xdef~{\noexpand\UTFviii@two@octets\noexpand~}}
\UTFviii@loop
    \count@"E0
    \@tempcnta"F0
    \def\UTFviii@tmp{\xdef~{\noexpand\UTFviii@three@octets\noexpand~}}
\UTFviii@loop
    \count@"F0
    \@tempcnta"F5
    \def\UTFviii@tmp{\xdef~{\noexpand\UTFviii@four@octets\noexpand~}}
\UTFviii@loop
    \count@"F5
    \@tempcnta"100
    \def\UTFviii@tmp{\xdef~{\noexpand\UTFviii@invalid@err\string~}}
\UTFviii@loop
\endgroup

% These are new work macros for the sequences as discussed above.

\long\def\UTFviii@two@octets{%
  \ifincsname
    \expandafter
    \UTF@twostring@octets
  \else
    \ifx\protect\@typeset@protect
    \else
      \expandafter\expandafter\expandafter
     \UTF@twoharmless@octets
    \fi
  \fi
  \UTFviii@two@octets@do
}


\long\def\UTFviii@three@octets{%
  \ifincsname
    \expandafter
    \UTF@threestring@octets
  \else
    \ifx\protect\@typeset@protect
    \else
      \expandafter\expandafter\expandafter
     \UTF@threeharmless@octets
    \fi
  \fi
  \UTFviii@three@octets@do
}


\long\def\UTFviii@four@octets{%
  \ifincsname
    \expandafter
    \UTF@fourstring@octets
  \else
    \ifx\protect\@typeset@protect
    \else
      \expandafter\expandafter\expandafter
     \UTF@fourharmless@octets
    \fi
  \fi
  \UTFviii@four@octets@do
}


% The \...@do are more or less what the original code was doing as
%  part of \UTFviii@...@octets. However #1 is now active (wasn't in
%  the original impl) so we better string that inside the cs. This
%  is faster than having it figure out that by itself that it is in a
%  csname.

\long\def\UTFviii@two@octets@do#1#2{\expandafter
    \UTFviii@defined\csname u8:\string#1\string#2\endcsname}
\long\def\UTFviii@three@octets@do#1#2#3{\expandafter
    \UTFviii@defined\csname u8:\string#1\string#2\string#3\endcsname}
\long\def\UTFviii@four@octets@do#1#2#3#4{\expandafter
    \UTFviii@defined\csname u8:\string#1\string#2\string#3\string#4\endcsname}


% These tempoarily prevent the active chars from expanding. (Maybe
%  using \unexpanded would be faster here?)

\long\def\UTF@twoharmless@octets#1#2{\noexpand#2\noexpand}
\long\def\UTF@threeharmless@octets#1#2#3{\noexpand#2\noexpand#3\noexpand}
\long\def\UTF@fourharmless@octets#1#2#3#4{\noexpand#2\noexpand#3\noexpand#4\noexpand}

% And the same with \string for use in \csname constructions.

\long\def\UTF@twostring@octets#1#2{\string#2\string}
\long\def\UTF@threestring@octets#1#2#3{\string#2\string#3\string}
\long\def\UTF@fourstring@octets#1#2#3#4{\string#2\string#3\string#4\string}


% The kernel already has saved away definitions for the starting code so
%  we have to refresh that (until the day this is properly integrated):

% if used in the kernel we also need this:
\let\UTFviii@two@octets@@\UTFviii@two@octets
\let\UTFviii@three@octets@@\UTFviii@three@octets
\let\UTFviii@four@octets@@\UTFviii@four@octets

% Done :-)


%-------------------------------------------------------------------------
%

% File name handling is done by generating a csname from the provided
%  file name (which means that utf8 octets gets turned into strings
%  due to the above procedure). By setting \escapchar to -1 we ensure
%  that we don't get a \ in front. As a result we end up with all
%  characters as catcode 12 (plus spaces). We then sometimes add
%  quotes around the contruct (removing any existing inner
%  quotes. Somes we only remove the quotes if they have been supplied
%  by the user. There is clearly some room for improvement.
%
% A side effect of the new code is that we will see quotes around file
%  name displays where there haven't been any before.

\def\set@curr@file#1{%
  \begingroup
    \escapechar\m@ne
    \xdef\@curr@file{\expandafter\string\csname #1\endcsname}%
  \endgroup
}

% quoting spaces
% a b c     -> "a b c"
% "a b c"   -> "a b c"
% a" "b" "c -> "a b c"
%           -> ""
\def\quote@name#1{"\quote@@name#1\@gobble""}
\def\quote@@name#1"{#1\quote@@name}

% removing quotes
%
\def\unquote@name#1{\quote@@name#1\@gobble"}


%-------------------------------------------------------------------------

% pass quoted file name as normal argument to allow recursive use

\let\IfFileExists@\IfFileExists

\def\IfFileExists#1{%
  \set@curr@file{#1}%
  \edef\q@curr@file{\expandafter\quote@name\expandafter{\@curr@file}}%
  \expandafter\IfFileExists@\expandafter{\q@curr@file}}



\long\def \InputIfFileExists#1#2{%
  \IfFileExists{#1}%
    {#2\@addtofilelist{#1}\@@input \@filef@und}}

\def\@iinput#1{%
  \InputIfFileExists{#1}{}%
  {\filename@parse\@curr@file
   \edef\reserved@a{\noexpand\@missingfileerror
     {\filename@area\filename@base}%
     {\ifx\filename@ext\relax tex\else\filename@ext\fi}}%
   \reserved@a}}


%-------------------------------------------------------------------------

\def\include#1{\relax
  \ifnum\@auxout=\@partaux
    \@latex@error{\string\include\space cannot be nested}\@eha
  \else
  \set@curr@file{#1 }%
  \expandafter\@include\@curr@file
  \fi}



% that zapspace ... hmmm

\def\includeonly#1{%
  \@partswtrue
  \set@curr@file{\zap@space#1 \@empty}%
  \let\@partlist\@curr@file
  }


\begingroup%
\@tempcnta=1
\loop
  \catcode\@tempcnta=12  %
  \advance\@tempcnta\@ne %
\ifnum\@tempcnta<32      %
\repeat                  %
\catcode`\*=11 %
\catcode`\^^M\active%
\catcode`\^^L\active\let^^L\relax%
\catcode`\^^I\active%
\gdef\filec@ntents#1{%
  \set@curr@file{#1}%
  \edef\q@curr@file{\expandafter\quote@name\expandafter{\@curr@file}}%
  \openin\@inputcheck\q@curr@file \space %
  \ifeof\@inputcheck%
    \@latex@warning@no@line%
        {Writing file `\@currdir\@curr@file'}%
    \chardef\reserved@c15 %
    \ch@ck7\reserved@c\write%
    \immediate\openout\reserved@c\q@curr@file\relax%
  \else%
    \closein\@inputcheck%
    \@latex@warning@no@line%
            {File `\@curr@file' already exists on the system.\MessageBreak%
             Not generating it from this source}%
    \let\write\@gobbletwo%
    \let\closeout\@gobble%
  \fi%
  \if@tempswa%
    \immediate\write\reserved@c{%
      \@percentchar\@percentchar\space%
          \expandafter\@gobble\string\LaTeX2e file `\@curr@file'^^J%
      \@percentchar\@percentchar\space  generated by the %
        `\@currenvir' \expandafter\@gobblefour\string\newenvironment^^J%
      \@percentchar\@percentchar\space from source `\jobname' on %
         \number\year/\two@digits\month/\two@digits\day.^^J%
      \@percentchar\@percentchar}%
  \fi%
  \let\do\@makeother\dospecials%
  \count@ 128\relax%
  \loop%
    \catcode\count@ 11\relax%
    \advance\count@ \@ne%
    \ifnum\count@<\@cclvi%
  \repeat%
  \edef\E{\@backslashchar end\string{\@currenvir\string}}%
  \edef\reserved@b{%
    \def\noexpand\reserved@b%
         ####1\E####2\E####3\relax}%
  \reserved@b{%
    \ifx\relax##3\relax%
      \immediate\write\reserved@c{##1}%
    \else%
      \edef^^M{\noexpand\end{\@currenvir}}%
      \ifx\relax##1\relax%
      \else%
          \@latex@warning{Writing text `##1' before %
             \string\end{\@currenvir}\MessageBreak as last line of \@curr@file}%
        \immediate\write\reserved@c{##1}%
      \fi%
      \ifx\relax##2\relax%
      \else%
         \@latex@warning{%
           Ignoring text `##2' after \string\end{\@currenvir}}%
      \fi%
    \fi%
    ^^M}%
  \catcode`\^^L\active%
  \let\L\@undefined%
  \def^^L{\expandafter\ifx\csname L\endcsname\relax\fi ^^J^^J}%
  \catcode`\^^I\active%
  \let\I\@undefined%
  \def^^I{\expandafter\ifx\csname I\endcsname\relax\fi\space}%
  \catcode`\^^M\active%
  \edef^^M##1^^M{%
    \noexpand\reserved@b##1\E\E\relax}}%
\endgroup%




% graphics



\AtBeginDocument{%
\def\Gin@getbase#1{%
  \edef\Gin@tempa{%
    \def\noexpand\@tempa####1#1\space{%
      \def\noexpand\Gin@base{"####1"}}}%
  \IfFileExists{\filename@area\filename@base#1}%
    {\Gin@tempa
\edef\uq@filef@und{\expandafter\unquote@name\expandafter{\@filef@und}}%
     \expandafter\@tempa\uq@filef@und
     \edef\Gin@ext{#1}}{}}%

\def\Ginclude@graphics#1{%
  \edef\Gin@extensions{\detokenize\expandafter{\Gin@extensions}}%
  \begingroup
  \let\input@path\Ginput@path
  \set@curr@file{#1}%
  \edef\uq@curr@file{\expandafter\unquote@name\expandafter{\@curr@file}}%
  \expandafter\filename@parse\expandafter{\uq@curr@file}%
  \edef\filename@area{\expandafter\quote@name\expandafter{\filename@area}}%
  \edef\filename@base{\expandafter\quote@name\expandafter{\filename@base}}%
  \ifx\filename@ext\relax
    \@for\Gin@temp:=\Gin@extensions\do{%
      \ifx\Gin@ext\relax
        \Gin@getbase\Gin@temp
      \fi}%
  \else
    \Gin@getbase{\Gin@sepdefault\filename@ext}%
    \ifx\Gin@ext\relax
       \@warning{File `#1' not found}%
       \def\Gin@base{\filename@area\filename@base}%
       \edef\Gin@ext{\Gin@sepdefault\filename@ext}%
    \fi
  \fi
    \ifx\Gin@ext\relax
         \@latex@error{File `#1' not found}%
         {I could not locate the file with any of these extensions:^^J%
          \Gin@extensions^^J\@ehc}%
    \else
       \@ifundefined{Gin@rule@\Gin@ext}%
         {\ifx\Gin@rule@*\@undefined
            \@latex@error{Unknown graphics extension: \Gin@ext}\@ehc
          \else
            \expandafter\Gin@setfile\Gin@rule@*{\Gin@base\Gin@ext}%
           \fi}%
         {\expandafter\expandafter\expandafter\Gin@setfile
             \csname Gin@rule@\Gin@ext\endcsname{\Gin@base\Gin@ext}}%
    \fi
  \endgroup}
}
\makeatother


