% \iffalse meta-comment
%
% Copyright (C) 1993-2021
% The LaTeX3 Project and any individual authors listed elsewhere
% in this file.
%
% This file is part of the LaTeX base system.
% -------------------------------------------
%
% It may be distributed and/or modified under the
% conditions of the LaTeX Project Public License, either version 1.3c
% of this license or (at your option) any later version.
% The latest version of this license is in
%    http://www.latex-project.org/lppl.txt
% and version 1.3c or later is part of all distributions of LaTeX
% version 2008 or later.
%
% This file has the LPPL maintenance status "maintained".
%
% The list of all files belonging to the LaTeX base distribution is
% given in the file `manifest.txt'. See also `legal.txt' for additional
% information.
%
% The list of derived (unpacked) files belonging to the distribution
% and covered by LPPL is defined by the unpacking scripts (with
% extension .ins) which are part of the distribution.
%
% \fi
% Filename: ltnews02.tex

% This is issue 2 of LaTeX News.

\documentclass
%   [lw35fonts]
   {ltnews}

\publicationmonth{December}
\publicationyear{1994}
\publicationissue{2}

\begin{document}

\maketitle

\section{Welcome to \LaTeXNews~2}

An issue of \emph{\LaTeXNews} will accompany every future release of
\LaTeX.  It will tell you about important events, such as major bug
fixes, newly available packages, or any other \LaTeX{} news.

\section{December 1994 release of \LaTeX}

December 1994 sees the second release of \LaTeXe.  We are on schedule
to deliver a release of \LaTeX{} every six months, in December and
June.

This release has seen quite a lot of activity, which is not too
surprising as it's only been a year since the first test release of
\LaTeXe.  We don't expect so much activity in the next six months.

Many of the changes are minor improvements and bug-fixes---see
\emph{\LaTeXe{} for authors} (\verb|usrguide.tex|), \emph{\LaTeXe{}
font selection} (\verb|fntguide.tex|) and our change log
(\verb|changes.txt|) for more details.

However, there are two important new packages available for \LaTeX:
\texttt{inputenc} and AMS-\LaTeX.

\section{Accented input}

One of the problems with writing non-English documents in \LaTeX{} is
the accent commands.  Reading documents containing text like
\verb|na\"\i ve| is frustrating, especially if your keyboard allows
you to type \texttt{na\"\i ve}.

In the past, \LaTeX{} has not supported input containing accented
characters such as \texttt{\"\i}, because Windows, Macintosh and Unix
all have different ways of dealing with accented input, called
\emph{input encodings}.

However, the \verb|inputenc| package allows you to specify which input
encoding your document is written with, for example to use the ISO
Latin-1 encoding, you type:
\begin{verbatim}
   \usepackage[latin1]{inputenc}
\end{verbatim}
At the moment, \verb|inputenc| supports the \verb|ascii| and
\verb|latin1| input encodings, but more will be added with future
releases.

The \verb|inputenc| package is
currently a test release.  The user interface for the full release will
be upwardly compatible with the test version.

\section{AMS-\LaTeX{}}

 AMS-\LaTeX{} is a set of miscellaneous extensions for \LaTeX{}
 distributed by the American Mathematical Society. They provide superior
 information structure and superior printed output for mathematical
 documents.

 There are far too many features of AMS-\LaTeX{} to list here.
 AMS-\LaTeX{} is described in the accompanying documentation,
 and in \emph{The \LaTeX{} Companion}.

 Version 1.2beta of AMS-\LaTeX{} was released for testing by intrepid
 users in October 1994.  The full release of AMS-\LaTeX{}~1.2 is expected
 in early January 1995.

 It will be divided into two bundles:
 \begin{itemize}

 \item the \verb|amsfonts| packages, which give access to
    hundreds of new mathematical symbols, and new math fonts
    such as blackboard bold and fraktur.

 \item the \verb|amsmath| packages, which provide finer control over
    mathematical typesetting, such as multi-line subscripts,
    enhanced theorem and proof environments,
    and improved displayed equations,

 \end{itemize}
 For compatibility with older documents, an \verb|amstex| package will be
 provided.

\section{\LaTeX{} on the internet}

\LaTeX{} has its own home page on the World Wide Web, with the URL:
\begin{verbatim}
   http://www.tex.ac.uk/CTAN/latex/
\end{verbatim}
This page describes \LaTeX{} and the \LaTeX3 project, and contains
pointers to other \LaTeX{} resources, such as the user guides, the
\TeX{} Frequently Asked Questions, and the \LaTeX{} bugs database.

The electronic home of anything \TeX-related is the Comprehensive
\TeX{} Archive Network (CTAN).  This is a network of cooperating ftp
sites, with over a gigabyte of \TeX{} material:
\begin{verbatim}
   ftp://ftp.tex.ac.uk/tex-archive/
   ftp://ftp.shsu.edu/tex-archive/
   ftp://ftp.dante.de/tex-archive/
\end{verbatim}
For more information, see the \LaTeX{} home page.

\section{Further information}

For more information on \TeX{} and \LaTeX, get in touch with your local
\TeX{} Users Group, or the international \TeX{} Users Group,
P.~O.~Box~869, Santa~Barbara, CA~93102-0869, USA, Fax:~+1~805~963~8358,
EMail:~tug@tug.org.


\end{document}
