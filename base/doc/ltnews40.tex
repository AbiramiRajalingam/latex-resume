% \iffalse meta-comment
%
% Copyright 2024
% The LaTeX Project and any individual authors listed elsewhere
% in this file.
%
% This file is part of the LaTeX base system.
% -——————————————
%
% It may be distributed and/or modified under the
% conditions of the LaTeX Project Public License, either version 1.3c
% of this license or (at your option) any later version.
% The latest version of this license is in
%    https://www.latex-project.org/lppl.txt
% and version 1.3c or later is part of all distributions of LaTeX
% version 2008 or later.
%
% This file has the LPPL maintenance status "maintained".
%
% The list of all files belonging to the LaTeX base distribution is
% given in the file `manifest.txt'. See also `legal.txt' for additional
% information.
%
% The list of derived (unpacked) files belonging to the distribution
% and covered by LPPL is defined by the unpacking scripts (with
% extension .ins) which are part of the distribution.
%
% \fi
% Filename: ltnews40.tex
%
% This is issue 40 of LaTeX News.

\NeedsTeXFormat{LaTeX2e}[2020-02-02]

\documentclass{ltnews}

%% Maybe needed only for Chris' inadequate system:
\providecommand\Dash {\unskip \textemdash}

%% NOTE: Chris' preferred hyphens!
%% \showhyphens{parameters}
%% \hyphenation{because}

\usepackage[T1]{fontenc}

\usepackage{lmodern,url,hologo}

\usepackage{csquotes}
\usepackage{multicol}
\usepackage{color}

\providecommand\hook[1]{\texttt{#1}}

\providecommand\meta[1]{$\langle$\textrm{\itshape#1}$\rangle$}
\providecommand\option[1]{\texttt{#1}}
\providecommand\env[1]{\texttt{#1}}
\providecommand\Arg[1]{\texttt\{\meta{#1}\texttt\}}


\providecommand\eTeX{\hologo{eTeX}}
\providecommand\XeTeX{\hologo{XeTeX}}
\providecommand\LuaTeX{\hologo{LuaTeX}}
\providecommand\pdfTeX{\hologo{pdfTeX}}
\providecommand\MiKTeX{\hologo{MiKTeX}}
\providecommand\CTAN{\textsc{ctan}}
\providecommand\TL{\TeX\,Live}


\providecommand\githubissue[2][]{\ifhmode\unskip\fi
     \quad\penalty500\strut\nobreak\hfill
     \mbox{\small\slshape(%
       \href{https://github.com/latex3/latex2e/issues/\getfirstgithubissue#2 \relax}%
          	    {github issue#1 #2}%
           )}%
     \par\smallskip}


% same for issues in the tagging repository:

\providecommand\taggingissue[2][]{\ifhmode\unskip\fi
     \quad\penalty500\strut\nobreak\hfill
     \mbox{\small\slshape(%
       \href{https://github.com/latex3/tagging-project/issues/\getfirstgithubissue#2 \relax}%
          	    {github tagging issue#1 #2}%
           )}%
     \par\smallskip}

%% But Chris has to mostly disable \href for his TEXPAD app:
%% \def\href #1#2{#2} % Only For Chris' deficient TeX engine

% simple solution right now (just link to the first issue if there are several)
\def\getfirstgithubissue#1 #2\relax{#1}

\providecommand\sxissue[1]{\ifhmode\unskip
     \else
       % githubissue preceding
       \vskip-\smallskipamount
       \vskip-\parskip
     \fi
     \quad\penalty500\strut\nobreak\hfill
     \mbox{\small\slshape(\url{https://tex.stackexchange.com/#1})}\par}

\providecommand\gnatsissue[2]{\ifhmode\unskip\fi
     \quad\penalty500\strut\nobreak\hfill
     \mbox{\small\slshape(%
       \href{https://www.latex-project.org/cgi-bin/ltxbugs2html?pr=#1\%2F\getfirstgithubissue#2 \relax}%
          	    {gnats issue #1/#2}%
           )}%
     \par}



\let\cls\pkg
\providecommand\env[1]{\texttt{#1}}
\providecommand\acro[1]{\textsc{#1}}

\vbadness=1400  % accept slightly empty columns


\let\finalpagebreak\pagebreak % for TUB (if they use it)
\let\finalvspace\vspace       % for document layout fixes

\makeatletter
% maybe not the greatest design but normally we wouldn't have subsubsections
\renewcommand{\subsubsection}{%
   \@startsection {subsubsection}{2}{0pt}{1.5ex \@plus 1ex \@minus .2ex}%
                  {-1em}{\@subheadingfont\colonize}%
}
\providecommand\colonize[1]{#1:}
\makeatother


% Undo ltnews's \verbatim@font with active < and >
\makeatletter
\def\verbatim@font{\normalsize\ttfamily}
\makeatother


%%%%%%%%%%%%%%%%%%%%%%%%%%%%%%%%%%%%%%%%%%%%%%%%%%%%%%%%%%%%%%%%%%%%%%%%%%%%%
\providecommand\tubcommand[1]{}
\tubcommand{\input{tubltmac}}

\publicationmonth{November}
%\publicationyear{2024  --- DRAFT version for upcoming release}
\publicationyear{2024}

\publicationissue{40}

\begin{document}

\maketitle
{\hyphenpenalty=10000 \exhyphenpenalty=10000 \spaceskip=3.33pt \hbadness=10000
\tableofcontents}

\setlength\rightskip{0pt plus 3em}

\medskip


\section{Thirty years of \LaTeXe{}}

In summer 1994, i.e., thirty years ago, \LaTeXe{} saw its first public
release. Back then it was meant to be an intermediate version (hence
the $\epsilon$) on the way to a major new version (the mythical
\LaTeX3) that we expected to take a couple of more years to reach
maturity. It took much more than that in the end\Dash nominally,
\LaTeXe{} is still with us today.

However, under the hood, \LaTeXe{} changed a lot throughout these
thirty years, as one can see, for example, when looking through the
forty newsletters~\cite{40:ltnews} that accompanied the \LaTeX{}
releases that happened in the meantime.

During the first two decades, the \LaTeX{} kernel was kept largely
stable with only minimal bug fix activities. During that period
additional functionality was mostly provided through new or
extended packages that could be loaded in the document preamble. This
included many of the ideas targeted for \LaTeX3{}, e.g., \pkg{expl3}
(\LaTeX3 programming language), \pkg{xparse} (new document command
interface), \pkg{xtemplate} (a configuration mechanism), and many
others.

Initially, this approach worked well and provided good backward
compatibility; however, over time it became apparent that keeping all
developments confined to packages was more and more problematical.
Features or bug fixes that should have been generally available, i.e.,
part of the kernel, were only available in packages, so a lot of
dependencies between packages were introduced and resulted in
convoluted code that was difficult to manage. For example,
\pkg{hyperref} had to rewrite a lot of kernel (and package) macros, so
the code and behavior of other packages had to change depending on
whether or not \pkg{hyperref} was loaded or not.

Thus, in 2015 the \LaTeX{} team decided to change the policy and
(re)start active kernel development, see~\cite{40:ltnews22}. To ensure
continuous backward compatibility we introduced at the same time the
\pkg{latexrelease} package that enables users to roll back changes to
the \LaTeX{} kernel to an earlier release, in case this is necessary
to successfully rerun a document produced at that time.

As a consequence of this policy change the last decade saw a larger
number of enhancements and corrections that were made part of the
\LaTeX{} kernel. Overall, we can confidently say that the new
approach has worked well and enabled us to modernize \LaTeX{} and ensure
that it remains relevant without compromising one of the cornerstones
of \LaTeX{}: its outstanding ability to reprocess old documents
written many years ago.

Being able to update and modernize the kernel sources allowed us to
embark in 2019 on the multi-year \enquote{\LaTeX{} Tagged PDF} project
with the goal of automatically providing accessible PDF documents with
\LaTeX{}. While there are several more project phases to complete, the
milestones already reached allow users to generate PDF/UA compliant
documents if the input is restricted to a (growing) subset of packages
and document classes; see next section and previous newsletters.

 A big change happened with the 2020-02-02 release as part of the project activity,
 albeit somewhat obfuscated by us as \enquote{Improved load-times for
   \pkg{expl3}}. While technically correct, what it really meant is
 that we had finally integrated the programming layer of \LaTeX3,
 i.e., the ideas originally sketched out around 1992.  Or saying it
 differently: with that date the original ideas for \LaTeX3 became a
 reality as part of the standard \LaTeX{} kernel.

 With the programming layer available under the hood we were then able to
 provide new concepts and extensions as part of \LaTeX{},
 e.g., the hook management system, a new mark mechanism, core
 functionality for tagging and PDF resource management, a consistent
 key/value interface, and more recently the socket and plug mechanism.

 More will follow while we continue to work on modernizing \LaTeX{} and
 bringing the Tagged PDF project to a truly successful completion\Dash so
 stay tuned and watch this space for future announcements in the next
 newsletters.

 

\section{News from the \enquote{\LaTeX{} Tagged PDF} project}

\subsection{Engine support:\ An important update}

As detailed below, work is progressing on the Tagged PDF
project. There are many drivers for this work, including legal changes
in many places which will increasingly require well-tagged PDFs
including full support for mathematics. As part of the work on this,
we are looking at the technical abilities of the \TeX{} engines.

With \XeTeX{}, it is impossible to reliably produce tagged PDFs due to
engine limitations. The increasing importance of tagged PDFs means
that this requires a move away from \XeTeX{}. We will continue to
address issues with \XeTeX{} support in team-maintained \LaTeX{} code
on a best-effort basis.  No \emph{new} functionality will be added for
\XeTeX{} by the \LaTeX{} team. It is likely that over time
functionality may become more restricted, and users are urged to
migrate \XeTeX{} documents to \LuaTeX{}.

For \pdfTeX{}, tagging is available and we are able to support
mathematics by including relevant \TeX{} source or by using
externally-generated MathML.  Only \LuaTeX{} is capable of
\emph{automatic} generation of MathML as part of a \LaTeX{} run. Thus
\pdfTeX{} continues to be supported for existing material, but for new
documents, moving to \LuaTeX{} is recommended.

We cannot make statements about the support for other engines such as
(u)p\TeX{}, as we don't use these programs nor have in depth knowledge
of their functionalities. To the best of our knowledge, core \LaTeX{}
works well with these engines, but if and to what extent tagging can
be supported will remain to be seen. If relevant information becomes
available to us we will provide an update in future editions of the
\LaTeX{} newsletter.


\subsection{Tagging support for external packages}

At \url{https://latex3.github.io/tagging-project/tagging-status/} we
show the status of many \LaTeX{} packages and classes with respect to
PDF tagging.  We also started to improve tagging support in external
packages. If the \texttt{firstaid} key is used in addition to the
\texttt{phase-III} key basic commands of several packages, including
\pkg{amsthm} and \pkg{fancyvrb}, can now be used.


\subsection{Improved table tagging}

The tagging of tabulars has been extended: it is now possible to tag
row headers and to create cells that span more than one row.

The interface to this functionality is not finalized but can be
accessed in the current release by specifying the row and columns to
be treated as headers. For example
%
\begin{verbatim}
  \tagpdfsetup{
   table/header-rows={1,2},
   table/header-columns={1} }
\end{verbatim}
would specify that in the following tables the first two rows and
first column of each row should be tagged as heading entries.

Similarly you may add a RowSpan attributes to tag a cell that spans
two rows using:
\begin{verbatim}
  \tagpdfsetup{table/multirow={2}}
\end{verbatim}


\subsection{Automatic MathML tagging}

When Lua\LaTeX{} is being used, and the \pkg{luamml} package is
available, and if the document uses the \pkg{unicode-math} package,
then the math module will automatically convert each math formula to
MathML and use it to attach MathML associated files (or MathML
Structure elements) to the tagged PDF. This new feature can be
disabled with \verb+\tagpdfsetup{math/mathml/luamml/load=false}+. More
options to configure MathML tagging can be found in the documentation
of \pkg{latex-lab-math}.


\subsection{Change behavior of tagging sockets with two arguments}

When calling tagging sockets with two arguments using
\cs{UseTaggingSocket} when tagging is suspended, previous versions of
\LaTeXe{} dropped both arguments.  This behavior has been changed to
drop the first argument and preserve the second one instead, thereby
allowing tagging sockets to be used to wrap existing content which
should still appear in a non-tagging context.

Since no tagging sockets currently provided by \LaTeX{} use two
arguments we do not expect this change to affect any existing
documents, but if a custom tagging socket has been defined outside of
the kernel it might need to be adapted to be compatible with the new
behavior.
%
\githubissue{1500}




\section{Changes to the \LaTeX{} kernel}

\subsection{Handling paragraph continuation}

Already \LaTeX~2.09 offered some automation to detect whether or not
text after a list or some other display environment is meant to be a
continuation of the current paragraph or should start a new one.  The
document-level syntax for this is that a blank line after such an
environment signals to \LaTeX{} that it should start a new paragraph;
whilst no blank line signals that there should be no new paragraph and
the text should be considered a continuation.

Unfortunately, there were a number of cases where the original 2.09
approach failed, e.g., with
\begin{flushleft}
  \ttfamily
\ \ \{\meta{local customizations}\\
\ \ \cs{begin}\{equation\} a<b \cs{end}\{equation\}\} \\
\ \ \meta{some text}
\end{flushleft}
the \meta{some text} incorrectly started a new paragraph.  Bug reports
about this behavior can be traced back to the time \LaTeXe{} was
developed, e.g., one test file from 1992 has a note that the above
case was unfortunately not resolvable despite some improvements made
back then.  The main cause of the issue (as you probably guessed) is
that the mechanism failed whenever the environment was executed within
a group (\texttt{\{...\}}, \cs{begingroup}/\cs{endgroup}, or
\cs{bgroup}/\cs{egroup} pair) that was closed before the next blank
line was reached.

While most of the time this could be visually corrected by adding some
explicit \cs{noindent}, the situation got worse when we tried to
implement tagged PDFs resulting in incorrect structures or worse.

We therefore made a new attempt to resolve this problem in every
situation and this new solution is rolled out in the current release.


\subsection{Avoid bogus \enquote{no item} error}

The commands \cs{addvspace} and \cs{addpenalty} generated the famous
error message \enquote{Something's wrong---perhaps a missing \cs{item}}
when they were encountered outside vertical mode. Most of the time
this error was bogus and if not, then it was generated several times
rather than once.

Once upon a time (in \LaTeX{}~2.09) it was necessary that these
commands were used only in vertical mode, but with \LaTeXe{} in 1994,
we changed the internals but simply overlooked that this error message
then had become useless. In this release, i.e.,~30 years too late, we
have finally lifted the ban and from now on this error should only
show up if there is indeed a missing \cs{item}.
%
\githubissue{1460}


\subsection[Switch to T1 as default encoding in documents
  using \cs{DocumentMetadata}]
           {Switch to T1 as default encoding in documents\\\mbox\qquad
  using \cs{DocumentMetadata}}

As it is well known, the font encoding \texttt{OT1} supports only 128
characters and has various problems and quirks notably for languages
different to English.  Nevertheless \texttt{OT1} is the default
encoding in \LaTeX{} and this cannot be easily changed without
affecting many documents as the \texttt{T1} version of the fonts have
slightly different metrics.

The introduction of the \cs{DocumentMetadata} command, which announces
\emph{new} code and changes that can also affect the layout gives us
now the opportunity to make this step. So with this version a use of
\cs{DocumentMetadata} with (pdf)\LaTeX{} will setup \texttt{T1} as
default font encoding.\footnote{The Unicode engines will continue to
use \texttt{TU} as the encoding.} To ensure that scalable fonts are used,
the package \pkg{cm-super} has to be installed. Users who want to
revert to the \texttt{OT1} encoding in their document can do so with
\verb+\usepackage[OT1]{fontenc}+.




\section{Code improvements}

\subsection[Avoiding key--value option clashes between classes and packages]
           {Avoiding key--value option clashes between classes\\\mbox\qquad and packages}

In \LaTeX{} News~35~\cite{40:ltnews35} we introduced key--value option
processing to the kernel. Following the standard for \LaTeXe{}
options, keyval options given to the \cs{documentclass} line were
treated as global and so parsed by every package. However, with
keyvals, the likelihood of a name clash between a class-specific
option and one used by a package is much higher than it is with simple
strings. We have therefore refined the mechanism in the current
release.

When a class uses the kernel keyval processor, any options it
recognizes are recorded and any packages using the keyval processor
will then \emph{skip} these \enquote{global} options. To allow for the
case where a class directly uses an option which should be global (for
example \texttt{draft}), a new key property \texttt{.pass-to-packages}
has been added. This can then be set to indicate that this key is not
to be skipped. For example
\begin{verbatim}
  \DeclareKeys{
    draft .if = {ifl@cls@draft},
    draft .pass-to-packages = true,
    mode  .store = \cls@mode
  }
\end{verbatim}
in a class would create two options, \texttt{draft} and
\texttt{mode}. The \texttt{draft} option will be treated in the normal
way by packages using keyvals, but they will ignore the \texttt{mode}
option: it is effectively marked as \enquote{private} to the class.
%
\githubissue{1279}

\subsection{Improvement to \textup{\XeTeX}\ \cs{showhyphens}}

When using \cs{showhyphens} with \XeTeX, missing character warnings
would be generated for any character not in Latin Modern. This has
been corrected and the warnings are suppressed.
%
\githubissue{1380}

\subsection{Improved error raised by empty hook}

When using the hook management, both hook and label names (if
specified) should be non-empty. Before, empty hook and empty label names both
raised the same label-specific error:
\begin{verbatim}
! LaTeX hooks Error: Empty code label on line ....
  Using 'top-level' instead.
\end{verbatim}
This has now been improved. Now an empty hook raises
\begin{verbatim}
! LaTeX hooks Error: Empty hook on line ....
\end{verbatim}
%
\githubissue{1423}

\subsection{Provide counter representations for link targets}

To create unique target names for links the package \pkg{hyperref}
uses a special counter representation \verb+\theH+\meta{counter}. To
ensure that this counter representation exists, \pkg{hyperref}
redefined the commands \verb+\@definecounter+, \verb+\@addtoreset+ and
\verb+\refstepcounter+. This counter representation is also needed for
the Tagged PDF project and so these augmented command definitions
have now been incorporated into the kernel.  Thus from now on every
\verb+\newcounter{+\meta{counter}\verb+}+ will not only define
\verb+\the+\meta{counter} but also \verb+\theH+\meta{counter}.

\subsection{Extending \cs{refstepcounter}}

For many years, the package \pkg{hyperref} had been redefining
\verb+\refstepcounter+ and adding code that creates link targets.  The
kernel definition has now been extended with socket interfaces that
will allow \pkg{hyperref} to avoid the redefinitions. The new
interfaces are also used by the Tagged PDF code that needs target
names to resolve references between structures.


\section{Bug fixes}

\subsection{Fix wrong file type in a rollback warning}

When \LaTeX{} is rolled back to date \meta{date1} and a class or
package with minimum date requirement \meta{date2} is to be loaded, a
rollback warning is raised if \meta{date2} is later than \meta{date1}:
\begin{verbatim}
LaTeX Warning: Suspicious rollback/min-date
               date given.
  A minimal date of YYYY-MM-DD has been
  specified for package '<pkgname>'.
  But this is in conflict with a rollback
  request to YYYY-MM-DD.
\end{verbatim}

In some cases this message showed a wrong file type, i.e.,
\verb|document class '<pkgname>'| or \verb|package '<clsname>'|.  This
has now been corrected.
%
\githubissue{870}

\subsection*{Fix existence check of document environments}

\cs{NewDocumentEnvironment} and friends define (or redefine) a
document environment using the space-trimmed \meta{envname}, but the
existence check for \meta{envname} was done without space
trimming. Thus when the user-specified \meta{envname} consists of
leading and/or trailing space(s), it may lead to erroneously silent
environment declaration.  For example, in
\begin{flushleft}
\ttfamily
\ \ \verb=\NewDocumentEnvironment{myenv}{}{begin}{end}=\\
\ \ \verb*=\NewDocumentEnvironment{ myenv }{}{begin}{end}=\kern-1cm
\end{flushleft}
the first line defines a new environment \env{myenv} but the second
line would check existence for \verb*| myenv | (which is not yet
defined), then redefine \env{myenv} environment without raising any
errors.  This has now been corrected.
%
\githubissue{1399}

\subsection{Handling of global keys with spaces}

If the global (class) options contained spaces around key names,
\cs{ProcessKeyOptions} would fail to remove known keys from the list
of unused global options and \cs{OptionNotUsed} would mistakenly add
space-surrounded key names to that list.  The first issue was corrected as
a hotfix in patch level 1 of the November 2023 release (but
unfortunately not mentioned in~\cite{40:ltnews38}) and the second in the current
release.
%
\githubissue{1238}

\subsection{File list entries for rolled back packages/classes}

When the rollback mechanism for packages and classes was introduced in
2018~\cite{40:ltnews28}, loading of the selected historic release was
not recorded in the file list used by \cs{listfiles}.  This has now
been corrected so that the extended usage~\cite{40:ltnews39}
\begin{verbatim}
  \listfiles[hashes,sizes]
\end{verbatim}
now gives more complete and less confusing info.
%
\githubissue{1413}

\subsection{\pkg{doc}:\ \cs{PrintDescribeMacro} in preamble}

In \pkg{doc} version 2 it was possible to alter the definition of
\cs{PrintDescribeMacro} and similar commands in preamble. In version 3
this stopped working because they were reset at the end of the
preamble. This has now been implemented differently and changes in the
preamble are possible again.
%
\githubissue{1000}


\subsection{Avoid low-level error if \cs{ShowHooks} is used late}

If \cs{ShowHooks} was used to examine a package hook after the
package was loaded, a low level error resulted. This has now been
corrected.
%
\githubissue{1513}

\subsection{Avoid code duplication in rollback}

When the kernel uses \cs{AddToHook} in a region that might be rolled
back (which happens in a few places) and a document requests a
rollback, then we have the situation that the hook already contains
code to which we added the same (or slightly different) code during
the rollback; this results in code duplication or, worse, in
errors. This has now been corrected by dropping any such code chunk
(if there is one) prior to adding the rollback code.
%
\githubissue{1407}


\subsection{Passing template keys using \cs{KeyValue}}

With the move of the template code to the kernel, internal
functions were reviewed to improve efficiency. However, there
was an oversight in how
passing key values from one setting to another was implemented,
such that using \cs{KeyValue} could result in an infinite
loop. This has now been fixed.
%
\githubissue{1486}



\section{Changes to packages in the \pkg{amsmath} category}

\subsection{Extend support for \cs{dots}}

The implementation of \cs{dots} in \pkg{amsmath} has the feature that
it selects different dots depending on the symbol that follows: e.g.,
dots between commas would normally be on the baseline, while dots
between binary or relational symbols would be raised. However, when
symbols such as \cs{cong} were protected from expansion in moving
arguments (so that they worked in places such as headings) it had the
unfortunate side-effect that the \cs{dots} magic stopped working for
them. This has now been corrected.
%
\githubissue{1265}



%\section{Changes to packages in the \pkg{graphics} category}

\section{Changes to packages in the \pkg{tools} category}

\subsection{Modification to generation of the \file{.tex} from \pkg{fileerr}}

The \pkg{fileerr} extraction has been modified to write
\texttt{rename-to-empty-base.tex} rather than \texttt{.tex} to comply
with an expected security change in \TeX{}~Live 2025.  The
\texttt{build.lua} file for the \pkg{tools} has been modified to
rename \texttt{rename-to-empty-base.tex} to \texttt{.tex} after
unpacking. However if using \textsf{docstrip} directly rather than
using \textsf{l3build} or the unpacked zip file from \CTAN{}, the user
must now rename the file and install as \texttt{.tex}.
%
\githubissue{1412}


\subsection{\pkg{array}:\ Improve \texttt{>\{...\}} specifier}

If the argument of \texttt{>\{...\}} ended with a command accepting a
trailing optional argument, e.g., defined for example with
\verb=\NewDocumentCommand\foo{o}{...}=, one could get low-level
parsing errors. This has now been corrected.
%
\githubissue{1468}


\subsection{\pkg{array}:\ Tagging support for \cs{cline}}

In the last release we added tagging support for \pkg{array},
\pkg{longtable} and other tabular packages, but we overlooked that the
kernel definition for \cs{cline} also needs modification because the
rule generated by the command needs to be tagged as an
artifact. Furthermore, the processing of a \cs{cline} looks to the
algorithm as if another row is added (which is technically what
happens), thus it was also necessary to decrement the internal row
counter to get a correct row count. This has now been corrected in
\pkg{array}, which is automatically loaded for tagging, so that all
these packages are now fully compatible with the tagging code if it is
turned on.
%
\taggingissue{134}



\subsection{\pkg{longtable}:\ Extend caption type}

The \pkg{longtable} package has been extended and now provides the command
\cs{LTcaptype} (stemming from the \pkg{ltcaption} package) to change
the counter and caption type used by the \cs{caption} command from
longtable.  So with \verb+\renewcommand\LTcaptype{figure}+, a
longtable will step the figure counter instead of the table counter
and produce an entry in the list of figures. An empty definition,
\verb+\renewcommand\LTcaptype{}+, will suppress increasing of the
counter. This makes it easy to define an unnumbered variant of
longtable:
\begin{verbatim}
  \newenvironment{longtable*}
    {\renewcommand\LTcaptype{}\longtable}
    {\endlongtable}
\end{verbatim}

\subsection{\pkg{longtable}:\ Prevent \cs{pagegoal} exceeding maximum value}

An internal guard has been added to avoid \TeX\ errors if
\verb=\pagegoal= is increased beyond the maximum value for a
\TeX\ dimension.
%
\githubissue{1495}



\section{Changes to \pkg{l3build}}

To support third-party developers testing their code against pre-release
\LaTeX{}, a new switch \verb|--dev| has been added to \pkg{l3build}.
This allows the developer to run
\begin{verbatim}
  l3build check
\end{verbatim}
to run their test suite against the current release of \LaTeX{} and
\begin{verbatim}
  l3build check --dev
\end{verbatim}
to run exactly the same tests using the development release of
\LaTeX{}.

%\section{Changes to files in the \pkg{cyrillic} category}

\begin{thebibliography}{9}\frenchspacing

%\fontsize{9.3}{11.3}\selectfont

\bibitem{40:Lamport}
Leslie Lamport.
\newblock \emph{{\LaTeX}: {A} Document Preparation System: User's Guide and Reference
  Manual}.
\newblock \mbox{Addison}-Wesley, Reading, MA, USA, 2nd edition, 1994.
\newblock ISBN 0-201-52983-1.
\newblock Reprinted with corrections in 1996.

\bibitem{40:ltnews} \LaTeX{} Project Team.
  \emph{\LaTeXe{} news 1--39}. June 2024.
  \url{https://latex-project.org/news/latex2e-news/ltnews.pdf}

\bibitem{40:ltnews22} \LaTeX{} Project Team.
  \emph{\LaTeXe{} news 22}. January 2015.
  \url{https://latex-project.org/news/latex2e-news/ltnews22.pdf}

\bibitem{40:ltnews28} \LaTeX{} Project Team.
  \emph{\LaTeXe{} news 28}. April 2018.
  \url{https://latex-project.org/news/latex2e-news/ltnews28.pdf}

\bibitem{40:ltnews35} \LaTeX{} Project Team.
  \emph{\LaTeXe{} news 35}. June 2022.
  \url{https://latex-project.org/news/latex2e-news/ltnews35.pdf}

\bibitem{40:ltnews38} \LaTeX{} Project Team.
  \emph{\LaTeXe{} news 38}. November 2023.
  \url{https://latex-project.org/news/latex2e-news/ltnews38.pdf}

\bibitem{40:ltnews39} \LaTeX{} Project Team.
  \emph{\LaTeXe{} news 39}. June 2024.
  \url{https://latex-project.org/news/latex2e-news/ltnews39.pdf}

\end{thebibliography}

\end{document}
