% \iffalse meta-comment
%
% Copyright (C) 1993-2024
% The LaTeX Project and any individual authors listed elsewhere
% in this file.
%
% This file is part of the LaTeX base system.
% -------------------------------------------
%
% It may be distributed and/or modified under the
% conditions of the LaTeX Project Public License, either version 1.3c
% of this license or (at your option) any later version.
% The latest version of this license is in
%    http://www.latex-project.org/lppl.txt
% and version 1.3c or later is part of all distributions of LaTeX
% version 2008 or later.
%
% This file has the LPPL maintenance status "maintained".
%
% The list of all files belonging to the LaTeX base distribution is
% given in the file `manifest.txt'. See also `legal.txt' for additional
% information.
%
% The list of derived (unpacked) files belonging to the distribution
% and covered by LPPL is defined by the unpacking scripts (with
% extension .ins) which are part of the distribution.
%
% \fi
% Filename: usrguide-historic.tex

\NeedsTeXFormat{LaTeX2e}[1995/12/01]

\documentclass{ltxguide}[2001/05/28]

\title{\LaTeX\ for authors --- historic version}

\author{\copyright~Copyright 1995--2023, \LaTeX\ Project Team.\\
   All rights reserved.%
   \footnote{This file may be distributed and/or modified under the
     conditions of the \LaTeX{} Project Public License, either version 1.3c
     of this license or (at your option) any later version. See the source
    \texttt{usrguide-historic.tex} for full details.}%
}

\date{09 November 2023}


\begin{document}

\maketitle

\tableofcontents

\section{Introduction}

Welcome to \LaTeXe, the new standard version of the \LaTeX{} Document
Preparation System.

This document describes how to take advantage of the new features of
\LaTeX, and how to process your old \LaTeX{} documents with
\LaTeXe. However, this document is only a brief introduction to the
new facilities and is intended for authors who are already familiar
with the old version of \LaTeX{}.  It is \emph{not} a reference manual
for \LaTeXe{} nor is it a complete introduction to \LaTeX.

It is somewhat of an historical document now, since \LaTeXe{} came into
existence in 1994.

\subsection[\LaTeXe---The new \LaTeX~release]
  {\LaTeXe---The new \LaTeX~release\\ (well, for more than 10 years now)}

The previous version of \LaTeX{} was known as \LaTeX~2.09.  Over the
years many extensions have been developed for \LaTeX.  This is, of
course, a sure sign of its continuing popularity but it has had one
unfortunate result: incompatible \LaTeX{} formats came into use at
different sites.  This included `standard \LaTeX~2.09', \LaTeX{} built
with the \emph{New Font Selection Scheme}~(\NFSS), \SLiTeX, \AmSLaTeX,
and so on.  Thus, to process documents from various places, a site
maintainer was forced to keep multiple versions of the \LaTeX{}
program.  In addition, when looking at a source file it was not always
clear for which format the document was written.

To put an end to this unsatisfactory situation, \LaTeXe{} has been
produced; it brings all such extensions back under a single format and
thus prevents the proliferation of mutually incompatible dialects of
\LaTeX~2.09.  With \LaTeXe{} the `new font selection scheme' is
standard and, for example, \textsf{amsmath} (formerly the \AmSLaTeX{}
format) or \textsf{slides} (formerly the \SLiTeX{} format) are simply
extensions, which may be loaded by documents using the same base format.

The introduction of a new release also made it possible to add a small
number of often-requested features and to make the task of writing
packages and classes simpler.

\subsection{\LaTeX3---The long-term future of \LaTeX}
\label{Sec:ltx3}

\LaTeXe{} is the consolidation step in a comprehensive
reimplementation of the \LaTeX{} system.  The next major release of
\LaTeX{} will be \LaTeX3, which will include a radical overhaul of the
document designers' and package writers' interface to \LaTeX.

\LaTeX3 is a long-term research project but, until it is completed,
the project team are committed to the active maintenance of \LaTeXe{}.
Thus the experience gained from the production and maintenance of
\LaTeXe{} will be a major influence on the design of \LaTeX3.
A brief description of the project can be found in the document
|ltx3info.tex|.

If you would like to support the project then you are welcome to send
donations to the \LaTeX\ Project Fund; this has been set up to help
the research team by financing various expenses associated with this
voluntary work of maintaining the current \LaTeX{} and developing
\LaTeX\ further.

The fund is administered by The \TeX{} Users Group and by various
local user groups.  Information about making donations and joining
these groups is available from:
\begin{quote}\small\label{addrs}
   \texttt{http://www.tug.org/lugs.html}
\end{quote}

The \LaTeX3{} project has its home page
on the World Wide Web at:
\begin{verbatim}
  http://www.latex-project.org/
\end{verbatim}
This page describes \LaTeX{} and the \LaTeX3 project, and contains
pointers to other \LaTeX{} resources, such as the user guides, the
\TeX{} Frequently Asked Questions, and the \LaTeX{} bugs database.

Older articles covering aspects of the \LaTeX3 project are also
available for anonymous ftp from the Comprehensive \TeX{} Archive, in
the directory:
\begin{verbatim}
  ctan:info/ltx3pub
\end{verbatim}
The file |ltx3pub.bib| in that directory contains an abstract of each
of the files.

\subsection{Overview}

This document contains an overview of the new structure and features
of \LaTeX.  It is \emph{not} a self-contained document, as it contains
only the features of \LaTeX{} which have changed since version 2.09.
You should read this document in conjunction with an introduction to
\LaTeX{}.

\begin{description}

\item[Section~\ref{Sec:class+packages}]
   contains an overview of the new structure of
   \LaTeX{} documents.  It describes how classes and packages work and
   how class and package options can be used.  It lists the standard
   packages and classes which come with \LaTeX.

\item[Section~\ref{Sec:commands}] describes the new commands available
   to authors in \LaTeXe.

\item[Section~\ref{Sec:209}] shows how to process old \LaTeX{}
   documents with \LaTeXe.

\item[Section~\ref{Sec:problems}] contains advice on dealing with
   problems you may encounter in running \LaTeXe.
   It lists some error messages which are new in \LaTeXe{} and
   it describes some of the more common problems and how to cure them,
   or where to find further information.

\end{description}

\subsection{Further information}

For a general introduction to \LaTeX, including the new features of
\LaTeXe, you should read \emph{\LaTeXbook}
by Leslie Lamport~\cite{A-W:LLa94}.

A more detailed description of the new features of \LaTeX, including an
overview of more than 200 packages and nearly 1000 ready to run examples, is
to be found in \emph{\LaTeXcomp second edition} by Frank Mittelbach and
Michel Goossens~\cite{A-W:MG2004}.

Packages and programs for producing and manipulating graphics are
discussed at length in \emph{\LaTeXGcomp} by Michel Goossens,
Sebastian Rahtz and Frank Mittelbach~\cite{A-W:GRM97}.

Solutions for publishing with \LaTeX{} on the World Wide Web are given
in \emph{\LaTeXWcomp} by Michel Goossens and Sebastian
Rahtz~\cite{A-W:GR99}.

For more information about the many new \LaTeX{} packages you should
read the package documentation, which should be available from the
same source as your copy of \LaTeX.

There are a number of documentation files which accompany every copy
of \LaTeX.  A copy of \emph{\LaTeX{} News} will come out with each
six-monthly release of \LaTeX; it will be found in the files
|ltnews*.tex|.  The class- and package-writer's guide \emph{\clsguide}
describes the new \LaTeX{} features for writers of document classes
and packages; it is in |clsguide.tex|.  The guide \emph{\fntguide}
describes the \LaTeX{} font selection scheme for class- and
package-writers; it is in |fntguide.tex|. Support for Cyrillic languages
in \LaTeX{} is described in \emph{\cyrguide}.

The documented source code (from the files used to produce the kernel
format via |latex.ltx|) is now available as
\emph{The \LaTeXe\ Sources}.  ]
This very large document also includes an index of
\LaTeX{} commands.  It can be typeset from the \LaTeX{} file
|source2e.tex| in the |base| directory, using the source files and
the class file |ltxdoc.cls| from this directory.

For more information about \TeX{} and \LaTeX{}, please contact your
local \TeX{} Users Group, or the international \TeX{} Users Group (see
page \pageref{addrs}).


\section{Classes and packages}
\label{Sec:class+packages}

This section describes the new structure of \LaTeX{} documents and the
new types of file: \emph{classes} and \emph{packages}.

\subsection{What are classes and packages?}

The main difference between \LaTeX~2.09 and \LaTeXe{} is in the
commands before |\begin{document}|.

In \LaTeX~2.09, documents had \emph{styles},
such as \textsf{article} or \textsf{book}, and \emph{options},
such as \textsf{twoside} or \textsf{epsfig}.
These were indicated by the |\documentstyle| command:
\begin{quote}
   |\documentstyle|\oarg{options}\arg{style}
\end{quote}
For example, to specify a two-sided article with encapsulated
PostScript figures, you said:
\begin{verbatim}
   \documentstyle[twoside,epsfig]{article}
\end{verbatim}
However, there were two different types of document style option:
\emph{built-in options} such as |twoside|; and \emph{packages} such as
|epsfig.sty|.  These were very different, since any \LaTeX{} document
style could use the \textsf{epsfig} package but only document styles
which declared the \textsf{twoside} option could use that option.

To avoid this confusion, \LaTeXe{} differentiates between built-in
options and packages.  These are given by the new |\documentclass| and
|\usepackage| commands:
\begin{quote}
   |\documentclass|\oarg{options}\arg{class} \\
   |\usepackage|\oarg{options}\arg{packages}
\end{quote}
For example, to specify a two-sided article with encapsulated
PostScript figures, you now write:
\begin{verbatim}
   \documentclass[twoside]{article}
   \usepackage{epsfig}
\end{verbatim}
You can load more than one package with a single |\usepackage|
command; for example, rather than writing:
\begin{verbatim}
   \usepackage{epsfig}
   \usepackage{multicol}
\end{verbatim}
you can specify:
\begin{verbatim}
   \usepackage{epsfig,multicol}
\end{verbatim}
Note that \LaTeXe{} still understands the \LaTeX~2.09 |\documentstyle|
command.  This command causes \LaTeXe{} to enter \emph{\LaTeX~2.09
compatibility mode}, which is described in Section~\ref{Sec:209}.

You should not, however, use the |\documentstyle| command for new
documents because this compatibility mode is very slow and the new
features of \LaTeXe{} are not available in this mode.

To help differentiate between classes and packages, document classes
now end with |.cls| rather than |.sty|.  Packages still end with
|.sty|, since most \LaTeX~2.09 packages work well with \LaTeXe.

\subsection{Class and package options}

In \LaTeX~2.09, only document styles could have options such as
|twoside| or |draft|.  In \LaTeXe{}, both classes and packages are
allowed to have options.  For example, to specify a two-sided article
with graphics using the |dvips| driver, you write:

\begin{verbatim}
   \documentclass[twoside]{article}
   \usepackage[dvips]{graphics}
\end{verbatim}
It is possible for packages to share common options.  For example,
you could, in addition, load the \textsf{color} package by specifying:
\begin{verbatim}
   \documentclass[twoside]{article}
   \usepackage[dvips]{graphics}
   \usepackage[dvips]{color}
\end{verbatim}
But because |\usepackage| allows more than one package to be listed,
this can be shortened to:
\begin{verbatim}
   \documentclass[twoside]{article}
   \usepackage[dvips]{graphics,color}
\end{verbatim}
In addition, packages will also use each option given to
the |\documentclass| command (if they know what to do with it), so you
could also write:
\begin{verbatim}
   \documentclass[twoside,dvips]{article}
   \usepackage{graphics,color}
\end{verbatim}
Class and package options are covered in more detail in
\emph{\LaTeXcomp} and in \emph{\clsguide}.

\subsection{Standard classes}

The following classes are distributed with \LaTeX:
\begin{description}

\item[article]  The |article| class described in \emph{\LaTeXbook}.
\item[book]     The |book| class described in \emph{\LaTeXbook}.
\item[report]   The |report| class described in \emph{\LaTeXbook}.
\item[letter]   The |letter| class described in \emph{\LaTeXbook}.
\item[slides]   The |slides| class described in \emph{\LaTeXbook},
   formerly \SLiTeX.
\item[proc]     A document class for proceedings, based on |article|.
   Formerly the |proc| package.
\item[ltxdoc]   The document class for documenting the \LaTeX{}
   program, based on |article|.
\item[ltxguide] The document class for \emph{\usrguide} and
   \emph{\clsguide}, based on |article|.  The document you are reading
   now uses the |ltxguide| class. The layout for this class is likely
   to change in future releases of \LaTeX.
\item[ltnews]   The document class for the \emph{\LaTeX{} News}
   information sheet, based on |article|. The layout for this class
   is likely to change in future releases of \LaTeX.
\item[minimal]
\NEWfeature{1995/12/01}
   This class is the bare minimum (3 lines) that is needed in a
   \LaTeX\ class file. It just sets the text width and height, and
   defines |\normalsize|.  It is principally intended for debugging
   and testing \LaTeX\ code in situations where you do not need to
   load a `full' class such as |article|. If, however, you are
   designing a completely new class that is aimed for documents with
   structure radically different from the structure supplied by the
   article class, then it may make sense to use this as a base and add
   to it code implementing the required structure, rather than
   starting from |article| and modifying the code there.
\end{description}

\subsection{Standard packages}
\label{Sec:st-pack}

The following packages are distributed with \LaTeX:
\begin{description}
\item[alltt]
\NEWfeature{1994/12/01}
   This package provides the |alltt| environment, which is like
   the |verbatim| environment except that |\|, |{|, and |}|
   have their usual meanings.  It is described in |alltt.dtx| and
   \emph{\LaTeXbook}.
\item[doc] This is the basic package for typesetting the documentation
   of \LaTeX{} programs.  It is described in |doc.dtx| and in
   \emph{\LaTeXcomp}.
\item[exscale]  This provides scaled versions of the math extension
   font.  It is described in |exscale.dtx| and \emph{\LaTeXcomp}.
 \item[fontenc] This is used to specify which font encoding \LaTeX{}
   should use.  It is described in |ltoutenc.dtx|.
\item[graphpap]
\NEWfeature{1994/12/01}
   This package defines the |\graphpaper| command; this
   can be used in a |picture| environment.
\item[ifthen]   Provides commands of the form `if\dots then do\dots
   otherwise do\dots'.
   It is described in |ifthen.dtx| and \emph{\LaTeXcomp}.
\item[inputenc]
\NEWfeature{1994/12/01}
   This is used to specify which input encoding \LaTeX{} should use.
   It is described in |inputenc.dtx|.
\item[latexsym] \LaTeXe{} no longer loads the \LaTeX{} symbol font by
   default.  To access it, you should use the |latexsym| package.  It
   is described in |latexsym.dtx| and in \emph{\LaTeXcomp}; see also
   Section~\ref{Sec:problems}.
 \item[makeidx] This provides commands for producing indexes.  It is
   described in \emph{\LaTeXbook} and in \emph{\LaTeXcomp}.
 \item[newlfont] This is used to emulate the font commands of
   \LaTeX~2.09 with the New Font Selection Scheme. It is described in
   \emph{\LaTeXcomp}.
 \item[oldlfont] This is used to emulate the font commands of
   \LaTeX~2.09.  It is described in \emph{\LaTeXcomp}.
 \item[showidx]
   This causes the argument of each |\index| command to
   be printed on the page where it occurs.
   It is described in \emph{\LaTeXbook}.
 \item[syntonly] This is used to process a document without
   typesetting it.  It is described in |syntonly.dtx| and in
   \emph{\LaTeXcomp}.
 \item[tracefnt] This allows you to control how much information about
   \LaTeX's font loading is displayed.  It is described in
   \emph{\LaTeXcomp}.
\end{description}

\subsection{Related software}

\NEWdescription{1998/12/01}
The following software should be available from the same distributor
as your copy of \LaTeXe.  You should obtain at least the
\textsf{graphics} and \textsf{tools} collections in order to have all
the files described in \emph{\LaTeXbook}.  The |amsmath| package (part
of \textsf{amslatex} and formerly known as |amstex|)
and \textsf{babel} are also mentioned in the list
of `standard packages' in section C.5.2 of that book.
\begin{description}
\item[amslatex]  Advanced mathematical typesetting from the American
  Mathematical Society. This includes the |amsmath| package; it
  provides many commands for typesetting mathematical formulas of
  higher complexity.  It is produced and supported by the American
  Mathematical Society and it is described in \emph{\LaTeXcomp}.
\item[babel]  This package and related files support typesetting in
  many languages. It is described in \emph{\LaTeXcomp}.
\item[cyrillic]
\NEWfeature{1998/12/01}
  Everything you need (except the fonts themselves) for
  typesetting with Cyrillic fonts.
\item[graphics]  This includes the |graphics| package which
  provides support for the inclusion and transformation of graphics,
  including files produced by other software. Also included, is the
  |color| package which provides support for typesetting in colour.
  Both these packages are described in \emph{\LaTeXbook}.
\item[psnfss]    Everything you need (except the fonts themselves) for
  typesetting with a large range of Type~1 (PostScript) fonts.
\item[tools]     Miscellaneous packages written by the \LaTeX3
  project team.
\end{description}
These packages come with documentation and each of them is also
described in at least one of the books \emph{\LaTeXcomp} and
\emph{\LaTeXbook}.

\subsubsection{Tools}

This collection of packages includes, at least, the following (some
files may have slightly different names on certain systems):

\begin{description}
\item[array]
        Extended versions of the environments |array|, |tabular|
        and |tabular*|, with many extra features.
\item[calc]
\NEWfeature{1996/12/01}
        Enables the use of certain algebraic notation when specifying
        values for lengths and counters.
\item[dcolumn]
        Alignment on `decimal points' in tabular entries. Requires the
        |array| package.
\item[delarray]
        Adds `large delimiters' around arrays. Requires |array|.
\item[hhline]
        Finer control over horizontal rules in tables. Requires |array|.
\item[longtable]
        Multi-page tables. (Does not require |array|, but it uses the
        extended features if both are loaded.)
\item[tabularx]
        Defines a |tabularx| environment that is similar to |tabular*|
        but it modifies the column widths, rather than the inter-column
        space, to achieve the desired table width.
\item[afterpage]
        Place text after the current page.
\item[bm]
        Access bold math symbols.
\item[enumerate]
        Extended version of the |enumerate| environment.
\item[fontsmpl]
        Package and test file for producing `font samples'.
\item[ftnright]
        Place all footnotes in the right-hand column in two-column mode.
\item[indentfirst]
        Indent the first paragraph of sections, etc.
\item[layout]
        Show the page layout defined by the current document class.
\item[multicol]
        Typeset text in columns, with the length of the columns
        `balanced'.
\item[rawfonts]
        Preload fonts using the old internal font names of \LaTeX~2.09.
        See Section~\ref{Sec:oldinternals}.
\item[somedefs]
       Selective handling of package options. (Used by the rawfonts
       package.)
\item[showkeys]
        Prints the `keys' used by |\label|, |\ref|, |\cite| etc.; useful
        whilst drafting.
\item[theorem]
       Flexible declaration of `theorem-like' environments.
\item[varioref]
       `Smart' handling of page references.
\item[verbatim]
        Flexible extension of the verbatim environment.
\item[xr]
       Cross reference other `external' documents.
\item[xspace]
       `Smart space' command that helps you to avoid the common mistake
       of missing spaces after command names.
\end{description}


\section{Commands}
\label{Sec:commands}

This section describes the new commands available in \LaTeXe.
They are covered in more detail in \emph{\LaTeXbook} and in
\emph{\LaTeXcomp}.

\subsection{Preamble commands}
\label{Sec:pre}

The changes to the preamble commands are intentionally designed to make
\LaTeXe{} documents look clearly different from old documents.  The
commands should be used only before |\begin{document}|.

\begin{decl}
|\documentclass| \oarg{option-list} \arg{class-name}
   \oarg{release-date}
\end{decl}

This command replaces the
\LaTeX~2.09 command |\documentstyle|.

There must be exactly one |\documentclass| command in a document; and
it should normally come before any other command.  (There are some
exceptions, e.g., you can have |filecontents| environments before it
or |\RequirePackage| but these should be only used in special
scenarios as discussed elsewhere.)

The \m{option-list} is a list of options, each of which may modify the
formatting of elements which are defined in the \m{class-name} file,
as well as those in all following |\usepackage| commands (see
below).

The optional argument \m{release-date} can be used to specify the
earliest desired release date of the class file; it should contain a
date in the format \textsc{yyyy/mm/dd}.  If a version of the class
older than this date is found, a warning is issued.

For example, to specify a two-column article, using a version of
|article.cls| released after June 1994, you specify:
\begin{verbatim}
   \documentclass[twocolumn]{article}[1994/06/01]
\end{verbatim}

\begin{decl}
|\documentstyle| \oarg{option-list} \arg{class-name}
\end{decl}

This command is still supported for compatibility with old files.  It
is essentially the same as |\documentclass| except that it invokes
\emph{\LaTeX~2.09 compatibility mode}.  It also causes any options in
the \m{option-list} that are not processed by the class file to be
loaded as packages after the class has been loaded. See
Section~\ref{Sec:209} for more details on \LaTeX~2.09 compatibility
mode.

\begin{decl}
|\usepackage| \oarg{option-list} \arg{package-name} \oarg{release-date}
\end{decl}

Any number of |\usepackage| commands is allowed. Each package file
(as denoted by \m{package-name}) defines new elements (or modifies
those defined in the class file loaded by the \m{class-name} argument
of the |\documentclass| command).  A package file thus extends the
range of documents which can be processed.

The \m{option-list} argument can contain a list of options, each of
which can modify the formatting of elements which are defined in this
\m{package-name} file.

As above, \m{release-date} can contain the earliest desired release
date of the package file in the format \textsc{yyyy/mm/dd}; if an
older version of the package is found, a warning is issued.

For example, to load the |graphics| package for the |dvips| driver,
using a version of |graphics.sty| released after June 1994, you write:
\begin{verbatim}
   \usepackage[dvips]{graphics}[1994/06/01]
\end{verbatim}
Each package is loaded only once.  If the same package is requested
more than once, nothing happens in the second or following attempt
unless the package has been requested with options that were not given
in the original |\usepackage|. If such extra options are specified
then an error message is produced. See Section~\ref{Sec:problems} how
to resolve this problem.

As well as processing the options given in the \m{option-list} of the
|\usepackage| command, each package processes the
\m{option-list} of the |\documentclass| command as well. This
means that any option which should be processed by every package (to
be precise, by every package that specifies an action for it) can be
specified just once, in the |\documentclass| command, rather than
being repeated for each package that needs it.

\begin{decl}
|\listfiles|
\end{decl}

If this command is placed in the preamble then a list of the files
read in (as a result of processing the document) will be displayed
on the terminal (and in the log file) at the end of the run. Where
possible, a short description will also be produced.

\NEWdescription{1995/12/01}
\emph{Warning}: this command will list only files which were read
using \LaTeX{} commands such as |\input|\arg{file} or
|\include|\arg{file}.  If the file was read using the primitive \TeX{}
syntax |\input |\emph{file} (without |{ }| braces around the file name)
then it will not be listed; failure to use the \LaTeX{} form with the
braces can cause more severe problems, possibly leading to overwriting
important files, so \textbf{always put in the braces}.


\begin{decl}
|\setcounter{errorcontextlines}| \arg{num}
\end{decl}

\TeX~3 introduced a new primitive |\errorcontextlines| which controls
the format of error messages. \LaTeXe\ provides an interface to this
through the standard |\setcounter| command. As most \LaTeX\ users do
not want to see the internal definitions of \LaTeX\ commands each time
they make an error, \LaTeXe{} sets this to $-1$ by default.


\subsection{Environments to write out support files}

\NEWfeature{2019}
%
Until the \LaTeX\ release in 2019 the |filecontents| environment was
restricted to a place before the |\documentclass| command. These days
it can be used anywhere, though we still think that in most cases it is
best to only use it a the top of your document or in the preamble.

\begin{decl}
|\begin{filecontents}| \oarg{option-list} \arg{file-name} \\
  \m{file-contents} \\
|\end{filecontents}|
\end{decl}

The |filecontents| environment is intended for bundling within a
single document file the contents of packages, options, or other
files.  When the document file is run through \LaTeXe{} the body of
this environment is written verbatim (preceded by a comment line) to a
file whose name is given as the environment's only argument.  However,
if that file already exists then nothing happens except for an
information message.

These days most UTF-8 text characters can be used in a
|filecontents| envi\-ronment---they will be written unchanged to the
output file.  However, tabs and form feeds produce a warning,
explaining that they are turned into spaces or blank lines,
respectively.

By default the environment does not overwrite an existing file and it
even refuses to write out the data if there exists a file that is
anywhere in the path that \TeX\ searches when inputting files.  With
the option |nosearch| you can ask it to look only into the current
directory and with the option |overwrite| (or |force|) you can request
it to write the file regardless. It will, however, never write to
|\jobname.tex| to avoid overwriting itself.

The |filecontents| environment is used for including \LaTeX{} files.
For other plain text files (such as Encapsulated PostScript files),
you should use the |filecontents*| environment which does not add a
comment line.




\subsection{Document structure}

The |book| document class introduces new commands to indicate
document structure.
\begin{decl}
|\frontmatter| \\ |\mainmatter| \\ |\backmatter|
\end{decl}
These commands indicate the beginning of the front matter (title page,
table of contents and prefaces), main matter (main text) and back
matter (bibliography, indexes and colophon).

\subsection{Definitions}

In \LaTeX, commands can have both mandatory and optional arguments,
for example in:
\begin{verbatim}
   \documentclass[11pt]{article}
\end{verbatim}
the |11pt| argument is optional, whereas the |article| class name is
mandatory.

In \LaTeX~2.09 users could define commands with arguments, but these
had to be mandatory arguments.  With \LaTeXe, users can now define
commands and environments which also have one optional argument.

\begin{decl}
|\newcommand| \arg{cmd} \oarg{num} \oarg{default} \arg{definition} \\
|\newcommand*| \arg{cmd} \oarg{num} \oarg{default} \arg{definition} \\
|\renewcommand| \arg{cmd} \oarg{num} \oarg{default} \arg{definition} \\
|\renewcommand*| \arg{cmd} \oarg{num} \oarg{default} \arg{definition}
\end{decl}

These commands have a new, second, optional argument; this is used for
defining commands which themselves take one optional argument.  This
new argument is best introduced by means of a simple (and hence not
very practical) example:
\begin{verbatim}
   \newcommand{\example}[2][YYY]{Mandatory arg: #2;
                                 Optional arg: #1.}
\end{verbatim}
This defines |\example| to be a command with two arguments, referred
to as |#1| and |#2| in the \arg{definition}---nothing new so far.  But
by adding a second optional argument to this |\newcommand| (the
|[YYY]|) the first argument (|#1|) of the newly defined command
|\example| is made optional with its default value being |YYY|.

Thus the usage of |\example| is either:
\begin{verbatim}
   \example{BBB}
\end{verbatim}
which prints:
\begin{quote}
   Mandatory arg: BBB;
   Optional arg: YYY.
\end{quote}
or:
\begin{verbatim}
   \example[XXX]{AAA}
\end{verbatim}
which prints:
\begin{quote}
   Mandatory arg: AAA;
   Optional arg: XXX.
\end{quote}

The default value of the optional argument is \texttt{YYY}.
This value is specified as the \oarg{default} argument of the
|\newcommand| that created |\example|.

As another more useful example, the definition:
\begin{verbatim}
   \newcommand{\seq}[2][n]{\lbrace #2_{0},\ldots,\,#2_{#1} \rbrace}
\end{verbatim}
means that the input |$\seq{a}$| produces
the formula $\lbrace a_{0},\ldots,\,a_{n} \rbrace$,
whereas the input |$\seq[k]{x}$| produces the formula
$\lbrace x_{0},\ldots,\,x_{k} \rbrace$.

In summary, the command:
\begin{quote}
   |\newcommand| \arg{cmd} \oarg{num} \oarg{default} \arg{definition}
\end{quote}
defines \m{cmd} to be a command with \m{num} arguments, the first of
which is optional and has default value \m{default}.

Note that there can only be one optional argument but, as before,
there can be up to nine arguments in total.

\begin{decl}
|\newenvironment|
 \arg{cmd} \oarg{num} \oarg{default} \arg{beg-def} \arg{end-def} \\
|\newenvironment*|
 \arg{cmd} \oarg{num} \oarg{default} \arg{beg-def} \arg{end-def} \\
|\renewenvironment|
 \arg{cmd} \oarg{num} \oarg{default} \arg{beg-def} \arg{end-def} \\
|\renewenvironment*|
 \arg{cmd} \oarg{num} \oarg{default} \arg{beg-def} \arg{end-def}
\end{decl}

\LaTeXe\ also supports the creation of environments that have one
optional argument.  Thus the syntax of these two commands has
been extended in the same way as that of |\newcommand|.

\begin{decl}
|\providecommand| \arg{cmd} \oarg{num} \oarg{default} \arg{definition} \\
|\providecommand*| \arg{cmd} \oarg{num} \oarg{default} \arg{definition}
\end{decl}

This takes the same arguments as |\newcommand|. If \m{cmd} is already
defined then the existing definition is kept; but if it is currently
undefined then the effect of |\providecommand| is to define \m{cmd}
just as if |\newcommand| had been used.

\NEWfeature{1994/12/01}
  All the above five `defining commands' now have \texttt{*}-forms that
  are usually the better form to use when defining commands with
  arguments, unless any of these arguments is intended to contain
  whole paragraphs of text.  Moreover, if you ever do find yourself
  needing to use the non-star form then you should ask whether
  that argument would not better be treated as the contents of a
  suitably defined environment.

\NEWfeature{1995/12/01}
  The commands produced by the above five `defining commands' are
  now robust.

\subsection{Boxes}

These next three commands for making LR-boxes all existed in
\LaTeX~2.09.  They have been enhanced in two ways.

\begin{decl}
|\makebox| \oarg{width} \oarg{pos}  \arg{text} \\
|\framebox| \oarg{width} \oarg{pos}  \arg{text} \\
|\savebox| \arg{cmd} \oarg{width} \oarg{pos}  \arg{text}
\end{decl}

One small but far-reaching change for \LaTeXe\ is that, within the
\m{width} argument only, four special lengths can be used.  These are
all dimensions of the box that would be produced by using simply
|\mbox|\arg{text}:
\begin{itemize}
\item []   |\height|\quad its height above the baseline;
\item []   |\depth|\quad its depth below the baseline;
\item []   |\totalheight|\quad the sum of |\height| and |\depth|;
\item []   |\width|\quad its width.
\end{itemize}
Thus, to put `hello' in the centre of a box of twice its natural
width, you would use:
\begin{verbatim}
   \makebox[2\width]{hello}
\end{verbatim}
Or you could put \textit{f} into a square box, like
this:
   \framebox{\makebox[\totalheight]{\itshape f\/}}
\begin{verbatim}
   \framebox{\makebox[\totalheight]{\itshape f\/}}
\end{verbatim}
Note that it is the total width of the framed box, including the
frame, which is set to |\totalheight|.

The other change is a new possibility for \m{pos}: |s| has been added
to |l| and |r|.  If \m{pos} is |s| then the text is stretched the full
length of the box, making use of any `rubber lengths' (including any
inter-word spaces) in the contents of the box.  If no such `rubber
length' is present, an `underfull box' will probably be produced.

\begin{decl}
|\parbox| \oarg{pos} \oarg{height} \oarg{inner-pos} \arg{width}
         \arg{text} \\
|\begin{minipage}|
         \oarg{pos} \oarg{height} \oarg{inner-pos} \arg{width}\\
\m{text}\\
|\end{minipage}|
\end{decl}

As for the box commands above, |\height|, |\width|, etc.~may be used
in the \oarg{height} argument to denote the natural dimensions of the
box.

The \m{inner-pos} argument is new in \LaTeXe.  It is the vertical
equivalent to the \m{pos} argument for |\makebox|, etc, determining
the position of \m{text} within the box.  The \m{inner-pos} may be any
one of |t|, |b|, |c|, or |s|, denoting top, bottom, centered, or
`stretched' alignment respectively.  When the \m{inner-pos} argument
is not specified, \LaTeX\ gives it same  value as \m{pos} (this could be
the latter's default value).


\begin{decl}
|\begin{lrbox}| \arg{cmd}\\
\m{text}\\
|\end{lrbox}|
\end{decl}

This is an environment which does not directly print anything.
Its effect is to save the typeset \m{text} in the bin \m{cmd}. Thus
it is like |\sbox| \arg{cmd} \arg{text}, except that any white space
before or after the contents \m{text} is ignored.

This is very useful as it enables both the |\verb| command and the
\texttt{verbatim} environment to be used within \m{text}.

It also makes it possible to define, for example, a `framed box'
environment.  This is done by first using this environment to save
some text in a bin \m{cmd} and then calling
|\fbox{\usebox{|\m{cmd}|}}|.

The following example defines an environment, called |fmpage|, that is
a framed version of |minipage|.
\begin{verbatim}
   \newsavebox{\fmbox}
   \newenvironment{fmpage}[1]
     {\begin{lrbox}{\fmbox}\begin{minipage}{#1}}
     {\end{minipage}\end{lrbox}\fbox{\usebox{\fmbox}}}
\end{verbatim}


\subsection{Measuring things}

The first of these next commands was in \LaTeX~2.09. The two new
commands are the obvious analogues.

\begin{decl}
|\settowidth|  \arg{length-cmd} \arg{lr text} \\
|\settoheight| \arg{length-cmd} \arg{lr text} \\
|\settodepth|  \arg{length-cmd} \arg{lr text}
\end{decl}

\subsection{Line endings}

\NEWdescription{1994/12/01}
The command |\\|, which is used to indicate a line-end in various
places, is now a robust command when used within arguments such as
section titles.

Also, because it is often necessary to distinguish which type of line
is to be ended, we have introduced the following new command; it
has the same argument syntax as that of |\\|.
\begin{decl}[1994/12/01]
|\tabularnewline| \oarg{vertical-space}
\end{decl}
One example of its use is when the text in the last
column of a |tabular| environment is set with |\raggedright|; then
|\tabularnewline| can be used to indicate the end of a row of the
|tabular|, whilst |\\| will indicate the end of a line of text in a
paragraph within the column. This command can be used in the |array|
environment as well as |tabular|, and also the extended versions of
these environments offered by the \textsf{array} and \textsf{longtable}
packages in the tools collection.

\subsection{Controlling page breaks}

Sometimes it is necessary, for a final version of a document, to
`help' \LaTeX\ break the pages in the best way. \LaTeX~2.09 had a
variety of commands for this situation: |\clearpage|, |\pagebreak|
etc.  \LaTeXe\ provides, in addition, commands which can produce
longer pages as well as shorter ones.


\begin{decl}
|\enlargethispage| \arg{size} \\
|\enlargethispage*| \arg{size}
\end{decl}

These commands increase the height of a page (from its normal value of
|\textheight|) by the specified amount \m{size}, a rigid length. This
change affects \emph{only} the current page.

This can be used, for example, to allow an extra line to be fitted
onto the page or, with a negative length, to produce a page
shorter than normal.

The star form also shrinks any vertical white space on the page as
much as possible, so as to fit the maximum amount of text on the
page.

\NEWdescription{1995/12/01}
These commands do not change the position of the footer text; thus, if
a page is lengthened too far, the main text may overprint the footer.

\subsection{Floats}

There is a new command, |\suppressfloats|, and a new `float
specifier'.  These will enable people to gain better
control of \LaTeX's float placement algorithm.

\begin{decl}
|\suppressfloats| \oarg{placement}
\end{decl}

This command stops any further floating environments from being placed
on the current page. With an optional argument, which should be either
|t| or |b| (not both), this restriction applies only to putting
further floats at the top or at the bottom.  Any floats which would
normally be placed on this page are placed on the next page instead.

\begin{decl}
The extra float location specifier: \ \texttt{!}
\end{decl}

This can be used, along with at least one of \texttt{h}, \texttt{t},
\texttt{b} and \texttt{p}, in the location optional argument of a
float.

If a \texttt{!} is present then, just for this particular float,
whenever it is processed by the float mechanism the following are
ignored:
\begin{itemize}
\item  all restrictions on the number of floats which can appear;
\item all explicit restrictions on the amount of space on a text page
which may be occupied by floats or must be occupied by text.
\end{itemize}
The mechanism will, however, still attempt to ensure that pages are
not overfull and that floats of the same type are printed in the
correct order.

Note that its presence has no effect on the production of float pages.

A \texttt{!} specifier overrides the effect of any |\suppressfloats|
command for this particular float.

\subsection{Font changing: text}

The font selection scheme used in \LaTeXe{} differs a lot from that
used in \LaTeX~2.09.  In this section, we give a brief description of
the new commands.  A more detailed description with examples is given
in \emph{\LaTeXcomp}, and the interface for class- and package-writers
is described in \emph{\fntguide}.

\begin{decl}
|\normalfont|\\
|\rmfamily|\\
|\sffamily|\\
|\ttfamily|\\
|\mdseries|\\
|\bfseries|\\
|\upshape|\\
|\itshape|\\
|\slshape|\\
|\scshape|
\end{decl}

These are font commands whose use is the same as the commands |\rm|,
|\bf|, etc.  The difference is that each command changes just one
attribute of the font (the attribute changed is part of the name).
One result of this is that, for example, |\bfseries\itshape| produces
both a change of series and a change of shape, to give a bold italic
font.

\begin{decl}
|\textnormal|\arg{text}\\
|\textrm|\arg{text}\\
|\textsf|\arg{text}\\
|\texttt|\arg{text}\\
|\textmd|\arg{text}\\
|\textbf|\arg{text}\\
|\textup|\arg{text}\\
|\textit|\arg{text}\\
|\textsl|\arg{text}\\
|\textsc|\arg{text}\\
|\emph|\arg{text}
\end{decl}

These are one-argument commands; they take as an argument the
text which is to be typeset in the particular font. They also
automatically insert italic corrections where appropriate; if you do
not like the result, you can add an italic correction with |\/| or
remove it with |\nocorr|.  The |\nocorr| should always be the first or
last thing within the \arg{text} argument.

\subsection{Font changing: math}

Most of the fonts used within math mode do not need to be explicitly
invoked; but to use letters from a range of fonts, the following
class of commands is provided.

\begin{decl}
|\mathrm| \arg{letters}\\
|\mathnormal| \arg{letters}\\
|\mathcal| \arg{letters}\\
|\mathbf| \arg{letters}\\
|\mathsf| \arg{letters}\\
|\mathtt| \arg{letters}\\
|\mathit| \arg{letters}
\end{decl}

These are also one-argument commands which take as an argument the
letters which are to be typeset in the particular font.  The argument
is processed in math mode so spaces within it will be ignored.  Only
letters, digits and accents have their font changed, for example
|$\mathbf{\tilde A \times 1}$| produces $\mathbf{\tilde A \times 1}$.

\subsection{Ensuring math mode}

\begin{decl}
|\ensuremath| \arg{math commands}
\end{decl}

In \LaTeX~2.09, if you wanted a command to work both in math mode and
in text mode, the suggested method was to define something like:
\begin{verbatim}
   \newcommand{\Gp}{\mbox{$G_p$}}
\end{verbatim}
Unfortunately, the |\mbox| stops |\Gp| changing size correctly in (for
instance) subscripts or a fraction.

In \LaTeXe{} you can define it thus:
\begin{verbatim}
   \newcommand{\Gp}{\ensuremath{G_p}}
\end{verbatim}
Now |\Gp| will work correctly in all contexts.

This is because the |\ensuremath| does nothing, producing simply
|G_p|, when |\Gp| is used within math mode; but it ensures that math
mode is entered (and exited) as required when |\Gp| is used in text
mode.


\subsection{Setting text superscripts}

\begin{decl}
|\textsuperscript| \arg{text}
\end{decl}

\NEWfeature{1995/06/01} In \LaTeX~2.09 textual superscripts such as
footnote markers were produced by internally entering math mode and
typesetting the number as a math superscript.  This normally looked
fine since the digits in math fonts are the same as those in text
fonts when Computer Modern fonts are used.  But when a different
document font (such as Times) is selected, the results look rather
strange.  For this reason the command |\textsuperscript| has been
introduced which typesets its argument in the current text font, in a
superscript position and in the correct size.

\subsection{Text commands: all encodings}

\NEWdescription{1994/12/01}
  One of the main differences between \LaTeXe{} and \LaTeX~2.09 is
  that \LaTeXe{} can deal with fonts in arbitrary \emph{encodings}.
  (A font encoding is the sequence of characters in the font---for
  example a Cyrillic font would have a different encoding from a Greek
  font.)

  The two major font encodings that are used for Latin languages such
  as English or German are |OT1| (Donald Knuth's 7-bit encoding, which
  has been used during most of \TeX's lifetime) and |T1| (the new
  8-bit `Cork' encoding).

  \LaTeX~2.09 only supported the |OT1| encoding, whereas \LaTeXe{} has
  support for both |OT1| and |T1| built-in.  The next section will
  cover the new commands which are available if you have |T1|-encoded
  fonts.  This section describes new commands which are available in
  all encodings.

  Most of these commands provide characters which were available in
  \LaTeX~2.09 already.  For example |\textemdash| gives an `em dash',
  which was available in \LaTeX~2.09 by typing |---|.  However, some
  fonts (for example a Greek font) may not have the |---| ligature,
  but you will still be able to access an em dash by typing
  |\textemdash|.

\begin{decl}[1994/12/01]
      |\r{<text>}|
\end{decl}
   This command gives a `ring' accent, for example `\r{o}' can be typed
   |\r{o}|.

\begin{decl}[1994/12/01]
      |\SS|
\end{decl}
  This command produces a German `SS', that is a capital `\ss'.  This
  letter can hyphenate differently from `SS', so is needed for entering
  all-caps German.

\begin{decl}[1994/12/01]
      |\textcircled{<text>}|
\end{decl}
   This command is used to build `circled characters' such as
   |\copyright|.  For example |\textcircled{a}| produces
   \textcircled{a}.

\begin{decl}[1994/12/01]
      |\textcompwordmark|
\end{decl}
  This command is used to separate letters which would normally
  ligature.  For example `f\textcompwordmark i' is produced with
  |f\textcompwordmark i|.  Note that the `f' and `i' have not
  ligatured to produce `fi'.  This is rarely useful in English
  (`shelf\textcompwordmark ful' is a rare example of where it might be
  used) but is used in languages such as German.

\begin{decl}[1994/12/01]
      |\textvisiblespace|
\end{decl}
   This command produces a `visible space' character
   `\textvisiblespace'.  This is sometimes used in computer listings,
   for example `type \textsf{hello\textvisiblespace world}'.

\begin{decl}[1994/12/01]
      |\textemdash|
      |\textendash|
      |\textexclamdown|
      |\textquestiondown| \\
      |\textquotedblleft|
      |\textquotedblright|
      |\textquoteleft|
      |\textquoteright|
\end{decl}
   These commands produce characters which would otherwise be
   accessed via ligatures:
   \begin{center}
      \begin{tabular}{ccl}
         \emph{ligature} & \emph{character} & \emph{command} \\
         |---| & --- & |\textemdash| \\
         |--|  & --  & |\textendash| \\
         |!`|  & !`  & |\textexclamdown| \\
         |?`|  & ?`  & |\textquestiondown| \\
         |``|  & ``  & |\textquotedblleft| \\
         |''|  & ''  & |\textquotedblright| \\
         |`|   & `   & |\textquoteleft| \\
         |'|   & '   & |\textquoteright|
      \end{tabular}
   \end{center}
   The reason for making these characters directly accessible is so
   that they will work in encodings which do not have these characters.

\begin{decl}[1994/12/01]
      |\textbullet|
      |\textperiodcentered|
\end{decl}
   These commands allow access to characters which were previously only
   available in math mode:
   \begin{center}
      \begin{tabular}{lcl}
         \emph{math command} & \emph{character} & \emph{text command} \\
          |\bullet|   & $\bullet$   & |\textbullet| \\
         |\cdot|     & $\cdot$     & |\textperiodcentered|
      \end{tabular}
   \end{center}

\begin{decl}[1995/12/01]
      |\textbackslash|
      |\textbar|
      |\textless|
      |\textgreater|
\end{decl}
   These commands allow access to ASCII characters which were
   only available in verbatim or math mode:
   \begin{center}
      \begin{tabular}{lcl}
         \emph{math command} & \emph{character} & \emph{text command} \\
          |\backslash|   & $\backslash$   & |\textbackslash| \\
          |\mid|     & $\mid$     & |\textbar| \\
          |<<| & $<$ & |\textless| \\
          |>>| & $>$ & |\textgreater|
      \end{tabular}
   \end{center}

\begin{decl}[1995/12/01]
      |\textasciicircum|
      |\textasciitilde|
\end{decl}
   These commands allow access to ASCII characters which were
   previously only available in verbatim:
   \begin{center}
      \begin{tabular}{cl}
         \emph{verbatim} & \emph{text command} \\
         |^| & |\textasciicircum| \\
         |~| & |\textasciitilde|
      \end{tabular}
   \end{center}

\begin{decl}[1995/12/01]
      |\textregistered|
      |\texttrademark|
\end{decl}
   These commands provide the `registered trademark' (R) and
   `trademark' (TM) symbols.

\subsection{Text commands: the T1 encoding}

\NEWdescription{1994/12/01}
  The |OT1| font encoding is fine for typesetting in English, but has
  problems when typesetting other languages.  The |T1| encoding solves
  some of these problems, by providing extra characters (such as `eth'
  and `thorn'), and it allows words containing accented letters to be
  hyphenated (as long as you have a package like |babel| which allows
  for non-American hyphenation).

  This section describes the commands you can use if you have the |T1|
  fonts.  To use them, you need to get the `ec fonts', or the
  |T1|-encoded PostScript fonts, as used by \textsf{psnfss}.
  All these fonts are
  available by anonymous ftp in the Comprehensive \TeX{} Archive, and
  are also available on the CD-ROMs \emph{4all \TeX} and
  \emph{\TeX{} Live} (both available from the \TeX{} Users Group).

   You can then select the |T1| fonts by saying:
\begin{verbatim}
   \usepackage[T1]{fontenc}
\end{verbatim}
   This will allow you to use the commands in this section.

   \emph{Note:} Since this document must be processable on any site
   running an up-to-date \LaTeX, it does not contain any characters
   that are present only in |T1|-encoded fonts.  This means that this
   document cannot show you what these glyphs look like!  If you want
   to see them then run \LaTeX{} on the document |fontsmpl| and
   respond `|cmr|' when it prompts you for a family name.

\begin{decl}[1994/12/01]
      |\k{<text>}|
\end{decl}
   This command produces an `ogonek' accent.

\begin{decl}[1994/12/01]
      |\DH|
      |\DJ|
      |\NG|
      |\TH|
      |\dh|
      |\dj|
      |\ng|
      |\th|
\end{decl}
   These commands produce characters `eth', `dbar', `eng', and `thorn'.

\begin{decl}[1994/12/01]
      |\guillemotleft|
      |\guillemotright|
      |\guilsinglleft|
      |\guilsinglright| \\
      |\quotedblbase|
      |\quotesinglbase|
      |\textquotedbl|
\end{decl}
   % A local hack (could be improved):
   \newcommand{\fauxguillemet}[1]{$\vcenter{\hbox{$\scriptscriptstyle#1$}}$}
   These commands produce various sorts of quotation mark.
   Rough representations of them are:
   \fauxguillemet\ll a\fauxguillemet\gg{}
   \fauxguillemet<a\fauxguillemet>
   ,\kern -0.1em,\kern 0.05em a\kern -0.05em``
   ,\kern 0.05em a\kern -0.05em` and |"|a|"|.

\NEWdescription{2001/06/01}
   There are therefore some extra short-form ligatures available for
   use in documents that will only be used with |T1|-encoded fonts.

   The guillemets |\guillemotleft| and |\guillemotright|%
   \footnote{We apologise once again for maintaining Adobe's
     enormous solipsism~(sic) of confusing a diving bird with
     punctuation marks!}
   can be obtained by typing |<<<<| and |>>>>| and |\quotedblbase|
   by typing |,,|\,.

   Also, unlike the unexpected results with
   |OT1|-encoded fonts, |<<| and |>>| will produce \textless{} and
   \textgreater{}.

   Note also that the single character |"| will no longer produce ''
   but rather |\textquotedbl|.

\subsection{Logos}

\begin{decl}
|\LaTeX|\\
|\LaTeXe|
\end{decl}

|\LaTeX| (producing `\LaTeX') is still the `main' logo command,
but if you need to refer to the new features, you can write
|\LaTeXe| (producing `\LaTeXe').

\subsection{Picture commands}

\begin{decl}
   |\qbezier[<N>](<AX>,<AY>)(<BX>,<BY>)(<CX>,<CY>)| \\
   | \bezier{<N>}(<AX>,<AY>)(<BX>,<BY>)(<CX>,<CY>)|
\end{decl}
The |\qbezier| command can be used in |picture| mode to draw a
quadratic Bezier curve from position |(<AX>,<AY>)| to |(<CX>,<CY>)| with
control point |(<BX>,<BY>)|.  The optional argument \m{N} gives the
number of points on the curve.

For example, the diagram:
\begin{center}
   \begin{picture}(50,50)
      \thicklines
      \qbezier(0,0)(0,50)(50,50)
      \qbezier[20](0,0)(50,0)(50,50)
      \thinlines
      \put(0,0){\line(1,1){50}}
   \end{picture}
\end{center}
is drawn with:
\begin{verbatim}
   \begin{picture}(50,50)
      \thicklines
      \qbezier(0,0)(0,50)(50,50)
      \qbezier[20](0,0)(50,0)(50,50)
      \thinlines
      \put(0,0){\line(1,1){50}}
   \end{picture}
\end{verbatim}
The |\bezier| command is the same, except that the argument \m{N} is not
optional.  It is provided for compatibility with the \LaTeX~2.09
|bezier| document style option.

\subsection{Old commands}

\begin{decl}
|\samepage|
\end{decl}

The |\samepage| command still exists but is no longer being
maintained.
This is because it only ever worked erratically; it does not
guarantee that there will be no page-breaks within its scope; and
it can cause footnotes and marginals to be wrongly placed.

We recommend using |\enlargethispage| in conjunction with page-break
commands such as |\newpage| and |\pagebreak| to help control page
breaks.

\begin{decl}
   |\SLiTeX|
\end{decl}
Since \SLiTeX{} no longer exists, the logo is no longer defined in the
\LaTeX{} kernel.  A suitable replacement is |\textsc{Sli\TeX}|.  The
\SLiTeX{} logo is defined in \LaTeX~2.09 compatibility mode.

\begin{decl}
|\mho| |\Join| |\Box| |\Diamond| |\leadsto| \\
|\sqsubset| |\sqsupset| |\lhd| |\unlhd| |\rhd| |\unrhd|
\end{decl}

These symbols are contained in the \LaTeX{} symbol font, which was
automatically loaded by \LaTeX~2.09.  However, \TeX{} has room for
only sixteen math font families; thus many users discovered that they
ran out.  Because of this, \LaTeX{} does not load the \LaTeX{} symbol
font unless you use the \textsf{latexsym} package.

These symbols are also made available, using different fonts, by the
\textsf{amsfonts} package, which also defines a large number of other
symbols.  It is supplied by the American Mathematical Society.

The \textsf{latexsym} package is loaded automatically in \LaTeX~2.09
compatibility mode.

\section{\LaTeX~2.09 documents}
\label{Sec:209}

\LaTeXe{} can process (almost) any \LaTeX~2.09 document, by entering
\emph{\LaTeX~2.09 compatibility mode}.  Nothing has changed, you run
\LaTeX{} in the same way you always did, and you will get much the
same results.

The reason for the `almost' is that some \LaTeX~2.09 packages made use
of low-level unsupported features of \LaTeX.  If you discover such a
package, you should find out if it has been updated to work with
\LaTeXe.  Most packages will still work with \LaTeXe---the easiest way
to find out whether a package still works is to try it!

\LaTeX~2.09 compatibility mode is a comprehensive emulation of
\LaTeX~2.09, but at the cost of time.  Documents can run up to 50\%
slower in compatibility mode than they did under \LaTeX~2.09.
In addition, many of the new features of \LaTeXe{} are not available in
\LaTeX~2.09 compatibility mode.

\subsection{Warning}

\NEWdescription{1995/12/01}
This \emph{\LaTeX~2.09 compatibility mode} is provided solely to allow
you to process 2.09 documents, i.e.~documents that were written (we
hope, a long time ago) for a very old system and therefore could be
processed by using a genuine antique \LaTeX~2.09 system.

This mode is therefore \emph{not} intended to provide access to the
enhanced features of \LaTeXe{}.  Thus it must not be used to process
new documents which masquerade as 2.09 documents (i.e.~they begin with
|\documentstyle|) but which could not be processed using that genuine
antique \LaTeX~2.09 system because they contain some new,
\LaTeXe{}-only, commands or environments.

To prevent such misuse of the system, and the consequent trouble it
causes when such misleadingly encoded documents are distributed, the
\emph{\LaTeX~2.09 compatibility mode} turns off most of these new
features and commands.  Any attempt to use them will give you an error
message and, moreover, many of them simply  will not work, whilst
others will produce unpredictable results.
So don't bother sending us any bug reports about such occurrences since
they are intentional.


\subsection{Font selection problems}
\label{Sec:fsprob}

When using compatibility mode, it is possible that you will find
problems with font-changing commands in some old documents.  These
problems are of two types:

\begin{itemize}
\item producing error messages;
\item not producing the font changes you expected.
\end{itemize}

In case of error messages it is possible that the document (or an old
style file used therein) contains references to old internal commands
which are no longer defined, see Section~\ref{Sec:oldinternals} for
more information if this is the case.

\NEWdescription{1995/12/01}
One example of the unexpected is if you use one of the new style of
math-mode font changing command as follows:
\begin{verbatim}
$ \mathbf{xy} A $
\end{verbatim}
You may well find that this behaves as if you had put:
\begin{verbatim}
$ \bf {xy} A $
\end{verbatim}
everything including the $A$ coming out bold.

\LaTeX~2.09 allowed sites to customize their \LaTeX{} installation,
which resulted in documents producing different results on different
\LaTeX{} installations.  \LaTeXe{} no longer allows so much
customization but, for compatibility with old documents, the local
configuration file |latex209.cfg| is loaded every time \LaTeXe{}
enters \LaTeX~2.09 compatibility mode.

For example, if your site was customized to use the New Font
Selection Scheme (\NFSS) with the |oldlfont| option, then you can
make \LaTeXe{} emulate this by creating a |latex209.cfg| file
containing the commands:
\begin{verbatim}
\ExecuteOptions{oldlfont}\RequirePackage{oldlfont}
\end{verbatim}
Similarly, to emulate \NFSS{} with the |newlfont| option, you can
create a |latex209.cfg| file containing:
\begin{verbatim}
\ExecuteOptions{newlfont}\RequirePackage{newlfont}
\end{verbatim}


\subsection{Native mode}
\label{Sec:native}

To run an old document faster, and use the new features of
\LaTeXe, you should try using \emph{\LaTeXe{} native mode}.
This is done by replacing the command:
\begin{quote}
   |\documentstyle[|\m{options}|,|\m{packages}|]|\arg{class}
\end{quote}
with:
\begin{quote}
   |\documentclass|\oarg{options}\arg{class} \\
   |\usepackage{latexsym,|\m{packages}|}|
\end{quote}
However, some documents which can be processed in \LaTeX~2.09
compatibility mode may not work in native mode.  Some \LaTeX~2.09
packages will only work with \LaTeXe{} in 2.09 compatibility mode.
Some documents will cause errors because of \LaTeXe's improved error
detection abilities.

But most \LaTeX~2.09 documents can be processed by \LaTeXe{}'s native
mode with the above change.  Again, the easiest way to find out
whether your documents can be processed in native mode is to try it!


\section{Local modifications}
\label{sec:loc}

\NEWdescription{1995/12/01}
There are two common types of local modifications that can be done
very simply.  Do not forget that documents produced using such
modifications will not be usable at other places (such documents are
called `non-portable').

One type of modification is the use of personal commands for commonly
used symbols or constructions.  These should be put into a package
file (for example, one called \texttt{mymacros.sty}) and loaded by
putting |\usepackage{mymacros}| in the document preamble.

Another type is a local document class that is very similar to one of
the standard classes but contains some straightforward modifications
such as extra environments, different values for some parameters, etc.
These should be put into a class file; here we shall describe a simple
method of constructing such a file using, as an example, a class
called \textsf{larticle} that is very similar to the \textsf{article}
class.

\NEWfeature{1995/12/01}
The class file called \texttt{larticle.cls} should (after the
preliminary identification commands) start as follows:
\begin{verbatim}
   \LoadClassWithOptions{article}
\end{verbatim}
This command should be followed by whatever additions and changes you
wish to make to the results of reading in the file
\texttt{article.sty}.

The effect of using the above |\LoadClassWithOptions| command is to
load the standard class file \textsf{article} with whatever options
are asked for by the document.  Thus a document using your
\textsf{larticle} class can specify any option that could be specified
when using the standard \textsf{article} class; for example:
\begin{verbatim}
   \documentclass[a4paper,twocolumn,dvips]{larticle}
\end{verbatim}


\section{Problems}
\label{Sec:problems}

This section describes some of the things which may go wrong when
using \LaTeXe, and what you can do about it.

\subsection{New error messages}

\LaTeXe{} has a number of new error messages.
Please also note that many error messages now produce further helpful
information if you press |h| in response to the error prompt.

\begin{decl}
|Option clash for package |\m{package}|.|
\end{decl}
The named package has been loaded twice with different options.  If
you enter |h| you will be told what the options were, for example, if
your document contained:
\begin{verbatim}
   \usepackage[foo]{fred}
   \usepackage[baz]{fred}
\end{verbatim}
then you will get the error message:
\begin{verbatim}
   Option clash for package fred.
\end{verbatim}
and typing |h| at the |?| prompt will give you:
\begin{verbatim}
   The package fred has already been loaded with options:
     [foo]
   There has now been an attempt to load it with options:
     [baz]
   Adding the line:
     \usepackage[foo,baz]{fred}
   to your document may fix this.
   Try typing <<return>> to proceed.
\end{verbatim}
The cure is, as suggested, to load the package with both sets of
options.  Note that since \LaTeX{} packages can call other packages,
it is possible to get a package option clash without explicitly
requesting the same package twice.

\begin{decl}
   |Command |\m{command}| not provided in base NFSS.|
\end{decl}
The \m{command} is not provided by default in \LaTeXe.  This error is
generated by using one of the commands:
\begin{verbatim}
   \mho \Join \Box \Diamond \leadsto
   \sqsubset \sqsupset \lhd \unlhd \rhd \unrhd
\end{verbatim}
which are now part of the \textsf{latexsym} package.
The cure is to add:
\begin{verbatim}
   \usepackage{latexsym}
\end{verbatim}
in the preamble of your document.

\begin{decl}
   |LaTeX2e command <command> in LaTeX 2.09 document.|
\end{decl}
The \m{command} is a \LaTeXe{} command but this is a \LaTeX~2.09
document.  The cure is to replace the command by a \LaTeX~2.09
command, or to run document in native mode, as described in
Section~\ref{Sec:native}.

\begin{decl}
   |NFSS release 1 command \newmathalphabet found.|
\end{decl}
The command |\newmathalphabet| was used by the New Font Selection
Scheme Release 1 but it has now been replaced by
|\DeclareMathAlphabet|, the use of which is described in
\emph{\fntguide}.

The best cure is to update the package which contained the
|\newmathalphabet| command.  Find out if there is a new release of
the package, or (if you wrote the package yourself) consult
\emph{\fntguide} for the new syntax of font commands.

If there is no updated version of the package then you can cure this
error by using the \textsf{newlfont} or \textsf{oldlfont} package,
which tells \LaTeX{} which version of |\newmathalphabet| should be
emulated.

You should use \textsf{oldlfont} if the document selects math fonts with
syntax such as this:
\begin{quote}
  |{\cal A}|, etc.
\end{quote}
Use \textsf{newlfont} if the document's syntax is like this:
\begin{quote}
  |\cal{A}|, etc.
\end{quote}


\begin{decl}
   |Text for \verb command ended by end of line.|
\end{decl}
The |\verb| command has been begun but not ended on that line.  This
usually means that you have forgotten to put in the end-character of
the |\verb| command.

\begin{decl}
   |Illegal use of \verb command.|
\end{decl}
The |\verb| command has been used inside the argument of another
command.  This has never been allowed in \LaTeX{}---often producing
incorrect output without any warning---and so \LaTeXe{} produces
an error message.

\subsection{Old internal commands}
\label{Sec:oldinternals}

A number of \LaTeX~2.09 internal commands have been removed, since
their functionality is now provided in a different way.  See
\emph{\clsguide} for more details of the new, supported interface for
class and package writers.

\begin{decl}
   |\tenrm| |\elvrm| |\twlrm| \dots\\
   |\tenbf| |\elvbf| |\twlbf| \dots\\
   |\tensf| |\elvsf| |\twlsf| \dots\\
   $\vdots$
\end{decl}
These commands provided access to the seventy fonts preloaded by
\LaTeX~2.09.  In contrast, \LaTeXe{} normally preloads at most
fourteen fonts, which saves a lot of font memory; but a consequence is
that any \LaTeX{} file which used the above commands to directly
access fonts will no longer work.

Their use will usually produce an error message such as:
\begin{verbatim}
   ! Undefined control sequence.
   l.5 \tenrm
\end{verbatim}
The cure for this is to update the document to use the new
font-changing commands provided by \LaTeXe; these are described in
\emph{\fntguide}.

If this is not possible then, as a last resort, you can
use the \textsf{rawfonts} package, which loads the
seventy \LaTeX~2.09 fonts and provides direct access to them using the
old commands.  This takes both time and memory.  If you do not
wish to load all seventy fonts, you can select some of them by using the
|only| option to \textsf{rawfonts}.  For example, to load only |tenrm|
and |tenbf| you write:
\begin{verbatim}
   \usepackage[only,tenrm,tenbf]{rawfonts}
\end{verbatim}

The \textsf{rawfonts} package is distributed as part of the \LaTeX{}
tools software, see Section~\ref{Sec:st-pack}.

\subsection{Old files}

One of the more common mistakes in running \LaTeX{} is to read in old
versions of packages instead of the new versions.  If you get an
incomprehensible error message from a standard package, make sure you
are loading the most recent version of the package.  You can find out
which version of the package has been loaded by looking in the log
file for a line like:
\begin{verbatim}
   Package: fred 1994/06/01 v0.01 Fred's package.
\end{verbatim}
You can use the \m{release-date} options to |\documentclass| and
|\usepackage| to make sure that you are getting a suitably recent copy
of the document class or package.  This is useful when sending a
document to another site, which may have out-of-date software.

\subsection{Where to go for more help}

If you can't find the answer for your problem here, try looking in
\emph{\LaTeXbook} or \emph{\LaTeXcomp}.  If you have a problem with
installing \LaTeX, look in the installation guide files which come with
the distribution.

If this doesn't help, contact your local \LaTeX{} guru or local
\LaTeX{} mailing list.

If you think you've discovered a bug then please report it!  First,
you should find out if the problem is with a third-party package or
class.  If the problem is caused by a package or class other than
those listed in Section~\ref{Sec:class+packages} then please report
the problem to the author of the package or class, not to the \LaTeX3
project team.

If the bug really is with core \LaTeX{} then you should create a
\emph{short}, \emph{self-contained} document which exhibits the
problem.  You should run a \emph{recent} (less than a year old)
version of \LaTeX{} on the file and then run \LaTeX{} on
|latexbug.tex|.  This will create an error report which you should
send, together with the sample document and log file, to the
\LaTeX{} bugs address which can be found in the file
|latexbug.tex| or |bugs.txt|.


\section{Enjoy!}
\label{Sec:enjoy}

We certainly hope you will enjoy using the new standard \LaTeX{} but,
if this is not possible, we hope that you will enjoy success and
fulfillment as a result of the documents which it will help you to
create.

If you find that the contribution of \LaTeX{} to your life is such
that you would like to support the work of the project team, then
please read Section~\ref{Sec:ltx3} and discover practical ways to do
this.

\begin{thebibliography}{1}

\bibitem{A-W:GRM97}
Michel Goossens, Sebastian Rahtz and Frank Mittelbach.
\newblock {\em The {\LaTeX} Graphics Companion}.
\newblock Addison-Wesley, Reading, Massachusetts, 1997.


\bibitem{A-W:GR99}
Michel Goossens and Sebastian Rahtz.
\newblock {\em The {\LaTeX} Web Companion}.
\newblock Addison-Wesley, Reading, Massachusetts, 1999.


\bibitem{A-W:DEK91}
Donald~E. Knuth.
\newblock {\em The \TeX book}.
\newblock Addison-Wesley, Reading, Massachusetts, 1986.
\newblock Revised to cover \TeX3, 1991.


\bibitem{A-W:LLa94}
Leslie Lamport.
\newblock {\em {\LaTeX:} A Document Preparation System}.
\newblock Addison-Wesley, Reading, Massachusetts, second edition, 1994.

\bibitem{A-W:MG2004}
Frank Mittelbach and Michel Goossens.
\newblock {\em The {\LaTeX} Companion second edition}.
\newblock With Johannes Braams, David Carlisle, and Chris Rowley.
\newblock Addison-Wesley, Reading, Massachusetts, 2004.


\end{thebibliography}

\end{document}
