% \iffalse meta-comment
%
% Copyright 2019-2021
% The LaTeX Project and any individual authors listed elsewhere
% in this file.
%
% This file is part of the LaTeX base system.
% -——————————————
%
% It may be distributed and/or modified under the
% conditions of the LaTeX Project Public License, either version 1.3c
% of this license or (at your option) any later version.
% The latest version of this license is in
%    https://www.latex-project.org/lppl.txt
% and version 1.3c or later is part of all distributions of LaTeX
% version 2008 or later.
%
% This file has the LPPL maintenance status "maintained".
%
% The list of all files belonging to the LaTeX base distribution is
% given in the file `manifest.txt'. See also `legal.txt' for additional
% information.
%
% The list of derived (unpacked) files belonging to the distribution
% and covered by LPPL is defined by the unpacking scripts (with
% extension .ins) which are part of the distribution.
%
% \fi
% Filename: ltnews35.tex
%
% This is issue 35 of LaTeX News.

\NeedsTeXFormat{LaTeX2e}[2020-02-02]

\documentclass{ltnews}

%%  Maybe needed only for Chris' inadequate system:  
\providecommand\Dash {\unskip \textemdash}

%% NOTE:  Chris' preferred hyphens!
%%\showhyphens{parameters}
%%  \hyphenation{because parameters parameter}

\usepackage[T1]{fontenc}

\usepackage{lmodern,url,hologo}

\usepackage{csquotes}
\usepackage{multicol}

\providecommand\hook[1]{\texttt{#1}}

\providecommand\meta[1]{$\langle$\textrm{\itshape#1}$\rangle$}
\providecommand\option[1]{\texttt{#1}}
\providecommand\env[1]{\texttt{#1}}
\providecommand\Arg[1]{\texttt\{\meta{#1}\texttt\}}


\providecommand\eTeX{\hologo{eTeX}}
\providecommand\XeTeX{\hologo{XeTeX}}
\providecommand\LuaTeX{\hologo{LuaTeX}}
\providecommand\pdfTeX{\hologo{pdfTeX}}
\providecommand\MiKTeX{\hologo{MiKTeX}}
\providecommand\CTAN{\textsc{ctan}}
\providecommand\TL{\TeX\,Live}
\providecommand\githubissue[2][]{\ifhmode\unskip\fi
     \quad\penalty500\strut\nobreak\hfill
     \mbox{\small\slshape(%
       \href{https://github.com/latex3/latex2e/issues/\getfirstgithubissue#2 \relax}%
          	    {github issue#1 #2}%
           )}%
     \par\smallskip}
%% But Chris has to mostly disable \href for his TEXPAD app:  
%%  \def\href #1{}  % Only For Chris' deficient TeX engine

% simple solution right now (just link to the first issue if there are more)
\def\getfirstgithubissue#1 #2\relax{#1}

\providecommand\sxissue[1]{\ifhmode\unskip\fi
     \quad\penalty500\strut\nobreak\hfill
     \mbox{\small\slshape(\url{https://tex.stackexchange.com/#1})}\par}

\providecommand\gnatsissue[2]{\ifhmode\unskip\fi
     \quad\penalty500\strut\nobreak\hfill
     \mbox{\small\slshape(%
       \href{https://www.latex-project.org/cgi-bin/ltxbugs2html?pr=#1\%2F\getfirstgithubissue#2 \relax}%
          	    {gnats issue #1/#2}%
           )}%
     \par}

\let\cls\pkg
\providecommand\env[1]{\texttt{#1}}
\providecommand\acro[1]{\textsc{#1}}

\vbadness=1400  % accept slightly empty columns


\makeatletter
% maybe not the greatest design but normally we wouldn't have subsubsections
\renewcommand{\subsubsection}{%
   \@startsection      {subsubsection}{2}{0pt}{1.5ex \@plus 1ex \@minus .2ex}%
      {-1em}{\@subheadingfont\colonize}%
}
\providecommand\colonize[1]{#1:}
\makeatother

\let\finalvspace\vspace          % for document layout fixes

% Undo ltnews's \verbatim@font with active < and >
\makeatletter
\def\verbatim@font{%
  \normalsize\ttfamily}
\makeatletter

%%%%%%%%%%%%%%%%%%%%%%%%%%%%%%%%%%%%%%%%%%%%%%%%%%%%%%%%%%%%%%%%%%%%%%%%%%%%%
\providecommand\tubcommand[1]{}
\tubcommand{\input{tubltmac}}

\publicationmonth{June}
\publicationyear{2022  --- DRAFT version for upcoming release}

\publicationissue{35}

\begin{document}

\tubcommand{\addtolength\textheight{4.2pc}}   % only for TUB

\maketitle
{\hyphenpenalty=10000 \spaceskip=3.33pt \hbadness=10000 \tableofcontents}

\setlength\rightskip{0pt plus 3em}


\medskip


\section{Introduction}

\section{Document metadata interface}

Until recently there was no dedicated location to declare
settings that affect a document as a whole. Settings had to be placed
somewhere in the preamble or as class options or sometimes even as
package options. For some such settings this may cause issues, e.g.,
setting the PDF version is only possible as long as the PDF output
file has not yet been opened which can be caused by loading one or the
other package.

For the \enquote{\LaTeX{} Tagged PDF project}~\cite{35:blueprint}
further metadata about the whole document (and its processing) need to
be specified and again this data should be all placed in a single
well-defined place.

For this reason we introduce the new command \cs{DocumentMetadata} to
unify all such settings in one place.  The command expects a key/value
list that describes all document metadata for the current document. It
is only allowed to be used at the very beginning of the document,
i.e., the declaration has to be placed \emph{before}
\cs{documentclass} and will issue an error if found later.


At this point in time we only provide the bare command in the format,
the actual processing of the key/value is defined externally and the
necessary code will be loaded if the command is used. This scheme is
chosen for two reasons: by adding the command in the kernel it is
available to everybody without the need to load a special package
using \cs{RequirePackage}. The actual processing, though, is external
so that we can easily extend the code (e.g., offering additional keys
or changing the internal processing) while the above mentioned project
is progressing. Both together allows users to immediately benefit from
intermediate results produced as part of the project, as well as
offering the \LaTeX{} Project Team the flexibility to enable such
intermediate results (for test purposes or even production use)
in-between and independently of regular \LaTeX{} releases. Over time,
tested and approved functionality can then seamlessly move into the
kernel at a later stage without any alterations to documents already
using it. At the same time, not using the new consolidated interface
means that existing documents are in no way affected by the work that
is carried out and is in a wider alpha or beta test phase.

\emph{either add info about some already existing keys or add a
  reference to the place the the right documentation about this is
  stored (to be determined).}


\section{The \pkg{latex-lab} bundle}

\emph{write about it}




\section{New or improved commands}

\subsection{Floating point and integer calculations}

The L3 programming layer offers expandable commands for calculating
floating point and integer values, but so far these functions have
only been available to programmers, because they require
\cs{ExplSyntaxOn} to be in force. To make them easily available at the
document-level, the small package \pkg{xfp} defined \cs{fpeval} and
\cs{inteval}.


An example of use could be the following:
\begin{verbatim}
\LaTeX{} can now compute: 
\[ \frac{\sin (3.5)}{2} + 2\cdot 10^{-3}
    = \fpeval{sin(3.5)/2 + 2e-3}         \]
\end{verbatim}
which produces the following output:
\begin{quote}
\LaTeX{} can now compute: 
\[ \frac{\sin (3.5)}{2} + 2\cdot 10^{-3}
   = \fpeval{sin(3.5)/2 + 2e-3}         \]
\end{quote}
These two commands have now been moved into the kernel and in addition
we also provide \cs{dimeval} and \cs{skipeval}. The details of their
syntax are described in \file{usrguide3.pdf}.  The command \cs{fpeval}
offers a rich syntax allows for extensive calculations whereas the
other three commands are essentially thin wrappers for \cs{numexpr},
\cs{dimexpr}, and \cs{glueexpr} \Dash therefore inheriting some syntax
peculiars and limitations in expressiveness.
\begin{verbatim}
\newcommand\calulateheight[1]{%
  \setlength\textheight{\dimeval{\topskip 
        + \baselineskip * \inteval{#1-1}}}}
\end{verbatim}
The above, for example, calculates the appropriate \cs{textheight} for
a given number of text lines.
%
\githubissue{711}


\subsection{???}
%
\githubissue{???}


\section{Code improvements}

\subsection{\class{ltxdoc} gets a \option{nocfg} option}

The \LaTeX{} sources are formatted with the \class{ltxdoc} class,
which supports loading a local config file \file{ltxdoc.cfg}. In the
past the \LaTeX{} sources used such a file but it was not distributed.
As a result reprocessing the \LaTeX{} sources elsewhere showed
formatting changes.  We now distribute this file which means that it
is loaded by default. With the option \option{nocfg} this can be
prevented.


\subsection{\pkg{doc} upgraded to version~3}

After roughly three decades the \pkg{doc} package gets a cautious
uplift, as already announced at the TUG conference 2019\Dash changes
to \pkg{doc} are obviously always done in a leisurely manner.

Given that most documentation is nowadays viewed on screen,
\pkg{hyperref} support is added and by default enabled (suppress it
with option \option{nohyperref} or alternatively with
\option{hyperref}\texttt{=false}) so the internal cross-reference are
properly resolved including those from the index back into the
document.

Furthermore, \pkg{doc} has now a general mechanism to define
additional \enquote{doc} elements besides the two \texttt{Macro} and
\texttt{Env} it did know in the past. This enables better
documentation because you can now clearly mark different types of
objects instead of simply calling them all \enquote{macros}.
If desired, they can be collected together under a heading
in the index so that you have a section just with your document
interface commands, or with all parameters, or \ldots

The code borrows ideas from Didier Verna's \pkg{dox} package (although
the document level interface is different) and it makes use of Heiko
Oberdiek's \pkg{hypdoc} package, which at some point in the future
will be completely integrated, given that its whole purpose it to
patch \pkg{doc}'s internal commands to make them \pkg{hyperref}-aware.

All changes are expected to be upward compatible, but if you run into
issues with older documentation using \pkg{doc} a simple and quick
solution is to load the package as follows:
\verb/\usepackage{doc}[=v2]/

\subsection{\pkg{doc} can now show dates in change log}

Up to now the change log was always sorted by version numbers
(ignoring the date that was given in the \cs{changes} command).  It
can now be sorted by both version and date if you specify the option
\option{reportchangedates} on package level and in that case the
changes are displayed with
\begin{quote}
  \meta{version} -- \meta{date}
\end{quote}
as the heading (instead of just \meta{version}), when using
\cs{PrintChanges}.
%
\githubissue{gh/531}



\subsection{Lua\TeX\ callback improvements}

The Lua\TeX\ callbacks \texttt{hpack\_quality} and \texttt{vpack\_quality} are
now \texttt{exclusive} and therefore only allow a single handler.
The previous type \texttt{list} resulted in incorrect parameters when multiple
handlers were set, therefore this only makes an existing restriction more
explicit.

Additionally the return value \texttt{true} for \texttt{list}
callbacks is now handled internally and no longer passed on to the
engine. This simplifies the handling of these callbacks and makes it
easier to provide consistent interfaces for user defined \texttt{list}
callbacks.

\subsection{Class \class{proc} supports \option{twoside}}

The document class \class{proc}, which is a small variation on the
\class{article} class, now supports the \option{twoside} option
displaying different data in the footer line on recto and verso pages.
%
\githubissue{gh/704}


\subsection{Croatian character support}
%
The default \pkg{inputenc} support has been extended to support the 9 characters
D\v Z, D\v z, d\v z, LJ, Lj, lj, NJ, Nj, nj, input as single UTF-8 codepoints
in the range U+01C4 to U+01CC.
%
\githubissue{gh/723}



\section{Bug fixes}

\subsection{Using \cs{DeclareUnicodeCharacter} with C1 control points}
An error in the UTF-8 handling for non-Unicode \TeX, has prevented
\cs{DeclareUnicodeCharacter} being used with characters in the range
hex 80 to 9F, this has been corrected in this release.
%
\githubissue{730}




\section{Changes to packages in the \pkg{amsmath} category}

\subsection{???}
%
\githubissue{???}


\section{Changes to packages in the \pkg{graphics} category}

\subsection{???}
%
\githubissue{???}



\section{Changes to packages in the \pkg{tools} category}

\subsection{\pkg{multicol}: Fix \cs{newcolumn}}

The recently added \cs{newcolumn} didn't work properly if used in
vertical mode, where it behaved like \cs{columnbreak}, i.e., spreading
the column material out instead running the column short.
%
\sxissue{q/624940}


\subsection{???}
%
\githubissue{???}



\medskip

\begin{thebibliography}{9}

\fontsize{9.3}{11.3}\selectfont

\bibitem{35:blueprint} Frank Mittelbach and Chris Rowley:
  \emph{\LaTeX{} Tagged PDF \Dash A blueprint for a large project}.\\
  \url{https://latex-project.org/publications/indexbyyear/2020/}

\bibitem{35:source2e}
  \emph{\LaTeX{} documentation on the \LaTeX{} Project Website}.\\
  \url{https://latex-project.org/help/documentation/}

\bibitem{35:ltnews31} \LaTeX{} Project Team:
  \emph{\LaTeXe{} news 31}.\\
  \url{https://latex-project.org/news/latex2e-news/ltnews31.pdf}

\bibitem{35:ltnews32} \LaTeX{} Project Team:
  \emph{\LaTeXe{} news 32}.\\
  \url{https://latex-project.org/news/latex2e-news/ltnews32.pdf}

\bibitem{35:ltnews33} \LaTeX{} Project Team:
  \emph{\LaTeXe{} news 33}.\\
  \url{https://latex-project.org/news/latex2e-news/ltnews33.pdf}

\end{thebibliography}



\end{document}


