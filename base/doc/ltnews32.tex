% \iffalse meta-comment
%
% Copyright 2019-2020
% The LaTeX3 Project and any individual authors listed elsewhere
% in this file.
%
% This file is part of the LaTeX base system.
% -------------------------------------------
%
% It may be distributed and/or modified under the
% conditions of the LaTeX Project Public License, either version 1.3c
% of this license or (at your option) any later version.
% The latest version of this license is in
%    https://www.latex-project.org/lppl.txt
% and version 1.3c or later is part of all distributions of LaTeX
% version 2008 or later.
%
% This file has the LPPL maintenance status "maintained".
%
% The list of all files belonging to the LaTeX base distribution is
% given in the file `manifest.txt'. See also `legal.txt' for additional
% information.
%
% The list of derived (unpacked) files belonging to the distribution
% and covered by LPPL is defined by the unpacking scripts (with
% extension .ins) which are part of the distribution.
%
% \fi
% Filename: ltnews32.tex
%
% This is issue 32 of LaTeX News.

\NeedsTeXFormat{LaTeX2e}[2020-02-02]

\documentclass{ltnews}
\usepackage[T1]{fontenc}

\usepackage{lmodern,url,hologo}

\usepackage{csquotes}
\usepackage{multicol}

\providecommand\meta[1]{$\langle$\textit{#1}$\rangle$}
\providecommand\option[1]{\texttt{#1}}
\providecommand\env[1]{\texttt{#1}}


\providecommand\XeTeX{\hologo{XeTeX}}
\providecommand\LuaTeX{\hologo{LuaTeX}}
\providecommand\pdfTeX{\hologo{pdfTeX}}
\providecommand\MiKTeX{\hologo{MiKTeX}}
\providecommand\CTAN{\textsc{ctan}}
\providecommand\TL{\TeX\,Live}
\providecommand\githubissue[2][]{\ifhmode\unskip\fi
     \quad\penalty500\strut\nobreak\hfill
     \mbox{\small\slshape(%
       \href{https://github.com/latex3/latex2e/issues/\getfirstgithubissue#2 \relax}%
          	    {github issue#1 #2}%
           )}%
     \par\smallskip}

% simple solution right now (just link to the first issue if there are more)
\def\getfirstgithubissue#1 #2\relax{#1}

\providecommand\sxissue[1]{\ifhmode\unskip\fi
     \quad\penalty500\strut\nobreak\hfill
     \mbox{\small\slshape(\url{https://tex.stackexchange.com/#1})}\par}

\providecommand\gnatsissue[2]{\ifhmode\unskip\fi
     \quad\penalty500\strut\nobreak\hfill
     \mbox{\small\slshape(%
       \href{https://www.latex-project.org/cgi-bin/ltxbugs2html?pr=#1\%2F#2}%
          	    {gnats issue #1/#2}%
           )}%
     \par}

\let\cls\pkg
\providecommand\env[1]{\texttt{#1}}

\vbadness=1400  % accept slightly empty columns


%%%%%%%%%%%%%%%%%%%%%%%%%%%%%%%%%%%%%%%%%%%%%%%%%%%%%%%%%%%%%%%%%%%%%%%%%%%%%
\iffalse % only for TUB production
\usepackage{graphicx}
\makeatletter
% Xe\TeX{} requires reflecting the first E, hence we complain if the
% graphics package is not present.  (For plain documents, this can be
% loaded via Eplain.)  Also, at Barbara's suggestion, if the current
% font is slanted, we rotate by 180 instead of reflecting so there is at
% least a chance to look ok.  (The magic values here seem more or less
% ok for \texttt{cmsl} and \texttt{cmti}.)
%
%    \begin{macrocode}
\def\tubreflect#1{%
  \@ifundefined{reflectbox}{%
    \TBerror{A graphics package must be loaded for \string\XeTeX}%
  }{%
    \ifdim \fontdimen1\font>0pt
      \raise 1.6ex \hbox{\kern.1em\rotatebox{180}{#1}}\kern-.1em
    \else
      \reflectbox{#1}%
    \fi
  }%
}
\def\tubhideheight#1{\setbox0=\hbox{#1}\ht0=0pt \dp0=0pt \box0 }
\def\XekernbeforeE{-.125em}
\def\XekernafterE{-.1667em}
\DeclareRobustCommand{\Xe}{\leavevmode
  \tubhideheight{\hbox{X%
    \setbox0=\hbox{\TeX}\setbox1=\hbox{E}%
    \ifdim \fontdimen1\font>0pt \def\XekernbeforeE{0em}\fi
    \lower\dp0\hbox{\raise\dp1\hbox{\kern\XekernbeforeE\tubreflect{E}}}%
    \kern\XekernafterE}}}
\def\XeTeX{\Xe\TeX}
\def\XeLaTeX{\Xe{\kern.11em \LaTeX}}
\fi
%%%%%%%%%%%%%%%%%%%%%%%%%%%%%%%%%%%%%%%%%%%%%%%%%%%%%%%%%%%%%%%%%%%%%%%%%%%%%

\publicationmonth{October}
\publicationyear{2020}

\publicationissue{32}

\begin{document}

%\addtolength\textheight{4.2pc}   % only for TUB

\maketitle
{\hyphenpenalty=10000 \spaceskip=3.33pt \hbadness=10000 \tableofcontents}

\setlength\rightskip{0pt plus 3em}


\medskip


\section{Introduction}

This document is under construction \ldots



\section{Other changes to the \LaTeX{} kernel}


\subsection{\cs{symbol} in math mode for large Unicode values}

The \LaTeXe{} kernel defines the command \cs{symbol}, which allows characters to be
typeset by entering their `slot number'. With the \LuaTeX{} and \XeTeX{} engines, these
slot numbers can extend to very large values to accomodate Unicode characters in the upper
Unicode planes (e.g., bold mathematical capital A is slot number \texttt{"1D400} in hex or
\texttt{119808} in decimal).
The \XeTeX{} engine did not allow \cs{symbol} in math mode for values above $2^{16}$, and
this limitation has now been lifted.
%
\githubissue{124}


\subsection{Spacing issues when using \cs{linethickness}}

In some circumstances the use of \cs{linethickness} introduced a
spurious space that shifted objects in a \texttt{picture} environments
to the right. This has been corrected.
%
\githubissue{270}


\subsection{Adjusting \texttt{fleqn}}

In \pkg{amsmath} the \cs{mathindent} parameter used with the
\texttt{fleqn} design is a rubber length paramter allowing for setting
it to a value such as \texttt{1em minus 1em}, i.e., so the the normal
indentation can be reduced in case of very wide math displays.
This is now also supported by the \LaTeX{} standard classes.

In addition a compressable space between formula and equation number
in the \texttt{equation} environment got added when the \texttt{fleqn}
option is used so that a very wide formula doesn't bump into the
equation number.
%
\githubissue{252}



\section{Changes to packages in the \pkg{tools} category}

\subsection{Support stretchable glue in \texttt{w}-columns}

If stretchable glue, e.g., \cs{dotfill}, is used in \env{tabular}
columns made with the \pkg{array} package, it stretches as it would in
normal paragraph text. The one exception were \texttt{w}-columns (but
not \texttt{W}-columns) were it got forced to its nominal width (which
in case if \cs{hfill} or \cs{dotfill} is 0\,pt). This has been
corrected and now \texttt{w}-columns behave like all other column
types in this respect.
%
\githubissue{270}



\section{Changes to packages in the \pkg{amsmath} category}

\subsection{Placement corrections for two accent commands}

The accent commands \cs{dddot} and \cs{ddddot} (producing triple and
quatruple dot accents) moved the base character vertically in certain
situations if it was a single glyph,
e.g., \verb=$Q \dddot{Q}$=
were not at the same baseline. This has been corrected.
%
\githubissue{126}


\end{document}
