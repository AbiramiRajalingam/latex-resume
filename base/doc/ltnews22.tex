% \iffalse meta-comment
%
% Copyright (C) 2015-2021
% The LaTeX3 Project and any individual authors listed elsewhere
% in this file.
%
% This file is part of the LaTeX base system.
% -------------------------------------------
%
% It may be distributed and/or modified under the
% conditions of the LaTeX Project Public License, either version 1.3c
% of this license or (at your option) any later version.
% The latest version of this license is in
%    http://www.latex-project.org/lppl.txt
% and version 1.3c or later is part of all distributions of LaTeX
% version 2008 or later.
%
% This file has the LPPL maintenance status "maintained".
%
% The list of all files belonging to the LaTeX base distribution is
% given in the file `manifest.txt'. See also `legal.txt' for additional
% information.
%
% The list of derived (unpacked) files belonging to the distribution
% and covered by LPPL is defined by the unpacking scripts (with
% extension .ins) which are part of the distribution.
%
% \fi
% Filename: ltnews22.tex
%
% This is issue 22 of LaTeX News.

\documentclass{ltnews}
\usepackage[T1]{fontenc}

\usepackage{lmodern,url,hologo}

\publicationmonth{January}
\publicationyear{2015}

\publicationissue{22}

\begin{document}

\maketitle

\section{New \LaTeXe\ bug-fix policy}

\subsection{Introduction}

For some years we have supplied bug fixes to the \LaTeXe\ kernel via the
\textsf{fixltx2e} package. This
kept the kernel stable, but at the expense of meaning that most users did not
benefit from bug fixes, and that some compromises
which were made to save space in the
machines of the time are still affecting most users today.

In this release we have started a new update policy. All the fixes previously
available via \textsf{fixltx2e} are now enabled \emph{by default} in the
format, as are some further extensions for extended \TeX\ engines, \eTeX,
\hologo{XeTeX} and \hologo{LuaTeX}. Compatibility and stability are still
important considerations, and while most users will not notice these
improvements, or will want to benefit from them, a new \textsf{latexrelease}
package is provided that will revert all the changes and re-instate the
definitions from earlier releases. The package can also be used with older
releases to effectively \emph{update} the kernel to be equivalent to this 2015
release.

A new document, \textsf{latexchanges}, is distributed with the release
that documents all the changes to documented commands since the 2014
\LaTeX\ release, and will be updated in future releases if further
changes have been made.

\subsection{The \textsf{latexrelease} package}

As noted above a new package is available to manage differences between \LaTeX\
releases. If you wish to revert all changes back to the definitions as they
were in previous releases you may start your document requesting the \LaTeX{} release from May~2014:
\begin{verbatim}
\RequirePackage[2014/05/01]{latexrelease}
\documentclass{article}
\end{verbatim}
Conversely if you start a large project now and want to protect yourself
against possible future changes, you may start your document
\begin{verbatim}
\RequirePackage[2015/01/01]{latexrelease}
\documentclass{article}
\end{verbatim}
Then the version of \textsf{latexrelease} distributed with any future \LaTeX\
release will revert any changes made in that format, and revert to the
definitions as they where at the beginning of 2015.

If you wish to share a document using the latest features with a
user restricted to using an older
format, you may use the form above and make the \textsf{latexrelease} package
available on the older installation.  The package will then update the format
definitions as needed to enable the older format to work as if dated on the date
specified in the package option.

\subsection{The \cs{IncludeInRelease} command}

The mechanism used in the \textsf{latexrelease} package is available for use in
package code. If in your \textsf{zzz} package you have
\begin{verbatim}
\RequirePackage{latexrelease}
\IncludeInRelease{2015/06/01}
    {\zzz}{\zzz definition}
  \def\zzz......new code
\EndIncludeInRelease
\IncludeInRelease{0000/00/00}
  {\zzz}{\zzz definition}
  \def\zzz....original
\EndIncludeInRelease
\end{verbatim}
then in a document using a format dated 2015/06/01 or later, the ``new code''
will be used, and for documents being processed with an older format, the
``original'' code will be used. Note the format date here may be the original
format date as shown at the start of every \LaTeX\ run, or a format date
specified as a package option to the \textsf{latexrelease} package.

So if the document has
\begin{verbatim}
\RequirePackage[2014/05/01]{latexrelease}
\documentclass{article}
\usepackage{zzz}
\end{verbatim}
then it will use the \emph{original} definition of \verb|\zzz| even if
processed with the current format, as the format acts as if dated 2014/05/01.


\subsection{Limitations of the approach}

The new concept provides full backward and forward compatibility for
the \LaTeX{} format, i.e., with the help of a current
\textsf{latexrelease} package the kernel can emulate all released
formats (starting with 2014/06/01\footnote{Patching an older format
  most likely works too, given that the changes in the past have been
  minimal, though this isn't guaranteed and hasn't been tested.}).


However, this is not necessarily true for all packages.
 Only if a package makes use of the \cs{IncludeInRelease}
functionality will it adjust to the
requested \LaTeX{} release date. Initially this will only be true for
a few selected packages and in general it may not even be advisable for packages
that have their own well-established release cycles and methods.

Thus, to regenerate a document with 100\,\% compatible behavior
it will still be necessary
to archive it together with all its inputs, for example, by archiving
the base distribution trees (and any modifications made). However, the
fact that a document requests a specific \LaTeX{} release date should
help identifying what release tree to use to achieve perfect accuracy.


\section{Updates to the kernel}

\subsection{Updates incorporated from \textsf{fixltx2e}}

The detailed list of changes incorporated from \textsf{fixltx2e} is
available in the new \textsf{latexchanges} document that is
distributed with this release. The main changes are that $2$-column
floats are kept in sequence with one column floats, corrections are
made to the \verb|\mark| system to ensure correct page headings in
$2$-column documents, several additional commands are made robust.

\subsection{\eTeX\ register allocation}

\LaTeX\ has traditionally used allocation routines inherited from plain \TeX\
that allocated registers in the range $0$--$255$. Almost all distributions have
for some years used \eTeX\ based formats (or \hologo{XeTeX} or \hologo{LuaTeX})
which have $2^{15}$ registers of each type ($2^{16}$ in the case of
\hologo{LuaTeX}). The \textsf{etex} package has been available to provided an
allocation mechanism for these extended registers but now the format will by
default allocate in a range suitable for the engine being used. The new
allocation mechanism is different than the \textsf{etex} package mechanism, and
supports \hologo{LuaTeX}'s full range and an allocation mechanism for \LaTeX\
floats as described below.

On \eTeX\ based engines, an additional command, \verb|\newmarks| is available
(as with the \textsf{etex} package) that allocates extended \eTeX\ marks, and
similarly if \hologo{XeTeX} is detected a new command
\verb|\newXeTeXintercharclass| is available, this is similar to the command
previously defined in the \texttt{xelatex.ini} file used to build the
\texttt{xelatex} format.

\subsection{Additional \LaTeX{} float storage}

\LaTeX's float placement algorithm needs to store floats (figures and tables)
until it finds a suitable page to output them. It allocates $18$ registers for
this storage, but this can often be insufficient.
 The contributed \textsf{morefloats} package has
been available to extend this list; however, it also only allocates from the
standard range $0$--$255$ so cannot take advantage of the extended registers.
The new allocation mechanism in this release incorporates a new command
\verb|\extrafloats|. If you get the error:
\mbox{\texttt{Too many unprocessed floats.}}
then you can add (say) \verb|\extrafloats{500}| to the document preamble to
make many more boxes available to hold floats.


\subsection{Built-in support for Unicode engines}

The kernel sources now detect the engine being used and adjust definitions
accordingly, this reduces the need for the ``\texttt{.ini}'' files used to make
the formats to patch definitions defined in \texttt{latex.ltx}.

As noted above the format now includes extended allocation routines.

The distribution includes a file \textsf{unicode-letters.def} derived from the
Unicode Consortium's Unicode Character Data files that details the upper and
lower case transformation data for the full Unicode range. This is used to set
the \verb|lccode| and \verb|uccode| values if a Unicode engine is being used,
rather than the values derived from the T1 font encoding which are used with
8-bit engines.

Finally \verb|\typein| is modified if \hologo{LuaTeX} is detected
such that it works with this engine.

\section{l3build}

This release has been tested and built using a new build system implemented in
\textsf{Lua}, intended to be run on the \textsf{texlua} interpreter distributed
with modern \TeX\ distributions. It is already separately available
from CTAN. This replaces earlier build systems (based at various times on
\textsf{make}, \textsf{cons}, and Windows \textsf{bat} files). It allows the
sources to be tested and packaged on a range of platforms (within the team,
OS X, Windows, Linux and Cygwin platforms are used). It also allows the
format to be tested on \hologo{XeTeX} and \hologo{LuaTeX} as well as the
standard \hologo{pdfTeX}/\eTeX\ engines.

\section{Hyperlinked documentation and TDS zip files}

As well as updating the build system, the team have looked again at exactly
what gets released to \textsc{ctan}. Taking inspiration from Heiko Oberdiek's
\textsf{latex-tds} bundle, the PDF documentation provided now includes
hyperlinks where appropriate. This has been done without modifying the sources
such that users without \textsf{hyperref} available can still typeset the
documentation using only the core distribution. At the same time, the release
now includes ready-to-install TDS-style zip files. This will be of principal
interest to \TeX\ system maintainers, but end users with older machines who
wish to manually update \LaTeX\ will also benefit.

\end{document}
