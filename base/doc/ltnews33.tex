% \iffalse meta-comment
%
% Copyright 2019-2020
% The LaTeX Project and any individual authors listed elsewhere
% in this file.
%
% This file is part of the LaTeX base system.
% -------------------------------------------
%
% It may be distributed and/or modified under the
% conditions of the LaTeX Project Public License, either version 1.3c
% of this license or (at your option) any later version.
% The latest version of this license is in
%    https://www.latex-project.org/lppl.txt
% and version 1.3c or later is part of all distributions of LaTeX
% version 2008 or later.
%
% This file has the LPPL maintenance status "maintained".
%
% The list of all files belonging to the LaTeX base distribution is
% given in the file `manifest.txt'. See also `legal.txt' for additional
% information.
%
% The list of derived (unpacked) files belonging to the distribution
% and covered by LPPL is defined by the unpacking scripts (with
% extension .ins) which are part of the distribution.
%
% \fi
% Filename: ltnews33.tex
%
% This is issue 33 of LaTeX News.

\NeedsTeXFormat{LaTeX2e}[2020-02-02]

\documentclass{ltnews}
\usepackage[T1]{fontenc}

\usepackage{lmodern,url,hologo}

\usepackage{csquotes}
\usepackage{multicol}

\providecommand\meta[1]{$\langle$\textrm{\itshape#1}$\rangle$}
\providecommand\option[1]{\texttt{#1}}
\providecommand\env[1]{\texttt{#1}}
\providecommand\Arg[1]{\texttt\{\meta{#1}\texttt\}}


\providecommand\eTeX{\hologo{eTeX}}
\providecommand\XeTeX{\hologo{XeTeX}}
\providecommand\LuaTeX{\hologo{LuaTeX}}
\providecommand\pdfTeX{\hologo{pdfTeX}}
\providecommand\MiKTeX{\hologo{MiKTeX}}
\providecommand\CTAN{\textsc{ctan}}
\providecommand\TL{\TeX\,Live}
\providecommand\githubissue[2][]{\ifhmode\unskip\fi
     \quad\penalty500\strut\nobreak\hfill
     \mbox{\small\slshape(%
       \href{https://github.com/latex3/latex2e/issues/\getfirstgithubissue#2 \relax}%
          	    {github issue#1 #2}%
           )}%
     \par\smallskip}

% simple solution right now (just link to the first issue if there are more)
\def\getfirstgithubissue#1 #2\relax{#1}

\providecommand\sxissue[1]{\ifhmode\unskip\fi
     \quad\penalty500\strut\nobreak\hfill
     \mbox{\small\slshape(\url{https://tex.stackexchange.com/#1})}\par}

\providecommand\gnatsissue[2]{\ifhmode\unskip\fi
     \quad\penalty500\strut\nobreak\hfill
     \mbox{\small\slshape(%
       \href{https://www.latex-project.org/cgi-bin/ltxbugs2html?pr=#1\%2F#2}%
          	    {gnats issue #1/#2}%
           )}%
     \par}

\let\cls\pkg
\providecommand\env[1]{\texttt{#1}}

\vbadness=1400  % accept slightly empty columns


%%%%%%%%%%%%%%%%%%%%%%%%%%%%%%%%%%%%%%%%%%%%%%%%%%%%%%%%%%%%%%%%%%%%%%%%%%%%%
\providecommand\tubcommand[1]{}
\tubcommand{\input{tubltmac}}

\publicationmonth{May}
\publicationyear{2021}

\publicationissue{33}

\begin{document}

\tubcommand{\addtolength\textheight{4.2pc}}   % only for TUB

\maketitle
{\hyphenpenalty=10000 \spaceskip=3.33pt \hbadness=10000 \tableofcontents}

\setlength\rightskip{0pt plus 3em}


\medskip


\section{Introduction}

\emph{to be written}




\section{Other changes to the \LaTeX{} kernel}

\subsection{Adjusting \env{itemize} labels with \cs{labelitemfont}}

The command \cs{labelitemfont} was in fact already introduced with the
\LaTeX\ release 2020-02-02, but back then we forgot to describe it, so
we do this now. Its purpose is to resolve some bad formatting issues
with the \env{itemize} environment and at the same time make it easier
to adjust its layout if necessary. What could happen in the past was the the
\env{itemize} labels, e.g., the \textbullet{}, would sometimes react to
surrounding font changes and could suddenly change shape, for example
to \textit{\textbullet}.

Now \cs{labelitemfont} is applied to each
label defaulting to \cs{normalfont} which will prevent this behavior.
By chosing a different settings other effects can be achieved, for example
\begin{verbatim}
  \renewcommand\labelitemfont
     {\normalfont\fontfamily{lmss}\selectfont}
  \renewcommand\labelitemfont
     {\rmfamily\normalshape}
\end{verbatim}
The first will take the symbols from Latin Modern Sans so that you get
%
\def\myfont#1{{\let\labelitemfont\empty\fontfamily{lmss}\selectfont#1}}
%
\myfont\labelitemi, \myfont\labelitemii, \myfont\labelitemiii\ and
\myfont\labelitemiv, while the second variant freezes the font family
and shape, but leave the series variable, so that an \env{itemize} in
a bold context would show bolder symbols. Making it empty would give
you the buggy old behavior back.
%
\githubissue{497}


\subsection{\cs{end}\texttt{\textbraceleft document\textbraceright}
  should always start in v-mode}

Until now \verb=\end{document}= executed the code from the
\cs{AtEndDocument} hook as its first action. This meant that it was
executed in horizontal mode if the user left no empty line after the
last paragraph.  As a result one could get a spurious space added, for
example, when that code contained a \cs{write} statement. This was
fixed and now \cs{enddocument} first issues a \cs{par} to ensure that
it always starts out in vertical mode.
%
\githubissue{385}



\subsection{Allow extra space between name and address in \pkg{letter} class}

The \cs{opening} command in the the \pkg{letter} class expects the
name and address to be separated by \verb=\\= but it didn't allow to
use an optional argument at this point to add some extra space after
the name. The coding has now been slightly altered to allow for this.
%
\githubissue{427}


\subsection{Add a Lua callback to ltshipout to provide a uniform location for applying custom attributes}

Just before shipping out a page, a new \LuaTeX{} callback
``\texttt{pre\_shipout\_filter}'' is now called to allow final
adjustments to the box to be shipped out. This is particularly for
Lua\TeX\ packages which flag certain elements of the page (e.g. using
attributes or properties) in order to apply certain effects to these
elements at shipout. An example for this is the \pkg{luacolor}
package which could insert the color commands using this callback.


\subsection{Improved copy\&paste support for \pdfTeX{} documents}

When compiling with \pdfTeX{}, additional information is added to the
PDF file to improve copying from and searching in text. This especially
allows ligatures to copy correctly from \pdfTeX{} generated PDF files in
most cases.

Since this has been integrated into the kernel, most documents should no
longer need to load the \pkg{cmap} package or input \texttt{glyphtounicode}.
%
\githubissue{465}


\subsection{Provide hook in \cs{selectfont}}

After \cs{selectfont} has altered the font we run a hook so that
packages can make final adjustments. This functionality was originally
provided by the \pkg{everysel} package, the new implementation is
slightly different and uses the standard hook management.
%
\githubissue{444}


\subsection{Delay change of font series and shape to \cs{selectfont} call}

With the NFSS extensions introduced in 2020 the font series and shape
settings be be influenced by changes to the font family. The setting
is therefore delayed until \cs{selectfont} is executed to avoid
unnecessary or incorrect substitutions that may otherwise happen due
to the order of declarations.
%
\githubissue{444}



\subsection{Allow \cs{nocite} in preamble}

A natural place for \verb=\nocite{*}= would be the preamble of the
document, but for historical reasons \LaTeX{} issued an error message
if it was placed there. From the new release on it is now allowed in
the preamble.
%
\githubissue{424}



\subsection{Shipping out a page while bypassing hooks}

In the 2020 October release several hooks were added to the page
shipout process, e.g., to add some background or foreground material
to some or all pages. We now also added a \cs{RawShipout} command that
bypasses most of these hooks during the shipout. Some essential
internal bookkeeping still takes place such as updating the
\texttt{totalpages} counter or adding \texttt{shipout/firstpage} or
\texttt{shipout/lastpage} material if the page happens to be the first
or last.

\subsection{Robust commands in filename arguments}
The filename handling has been modified so that \verb|\string| is
applied while normalising robust commands while detetermining the file
name.  Previously \verb|\input{\sqrt{2}}| would cause \LaTeX\ to loop indefinitely.
With the new behaviour it accesses \verb|sqrt {2}.tex|.
%
\githubissue{481}


\subsection{Additional support for Unicode characters from the
  Latin Extended Additional block}

\LaTeX\ is quite capable of typesetting characters such as
\enquote{\d{m}}, but until now it lacked the Unicode mappings for some
characters that are used to write Sanskrit words in Latin
transliteration (as seen in books about yoga, Buddhist philosophy,
etc.). These have now been added so that such characters can be
entered directly instead of resorting to \verb=\d{m}= and so forth.
%
\githubissue{484}


\subsection{Always have color groups set up}

To use color in \LaTeX{} certain constructs, especially boxes, need an
extra layer of groups to ensure that the color setting does not
\emph{escape} and continue outside the box when it shouldn't. To
arrange for this the \LaTeX{} kernel defined a number of commands, e.g.,
\cs{color@begingroup} to be used in such places. They have been
initally no-ops and only the color packages redefined them to become
real groups. This arrangement complicates the coding as one has to
account for a group being there (or not there) depending of what is
loaded in the document. So now the kernel already adds the groups.
%
\githubissue{488}


\subsection{Execute \cs{par} at the end of \cs{marginpar} arguments}

In preparation for tagged PDF it is important to properly tag all
paragraphs and this requires running code at the beinning and end of
each. At the end of a paragraph this is done inside the \cs{par}
command, but the way \cs{marginpar} was coded, \LaTeX{} ended the
marginal note without ever explicitly calling \cs{par}. This has now
been changed.
%
\githubissue{489}



\subsection{\ldots}

\emph{to be written}
%
\githubissue{xxx}



\section{Changes to packages in the \pkg{graphics} category}

\subsection{\ldots}

%
\githubissue{xxx}



\section{Changes to packages in the \pkg{tools} category}

\subsection{\pkg{layout}: Support Japanese as a language option}
The package now recognizes \texttt{japanese} as a language option.
%
\githubissue{353}


\subsection{\ldots}

%
\githubissue{xxx}




\section{Changes to packages in the \pkg{amsmath} category}

\subsection{\ldots}

%
\githubissue{xxx}





\end{document}


\medskip
\begin{thebibliography}{9}

\fontsize{9.3}{11.3}\selectfont

\bibitem{32:ltnews31} \LaTeX{} Project Team:
  \emph{\LaTeXe{} news 31}.\\
  \url{https://latex-project.org/news/latex2e-news/ltnews31.pdf}

\bibitem{32:site-doc}
  \emph{\LaTeX{} documentation on the \LaTeX{} Project Website}.\\
  \url{https://latex-project.org/help/documentation/}

\bibitem{32:issue-tracker}
  \emph{\LaTeX{} issue tracker}.
  \url{https://github.com/latex3/latex2e/issues/}

\bibitem{32:babel}
  Javier Bezos and Johannes Braams.
  \emph{Babel---Localization and internationalization}.\\
  \url{https://www.ctan.org/pkg/babel}

\end{thebibliography}

