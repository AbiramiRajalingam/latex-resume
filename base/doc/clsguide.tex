% \iffalse meta-comment
%
% Copyright (C) 2023
% The LaTeX Project and any individual authors listed elsewhere
% in this file.
%
% This file is part of the LaTeX base system.
% -------------------------------------------
%
% It may be distributed and/or modified under the
% conditions of the LaTeX Project Public License, either version 1.3c
% of this license or (at your option) any later version.
% The latest version of this license is in
%    http://www.latex-project.org/lppl.txt
% and version 1.3c or later is part of all distributions of LaTeX
% version 2008 or later.
%
% This file has the LPPL maintenance status "maintained".
%
% The list of all files belonging to the LaTeX base distribution is
% given in the file `manifest.txt'. See also `legal.txt' for additional
% information.
%
% The list of derived (unpacked) files belonging to the distribution
% and covered by LPPL is defined by the unpacking scripts (with
% extension .ins) which are part of the distribution.
%
% \fi
% Filename: clsguide.tex

\documentclass{ltxguide}

\usepackage[T1]{fontenc}  % needed for \textbackslash in tt
\usepackage{csquotes}

\title{\LaTeX\ for package and class authors --- current version}
\author{\copyright~Copyright 2023, \LaTeX\ Project Team.\\
   All rights reserved.%
   \footnote{This file may distributed and/or modified under the
     conditions of the \LaTeX{} Project Public License, either version 1.3c
     of this license or (at your option) any later version. See the source
    \texttt{usrguide.tex} for full details.}%
}

\date{2022-07-05}

\NewDocumentCommand\cs{m}{\texttt{\textbackslash\detokenize{#1}}}
\NewDocumentCommand\marg{m}{\arg{#1}}
\NewDocumentCommand\meta{m}{\ensuremath{\langle}\textit{#1}\ensuremath{\rangle}}
\NewDocumentCommand\pkg{m}{\textsf{#1}}
\NewDocumentCommand\text{m}{\ifmmode\mbox{#1}\else#1\fi}
% Fix a 'feature'
\makeatletter
\renewcommand \verbatim@font {\normalfont \ttfamily}
\makeatother

\begin{document}

\maketitle

\tableofcontents

\section{Introduction}

\LaTeXe{} was released in 1994 and added a number of then-new concepts to
\LaTeX{}. For package and class authors, these are described in
\texttt{clsguide-historic}, which has largely remained unchanged. Since then,
the \LaTeX{} team have worked on a number of ideas, firstly a programming
language for \LaTeX{} (L3 programming layer) and then a range of tools for
authors which build on that language. Here, we describe the current, stable set
of tools provided by the \LaTeX{} kernel for packages and class developers. We
assume familiarity with general \LaTeX{} usage as a document author, and that
the material here is read in conjunction with \texttt{usrguide}, which provides
information for general \LaTeX{} users on up-to-date approaches to creating
commands, etc.

\end{document}