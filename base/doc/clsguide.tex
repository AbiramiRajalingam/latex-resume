% \iffalse meta-comment
%
% Copyright (C) 2023
% The LaTeX Project and any individual authors listed elsewhere
% in this file.
%
% This file is part of the LaTeX base system.
% -------------------------------------------
%
% It may be distributed and/or modified under the
% conditions of the LaTeX Project Public License, either version 1.3c
% of this license or (at your option) any later version.
% The latest version of this license is in
%    http://www.latex-project.org/lppl.txt
% and version 1.3c or later is part of all distributions of LaTeX
% version 2008 or later.
%
% This file has the LPPL maintenance status "maintained".
%
% The list of all files belonging to the LaTeX base distribution is
% given in the file `manifest.txt'. See also `legal.txt' for additional
% information.
%
% The list of derived (unpacked) files belonging to the distribution
% and covered by LPPL is defined by the unpacking scripts (with
% extension .ins) which are part of the distribution.
%
% \fi
% Filename: clsguide.tex

\documentclass{ltxguide}

\usepackage[T1]{fontenc}  % needed for \textbackslash in tt
\usepackage{csquotes}

\title{\LaTeX\ for package and class authors --- current version}
\author{\copyright~Copyright 2023, \LaTeX\ Project Team.\\
   All rights reserved.%
   \footnote{This file may distributed and/or modified under the
     conditions of the \LaTeX{} Project Public License, either version 1.3c
     of this license or (at your option) any later version. See the source
    \texttt{usrguide.tex} for full details.}%
}

\date{2022-07-05}

\NewDocumentCommand\cs{m}{\texttt{\textbackslash\detokenize{#1}}}
\NewDocumentCommand\marg{m}{\arg{#1}}
\NewDocumentCommand\meta{m}{\ensuremath{\langle}\textit{#1}\ensuremath{\rangle}}
\NewDocumentCommand\pkg{m}{\textsf{#1}}
\NewDocumentCommand\text{m}{\ifmmode\mbox{#1}\else#1\fi}
% Fix a 'feature'
\makeatletter
\renewcommand \verbatim@font {\normalfont \ttfamily}
\makeatother

\begin{document}

\maketitle

\tableofcontents

\section{Introduction}

\LaTeXe{} was released in 1994 and added a number of then-new concepts to
\LaTeX{}. For package and class authors, these are described in
\texttt{clsguide-historic}, which has largely remained unchanged. Since then,
the \LaTeX{} team have worked on a number of ideas, firstly a programming
language for \LaTeX{} (L3 programming layer) and then a range of tools for
authors which build on that language. Here, we describe the current, stable set
of tools provided by the \LaTeX{} kernel for packages and class developers. We
assume familiarity with general \LaTeX{} usage as a document author, and that
the material here is read in conjunction with \texttt{usrguide}, which provides
information for general \LaTeX{} users on up-to-date approaches to creating
commands, etc.

\section{Writing classes and packages}
\label{Sec:writing}

This section covers some general points concerned with writing
\LaTeX{} classes and packages.

\subsection{Is it a class or a package?}
\label{Sec:classorpkg}

The first thing to do when you want to put some new \LaTeX{} commands
in a file is to decide whether it should be a \emph{document class} or a
\emph{package}.  The rule of thumb is:
\begin{quote}
  If the commands could be used with any document class, then make
  them a package; and if not, then make them a class.
\end{quote}

There are two major types of class: those like |article|, |report| or
|letter|, which are free-standing; and those which are extensions or
variations of other classes---for example, the |proc| document class,
which is built on the |article| document class.

Thus, a company might have a local |ownlet| class for printing letters
with their own headed note-paper.  Such a class would build on top of
the existing |letter| class but it cannot be used with any other
document class, so we have |ownlet.cls| rather than |ownlet.sty|.

The |graphics| package, in contrast, provides commands for including
images into a \LaTeX{} document.  Since these commands can be used
with any document class, we have |graphics.sty| rather than
|graphics.cls|.

\subsection{Using `docstrip' and `doc'}

If you are going to write a large class or package for \LaTeX{} then
you should consider using the |doc| software which comes with
\LaTeX{}.
\LaTeX{} classes and packages written using this can be
processed in two ways: they can be run through \LaTeX{}, to produce
documentation; and they can be processed with |docstrip|, to produce
the |.cls| or |.sty| file.

The |doc| software can automatically generate indexes of definitions,
indexes of command use, and change-log lists.  It is very useful for
maintaining and documenting large \TeX{} sources.

The documented sources of the \LaTeX{} kernel itself, and of the
standard classes, etc, are |doc| documents; they are in the |.dtx|
files in the distribution.  You can, in fact, typeset the source code
of the kernel as one long document, complete with index, by running
\LaTeX{} on |source2e.tex|.  Typesetting these documents uses the
class file |ltxdoc.cls|.

For more information on |doc| and |docstrip|, consult the files
|docstrip.dtx|, |doc.dtx|, and \emph{\LaTeXcomp}.  For examples of its
use, look at the |.dtx| files.

\subsection{Command names}

\LaTeX{} has three types of command.

There are the author commands, such as |\section|, |\emph| and
|\times|:  most of these have short names, all in lower case.

There are also the class and package writer commands:
most of these have long mixed-case names such as the following.
\begin{verbatim}
   \InputIfFileExists  \RequirePackage  \PassOptionsToClass
\end{verbatim}

Finally, there are the internal commands used in the \LaTeX{}
implementation, such as |\@tempcnta|, |\@ifnextchar| and |\@eha|:
most of these commands contain |@| in their name, which means they
cannot be used in documents, only in class and package files.

Unfortunately, for historical reasons the distinction between these
commands is often blurred.  For example, |\hbox| is an internal
command which should only be used in the \LaTeX{} kernel, whereas
|\m@ne| is the constant $-1$ and could have been |\MinusOne|.

However, this rule of thumb is still useful: if a command has |@| in
its name then it is not part of the supported \LaTeX{} language---and
its behaviour may change in future releases!  If a command is
mixed-case, or is described in \emph{\LaTeXbook}, then you can rely on
future releases of \LaTeX{} supporting the command.

\subsection{Programming support}

As noted in the introduction, the \LaTeX{} kernel today loads dedicated support
from programming, here referred to as the L3 programming layer but also often
called \pkg{expl3}. Detailes of the general approach taken by the L3
programming layer are given in the document \texttt{expl3}, while a reference
for all current code interfaces is available as \texttt{interface3}. This layer
contains two types of command: a documented set of commands making up the API
and a large number of private internal commands. The latter all start with two
underscores and should not be used outside of the code module which defines
them. This more structured approach means that using the L3 programming layer
does not suffer from the somewhat fluid situation mentioned above with
`\texttt{@} commands'.

We do not cover the detail of using the L3 programming layer here. A good
introduction to the approach is provided at
\url{https://www.alanshawn.com/latex3-tutorial/}.

\subsection{Box commands and color}
\label{Sec:color}

Even if you do not intend to use color in your own documents, by taking note of
the points in this section you can ensure that your class or package is
compatible with the |color| package. This may benefit people using your class
or package who have access to color printers.

The simplest way to ensure `color safety' is to always use \LaTeX{} box
commands rather than \TeX{} primitives, that is use |\sbox| rather than
|\setbox|, |\mbox| rather than |\hbox| and |\parbox| or the |minipage|
environment rather than |\vbox|. The \LaTeX{} box commands have new options
which mean that they are now as powerful as the \TeX{} primitives.

As an example of what can go wrong, consider that in |{\ttfamily <text>}| the
font is restored just \emph{before} the |}|, whereas in the similar looking
construction |{\color{green} <text>}| the color is restored just \emph{after}
the final |}|. Normally this distinction does not matter at all; but consider a
primitive \TeX{} box assignment such as:
\begin{verbatim}
   \setbox0=\hbox{\color{green} <text>}
\end{verbatim}
Now the color-restore occurs after the |}| and so is \emph{not} stored in the
box. Exactly what bad effects this can have depends on how color is
implemented: it can range from getting the wrong colors in the rest of the
document, to causing errors in the dvi-driver used to print the document.

Also of interest is the command |\normalcolor|. This is normally just |\relax|
(i.e., does nothing) but you can use it rather like |\normalfont| to set
regions of the page such as captions or section headings to the `main document
color'.

\subsection{General style}
\label{Sec:general}

\subsubsection{Loading other files}
\label{Sec:loading}

\LaTeX{} provides these commands:
\begin{verbatim}
   \LoadClass        \LoadClassWithOptions
   \RequirePackage   \RequirePackageWithOptions
\end{verbatim}
for using classes or packages inside other classes or packages. We recommend
strongly that you use them, rather than the primitive |\input| command, for a
number of reasons.

Files loaded with |\input <filename>| will not be listed in the |\listfiles|
list.

If a package is always loaded with |\RequirePackage...| or |\usepackage| then,
even if its loading is requested several times, it will be loaded only once. By
contrast, if it is loaded with |\input| then it can be loaded more than once;
such an extra loading may waste time and memory and it may produce strange
results.

If a package provides option-processing then, again, strange results are
possible if the package is |\input| rather than loaded by means of
|\usepackage| or |\RequirePackage...|.

If the package |foo.sty| loads the package |baz.sty| by use of |\input baz.sty|
then the user will get a warning:
\begin{verbatim}
   LaTeX Warning: You have requested package `foo',
                  but the package provides `baz'.
\end{verbatim}
Thus, for several reasons, using |\input| to load packages is not a good idea.
Unfortunately, if you are upgrading the file |myclass.sty| to a class file then
you have to make sure that any old files which contain |\input myclass.sty|
still work.

For example, |article.sty| contains just the following lines:
\begin{verbatim}
   \NeedsTeXFormat{LaTeX2e}
   \@obsoletefile{article.cls}{article.sty}
   \LoadClass{article}
\end{verbatim}
You may wish to do the same or, if you think that it is safe to do so, you may
decide to just remove |myclass.sty|.

\subsubsection{Make it robust}

We consider it good practice, when writing packages and classes, to use
\LaTeX{} commands as much as possible.

Thus, instead of using |\def...| we recommend using one of |\newcommand|,
|\renewcommand| or |\providecommand| for programming and for defining document
interfaces |\NewDocumentCommand|, etc. (see \texttt{usrguide} for details of
these commands).

When you define an environment, use |\NewDocumentEnviornment| or
|\ReewDocumentEnviornment| instead |\def\foo{...}| and |\def\endfoo{...}|.

If you need to set or change the value of a \m{dimen} or \m{skip} register, use
|\setlength|.

To manipulate boxes, use \LaTeX{} commands such as |\sbox|, |\mbox| and
|\parbox| rather than |\setbox|, |\hbox| and |\vbox|.

Use |\PackageError|, |\PackageWarning| or |\PackageInfo| (or the equivalent
class commands) rather than |\@latexerr|, |\@warning| or |\wlog|.

The advantage of this kind of practice is that your code is more readable and
accessible to other experienced \LaTeX{} programmers.

\subsubsection{Make it portable}

It is also sensible to make your files are as portable as possible. To ensure
this, files must not have the same name as a file in the standard \LaTeX{}
distribution, however similar its contents may be to one of these files. It is
also still lower risk to stick to file names which use only the ASCII range:
whilst \LaTeX{} works natively with UTF-8, the same cannot be said with
certainty for all tools. For the same reason, avoid spaces in file names.

It is also useful if local classes or packages have a common prefix, for
example the University of Nowhere classes might begin with |unw|. This helps to
avoid every University having its own thesis class, all called |thesis.cls|.

If you rely on some features of the \LaTeX{} kernel, or on a package,
please specify the release-date you need.  For example, the package
error commands were introduced in the June 2022 release so, if you use
them then you should put:
\begin{verbatim}
   \NeedsTeXFormat{LaTeX2e}[2022-06-01]
\end{verbatim}

\subsubsection{Useful hooks}

It is sometimes necessary for a package to arrange for code to be
executed at the start or end of the preamble, at the end of the document
or at the start of every use of an environment. This can be carried
out by using hooks. As a document author, you will likely be familiar with
|\AtBeginDocument|, a wrapper around the more powerful command |\AddToHook|.
The \LaTeX{} kernel provides a large number of dedicated hooks (applying in a 
a pre-defined location) and generic hooks (applying to arbitrary commands):
the interface for using these is described in \texttt{lthooks} . There are
also hooks to apply to files, described in \texttt{ltfilehooks}.

\section{The structure of a class or package}
\label{Sec:structure}

The outline of a class or package file is:
\begin{description}
\item[Identification] The file says that it is a \LaTeXe{} package or
   class, and gives a short description of itself.
\item[Preliminary declarations]
   Here the file declares some commands and can also load
   other files.  Usually these commands will be just those needed for
   the code used in the declared options.
\item[Options] The file declares and processes its options.
\item[More declarations] This is where the file does most of its work:
   declaring new variables, commands and fonts; and loading other files.
\end{description}

\subsection{Identification}

The first thing a class or package file does is identify itself.
Package files do this as follows:
\begin{verbatim}
   \NeedsTeXFormat{LaTeX2e}
   \ProvidesPackage{<package>}[<date> <other information>]
\end{verbatim}
For example:
\begin{verbatim}
   \NeedsTeXFormat{LaTeX2e}
   \ProvidesPackage{latexsym}[2002-06-01 Standard LaTeX package]
\end{verbatim}
Class files do this as follows:
\begin{verbatim}
   \NeedsTeXFormat{LaTeX2e}
   \ProvidesClass{<class-name>}[<date> <other information>]
\end{verbatim}
For example:
\begin{verbatim}
   \NeedsTeXFormat{LaTeX2e}
   \ProvidesClass{article}[2002-06-01 Standard LaTeX class]
\end{verbatim}
The \m{date} should be given in the form `\textsc{YYYY-MM-DD}' and must be
present if the optional argument is used (this is also true for the
|\NeedsTeXFormat| command). Any derivation from this syntax will result in
low-level \TeX{} errors---the commands expect a valid syntax to speed up the
daily usage of the package or class and make no provision for the case that the
developer made a mistake!

This date is checked whenever a user specifies a date in their |\documentclass|
or |\usepackage| command. For example, if you wrote:
\begin{verbatim}
   \documentclass{article}[2002-06-01]
\end{verbatim}
then users at a different location would get a warning that their copy of
|article| was out of date.

The description of a class is displayed when the class is used. The description
of a package is put into the log file. These descriptions are also displayed by
the |\listfiles| command. The phrase \texttt{Standard LaTeX} \textbf{must not}
be used in the identification banner of any file other than those in the
standard \LaTeX{} distribution.

\subsection{Using classes and packages}

A \LaTeX{} package or class can load a package as follows:
\begin{verbatim}
   \RequirePackage[<options>]{<package>}[<date>]
\end{verbatim}
For example:
\begin{verbatim}
   \RequirePackage{ifthen}[2002-06-01]
\end{verbatim}
This command has the same syntax as the author command |\usepackage|. It allows
packages or classes to use features provided by other packages. For example, by
loading the |ifthen| package, a package writer can use the `if\dots then\dots
else\dots' commands provided by that package.

A \LaTeX{} class can load one other class as follows:
\begin{verbatim}
   \LoadClass[<options>]{<class-name>}[<date>]
\end{verbatim}
For example:
\begin{verbatim}
   \LoadClass[twocolumn]{article}
\end{verbatim}
This command has the same syntax as the author command |\documentclass|. It
allows classes to be based on the syntax and appearance of another class. For
example, by loading the |article| class, a class writer only has to change the
bits of |article| they don't like, rather than writing a new class from
scratch.

The following commands can be used in the common case that you want to
simply load a class or package file with exactly those options that
are being used by the current class.
\begin{verbatim}
   \LoadClassWithOptions{<class-name>}[<date>]
   \RequirePackageWithOptions{<package>}[<date>]
\end{verbatim}
For example:
\begin{verbatim}
   \LoadClassWithOptions{article}
   \RequirePackageWithOptions{graphics}[1995/12/01]
\end{verbatim}

\subsection{Declaring options}

Packages and classes can declare options and these can be specified by authors;
for example, the |twocolumn| option is declared by the |article| class. Note
that the name of an option should contain only those characters allowed in a
`\LaTeX{} name'; in particular it must not contain any control sequences.

An option is declared as follows:
\begin{verbatim}
   \DeclareOption{<option>}{<code>}
\end{verbatim}
For example, the |dvips| option (slightly simplified)
to the |graphics| package is implemented as:
\begin{verbatim}
   \DeclareOption{dvips}{%% dvips.def Copyright (C) 1994      David Carlisle Sebastian Rahtz
%%           Copyright (C) 1995-1999 David Carlisle
%%           Copyright (C) 2000-2017 David Carlisle, LaTeX3 Project
%%
%% This work may be distributed and/or modified under the
%% conditions of the LaTeX Project Public License, either version 1.3c
%% of this license or (at your option) any later version.
%% The latest version of this license is in
%%   http://www.latex-project.org/lppl.txt
%% and version 1.3c or later is part of all distributions of LaTeX
%% version 2005/12/01 or later.
%%
%% This work has the LPPL maintenance status `maintained'.
%%
%% The Current Maintainer of this work is the LaTeX3 Project.
%%
%% https://github.com/latex3/graphics-def/issues
%%
\ProvidesFile{dvips.def}
  [2017/06/20 v3.1d Graphics/color driver for dvips]
\def\GPT@space{ }
\def\c@lor@arg#1{%
  \dimen@#1\p@
  \ifdim\dimen@<\z@\dimen@\maxdimen\fi
  \ifdim\dimen@>\p@
    \PackageError{color}{Argument `#1' not in range [0,1]}\@ehd
  \fi}
\def\color@gray#1#2{%
  \c@lor@arg{#2}%
  \edef#1{gray #2}%
  }
\def\color@cmyk#1#2{\c@lor@@cmyk#2\@@#1}
\def\c@lor@@cmyk#1,#2,#3,#4\@@#5{%
  \c@lor@arg{#4}%
  \c@lor@arg{#1}%
  \c@lor@arg{#2}%
  \c@lor@arg{#3}%
  \edef#5{cmyk #1 #2 #3 #4}%
  }
\def\color@rgb#1#2{\c@lor@@rgb#2\@@#1}
\def\c@lor@@rgb#1,#2,#3\@@#4{%
  \c@lor@arg{#1}%
  \c@lor@arg{#2}%
  \c@lor@arg{#3}%
  \edef#4{rgb #1 #2 #3}%
  }
\def\color@RGB#1#2{\c@lor@@RGB#2\@@#1}
\def\c@lor@@RGB#1,#2,#3\@@#4{%
 \c@lor@RGB@rgb{#1}\@tempa
 \c@lor@RGB@rgb{#2}\@tempb
 \c@lor@RGB@rgb{#3}\@tempc
 \c@lor@@rgb\@tempa,\@tempb,\@tempc\@@#4%
  }
\def\c@lor@RGB@rgb#1#2{%
  \dimen@#1\p@
  \divide\dimen@\@cclv
  \edef#2{\strip@pt\dimen@}}
\def\color@hsb#1#2{\c@lor@@hsb#2\@@#1}
\def\c@lor@@hsb#1,#2,#3\@@#4{%
  \c@lor@arg{#1}%
  \c@lor@arg{#2}%
  \c@lor@arg{#3}%
  \edef#4{hsb #1 #2 #3}%
  }
\def\color@named#1#2{\c@lor@@named#2,,\@@#1}
\def\c@lor@@named#1,#2,#3\@@#4{%
  \@ifundefined{col@#1}%
    {\PackageError{color}{Undefined color `#1'}\@ehd}%
  {\edef#4{ #1}}%
  }
\def\c@lor@to@ps#1 #2\@@{\csname c@lor@ps@#1\endcsname#2 \@@}
\def\c@lor@ps@#1 #2\@@{TeXDict begin #1 end}
\def\c@lor@ps@rgb#1\@@{#1 setrgbcolor}
\def\c@lor@ps@hsb#1\@@{#1 sethsbcolor}
\def\c@lor@ps@cmyk#1\@@{#1 setcmykcolor}
\def\c@lor@ps@gray#1\@@{#1 setgray}
\def\current@color{ Black}
\def\set@color{%
 \special{color push  \current@color
                          }\aftergroup\reset@color}
\def\reset@color{\special{%
        color pop}}
\def\set@page@color{\special{%
        background \current@color}}
\def\define@color@named#1#2{%
  \expandafter\let\csname col@#1\endcsname\@nnil}
      \def\no@page@color{\special{background \string"newpath clip}}
\def\Ginclude@eps#1{%
 \message{<#1>}%
  \bgroup
  \def\@tempa{!}%
  \dimen@\Gin@req@width
  \dimen@ii.1bp%
  \divide\dimen@\dimen@ii
  \@tempdima\Gin@req@height
  \divide\@tempdima\dimen@ii
    \special{PSfile="#1"\GPT@space
      llx=\Gin@llx\GPT@space
      lly=\Gin@lly\GPT@space
      urx=\Gin@urx\GPT@space
      ury=\Gin@ury\GPT@space
      \ifx\Gin@scalex\@tempa\else rwi=\number\dimen@\GPT@space\fi
      \ifx\Gin@scaley\@tempa\else rhi=\number\@tempdima\GPT@space\fi
      \ifGin@clip clip\fi}%
  \egroup}
\def\Ginclude@bmp#1{%
  \message{<#1>}%
  \dimen@\Gin@req@height
  \advance\dimen@ by-\Gin@lly bp
  \kern-\Gin@llx bp\raise\Gin@req@height\hbox{%
   \ifdim\Gin@urx bp=\z@
     \ifdim\Gin@ury bp=\z@
        \special{em: graph #1}%
     \else
        \special{em: graph #1,\Gin@urx bp}%
     \fi
  \else
        \special{em: graph #1,\Gin@urx bp,\Gin@ury bp}%
  \fi
 }%
}
\def\oztex@include#1#2{%
 \dimen@1bp%
 \divide\Gin@req@width\dimen@
 \divide\Gin@req@height\dimen@
 \special{#1=#2\GPT@space
   \@width=\number\Gin@req@width \GPT@space
   \@height=\number\Gin@req@height}}
\def\Ginclude@pntg{\oztex@include{pntg}}
\def\Ginclude@pict{\oztex@include{pict}}
\def\Grot@start{%
 \special{ps: gsave currentpoint
 currentpoint translate \Grot@angle\GPT@space neg
 rotate neg exch neg exch translate}}
\def\Grot@end{\special{ps: currentpoint grestore moveto}}
\def\Gscale@start{\special{ps:  currentpoint currentpoint translate
  \Gscale@x\GPT@space \Gscale@y\GPT@space scale neg exch neg exch translate}}
\def\Gscale@end{\special{ps:  currentpoint currentpoint translate
  1 \Gscale@x\GPT@space div 1 \Gscale@y\GPT@space div scale
  neg exch neg exch translate}}
\def\Gin@PS@raw#1{\special{ps: #1}}
\def\Gin@PS@restored#1{\special{" #1}}
\def\Gin@PS@literal@header#1{\AtBeginDvi{\special{! #1}}}
\def\Gin@PS@file@header#1{\AtBeginDvi{\special{header=#1}}}
\@ifundefined{ifGin@setpagesize}
  {\expandafter\let\csname ifGin@setpagesize\expandafter\endcsname
                    \csname iftrue\endcsname}
  {}
\ifGin@setpagesize
\ifx\paperwidth\@undefined\else
  \AtBeginDocument{%
% If a package has changed \mag, assume it knows what it is doing
% and leave page size alone
\ifnum\mag=\@m
  \AtBeginDvi{%
    \begingroup
      \@ifundefined{stockwidth}{}
        {%
          \paperwidth\stockwidth
          \paperheight\stockheight
        }%
    \ifdim\paperwidth>\z@
      \ifdim\paperheight>\z@
        \special{papersize=\the\paperwidth,\the\paperheight}%
      \fi
    \fi
    \endgroup}%
    \fi
    }
\fi
\fi
\def\Gin@extensions{.eps,.ps,.eps.gz,.ps.gz,.eps.Z,.mps}
\@namedef{Gin@rule@.ps}#1{{eps}{.ps}{#1}}
\@namedef{Gin@rule@.eps}#1{{eps}{.eps}{#1}}
\@namedef{Gin@rule@.mps}#1{{eps}{.mps}{#1}}
\@namedef{Gin@rule@.pz}#1{{eps}{.bb}{#1}}
\@namedef{Gin@rule@.eps.Z}#1{{eps}{.eps.bb}{#1}}
\@namedef{Gin@rule@.ps.Z}#1{{eps}{.ps.bb}{#1}}
\@namedef{Gin@rule@.ps.gz}#1{{eps}{.ps.bb}{#1}}
\@namedef{Gin@rule@.eps.gz}#1{{eps}{.eps.bb}{#1}}
\@namedef{Gin@rule@*}#1{{eps}{\Gin@ext}{#1}}
\@namedef{Gin@rule@.pcx}#1{{bmp}{}{#1}}
\@namedef{Gin@rule@.bmp}#1{{bmp}{}{#1}}
\@namedef{Gin@rule@.msp}#1{{bmp}{}{#1}}
\@namedef{Gin@rule@.pict}#1{{pict}{}{#1}}
\@namedef{Gin@rule@.pntg}#1{{pntg}{}{#1}}
\endinput
}
\end{verbatim}
This means that when an author writes |\usepackage[dvips]{graphics}|, the file
|dvips.def| is loaded. As another example, the |a4paper| option is declared in
the |article| class to set the |\paperheight| and |\paperwidth| lengths:
\begin{verbatim}
   \DeclareOption{a4paper}{%
      \setlength{\paperheight}{297mm}%
      \setlength{\paperwidth}{210mm}%
   }
\end{verbatim}
Sometimes a user will request an option which the class or package has not
explicitly declared. By default this will produce a warning (for classes) or
error (for packages); this behaviour can be altered as follows:
\begin{verbatim}
   \DeclareOption*{<code>}
\end{verbatim}
For example, to make the package |fred| produce a warning rather than an error
for unknown options, you could specify:
\begin{verbatim}
   \DeclareOption*{%
      \PackageWarning{fred}{Unknown option `\CurrentOption'}%
   }
\end{verbatim}
Then, if an author writes |\usepackage[foo]{fred}|, they will get a warning
\texttt{Package fred Warning: Unknown option `foo'.} As another example, the
|fontenc| package tries to load a file |<ENC>enc.def| whenever the \m{ENC}
option is used. This can be done by writing:
\begin{verbatim}
   \DeclareOption*{%
      \input{\CurrentOption enc.def}%
   }
\end{verbatim}

It is possible to pass options on to another package or class, using the
command |\PassOptionsToPackage| or |\PassOptionsToClass| (note that this is a
specialised operation that works only for option names). For example, to pass
every unknown option on to the |article| class, you can use:
\begin{verbatim}
   \DeclareOption*{%
      \PassOptionsToClass{\CurrentOption}{article}%
   }
\end{verbatim}
If you do this then you should make sure you load the class at some later
point, otherwise the options will never be processed!

So far, we have explained only how to declare options, not how to execute them.
To process the options with which the file was called, you should use:
\begin{verbatim}
   \ProcessOptions\relax
\end{verbatim}
This executes the \m{code} for each option that was both specified and declared
(see Section~\ref{Sec:commands.options} for details of how this is done).

For example, if the |jane| package file contains:
\begin{verbatim}
   \DeclareOption{foo}{\typeout{Saw foo.}}
   \DeclareOption{baz}{\typeout{Saw baz.}}
   \DeclareOption*{\typeout{What's \CurrentOption?}}
   \ProcessOptions\relax
\end{verbatim}
and an author writes |\usepackage[foo,bar]{jane}|, then they will see the
messages \texttt{Saw foo.} and \texttt{What's bar?}

\subsection{Example: a local letter class}

A company may have its own letter class, for setting letters in the
company style.  This section shows a simple implementation of such a
class, although a real class would need more structure.

The class begins by announcing itself as |neplet.cls|.
\begin{verbatim}
   \NeedsTeXFormat{LaTeX2e}
   \ProvidesClass{neplet}[2022-06-01 NonExistent Press letter class]
\end{verbatim}
Then this next bit passes any options on to the |letter| class, which is loaded
with the |a4paper| option.
\begin{verbatim}
   \DeclareOption*{\PassOptionsToClass{\CurrentOption}{letter}}
   \ProcessOptions\relax
   \LoadClass[a4paper]{letter}
\end{verbatim}
In order to use the company letter head, it redefines the |firstpage| page
style: this is the page style that is used on the first page of letters.
\begin{verbatim}
   \renewcommand{\ps@firstpage}{%
      \renewcommand{\@oddhead}{<letterhead goes here>}%
      \renewcommand{\@oddfoot}{<letterfoot goes here>}%
   }
\end{verbatim}
And that's it!

\subsection{Example: a newsletter class}

A simple newsletter can be typeset with \LaTeX{}, using a variant of the
|article| class. The class begins by announcing itself as |smplnews.cls|.
\begin{verbatim}
   \NeedsTeXFormat{LaTeX2e}
   \ProvidesClass{smplnews}[2022-06-01 The Simple News newsletter class]

   \newcommand{\headlinecolor}{\normalcolor}
\end{verbatim}
It passes most specified options on to the |article| class: apart from the
|onecolumn| option, which is switched off, and the |green| option, which sets
the headline in green.
\begin{verbatim}
   \DeclareOption{onecolumn}{\OptionNotUsed}
   \DeclareOption{green}{\renewcommand{\headlinecolor}{\color{green}}}

   \DeclareOption*{\PassOptionsToClass{\CurrentOption}{article}}

   \ProcessOptions\relax
\end{verbatim}
It then loads the class |article| with the option |twocolumn|.
\begin{verbatim}
   \LoadClass[twocolumn]{article}
\end{verbatim}
Since the newsletter is to be printed in colour, it now loads the |color|
package. The class does not specify a device driver option since this should be
specified by the user of the |smplnews| class.
\begin{verbatim}
   \RequirePackage{color}
\end{verbatim}
The class then redefines |\maketitle| to produce the title in 72\,pt Helvetica
bold oblique, in the appropriate colour.
\begin{verbatim}
   \renewcommand{\maketitle}{%
      \twocolumn[%
         \fontsize{72}{80}\fontfamily{phv}\fontseries{b}%
         \fontshape{sl}\selectfont\headlinecolor
         \@title
      ]%
   }
\end{verbatim}
It redefines |\section| and switches off section numbering.
\begin{verbatim}
   \renewcommand{\section}{%
      \@startsection
         {section}{1}{0pt}{-1.5ex plus -1ex minus -.2ex}%
         {1ex plus .2ex}{\large\sffamily\slshape\headlinecolor}%
   }

   \setcounter{secnumdepth}{0}
\end{verbatim}
It also sets the three essential things.
\begin{verbatim}
   \renewcommand{\normalsize}{\fontsize{9}{10}\selectfont}
   \setlength{\textwidth}{17.5cm}
   \setlength{\textheight}{25cm}
\end{verbatim}
In practice, a class would need more than this: it would provide commands for
issue numbers, authors of articles, page styles and so on; but this skeleton
gives a start. The |ltnews| class file is not much more complex than this one.

\end{document}
