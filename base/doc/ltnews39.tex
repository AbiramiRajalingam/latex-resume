% \iffalse meta-comment
%
% Copyright 2023
% The LaTeX Project and any individual authors listed elsewhere
% in this file.
%
% This file is part of the LaTeX base system.
% -——————————————
%
% It may be distributed and/or modified under the
% conditions of the LaTeX Project Public License, either version 1.3c
% of this license or (at your option) any later version.
% The latest version of this license is in
%    https://www.latex-project.org/lppl.txt
% and version 1.3c or later is part of all distributions of LaTeX
% version 2008 or later.
%
% This file has the LPPL maintenance status "maintained".
%
% The list of all files belonging to the LaTeX base distribution is
% given in the file `manifest.txt'. See also `legal.txt' for additional
% information.
%
% The list of derived (unpacked) files belonging to the distribution
% and covered by LPPL is defined by the unpacking scripts (with
% extension .ins) which are part of the distribution.
%
% \fi
% Filename: ltnews39.tex
%
% This is issue 39 of LaTeX News.

\NeedsTeXFormat{LaTeX2e}[2020-02-02]

\documentclass{ltnews}

%%  Maybe needed only for Chris' inadequate system:
\providecommand\Dash {\unskip \textemdash}

%% NOTE:  Chris' preferred hyphens!
%%\showhyphens{parameters}
%%  \hyphenation{because parameters parameter}

\usepackage[T1]{fontenc}

\usepackage{lmodern,url,hologo}

\usepackage{csquotes}
\usepackage{multicol}
\usepackage{color}

\providecommand\hook[1]{\texttt{#1}}

\providecommand\meta[1]{$\langle$\textrm{\itshape#1}$\rangle$}
\providecommand\option[1]{\texttt{#1}}
\providecommand\env[1]{\texttt{#1}}
\providecommand\Arg[1]{\texttt\{\meta{#1}\texttt\}}


\providecommand\eTeX{\hologo{eTeX}}
\providecommand\XeTeX{\hologo{XeTeX}}
\providecommand\LuaTeX{\hologo{LuaTeX}}
\providecommand\pdfTeX{\hologo{pdfTeX}}
\providecommand\MiKTeX{\hologo{MiKTeX}}
\providecommand\CTAN{\textsc{ctan}}
\providecommand\TL{\TeX\,Live}
\providecommand\githubissue[2][]{\ifhmode\unskip\fi
     \quad\penalty500\strut\nobreak\hfill
     \mbox{\small\slshape(%
       \href{https://github.com/latex3/latex2e/issues/\getfirstgithubissue#2 \relax}%
          	    {github issue#1 #2}%
           )}%
     \par\smallskip}
%% But Chris has to mostly disable \href for his TEXPAD app:
%%  \def\href #1{}  % Only For Chris' deficient TeX engine

% simple solution right now (just link to the first issue if there are more)
\def\getfirstgithubissue#1 #2\relax{#1}

\providecommand\sxissue[1]{\ifhmode\unskip
     \else
       % githubissue preceding
       \vskip-\smallskipamount
       \vskip-\parskip
     \fi
     \quad\penalty500\strut\nobreak\hfill
     \mbox{\small\slshape(\url{https://tex.stackexchange.com/#1})}\par}

\providecommand\gnatsissue[2]{\ifhmode\unskip\fi
     \quad\penalty500\strut\nobreak\hfill
     \mbox{\small\slshape(%
       \href{https://www.latex-project.org/cgi-bin/ltxbugs2html?pr=#1\%2F\getfirstgithubissue#2 \relax}%
          	    {gnats issue #1/#2}%
           )}%
     \par}

\let\cls\pkg
\providecommand\env[1]{\texttt{#1}}
\providecommand\acro[1]{\textsc{#1}}

\vbadness=1400  % accept slightly empty columns


\let\finalpagebreak\pagebreak % for TUB (if they use it)

\makeatletter
% maybe not the greatest design but normally we wouldn't have subsubsections
\renewcommand{\subsubsection}{%
   \@startsection      {subsubsection}{2}{0pt}{1.5ex \@plus 1ex \@minus .2ex}%
      {-1em}{\@subheadingfont\colonize}%
}
\providecommand\colonize[1]{#1:}
\makeatother

\let\finalvspace\vspace          % for document layout fixes

% Undo ltnews's \verbatim@font with active < and >
\makeatletter
\def\verbatim@font{%
  \normalsize\ttfamily}
\makeatletter

%%%%%%%%%%%%%%%%%%%%%%%%%%%%%%%%%%%%%%%%%%%%%%%%%%%%%%%%%%%%%%%%%%%%%%%%%%%%%
\providecommand\tubcommand[1]{}
\tubcommand{\input{tubltmac}}

\publicationmonth{June}
\publicationyear{2024  --- DRAFT version for upcoming release}
%\publicationyear{2024}

\publicationissue{39}

\begin{document}

\tubcommand{\addtolength\textheight{4.2pc}}   % only for TUB

\maketitle
{\hyphenpenalty=10000 \exhyphenpenalty=10000 \spaceskip=3.33pt \hbadness=10000
\tableofcontents}

\setlength\rightskip{0pt plus 3em}

\medskip

\section{Introduction}



\section{New or improved commands}


\section{Code improvements}

\subsection{Keep track of lost glyphs}

A while ago we changed the \LaTeX{} default value for
\cs{tracinglostchars} from \texttt{1} to \texttt{2} so that missing
glyphs generate at least a warning, but we forgot to make the same
change to \cs{tracingnone}. Thus, when issuing that command \LaTeX{}
stopped generating warnings about missing glyphs. This has now been
corrected.
%
\githubissue{549}

\subsection{Improve \pkg{fontenc} error message}

If the \pkg{fontenc} is asked to load a font encoding for which it
doesn't find a suitable \texttt{.def} file it generates an error
message indicating that the encoding name might be misspelled. That
is, of course, one of the possible causes, but another one is that the
installation is missing a necessary support package, e.g., that no
support for Cyrillic fonts has been installed. The error message text
has therefore been extended to explain the issue more generally.
%
\githubissue{1102}


\subsection{Warn if counter names are problematic}

In the past it was possible to declare, for example,
\verb/\newcounter{index}/ with the side-effect that this defines
\cs{theindex}, even though \LaTeX{} has a \env{theindex} environment
that then got clobbered by the declaration.
%
This has now been changed: if \cs{the}\meta{counter} is already
defined it is not altered, but instead a warning message is displayed.
%
\githubissue{823}

\subsection{Extended information in \cs{listfiles}}

The \cs{listfiles} command provides useful information when finding issues
related to variation in package versions. However, this has to date relied on
the information in the \cs{ProvidesPackage} line, or similar: that can be
misleading if for example a file has been edited locally. We have now extended
\cs{listfiles} to take an optional argument which will then include the MD5
hash of each file (and the size of each file) in the \texttt{.log}. Thus for
example you can use
\begin{verbatim}
\listfiles[hashes,sizes]
\end{verbatim}
to get both the file sizes and file hashes in the \texttt{.log} as well as
the standard release information.
%
\githubissue{945}

\subsection{Optimise creation of simple document commands}

Creating document commands using \cs{NewDocumentCommand}, etc., provides a very
flexible way of grabbing arguments. When the document command only takes simple
mandatory arguments, this has to-date added an overhead that could be avoided.
We have now refined the internal code path such that \enquote{simple} document
commands avoid almost any overhead at point-of-use, making the results
essentially as efficient as using \cs{newcommand} for low-level \TeX{}
constructs. Note that as \cs{NewDocumentCommand} makes engine-robust commands,
the direct equivalent to \cs{newcommand} is \cs{NewExpandableDocumentCommand}.
%
\githubissue{1189}


%\section{Bug fixes}

%\section{Changes to packages in the \pkg{amsmath} category}

%\section{Changes to packages in the \pkg{graphics} category}



\section{Changes to packages in the \pkg{tools} category}

\subsection{\pkg{verbatim}: \cs{verb} showed visible spaces}

A recent change in the kernel was not reflected in the \pkg{verbatim}
package with the result that \cs{verb} showed visible spaces
(\verb*/ /) after the package was loaded. This has already been corrected
in a hotfix for release 2023-11.
%
\githubissue{1160}


\subsection{\pkg{multicol}: \cs{columnbreak} interfers with mark mechanism}

The \pkg{multicol} package has to keep track of marks (from
\cs{markright} or \cs{markboth}) as part of its output routine code
and can't rely on \LaTeX{} handling that automatically. It does so by
artifically splitting page data with \cs{vsplit} to extract the mark
data. With the introduction of \cs{columnbreak} that code failed
sometimes, because it was not seeing any mark that followed such a
forced column break.

This has now been corrected, but there is further work to do, because
as of now \pkg{multicol} does not handle marks using the new mark
mechanism at all.
%
\githubissue{1130}


\section{Changes to files in the \pkg{cyrillic} category}

\subsection{Correct definition of \cs{k}}

Ages ago, the encoding specific definitions for various accent
commands were changed to guard against altering some parameter values
non-locally by mistake. For some reason the definition for \cs{k} in
the Cyrillic encodings \texttt{T2A}, \texttt{T2B}, and \texttt{T2C}
didn't get this treatment. This oversight has now been corrected.
%
\githubissue{1148}



\begin{thebibliography}{9}

%\fontsize{9.3}{11.3}\selectfont


%\bibitem{39:Lamport}
%Leslie Lamport.
%\newblock \emph{{\LaTeX}: {A} Document Preparation System: User's Guide and Reference
%  Manual}.
%\newblock \mbox{Addison}-Wesley, Reading, MA, USA, 2nd edition, 1994.
%\newblock ISBN 0-201-52983-1.
%\newblock Reprinted with corrections in 1996.
%
%
%\bibitem{39:ltnews} \LaTeX{} Project Team.
%  \emph{\LaTeXe{} news 1--39}.\\
%  \url{https://latex-project.org/news/latex2e-news/ltnews.pdf}
%
%\bibitem{39:blueprint} Frank Mittelbach and Chris Rowley.
%  \emph{\LaTeX{} Tagged PDF \Dash A blueprint for a large project}.\\
%  \url{https://latex-project.org/publications/indexbyyear/2020/}

\end{thebibliography}

\end{document}
