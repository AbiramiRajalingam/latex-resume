% \iffalse meta-comment
%
% Copyright 2023
% The LaTeX Project and any individual authors listed elsewhere
% in this file.
%
% This file is part of the LaTeX base system.
% -——————————————
%
% It may be distributed and/or modified under the
% conditions of the LaTeX Project Public License, either version 1.3c
% of this license or (at your option) any later version.
% The latest version of this license is in
%    https://www.latex-project.org/lppl.txt
% and version 1.3c or later is part of all distributions of LaTeX
% version 2008 or later.
%
% This file has the LPPL maintenance status "maintained".
%
% The list of all files belonging to the LaTeX base distribution is
% given in the file `manifest.txt'. See also `legal.txt' for additional
% information.
%
% The list of derived (unpacked) files belonging to the distribution
% and covered by LPPL is defined by the unpacking scripts (with
% extension .ins) which are part of the distribution.
%
% \fi
% Filename: ltnews39.tex
%
% This is issue 39 of LaTeX News.

\NeedsTeXFormat{LaTeX2e}[2020-02-02]

\documentclass{ltnews}

%%  Maybe needed only for Chris' inadequate system:
\providecommand\Dash {\unskip \textemdash}

%% NOTE:  Chris' preferred hyphens!
%%\showhyphens{parameters}
%%  \hyphenation{because parameters parameter}

\usepackage[T1]{fontenc}

\usepackage{lmodern,url,hologo}

\usepackage{csquotes}
\usepackage{multicol}
\usepackage{color}

\providecommand\hook[1]{\texttt{#1}}

\providecommand\meta[1]{$\langle$\textrm{\itshape#1}$\rangle$}
\providecommand\option[1]{\texttt{#1}}
\providecommand\env[1]{\texttt{#1}}
\providecommand\Arg[1]{\texttt\{\meta{#1}\texttt\}}


\providecommand\eTeX{\hologo{eTeX}}
\providecommand\XeTeX{\hologo{XeTeX}}
\providecommand\LuaTeX{\hologo{LuaTeX}}
\providecommand\pdfTeX{\hologo{pdfTeX}}
\providecommand\MiKTeX{\hologo{MiKTeX}}
\providecommand\CTAN{\textsc{ctan}}
\providecommand\TL{\TeX\,Live}
\providecommand\githubissue[2][]{\ifhmode\unskip\fi
     \quad\penalty500\strut\nobreak\hfill
     \mbox{\small\slshape(%
       \href{https://github.com/latex3/latex2e/issues/\getfirstgithubissue#2 \relax}%
          	    {github issue#1 #2}%
           )}%
     \par\smallskip}
%% But Chris has to mostly disable \href for his TEXPAD app:
%%  \def\href #1{}  % Only For Chris' deficient TeX engine

% simple solution right now (just link to the first issue if there are more)
\def\getfirstgithubissue#1 #2\relax{#1}

\providecommand\sxissue[1]{\ifhmode\unskip
     \else
       % githubissue preceding
       \vskip-\smallskipamount
       \vskip-\parskip
     \fi
     \quad\penalty500\strut\nobreak\hfill
     \mbox{\small\slshape(\url{https://tex.stackexchange.com/#1})}\par}

\providecommand\gnatsissue[2]{\ifhmode\unskip\fi
     \quad\penalty500\strut\nobreak\hfill
     \mbox{\small\slshape(%
       \href{https://www.latex-project.org/cgi-bin/ltxbugs2html?pr=#1\%2F\getfirstgithubissue#2 \relax}%
          	    {gnats issue #1/#2}%
           )}%
     \par}

\let\cls\pkg
\providecommand\env[1]{\texttt{#1}}
\providecommand\acro[1]{\textsc{#1}}

\vbadness=1400  % accept slightly empty columns


\let\finalpagebreak\pagebreak % for TUB (if they use it)

\makeatletter
% maybe not the greatest design but normally we wouldn't have subsubsections
\renewcommand{\subsubsection}{%
   \@startsection      {subsubsection}{2}{0pt}{1.5ex \@plus 1ex \@minus .2ex}%
      {-1em}{\@subheadingfont\colonize}%
}
\providecommand\colonize[1]{#1:}
\makeatother

\let\finalvspace\vspace          % for document layout fixes

% Undo ltnews's \verbatim@font with active < and >
\makeatletter
\def\verbatim@font{%
  \normalsize\ttfamily}
\makeatletter

%%%%%%%%%%%%%%%%%%%%%%%%%%%%%%%%%%%%%%%%%%%%%%%%%%%%%%%%%%%%%%%%%%%%%%%%%%%%%
\providecommand\tubcommand[1]{}
\tubcommand{\input{tubltmac}}

\publicationmonth{June}
\publicationyear{2024  --- DRAFT version for upcoming release}
%\publicationyear{2024}

\publicationissue{39}

\begin{document}

\tubcommand{\addtolength\textheight{4.2pc}}   % only for TUB

\maketitle
{\hyphenpenalty=10000 \exhyphenpenalty=10000 \spaceskip=3.33pt \hbadness=10000
\tableofcontents}

\setlength\rightskip{0pt plus 3em}

\medskip

\section{Introduction}

\emph{to write}


\section{Enhancements to the new mark mechanism}

In June 2022 we introduced a new mark mechanism in
\LaTeX{}~\cite[p.~76]{39:ltnews} that allows keeping track of multiple
independent marks. It also  properly supports top marks, something that wasn't
reliably possible with \LaTeX{} before.

There was, however, one limitation: to retrieve the marks from the
page data it was necessary to \cs{vsplit} that data artificially so
that \TeX{} would produce split marks that the mechanism could then
use. Unfortunately, \TeX{} gets very upset if it finds infinite
negative glue (e.g., from \cs{vss}) within this data. This is not
totally surprising because such glue would allow splitting off any
amount of material as such glue would hide the size of it. \TeX{}
therefore responds with an error message if it find such glue while
doing a \cs{vsplit} operation (and it does so even if a later glue
item cancels the infinite glue).

To account for this, the code in 2022 attempted to detect this
situation beforehand and if so did not do any splitting but, of
course, it would then also not extract any mark information.

In this release the approach has been changed and we always do a
\cs{vsplit} operation and thus always get the right mark data
extracted. While it is not possible to avoid upsetting \TeX{} in case
we have infinite negative glue present, it is possible to hide this
(more or less) from the user.\footnote{A note to \pkg{l3build} users
that make use of its testing capabilities: the new mechanism
temporarily changes \cs{interactionmode} and, for implementation
reasons in \TeX{}, that results in extra newlines in the \texttt{.log}
file, so instead of seeing \texttt{[1] [2]} you will see each on
separate lines. This means that test files might show difference of
that nature, once the code is active, and must therefore be
regenerated as necessary.}  With the new code \TeX{} will neither stop
nor show anything on the terminal. What we can't do, though, is to
avoid that an error is written to the log file, but to make it clear
that this error is harmless and should be ignored we have arranged the
code so that the error message, if it is issued, takes the following
format:
\begin{verbatim}
! Infinite glue shrinkage found in box being split.
<argument> Infinite shrink error above ignored ! 
l. ...  }
\end{verbatim}
Not perfect (especially the somewhat unmotivated \texttt{<argument>}),
but you can only do so much if error messages and their texts are
hard-wired in the engine.

So why all this? There are two reasons: we do not lose marks in edge
cases any longer and perhaps more importantly we are now also reliably
able to extract marks from arbitrarily boxed data, something that
wasn't possible at all before. This is, for example, necessary to
support extended marks in \env{multicols} environments or extract them
from floats, marginpars, etc.

Details about the implementation can be found in \texttt{texdoc
  ltmarks-code} or in the shorter \texttt{texdoc ltmarks-doc} (which
only describes the general concepts and  the command interfaces).

\section{New or improved commands}

\subsection{\pkg{doc}: provide \cs{ProvideDocElement}}

Beside \cs{NewDocElement} and \cs{RenewDocElement} we now also offer a
\cs{ProvideDocElement} declaration that does nothing unless the doc
element could be declared with \cs{NewDocElement}. This can be useful
if documentation files are processed both individually as well as
combined.


\subsection{\pkg{doc}: better support for \pkg{upquote}}

In \LaTeX{} News~37~\cite{39:ltnews37} we wrote that support for the
\pkg{upquote} package was added to the \pkg{doc} package, but back
then this was only done for \cs{verb}, and the \env{verbatim}
environments. However, the bulk of code in a typical \texttt{.dtx} file is
within \env{macrocode} or \env{macrocode*} environments which were not
affected by adding \pkg{upquote}. We have now updated those, such that
\pkg{upquote} alters the quote characters in these environments as
well.
%
\githubissue{1230}


\section{Code improvements}

\subsection{Loading packages at the top level}
Classes and packages should only be loaded with \cs{documentclass}, \cs{usepackage},
or class interface commands such as \cs{LoadClass} or \cs{RequirePackageWithOptions}
at the top level, not inside a group. Previously \LaTeX\ did not check this,
which would often lead to low level errors later on if package declarations
were reverted as a group ended.
\LaTeX\ now checks the group level and an error is thrown
if the class or package is loaded in a group.
%
\githubissue{1185}

\subsection{Keep track of lost glyphs}

A while ago we changed the \LaTeX{} default value for
\cs{tracinglostchars} from \texttt{1} to \texttt{2} so that missing
glyphs generate at least a warning, but we forgot to make the same
change to \cs{tracingnone}. Thus, when issuing that command \LaTeX{}
stopped generating warnings about missing glyphs. This has now been
corrected.
%
\githubissue{549}

\subsection{Improve \pkg{fontenc} error message}

If the \pkg{fontenc} is asked to load a font encoding for which it
doesn't find a suitable \texttt{.def} file it generates an error
message indicating that the encoding name might be misspelled. That
is, of course, one of the possible causes, but another one is that the
installation is missing a necessary support package, e.g., that no
support for Cyrillic fonts has been installed. The error message text
has therefore been extended to explain the issue more generally.
%
\githubissue{1102}


\subsection{Warn if counter names are problematic}

In the past it was possible to declare, for example,
\verb/\newcounter{index}/ with the side-effect that this defines
\cs{theindex}, even though \LaTeX{} has a \env{theindex} environment
that then got clobbered by the declaration.
%
This has now been changed: if \cs{the}\meta{counter} is already
defined it is not altered, but instead a warning message is displayed.
%
\githubissue{823}

\subsection{Extended information in \cs{listfiles}}

The \cs{listfiles} command provides useful information when finding issues
related to variation in package versions. However, this has to date relied on
the information in the \cs{ProvidesPackage} line, or similar: that can be
misleading if for example a file has been edited locally. We have now extended
\cs{listfiles} to take an optional argument which will then include the MD5
hash of each file (and the size of each file) in the \texttt{.log}. Thus for
example you can use
\begin{verbatim}
\listfiles[hashes,sizes]
\end{verbatim}
to get both the file sizes and file hashes in the \texttt{.log} as well as
the standard release information.
%
\githubissue{945}

\subsection{Optimize creation of simple document commands}

Creating document commands using \cs{NewDocumentCommand}, etc., provides a very
flexible way of grabbing arguments. When the document command only takes simple
mandatory arguments, this has to-date added an overhead that could be avoided.
We have now refined the internal code path such that \enquote{simple} document
commands avoid almost any overhead at point-of-use, making the results
essentially as efficient as using \cs{newcommand} for low-level \TeX{}
constructs. Note that as \cs{NewDocumentCommand} makes engine-robust commands,
the direct equivalent to \cs{newcommand} is \cs{NewExpandableDocumentCommand}.
%
\githubissue{1189}

\subsection{Declaring appropriate sub-encodings for \texttt{TS1} symbol fonts}

In 2020 we incorporated support for the \texttt{TS1} symbol encoding
directly into the kernel and in this way removed the need to load the
\pkg{textcomp} package~\cite{39:ltnews31} to make commands such as
\cs{texteuro} available.

There is, however, a big problem with this \texttt{TS1} symbol
encoding: only very few fonts can provide every glyph that is supposed
to be part of \texttt{TS1}. This means that changing font families
might result in certain symbols becoming unavailable. This can be a
major disaster if, for example, the \cs{texteuro} (\texteuro) or the
\cs{textohm} (\textohm) are no longer printed in your document, just
because you altered your text font family.

To mitigate this problem, we also introduced in 2020 the declaration
\cs{DeclareEncodingSubset}. This declaration is supposed to be used in
font definition files for the \texttt{TS1} encoding to specify which
subset (we have defined 10 common ones) a specific font implements. If
such a declaration is used then missing symbols are automatically
taking from a fallback font. While this is not perfect, it is the best
you can do other than painstrickenly checking that your document only
uses glyphs that the font supports and if necessary switch to a
different font or avoid the missing symbols. See also the discussion
in \cite{39:ltnews33}.

To jumpstart the process we also added declarations to the \LaTeX{}
kernel for most of the fonts found in \TeX{}Live at the time\Dash with
the assumption that such declarations would over time be superseded by
declarations in the \texttt{.fd} files. Unfortunately, this hasn't
happened yet (or not often) and so many of the initial declarations
went stale: several fonts got new glyphs added to them (so their
sub-encoding should have been changed but didn't); others (mainly due
to license issues) changed the family name and thus our declarations
became useless and the renamed fonts (now without a declaration) ended
up in the default sub-encoding which offers only few glyphs; yet
others such as CharisSIL (which triggered the GitHub issue) were
simply not around at the time.

We have therefore, again attempted to provide the (currently) correct
declarations, but it is obvious that this is not a workable
process. As we do not maintain the fonts we do not have the
information that something has changed, and to regularly check the
ever growing font support bundles is simply not possible. It is
therefore very important that maintainers of font packages do not only
provide \texttt{.fd} files but also add such a declaration to every
\texttt{TS1...fd} font definition file that they distribute.

To simplify this process, we now provide a simple \LaTeX{} file
(\texttt{checkencodingsubset.tex}) for determining the correct (safe)
sub-encoding. If run, it asks for a font family and then outputs its
findings, for example, for \texttt{AlgolRevived-TLF} you will get:
\begin{verbatim}
-----------------------------------------
Testing font family AlgolRevived-TLF
(currently TS1-sub-encoding 9)
-----------------------------------------
Some glyphs are missing from sub-encoding 8:
   ==> \textcelsius (137) is missing
   ==> \texttwosuperior (178) is missing
   ==> \textthreesuperior (179) is missing
   ==> \textonesuperior (185) is missing
Some glyphs are missing from sub-encoding 7:
   ==> \texteuro (191) is missing
All glyphs between sub-encoding 6 and 7 exist
All glyphs between sub-encoding 5 and 6 exist
All glyphs between sub-encoding 4 and 5 exist
Some glyphs are missing from sub-encoding 3:
   ==> \textwon (142) is missing
All glyphs between sub-encoding 2 and 3 exist
Some glyphs are missing from sub-encoding 1:
   ==> \textmho (77) is missing
   ==> \textpertenthousand (152) is missing
All glyphs between sub-encoding 0 and 1 exist
All glyphs in core exist
-----------------------------------------
TS1 encoding subset for AlgolRevived-TLF (ok)
Use sub-encoding 9
-----------------------------------------
\end{verbatim}
This output is meant for human consumption, e.g., you see which glyphs
are missing and why a certain sub-encoding is suggested, but it is not
that hard to use it in a script and extract the suggested sub-encoding
by grepping for the line starting with \texttt{Use sub-encoding}.

Of course, this check will only work if the missing glyphs are really
missing: some fonts placed \enquote{tofu}\footnote{Little squares to
indicate a missing symbol.} into such slots and in this case it looks
to \TeX{} as if the glyph is provided. For example, for the old
Palatino fonts (family \texttt{ppl}) it would report
\begin{verbatim}
-----------------------------------------
TS1 encoding subset for ppl (bad)
Use sub-encoding 0 (not 5)
-----------------------------------------
\end{verbatim}
i.e., all glyphs are provided, while in reality more than twenty are
missing and sub-encoding 5 as declared in the kernel is correct.
%
\githubissue{1257}



%\section{Bug fixes}

%\section{Changes to packages in the \pkg{amsmath} category}

%\section{Changes to packages in the \pkg{graphics} category}



\section{Changes to packages in the \pkg{tools} category}

\subsection{\pkg{verbatim}: \cs{verb} showed visible spaces}

A recent change in the kernel was not reflected in the \pkg{verbatim}
package with the result that \cs{verb} showed visible spaces
(\verb*/ /) after the package was loaded. This has already been corrected
in a hotfix for release 2023-11.
%
\githubissue{1160}

\subsection{\pkg{verbatim}: Support tabs in \cs{verbatiminput*}}

Mimicking the kernel update (November, 2023) that allowed \cs{verb*}
to mark tabs as spaces, the \pkg{verbatim} package has been updated
such that \cs{verbatiminput*} marks tabs as spaces as well.
%
\githubissue{1245}


\subsection{\pkg{multicol}: \cs{columnbreak} interferes with mark mechanism}

The \pkg{multicol} package has to keep track of marks (from
\cs{markright} or \cs{markboth}) as part of its output routine code
and can't rely on \LaTeX{} handling that automatically. It does so by
artificially splitting page data with \cs{vsplit} to extract the mark
data. With the introduction of \cs{columnbreak} that code failed
sometimes, because it was not seeing any mark that followed such a
forced column break.

This has now been corrected, but there is further work to do, because
as of now \pkg{multicol} does not yet handle marks using the new mark
mechanism\Dash see the discussion at the beginning of the newsletter.
%
\githubissue{1130}

\subsection{\pkg{showkeys}: Allow \cs{newline} in \pkg{amsthm} to work}
%
Previously \pkg{showkeys} added an extra box layer which disabled the \cs{newline}
of \pkg{amsthm} theorem styles. This extra box has now been avoided.
%
\githubissue{1123}


\section{Changes to files in the \pkg{cyrillic} category}

\subsection{Correct definition of \cs{k}}

Ages ago, the encoding specific definitions for various accent
commands were changed to guard against altering some parameter values
non-locally by mistake. For some reason the definition for \cs{k} in
the Cyrillic encodings \texttt{T2A}, \texttt{T2B}, and \texttt{T2C}
didn't get this treatment. This oversight has now been corrected.
%
\githubissue{1148}



\begin{thebibliography}{9}

%\fontsize{9.3}{11.3}\selectfont


%\bibitem{39:Lamport}
%Leslie Lamport.
%\newblock \emph{{\LaTeX}: {A} Document Preparation System: User's Guide and Reference
%  Manual}.
%\newblock \mbox{Addison}-Wesley, Reading, MA, USA, 2nd edition, 1994.
%\newblock ISBN 0-201-52983-1.
%\newblock Reprinted with corrections in 1996.
%

\bibitem{39:ltnews} \LaTeX{} Project Team.
  \emph{\LaTeXe{} news 1--39}. June, 2024.
  \url{https://latex-project.org/news/latex2e-news/ltnews.pdf}
%
%
\bibitem{39:ltnews31} \LaTeX{} Project Team.
  \emph{\LaTeXe{} news 31}. February, 2020.
  \url{https://latex-project.org/news/latex2e-news/ltnews31.pdf}

\bibitem{39:ltnews33} \LaTeX{} Project Team.
  \emph{\LaTeXe{} news 33}. June 2021.\\
  \url{https://latex-project.org/news/latex2e-news/ltnews33.pdf}

\bibitem{39:ltnews37} \LaTeX{} Project Team.
  \emph{\LaTeXe{} news 37}. June 2023.\\
  \url{https://latex-project.org/news/latex2e-news/ltnews37.pdf}

%\bibitem{39:blueprint} Frank Mittelbach and Chris Rowley.
%  \emph{\LaTeX{} Tagged PDF \Dash A blueprint for a large project}.\\
%  \url{https://latex-project.org/publications/indexbyyear/2020/}

\end{thebibliography}

\end{document}
