% \iffalse meta-comment
%
% Copyright 2023
% The LaTeX Project and any individual authors listed elsewhere
% in this file.
%
% This file is part of the LaTeX base system.
% -——————————————
%
% It may be distributed and/or modified under the
% conditions of the LaTeX Project Public License, either version 1.3c
% of this license or (at your option) any later version.
% The latest version of this license is in
%    https://www.latex-project.org/lppl.txt
% and version 1.3c or later is part of all distributions of LaTeX
% version 2008 or later.
%
% This file has the LPPL maintenance status "maintained".
%
% The list of all files belonging to the LaTeX base distribution is
% given in the file `manifest.txt'. See also `legal.txt' for additional
% information.
%
% The list of derived (unpacked) files belonging to the distribution
% and covered by LPPL is defined by the unpacking scripts (with
% extension .ins) which are part of the distribution.
%
% \fi
% Filename: ltnews39.tex
%
% This is issue 39 of LaTeX News.

\NeedsTeXFormat{LaTeX2e}[2020-02-02]

\documentclass{ltnews}

%%  Maybe needed only for Chris' inadequate system:
\providecommand\Dash {\unskip \textemdash}

%% NOTE:  Chris' preferred hyphens!
%%\showhyphens{parameters}
%%  \hyphenation{because parameters parameter}

\usepackage[T1]{fontenc}

\usepackage{lmodern,url,hologo}

\usepackage{csquotes}
\usepackage{multicol}
\usepackage{color}

\providecommand\hook[1]{\texttt{#1}}

\providecommand\meta[1]{$\langle$\textrm{\itshape#1}$\rangle$}
\providecommand\option[1]{\texttt{#1}}
\providecommand\env[1]{\texttt{#1}}
\providecommand\Arg[1]{\texttt\{\meta{#1}\texttt\}}


\providecommand\eTeX{\hologo{eTeX}}
\providecommand\XeTeX{\hologo{XeTeX}}
\providecommand\LuaTeX{\hologo{LuaTeX}}
\providecommand\pdfTeX{\hologo{pdfTeX}}
\providecommand\MiKTeX{\hologo{MiKTeX}}
\providecommand\CTAN{\textsc{ctan}}
\providecommand\TL{\TeX\,Live}
\providecommand\githubissue[2][]{\ifhmode\unskip\fi
     \quad\penalty500\strut\nobreak\hfill
     \mbox{\small\slshape(%
       \href{https://github.com/latex3/latex2e/issues/\getfirstgithubissue#2 \relax}%
          	    {github issue#1 #2}%
           )}%
     \par\smallskip}
%% But Chris has to mostly disable \href for his TEXPAD app:
%%  \def\href #1{}  % Only For Chris' deficient TeX engine

% simple solution right now (just link to the first issue if there are more)
\def\getfirstgithubissue#1 #2\relax{#1}

\providecommand\sxissue[1]{\ifhmode\unskip
     \else
       % githubissue preceding
       \vskip-\smallskipamount
       \vskip-\parskip
     \fi
     \quad\penalty500\strut\nobreak\hfill
     \mbox{\small\slshape(\url{https://tex.stackexchange.com/#1})}\par}

\providecommand\gnatsissue[2]{\ifhmode\unskip\fi
     \quad\penalty500\strut\nobreak\hfill
     \mbox{\small\slshape(%
       \href{https://www.latex-project.org/cgi-bin/ltxbugs2html?pr=#1\%2F\getfirstgithubissue#2 \relax}%
          	    {gnats issue #1/#2}%
           )}%
     \par}

\let\cls\pkg
\providecommand\env[1]{\texttt{#1}}
\providecommand\acro[1]{\textsc{#1}}

\vbadness=1400  % accept slightly empty columns


\let\finalpagebreak\pagebreak % for TUB (if they use it)

\makeatletter
% maybe not the greatest design but normally we wouldn't have subsubsections
\renewcommand{\subsubsection}{%
   \@startsection      {subsubsection}{2}{0pt}{1.5ex \@plus 1ex \@minus .2ex}%
      {-1em}{\@subheadingfont\colonize}%
}
\providecommand\colonize[1]{#1:}
\makeatother

\let\finalvspace\vspace          % for document layout fixes

% Undo ltnews's \verbatim@font with active < and >
\makeatletter
\def\verbatim@font{%
  \normalsize\ttfamily}
\makeatletter

%%%%%%%%%%%%%%%%%%%%%%%%%%%%%%%%%%%%%%%%%%%%%%%%%%%%%%%%%%%%%%%%%%%%%%%%%%%%%
\providecommand\tubcommand[1]{}
\tubcommand{\input{tubltmac}}

\publicationmonth{June}
\publicationyear{2024  --- DRAFT version for upcoming release}
%\publicationyear{2024}

\publicationissue{39}

\begin{document}

\tubcommand{\addtolength\textheight{4.2pc}}   % only for TUB

\maketitle
{\hyphenpenalty=10000 \exhyphenpenalty=10000 \spaceskip=3.33pt \hbadness=10000
\tableofcontents}

\setlength\rightskip{0pt plus 3em}

\medskip

\section{Introduction}


\section{Enhancements to the new mark mechanism}

In June 2022 we introduced a new mark mechanism in
\LaTeX{}~\cite[p.~76]{39:ltnews} that allows keeping track of multiple
independent marks. It also  properly supports top marks, something that wasn't
reliably possible with \LaTeX{} before.

There was, however, one limitation: to retrieve the marks from the
page data it was necessary to \cs{vsplit} that data artificially so
that \TeX{} would produce split marks that the mechanism could then
use. Unfortunately, \TeX{} gets very upset if it finds infinite
negative glue (e.g., from \cs{vss}) within this data. This is not
totally surprising because such glue would allow splitting off any
amount of material as such glue would hide the size of it. \TeX{}
therefore responds with an error message if it find such glue while
doing a \cs{vsplit} operation (and it does so even if a later glue
item cancels the infinite glue).

To account for this, the code in 2022 attempted to detect this
situation beforehand and if so did not do any splitting but, of
course, it would then also not extract any mark information.

In this release the approach has been changed and we always do a
\cs{vsplit} operation and thus always get the right mark data
extracted. While it is not possible to avoid upsetting \TeX{} in case
we have infinite negative glue present, it is possible to hide this
(more or less) from the user.\footnote{A note to \pkg{l3build} users
that make use of its testing capabilities: the new mechanism
temporarily changes \cs{interactionmode} and, for implementation
reasons in \TeX{}, that results in extra newlines in the \texttt{.log}
file, so instead of seeing \texttt{[1] [2]} you will see each on
separate lines. This means that test files might show difference of
that nature, once the code is active, and must therefore be
regenerated as necessary.}  With the new code \TeX{} will neither stop
nor show anything on the terminal. What we can't do, though, is to
avoid that an error is written to the log file, but to make it clear
that this error is harmless and should be ignored we have arranged the
code so that the error message, if it is issued, takes the following
format:
\begin{verbatim}
! Infinite glue shrinkage found in box being split.
<argument> Infinite shrink error above ignored ! 
l. ...  }
\end{verbatim}
Not perfect (especially the somewhat unmotivated \texttt{<argument>}),
but you can only do so much if error messages and their texts are
hard-wired in the engine.

So why all this? There are two reasons: we do not lose marks in edge
cases any longer and perhaps more importantly we are now also reliably
able to extract marks from arbitrarily boxed data, something that
wasn't possible at all before. This is, for example, necessary to
support extended marks in \env{multicols} environments or extract them
from floats, marginpars, etc.

\section{New or improved commands}


\section{Code improvements}

\subsection{Keep track of lost glyphs}

A while ago we changed the \LaTeX{} default value for
\cs{tracinglostchars} from \texttt{1} to \texttt{2} so that missing
glyphs generate at least a warning, but we forgot to make the same
change to \cs{tracingnone}. Thus, when issuing that command \LaTeX{}
stopped generating warnings about missing glyphs. This has now been
corrected.
%
\githubissue{549}

\subsection{Improve \pkg{fontenc} error message}

If the \pkg{fontenc} is asked to load a font encoding for which it
doesn't find a suitable \texttt{.def} file it generates an error
message indicating that the encoding name might be misspelled. That
is, of course, one of the possible causes, but another one is that the
installation is missing a necessary support package, e.g., that no
support for Cyrillic fonts has been installed. The error message text
has therefore been extended to explain the issue more generally.
%
\githubissue{1102}


\subsection{Warn if counter names are problematic}

In the past it was possible to declare, for example,
\verb/\newcounter{index}/ with the side-effect that this defines
\cs{theindex}, even though \LaTeX{} has a \env{theindex} environment
that then got clobbered by the declaration.
%
This has now been changed: if \cs{the}\meta{counter} is already
defined it is not altered, but instead a warning message is displayed.
%
\githubissue{823}



%\section{Bug fixes}

%\section{Changes to packages in the \pkg{amsmath} category}

%\section{Changes to packages in the \pkg{graphics} category}



\section{Changes to packages in the \pkg{tools} category}

\subsection{\pkg{verbatim}: \cs{verb} showed visible spaces}

A recent change in the kernel was not reflected in the \pkg{verbatim}
package with the result that \cs{verb} showed visible spaces
(\verb*/ /) after the package was loaded. This has already been corrected
in a hotfix for release 2023-11.
%
\githubissue{1160}


\subsection{\pkg{multicol}: \cs{columnbreak} interferes with mark mechanism}

The \pkg{multicol} package has to keep track of marks (from
\cs{markright} or \cs{markboth}) as part of its output routine code
and can't rely on \LaTeX{} handling that automatically. It does so by
artificially splitting page data with \cs{vsplit} to extract the mark
data. With the introduction of \cs{columnbreak} that code failed
sometimes, because it was not seeing any mark that followed such a
forced column break.

This has now been corrected, but there is further work to do, because
as of now \pkg{multicol} does not yet handle marks using the new mark
mechanism\Dash see the discussion at the beginning of the newsletter.
%
\githubissue{1130}


\section{Changes to files in the \pkg{cyrillic} category}

\subsection{Correct definition of \cs{k}}

Ages ago, the encoding specific definitions for various accent
commands were changed to guard against altering some parameter values
non-locally by mistake. For some reason the definition for \cs{k} in
the Cyrillic encodings \texttt{T2A}, \texttt{T2B}, and \texttt{T2C}
didn't get this treatment. This oversight has now been corrected.
%
\githubissue{1148}



\begin{thebibliography}{9}

%\fontsize{9.3}{11.3}\selectfont


%\bibitem{39:Lamport}
%Leslie Lamport.
%\newblock \emph{{\LaTeX}: {A} Document Preparation System: User's Guide and Reference
%  Manual}.
%\newblock \mbox{Addison}-Wesley, Reading, MA, USA, 2nd edition, 1994.
%\newblock ISBN 0-201-52983-1.
%\newblock Reprinted with corrections in 1996.
%

\bibitem{39:ltnews} \LaTeX{} Project Team.
  \emph{\LaTeXe{} news 1--39}.\\
  \url{https://latex-project.org/news/latex2e-news/ltnews.pdf}
%
%\bibitem{39:blueprint} Frank Mittelbach and Chris Rowley.
%  \emph{\LaTeX{} Tagged PDF \Dash A blueprint for a large project}.\\
%  \url{https://latex-project.org/publications/indexbyyear/2020/}

\end{thebibliography}

\end{document}
