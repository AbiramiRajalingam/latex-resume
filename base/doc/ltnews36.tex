% \iffalse meta-comment
%
% Copyright 2021-2022
% The LaTeX Project and any individual authors listed elsewhere
% in this file.
%
% This file is part of the LaTeX base system.
% -——————————————
%
% It may be distributed and/or modified under the
% conditions of the LaTeX Project Public License, either version 1.3c
% of this license or (at your option) any later version.
% The latest version of this license is in
%    https://www.latex-project.org/lppl.txt
% and version 1.3c or later is part of all distributions of LaTeX
% version 2008 or later.
%
% This file has the LPPL maintenance status "maintained".
%
% The list of all files belonging to the LaTeX base distribution is
% given in the file `manifest.txt'. See also `legal.txt' for additional
% information.
%
% The list of derived (unpacked) files belonging to the distribution
% and covered by LPPL is defined by the unpacking scripts (with
% extension .ins) which are part of the distribution.
%
% \fi
% Filename: ltnews36.tex
%
% This is issue 36 of LaTeX News.

\NeedsTeXFormat{LaTeX2e}[2020-02-02]

\documentclass{ltnews}

%%  Maybe needed only for Chris' inadequate system:
\providecommand\Dash {\unskip \textemdash}

%% NOTE:  Chris' preferred hyphens!
%%\showhyphens{parameters}
%%  \hyphenation{because parameters parameter}

\usepackage[T1]{fontenc}

\usepackage{lmodern,url,hologo}

\usepackage{csquotes}
\usepackage{multicol}
\usepackage{color}

\providecommand\hook[1]{\texttt{#1}}

\providecommand\meta[1]{$\langle$\textrm{\itshape#1}$\rangle$}
\providecommand\option[1]{\texttt{#1}}
\providecommand\env[1]{\texttt{#1}}
\providecommand\Arg[1]{\texttt\{\meta{#1}\texttt\}}


\providecommand\eTeX{\hologo{eTeX}}
\providecommand\XeTeX{\hologo{XeTeX}}
\providecommand\LuaTeX{\hologo{LuaTeX}}
\providecommand\pdfTeX{\hologo{pdfTeX}}
\providecommand\MiKTeX{\hologo{MiKTeX}}
\providecommand\CTAN{\textsc{ctan}}
\providecommand\TL{\TeX\,Live}
\providecommand\githubissue[2][]{\ifhmode\unskip\fi
     \quad\penalty500\strut\nobreak\hfill
     \mbox{\small\slshape(%
       \href{https://github.com/latex3/latex2e/issues/\getfirstgithubissue#2 \relax}%
          	    {github issue#1 #2}%
           )}%
     \par\smallskip}
%% But Chris has to mostly disable \href for his TEXPAD app:
%%  \def\href #1{}  % Only For Chris' deficient TeX engine

% simple solution right now (just link to the first issue if there are more)
\def\getfirstgithubissue#1 #2\relax{#1}

\providecommand\sxissue[1]{\ifhmode\unskip
     \else
       % githubissue preceding
       \vskip-\smallskipamount
       \vskip-\parskip
     \fi
     \quad\penalty500\strut\nobreak\hfill
     \mbox{\small\slshape(\url{https://tex.stackexchange.com/#1})}\par}

\providecommand\gnatsissue[2]{\ifhmode\unskip\fi
     \quad\penalty500\strut\nobreak\hfill
     \mbox{\small\slshape(%
       \href{https://www.latex-project.org/cgi-bin/ltxbugs2html?pr=#1\%2F\getfirstgithubissue#2 \relax}%
          	    {gnats issue #1/#2}%
           )}%
     \par}

\let\cls\pkg
\providecommand\env[1]{\texttt{#1}}
\providecommand\acro[1]{\textsc{#1}}

\vbadness=1400  % accept slightly empty columns


\makeatletter
% maybe not the greatest design but normally we wouldn't have subsubsections
\renewcommand{\subsubsection}{%
   \@startsection      {subsubsection}{2}{0pt}{1.5ex \@plus 1ex \@minus .2ex}%
      {-1em}{\@subheadingfont\colonize}%
}
\providecommand\colonize[1]{#1:}
\makeatother

\let\finalvspace\vspace          % for document layout fixes

% Undo ltnews's \verbatim@font with active < and >
\makeatletter
\def\verbatim@font{%
  \normalsize\ttfamily}
\makeatletter

%%%%%%%%%%%%%%%%%%%%%%%%%%%%%%%%%%%%%%%%%%%%%%%%%%%%%%%%%%%%%%%%%%%%%%%%%%%%%
\providecommand\tubcommand[1]{}
\tubcommand{\input{tubltmac}}

\publicationmonth{November}
\publicationyear{2022  --- DRAFT version for upcoming release}

\publicationissue{36}

\begin{document}

\tubcommand{\addtolength\textheight{4.2pc}}   % only for TUB

\maketitle
{\hyphenpenalty=10000 \exhyphenpenalty=10000 \spaceskip=3.33pt \hbadness=10000
\tableofcontents}

\setlength\rightskip{0pt plus 3em}


\medskip


\section{Introduction}

\emph{to be written}


\subsection{Auto-detecting key--value arguments}

To allow extension of the core \LaTeX{} syntax, \pkg{ltcmd} now supports
a \texttt{={...}} modifier when grabbing arguments. This modifier instructs
\LaTeX{} that the argument should be passed to the underlying code as
a set of keyvals. If the argument does not \enquote{look like} a set
of keyvals, it will be converted into a single key--value pair, with
the argument to \texttt{=} specifying the name of that key. For
example, the \cs{caption} command could be defined as
\begin{verbatim}
  \DeclareDocumentCommand\caption
         {s ={short-text}+O{#3} +m}
         {...}
\end{verbatim}
which would mean that if the optional argument does \emph{not}
contain keyval data, it will be converted to a single keyval
pair with the key name \texttt{short-text}.

Arguments which begin with \texttt{=,} are always interpreted as
keyvals even if they do not contain further \texttt{=} signs.
Any \texttt{=} signs enclosed within \verb|$...$| or \verb|\(...\)|,
i.e.~in inline math mode, are ignored, meaning that
only \texttt{=} outside of math mode will generally cause
interpretation as keyval material.

In case the argument contains a \enquote{textual} \texttt{=} sign that
is mistaken as key/value indicator you can hide it using a brace
group as you would do in other places, e.g.,
\begin{verbatim}
\caption[{Use of = signs}]
        {Use of = signs in optional arguments}
\end{verbatim}
However, because a \texttt{=} sign in math mode are already ignored, this
should seldom be necessary.

\subsection{Encoding subsets for \texttt{TS1} encoded fonts}

The text companion encoding \texttt{TS1} is unfortunately not very
faithfully supported in fonts that are not close cousins to the
Computer Modern fonts. It was therefore necessary to provide the
notion of \enquote{sub-encodings} on a per font basis. These
sub-encodings are declared for a font family with the help of a
\cs{DeclareEncodingSubset} declaration, see \cite{36:fntguide} for
details.

Maintainers of font bundles that include \texttt{TS1} encoded font
files should add an appropriate declaration into the corresponding
\texttt{ts1}\textit{family}\texttt{.fd} file, because otherwise the
default subencoding is assumed, which is probably disabling too many
glyphs that are actually available in the font.\footnote{The \LaTeX{}
  format contains declarations for many font families already.  This
  was done in 2020 to quickstart the use of the symbols in the kernel,
  but it is really the wrong place for such declarations. Thus, for
  new fonts the declarations should be placed into the corresponding
  \texttt{.fd} files.}
%
\githubissue{905}



\section{New or improved commands}


\section{Code improvements}

\subsection{Support for slanted small caps in the EC-fonts }
Since some time \LaTeX{} supports the combination of the shapes
small caps and italic/slanted. The EC-fonts contain slanted small caps fonts
but using them required the loading of an external package. Suitable font definitions
have now been added to \pkg{t1cmd.fd} and so from now on
\begin{verbatim}
 \textsc{\textsl{slanted small}}
 \textsc{\textit{italic small caps}}
 \bfseries
 \textsc{\textsl{bold slanted small caps}}
 \textsc{\textit{bold italic small caps}}
\end{verbatim}
will give the expected result: {\fontfamily{cmr}
\textsc{\textsl{slanted small}} \textsc{\textit{italic small caps}}
\bfseries
\textsc{\textsl{bold slanted small caps}} \textsc{\textit{bold italic small caps}}}
%
\githubissue{782}


\subsection{EC sans serif at small sizes}

The EC (T1 encoded Computer Modern) sans serif fonts have errors at
small sizes: the medium weight is bolder and wider than the bold
extended. This makes them unusable at these small sizes. The default
\texttt{.fd} file has therefore been adjusted to use a scaled down 8pt
font instead.
%
\githubissue{879}



\subsection{\LuaTeX\ callback efficiency improvement}

The mechanism for providing the
\texttt{pre/post\_mlist\_to\_hlist\_filter} callbacks in \LuaTeX\ has
been improved to make it more reusable and to avoid overhead if these
callbacks are not used.
%
\githubissue{830}





\section{Bug fixes}


\section{Changes to packages in the \pkg{amsmath} category}


\section{Changes to packages in the \pkg{graphics} category}

\subsection{Fix a \cs{mathcolor} bug}

The \cs{mathcolor} command intorduced in \cite{36:ltnews35} needs to
scan for following sub and superscripts, but if it did so at the end
of an alignment cell, e.g., in a \texttt{array} environment, the
\texttt{\&} was evaluated too early causing some internal errors. This
is now properly guarded for.
%
\githubissue{901}



\section{Changes to packages in the \pkg{tools} category}


\medskip

\begin{thebibliography}{9}

\fontsize{9.3}{11.3}\selectfont

%\bibitem{36:blueprint} Frank Mittelbach and Chris Rowley:
%  \emph{\LaTeX{} Tagged PDF \Dash A blueprint for a large project}.\\
%  \url{https://latex-project.org/publications/indexbyyear/2020/}

%\bibitem{36:source2e}
%  \emph{\LaTeX{} documentation on the \LaTeX{} Project Website}.\\
%  \url{https://latex-project.org/help/documentation/}

%\bibitem{36:ltnews31} \LaTeX{} Project Team:
%  \emph{\LaTeXe{} news 31}.\\
%  \url{https://latex-project.org/news/latex2e-news/ltnews31.pdf}

%\bibitem{36:ltnews32} \LaTeX{} Project Team:
%  \emph{\LaTeXe{} news 32}.\\
%  \url{https://latex-project.org/news/latex2e-news/ltnews32.pdf}

%\bibitem{36:ltnews33} \LaTeX{} Project Team:
%  \emph{\LaTeXe{} news 33}.\\
%  \url{https://latex-project.org/news/latex2e-news/ltnews33.pdf}

\bibitem{36:ltnews35} \LaTeX{} Project Team:
  \emph{\LaTeXe{} news 35}.\\
  \url{https://latex-project.org/news/latex2e-news/ltnews35.pdf}

\bibitem{36:fntguide} \LaTeX{} Project Team:
  \emph{\LaTeXe{} font selection}.\\
  \url{https://latex-project.org/help/documentation/}

%\bibitem{36:ltfilehook-doc} Frank Mittelbach, Phelype Oleinik, \LaTeX{}~Project~Team:
%  \emph{The \texttt{\upshape ltfilehook} documentation}.\\
%  Run \texttt{texdoc} \texttt{ltfilehook-doc} to view.
\end{thebibliography}



\end{document}
