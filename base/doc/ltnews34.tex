% \iffalse meta-comment
%
% Copyright 2019-2021
% The LaTeX Project and any individual authors listed elsewhere
% in this file.
%
% This file is part of the LaTeX base system.
% -------------------------------------------
%
% It may be distributed and/or modified under the
% conditions of the LaTeX Project Public License, either version 1.3c
% of this license or (at your option) any later version.
% The latest version of this license is in
%    https://www.latex-project.org/lppl.txt
% and version 1.3c or later is part of all distributions of LaTeX
% version 2008 or later.
%
% This file has the LPPL maintenance status "maintained".
%
% The list of all files belonging to the LaTeX base distribution is
% given in the file `manifest.txt'. See also `legal.txt' for additional
% information.
%
% The list of derived (unpacked) files belonging to the distribution
% and covered by LPPL is defined by the unpacking scripts (with
% extension .ins) which are part of the distribution.
%
% \fi
% Filename: ltnews34.tex
%
% This is issue 34 of LaTeX News.

\NeedsTeXFormat{LaTeX2e}[2020-02-02]

\documentclass{ltnews}

%%CCC  Temporary definitions:
\providecommand\Dash {\unskip ---}



%% NOTE:  Chris' preferred hyphens!
%%\showhyphens{parameters}
\hyphenation{because parameters parameter}

\usepackage[T1]{fontenc}

\usepackage{lmodern,url,hologo}

\usepackage{csquotes}
\usepackage{multicol}

\providecommand\hook[1]{\texttt{#1}}

\providecommand\meta[1]{$\langle$\textrm{\itshape#1}$\rangle$}
\providecommand\option[1]{\texttt{#1}}
\providecommand\env[1]{\texttt{#1}}
\providecommand\Arg[1]{\texttt\{\meta{#1}\texttt\}}


\providecommand\eTeX{\hologo{eTeX}}
\providecommand\XeTeX{\hologo{XeTeX}}
\providecommand\LuaTeX{\hologo{LuaTeX}}
\providecommand\pdfTeX{\hologo{pdfTeX}}
\providecommand\MiKTeX{\hologo{MiKTeX}}
\providecommand\CTAN{\textsc{ctan}}
\providecommand\TL{\TeX\,Live}
\providecommand\githubissue[2][]{\ifhmode\unskip\fi
     \quad\penalty500\strut\nobreak\hfill
     \mbox{\small\slshape(%
       \href{https://github.com/latex3/latex2e/issues/\getfirstgithubissue#2 \relax}%
          	    {github issue#1 #2}%
           )}%
     \par\smallskip}

% simple solution right now (just link to the first issue if there are more)
\def\getfirstgithubissue#1 #2\relax{#1}

\providecommand\sxissue[1]{\ifhmode\unskip\fi
     \quad\penalty500\strut\nobreak\hfill
     \mbox{\small\slshape(\url{https://tex.stackexchange.com/#1})}\par}

\providecommand\gnatsissue[2]{\ifhmode\unskip\fi
     \quad\penalty500\strut\nobreak\hfill
     \mbox{\small\slshape(%
       \href{https://www.latex-project.org/cgi-bin/ltxbugs2html?pr=#1\%2F\getfirstgithubissue#2 \relax}%
          	    {gnats issue #1/#2}%
           )}%
     \par}

\let\cls\pkg
\providecommand\env[1]{\texttt{#1}}
\providecommand\acro[1]{\textsc{#1}}

\vbadness=1400  % accept slightly empty columns


\makeatletter
% maybe not the greatest design but normally we wouldn't have subsubsections
\renewcommand{\subsubsection}{%
   \@startsection      {subsubsection}{2}{0pt}{1.5ex \@plus 1ex \@minus .2ex}%
      {-1em}{\@subheadingfont\colonize}%
}
\providecommand\colonize[1]{#1:}
\makeatother

\let\finalvspace\vspace          % for document layout fixes



%%%%%%%%%%%%%%%%%%%%%%%%%%%%%%%%%%%%%%%%%%%%%%%%%%%%%%%%%%%%%%%%%%%%%%%%%%%%%
\providecommand\tubcommand[1]{}
\tubcommand{\input{tubltmac}}

\publicationmonth{November}
\publicationyear{2021 --- Draft Version} 

\publicationissue{34}

\begin{document}

\tubcommand{\addtolength\textheight{4.2pc}}   % only for TUB

\maketitle
{\hyphenpenalty=10000 \spaceskip=3.33pt \hbadness=10000 \tableofcontents}

\setlength\rightskip{0pt plus 3em}


\medskip


\section{Introduction}

\emph{write}

\section{???}

\emph{write}

\section{???}

\emph{write}

\section{Hook business}

\subsection{Changed how \cs{RemoveFromHook} treats code that isn't in the hook}

In the first version of \cs{RemoveFromHook}, in case the code label
being removed didn't exist in the hook, a ``removal order'' would be
queued, and the next time something tried to add that label to the hook,
the \cs{AddToHook} would be cancelled by the removal order, and no code
would be added that once.  This was so that in principle package loading
order wouldn't matter.  However this implementation didn't work quite as
intended, because while two \cs{AddToHook} to a given label would be
removed by a single \cs{RemoveFromHook}, one \cs{RemoveFromHook} could
not cancel two \cs{AddToHook} to that label, and this asymmetry caused
confusion and was a recipe for further problems.

The implementation was changed and now \cs{RemoveFromHook} only removes
labels that already exist in a hook, and will display a warning if they
don't.  For usage across packages, for removing code in a hook, the
\texttt{voids} relation should be used instead:  this relation is
non-destructive (meaning it can be later reverted with another one), and
it is truly independent of package loading order, so it should be
preferred.
%
\githubissue{625}

\section{New or improved commands}


\subsection{Added \cs{PackageNote} and \cs{ClassNote}}

\LaTeX{} offers \cs{PackageError} to signal errors that stop
processing, \cs{PackageWarning} that generates a warning message on the
terminal, but continues with the processing and also \cs{PackageInfo}
to provide some information that is only written to the \texttt{.log}
file. What hasn't existed up to now is a way to provide some
information on the terminal that is identifying itself as coming from
a specific package but which isn't claiming to be a warning. Thus,
packages that wanted to write to the terminal used \cs{PackageWarning}
even though the information wasn't really warning the user.  For this
we now have \cs{PackageNote} and \cs{PackageNoteNoLine}, that identify
themselves as \enquote{informational}, but still go to the terminal and
not only to the transcript.

Similar commands exist for classes and there we have added the missing
\cs{ClassNote} and \cs{ClassNoteNoLine} as well.
%
\githubissue{613}


\subsection{New implementation for \cs{counterwithin}}

New implementation for \cs{counterwithout} and \cs{counterwithin} with
an additional optional arg so it becomes a drop-in replacement for
amsmath \cs{numberwithin}.

\emph{write appropriate description}



\subsection{New default for \cs{tracinglostchars}}

In 2021 the \TeX{} engines got enhanced so that \cs{tracinglostchars}
is now also supporting the value \texttt{3}, turning missing
characters into errors and not just warnings. This change made us
realize that \LaTeX{} should use a better default for this parameter
(so far the warning was only written to the transcript file).
Using the now available \texttt{3} would really be the best, but for
compatibility reasons we only set it to \texttt{2} in the kernel.
However, we recommend adding \cs{tracinglostchars}\texttt{=3} to the
preamble of documents, because missing glyphs in the output are an
error and should therefore be properly looked at.







\subsection{???}

%
\githubissue{000}




\section{Code improvements}

\subsection{???}

%
\githubissue{000}


\subsection{Additional Extended Latin characters predefined}
Some additional characters such as \'k (U+1E131) are now pre-defined and
will work without needing \verb|\DeclareUnicodeCharacter| declarations.
%
\githubissue{593}

\section{Changes to packages in the \pkg{graphics} category}


\subsection{???}

%
\githubissue{000}



\section{Changes to packages in the \pkg{tools} category}

\subsection{\pkg{varioref}: Improve missing label handling}

If an undefined label is referenced, \pkg{varioref} makes a default
definition so that later processing finds the right structure (two
brace groups inside \cs{r@}\meta{label}) However, if \pkg{nameref} or
\pkg{hyperref} is loaded, the data structure changes to five
arguments, resulting in low-evel errors in some cases. The code has
been changed to avoid these errors.
%
\sxissue{603948}


\subsection{\pkg{array}: Cancel  \cs{mathsurround} for \env{tabular}}

A \env{tabular} environment is internally typeset as an \env{array}
environment with special settings and therefore in math mode. This
math group should not get any \cs{mathsuround} added as it isn't a
real formula because otherwise the spacing around the \env{tabular}
changes. This bug has been there forever (which means not many people
use \cs{mathsurround} or noticed the difference). Anyhow, this now got
fixed.
%
\githubissue{614}



\subsection{???}

%
\githubissue{000}


\section{Changes to packages in the \pkg{amsmath} category}

\subsection{???}

%
\githubissue{000}




\medskip

\begin{thebibliography}{9}

\fontsize{9.3}{11.3}\selectfont

\bibitem{34:blueprint} Frank Mittelbach and Chris Rowley:
  \emph{\LaTeX{} Tagged PDF \Dash A blueprint for a large project}.\\
  \url{https://latex-project.org/publications/indexbyyear/2020/}

\bibitem{34:source2e}
  \emph{\LaTeX{} documentation on the \LaTeX{} Project Website}.\\
  \url{https://latex-project.org/help/documentation/}

\bibitem{34:ltnews32} \LaTeX{} Project Team:
  \emph{\LaTeXe{} news 32}.\\
  \url{https://latex-project.org/news/latex2e-news/ltnews32.pdf}

\end{thebibliography}



\end{document}
