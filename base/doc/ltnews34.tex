% \iffalse meta-comment
%
% Copyright 2019-2021
% The LaTeX Project and any individual authors listed elsewhere
% in this file.
%
% This file is part of the LaTeX base system.
% -------------------------------------------
%
% It may be distributed and/or modified under the
% conditions of the LaTeX Project Public License, either version 1.3c
% of this license or (at your option) any later version.
% The latest version of this license is in
%    https://www.latex-project.org/lppl.txt
% and version 1.3c or later is part of all distributions of LaTeX
% version 2008 or later.
%
% This file has the LPPL maintenance status "maintained".
%
% The list of all files belonging to the LaTeX base distribution is
% given in the file `manifest.txt'. See also `legal.txt' for additional
% information.
%
% The list of derived (unpacked) files belonging to the distribution
% and covered by LPPL is defined by the unpacking scripts (with
% extension .ins) which are part of the distribution.
%
% \fi
% Filename: ltnews34.tex
%
% This is issue 34 of LaTeX News.

\NeedsTeXFormat{LaTeX2e}[2020-02-02]

\documentclass{ltnews}

%%CCC  Temporary definitions:
\providecommand\Dash {\unskip ---}



%% NOTE:  Chris' preferred hyphens!
%%\showhyphens{parameters}
\hyphenation{because parameters parameter}

\usepackage[T1]{fontenc}

\usepackage{lmodern,url,hologo}

\usepackage{csquotes}
\usepackage{multicol}

\providecommand\hook[1]{\texttt{#1}}

\providecommand\meta[1]{$\langle$\textrm{\itshape#1}$\rangle$}
\providecommand\option[1]{\texttt{#1}}
\providecommand\env[1]{\texttt{#1}}
\providecommand\Arg[1]{\texttt\{\meta{#1}\texttt\}}


\providecommand\eTeX{\hologo{eTeX}}
\providecommand\XeTeX{\hologo{XeTeX}}
\providecommand\LuaTeX{\hologo{LuaTeX}}
\providecommand\pdfTeX{\hologo{pdfTeX}}
\providecommand\MiKTeX{\hologo{MiKTeX}}
\providecommand\CTAN{\textsc{ctan}}
\providecommand\TL{\TeX\,Live}
\providecommand\githubissue[2][]{\ifhmode\unskip\fi
     \quad\penalty500\strut\nobreak\hfill
     \mbox{\small\slshape(%
       \href{https://github.com/latex3/latex2e/issues/\getfirstgithubissue#2 \relax}%
          	    {github issue#1 #2}%
           )}%
     \par\smallskip}

% simple solution right now (just link to the first issue if there are more)
\def\getfirstgithubissue#1 #2\relax{#1}

\providecommand\sxissue[1]{\ifhmode\unskip\fi
     \quad\penalty500\strut\nobreak\hfill
     \mbox{\small\slshape(\url{https://tex.stackexchange.com/#1})}\par}

\providecommand\gnatsissue[2]{\ifhmode\unskip\fi
     \quad\penalty500\strut\nobreak\hfill
     \mbox{\small\slshape(%
       \href{https://www.latex-project.org/cgi-bin/ltxbugs2html?pr=#1\%2F\getfirstgithubissue#2 \relax}%
          	    {gnats issue #1/#2}%
           )}%
     \par}

\let\cls\pkg
\providecommand\env[1]{\texttt{#1}}
\providecommand\acro[1]{\textsc{#1}}

\vbadness=1400  % accept slightly empty columns


\makeatletter
% maybe not the greatest design but normally we wouldn't have subsubsections
\renewcommand{\subsubsection}{%
   \@startsection      {subsubsection}{2}{0pt}{1.5ex \@plus 1ex \@minus .2ex}%
      {-1em}{\@subheadingfont\colonize}%
}
\providecommand\colonize[1]{#1:}
\makeatother

\let\finalvspace\vspace          % for document layout fixes

% Undo ltnews's \verbatim@font with active < and >
\makeatletter
\def\verbatim@font{%
  \normalsize\ttfamily}
\makeatletter

%%%%%%%%%%%%%%%%%%%%%%%%%%%%%%%%%%%%%%%%%%%%%%%%%%%%%%%%%%%%%%%%%%%%%%%%%%%%%
\providecommand\tubcommand[1]{}
\tubcommand{\input{tubltmac}}

\publicationmonth{November}
\publicationyear{2021 --- Draft Version} 

\publicationissue{34}

\begin{document}

\tubcommand{\addtolength\textheight{4.2pc}}   % only for TUB

\maketitle
{\hyphenpenalty=10000 \spaceskip=3.33pt \hbadness=10000 \tableofcontents}

\setlength\rightskip{0pt plus 3em}


\medskip


\section{Introduction}

\emph{write}

\section{???}

\emph{write}




\section{Hook business}



\subsection{Clear extra hook code for next invocation}

There are a few use cases where it would be helpful if one can cancel
an earlier use of \cs{AddToHookNext}, for example, when a page is
discarded with \cs{DiscardShipoutBox} because only some pages of the
document are printed. For such situations the new command
\cs{ClearHookNext} is provided.
%
\githubissue{565}



\subsection{Class, package, and include hook improvements}

Classes, packages and include files can only be loaded once in a
\LaTeX{} document. For that reasons hooks that are specific to such
files have been made one-time hooks. Beside being more efficient this
supports the following important use case
\begin{verbatim}
\AddToHook{package/after/varioref}
  { ... apply my customizations if the package
      gets loaded (or was loaded already) ... }
\end{verbatim}
without the need to first test if the package was already loaded
before.
%
\githubissue{623}





\subsection{???}

%
\githubissue{000}






\section{New or improved commands}


\subsection{Added \cs{PackageNote} and \cs{ClassNote}}

\LaTeX{} offers \cs{PackageError} to signal errors that stop
processing, \cs{PackageWarning} that generates a warning message on the
terminal, but continues with the processing and also \cs{PackageInfo}
to provide some information that is only written to the \texttt{.log}
file. What hasn't existed up to now is a way to provide some
information on the terminal that is identifying itself as coming from
a specific package but which isn't claiming to be a warning. Thus,
packages that wanted to write to the terminal used \cs{PackageWarning}
even though the information wasn't really warning the user.  For this
we now have \cs{PackageNote} and \cs{PackageNoteNoLine}, that identify
themselves as \enquote{informational}, but still go to the terminal and
not only to the transcript.

Similar commands exist for classes and there we have added the missing
\cs{ClassNote} and \cs{ClassNoteNoLine} as well.
%
\githubissue{613}


\subsection{New implementation for \cs{counterwithin}}

New implementation for \cs{counterwithout} and \cs{counterwithin} with
an additional optional arg so it becomes a drop-in replacement for
amsmath \cs{numberwithin}.

\emph{write appropriate description}



\subsection{New default for \cs{tracinglostchars}}

In 2021 the \TeX{} engines got enhanced so that \cs{tracinglostchars}
is now also supporting the value \texttt{3}, turning missing
characters into errors and not just warnings. This change made us
realize that \LaTeX{} should use a better default for this parameter
(so far the warning was only written to the transcript file).
Using the now available \texttt{3} would really be the best, but for
compatibility reasons we only set it to \texttt{2} in the kernel.
However, we recommend adding \cs{tracinglostchars}\texttt{=3} to the
preamble of documents, because missing glyphs in the output are an
error and should therefore be properly looked at.






\subsection{Provide tests for package and class loading}

To test if a package was loaded you can now use \cs{IfPackageLoadedTF}
\Arg{package} \Arg{true} \Arg{false} and based on the result execute
different code. It is also possible to check if the package was loaded
with certain options. This is done with
\cs{IfPackageLoadedWithOptionsTF}. It takes four arguments:
\Arg{package}\Arg{option-list}\Arg{true}\Arg{false}. It uses the
\meta{false} code if at least one option in the \meta{option-list}
has not been used during loading or if the package hasn't been loaded
at all.

For classes similar commands (\texttt{Package} replaced by
\texttt{Class} in the name) are provided.
%
\githubissue{621}









\subsection{New \cs{showfloat} command}

The package \pkg{fltrace} offers a (fairly low-level but very
detailed) way to trace \LaTeX's float mechanism. This can help to
understand why a certain float is placed into a certain region or why
it shows up unexpectedly on a later page.  \LaTeX{} stores floats in
registers named \cs{bx@A}, \cs{bx@B}, etc., and these names show up in
the tracing information.
%
To display their contents you can now say
\verb=\showfloat{=\textit{uc-letter}\verb=}= where \textit{uc-letter}
is the uppercase letter (or letters) after \texttt{bx@} in the
register name shown in the tracing. The command is generally
available, whether or not you have loaded \pkg{fltrace}, because it is
also useful when interpreting the tracing output of the
\pkg{fewerfloatpages} package.


\subsection{Add \pkg{ltcmd} support for \cs{NewCommandCopy} and \cs{ShowCommand}}

Since the 2020-10-01 release (see~\cite{34:ltnews32}), \LaTeX{} provides
\cs{NewCommandCopy} to copy robust commands, and \cs{ShowCommand} to
show their definition in the terminal.  In the same release, the
\pkg{xparse} package was integrated in the kernel (now called
\pkg{ltcmd}).  But since then, the support for \cs{NewCommandCopy} and
\cs{ShowCommand} was not implemented in \pkg{ltcmd}.  The present
\LaTeX{} release ships with that support implemented, so now commands
defined with \pkg{xparse}/\pkg{ltcmd} can be copied and their definition
can be easily shown in the terminal without \cs{csname} gymnastics.
%
\githubissue{569}





\subsection{???}

%
\githubissue{000}




\section{Code improvements}


\subsection{Additional Extended Latin characters predefined}
Some additional characters such as \'k (U+1E131) are now pre-defined and
will work without needing \verb|\DeclareUnicodeCharacter| declarations.
%
\githubissue{593}


\subsection{Use OpenType version of Latin Modern Upright Italic}
When Latin Modern is used with the TU encoding under \XeTeX\ or \LuaTeX\
and fontshape \texttt{ui} is requested, \LaTeX\ now uses the OpenType
version instead of substituting the (T1 encoded) Type 1 version.


\subsection{Pick up all arguments to \cs{contentsline}}

The \cs{contentsline} commands in the TOC file are always followed by
four arguments, the last one being empty by default and only used by
\pkg{hyperref}. The \cs{contentsline} command itself only used the
first three arguments and relied on the fourth being empty (and thus
doing no harm). But this assumption is not always correct, e.g., if
you use \pkg{hyperref} and then remove it from the preamble.

So now we pick up all four arguments and save the last one away, so
that it can be used by \pkg{hyperref}.
%
\githubissue{633}


\subsection{???}

%
\githubissue{000}




\section{Bug fixes}

\subsection{Replicate argument processors for all embellishments}

There was a bug in \pkg{ltcmd} (former \pkg{xparse}) that caused
commands to misbehave if they were defined with embellishments and
argument processors.  In that case, only one (possibly void) argument
processor would be added to the full set of embellishment arguments,
resulting in too few processors in some cases, leading to unpredictable
behavior.  This bug has been fixed by applying the same argument
processors to all the embellishments in a set, so a declaration like:
\begin{verbatim}
\NewDocumentCommand\foo{>{\TrimSpaces}e{_^}}
  {(#1)[#2]}
\foo^{ a }_{ b }
\end{verbatim}
will now correctly apply \cs{TrimSpaces} to both arguments.
%
\githubissue{639}





\section{Changes to packages in the \pkg{graphics} category}


\subsection{???}

%
\githubissue{000}



\section{Changes to packages in the \pkg{tools} category}

\subsection{\pkg{varioref}: Improve missing label handling}

If an undefined label is referenced, \pkg{varioref} makes a default
definition so that later processing finds the right structure (two
brace groups inside \cs{r@}\meta{label}) However, if \pkg{nameref} or
\pkg{hyperref} is loaded, the data structure changes to five
arguments, resulting in low-evel errors in some cases. The code has
been changed to avoid these errors.
%
\sxissue{603948}


\subsection{\pkg{array}: Cancel  \cs{mathsurround} for \env{tabular}}

A \env{tabular} environment is internally typeset as an \env{array}
environment with special settings and therefore in math mode. This
math group should not get any \cs{mathsuround} added as it isn't a
real formula because otherwise the spacing around the \env{tabular}
changes. This bug has been there forever (which means not many people
use \cs{mathsurround} or noticed the difference). Anyhow, this now got
fixed.
%
\githubissue{614}



\subsection{???}

%
\githubissue{000}


\section{Changes to packages in the \pkg{amsmath} category}

\subsection{???}

%
\githubissue{000}




\medskip

\begin{thebibliography}{9}

\fontsize{9.3}{11.3}\selectfont

\bibitem{34:blueprint} Frank Mittelbach and Chris Rowley:
  \emph{\LaTeX{} Tagged PDF \Dash A blueprint for a large project}.\\
  \url{https://latex-project.org/publications/indexbyyear/2020/}

\bibitem{34:source2e}
  \emph{\LaTeX{} documentation on the \LaTeX{} Project Website}.\\
  \url{https://latex-project.org/help/documentation/}

\bibitem{34:ltnews32} \LaTeX{} Project Team:
  \emph{\LaTeXe{} news 32}.\\
  \url{https://latex-project.org/news/latex2e-news/ltnews32.pdf}

\end{thebibliography}



\end{document}
