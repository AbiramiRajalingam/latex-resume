% \iffalse meta-comment
%
% Copyright (C) 1993-2023
% The LaTeX Project and any individual authors listed elsewhere
% in this file.
%
% This file is part of the LaTeX base system.
% -------------------------------------------
%
% It may be distributed and/or modified under the
% conditions of the LaTeX Project Public License, either version 1.3c
% of this license or (at your option) any later version.
% The latest version of this license is in
%    http://www.latex-project.org/lppl.txt
% and version 1.3c or later is part of all distributions of LaTeX
% version 2008 or later.
%
% This file has the LPPL maintenance status "maintained".
%
% The list of all files belonging to the LaTeX base distribution is
% given in the file `manifest.txt'. See also `legal.txt' for additional
% information.
%
% The list of derived (unpacked) files belonging to the distribution
% and covered by LPPL is defined by the unpacking scripts (with
% extension .ins) which are part of the distribution.
%
% \fi
% Filename: ltnews03.tex

% This is issue 3 of LaTeX News.

\documentclass
%   [lw35fonts]
   {ltnews}

\publicationmonth{June}
\publicationyear{1995}
\publicationissue{3}


\providecommand\pkg[1]{\texttt{#1}}
\providecommand\cls[1]{\texttt{#1}}
\providecommand\option[1]{\texttt{#1}}
\providecommand\env[1]{\texttt{#1}}
\providecommand\file[1]{\texttt{#1}}


\begin{document}

\maketitle

\section{Welcome to \LaTeXNews~3}

An issue of \emph{\LaTeXNews} will accompany every future release of
\LaTeX.  It will tell you about important events, such as major bug
fixes, newly available packages, or any other \LaTeX{} news.

\section{June 1995 release of \LaTeX}

June 1995 sees the third release of \LaTeXe.  We are on schedule
to deliver a release of \LaTeX{} every six months, in December and
June.

In the last \emph{\LaTeXNews}, we said ``we don't expect so much
activity in the next six months,'' which has turned out not to be
true!

\section{Additional input encodings}

In the last release of \LaTeX{} we distributed a test version of the
\pkg{inputenc} package which allows the use of input characters
other than just a--z and A--Z.  The package has proved to be robust,
so we are now distributing an expanded version.
The new release comes with a number of input encodings:

\begin{itemize}
\item \option{ascii} the standard encoding,
\item \option{latin1} the ISO Western European alphabet,
\item \option{latin2} the ISO Eastern European alphabet,
\item \option{cp437} the IBM codepage 437,
\item \option{cp850} the IBM codepage 850, and
\item \option{applemac} the Apple Macintosh encoding.
\end{itemize}
These can be used by specifying an option to the \pkg{inputenc}
package, for example:
\begin{verbatim}
   \usepackage[latin1]{inputenc}
\end{verbatim}
The new input encodings are currently being tested, but we don't
expect any major changes.

\section{\LaTeX\ getting smaller}

In the past releases of \LaTeXe, the amount of memory \LaTeX{}
requires has increased, but we are pleased to say that this trend has
been reversed.  We hope that future releases of \LaTeX{} will continue
to get smaller.

For example, on this document, the December 1994 release used 52,622
words of memory, and the June 1995 release uses 51,216 words of
memory, which is a 2.7\% reduction.

We are currently experimenting with other ways of reducing the size of
\LaTeX.  For example, we are experimenting with an option to remove
the \env{picture} and \env{tabbing} environments from the
\LaTeX{} kernel, and to load them from a file the first time they are
used.  This should help \LaTeX{} to run on machines with limited
memory.  See \file{autoload.txt} for details.

\section{Distribution and modification}

One topic of discussion that has kept us busy is the distribution and
modification conditions of \LaTeX.  We are committed to keeping
\LaTeX{} as free reliable software, and ensuring that (as far as
possible) \LaTeX{} documents will produce the same results on all
systems.

The modification conditions are currently under discussion, and we
would like to hear from anyone interested.  Please read
\file{modguide.tex} for more information.

\section{AMS-\LaTeX\ full release}

The AMS-\LaTeX\ packages were still in beta test in the December 1994
release of \LaTeX, and the full release came out in January 1995.

AMS-\LaTeX\ is described in the \emph{User's Guide}
(\file{amsldoc.tex}) and in \emph{The \LaTeX{} Companion}.

\section{PostScript fonts}

There is a new test release of the PSNFSS packages for accessing
PostScript fonts in \LaTeXe.  This includes an update to all of the
fonts, to remove many of the underfull and overfull \verb|\hbox|
warnings, and improve the setting of non-English languages.

The new release of \LaTeX{} removes all of the `hidden' uses of
Computer Modern mathematics.  For example, the footnote markers used
to use math mode, so always used Computer Modern digits rather than
ones from the current text font.  This has now been fixed.

\section{Further information}

For more information on \TeX{} and \LaTeX, get in touch with your local
\TeX{} Users Group, or the international \TeX{} Users Group,
P.~O.~Box~869, Santa~Barbara, CA~93102-0869, USA, Fax:~+1~805~963~8358,
EMail:~tug@tug.org.

The \LaTeX{} home page is \verb|http://www.tex.ac.uk/ctan/latex/|
and contains links to other WWW resources for \LaTeX.

\end{document}
