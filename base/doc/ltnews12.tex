% \iffalse meta-comment
%
% Copyright (C) 1993-2021
% The LaTeX Project and any individual authors listed elsewhere
% in this file.
%
% This file is part of the LaTeX base system.
% -------------------------------------------
%
% It may be distributed and/or modified under the
% conditions of the LaTeX Project Public License, either version 1.3c
% of this license or (at your option) any later version.
% The latest version of this license is in
%    http://www.latex-project.org/lppl.txt
% and version 1.3c or later is part of all distributions of LaTeX
% version 2008 or later.
%
% This file has the LPPL maintenance status "maintained".
%
% The list of all files belonging to the LaTeX base distribution is
% given in the file `manifest.txt'. See also `legal.txt' for additional
% information.
%
% The list of derived (unpacked) files belonging to the distribution
% and covered by LPPL is defined by the unpacking scripts (with
% extension .ins) which are part of the distribution.
%
% \fi
% Filename: ltnews12.tex
%
% This is issue 12 of LaTeX News.

\documentclass
%    [lw35fonts]     % uncomment this line to get Times
   {ltnews}[1999/12/01]

% \usepackage[T1]{fontenc}

\publicationmonth{December}
\publicationyear{1999}
\publicationissue{12}

% Should go to .cls:
\newcommand{\acro}[1]{\textsc{\MakeLowercase{#1}}}

\begin{document}

\maketitle

\section{LPPL update}

Since the release of the \LaTeX{} Project Public Licence version~1.1,
we have received a small number of queries which resulted in some
minor changes to improve the wording or explain the intentions better.
As a consequence this release now contains LPPL~1.2 in the file
\file{lppl.txt} and the previous versions as \file{lppl-1-0.txt} and
\file{lppl-1-1.txt}.

\section{fixltx2e package}

    This package provides fixes to \LaTeXe{} which are desirable but
    cannot be integrated into the \LaTeXe{} kernel directly as they
    would produce a version incompatible to earlier releases (either
    in formatting or functionality).

    By having these fixes in the form of a package, users can benefit
    from them without the danger that their documents will fail, or
    produce unexpected results, at other sites; this works because a
    document will contain a clear indication (the \verb=\usepackage=
    line, preferably with a required date) that at least some of these
    fixes are required to format it.

\section{Outcome of TUG '99 (Vancouver)}

The slides from the \acro{TUG}'99 presentation we gave on \emph{a
new interface for \LaTeX\ class designers} are available from the
\LaTeX\ Project website; look for the file \verb|tug99.pdf| at:
\begin{quote}
   \url{http://www.latex-project.org/talks/}
\end{quote}

Please note that this document was intended only to be informal
``speaker's notes'' for our own use. We decided to make them available (the
speaker's notes as well as the slides that were presented) because several
people requested copies after the talk. However, they are \emph{not} in
a polished copy-edited form and are not intended for publication.

Prototype implementations of parts of this interface are now available from:
\begin{quote}
   \url{http://www.latex-project.org/code/experimental/}
\end{quote}

We are continuing to add new material at this location so as to
stimulate further discussion of the underlying concepts. As of
December 1, 1999
the following parts can be downloaded.
\begin{description}

 \item[xparse] Prototype implementation of the interface for declaring
  document command syntax. See the \texttt{.dtx} files for
  documentation.

 \item[template] Prototype implementation of the template interface
  (needs parts of \texttt{xparse}).

  The file \texttt{template.dtx} in that directory has a large section
  of documentation at the front describing the commands in the
  interface and giving a `worked example' building up some templates
  for caption formatting.

 \item[xcontents] Interface description for table of contents data (no
  code yet).  Coding examples have been thoroughly discussed on the
  \texttt{latex-l} list.

 \item[xfootnote] Working examples for generating footnotes,
  etc. Needs \texttt{xparse} and \texttt{template}.

 \end{description}
All examples are organised in subdirectories and additionally
 available as \texttt{gzip} \texttt{tar} files.

 Please remember
that this material is intended only for experimentation and comments;
thus any aspect of it, e.g., the user interface or the functionality,
may change and, in fact, is very likely to change.
For this reason it is explicitly forbidden to place this material on
\acro{CD-ROM} distributions or public servers.

These concepts, as well as their implementation, are under discussion
on the list \texttt{LATEX-L}. You can join this list, which is
intended solely for discussing ideas and concepts for future versions
of \LaTeX, by sending mail to
%\begin{quote}
  \email{listserv@URZ.UNI-HEIDELBERG.DE}
%\end{quote}
containing the line
  \begin{quote}
    \texttt{SUBSCRIBE LATEX-L} \textit{Your Name}
  \end{quote}

This list is archived and, after subscription, you can retrieve older
posts to it by sending mail to the above address, containing a command
such as:
\begin{quote}
\texttt{GET LATEX-L LOGyymm}
\end{quote}
where \texttt{yy}=Year and \texttt{mm}=Month, e.g.
\begin{quote}
\texttt{GET LATEX-L LOG9910}
\end{quote}
for all messages sent in October 1999.


\end{document}
