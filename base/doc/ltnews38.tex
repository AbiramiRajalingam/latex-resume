% \iffalse meta-comment
%
% Copyright 2021-2023
% The LaTeX Project and any individual authors listed elsewhere
% in this file.
%
% This file is part of the LaTeX base system.
% -——————————————
%
% It may be distributed and/or modified under the
% conditions of the LaTeX Project Public License, either version 1.3c
% of this license or (at your option) any later version.
% The latest version of this license is in
%    https://www.latex-project.org/lppl.txt
% and version 1.3c or later is part of all distributions of LaTeX
% version 2008 or later.
%
% This file has the LPPL maintenance status "maintained".
%
% The list of all files belonging to the LaTeX base distribution is
% given in the file `manifest.txt'. See also `legal.txt' for additional
% information.
%
% The list of derived (unpacked) files belonging to the distribution
% and covered by LPPL is defined by the unpacking scripts (with
% extension .ins) which are part of the distribution.
%
% \fi
% Filename: ltnews38.tex
%
% This is issue 38 of LaTeX News.

\NeedsTeXFormat{LaTeX2e}[2020-02-02]

\documentclass{ltnews}

%%  Maybe needed only for Chris' inadequate system:
\providecommand\Dash {\unskip \textemdash}

%% NOTE:  Chris' preferred hyphens!
%%\showhyphens{parameters}
%%  \hyphenation{because parameters parameter}

\usepackage[T1]{fontenc}

\usepackage{lmodern,url,hologo}

\usepackage{csquotes}
\usepackage{multicol}
\usepackage{color}

\providecommand\hook[1]{\texttt{#1}}

\providecommand\meta[1]{$\langle$\textrm{\itshape#1}$\rangle$}
\providecommand\option[1]{\texttt{#1}}
\providecommand\env[1]{\texttt{#1}}
\providecommand\Arg[1]{\texttt\{\meta{#1}\texttt\}}


\providecommand\eTeX{\hologo{eTeX}}
\providecommand\XeTeX{\hologo{XeTeX}}
\providecommand\LuaTeX{\hologo{LuaTeX}}
\providecommand\pdfTeX{\hologo{pdfTeX}}
\providecommand\MiKTeX{\hologo{MiKTeX}}
\providecommand\CTAN{\textsc{ctan}}
\providecommand\TL{\TeX\,Live}
\providecommand\githubissue[2][]{\ifhmode\unskip\fi
     \quad\penalty500\strut\nobreak\hfill
     \mbox{\small\slshape(%
       \href{https://github.com/latex3/latex2e/issues/\getfirstgithubissue#2 \relax}%
          	    {github issue#1 #2}%
           )}%
     \par\smallskip}
%% But Chris has to mostly disable \href for his TEXPAD app:
%%  \def\href #1{}  % Only For Chris' deficient TeX engine

% simple solution right now (just link to the first issue if there are more)
\def\getfirstgithubissue#1 #2\relax{#1}

\providecommand\sxissue[1]{\ifhmode\unskip
     \else
       % githubissue preceding
       \vskip-\smallskipamount
       \vskip-\parskip
     \fi
     \quad\penalty500\strut\nobreak\hfill
     \mbox{\small\slshape(\url{https://tex.stackexchange.com/#1})}\par}

\providecommand\gnatsissue[2]{\ifhmode\unskip\fi
     \quad\penalty500\strut\nobreak\hfill
     \mbox{\small\slshape(%
       \href{https://www.latex-project.org/cgi-bin/ltxbugs2html?pr=#1\%2F\getfirstgithubissue#2 \relax}%
          	    {gnats issue #1/#2}%
           )}%
     \par}

\let\cls\pkg
\providecommand\env[1]{\texttt{#1}}
\providecommand\acro[1]{\textsc{#1}}

\vbadness=1400  % accept slightly empty columns


\makeatletter
% maybe not the greatest design but normally we wouldn't have subsubsections
\renewcommand{\subsubsection}{%
   \@startsection      {subsubsection}{2}{0pt}{1.5ex \@plus 1ex \@minus .2ex}%
      {-1em}{\@subheadingfont\colonize}%
}
\providecommand\colonize[1]{#1:}
\makeatother

\let\finalvspace\vspace          % for document layout fixes

% Undo ltnews's \verbatim@font with active < and >
\makeatletter
\def\verbatim@font{%
  \normalsize\ttfamily}
\makeatletter

%%%%%%%%%%%%%%%%%%%%%%%%%%%%%%%%%%%%%%%%%%%%%%%%%%%%%%%%%%%%%%%%%%%%%%%%%%%%%
\providecommand\tubcommand[1]{}
\tubcommand{\input{tubltmac}}

\publicationmonth{November}
\publicationyear{2023  --- DRAFT version for upcoming release}

\publicationissue{38}

\begin{document}

\tubcommand{\addtolength\textheight{4.2pc}}   % only for TUB

\maketitle
{\hyphenpenalty=10000 \exhyphenpenalty=10000 \spaceskip=3.33pt \hbadness=10000
\tableofcontents}

\setlength\rightskip{0pt plus 3em}

\medskip

\section{Introduction}


\section{New or improved commands}

%
\githubissue{xxx}

\section{Code improvements}

\subsection{Support for tabs in \cs{verb*}}

\LaTeX{} converts tabs to spaces when typsetting. The same has been true to
date inside \cs{verb}, but was done in a way that meant that they remained
as normal spaces even in \cs{verb*}. We have now adjusted the code so that
tabs are retained within \cs{verb} and \cs{verb*} independently from spaces,
and are set up to print in the same way spaces do. This means that they
now generate visible spaces inside \cs{verb*}, and their behavior can be
adjusted if required to be different from that of spaces.
%
\githubissue{1085}

\subsection{Unification of space and tab treatment by \cs{verb} and
  \cs{NewDocumentCommand}}

The \cs{verb(*)} command has always collected spaces as active characters, and
the change outlined above extends this to tabs. This is what enables \cs{verb*}
to easily print visible spaces. We have adjusted the implementation of the
\texttt{v}-type argument in \cs{NewDocumentCommand} to do the same. This will
\emph{require sources to be updated} if the normal definition of an active
space has been modified. However, the team believe that overall the benefit of
better consistency makes this a necessary change.

\subsection{Handling of end-of-lines in \cs{NewDocumentCommand} \texttt{+v}
arguments}

The \texttt{+v} argument type provided by \cs{NewDocumentCommand}, etc., allows
grabbing of multiple lines of text in a verbatim-like argument. Almost always,
the result of this grabbing will be used in a typesetting context. Previously,
the end-of-line characters were stored literally as category code~12
(\enquote{other}) \verb|^^M| tokens. However, these are difficult to work with
in general. We have now revised this behavior, such that end-of-line characters
are converted to the \cs{par} command when parsed by \texttt{+v}-type
arguments. As with the previous change, this may require adjustment in the
source of some documents, but the enhanced ability of users and programmers to
exploit the \texttt{+v}-type argument means we believe it is necessary.

\section{Bug fixes}

\section{Changes to packages in the \pkg{amsmath} category}

\section{Changes to packages in the \pkg{graphics} category}

\section{Changes to packages in the \pkg{tools} category}


%\medskip

\begin{thebibliography}{9}

\fontsize{9.3}{11.3}\selectfont

\end{thebibliography}

\end{document}
