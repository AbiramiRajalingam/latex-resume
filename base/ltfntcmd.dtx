% \iffalse meta-comment
%
% Copyright (C) 1993-2021
% The LaTeX Project and any individual authors listed elsewhere
% in this file.
%
% This file is part of the LaTeX base system.
% -------------------------------------------
%
% It may be distributed and/or modified under the
% conditions of the LaTeX Project Public License, either version 1.3c
% of this license or (at your option) any later version.
% The latest version of this license is in
%    https://www.latex-project.org/lppl.txt
% and version 1.3c or later is part of all distributions of LaTeX
% version 2008 or later.
%
% This file has the LPPL maintenance status "maintained".
%
% The list of all files belonging to the LaTeX base distribution is
% given in the file `manifest.txt'. See also `legal.txt' for additional
% information.
%
% The list of derived (unpacked) files belonging to the distribution
% and covered by LPPL is defined by the unpacking scripts (with
% extension .ins) which are part of the distribution.
%
% \fi
% \iffalse
%%% From File: ltfntcmd.dtx
%%
%<*driver>
% \fi
\ProvidesFile{ltfntcmd.dtx}
             [2021/09/12 v3.5a LaTeX Kernel (Font commands)]
% \iffalse
\documentclass{ltxdoc}
\GetFileInfo{ltfntcmd.dtx}
\title{\filename}
\date{\filedate}
\author{Frank Mittelbach, Chris Rowley}
\begin{document}
 \MaintainedByLaTeXTeam{latex}
 \maketitle
 \DocInput{\filename}
\end{document}
%</driver>
% \fi
%
%
% \iffalse
% Copyright (C) 1994-1995 Frank Mittelbach and Chris Rowley
% all rights reserved.
% \fi
%
% \changes{v2.0b}{1993/05/05}{Removed all LaTeX related cmds}
% \changes{v3.0a}{1993/12/11}{Complete reworking of all text commands,
% using just one creator function}
% \changes{v3.0a}{1993/12/11}{newfontswitch command corrected and
% changed}
% \changes{v3.0a}{1993/12/11}{newcommands replaced by defs}
% \changes{v3.0a}{1993/12/11}{italic correction now put in front of
% penalty before glue}
% \changes{v3.0b}{1993/12/19}{\cs{@pdef} command added}
% \changes{v3.0b}{1993/12/19}{Other tidying}
% \changes{v3.0b}{1993/12/19}{Made \cs{@newfontswitch} produce an
% error if command already exists, and added \cs{@renewfontswitch},
% ASAJ}
% \changes{v3.0b}{1993/12/19}{Untidying added, so this is now a
%    TEMPORARY version.}
% \changes{v3.0b}{1993/12/19}{Some more tidying done}
% \changes{v3.1b}{1994/01/31}{\cs{@normalsize} no longer defined}
% \changes{v3.2a}{1994/03/15}{Adapted to mass formatting}
% \changes{v3.2a}{1994/03/15}{Removed defs of short-forms and all
% sizes except \cs{normalize}}
% \changes{v3.2a}{1994/03/15}{Removed \cs{@renewfontswitch}}
% \changes{v3.2a}{1994/03/15}{Changed \cs{/} to \cs{@@italiccorr}}
% \changes{v3.2b}{1994/04/14}{Macros renamed to non-private forms, JB}
% \changes{v3.3a}{1994/04/20}{New implementation of \cs{nocorr}}
% \changes{v3.3a}{1994/04/20}{Documentation up-dated}
% \changes{v3.3b}{1994/04/30}{Documentation up-dated and tidied}
% \changes{v3.3b}{1994/04/30}{Title changed}
% \changes{v3.3b}{1994/04/30}{Prefix frag@ changed to frag in
% \cs{@protecteddef}}
% \changes{v3.3b}{1994/04/30}{Warning changed to info message in
% \cs{@protecteddef}}
% \changes{v3.3c}{1994/05/05}{Corrected \cs{@fontswitch}}
% \changes{v3.3d}{1994/05/08}{Removed \cs{@undefinedfonterror}}
% \changes{v3.3f}{1994/05/09}{Replaced all \cs{next} by
%                          \cs{@let@token} and undo change 3.3e,
%                          whatever that was.}
% \changes{v3.3g}{1994/05/13}{Replaced \cs{@protecteddef} by
%                             \cs{DeclareRobustCommand}}
% \changes{v3.3h}{1994/05/20}{Use new error commands}
% \changes{v3.3i}{1994/05/24}{Tidying and typos fixed}
% \changes{v3.3j}{1994/05/25}{Insertion of \cs{aftergroup}s to
% implement \cs{nocorr} moved to the end of the group}
% \changes{v3.3k}{1994/06/09}{Tidying and typos fixed in documentation}
% \changes{v3.3l}{1994/06/18}{Added check for empty text}
% \changes{v3.3m}{1994/06/22}{Removed space from \cs{nfss@text}}
% \changes{v3.3m}{1994/06/22}{Renamed \cs{check@nocorr}}
% \changes{v3.3p}{1994/11/17}
%         {\cs{@tempa} to \cs{reserved@a}}
% \changes{v3.3v}{1997/06/16}{Fix typo in documentation.}
% \changes{v3.3x}{1998/08/17}{(RmS) Minor documentation fixes.}
%
% \def \ie {i.e.~}
% \def \eg {e.g.~}
%
% \def \dst {{\normalfont\scshape docstrip}}
% \def \NFSS {\textsf{NFSS}}
%
%
% \begin{abstract}
%   The commands defined in this file \texttt{ltfntcmd} are
%   part of the kernel code for \LaTeXe/NFSS2.
%
%   It is also meant to serve as documentation for package writers since
%   it demonstrates how to define high-level font changing commands
%   using a small number of creator functions.
% \end{abstract}
%
% \section{Introduction}
%
% Font changes such as |\bfseries|, |\sffamily|, etc.\ are
% declarations; this means that their scope is delimited by the
% grouping structure, either by the next |\end| of some environment or
% by explicitly using a group, e.g., writing something like
% |{\bfseries...}| in the source. If you make the mistake of writing
% |\bfseries{...}| (thinking of |\bfseries| as a command with one
% argument) then the result is rather striking.
%
% Font declarations are an artifact of the \TeX{} system and for
% several reasons it is better to avoid them on the user level
% whenever possible.  In \LaTeX3 they will probably all be
% replaced by environments and by font commands taking one argument.
%
% This file defines a creator function for such declarative font
% switches.  This function creates commands which can be used in
% both math and text.
%
% This file also defines a number of high-level commands (all
% starting with |\text..|) that have one argument and typeset this
% argument in the requested way. Thus these commands are for
% typesetting short pieces of text in a specific family, series or
% shape.  These are all produced as examples of the use of a creator
% function which is itself also defined in this file.
%
% Table~\ref{tab:fontcmds} shows all these high-level commands in
% action.  A further advantage of using these commands is that they
% automatically take care of any necessary italic correction on either
% side of their argument.
% \begin{table}[t]
% \begin{center}
% \begin{tabular}{lll}
% \textit{Command} & \textit{Corresponds to }& \textit{Action} \\[4pt]
% |\textnormal{..}| & |\normalfont| &
%            Typeset argument in normal family  \\
% |\textrm{..}| & |\rmfamily| &
%            Typeset argument in roman family  \\
% |\textsf{..}| & |\sffamily| &
%            Typeset argument in \textsf{sans serif} family  \\
% |\texttt{..}| & |\ttfamily| &
%            Typeset argument in \texttt{typewriter} family
%                                                            \\[2pt]
% |\textmd{..}| & |\mdseries| &
%             Typeset argument in medium series    \\
% |\textbf{..}| & |\bfseries| &
%             Typeset argument in \textbf{bold} series    \\[2pt]
% |\textup{..}| & |\upshape| &
%             Typeset argument in normal shape  \\
% |\textit{..}| & |\itshape| &
%             Typeset argument in \textit{italic} shape  \\
% |\textsl{..}| & |\slshape| &
%             Typeset argument in \textsl{slanted} shape  \\
% |\textsc{..}| & |\scshape| &
%             Typeset argument in \textsc{small caps} shape
%                                                            \\[2pt]
% |\emph{..}| & |\em| &
%              Typeset argument \emph{emphasized}
% \end{tabular}
% \end{center}
% \caption{Font-change commands with arguments}
% \label{tab:fontcmds}
% \begin{quote}
% The font change commands provided here
% all start with |\text..| to emphasize that they
% are for use in normal text and to be easily memorable. They
% automatically take care of any necessary italic correction on either
% side of the argument.
% \end{quote}
% \end{table}
%
% Thus, when using such commands, one does not have to worry about
% forgetting the italic correction when changing fonts.  Only in very
% few situations is this additional space wrong but, for example, most
% typographers recommend omitting the italic correction if a small
% punctuation character, like a comma, directly follows the font
% change.  Since the amount of correction required is partly a matter
% of taste, you can define in what situations the italic correction
% should be suppressed. This is done by putting the characters that
% should cancel a preceding italic correction in the list
% |\nocorrlist|.\footnote{Any package that changes the
% \texttt{\string\catcode} of a character inside
% \texttt{\string\nocorrlist} must then explicitly reset the list.
% Otherwise the changed character will no longer be recognized
% by the suppression algorithm.}
% The default definition for this list is produced by the following.
% \begin{verbatim}
% \newcommand \nocorrlist {,.}
% \end{verbatim}
% It is best to declare the most often used characters first, because
% this will make the processing slightly faster.  For example,
% \begin{verbatim}
% \emph{When using the \NFSS{} high-level commands,
% the \emph{proper} use of italic corrections is
% automatically taken care of}. Only
% \emph{sometimes} one has to help \LaTeX{} by
% adding a \verb=\nocorr= command.
% \end{verbatim}
% which results in:
% \begin{quote}
% \emph{When using the \NFSS{} high-level commands,
% the \emph{proper} use of italic corrections is
% automatically taken care of}. Only
% \emph{sometimes} one has to help \LaTeX{} by
% adding a \verb=\nocorr= command.
% \end{quote}
%
% In contrast, the use of the declaration forms is often more
% appropriate when you define your own commands or environments.
% \begin{verbatim}
% \newenvironment{bfitemize}{\begin{itemize}\normalfont\bfseries}
%                           {\end{itemize}}
% \begin{bfitemize}
% \item This environment produces boldface items.
% \item It is defined in terms of \LaTeX's
%   \texttt{itemize} environment and NFSS
%   declarations.
% \end{bfitemize}
% \end{verbatim}
% This gives:
% \begin{quote}
% \newenvironment{bfitemize}
%   {\begin{itemize}\normalfont\bfseries}
%   {\end{itemize}}
% \begin{bfitemize}
% \item This environment produces boldface items.
% \item It is defined in terms of \LaTeX's
%   \texttt{itemize} environment and NFSS
%   declarations.
% \end{bfitemize}
% \end{quote}
%
% In addition to global customization of when to insert the italic
% correction, it is of course sometimes necessary to explicitly insert
% one with |\/|.
%
% It is also possible to suppress the italic correction
% in individual instances. For this, the command |\nocorr| is provided.
%
% The |\nocorr| must appear as the first or last token inside the
% braces of the argument of the |\text...| commands, at that end of
% the text where you wish to suppress the italic correction.
%
%
% It is worth pointing out here that inserting a |\/| in places where
% it can have no function (\ie anywhere except immediately after a
% slanted letter) is not an error---it will just be silently ignored.
% Unfortunately this is not true if the redefinition of |\/| in {\tt
% amstex.sty} is used as this version can cause space to be removed
% immediately before the |\/|.
%
% \MaybeStop{}
%
%
% \section{The implementation}
%
% \changes{v3.3z}{2003/01/01}{Code checked and documentation extended
%  by Chris}
%
%  \begin{macro}{\DeclareTextFontCommand}
% \changes{v3.0a}{1993/12/11}{Macro changed}
% \changes{v3.0a}{1993/12/11}{Macro changed}
%    This is the creator function for |\text..| commands.
%    It gives a warning if |\foo| or |\fragfoo| is already defined.
% \changes{v3.0b}{1993/12/19}{Corrected and tidied}
%
%    In math mode it simply puts the font declaration and text into a
%    box (possibly an automagically sized one).
%
%    Otherwise it first scans the text to see where |\nocorr| occurs
%    within it.  This sets the |\check@ic| commands to do what is
%    necessary concerning the italic correction at both ends.
%
%    The algorithm for deciding whether to put in an italic correction
%    is not very subtle: one is added whenever the newly current font
%    is not itself positively sloped, unless the next token is a
%    character in the `nocorr' list.  At the end of the text this
%    is done after closing the group so as to check the `outer font'.
%    Note that this
%    will often result in adding an italic correction token after a
%    character in an unsloped font; we believe (in early 2003) that
%    this is perhaps inefficient but not dangerous.
%
%    It also now checks for empty contents of the text command and optimizes
%    this case.  Some care is also taken to check that doing dangerous
%    things in vertical mode is avoided.
%
%    The italic correction token is added to the horizontal list
%    before (in the list) an immediately preceding non-zero glob of
%    glue (skip) and any non-zero penalty preceding that since, in
%    the typical case, this puts it immediately
%    after the last character in the preceding word.
%
%
% \changes{v3.3j}{1994/05/25}{Insertion of \cs{aftergroup}s moved}
% \changes{v3.3j}{1994/05/25}{\cs{expandafter} inserted}
% \changes{v3.3m}{1994/06/22}{Removed space from \cs{nfss@text}}
% \changes{v3.3u}{1996/10/08}{Removed \cs{check@icr} when in vmode
% since it causes various errors (see pr/2157)}
% \changes{v3.3w}{1997/10/17}{Reinstalled \cs{check@icr} as check is
%    now done in \cs{check@nocorr@} (see PR/2646).}
% \changes{v3.3y}{2000/01/30}{Use \cs{hmode@bgroup} now (pr/3160)}
%
%    Note that it is necessary to put in the |\aftergroup\maybe@ic| at
%    the end of the group so that it comes after any other aftergroup
%    tokens and immediately before the following tokens.
%    It is also necessary to remove the |\fi| from the token list before
%    the group ends; this is done by adding an |\expandafter| just
%    before the closing brace.
%
%    \begin{macrocode}
%<*2ekernel>
\def \DeclareTextFontCommand #1#2{%
  \DeclareRobustCommand#1[1]{%
    \ifmmode
      \nfss@text{#2##1}%
    \else
      \hmode@bgroup
       \text@command{##1}%
       #2\check@icl ##1\check@icr
       \expandafter
      \egroup
    \fi
                       }%
}
%    \end{macrocode}
%  \end{macro}
%
%  \begin{macro}{\textrm}
%  \begin{macro}{\textsf}
%  \begin{macro}{\texttt}
%  \begin{macro}{\textnormal}
%    Now we define the |\text|\meta{family} commands in terms of the
%    above; |\texttt| does not look very nice!
%    \begin{macrocode}
\DeclareTextFontCommand{\textrm}{\rmfamily}
\DeclareTextFontCommand{\textsf}{\sffamily}
\DeclareTextFontCommand{\texttt}{\ttfamily}
\DeclareTextFontCommand{\textnormal}{\normalfont}
%    \end{macrocode}
%  \end{macro}
%  \end{macro}
%  \end{macro}
%  \end{macro}
%
%  \begin{macro}{\textbf}
%  \begin{macro}{\textmd}
%    For the series attribute:
%    \begin{macrocode}
\DeclareTextFontCommand{\textbf}{\bfseries}
\DeclareTextFontCommand{\textmd}{\mdseries}
%    \end{macrocode}
%  \end{macro}
%  \end{macro}
%
%  \begin{macro}{\textit}
%  \begin{macro}{\textsl}
%  \begin{macro}{\textsc}
%  \begin{macro}{\textup}
% \changes{v3.0a}{1993/12/11}{Macros changed}
%    And for the shapes:
%    \begin{macrocode}
\DeclareTextFontCommand{\textit}{\itshape}
\DeclareTextFontCommand{\textsl}{\slshape}
\DeclareTextFontCommand{\textsc}{\scshape}
\DeclareTextFontCommand{\textup}{\upshape}
%    \end{macrocode}
%  \end{macro}
%  \end{macro}
%  \end{macro}
%  \end{macro}


%  \begin{macro}{textulc}
%  \begin{macro}{textsw}
%  \begin{macro}{textssc}
%
% \changes{v3.4c}{2019/12/17}{Macro added}
%    \begin{macrocode}
%</2ekernel>
%<*2ekernel|latexrelease>
%<latexrelease>\IncludeInRelease{2020/02/02}%
%<latexrelease>                 {\textulc}{Additional text commands}%
\DeclareTextFontCommand{\textulc}{\ulcshape}
\DeclareTextFontCommand{\textsw}{\swshape}
\DeclareTextFontCommand{\textssc}{\sscshape}
%</2ekernel|latexrelease>
%<latexrelease>\EndIncludeInRelease
%<latexrelease>\IncludeInRelease{0000/00/00}%
%<latexrelease>                 {\textulc}{Additional text commands}%
%<latexrelease>
%<latexrelease>\let\textulc\@undefined
%<latexrelease>\let\textsw\@undefined
%<latexrelease>\let\textssc\@undefined
%<latexrelease>\EndIncludeInRelease
%<*2ekernel>
%    \end{macrocode}
%  \end{macro}
%  \end{macro}
%  \end{macro}
%
%  \begin{macro}{\emph}
% \changes{v3.0a}{1993/12/11}{Macro changed}
%    Finally we have the |\em| font change declaration of \LaTeX. The
%    corresponding definition with argument is
%    \begin{macrocode}
\DeclareTextFontCommand{\emph}{\em}
%    \end{macrocode}
%  \end{macro}
%
%  \begin{macro}{\nocorr}
%  This is just a label, so it does nothing; it should also be
%  unexpandable.
%    \begin{macrocode}
\let \nocorr \relax
%    \end{macrocode}
%  \end{macro}
%
%  \begin{macro}{\check@icl}
%  \begin{macro}{\check@icr}
% \changes{v3.3j}{1994/05/25}{Macros added}
% \changes{v3.3t}{1996/05/09}{Default definitions added}
%  We define these defaults in case some error causes them to be
%  expanded at the wrong time.
%    \begin{macrocode}
\let \check@icl \@empty
\let \check@icr \@empty
%    \end{macrocode}
%  \end{macro}
%  \end{macro}
%
%  \begin{macro}{\text@command}
%  \begin{macro}{\check@nocorr@}
% \changes{v3.3a}{1994/04/20}{Macros added}
% \changes{v3.3m}{1994/06/22}{Renamed \cs{check@nocorr} to
% \cs{text@command} to improve \cs{long} error message}
% \changes{v3.3j}{1994/05/25}{Insertion of \cs{aftergroup}s moved
% and defaults set up for efficiency}
%    This checks for a |\nocorr| as the first token in its argument
%    and also for one in any other position not protected within
%    braces (the latter is treated as if it were at the end of the
%    argument).
%
%    Is this the correct action in the `empty' case? It is efficient but
%    typographically it is, strictly, incorrect!
% \changes{v3.3l}{1994/06/18}{Added check for empty text}
% \changes{v3.5a}{2021/09/12}{use \cs{unexpanded} to make \# safe}
%    \begin{macrocode}
\def \text@command #1{%
  \edef \reserved@a {\unexpanded{#1}}%
  \ifx \reserved@a \@empty
    \let \check@icl \@empty
    \let \check@icr \@empty
  \else
%    \end{macrocode}
%    |\space| is a reserved word in \LaTeX{} or actually already in
%    plain \TeX. If somebody really redefines it so many things will
%    break that I don't see any reason to make this routine here
%    slower than necessary.
% \changes{v3.3q}{1994/12/10}{Use \cs{space} command for comparison}
%    \begin{macrocode}
%    \def \reserved@b { }%
%    \ifx \reserved@a \reserved@b
    \ifx \reserved@a \space
      \let \check@icl \@empty
      \let \check@icr \@empty
    \else
      \check@nocorr@ #1\nocorr\@nil
    \fi
  \fi
}
\def \check@nocorr@ #1#2\nocorr#3\@nil {%
%    \end{macrocode}
%    The two checks are initialised here to their values in
%    the normal case.
% \changes{v3.3w}{1997/10/17}{Check for vertical mode moved here, from
%    \cs{DeclareTextFontCommand} (see PR/2646).}
%    \begin{macrocode}
  \let \check@icl \maybe@ic
  \def \check@icr {\ifvmode \else \aftergroup \maybe@ic \fi}%
  \def \reserved@a {\nocorr}%
  \def \reserved@b {#1}%
  \def \reserved@c {#3}%
  \ifx \reserved@a \reserved@b
    \ifx \reserved@c \@empty
%    \end{macrocode}
%    In this case there is a |\nocorr| at the start but not at the
%    end, so |\check@icl| should be empty.
%    \begin{macrocode}
      \let \check@icl \@empty
    \else
%    \end{macrocode}
%    Otherwise there is a |\nocorr| both at the start and
%    elsewhere, so no italic corrections should be added.
%    \begin{macrocode}
      \let \check@icl \@empty
      \let \check@icr \@empty
    \fi
  \else
    \ifx \reserved@c \@empty
%    \end{macrocode}
%    In this case there is no |\nocorr| anywhere, so we need to check
%    for an italic correction at both the beginning and the end.  This
%    has been set up as the default so no code is needed here.
%    \begin{macrocode}
    \else
%    \end{macrocode}
%    In this case there is no |\nocorr| at the start but there is one
%    elsewhere, so no |\aftergroup| is needed.
%    \begin{macrocode}
      \let \check@icr \@empty
    \fi
  \fi
}
%    \end{macrocode}
%  \end{macro}
%  \end{macro}
%
% \begin{macro}{\ifmaybe@ic}
% \changes{v3.4a}{2009/12/14}{Macro added}
%    Switch used solely within |\maybe@ic| not interfering with other
%    switches.
%    \begin{macrocode}
\newif\ifmaybe@ic
%    \end{macrocode}
% \end{macro}

%  \begin{macro}{\maybe@ic}
% \changes{v3.0a}{1993/12/11}{Macro name changed}
%  \begin{macro}{\maybe@ic@}
% \changes{v3.0a}{1993/12/11}{Macro and name changed}
% \changes{v3.4a}{2009/12/14}{Use switch \cs{ifmaybe@ic} instead of
%    \cs{if@tempswa}}
% \task{CAR}{Make this more efficient by checking for non-characters
% first and using \cs{if} test?}
%    These macros implement the italic correction.
%    \begin{macrocode}
\def \maybe@ic {\futurelet\@let@token\maybe@ic@}
\def \maybe@ic@ {%
%    \end{macrocode}
%    We first check to see if the current font is positively sloped.
%    (But do not forget the message Rainer sent about an upright font
%    with non-zero slope!  Or is this an urban myth?) It has been
%    suggested that this should test against a small positive value,
%    but what?
%    \begin{macrocode}
  \ifdim  \fontdimen\@ne\font>\z@
  \else
    \maybe@ictrue
%    \end{macrocode}
%    It would be possible, but probably not worthwhile, to continue the
%    forward scan beyond any closing braces.
%
% \changes{v3.3a}{1994/04/20}{\cs{nocorr} etc removed from list of
%                 tokens to check, leaving only punctuation characters}
%    \begin{macrocode}
    \expandafter\@tfor\expandafter\reserved@a\expandafter:\expandafter=%
        \nocorrlist
%    \end{macrocode}
%    We have to hide the |\@let@token| in the macro |\t@st@ic| rather
%    than testing it directly in the loop since it might be |\let|
%    to a |\fi| or |\else|, which would result in chaos.
% \changes{v2.1a}{1993/11/24}{Use \cs{t@st@ic}}
%    \begin{macrocode}
    \do \t@st@ic
%    \end{macrocode}
%    Frank thinks that the next bit is inefficient if done after
%    the second change.  Chris thinks that most all of this is
%    inefficient for the commonest cases: but that is the price of a
%    cleverer algorithm.  It is certainly needed to deal with the use
%    of |\nolinebreak|.
%
%    \begin{macrocode}
    \ifmaybe@ic \sw@slant \fi
  \fi
}
%    \end{macrocode}
%  \end{macro}
%  \end{macro}
%
%  \begin{macro}{\t@st@ic}
% \changes{v2.1a}{1993/11/24}{Macro added}
% \changes{v3.3s}{1995/07/05}{Renamed from \cs{test@next}}
%    The next token in the input stream is stored in |\@let@token| via a
%    |\let|, the current token from |\nocorrlist| is stored via |\def|
%    in |\reserved@a|. To compare them we have to fiddle around a bit.
%
%    If the only things to check were characters then this could be
%    done via an |\if| thus their catcodes would not matter; but this
%    will not work whilst |\futurelet| is used above.
% \changes{v3.4a}{2009/12/14}{Use switch \cs{ifmaybe@ic} instead of
%    \cs{if@tempswa}}
%    \begin{macrocode}
\def \t@st@ic {%
  \expandafter\let\expandafter\reserved@b\expandafter=\reserved@a\relax
  \ifx\reserved@b\@let@token
%    \end{macrocode}
%    If they are the same we record the fact and jump out of the loop.
% \changes{v3.3s}{1995/07/03}{Use clean interface for jump}
%    \begin{macrocode}
    \maybe@icfalse
    \@break@tfor
  \fi
}
%    \end{macrocode}
%  \end{macro}
%
%  \begin{macro}{\sw@slant}
% \changes{v3.0a}{1993/12/11}{Macro changed}
%  \begin{macro}{\fix@penalty}
% \changes{v3.0a}{1993/12/11}{Macro added}
%    The definition of the mysterious |\sw@slant| command is as follows.
%    \begin{macrocode}
\def \sw@slant {%
%    \end{macrocode}
%    It is surely correct to put in an italic correction when there
%    is no skip.  If the last thing on the list is actually a zero
%    skip (including things whose dimension part is zero, such as
%    |\hfill|), or anything other than a character, then the
%    italic correction will have no effect.
%
%    In order to work correctly with unbreakable spaces from |~|
%    (and other common forms of line-breaking control) we also move
%    back across a penalty before the glue.
%    \begin{macrocode}
  \ifdim \lastskip=\z@
    \fix@penalty
  \else
    \skip@ \lastskip
    \unskip
    \fix@penalty
    \hskip \skip@
  \fi
}
%    \end{macrocode}
%    The above code means: ``If there is a non-zero space just before
%    the current position (|\ifdim...|) save the amount of that space
%    (|\skip@\lastskip|), remove it (|\unskip|), then do a similar
%    thing if there is a penalty just before the skip,
%    and finally put the space back in.''
%
%    Since zero glue cannot be distinguished in this context from no
%    glue, we dare not put in an |\hskip| in this case as this may
%    produce an unwanted breakpoint.  This is not satisfactory.
%
%    The penalty before the glue is handled similarly, with the same
%    caveats concerning the zero case.   Is this the first recorded
%    use of |\unpenalty| in standard \LaTeX{} code?
%    \begin{macrocode}
\def \fix@penalty {%
  \ifnum \lastpenalty=\z@
    \@@italiccorr
  \else
    \count@ \lastpenalty
    \unpenalty
    \@@italiccorr
    \penalty \count@
  \fi
}
%    \end{macrocode}
%  \end{macro}
%  \end{macro}
%
%  \begin{macro}{\nocorrlist}
%    This holds the list of characters that should prevent italic
%    correction.  They should be ordered by decreasing frequency of
%    use. If any such character is made active later on one needs
%    to redefine the list so that the active character becomes part
%    of it.
% \changes{v3.1a}{1994/01/27}{Only ., used as default for cm fonts}
%    \begin{macrocode}
\def \nocorrlist {,.}
%    \end{macrocode}
%  \end{macro}
%
%
%  \begin{macro}{\nfss@text}
%    This command will by default behave like a \LaTeX{} |\mbox| but
%    may be redefined by packages such as |amstext.sty| to be a bit
%    cleverer.
%    \begin{macrocode}
\ifx \nfss@text\@undefined
  \def \nfss@text {\leavevmode\hbox}
\fi
%    \end{macrocode}
%  \end{macro}
%
%
%  \begin{macro}{\DeclareOldFontCommand}
% \changes{v3.2b}{1994/04/14}{Renamed from \cs{@newfontswitch}}
%    This is the function used to create declarative font-changing
%    commands that can also be used to change alphabets in math-mode.
%
%    Usage: |\DeclareOldFontCommand \fn{|\meta{font-change decls}|}|
%                                        \meta{math-alphabet}
%
%    Here |\fn| is the font-declaration command being defined,
%    \meta{font-change decls} is the declaration it will expand to in
%    text-mode, and \meta{math-alphabet} is the (single) math alphabet
%    specifier which is to be used in math-mode.
%
%    It does not care whether the command being defined already
%    exists but it does give a warning if it redefines anything.
%
%    Here are some typical examples of its use in conjunction with
%    more basic NFSS2 font commands.
%    \begin{verbatim}
%    \DeclareOldFontCommand{\rm}{\normalfont\rmfamily}{\mathrm}
%    \DeclareOldFontCommand{\sf}{\normalfont\sffamily}{\mathsf}
%    \DeclareOldFontCommand{\tt}{\normalfont\ttfamily}{\mathtt}
%    \end{verbatim}
%
% \changes{v3.0b}{1993/12/19}{Corrected and tidied}
%    \begin{macrocode}
\def \DeclareOldFontCommand #1#2#3{%
  \DeclareRobustCommand #1{\@fontswitch {#2}{#3}}%
}
%    \end{macrocode}
%  \end{macro}
%
%  \begin{macro}{\@fontswitch}
%  \begin{macro}{\@@math@egroup}
%  \begin{macro}{\@@math@egroup}
%    These two commands actually do the necessary tests and
%    declarative \mbox{font-} or alphabet-changing.
%
% \changes{v3.3c}{1994/05/05}{Corrected \cs{@fontswitch} and added
% saved versions}
% \changes{v3.3q}{1994/12/10}{Don't read arguments}
% \changes{v3.3r}{1995/04/02}{Read them again to be able to add \cs{relax}.}
%    \begin{macrocode}
\def \@fontswitch #1#2{%
  \ifmmode
     \let \math@bgroup \relax
     \def \math@egroup {\let \math@bgroup \@@math@bgroup
                        \let \math@egroup \@@math@egroup}%
%    \end{macrocode}
%    We need to have a |\relax| in the following line in case the |#2|
%    is something like |\mathsf| grabbing the next token as an
%    argument. For this reason the code also uses explicit arguments
%    again (see pr/1275).
%    \begin{macrocode}
     #2\relax
  \else
     #1%
  \fi
}
\let \@@math@bgroup \math@bgroup
\let \@@math@egroup \math@egroup
%    \end{macrocode}
%  \end{macro}
%  \end{macro}
%  \end{macro}
%
% \changes{v2.1b}{1993/12/08}{Macros \cs{rm}, \cs{bf} and \cs{sf}
%                 moved to classes.dtx}
%
%    These commands are available only in the preamble.
%    \begin{macrocode}
\@onlypreamble \DeclareTextFontCommand
\@onlypreamble \DeclareOldFontCommand
%    \end{macrocode}
%
% \section{Initialization}
% \changes{v3.0b}{1993/12/19}{Added by ASAJ.}
% \changes{v3.0b}{1993/12/19}{Wording changes by CAR.}
%
% \begin{macro}{\normalsize}
% \changes{v3.3d}{1994/05/08}{Removed \cs{@undefinedfonterror}}
% \changes{v3.3o}{1994/11/12}{Added \cs{MessageBreak}}
%    This is defined to produce an error.
%    \begin{macrocode}
\def\normalsize{%
  \@latex@error {The font size command \protect\normalsize\space
              is not defined:\MessageBreak
              there is probably something wrong with
              the class file}\@eha
}
%</2ekernel>
%    \end{macrocode}
% \end{macro}
%
% \Finale
\endinput
