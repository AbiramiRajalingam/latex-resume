% \iffalse meta-comment
%
% Copyright (C) 1993-2024
% The LaTeX Project and any individual authors listed elsewhere
% in this file.
%
% This file is part of the LaTeX base system.
% -------------------------------------------
%
% It may be distributed and/or modified under the
% conditions of the LaTeX Project Public License, either version 1.3c
% of this license or (at your option) any later version.
% The latest version of this license is in
%    https://www.latex-project.org/lppl.txt
% and version 1.3c or later is part of all distributions of LaTeX
% version 2008 or later.
%
% This file has the LPPL maintenance status "maintained".
%
% The list of all files belonging to the LaTeX base distribution is
% given in the file `manifest.txt'. See also `legal.txt' for additional
% information.
%
% The list of derived (unpacked) files belonging to the distribution
% and covered by LPPL is defined by the unpacking scripts (with
% extension .ins) which are part of the distribution.
%
% \fi
%
% \iffalse
% Copyright (C) 1994-97 LaTeX Project, Frank Mittelbach
% and Rainer Sch\"opf, all rights reserved.
%
%    \begin{macrocode}
%<+class>\NeedsTeXFormat{LaTeX2e}[1997/06/01]
%    \end{macrocode}
%
%    Announce the Class name and its version:
% \changes{v2.3b}{1994/05/01}{Removed the use of \cs{fileversion}
%    c.s.}
% \changes{v2.3b}{1994/05/01}{Added \cs{ProvidesFile} to slides.def.}
% \changes{v2.3o}{1995/05/17}{Replaced all \cs{hbox to} with
%    \cs{hb@xt@}.}
% \changes{v2.3o}{1995/05/17}{Cleaned up the \cs{changes} entries.}
% \changes{v2.3p}{1995/05/23}{Corrected some brace mismatches.}
% \changes{v2.3q}{1995/09/20}{Replaced \cs{@tempa} by \cs{reserved@a}.}
% \changes{v2.3r}{1995/09/27}{Globally replaced scale factor 19.91 by
%          19.907 in \cs{DeclareFontShape}, as this gives better
%          rounded font sizes at 600dpi (suggested by Denis Roegel).}
%    \begin{macrocode}
%<+class>\ProvidesClass{slides}
%<+cmd>\ProvidesFile{slides.def}
%<*driver>
       \ProvidesFile{slides.drv}
%</driver>
              [2022/05/18 v2.4b
%<+class>               Standard LaTeX document class]
%<+cmd>               SLiTeX definitions]
%    \end{macrocode}
%
% \section{A driver for this document}
%
% \changes{v2.3f}{1994/05/26}{Moved the driver to the front of the
%    file; it doesn't print any longer}
%
% The next bit of code contains the documentation driver file for
% \TeX{}, i.e., the file that will produce the documentation you are
% currently reading. It can be extracted from this file by the
% \dst{} program.
%    \begin{macrocode}
%<*driver>
]
\documentclass{ltxdoc}
%    \end{macrocode}
%    The following command retrieves the date and version information
%    from the file.
%    \begin{macrocode}
\GetFileInfo{slides.drv}
%    \end{macrocode}
%    Some commonly used abbreviations:
% \changes{v2.3h}{1994/06/01}{Added definition of \cs{SLiTeX}.}
%    \begin{macrocode}
\DeclareRobustCommand{\SLiTeX}{{%
   \normalfont S\kern -.06em
   {\scshape l\kern -.035emi}\kern -.06em
   \TeX
}}
\newcommand*{\Lopt}[1]{\textsf {#1}}
\newcommand*{\file}[1]{\texttt {#1}}
\newcommand*{\Lcount}[1]{\textsl {\small#1}}
\newcommand*{\pstyle}[1]{\textsl {#1}}
\newcommand*{\dst}{{\normalfont\scshape docstrip}}
%
%    \end{macrocode}
%    Now start the document and input this file.
% \changes{v2.3d}{1994/05/12}{Added a missing \cs{begin\{macrocode\}}}
%    \begin{macrocode}
\begin{document}
   \DocInput{slides.dtx}
\end{document}
%</driver>
%    \end{macrocode}
%\fi
%
%
% \changes{v2.1b}{1993/12/13}{Removed \cs{CodelineIndex} from the
%    driver code.}
% \changes{v2.1e}{1993/12/16}{Removed again different definitions for
%    font commands font commands in non-compatibility mode.}
%
% \changes{v2.2}{1993/12/18}{Changes to make it work with
%    compatibility mode.}
% \changes{v2.2a}{1993/12/19}{Removed float parms}
% \changes{v2.2d}{1994/01/31}{Removed \cs{@normalsize definition}}
% \changes{v2.2g}{1994/03/01}{Removed option makeidx.}
% \changes{v2.2g}{1994/03/01}{Moved driver up front.}
% \changes{v2.2g}{1994/03/01}{Renamed files \file{slides.ltx} and
%    \file{sfontdef.ltx} to \file{slides.def} and \file{sfonts.def}.}
% \changes{v2.2l}{1994/03/16}{Removed \cs{typeouts}}
% \changes{v2.3}{1994/03/16}{Removed root/slide-file structure except
%    for compatibility mode. (LL)}
% \changes{v2.3}{1994/03/16}{Added clock option (LL)}
% \changes{v2.3}{1994/03/16}{Modified slide, overlay, note
%    environments. (LL)}
% \changes{v2.3}{1994/03/16}{Added titlepage option and
%    \cs{maketitle}. (LL)}
% \changes{v2.3c}{1994/05/06}{Changed documentation to use
%    `environment' instead of `macro' environment for environments}
% \changes{v2.3f}{1994/05/26}{Wrapped two long lines}
% \changes{v2.3l}{1994/12/16}{Use \cs{newcommand*} to define commands
%    with an argument}
% \changes{v2.3n}{1995/04/02}{a slight documentation fix (PR 1517)}
%
% \title{Producing slides with \LaTeXe{}}
% \author{Frank Mittelbach}
% \date{\filedate}
%
% \maketitle
% \MaintainedByLaTeXTeam{latex}
%
% \section{Introduction}
%
% With \LaTeXe{} it is now no longer necessary to maintain a special
% format for producing overhead slides. Instead the standard format
% may be used and internally only different font definition files come
% into play.
%
% \section{Usage}
%
% For producing slides you have to use |slides| as the
% document class. This class is very similar to the |slides| style
% that came with \SliTeX{}, in fact it is basically a copy changed to
% work under \LaTeXe{}.\footnote{Therefore you should compare the
% new class with old \SliTeX{} styles
% in case you have local slide classes to see what you have to change
% in order to use them with \LaTeXe{}.} Thus you have to say something
% like
% \begin{verbatim}
%   \documentclass[...]{slides}
% \end{verbatim}
% and process this with \LaTeXe.
%
% \section{Fonts}
%
% Note, that with NFSS you can easily produce slides with special
% fonts just by calling an appropriate style file (like |times|) in a
% |\usepackage| command. This works, for example, with all
% fonts that are defined to be scaleable (e.g., PostScript fonts) since
% they can be used at any size by NFSS.
%
% However, packages like |pandora| won't work because the standard
% |.fd| files shipped with NFSS only contain small sizes. You can, of
% course, produce additional sizes and change the |.fd| files
% accordingly so that they would be useable for slides as well.
%
% \section{Invisible text and color separation}
%
% In the original \SliTeX{} it was possible to produce invisible text
% using the |\invisible| command, so that one was able to put several
% slides on top of each other (with each slides showing additional
% details, etc.). It was also possible to produce `color' slides. This
% was done by producing individual slides one for each color and
% placing them on top of each other.
%
% The availability of color printers and the |color| package make
% color separation obsolete, so it has been removed.  Although the
% |color| has also made |\invisible| obsolete, the command is
% retained in the \LaTeXe{} implementation, but there
% are a few restrictions. Invisible fonts are implemented as special
% shapes where the shape names are build by prefixing the normal shape
% name with an uppercase |I|. For example, the `normal invisible
% shape' would be |In|. When \LaTeX{} is requested to typeset
% invisible it will thus change the current shape attribute in this
% manner. To make this work it is necessary that the resulting font
% shape group is defined. If not, the normal font substitution
% mechanism of \LaTeXe{} will change the attribute until it finds a
% usable font shape group with the result that the text may become
% visible.
%
% As long as you use the standard fonts for slides this is not a
% problem because all the visible font shape groups have invisible
% counterparts. However, if you decide on using special fonts, e.g.,
% PostScript fonts, your |\DeclareFontShape| settings may not contain
% invisible font shape groups and thus you may be unable to use these
% features without adding additional |\DeclareFontShape| commands to
% your |.fd| files or the preamble of your document.
%
% \MaybeStop{}
%
%
% \section{The Implementation}
%
% \begin{quote}
% \textbf{Warning:} The implementation is still very experimental and
% may change internally very much. It currently basically consists of a
% slightly modified copy of |slides.sty| (which then forms
% |slides.cls|) followed by a slightly changed copy of |slitex.tex|.
% Documentation is practically
% non-existing.  Everybody is invited to help changing this!
% \end{quote}
%
% The code is divided into two parts, we first implement the class
% related functions and declarations and then define low level stuff
% that is necessary within every class. By placing such commands into
% a separate file it will be possible to share it with other slide
% classes.
%
% \subsection{The class code}
%
% At this point we input the redefinitions that are necessary for
% \SLiTeX.
% \changes{v2.3g}{1994/05/26}{Use \cs{input} instead of \cs{@@input}}
%    \begin{macrocode}
%<*class>
%%
%% This file will generate fast loadable files and documentation
%% driver files from the doc files in this package when run through
%% LaTeX or TeX.
%%
%% Copyright (C) 1993-2022
%% The LaTeX Project and any individual authors listed elsewhere
%% in this file.
%%
%% This file is part of the LaTeX base system.
%% -------------------------------------------
%%
%% It may be distributed and/or modified under the
%% conditions of the LaTeX Project Public License, either version 1.3c
%% of this license or (at your option) any later version.
%% The latest version of this license is in
%%    https://www.latex-project.org/lppl.txt
%% and version 1.3c or later is part of all distributions of LaTeX
%% version 2008 or later.
%%
%% This file has the LPPL maintenance status "maintained".
%%
%% As this file contains legal notices, it is NOT PERMITTED to modify
%% this file in any way that the legal information placed into
%% generated files is changed (i.e., the files generated when the
%% original file is executed). This restriction does not apply if
%% (parts of) the content is reused in a different WORK producing its
%% own generated files.
%%
%% The list of all files belonging to the LaTeX base distribution is
%% given in the file `manifest.txt'. See also `legal.txt' for additional
%% information.
%%
%%
%%
%%
%% --------------- start of docstrip commands ------------------
%%

\input docstrip

\preamble

This is a generated file.

The source is maintained by the LaTeX Project team and bug
reports for it can be opened at https://latex-project.org/bugs.html
(but please observe conditions on bug reports sent to that address!)


Copyright (C) 1993-2022
The LaTeX Project and any individual authors listed elsewhere
in this file.

This file was generated from file(s) of the LaTeX base system.
--------------------------------------------------------------

It may be distributed and/or modified under the
conditions of the LaTeX Project Public License, either version 1.3c
of this license or (at your option) any later version.
The latest version of this license is in
   https://www.latex-project.org/lppl.txt
and version 1.3c or later is part of all distributions of LaTeX
version 2008 or later.

This file has the LPPL maintenance status "maintained".

This file may only be distributed together with a copy of the LaTeX
base system. You may however distribute the LaTeX base system without
such generated files.

The list of all files belonging to the LaTeX base distribution is
given in the file `manifest.txt'. See also `legal.txt' for additional
information.

The list of derived (unpacked) files belonging to the distribution
and covered by LPPL is defined by the unpacking scripts (with
extension .ins) which are part of the distribution.
\endpreamble

\keepsilent
\usedir{tex/latex/base}

\Msg{*** Generating slide cmd files and styles ***}

\generate{
   \file{slides.cls}{
        \from{slides.dtx}{class}}
   \file{slides.def}{
        \from{slides.dtx}{cmd}}
   }

\Msg{*** Generating default fonts setup file for slides ***}

\preamble

This is a generated file.

Copyright (C) 1993-2022
The LaTeX Project and any individual authors listed elsewhere
in this file.

This file was generated from file(s) of the LaTeX base system.
--------------------------------------------------------------

It may be distributed and/or modified under the
conditions of the LaTeX Project Public License, either version 1.3c
of this license or (at your option) any later version.
The latest version of this license is in
   https://www.latex-project.org/lppl.txt
and version 1.3c or later is part of all distributions of LaTeX
version 2008 or later.

This file may only be distributed together with a copy of the LaTeX
base system. You may however distribute the LaTeX base system without
such generated files.

The list of all files belonging to the LaTeX base distribution is
given in the file `manifest.txt'. See also `legal.txt' for additional
information.

Details of how to use a configuration file to modify this part of
the system are in the document `cfgguide.tex'.


\endpreamble



\Msg{*** Generating .fd files ***}

\declarepreamble\fdpreamble

This is a generated file.

The source is maintained by the LaTeX Project team and bug
reports for it can be opened at https://latex-project.org/bugs.html
(but please observe conditions on bug reports sent to that address!)


Copyright (C) 1993-2022
The LaTeX Project and any individual authors listed elsewhere
in this file.

This file was generated from file(s) of the LaTeX base system.
--------------------------------------------------------------

It may be distributed and/or modified under the
conditions of the LaTeX Project Public License, either version 1.3c
of this license or (at your option) any later version.
The latest version of this license is in
   https://www.latex-project.org/lppl.txt
and version 1.3c or later is part of all distributions of LaTeX
version 2008 or later.

This file may only be distributed together with a copy of the LaTeX
base system. You may however distribute the LaTeX base system without
such generated files.

The list of all files belonging to the LaTeX base distribution is
given in the file `manifest.txt'. See also `legal.txt' for additional
information.

In particular, permission is granted to customize the declarations in
this file to serve the needs of your installation.

However, a modified version of this file under its original name
should not be distributed as part of a standard LaTeX distribution, as
the modification will be nontransparent for the user of that
distribution (and thus violating clause 6a of the LPPL license),
making successful document exchange impossible.

\endpreamble

\usedir{tex/latex/base}

\generate{
   \file{sfonts.def}{
        \from{slifonts.fdd}{main}}
   \usepreamble\fdpreamble  % change of preamble !
   \file{ot1lcmss.fd}{
        \from{slifonts.fdd}{lcmss,fd}}
   \file{ot1lcmtt.fd}{
        \from{slifonts.fdd}{lcmtt,fd}}
   \file{omllcmm.fd}{
        \from{slifonts.fdd}{lcmm,fd}}
   \file{omslcmsy.fd}{
        \from{slifonts.fdd}{lcmsy,fd}}
   \file{omxlcmex.fd}{
        \from{slifonts.fdd}{lcmex,fd}}
   \file{ullasy.fd}{
        \from{slifonts.fdd}{llasy,fd}}
   }


\ifToplevel{
\Msg{***********************************************************}
\Msg{*}
\Msg{* To finish the installation you have to move the following}
\Msg{* files into a directory searched by TeX:}
\Msg{*}
\Msg{* \space\space All *.fd}
\Msg{* \space\space All *.def}
\Msg{* \space\space slides.cls}
\Msg{*}
\Msg{* To produce the documentation run the files ending with}
\Msg{* `.drv' through LaTeX.}
\Msg{*}
\Msg{* Happy TeXing}
\Msg{***********************************************************}
}

\endbatchfile

%    \end{macrocode}
%
%
% Now we are ready for setting up the font tables. As usual, we first
% look for a local configuration file |sfonts.cfg|. If there isn't
% one, we fall back to the default one (|sfonts.def|).
% \changes{v2.2i}{1994/03/08}{Corrected first argument of
%    \cs{IfFileExists}: \file{sfonts.def} to \file{sfonts.cfg}.}
% \changes{v2.3g}{1994/05/26}{Use \cs{InputIfFileExists} instead of
%    \cs{IfFileExists} and \cs{input} instead of \cs{@@input}}
%    \begin{macrocode}
\InputIfFileExists{sfonts.cfg}
           {\typeout{**************************************^^J%
                     *^^J%
                     * Local config file sfonts.cfg used^^J%
                     *^^J%
                     **************************************}}%
           {\input{sfonts.def}}
%    \end{macrocode}
%
% \section{Declaration of Options}
%
% We declare a few options as illegal.
%
%
% \subsection{Setting Paper Sizes}
%
%    The variables |\paperwidth| and |\paperheight| should reflect the
%    physical paper size after trimming. For desk printer output this
%    is usually the real paper size since there is no post-processing.
%    Classes for real book production will probably add other paper
%    sizes and additionally the production of crop marks for trimming.
% \changes{v2.1d}{1993/12/14}{Corrected typo, A4 is not 279 mm high}
%    \begin{macrocode}
\DeclareOption{a4paper}
   {\setlength\paperheight {297mm}%
    \setlength\paperwidth  {210mm}}
\DeclareOption{a5paper}
   {\setlength\paperheight {210mm}%
    \setlength\paperwidth  {148mm}}
\DeclareOption{b5paper}
   {\setlength\paperheight {250mm}%
    \setlength\paperwidth  {176mm}}
\DeclareOption{letterpaper}
   {\setlength\paperheight {11in}%
    \setlength\paperwidth  {8.5in}}
\DeclareOption{legalpaper}
   {\setlength\paperheight {14in}%
    \setlength\paperwidth  {8.5in}}
\DeclareOption{executivepaper}
   {\setlength\paperheight {10.5in}%
    \setlength\paperwidth  {7.25in}}
%    \end{macrocode}
%
%    The option \Lopt{landscape} switches the values of |\paperheight|
%    and |\paperwidth|, assuming the dimensions were given for portrait
%    paper.
%    \begin{macrocode}
\DeclareOption{landscape}
   {\setlength\@tempdima   {\paperheight}%
    \setlength\paperheight {\paperwidth}%
    \setlength\paperwidth  {\@tempdima}}
%    \end{macrocode}
%
%  \subsection{The clock option}

% The option \Lopt{clock} prints the time at the bottom of each note.
% We also define here the commands and counters used to keep track of
% time.
%    \begin{macrocode}
\newif\if@clock \@clockfalse
\DeclareOption{clock}{\@clocktrue
  \AtEndDocument{\typeout{\@arabic\c@minutes\space minutes}}
}%
\newcounter{minutes}%
\newcounter{seconds}%
\newcommand*{\settime}[1]{\setcounter{seconds}{0}\addtime{#1}}%
\newcommand*{\addtime}[1]{\addtocounter{seconds}{#1}%
  \setcounter{minutes}{\value{seconds}}%
  \global \divide \value{minutes} by 60\relax}

%    \end{macrocode}
%
%  \subsection{Two-side or one-side printing}
%
% Two-sided printing is not allowed, so don't declare an option.
% But it is necessary to initialize the switch.
% \changes{v2.2h}{1994/03/07}{Removed declared option twoside.}
%    \begin{macrocode}
\@twosidefalse
%    \end{macrocode}
%
%
%  \subsection{Draft option}
%
%    If the user requests \Lopt{draft} we show any overfull boxes.
%    We could probably add some more interesting stuff to this option.
%    \begin{macrocode}
\DeclareOption{draft}{\setlength\overfullrule{5pt}}
\DeclareOption{final}{\setlength\overfullrule{0pt}}
%    \end{macrocode}
%
%  \subsection{Titlepage option}
%    The default is for a |\maketitle| command to make a new page.
%    \begin{macrocode}
\newif\if@titlepage
\@titlepagetrue
\DeclareOption{titlepage}{\@titlepagetrue}
\DeclareOption{notitlepage}{\@titlepagefalse}
%    \end{macrocode}
%
%  \subsection{Twocolumn printing}
%
%    Two-column printing is again forbidden.
% \changes{v2.3e}{1994/05/22}{twocolumn produces only a warning}
%    \begin{macrocode}
\DeclareOption{onecolumn}{}
\DeclareOption{twocolumn}{%
    \ClassWarning{slides}{No 'twocolumn' layout for slides}}
%    \end{macrocode}
%
%  \subsection{Equation numbering on the left}
%
%    The option \Lopt{leqno} can be used to get the equation numbers
%    on the left side of the equation.
% \changes{v2.2g}{1994/03/01}{Option leqno loads \file{leqno.clo}
%    file.}
%    \begin{macrocode}
\DeclareOption{leqno}{\input{leqno.clo}}
%    \end{macrocode}
%
%  \subsection{Flush left displays}
%
%    The option \Lopt{fleqn} redefines the displayed math environments
%    in such a way that they come out flush left, with an indentation
%    of |\mathindent| from the prevailing left margin.
% \changes{v2.2g}{1994/03/01}{Added fleqn option.}
%    \begin{macrocode}
\DeclareOption{fleqn}{\input{fleqn.clo}}
%    \end{macrocode}
%
%
% \section{Executing Options}
%
%    Here we execute the default options to initialize certain
%    variables.
%    \begin{macrocode}
\ExecuteOptions{letterpaper,final}
%    \end{macrocode}
%
%    The |\ProcessOptions| command causes the execution of the code
%    for every option \Lopt{FOO}
%    which is declared and for which the user typed
%    the \Lopt{FOO} option in his
%    |\documentclass| command.  For every option \Lopt{BAR} he typed,
%    which is not declared, the option is assumed to be a global option.
%    All options will be passed as document options to any
%    |\usepackage| command in the document preamble.
%    \begin{macrocode}
\ProcessOptions
%    \end{macrocode}
%
%  \section{Loading Packages}
%
%  The standard class files do not load additional packages.
%
%
% \section{Document Layout}
%
%
%  In this section we are finally dealing with the nasty typographical
%  details.
%
%
%
% \subsection{Fonts}
%
%    \begin{macrocode}
% FMi:
\def\rmdefault{lcmss}        % no roman
\def\sfdefault{lcmss}
\def\ttdefault{lcmtt}
\def\itdefault{sl}
\def\sldefault{sl}
\def\bfdefault{bx}
%    \end{macrocode}
%    As |\fontshape| gets redefined we need to make sure that the default
%    for |\upshape| is no longer \texttt{up} but again \texttt{n}.
% \changes{v2.4b}{2019/11/04}{Explicitly set \cs{updefault}}
%    \begin{macrocode}
\def\updefault{n}
%    \end{macrocode}
%
%
%  Since the number of parameters to set are very large it seems
%    reasonable to set up one command |\@setfontsize@parms| which will
%    do the work for us.
%
%    \LaTeX\ offers the user commands to change the size of the font,
%    relative to the `main' size. Each relative size changing command
%    |\size| executes the command
%    |\@setfontsize||\size|\meta{font-size}\meta{baselineskip} where:
%
%    \begin{description}
%    \item[\meta{font-size}] The absolute size of the font to use from
%        now on.
%
%    \item[\meta{baselineskip}] The normal value of |\baselineskip|
%        for the size of the font selected. (The actual value will be
%        |\baselinestretch| * \meta{baselineskip}.)
%    \end{description}
%
%    A number of commands, defined in the \LaTeX{} kernel, shorten the
%    following  definitions and are used throughout. They are:
% \begin{center}
% \begin{tabular}{ll@{\qquad}ll@{\qquad}ll}
%  \verb=\@vpt= & 5 & \verb=\@vipt= & 6 & \verb=\@viipt= & 7 \\
%  \verb=\@viiipt= & 8 & \verb=\@ixpt= & 9 & \verb=\@xpt= & 10 \\
%  \verb=\@xipt= & 10.95 & \verb=\@xiipt= & 12 & \verb=\@xivpt= & 14.4\\
%  ...
%  \end{tabular}
%  \end{center}
%
% \begin{macro}{\ifourteenpt}
% \begin{macro}{\iseventeenpt}
% \begin{macro}{\itwentypt}
% \begin{macro}{\itwentyfourpt}
% \begin{macro}{\itwentyninept}
% \begin{macro}{\ithirtyfourpt}
% \begin{macro}{\ifortyonept}
%    For \SLiTeX{}, however, these are not sufficient, and we therefore
%    need to add a few extra, larger, sizes.
%    \begin{macrocode}
\def\ifourteenpt{13.82}
\def\iseventeenpt{16.59}
\def\itwentypt{19.907}
\def\itwentyfourpt{23.89}
\def\itwentyninept{28.66}
\def\ithirtyfourpt{34.4}
\def\ifortyonept{41.28}
%    \end{macrocode}
% \end{macro}
% \end{macro}
% \end{macro}
% \end{macro}
% \end{macro}
% \end{macro}
% \end{macro}
%
% \begin{macro}{\@setfontsize@parms}
%    This routine is used in \SliTeX{} to interface font size setting
%    it is modeled after the settings I found in \texttt{slides.sty}, so
%    it probably needs an update. But any class is free to redefine
%    it, as it is used only as an abbreviation.
%    It's syntax is:
% \begin{quote}
%     |\@setfontsize@parms| \\
%     |   |\meta{lineskip} \\
%     |   |\meta{parskip} \\
%     |   |\meta{abovedisplayskip}  \\
%     |   |\meta{belowdisplayskip}  \\
%     |   |\meta{abovedisplayshortskip} \\
%     |   |\meta{belowdisplayshortskip} \\
%     |   |\meta{strut ht} \meta{strut dp}   (without pt)
% \end{quote}
%
% For NFSS1 a similar style existed which did run both with a
% \SliTeX{} with old font selection or with NFSS1. But when no
% separate format is made this doesn't make much sense.
% So the following note is history and would only be true if all NFSS
% stuff would be removed from the file and placed into the format.
% \begin{quote}\small
% Note: To interface the old \texttt{sfonts.tex} the \meta{size} must be
%    hidden in commands denoting the size by its name prefixed with
%    `i', i.e.\ 20pt size is called |\itwentypt| at this point. The
%    NFSS interface will define those sizes to expand to the internal
%    size, e.g.\ 20 but for the old sfonts the command name, e.g.
%    |\itwentypt|, will be used to construct the name |\twentypt| etc.
%
%    This is a crude interface to the old \texttt{sfonts.tex}. It will
%    be a bit slower than the old one because it must define |\@tiny|
%    etc.\ every time a size changes.
% \end{quote}
%
% If classes are set up that are only for use with NFSS then the second
% argument may be an ordinary font size!
% \changes{v2.0d}{1993/11/12}{Replaced all pt by \cs{p@}, corrected
%    definition for \cs{tiny}.}
%
%    \begin{macrocode}
\def\@setfontsize@parms#1#2#3#4#5#6#7#8{%
   \lineskip #1\relax%
   \parskip #2\relax
   \abovedisplayskip #3\relax
   \belowdisplayskip #4\relax
   \abovedisplayshortskip #5\relax
   \belowdisplayshortskip #6\relax
%
%    \end{macrocode}
%    I don't see a reason why the |\strutbox| has a dim different from
%    |\baselineskip| but we will leave it for the moment
%    \begin{macrocode}
  \setbox\strutbox=\hbox{\vrule \@height#7\p@\@depth#8\p@\@width\z@}%
  \baselineskip\baselinestretch\baselineskip
  \normalbaselineskip\baselineskip}
%    \end{macrocode}
%  \end{macro}
%
%    Setting size relations for math scripts:
% \changes{v2.2e}{1994/02/07}{Corrected entry for size 23.89.}
%    \begin{macrocode}
\DeclareMathSizes{13.82}{13.82}{10}{7}
\DeclareMathSizes{16.59}{16.59}{12}{7}
\DeclareMathSizes{19.907}{19.907}{16.59}{13.82}
\DeclareMathSizes{23.89}{23.89}{19.907}{16.59}
\DeclareMathSizes{28.66}{28.66}{23.89}{19.907}
\DeclareMathSizes{34.4}{34.4}{28.66}{23.89}
\DeclareMathSizes{41.28}{41.28}{34.4}{28.66}
%    \end{macrocode}
%
% \begin{macro}{\normalsize}
%    \begin{macrocode}
\def\normalsize{%
      \@setfontsize\normalsize\itwentypt{28\p@ plus3\p@ minus4\p@}%
%            {20}{30\p@ plus3\p@ minus3\p@}%  made a bit shorter
      \@setfontsize@parms
            {2pt}%
            {30\p@ plus18\p@ minus9\p@}%
            {15\p@ plus3\p@ minus3\p@}%
            {10\p@ plus3\p@ minus3\p@}%
            {10\p@ plus3\p@}
            \abovedisplayshortskip
            {17}{7}}
%    \end{macrocode}
% \end{macro}
%
%
%    We initially choose the normalsize font.
%    \begin{macrocode}
\normalsize
%    \end{macrocode}
%
% \begin{macro}{\small}
%    \begin{macrocode}
\def\small{\@setfontsize\small\iseventeenpt{19\p@ plus3\p@ minus\p@}%
           \@setfontsize@parms
            {2\p@}%
            {15\p@ plus15\p@ minus7\p@}%
            {12\p@ plus3\p@ minus3\p@}%
            {9\p@ plus3\p@ minus3\p@}%
            {6\p@ plus3\p@}%
            \abovedisplayshortskip
            {13.5}{5.6}}
%    \end{macrocode}
% \end{macro}
%
% \begin{macro}{\footnotesize}
% \begin{macro}{\scriptsize}
%    \begin{macrocode}
\let\footnotesize=\small
\let\scriptsize=\small
%    \end{macrocode}
% \end{macro}
% \end{macro}
%
% \begin{macro}{\tiny}
%    \begin{macrocode}
\def\tiny{\@setfontsize\tiny\ifourteenpt{16\p@ plus2\p@ minus\p@}%
          \@setfontsize@parms
            {2pt}%
            {14\p@ plus3\p@ minus10\p@}%
            {11\p@ plus3\p@ minus10\p@}%
            \abovedisplayskip
            {8\p@ plus3\p@ minus5\p@}%
            {\z@ plus3\p@}%
            {10}{4}}
%    \end{macrocode}
% \end{macro}
%
% Actually copying the code above would be better because this would
% correct the error message. Maybe one should remove the first
% argument of |\set@font@size@parms|.
%
% \begin{macro}{\large}
% \begin{macro}{\Large}
% \begin{macro}{\LARGE}
% \begin{macro}{\huge}
% \begin{macro}{\Huge}
%    \begin{macrocode}
\def\large{\@setfontsize\large\itwentyfourpt{42\p@ plus8\p@ minus5\p@}%
           \@setfontsize@parms
            {2\p@}%
            {40\p@ plus20\p@ minus4\p@}%
            {20\p@ plus8\p@ minus3\p@}%
            \abovedisplayskip
            {10\p@ plus5\p@}%
            \abovedisplayshortskip
            {20}{8.5}}

\def\Large{\@setfontsize\Large\itwentyninept{48\p@ plus10\p@ minus6\p@}%
           \@setfontsize@parms
            {2\p@}%
            {48\p@ plus30\p@ minus6\p@}%
            {24\p@ plus10\p@ minus6\p@}%
            \abovedisplayskip
            {12\p@ plus8\p@}%
            \abovedisplayshortskip
            {27}{11}}

\def\LARGE{\@setfontsize\LARGE\ithirtyfourpt{52\p@ plus10\p@ minus6\p@}%
           \@setfontsize@parms
            {2\p@}%
            {52\p@ plus30\p@ minus6\p@}%
            {24\p@ plus10\p@ minus6\p@}%
            \abovedisplayskip
            {12\p@ plus8\p@}%
            \abovedisplayshortskip
            {27}{11}}

\def\huge{\@setfontsize\huge\ifortyonept{60\p@ plus10\p@ minus6\p@}%
           \@setfontsize@parms
            {2\p@}%
            {60\p@ plus30\p@ minus6\p@}%
            {24\p@ plus10\p@ minus6\p@}%
            \abovedisplayskip
            {12\p@ plus8\p@}%
            \abovedisplayshortskip
            {27}{11}}

\let\Huge\huge
%    \end{macrocode}
% \end{macro}
% \end{macro}
% \end{macro}
% \end{macro}
% \end{macro}
%
% \subsection{Paragraphing}
%
% \begin{macro}{\baselinestretch}
%    This is used as a multiplier for |\baselineskip|. The default is
%    to {\em not\/} stretch the baselines.
%    \begin{macrocode}
\renewcommand\baselinestretch{}
%    \end{macrocode}
% \end{macro}
%
% \begin{macro}{\parindent}
%    |\parindent| is the width of the paragraph indentation.
%    \begin{macrocode}
\setlength\parindent{\z@}
%    \end{macrocode}
% \end{macro}
%
% \begin{macro}{\@lowpenalty}
% \begin{macro}{\@medpenalty}
% \begin{macro}{\@highpenalty}%
%    The commands |\nopagebreak| and |\nolinebreak| put in penalties
%    to discourage these breaks at the point they are put in.
%    They use |\@lowpenalty|, |\@medpenalty| or |\@highpenalty|,
%    dependent on their argument.
%    \begin{macrocode}
\@lowpenalty   51
\@medpenalty  151
\@highpenalty 301
%    \end{macrocode}
% \end{macro}
% \end{macro}
% \end{macro}
%
% \begin{macro}{\clubpenalty}
% \begin{macro}{\widowpenalty}
%    These penalties are use to discourage club and widow lines.
%    Because we use their default values we only show them here,
%    commented out.
%    \begin{macrocode}
% \clubpenalty  150
% \widowpenalty 150
%    \end{macrocode}
% \end{macro}
% \end{macro}
%
% \begin{macro}{\displaywidowpenalty}
% \begin{macro}{\predisplaypenalty}
% \begin{macro}{\postdisplaypenalty}
%    Discourage (but not so much) widows in front of a math display
%    and forbid breaking directly in front of a display. Allow break
%    after a display without a penalty. Again the default values are
%    used, therefore we only show them here.
%    \begin{macrocode}
% \displaywidowpenalty 50
% \predisplaypenalty   10000
% \postdisplaypenalty  0
%    \end{macrocode}
% \end{macro}
% \end{macro}
% \end{macro}
%
% \begin{macro}{\interlinepenalty}
%    Allow the breaking of a page in the middle of a paragraph.
%    \begin{macrocode}
% \interlinepenalty 0
%    \end{macrocode}
% \end{macro}
%
%
% \begin{macro}{\brokenpenalty}
%    We allow the breaking of a page after a hyphenated line.
%    \begin{macrocode}
% \brokenpenalty 0
%    \end{macrocode}
% \end{macro}
%
%
% \subsection{Page Layout}
%
%    All margin dimensions are measured from a point one inch from the
%    top and lefthand side of the page.
%
% \subsubsection{Vertical spacing}
%
% \begin{macro}{\headheight}
% \begin{macro}{\headsep}
% \begin{macro}{\topskip}
%    The |\headheight| is the height of the box that will contain the
%    running head. The |\headsep| is the distance between the bottom
%    of the running head and the top of the text. |\topskip| is the
%    |\baselineskip| for the first line on a page.
%    \begin{macrocode}
\setlength\headheight{14\p@}
\setlength\headsep   {15\p@}
\setlength\topskip   {30\p@}
%    \end{macrocode}
% \end{macro}
% \end{macro}
% \end{macro}
%
% \begin{macro}{\footskip}
%    The distance from the baseline of the box which contains the
%    running footer to the baseline of last line of text is controlled
%    by the |\footskip|.
%    Bottom of page:
%    \begin{macrocode}
\setlength\footskip{25\p@}   %
%    \end{macrocode}
% \end{macro}
%
% \begin{macro}{\maxdepth}
% \begin{macro}{\@maxdepth}
% \changes{v2.3c}{1994/05/06}{Added setting of \cs{maxdepth} and
%    \cs{@maxdepth}}
%    The \TeX\ primitive register |\maxdepth| has a function that is
%    similar to that of |\topskip|. The register |\@maxdepth| should
%    always contain a copy of |\maxdepth|. In both plain \TeX\ and
%    \LaTeX~2.09 |\maxdepth| had a fixed value of \texttt{4pt}; in
%    native \LaTeX2e\ mode we let the value depend on the typesize. We
%    set it so that |\maxdepth| $+$ |\topskip| $=$ typesize $\times
%    1.5$. As it happens, in these classes |\topskip| is equal to the
%    typesize, therefor we set |\maxdepth| to half the value of
%    |\topskip|.
%    \begin{macrocode}
\if@compatibility
  \setlength\maxdepth{4\p@}
\else
  \setlength\maxdepth{.5\topskip}
\fi
\setlength\@maxdepth\maxdepth
%    \end{macrocode}
% \end{macro}
% \end{macro}
%
% \subsubsection{The dimension of text}
%
% \begin{macro}{\textwidth}
%    When we are in compatibility mode we have to make sure that the
%    dimensions of the printed area are not different from what the
%    user was used to see.
%
%    \begin{macrocode}
\if@compatibility
  \setlength\textwidth{460\p@}
%    \end{macrocode}
%    When we are not in compatibility mode we can set some of the
%    dimensions differently, taking into account the paper size for
%    instance.
%    \begin{macrocode}
\else
%    \end{macrocode}
%    First, we calculate the maximum textwidth, which depends on the
%    papersize. Then we calculate the approximate length of 65
%    characters, which should be the maximum length of a line of text.
%    The calculated values are stored in |\@tempdima| and |\@tempdimb|.
%    \begin{macrocode}
  \setlength\@tempdima{\paperwidth}
  \addtolength\@tempdima{-2in}
  \setbox\@tempboxa\hbox{\rmfamily im}
  \setlength\@tempdimb{.5\wd\@tempboxa}
  \setlength\@tempdimb{65\@tempdimb}
%    \end{macrocode}
%
%    Now we can set the |\textwidth|, depending on whether we will be
%    setting one or two columns.
%
%    The text should not be wider than the minimum
%    of the paperwidth (minus 2 inches for the margins) and the
%    maximum length of a line as defined by the number of characters.
%    \begin{macrocode}
  \ifdim\@tempdima>\@tempdimb\relax
    \setlength\textwidth{\@tempdimb}
  \else
    \setlength\textwidth{\@tempdima}
  \fi
\fi
%    \end{macrocode}
%
%    Here we modify the width of the text a little to be a whole
%    number of points.
%    \begin{macrocode}
\@settopoint\textwidth
%    \end{macrocode}
% \end{macro}
%
% \begin{macro}{\columnwidth}
% \begin{macro}{\columnsep}
% \begin{macro}{\columnseprule}
%    \begin{macrocode}
\columnwidth \textwidth
\columnsep 10pt
\columnseprule \z@
%    \end{macrocode}
% \end{macro}
% \end{macro}
% \end{macro}
% \begin{macro}{\textheight}
%    Now that we have computed the width of the text, we have to take
%    care of the height. The |\textheight| is the height of text
%    (including footnotes and figures, excluding running head and
%    foot).
%
%    First make sure that the compatibility mode gets the same
%    dimensions as we had with \LaTeX2.09. The number of lines was
%    calculated as the floor of the old |\textheight| minus
%    |\topskip|, divided by |\baselineskip| for |\normalsize|. The
%    old value of |\textheight| was 528pt.
%
%    \begin{macrocode}
\if@compatibility
  \setlength\textheight{600\p@}
%    \end{macrocode}
%
%    Again we compute this, depending on the papersize and depending
%    on the baselineskip that is used, in order to have a whole number
%    of lines on the page.
%    \begin{macrocode}
\else
  \setlength\@tempdima{\paperheight}
%    \end{macrocode}
%
%    We leave at least a 1 inch margin on the top and the bottom of
%    the page.
%    \begin{macrocode}
  \addtolength\@tempdima{-2in}
%    \end{macrocode}
%
%    We also have to leave room for the running headers and footers.
%    \begin{macrocode}
  \addtolength\@tempdima{-1in}
%    \end{macrocode}
%
%    Then we divide the result by the current |\baselineskip| and
%    store this in the count register |\@tempcnta|, which then
%    contains the number of lines that fit on this page.
%    \begin{macrocode}
  \divide\@tempdima\baselineskip
  \@tempcnta=\@tempdima
%    \end{macrocode}
%
%    From this we can calculate the height of the text.
%    \begin{macrocode}
  \setlength\textheight{\@tempcnta\baselineskip}
\fi
%    \end{macrocode}
%
%    The first line on the page has a height of |\topskip|.
%    \begin{macrocode}
\advance\textheight by \topskip
%    \end{macrocode}
% \end{macro}
%
% \subsubsection{Margins}
%
% \begin{macro}{\oddsidemargin}
% \begin{macro}{\evensidemargin}
% \begin{macro}{\marginparwidth}
%    First we give the values for the compatibility mode.
%
%    Values for two-sided printing:
%    \begin{macrocode}
\if@compatibility
   \setlength\oddsidemargin  {17\p@}
   \setlength\evensidemargin {17\p@}
   \setlength\marginparwidth {20\p@}
\else
%    \end{macrocode}
%
%    When we are not in compatibility mode we can take the dimensions
%    of the selected paper into account.
%
%    We center the text on the page, by
%    calculating the difference between |textwidth| and
%    |\paperwidth|$-$|2in|. Half of that difference is then used for
%    the margin. The amount of space that can be used for marginal
%    notes is at least 0.8~inch, to which we add any `leftover' space.
%    \begin{macrocode}
  \setlength\@tempdima        {\paperwidth}
  \addtolength\@tempdima      {-2in}
  \addtolength\@tempdima      {-\textwidth}
  \setlength\oddsidemargin    {.5\@tempdima}
  \setlength\marginparwidth   {.8in}
  \addtolength\marginparwidth {.5\@tempdima}
%    \end{macrocode}
%
%    The |\evensidemargin| can now be computed from the values set
%    above.
%    \begin{macrocode}
\setlength\evensidemargin  {\paperwidth}
\addtolength\evensidemargin{-2in}
\addtolength\evensidemargin{-\textwidth}
\addtolength\evensidemargin{-\oddsidemargin}
\fi
%    \end{macrocode}
% \end{macro}
% \end{macro}
% \end{macro}
%
% \begin{macro}{\marginparsep}
% \begin{macro}{\marginparpush}
%    The horizontal space between the main text and marginal notes is
%    determined by |\marginparsep|, the minimum vertical separation
%    between two marginal notes is controlled by |\marginparpush|.
%    \begin{macrocode}
\setlength\marginparsep {5\p@}
\setlength\marginparpush{5\p@}
%    \end{macrocode}
% \end{macro}
% \end{macro}
%
% \begin{macro}{\topmargin}
%    The |\topmargin| is the distance between the top of `the
%    printable area' --which is 1 inch below the top of the paper--
%    and the top of the box which contains the running head.
%
%    It can now be computed from the values set above.
%    \begin{macrocode}
\if@compatibility
  \setlength\topmargin{-10pt}
\else
  \setlength\topmargin{\paperheight}
  \addtolength\topmargin{-2in}
  \addtolength\topmargin{-\headheight}
  \addtolength\topmargin{-\headsep}
  \addtolength\topmargin{-\textheight}
  \addtolength\topmargin{-\footskip}     % this might be wrong!
%    \end{macrocode}
%    By changing the factor in the next line the complete page
%    can be shifted vertically.
%    \begin{macrocode}
  \addtolength\topmargin{-.5\topmargin}
\fi
\@settopoint\topmargin
%    \end{macrocode}
% \end{macro}
%
%
% \subsubsection{Footnotes}
%
% \begin{macro}{\footnotesep}
%    |\footnotesep| is the height of the strut placed at the beginning
%    of every footnote. It equals the  height of a normal
%    |\footnotesize| strut in this
%    class, thus no extra space occurs between footnotes.
%    \begin{macrocode}
\setlength\footnotesep{20\p@}
%    \end{macrocode}
% \end{macro}
%
% \begin{macro}{\footins}
%    |\skip\footins| is the space between the last line of the main
%    text and the top of the first footnote.
%    \begin{macrocode}
\setlength{\skip\footins}{10\p@ \@plus 2\p@ \@minus 4\p@}
%    \end{macrocode}
% \end{macro}
%
%
% \subsection{Page Styles}
%
%    The page style \pstyle{foo} is defined by defining the command
%    |\ps@foo|.   This command should make only local definitions.
%    There should be no stray spaces in the definition, since they
%    could lead to mysterious extra spaces in the output (well, that's
%    something that should be always avoided).
%
% \begin{macro}{\@evenhead}
% \begin{macro}{\@oddhead}
% \begin{macro}{\@evenfoot}
% \begin{macro}{\@oddfoot}
%    The |\ps@...| command defines the macros |\@oddhead|,
%    |\@oddfoot|, |\@evenhead|, and |\@evenfoot| to define the running
%    heads and feet---e.g., |\@oddhead| is the macro to produce the
%    contents of the heading box for odd-numbered pages.  It is called
%    inside an |\hbox| of width |\textwidth|.
% \end{macro}
% \end{macro}
% \end{macro}
% \end{macro}
%
%
%  The page styles of slides is determined by the 'slide' page style,
% the slide environment executing a |\thispagestyle{slide}| command.
% The page styles of overlays and notes are similarly determined by
% 'overlay' and 'note' page styles.  The command standard 'headings',
% 'plain' and 'empty' page styles work by redefining the 'slide',
% 'overlay', and 'note' styles.
%
% \changes{v2.3q}{1995/09/20}{Wrap some long lines, and use \cs{null}.}
% \begin{macro}{\ps@headings}
%    \begin{macrocode}
\if@compatibility
\def\ps@headings{%
\def\ps@slide{\def\@oddfoot{\@mainsize +\hfil\hb@xt@3em{\theslide
                                                          \hss}}%
 \def\@oddhead{\@mainsize +\hfil +}%
 \def\@evenfoot{\@mainsize +\hfil\hb@xt@3em{\theslide\hss}}%
 \def\@evenhead{\@mainsize +\hfil +}}

\def\ps@overlay{\def\@oddfoot{\@mainsize +\hfil\hb@xt@3em{\theoverlay
                                                           \hss}}%
 \def\@oddhead{\@mainsize +\hfil +}%
 \def\@evenfoot{\@mainsize +\hfil\hb@xt@3em{\theoverlay\hss}}%
 \def\@evenhead{\@mainsize +\hfil +}}
\def\ps@note{\def\@oddfoot{\@mainsize \hbox{}\hfil\thenote}%
 \def\@oddhead{}%
 \def\@evenfoot{\@mainsize \hbox{}\hfil\thenote}%
 \def\@evenhead{}}}
%
\else %%if@compatibility
%
\def\ps@headings{%
  \def\ps@slide{%
    \def\@oddfoot{\@mainsize \mbox{}\hfil\hb@xt@3em{\theslide\hss}}%
    \def\@oddhead{}%
    \def\@evenfoot{\@mainsize \mbox{}\hfil\hb@xt@3em{\theslide\hss}}%
    \def\@evenhead{}}

  \def\ps@overlay{%
    \def\@oddfoot{\@mainsize \mbox{}\hfil\hb@xt@3em{\theoverlay\hss}}%
    \def\@oddhead{}%
    \def\@evenfoot{\@mainsize \mbox{}\hfil\hb@xt@3em{\theoverlay\hss}}%
    \def\@evenhead{}}

  \def\ps@note{%
    \def\@oddfoot{%
      \@mainsize
      \if@clock
        \fbox{\large \@arabic\c@minutes\space min}%
      \else
        \null
      \fi
      \hfil\thenote}%
    \def\@oddhead{}%
    \def\@evenfoot{%
      \@mainsize
      \if@clock
        \fbox{\large \@arabic\c@minutes\space min}%
      \else
        \null
      \fi
      \hfil\thenote}%
    \def\@evenhead{}}}
\fi  %% if@compatibility
%    \end{macrocode}
% \end{macro}
%
% \begin{macro}{\ps@plain}
%    \begin{macrocode}
\def\ps@plain{\def\ps@slide{%
 \def\@oddfoot{\@mainsize \mbox{}\hfil\hb@xt@3em{\theslide\hss}}%
 \def\@oddhead{}%
 \def\@evenfoot{\@mainsize \mbox{}\hfil\hb@xt@3em{\theslide\hss}}%
 \def\@evenhead{}}
\def\ps@overlay{\def\@oddfoot{\@mainsize
   \mbox{}\hfil\hb@xt@3em{\theoverlay\hss}}%
 \def\@oddhead{}%
 \def\@evenfoot{\@mainsize \mbox{}\hfil\hb@xt@3em{\theoverlay\hss}}%
 \def\@evenhead{}}
\def\ps@note{\def\@oddfoot{\@mainsize \hbox{}\hfil\thenote}%
 \def\@oddhead{}%
 \def\@evenfoot{\@mainsize \hbox{}\hfil\thenote}%
 \def\@evenhead{}}}
%    \end{macrocode}
% \end{macro}
%
% \begin{macro}{\ps@empty}
%    \begin{macrocode}
\def\ps@empty{%
\def\ps@slide{\def\@oddhead{}\def\@oddfoot{}%
\def\@evenhead{}\def\@evenfoot{}}%
\def\ps@overlay{\def\@oddhead{}\def\@oddfoot{}%
\def\@evenhead{}\def\@evenfoot{}}%
\def\ps@note{\def\@oddhead{}\def\@oddfoot{}%
\def\@evenhead{}\def\@evenfoot{}}}
%    \end{macrocode}
% \end{macro}
%
% Default definition the 'slide', 'overlay', and 'note' page styles.
%    \begin{macrocode}
\ps@headings
%    \end{macrocode}
% Set ordinary page style to 'empty'
%    \begin{macrocode}
\let\@oddhead\@empty\let\@oddfoot\@empty
\let\@evenhead\@empty\let\@evenfoot\@empty
%    \end{macrocode}
%
%
% \subsection{Providing math {\em versions}}
%
%  \LaTeX{} provides two {\em versions\/}. We call them
%  \textsf{normal} and \textsf{bold}, respectively.
%  \SliTeX{} does not have a \textsf{bold} version. But we treat the
%  invisible characters as a version. The only thing we have to take
%  care of is to ensure that we have exactly the same fonts in both
%  versions available.
%
%    \begin{macrocode}
\DeclareMathVersion{invisible}
%    \end{macrocode}
%
% Now we define the basic {\em math groups\/} used by \LaTeX{}.  Later
% on, in packages some other {\em math groups}, e.g., the AMS
% symbol fonts, will be defined.
%
%  As a default I used serif fonts for mathgroup 0 to get things like
% \verb+\log+ look right.
%    \begin{macrocode}
\SetSymbolFont{operators}{normal}
                 {OT1}{lcmss}{m}{n}

\SetSymbolFont{letters}{normal}
                 {OML}{lcmm}{m}{it}
\SetSymbolFont{symbols}{normal}
                 {OMS}{lcmsy}{m}{n}
\SetSymbolFont{largesymbols}{normal}
                 {OMX}{lcmex}{m}{n}

\SetSymbolFont{operators}{invisible}
                 {OT1}{lcmss}{m}{In}
\SetSymbolFont{letters}{invisible}
                 {OML}{lcmm}{m}{Iit}
\SetSymbolFont{symbols}{invisible}
                 {OMS}{lcmsy}{m}{In}
\SetSymbolFont{largesymbols}{invisible}
                  {OMX}{lcmex}{m}{In}


\def\@mainsize{\visible\tiny}
%    \end{macrocode}
%
%
%  \subsection{Environments}
%
% \begin{environment}{titlepage}
%    This environment starts a new page, with pagestyle \pstyle{empty}
%    and sets the page counter to 0.
%    \begin{macrocode}
\newenvironment{titlepage}
     {\newpage
      \thispagestyle{empty}%
      \setcounter{page}{\z@}}
     {\newpage}
%    \end{macrocode}
% \end{environment}
%
% \subsubsection{General List Parameters}
%
%    The following commands are used to set the default values for the
%    list environment's parameters. See the \LaTeX{} manual for an
%    explanation of the meaning of the parameters.
%
% \begin{macro}{\leftmargini}
% \begin{macro}{\leftmarginii}
% \begin{macro}{\leftmarginiii}
% \begin{macro}{\leftmarginiv}
% \begin{macro}{\leftmarginv}
% \begin{macro}{\leftmarginvi}
%    \begin{macrocode}
\setlength\leftmargini   {38\p@}
\setlength\leftmarginii  {30\p@}
\setlength\leftmarginiii {20\p@}
\setlength\leftmarginiv  {15\p@}
\setlength\leftmarginv   {15\p@}
\setlength\leftmarginvi  {10\p@}
%    \end{macrocode}
% \end{macro}
% \end{macro}
% \end{macro}
% \end{macro}
% \end{macro}
% \end{macro}
%
% \begin{macro}{\@listi}
% \begin{macro}{\@listii}
% \begin{macro}{\@listiii}
% \begin{macro}{\@listiv}
% \begin{macro}{\@listv}
% \begin{macro}{\@listvi}
%    These commands set the values of |\leftmargin|, |\parsep|,
%    |\topsep|, and |\itemsep| for the various levels of lists.
% \changes{v2.3z}{1997/08/15}{Add initialization of \cs{leftmargin} to
%    \cs@listi.}
%    It is even necessary to initialize |\leftmargin| in |\@listi|,
%    i.e. for a level one list, as a list environment may appear
%    inside a \texttt{trivlist}, for example inside a \texttt{theorem}
%    environment.
%    \begin{macrocode}
\def\@listi{\leftmargin\leftmargini
            \parsep .5\parskip
            \topsep \parsep
            \itemsep\parskip
            \partopsep \z@}

\def\@listii{\leftmargin\leftmarginii
             \labelwidth\leftmarginii
             \advance\labelwidth-\labelsep
             \parsep .5\parskip
             \topsep \parsep
             \itemsep\parskip}
\def\@listiii{\leftmargin\leftmarginiii
              \labelwidth\leftmarginiii
              \advance\labelwidth-\labelsep}
\def\@listiv{\leftmargin\leftmarginiv
             \labelwidth\leftmarginiv
             \advance\labelwidth-\labelsep}
\def\@listv{\leftmargin\leftmarginv
            \labelwidth\leftmarginv
            \advance\labelwidth-\labelsep}
\def\@listvi{\leftmargin\leftmarginvi
             \labelwidth\leftmarginvi
             \advance\labelwidth-\labelsep}
%    \end{macrocode}
% \end{macro}
% \end{macro}
% \end{macro}
% \end{macro}
% \end{macro}
% \end{macro}
%
%    Here we initialize |\leftmargin| and |\labelwidth|.
%    \begin{macrocode}
\leftmargin\leftmargini
\labelwidth\leftmargini\advance\labelwidth-\labelsep
%    \end{macrocode}
%
%
%    \subsubsection{Paragraph-formatting environments}
%
% \begin{environment}{verse}
%    Inside a |verse| environment, |\\| ends a line, and
%    line continuations are indented further.
%    A blank line makes new paragraph with |\parskip| space.
%    \begin{macrocode}
\newenvironment{verse}{\let\\=\@centercr
                       \list{}{\itemsep       \z@
                               \itemindent    -15\p@
                               \listparindent \itemindent
                               \rightmargin   \leftmargin
                               \advance\leftmargin 15\p@}%
                       \item[]}
                      {\endlist}
%    \end{macrocode}
% \end{environment}
%
% \begin{environment}{quotation}
%    The |quotation| environment fills lines, indents paragraphs.
%    \begin{macrocode}
\newenvironment{quotation}{\list{}{\listparindent 20\p@
                                   \itemindent\listparindent
                                   \rightmargin\leftmargin}%
                           \item[]}
                          {\endlist}
%    \end{macrocode}
% \end{environment}
%
% \begin{environment}{quote}
%    The |quote| environment is the same as the |quotation| environment,
%    except that there is no paragraph indentation.
%    \begin{macrocode}
\newenvironment{quote}{\list{}{\rightmargin\leftmargin}\item[]}
                      {\endlist}
%    \end{macrocode}
% \end{environment}
%
%
%    \subsubsection{List-making environments}
%
%
% \begin{environment}{description}
%    The description environment is defined here -- while the itemize
%    and enumerate environments are defined in the \LaTeX{} format.
%
%    \begin{macrocode}
\newenvironment{description}{\list{}{\labelwidth\z@
                                     \itemindent-\leftmargin
                                     \let\makelabel\descriptionlabel}}
                            {\endlist}
%    \end{macrocode}
% \end{environment}
%
% \begin{macro}{\descriptionlabel}
%  To change the formatting of the label, you must redefine
%  |\descriptionlabel|.
%    \begin{macrocode}
\newcommand*{\descriptionlabel}[1]{\hspace\labelsep
                                \normalfont\bfseries #1}

%    \end{macrocode}
% \end{macro}
%
% \subsubsection{Enumerate}
%
%    The enumerate environment uses  four counters: \Lcount{enumi},
%    \Lcount{enumii}, \Lcount{enumiii} and \Lcount{enumiv}, where
%    \Lcount{enumN} controls the numbering of the Nth level
%    enumeration.
%
% \begin{macro}{\theenumi}
% \begin{macro}{\theenumii}
% \begin{macro}{\theenumiii}
% \begin{macro}{\theenumiv}
%    The counters are already defined in the \LaTeX{} format, but their
%    representation is changed here.
%
%    \begin{macrocode}
\renewcommand\theenumi{\@arabic\c@enumi}
\renewcommand\theenumii{\@alph\c@enumii}
\renewcommand\theenumiii{\@roman\c@enumiii}
\renewcommand\theenumiv{\@Alph\c@enumiv}
%    \end{macrocode}
% \end{macro}
% \end{macro}
% \end{macro}
% \end{macro}
%
% \begin{macro}{\labelenumi}
% \begin{macro}{\labelenumii}
% \begin{macro}{\labelenumiii}
% \begin{macro}{\labelenumiv}
%    The label for each item is generated by the four commands
%    |\labelenumi| \ldots\ |\labelenumiv|.
% \changes{v2.3k}{1994/12/12}{Handle the \cs{label...} commands as in
%    the other class files}
%    \begin{macrocode}
\newcommand\labelenumi{\theenumi.}
\newcommand\labelenumii{(\theenumii)}
\newcommand\labelenumiii{\theenumiii.}
\newcommand\labelenumiv{\theenumiv.}
%    \end{macrocode}
% \end{macro}
% \end{macro}
% \end{macro}
% \end{macro}
%
% \begin{macro}{\p@enumii}
% \begin{macro}{\p@enumiii}
% \begin{macro}{\p@enumiv}
%    The expansion of |\p@enumN||\theenumN| defines the output of a
%    |\ref| command when referencing an item of the Nth level of an
%    enumerated list.
%    \begin{macrocode}
\renewcommand\p@enumii{\theenumi}
\renewcommand\p@enumiii{\theenumi(\theenumii)}
\renewcommand\p@enumiv{\p@enumiii\theenumiii}
%    \end{macrocode}
% \end{macro}
% \end{macro}
% \end{macro}
%
% \subsubsection{Itemize}
%
%
% \begin{macro}{\labelitemi}
% \begin{macro}{\labelitemii}
% \changes{v2.3x}{1996/08/25}{replaced -{}- with \cs{textendash}}
% \begin{macro}{\labelitemiii}
% \begin{macro}{\labelitemiv}
%    Itemization is controlled by four commands: |\labelitemi|,
%    |\labelitemii|, |\labelitemiii|, and |\labelitemiv|, which\
%    define the labels of the various itemization levels.
%    \begin{macrocode}
\newcommand\labelitemi{$\m@th\bullet$}
\newcommand\labelitemii{\normalfont\bfseries \textendash}
\newcommand\labelitemiii{$\m@th\ast$}
\newcommand\labelitemiv{$\m@th\cdot$}
%    \end{macrocode}
%    \end{macro}
%    \end{macro}
%    \end{macro}
%    \end{macro}
%
% \subsection{Setting parameters for existing environments}
%
% \subsubsection{Array and tabular}
%
% \begin{macro}{\arraycolsep}
%    The columns in an array environment are separated by
%    2|\arraycolsep|.% Array and tabular environment parameters
%    \begin{macrocode}
\setlength\arraycolsep{8\p@}
%    \end{macrocode}
% \end{macro}
%
% \begin{macro}{\tabcolsep}
%    The columns in an tabular environment are separated by
%    2|\tabcolsep|.
%    \begin{macrocode}
\setlength\tabcolsep{10\p@}
%    \end{macrocode}
% \end{macro}
%
% \begin{macro}{\arrayrulewidth}
%    The width of rules in the array and tabular environments is given
%    by  the length parameter|\arrayrulewidth|.
%    \begin{macrocode}
\setlength\arrayrulewidth{.6\p@}
%    \end{macrocode}
% \end{macro}
%
% \begin{macro}{\doublerulesep}
%    The space between adjacent rules in the array and tabular
%    environments is given by |\doublerulesep|.
%    \begin{macrocode}
\setlength\doublerulesep{3\p@}
%    \end{macrocode}
% \end{macro}
%
% \subsubsection{Tabbing}
%
% \begin{macro}{\tabbingsep}
%    This controls the space that the |\'| command puts in. (See
%    \LaTeX{} manual for an explanation.)
%    \begin{macrocode}
\labelsep 10pt
\setlength\tabbingsep{\labelsep}
%    \end{macrocode}
% \end{macro}
%
% \subsubsection{Minipage}
%
% \begin{macro}{\@minipagerestore}
%    The macro |\@minipagerestore| is called upon entry to a minipage
%    environment to set up things that are to be handled differently
%    inside a minipage environment. In the current styles, it does
%    nothing.
% \end{macro}
%
% \begin{macro}{\@mpfootins}
%    Minipages have their own footnotes; |\skip||\@mpfootins| plays
%    same r\^ole for footnotes in a minipage as |\skip||\footins| does
%    for ordinary footnotes.
%
%    \begin{macrocode}
\skip\@mpfootins = \skip\footins
%    \end{macrocode}
% \end{macro}
%
% \subsubsection{Framed boxes}
%
% \begin{macro}{\fboxsep}
%    The space left by |\fbox| and |\framebox| between the box and the
%    text in it.
% \begin{macro}{\fboxrule}
%    The width of the rules in the box made by |\fbox| and |\framebox|.
%    \begin{macrocode}
\setlength\fboxsep{5\p@}
\setlength\fboxrule{.6\p@}
%    \end{macrocode}
% \end{macro}
% \end{macro}
%
% \begin{macro}{\theequation}
%    The equation number will be typeset as arabic numerals.
%    \begin{macrocode}
\def\theequation{\@arabic\c@equation}
%    \end{macrocode}
% \end{macro}
%
%
% \begin{macro}{\jot}
%    |\jot| is the extra space added between lines of an eqnarray
%    environment. The default value is used.
%    \begin{macrocode}
% \setlength\jot{3pt}
%    \end{macrocode}
% \end{macro}
%
% \begin{macro}{\@eqnnum}
%    The macro |\@eqnnum| defines how equation numbers are to appear in
%    equations. Again the default is used.
%
%    \begin{macrocode}
% \def\@eqnnum{(\theequation)}
%    \end{macrocode}
% \end{macro}
%
%
% \subsection{Font changing}
%
% \changes{v2.3j}{1994/11/10}{fixed a few typos}
%
%    Here we supply the declarative font changing commands that were
%    common in \LaTeX\ version 2.09 and earlier. These commands work
%    in text mode \emph{and} in math mode. They are provided for
%    compatibility, but one should start using the |\text...| and
%    |\math...| commands instead. These commands are redefined using
%    |\DeclareOldFontCommand|, a command with three arguments: the
%    user command to be defined, \LaTeX\ commands to execute in text
%    mode and \LaTeX\ commands to execute in math mode.
%
%  \begin{macro}{\rm}
%  \begin{macro}{\tt}
%  \begin{macro}{\sf}
% \changes{v2.2}{1993/12/18}{Changed \cs{@newfontswitch} to
%    \cs{@renewfontswitch}.}
% \changes{v2.3a}{1994/04/14}{\cs{@renewfontswitch} has become
%    \cs{DeclareOldFontCommand}}
%
%    The commands to change the family. When in compatibility mode we
%    select the `default' font first, to get \LaTeX2.09 behaviour.
%    \begin{macrocode}
\DeclareOldFontCommand{\rm}{\normalfont\rmfamily}{\mathrm}
\DeclareOldFontCommand{\sf}{\normalfont\sffamily}{\mathsf}
\DeclareOldFontCommand{\tt}{\normalfont\ttfamily}{\mathtt}
%    \end{macrocode}
%  \end{macro}
%  \end{macro}
%  \end{macro}
%
%  \begin{macro}{\bf}
%    The command to change to the bold series. One should use
%    |\mdseries| to explicitly switch back to medium series.
% \changes{v2.2}{1993/12/18}{Changed \cs{@newfontswitch} to
%    \cs{@renewfontswitch}.}
%    \begin{macrocode}
\DeclareOldFontCommand{\bf}{\normalfont\bfseries}{\mathbf}
%    \end{macrocode}
%  \end{macro}
%
%  \begin{macro}{\sl}
%  \begin{macro}{\it}
%  \begin{macro}{\sc}
% \changes{v2.2}{1993/12/18}{Changed \cs{@newfontswitch} to
%    \cs{@renewfontswitch}.}
%    And the commands to change the shape of the font. The slanted and
%    small caps shapes are not available by default as math alphabets,
%    so those changes do nothing in math mode. One should use
%    |\upshape| to explicitly change back to the upright shape.
%    \begin{macrocode}
\DeclareOldFontCommand{\it}{\normalfont\itshape}{\mathit}
\DeclareOldFontCommand{\sl}{\normalfont\slshape}{\relax}
\DeclareOldFontCommand{\sc}{\normalfont\scshape}{\relax}
%    \end{macrocode}
%  \end{macro}
%  \end{macro}
%  \end{macro}
%
% \begin{macro}{\cal}
% \changes{v2.1d}{1993/12/14}{Macro added}
% \begin{macro}{\mit}
% \changes{v2.1d}{1993/12/14}{Macro added}
%
%    The commands |\cal| and |\mit| should only be used in math mode,
%    outside math mode they have no effect. Currently the New Font
%    Selection Scheme defines these commands to generate warning
%    messages. Therefore we have to define them `by hand'.
% \changes{v2.3k}{1994/12/12}{Now define \cs{cal} and \cs{mit} using
%    \cs{DeclareRobustCommand*}}
%    \begin{macrocode}
\DeclareRobustCommand*{\cal}{\@fontswitch{\relax}{\mathcal}}
\DeclareRobustCommand*{\mit}{\@fontswitch{\relax}{\mathnormal}}
%    \end{macrocode}
%  \end{macro}
%  \end{macro}
%
% \subsection{Footnotes}
%
%
% \begin{macro}{\footnoterule}
%    Usually, footnotes are separated from the main body of the text
%    by a small rule. This rule is drawn by the macro |\footnoterule|.
%    We have to make sure that the rule takes no vertical space (see
%    \file{plain.tex}).  The resulting rule will appear on all color
%    layers, so it's best not to draw a rule.
%    \begin{macrocode}
\renewcommand\footnoterule{}
% \let \footnoterule = \relax
%    \end{macrocode}
% \end{macro}
%
% \begin{macro}{\c@footnote}
% \begin{macro}{\thefootnote}
%    Footnotes are numbered within slides, overlays, and notes and
%    numbered with $\ast$, $\dagger$, etc.
%    \begin{macrocode}
% \newcounter{footnote}
\def\thefootnote{\fnsymbol{footnote}}
\@addtoreset{footnote}{slide}
\@addtoreset{footnote}{overlay}
\@addtoreset{footnote}{note}
%    \end{macrocode}
% \end{macro}
% \end{macro}
%
%
% \begin{macro}{\@makefntext}
% \changes{v2.2i}{1994/03/08}{Always call \cs{@makefnmark}.}
%    The footnote mechanism of \LaTeX{} calls the macro |\@makefntext|
%    to produce the actual footnote. The macro gets the text of the
%    footnote as its argument and should use |\@makefnmark| to produce
%    the mark of the footnote. The macro |\@makefntext| is called when
%    effectively inside a |\parbox| of width |\columnwidth| (i.e.,
%    with |\hsize| = |\columnwidth|).
%
%   An example of what can be achieved is given by the following piece
%   of \TeX\ code.
% \begin{verbatim}
%          \long\def\@makefntext#1{%
%             \@setpar{\@@par
%                      \@tempdima = \hsize
%                      \advance\@tempdima-10pt
%                      \parshape \@ne 10pt \@tempdima}%
%             \par
%             \parindent 1em\noindent
%             \hbox to \z@{\hss\@makefnmark}#1}
% \end{verbatim}
%    The effect of this definition is that all lines of the footnote
%    are indented by 10pt, while the first line of a new paragraph is
%    indented by 1em. To change these dimensions, just substitute the
%    desired value for `10pt' (in both places) or `1em'.  The mark is
%    flushright against the footnote.
%
%    In these document classes we use a simpler macro, in which the
%    footnote text is set like an ordinary text paragraph, with no
%    indentation except on the first line of a paragraph, and the
%    first line of the footnote. Thus, all the macro must do is set
%    |\parindent| to the appropriate value for succeeding paragraphs
%    and put the proper indentation before the mark.
%
%    \begin{macrocode}
\long\def\@makefntext#1{
    \noindent
    \hangindent 10\p@
    \hb@xt@10\p@{\hss\@makefnmark}#1}
%    \end{macrocode}
% \end{macro}
%
% \begin{macro}{\@makefnmark}
%    The footnote markers that are printed in the text to point to the
%    footnotes should be produced by the macro |\@makefnmark|. We use
%    the default definition for it.
%    \begin{macrocode}
%\def\@makefnmark{\hbox{$^{\@thefnmark}\m@th$}}
%    \end{macrocode}
% \end{macro}
%
%
% \subsection{The title}
%     The commands |\title|, |\author|, and |\date| are already
%     defined, so here we just define |\maketitle|.
%
% \changes{v2.3s}{1995/10/10}{Move \cs{par} inside the scope of
%               \cs{Large}, to get even line spacing.}
%    \begin{macrocode}
\newcommand\maketitle{{\centering {\Large \@title \par}%
     \@author \par \@date\par}%
     \if@titlepage \break \fi}
%    \end{macrocode}

% \section{Initialization}
%
% \subsection{Date}
%
% \begin{macro}{\today}
%    This macro uses the \TeX\ primitives |\month|, |\day| and |\year|
%    to provide the date of the \LaTeX-run.
%    \begin{macrocode}
\newcommand\today{\ifcase\month\or
  January\or February\or March\or April\or May\or June\or
  July\or August\or September\or October\or November\or December\fi
  \space\number\day, \number\year}
%    \end{macrocode}
% \end{macro}
%
%
% Default initializations
%
%    \begin{macrocode}
\pagenumbering{arabic}
\onecolumn
%</class>
%    \end{macrocode}
%
% \subsection{Basic code}
%
% The code below is basically a copy of |slitex.tex| with some
% changes.
%
% Global changes so far:
%
% \changes{FMi}{1990/06/01}{\cs{gdef}\cs{@slidesw} ... replaced by a
%    \cs{newifG} which is similar to \cs{newif} but uses \cs{global}
%    inside.}
%
%
% \subsubsection{Hacks for slide macros}
%
%    \begin{macrocode}
%<*cmd>
\message{hacks,}

\outer\def\newifG#1{\count@\escapechar \escapechar\m@ne
  \expandafter\expandafter\expandafter
   \edef\@ifG#1{true}{\global\let\noexpand#1\noexpand\iftrue}%
  \expandafter\expandafter\expandafter
   \edef\@ifG#1{false}{\global\let\noexpand#1\noexpand\iffalse}%
  \@ifG#1{false}\escapechar\count@} % the condition starts out false
\def\@ifG#1#2{\csname\expandafter\ifG@\string#1#2\endcsname}
{\uccode`1=`i \uccode`2=`f \uccode`3=`G \uppercase{\gdef\ifG@123{G}}}
 % `ifG' is required

\def\@gobbletoend#1{\def\@argend{#1}\@ggobtoend}

\long\def\@ggobtoend#1\end#2{\fi\def\reserved@a{#2}%
\ifx\reserved@a\@argend\else\@ggobtoend\fi}
%    \end{macrocode}
% FMi: I don't see any reason for this command since |\fi| is hidden
%      anyway in the replacement text
% |\def\@xfi{\fi}|
%    \begin{macrocode}
 \message{slides,}
%    \end{macrocode}
%
% \subsubsection{Slide macros}
%
%
% Switches:\\
% \begin{tabular}{ll}
%   |@bw|           & true if making black and white slides \\
%   |@visible|      & true if visible output to be produced.\\
%   |@makingslides| & true if making a slide/overlay/note
% \end{tabular}
%
%    \begin{macrocode}
\newif\if@bw
\newif\if@visible
\newif\if@onlyslidesw \@onlyslideswfalse
\newif\if@onlynotesw  \@onlynoteswfalse
\newif\if@makingslides
%    \end{macrocode}
% FMi: |\newifG| replaces |\gdef\@slidesw{T}| stuff
%    \begin{macrocode}
\newifG\ifG@slidesw
%    \end{macrocode}
% Counters\\
% \begin{tabular}{ll}
%  slide   & slide number\\
%  overlay & overlay number for a slide\\
%  note    & note number for a slide
% \end{tabular}
%
%    \begin{macrocode}
\countdef\c@slide=0 \c@slide=0
\def\cl@slide{}
\countdef\c@overlay=1 \c@overlay=0
\def\cl@overlay{}
\countdef\c@note=2 \c@note=0
\def\cl@note{}
%    \end{macrocode}
%  Add these counters explicitly to the `ckpt list' so that the
%  |\include| mechanism works.
% \changes{v2.3u}{1996/05/09}{Make include work: pr/2140(CAR)}
% \changes{v2.3y}{1997/05/09}{Make include work properly: add the
% counters in case some are already in there: pr/2140+2474(CAR)}
% \changes{v2.3v}{1996/05/11}{Do not add page counter here as it is
%   added below(CAR)}
%    \begin{macrocode}
\g@addto@macro\cl@@ckpt{\@elt{slide}\@elt{overlay}\@elt{note}}
\@addtoreset{overlay}{slide}
\@addtoreset{note}{slide}
%    \end{macrocode}
% Redefine page counter to some other number.
% The page counter will always be zero except when putting out an
% extra page for a slide, note or overlay.
%    \begin{macrocode}
\@definecounter{page}
\@addtoreset{page}{slide}
\@addtoreset{page}{note}
\@addtoreset{page}{overlay}

\def\theslide{\@arabic\c@slide}
\def\theoverlay{\theslide-\@alph\c@overlay}
\def\thenote{\theslide-\@arabic\c@note}
%    \end{macrocode}
% \begin{verbatim}
% \@setlimits \LIST \LOW \HIGH
%
%    Assumes that \LIST = RANGE1,RANGE2,...,RANGEn  (n>0)
%    Where RANGEi = j or j-k.
%
%    Then \@setlimits  globally sets
%        (i) \LIST := RANGE2, ... , RANGEn
%       (ii) \LOW  := p
%      (iii) \HIGH := q
%   where either RANGE1 = p-q   or  RANGE1 = p  and  q=p.
%\end{verbatim}
%    \begin{macrocode}
\def\@sl@getargs#1-#2-#3\relax#4#5{\xdef#4{#1}\xdef#5{#2}}
\def\@sl@ccdr#1,#2\relax#3#4{\xdef#3{#1-#1-}\xdef#4{#2}}

\def\@setlimits #1#2#3{\expandafter\@sl@ccdr#1\relax\@sl@gtmp #1%
\expandafter\@sl@getargs\@sl@gtmp\relax#2#3}
%    \end{macrocode}
% \begin{verbatim}
% \onlyslides{LIST} ::=
%  BEGIN
%    @onlyslidesw := true
%    \@doglslidelist :=G LIST,999999,999999
%   if @onlynotesw = true
%     else @onlynotesw := true
%          \@doglnotelist :=G LIST,999999,999999
%   fi
%   message: Only Slides LIST
%  END
%\end{verbatim}
%    \begin{macrocode}
\def\onlyslides#1{\@onlyslideswtrue
   \gdef\@doglslidelist{#1,999999,999999}%
   \if@onlynotesw \else
      \@onlynoteswtrue\gdef\@doglnotelist{999999,999999}\fi
   \typeout{Only Slides #1}}
%    \end{macrocode}
%\begin{verbatim}
% \onlynotes{LIST} ::=
%  BEGIN
%    @onlynotesw := true
%    \@doglnotelist :=G LIST,999999,999999
%   if @onlyslidesw = true
%     else \@onlyslidesw := true
%          \@doglslidelist{999999,999999}
%   fi
%   message: Only Notes LIST
%  END
%\end{verbatim}
%    \begin{macrocode}
\def\onlynotes#1{\@onlynoteswtrue
   \gdef\@doglnotelist{#1,999999,999999}%
   \if@onlyslidesw \else
      \@onlyslideswtrue\gdef\@doglslidelist{999999,999999}\fi
   \typeout{Only Notes #1}}
%    \end{macrocode}
%\begin{verbatim}
% \setupcounters ::=    (similar to old \blackandwhite #1  ::= )
%    \newpage
%    page counter := 0
%    @bw := T
%    @visible := T
%    if @onlyslidesw = true
%      then  \@doslidelist := \@doglslidelist
%            \@setlimits\@doslidelist\@doslidelow\@doslidehigh
%    fi
%    if @onlynotesw = true
%      then  \@donotelist := \@doglnotelist
%            \@setlimits\@donotelist\@donotelow\@donotehigh
%    fi
%    \normalsize    % Note, this sets font to \rmfamily , which sets
%                     % \@currfont to \rmfamily
%    counter slidenumber := 0
%    counter note        := 0
%    counter overlay     := 0
%    @makingslides       := F  %% \blackandwhite: @makingslides  := T
%                              %%                 input #1
%                              %%                 @makingslides  := F
%\end{verbatim}
%    \begin{macrocode}
\if@compatibility
% In compatibility mode, need to define \verb+\blackandwhite+,
% \verb+\colors+, \verb+\colorslides+, etc.
\def\blackandwhite#1{\newpage\setcounter{page}{0}\@bwtrue\@visibletrue
\if@onlyslidesw \xdef\@doslidelist{\@doglslidelist}%
\@setlimits\@doslidelist\@doslidelow\@doslidehigh\fi
\if@onlynotesw \xdef\@donotelist{\@doglnotelist}%
\@setlimits\@donotelist\@donotelow\@donotehigh\fi
\normalsize\setcounter{slide}{0}\setcounter{overlay}{0}%
\setcounter{note}{0}\@makingslidestrue\input #1\@makingslidesfalse}
%    \end{macrocode}
%\begin{verbatim}
% \colors{COLORS} ::=
%  for \@colortemp := COLORS
%     do \csname \@colortemp \endcsname == \@color{\@colortemp} od
%  if \@colorlist = empty
%     then \@colorlist := COLORS
%     else \@colorlist := \@colorlist , COLORS
%  fi
%\end{verbatim}
%
%    \begin{macrocode}
\def\colors#1{\@for\@colortemp:=#1\do{\expandafter
  \xdef\csname\@colortemp\endcsname{\noexpand\@color{\@colortemp}}}\ifx
  \@colorlist\@empty \gdef\@colorlist{#1}%
    \else \xdef\@colorlist{\@colorlist,#1}\fi}

\def\@colorlist{}
%    \end{macrocode}
%\begin{verbatim}
% \colorslides{FILE} ::=
%    \newpage
%    page counter := 0
%    @bw := F
%    for \@currcolor := \@colorlist
%      do  @visible := T
%          if @onlyslidesw = true
%            then  \@doslidelist := \@doglslidelist
%                  \@setlimits\@doslidelist\@doslidelow\@doslidehigh
%          fi
%          if @onlynotesw = true
%            then  \@donotelist := \@doglnotelist
%                  \@setlimits\@donotelist\@donotelow\@donotehigh
%          fi
%          \normalsize
%          counter slide := 0
%          counter overlay := 0
%          counter note    := 0
%          type message
%          generate color layer output page
%          @makingslides := T
%          input #1
%          @makingslides := F
%      od
%\end{verbatim}
%    \begin{macrocode}
\def\colorslides#1{\newpage\setcounter{page}{0}\@bwfalse
\@for\@currcolor:=\@colorlist\do
{\@visibletrue
\if@onlyslidesw \xdef\@doslidelist{\@doglslidelist}%
\@setlimits\@doslidelist\@doslidelow\@doslidehigh\fi
\if@onlynotesw \xdef\@donotelist{\@doglnotelist}%
\@setlimits\@donotelist\@donotelow\@donotehigh\fi
\normalsize\setcounter{slide}{0}\setcounter{overlay}{0}%
\setcounter{note}{0}\typeout{color \@currcolor}%
\newpage
\begin{huge}%
\begin{center}%
COLOR LAYER\\[.75in]%
\@currcolor
\end{center}%
\end{huge}%
\newpage
\@makingslidestrue
\input #1
\@makingslidesfalse}}
%
\else  %% if@compatibility
%
\def\setupcounters{\newpage\setcounter{page}{0}\@bwtrue\@visibletrue
\if@onlyslidesw \xdef\@doslidelist{\@doglslidelist}%
\@setlimits\@doslidelist\@doslidelow\@doslidehigh\fi
\if@onlynotesw \xdef\@donotelist{\@doglnotelist}%
\@setlimits\@donotelist\@donotelow\@donotehigh\fi
\normalsize\setcounter{slide}{0}\setcounter{overlay}{0}%
\setcounter{note}{0}\@makingslidesfalse}

\AtBeginDocument{\setupcounters}
\fi %% if@compatibility
%    \end{macrocode}
%\begin{verbatim}
% \slide COLORS ::=
%  BEGIN
% \changes{v2.3}{1994/03/16}{Moved \cs{newpage} up front, here and in
%    \cs{note} and \cs{overlay}}
%   \par\break
%   \stepcounter{slide}
%   \setcounter{page}{0}                    % in case of non-slide pages
%   \@slidesw :=G T
%   if @onlyslidesw = true                  % set \@slidesw = T iff
%     then                                  % page to be output
%       while \c@slide > \@doslidehigh
%          do  \@setlimits\@doslidelist\@doslidelow\@doslidehigh  od
%       if \c@slide < \@doslidelow
%         then \@slidesw := F
%       fi
%   fi
%   if \@slidesw = T
%      then \@slidesw :=G F
%           \begingroup
%              if @bw = true
%                then  \@slidesw :=G T
%                else \@color{COLORS}
%                     \if@visible then \@slidesw :=G T \fi
%              fi
%            \endgroup
%  fi
%  if \@slidesw = T
%    then @makingslides  := T
%         \thispagestyle{slide}
%    else \end{slide}
%          \@gobbletoend{slide}
%  fi
% END
%
% \endslide ::=
%  BEGIN
%    \par\break
%  END
%\end{verbatim}
%    \begin{macrocode}
\if@compatibility
\def\slide#1{\stepcounter{slide}\G@slideswtrue\if@onlyslidesw
\@whilenum \c@slide >\@doslidehigh\relax
\do{\@setlimits\@doslidelist\@doslidelow\@doslidehigh}\ifnum
\c@slide <\@doslidelow\relax\G@slideswfalse\fi\fi
\ifG@slidesw
  \G@slideswfalse
% FMi this is only a hack at the moment to get things running.
%  \begingroup
  \if@bw\G@slideswtrue\else
    \@color{#1}\if@visible \G@slideswtrue \fi
  \fi
%  \endgroup
\fi
\ifG@slidesw \newpage\thispagestyle{slide}%
%    \end{macrocode}
%    This will set up the last color specified in the argument to
%    \verb+\slide+ as the current color. If only back and white  slides
%    are prepared \verb+\last@color+ will be empty and effectively
%    \verb+\relax+ will be generated (hopefully).
%
%    We need to reset to a default font at the beginning of a slide.
%    (not done yet).
%    \begin{macrocode}
\csname \last@color \endcsname
%    \end{macrocode}
%    \begin{macrocode}
\else\end{slide}\@gobbletoend{slide}\fi}
%
\else  %% if@compatibility
%
\def\slide{\par\break
\stepcounter{slide}\setcounter{page}{0}\G@slideswtrue\if@onlyslidesw
\@whilenum \c@slide >\@doslidehigh\relax
\do{\@setlimits\@doslidelist\@doslidelow\@doslidehigh}\ifnum
\c@slide <\@doslidelow\relax\G@slideswfalse\fi\fi
\ifG@slidesw
  \G@slideswfalse
% FMi this is only a hack at the moment to get things running.
%  \begingroup
  \if@bw\G@slideswtrue\else
    \if@visible \G@slideswtrue \fi
  \fi
%  \endgroup
\fi
\ifG@slidesw \@makingslidestrue\thispagestyle{slide}%
%    \end{macrocode}
%    This will set up the last color specified in the argument to
%    \verb+\slide+ as the current color. If only back and white  slides
%    are prepared \verb+\last@color+ will be empty and effectively
%    \verb+\relax+ will be generated (hopefully).
%
%    We need to reset to a default font at the beginning of a slide.
%    (not done yet).
%    \begin{macrocode}
\csname \last@color \endcsname
%    \end{macrocode}
%    \begin{macrocode}
\else\end{slide}\@gobbletoend{slide}\fi}
\fi %% if@compatibility

\let\last@color\@empty

\def\endslide{\par\break}
%    \end{macrocode}
%\begin{verbatim}
% \overlay COLORS ::=
%  BEGIN
%   \par\break
%   \stepcounter{overlay}
%   \setcounter{page}{0}                    % in case of non-slide pages
%   \@slidesw :=G T
%   if @onlyslidesw = T                     % set \@slidesw = T iff
%     then                                  % page to be output
%       if \c@slide < \@doslidelow
%         then \@slidesw :=G F
%       fi
%  fi
%  if \@slidesw = T
%    \@slidesw :=G F
%    \begingroup
%      if @bw = true
%          then  \@slidesw :=G T
%          else  \@color{COLORS}
%                \if@visible then \@slidesw :=G T \fi
%      fi
%    \endgroup
%  fi
%  if \@slidesw = T
%     then @makingslides  := T
%          \thispagestyle{overlay}
%     else \end{overlay}
%          \@gobbletoend{overlay}
%  fi
% END
%
% \endoverlay ::=
%  BEGIN
%    \par\break
%  END
%\end{verbatim}
%    \begin{macrocode}
\if@compatibility
\def\overlay#1{\stepcounter{overlay}\G@slideswtrue%
\if@onlyslidesw\ifnum \c@slide <\@doslidelow\relax
\G@slideswfalse\fi\fi
\ifG@slidesw \G@slideswfalse\begingroup\if@bw\G@slideswtrue%
\else\@color{#1}\if@visible \G@slideswtrue\fi\fi\endgroup\fi
\ifG@slidesw \newpage\thispagestyle{overlay}%
\else\end{overlay}\@gobbletoend{overlay}\fi}
%
\else %%if@compatibility
%
\def\overlay{\par\break
  \stepcounter{overlay}%
  \setcounter{page}{0}%
  \G@slideswtrue%
  \if@onlyslidesw\ifnum \c@slide <\@doslidelow\relax
    \G@slideswfalse\fi\fi
  \ifG@slidesw \G@slideswfalse
    \begingroup\if@bw\G@slideswtrue%
               \else\if@visible \G@slideswtrue\fi\fi
    \endgroup\fi
  \ifG@slidesw \@makingslidestrue\thispagestyle{overlay}%
  \else\end{overlay}\@gobbletoend{overlay}\fi}
\fi %%if@compatibility

\def\endoverlay{\par\break}
%    \end{macrocode}
%
% \changes{v2.0d}{1993/11/12}{Removed extra blank.}
%\begin{verbatim}
% \note ::=
%  BEGIN
%   \par\break
%   \stepcounter{note}
%   \setcounter{page}{0}                    % in case of non-slide pages
%   if @bw = T
%     then
%       \@slidesw :=G T
%       if @onlynotesw = true                  % set \@notesw = T iff
%         then                                 % page to be output
%           while \c@slide > \@donotehigh
%              do  \@setlimits\@donotelist\@donotelow\@donotehigh  od
%           if \c@slide < \@donotelow
%             then \@slidesw :=G F
%           fi
%       fi
%     else \@slidesw :=G F
%  fi
%  if \@slidesw = T
%     then @makingslides   := T
%          \thispagestyle{note}
%     else \end{note}
%          \@gobbletoend{note}
%  fi
% END
%
% \endnote ::=
%  BEGIN
%    \par\break
%  END
%\end{verbatim}
%    \begin{macrocode}
\if@compatibility
\def\note{\stepcounter{note}%
   \if@bw
      \G@slideswtrue
      \if@onlynotesw\@whilenum \c@slide >\@donotehigh\relax
      \do{\@setlimits\@donotelist\@donotelow\@donotehigh}\ifnum
        \c@slide <\@donotelow\relax \G@slideswfalse\fi\fi
      \else\G@slideswfalse\fi
      \ifG@slidesw \newpage\thispagestyle{note}\else
      \end{note}\@gobbletoend{note}\fi}
%
\else %%if@compatibility
%
\def\note{\par\break\stepcounter{note}\setcounter{page}{0}%
   \if@bw
      \G@slideswtrue
      \if@onlynotesw\@whilenum \c@slide >\@donotehigh\relax
      \do{\@setlimits\@donotelist\@donotelow\@donotehigh}\ifnum
        \c@slide <\@donotelow\relax \G@slideswfalse\fi\fi
      \else\G@slideswfalse\fi
      \ifG@slidesw \@makingslidestrue\thispagestyle{note}\else
      \end{note}\@gobbletoend{note}\fi}
\fi %%if@compatibility

\def\endnote{\par\break}
%    \end{macrocode}
%\begin{verbatim}
% \@color{COLORS} ::=
%  BEGIN
%   if math mode
%     then type warning
%   fi
%   if @bw
%     then \visible
%     else \invisible
%         for \last@color := COLORS
%          do if \last@color = \@currcolor
%               then \visible
%             fi
%          od
%   fi
%   \ignorespaces
% END
%\end{verbatim}
% FMi: |\last@color| will be used in |\slide| to set up first
%      color if no color is given.
% I suppose that this is much too complicated. |\else\@tempswafalse|
% would produce the same effect I imagine.
%    \begin{macrocode}
\def\@color#1{\@mmodetest
 {\if@bw \@tempswatrue \else \@tempswafalse
   \@for \reserved@a :=#1\do{\ifx\reserved@a\@currcolor\@tempswatrue\fi
                         \let\last@color\reserved@a}\fi
  \if@tempswa \visible \else \invisible \fi
  \ignorespaces}}

\def\@mmodetest#1{\ifmmode\ClassWarning{slides}{Color-changing command
       in math mode has been ignored}\else #1\fi}

\def\invisible{\@mmodetest
  {\if@visible
     \@visiblefalse
     \fontshape\f@shape\selectfont
     \mathversion{invisible}%
   \fi
   \ignorespaces}}

\def\visible{\@mmodetest
  {\if@visible
   \else
     \@visibletrue
%    \end{macrocode}
%    Here is the \LaTeXe{} interface hidden. We use a trick to provide
%    ourselves with a sort of additional attribute without making the
%    current mechanism even larger. The trick is that we denote
%    invisible by putting an uppercase |I| in front of the shape name
%    for invisible shapes and remove it again if we want to become
%    visible.
%    \begin{macrocode}
     \fontshape{\expandafter\@gobble\f@shape}\selectfont
     \mathversion{normal}%
   \fi
   \ignorespaces}}

\def\fontshape#1{\edef\f@shape{\if@visible \else I\fi #1}}
%    \end{macrocode}
%
% \subsection{Macros for font handling}
%
%    We let |\familydefault| point at |\sfdefault|, to make it easier
%    to use the document class slides with packages that set up other
%    fonts.
%    \begin{macrocode}
\renewcommand{\familydefault}{\sfdefault}
%    \end{macrocode}
%
%  \changes{v2.3l}{1994/12/16}{Added the declaration of the lasy font
%    family}
%    The \texttt{latexsym} package, which is needed to be able to access
%    the \LaTeX\ symbol fonts (lasy), sets things up so that for sizes
%    larger then 10 point magnifications of \texttt{lasy10} are
%    used. For slides we want to use magnifications of \texttt{lasy8},
%    so we set up the lasy family here to prevent \LaTeX\ from loading
%    \texttt{Ulasy.fd}.
%    \begin{macrocode}
\DeclareFontFamily{U}{lasy}{}{}
\DeclareFontShape{U}{lasy}{m}{n}{%
      <12><13.82><16.59><19.907><23.89><28.66><34.4><41.28>lasy8
}{}
\DeclareFontShape{U}{lasy}{m}{In}{%
      <13.82><16.59><19.907><23.89><28.66><34.4><41.28>ilasy8
}{}
%    \end{macrocode}
%
%    \begin{macrocode}
\message{picture,}
%    \end{macrocode}
%
%  \subsubsection{Modifications to the picture environment}
%
%  Below are the new definitions of the picture-drawing macros
%  required for SLiTeX.  Only those commands that actually
%  draw something must be changed so that they do not produce
%  any output when the |@visible| switch is false.
%
% \changes{v2.2j}{1994/03/11}{Corrected \cs{@oval}, like previous
%    change to the \LaTeX{} format.}
% \changes{v2.4a}{2016/03/29}{Initialize in \cs{@oval}tests added for
%    latex/4452}
%    \begin{macrocode}
\def\line(#1,#2)#3{\if@visible\@xarg #1\relax \@yarg #2\relax
\@linelen #3\unitlength
\ifnum\@xarg =\z@ \@vline
  \else \ifnum\@yarg =\z@ \@hline \else \@sline\fi
\fi\fi}

\def\vector(#1,#2)#3{\if@visible\@xarg #1\relax \@yarg #2\relax
\@linelen #3\unitlength
\ifnum\@xarg =\z@ \@vvector
  \else \ifnum\@yarg =\z@ \@hvector \else \@svector\fi
\fi\fi}

\def\dashbox#1(#2,#3){%
\leavevmode\if@visible\hb@xt@\z@{\baselineskip \z@
\lineskip \z@
\@dashdim #2\unitlength
\@dashcnt \@dashdim \advance\@dashcnt 200
\@dashdim #1\unitlength\divide\@dashcnt \@dashdim
\ifodd\@dashcnt\@dashdim\z@
\advance\@dashcnt \@ne \divide\@dashcnt \tw@
\else \divide\@dashdim \tw@ \divide\@dashcnt \tw@
\advance\@dashcnt \m@ne
\setbox\@dashbox \hbox{\vrule \@height \@halfwidth \@depth \@halfwidth
\@width \@dashdim}\put(0,0){\copy\@dashbox}%
\put(0,#3){\copy\@dashbox}%
\put(#2,0){\hskip-\@dashdim\copy\@dashbox}%
\put(#2,#3){\hskip-\@dashdim\box\@dashbox}%
\multiply\@dashdim \thr@@
\fi
\setbox\@dashbox \hbox{\vrule \@height \@halfwidth \@depth \@halfwidth
\@width #1\unitlength\hskip #1\unitlength}\@tempcnta\z@
\put(0,0){\hskip\@dashdim \@whilenum \@tempcnta <\@dashcnt
\do{\copy\@dashbox\advance\@tempcnta \@ne }}\@tempcnta\z@
\put(0,#3){\hskip\@dashdim \@whilenum \@tempcnta <\@dashcnt
\do{\copy\@dashbox\advance\@tempcnta \@ne }}%
\@dashdim #3\unitlength
\@dashcnt=\@dashdim \advance\@dashcnt 200
\@dashdim #1\unitlength\divide\@dashcnt \@dashdim
\ifodd\@dashcnt \@dashdim=\z@
\advance\@dashcnt \@ne \divide\@dashcnt \tw@
\else
\divide\@dashdim \tw@ \divide\@dashcnt \tw@
\advance\@dashcnt \m@ne
\setbox\@dashbox\hbox{\hskip -\@halfwidth
\vrule \@width \@wholewidth
\@height \@dashdim}\put(0,0){\copy\@dashbox}%
\put(#2,0){\copy\@dashbox}%
\put(0,#3){\lower\@dashdim\copy\@dashbox}%
\put(#2,#3){\lower\@dashdim\copy\@dashbox}%
\multiply\@dashdim \thr@@
\fi
\setbox\@dashbox\hbox{\vrule \@width \@wholewidth
\@height #1\unitlength}\@tempcnta\z@
\put(0,0){\hskip -\@halfwidth \vbox{\@whilenum \@tempcnta <\@dashcnt
\do{\vskip #1\unitlength\copy\@dashbox\advance\@tempcnta \@ne }%
\vskip\@dashdim}}\@tempcnta\z@
\put(#2,0){\hskip -\@halfwidth \vbox{\@whilenum \@tempcnta <\@dashcnt
\relax\do{\vskip #1\unitlength\copy\@dashbox\advance\@tempcnta \@ne }%
\vskip\@dashdim}}}\fi\@makepicbox(#2,#3)}
%    \end{macrocode}
% (re)declare these booleans as they not defined in old format
% (or with latexrelease package)
%    \begin{macrocode}
\newif\if@ovvline \@ovvlinetrue
\newif\if@ovhline \@ovhlinetrue
%    \end{macrocode}
%    \begin{macrocode}
\def\@oval(#1,#2)[#3]{\if@visible\begingroup \boxmaxdepth \maxdimen
  \@ovttrue \@ovbtrue \@ovltrue \@ovrtrue
%    \end{macrocode}
%    \begin{macrocode}
  \@ovvlinefalse \@ovhlinefalse
%    \end{macrocode}
%    \begin{macrocode}
  \@tfor\reserved@a :=#3\do
     {\csname @ov\reserved@a false\endcsname}%
  \@ovxx#1\unitlength \@ovyy #2\unitlength
%    \end{macrocode}
%    \begin{macrocode}
  \@tempdimb \ifdim \@ovyy >\@ovxx \@ovxx \@ovvlinetrue
  \else \@ovyy \ifdim \@ovyy =\@ovxx \else \@ovhlinetrue \fi\fi
%    \end{macrocode}
%    \begin{macrocode}
  \advance \@tempdimb -2\p@
  \@getcirc \@tempdimb
  \@ovro \ht\@tempboxa \@ovri \dp\@tempboxa
  \@ovdx\@ovxx \advance\@ovdx -\@tempdima \divide\@ovdx \tw@
  \@ovdy\@ovyy \advance\@ovdy -\@tempdima \divide\@ovdy \tw@
%    \end{macrocode}
%    \begin{macrocode}
  \ifdim \@ovdx >\z@ \@ovhlinetrue \fi
  \ifdim \@ovdy >\z@ \@ovvlinetrue \fi
%    \end{macrocode}
%    \begin{macrocode}
  \@circlefnt \setbox\@tempboxa
  \hbox{\if@ovr \@ovvert32\kern -\@tempdima \fi
  \if@ovl \kern \@ovxx \@ovvert01\kern -\@tempdima \kern -\@ovxx \fi
  \if@ovt \@ovhorz \kern -\@ovxx \fi
  \if@ovb \raise \@ovyy \@ovhorz \fi}\advance\@ovdx\@ovro
  \advance\@ovdy\@ovro \ht\@tempboxa\z@ \dp\@tempboxa\z@
  \@put{-\@ovdx}{-\@ovdy}{\box\@tempboxa}%
  \endgroup\fi}

\def\@circle#1{\if@visible \begingroup \boxmaxdepth \maxdimen
   \@tempdimb #1\unitlength
   \ifdim \@tempdimb >15.5\p@\relax \@getcirc\@tempdimb
      \@ovro\ht\@tempboxa
     \setbox\@tempboxa\hbox{\@circlefnt
      \advance\@tempcnta\tw@ \char \@tempcnta
      \advance\@tempcnta\m@ne \char \@tempcnta \kern -2\@tempdima
      \advance\@tempcnta\tw@
      \raise \@tempdima \hbox{\char\@tempcnta}\raise \@tempdima
        \box\@tempboxa}\ht\@tempboxa\z@ \dp\@tempboxa\z@
      \@put{-\@ovro}{-\@ovro}{\box\@tempboxa}%
   \else  \@circ\@tempdimb{96}\fi\endgroup\fi}

%    \end{macrocode}
%
% \changes{v2.0d}{1993/11/12}{Removed extra blank.}
%    \begin{macrocode}
\def\@dot#1{%
  \if@visible\@tempdimb #1\unitlength \@circ\@tempdimb{112}\fi}
%    \end{macrocode}
% \changes {v2.3m}{1994/03/20}{(DPC) Remove old \cs{@iframebox} and
%    \cs{fbox} defn}
% \changes {v2.3m}{1994/03/20}{(DPC) Add \cs{@frameb@x} defn.}
%    \begin{macrocode}
\def\@frameb@x#1{%
  \@tempdima\fboxrule
  \advance\@tempdima\fboxsep
  \advance\@tempdima\dp\@tempboxa
  \leavevmode
  \hbox{%
    \lower\@tempdima\hbox{%
      \vbox{%
        \if@visible\hrule\@height\else\vskip\fi\fboxrule
        \hbox{%
          \if@visible\vrule\@width\else\hskip\fi\fboxrule
          #1%
          \vbox{%
            \vskip\fboxsep
            \box\@tempboxa
            \vskip\fboxsep}%
          #1%
          \if@visible\vrule\@width\else\hskip\fi\fboxrule}%
        \if@visible\hrule\@height\else\vskip\fi\fboxrule}}}}

%    \end{macrocode}
%
%    \begin{macrocode}
\long\def\frame#1{\if@visible\leavevmode
\vbox{\vskip-\@halfwidth\hrule \@height\@halfwidth \@depth \@halfwidth
  \vskip-\@halfwidth\hbox{\hskip-\@halfwidth \vrule \@width\@wholewidth
  \hskip-\@halfwidth #1\hskip-\@halfwidth \vrule \@width \@wholewidth
  \hskip -\@halfwidth}\vskip -\@halfwidth\hrule \@height \@halfwidth
  \@depth \@halfwidth\vskip -\@halfwidth}\else #1\fi}
%    \end{macrocode}
%
% \changes{v2.0d}{1993/11/12}{Corrected \cs{@math} to \cs{m@th} in
%    definition of \cs{underline}.}
%
%    \begin{macrocode}
\message{mods,}
%    \end{macrocode}
%
%
%  \subsubsection{Other modifications to \TeX{} and \LaTeX{} commands}
%
% |\rule|
%    \begin{macrocode}
\def\@rule[#1]#2#3{\@tempdima#3\advance\@tempdima #1\leavevmode
 \hbox{\if@visible\vrule
  \@width#2 \@height\@tempdima \@depth-#1\else
\vrule \@width \z@ \@height\@tempdima \@depth-#1\vrule
 \@width#2 \@height\z@\fi}}

% \_  (Added 10 Nov 86)

\def\_{\leavevmode \kern.06em \if@visible\vbox{\hrule \@width.3em}\else
   \vbox{\hrule \@height \z@ \@width.3em}\vbox{\hrule \@width \z@}\fi}
%    \end{macrocode}
%\begin{verbatim}
% \overline, \underline, \frac and \sqrt
%
% \@mathbox{STYLE}{BOX}{MTEXT} : Called in math mode, typesets MTEXT and
%   stores result in BOX, using STYLE.
%
% \@bphant{BOX}    : Creates a phantom with dimensions BOX.
% \@vbphant{BOX}   : Creates a phantom with ht of BOX and zero width.
% \@hbphant{BOX}   : Creates a phantom with width of BOX
%                    and zero ht & dp.
% \@hvsmash{STYLE}{MTEXT} : Creates a copy of MTEXT with zero height and
%                           width in STYLE.
%\end{verbatim}
%    \begin{macrocode}
\def\@mathbox#1#2#3{\setbox#2\hbox{$\m@th#1{#3}$}}

\def\@vbphantom#1{\setbox\tw@\null \ht\tw@\ht #1\dp\tw@\dp #1%
   \box\tw@}

\def\@bphantom#1{\setbox\tw@\null
   \wd\tw@\wd #1\ht\tw@\ht #1\dp\tw@\dp #1%
   \box\tw@}

\def\@hbphantom#1{\setbox\tw@\null \wd\tw@\wd #1\ht\tw@\z@ \dp\tw@\z@
   \box\tw@}

\def\@hvsmash#1#2{\@mathbox#1\z@{#2}\ht\z@\z@ \dp\z@\z@ \wd\z@\z@
   \box\z@}

\def\underline#1{\relax\ifmmode
  \@xunderline{#1}\else $\m@th\@xunderline{\hbox{#1}}$\relax\fi}

\def\@xunderline#1{\mathchoice{\@xyunderline\displaystyle{#1}}%
   {\@xyunderline
    \textstyle{#1}}{\@xyunderline\scriptstyle{#1}}{\@xyunderline
      \scriptscriptstyle{#1}}}

\def\@xyunderline#1#2{%
   \@mathbox#1\@smashboxa{#2}\@hvsmash#1{\copy\@smashboxa}%
   \if@visible \@hvsmash#1{\@@underline{\@bphantom\@smashboxa}}\fi
  \@mathbox#1\@smashboxb{\@@underline{\box\@smashboxa}}%
   \@bphantom\@smashboxb}

\let\@@overline=\overline

\def\overline#1{\mathchoice{\@xoverline\displaystyle{#1}}{\@xoverline
    \textstyle{#1}}{\@xoverline\scriptstyle{#1}}{\@xoverline
      \scriptscriptstyle{#1}}}

\def\@xoverline#1#2{%
   \@mathbox#1\@smashboxa{#2}\@hvsmash#1{\copy\@smashboxa}%
   \if@visible \@hvsmash#1{\@@overline{\@bphantom\@smashboxa}}\fi
   \@mathbox#1\@smashboxb{\@@overline{\box\@smashboxa}}%
   \@bphantom\@smashboxb}
%    \end{macrocode}
%
% \changes{v2.0b}{1993/04/14}{Corrected \cs{frac} command.}
% \changes{v2.0d}{1993/11/12}{Removed \cs{vcenter} in \cs{@frac}.}
%
%\begin{verbatim}
% \@frac {STYLE}{DENOMSTYLE}{NUM}{DEN}{FONTSIZE} :
%   Creates \frac{NUM}{DENOM}
%   in style STYLE with NUM and DENOM in style DENOMSTYLE
%   FONTSIZE should be \textfont \scriptfont or \scriptscriptfont
%\end{verbatim}
%    Added a group around the first argument of |\frac| to prevent
%    changes (for example font changes) to modify the contents of the
%    second argument.
% \changes{v2.1c}{1993/12/13}{Added group around first arg.}
%    \begin{macrocode}
\def\frac#1#2{\mathchoice
   {\@frac\displaystyle\textstyle{#1}{#2}\textfont}{\@frac
         \textstyle\scriptstyle{#1}{#2}\textfont}{\@frac
         \scriptstyle\scriptscriptstyle{#1}{#2}\scriptfont}{\@frac
         \scriptscriptstyle\scriptscriptstyle{#1}{#2}\scriptscriptfont}}

\def\@frac#1#2#3#4#5{%
   \@mathbox#1\@smashboxc{{\begingroup#3\endgroup\over#4}}%
   \setbox\tw@\null
   \ht\tw@ \ht\@smashboxc
   \dp\tw@ \dp\@smashboxc
   \wd\tw@ \wd\@smashboxc
   \box\if@visible\@smashboxc\else\tw@\fi}

\def\r@@t#1#2{\setbox\z@\hbox{$\m@th#1\@xysqrt#1{#2}$}%
  \dimen@\ht\z@ \advance\dimen@-\dp\z@
  \mskip5mu\raise.6\dimen@\copy\rootbox \mskip-10mu\box\z@}
%    \end{macrocode}
% \changes{v2.3w}{1996/05/15}{Removed use of obsolete command
%                             \cs{@@sqrt} (CAR)}
%    \begin{macrocode}
\def\sqrt{\@ifnextchar[{\@sqrt}{\@xsqrt}}
\def\@sqrt[#1]{\root #1\of}
\def\@xsqrt#1{\mathchoice{\@xysqrt\displaystyle{#1}}{\@xysqrt
     \textstyle{#1}}{\@xysqrt\scriptstyle{#1}}{\@xysqrt
    \scriptscriptstyle{#1}}}
\def\@xysqrt#1#2{\@mathbox#1\@smashboxa{#2}\if@visible
    \@hvsmash#1{\sqrtsign{\@bphantom\@smashboxa}}\fi
    \phantom{\sqrtsign{\@vbphantom\@smashboxa}}\box\@smashboxa}

\newbox\@smashboxa
\newbox\@smashboxb
\newbox\@smashboxc
%    \end{macrocode}
%
% array and tabular environments: changes to `\verb+|+', |\hline|,
% |\cline|, and |\vline|,
% added 8 Jun 88
%    \begin{macrocode}
\def\@arrayrule{\if@visible\@addtopreamble{\hskip -.5\arrayrulewidth
   \vrule \@width \arrayrulewidth\hskip -.5\arrayrulewidth}\fi}
%    \end{macrocode}
%
% \changes{v2.3t}{1996/01/31}{Change in \cs{cline} calling interface}
%    \begin{macrocode}
\def\cline#1{\if@visible\@cline#1\@nil\fi}

\def\hline{\noalign{\ifnum0=`}\fi
    \if@visible \hrule \@height \arrayrulewidth
      \else \hrule \@width \z@
    \fi
    \futurelet \reserved@a\@xhline}

\def\vline{\if@visible \vrule \@width \arrayrulewidth
            \else \vrule \@width \arrayrulewidth \@height \z@
           \@depth \z@ \fi}
%    \end{macrocode}
%
%    \begin{macrocode}
\message{output,}
%    \end{macrocode}
%
% \subsubsection{Changes to \LaTeX{} output routine}
%
%\begin{verbatim}
%  \@makecol ==
%    BEGIN
% % Following test added for slides to check if extra page
%     if @makingslides = T
%     then if \c@page > 0
%             then if \c@note > 0
%                     then type 'Note \thenote too long.'
%                     else if \c@overlay > 0
%                            then type 'Overlay \theoverlay too long.'
%                            else type 'Slide \theslide too long'
%     fi     fi     fi     fi
%     ifvoid \insert\footins
%        then  \@outputbox := \box255
%        else  \@outputbox := \vbox {\unvbox255
%                                    \vskip \skip\footins
%                                    \footnoterule
%                                    \unvbox\@footinsert
%                                   }
%    fi
%    \@freelist :=G \@freelist * \@midlist
%    \@midlist  :=G empty
%    \@combinefloats
%    \@outputbox := \vbox to \@colht{\boxmaxdepth := \maxdepth
%                                     \vfil     %%\vfil added for slides
%                                     \unvbox\@outputbox
%                                     \vfil }   %%\vfil added for slides
%    \maxdepth :=G \@maxdepth
%    END
%\end{verbatim}
% FMi simple hack to allow none centered slides Should be revised
% of course.
%    \begin{macrocode}
\let\@topfil\vfil

\def\@makecol{\if@makingslides\ifnum\c@page>\z@ \@extraslide\fi\fi
\ifvoid\footins \setbox\@outputbox\box\@cclv \let\@botfil\vfil
   \else\let\@botfil\relax\setbox\@outputbox
     \vbox{\unvbox\@cclv\vfil
           \vskip\skip\footins\footnoterule\unvbox\footins\vskip
            \z@ plus.1fil\relax}\fi
  \xdef\@freelist{\@freelist\@midlist}\gdef\@midlist{}\@combinefloats
     \setbox\@outputbox\vbox to\@colht{\boxmaxdepth\maxdepth
        \@topfil\unvbox\@outputbox\@botfil}\global\maxdepth\@maxdepth}

\def\@extraslide{\ifnum\c@note>\z@
    \ClassWarning{slides}{Note \thenote\space too long}\else
     \ifnum\c@overlay>\z@
        \ClassWarning{slides}{Overlay \theoverlay\space too long}\else
        \ClassWarning{slides}{Slide \theslide\space too long}\fi\fi}
%    \end{macrocode}
%
%    \begin{macrocode}
\message{init}
%    \end{macrocode}
%
%
% \subsubsection{Special \SLiTeX{} initializations}
%
%FMi why not allow for ref's ?
%    \begin{macrocode}
%   \nofiles

\@visibletrue
%</cmd>
%    \end{macrocode}
%
%
% \Finale
%
\endinput
