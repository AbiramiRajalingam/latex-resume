% \iffalse meta-comment
%
% Copyright (C) 1993-2024
% The LaTeX Project and any individual authors listed elsewhere
% in this file.
%
% This file is part of the LaTeX base system.
% -------------------------------------------
%
% It may be distributed and/or modified under the
% conditions of the LaTeX Project Public License, either version 1.3c
% of this license or (at your option) any later version.
% The latest version of this license is in
%    https://www.latex-project.org/lppl.txt
% and version 1.3c or later is part of all distributions of LaTeX
% version 2008 or later.
%
% This file has the LPPL maintenance status "maintained".
%
% The list of all files belonging to the LaTeX base distribution is
% given in the file `manifest.txt'. See also `legal.txt' for additional
% information.
%
% The list of derived (unpacked) files belonging to the distribution
% and covered by LPPL is defined by the unpacking scripts (with
% extension .ins) which are part of the distribution.
%
% \fi
% \iffalse
%%% From File: ltfssbas.dtx
%% Copyright (C) 1989-2002 Frank Mittelbach and Rainer Sch\"opf
%
%<*driver>
% \fi
%
%
\ProvidesFile{ltfssbas.dtx}
             [2024/02/08 v3.2l LaTeX Kernel (NFSS Basic Macros)]
% \iffalse
\documentclass{ltxdoc}
\begin{document}
\DocInput{ltfssbas.dtx}
\end{document}
%</driver>
% \fi
%
% \iffalse
%<+checkmem>\def\CHECKMEM{\tracingstats=2
%<+checkmem>  \newlinechar=`\^^J
%<+checkmem>  \message{^^JMemory usage: \filename}\shipout\hbox{}}
%<+checkmem>\CHECKMEM
% \fi
%
%
%
% \GetFileInfo{ltfssbas.dtx}
% \title{A new font selection scheme for \TeX{} macro packages\\
%        (Basic Macros)\thanks
%       {This file has version number
%       \fileversion\ dated \filedate}}
%
% \author{Frank Mittelbach \and Rainer Sch\"opf}
%
% \MaintainedByLaTeXTeam{latex}
% \maketitle
%
% This file contains the main implementation of the `low level' font
% selection commands. See other parts of the \LaTeX\
% distribution, or  \emph{The \LaTeX\ Companion} for higher level
% documentation of the \LaTeX\ `New' Font Selection Scheme.
%
% \begin{quote}
%   \textbf{Warning:}
%  The macro documentation is still basically the documentation from the
%  first NFSS release and therefore in some cases probably not
%  completely accurate.
% \end{quote}
%
% \MaybeStop{}
%
% \changes{v3.0v}{1998/08/17}{(RmS) Documentation fixes.}
% \changes{v3.0j}{1995/10/22}{(RmS) New size function macro
%    \cs{genb@sfcnt} needs to be disabled at \cs{document}.}
% \changes{v3.0i}{1995/10/20}
%         {(DPC) Modify autoload code, change \cs{undefined}}
% \changes{v3.0g}{1995/10/04}{Modify autoload code}
% \changes{v3.0f}{1995/08/24}{Added autoload code}
% \changes{v3.0d}{1995/07/13}
%         {minor documentation changes}
% \changes{v3.0c}{1995/06/15}
%         {(DPC) minor documentation changes}
% \changes{v3.0a}{1995/05/24}
%         {(DPC) Make file from previous file, fam.dtx 1995/05/20 v2.2d}
%
%
% \changes{v2.1x}{1994/11/18}{(DPC) use \cs{reserved@f} not \cs{next}}
% \changes{v2.1q}{1994/05/20}{Use new error commands}
% \changes{v2.1o}{1994/05/15}{encoding cmds changed to {enc}-cmd}
% \changes{v2.1k}{1994/05/13}{Remove File identification `typeout'}
% \changes{v2.1j}{1994/05/12}{New baselinestretch concept}
% \changes{v2.1j}{1994/05/12}{Replaced hand-protected commands by
%                             \cs{DeclareRobustCommand} defs}
% \changes{v2.1h}{1994/04/11}{Added \cs{defaultscriptratio} and
%    \cs{defaultscriptscriptratio}.  ASAJ.}
% \changes{v2.1g}{1994/03/13}{add 2ekernel module to omit repeated code}
% \changes{v2.1a}{1994/01/17}{New math font setup}
% \changes{v2.0c}{1993/03/18}{Changed all \cs{@tempdima} in
%               \cs{@tempdimb} to avoid killing \cs{numberline}}
% \changes{v1.91a}{1992/09/22}{Introduced \cs{tf@size} for math size.}
% \changes{v1.3a}{1991/11/27}{All \cs{family}, \cs{shape} etc.
%                           renamed to \cs{fontfamily} etc.}
% \changes{v1.2i}{1990/04/01}{Code added from tracefnt.dtx.}
% \changes{v1.2i}{1990/04/01}{Support for TeX3.}
% \changes{v1.2g}{1990/02/16}{Support for changes of \cs{baselineskip}
%          without changing the size.}
%
% \changes{v1.2c}{1990/01/23}{\cs{no@version@warning} renamed to
%                           \cs{no@alphabet@error}.}
% \changes{v1.0s}{1989/11/25}{All \cs{edef}\cs{font@name} changed to
%                           \cs{xdef}\cs{font@name}.
%                           Necessary after introduction of
%                           \cs{begingroup}/\cs{endgroup} in v1.0q.}
% \changes{v1.0s}{1989/11/25}{extra// $\to$ + in \cs{extra@def}.}
% \changes{v1.0o}{1989/11/08}{First parameter of
%                           \cs{define@mathalphabet}
%                           and \cs{define@mathgroup} changed
%                           from string to control sequence.}
% \changes{v1.0m}{1989/09/14}
%      {Global replacement: \cs{group} to \cs{mathgroup}}
% \changes{v1.0k}{1989/05/22}{Lines longer than 72 characters folded.}
% \changes{v1.0j}{1989/05/01}{Default for \cs{baselinestretch} added.}
% \changes{v1.0i}{1989/04/29}
%      {Removed the \cs{halign} \cs{noalign} correction
%                           (wasn't bugfree)}
% \changes{v1.0h}{1989/04/29}{Documented problem with \cs{halign}, and
%                           \cs{noalign}}
% \changes{v1.0g}{1989/04/27}{Documentation revised.}
% \changes{v1.0f}{1989/04/23}{\% in \cs{getanddefinefonts} added.}
% \changes{v1.0e}{1989/04/21}{Documentation is fun!!
%                       Parameters of \cs{define@mathalphabet} changed.}
% \changes{v1.0d}{1989/04/19}{Even more doc.}
% \changes{v1.0c}{1989/04/14}{More documentation added.}
% \changes{v1.0b}{1989/04/10}{\cs{preload@sizes} added.}
% \changes{v1.0b}{1989/04/10}{\cs{wrong@fontshape} changed to define
%                            substitution font/shape macro.}
% \changes{v1.0a}{1989/04/10}{Starting with version numbers!!
%                           \cs{ifmmode} added in \cs{math@group}}
% \changes{v2.1w}{1994/11/17}{\cs{@tempa} to \cs{reserved@a}}
% \changes{v3.0z}{2015/02/21}{Removed autoload code}
% \changes{v3.2d}{2019/08/27}{Make various commands robust}
%
%
%
%
% \section{Preliminary macros}
%
% We define a number
% of macros that will be used later.
%
%
% \begin{macro}{\@nomath}
%    |\@nomath| is used by most macros that will have no effect
%    in math mode. It issues a warning message.
%    \begin{macrocode}
%<*2ekernel>
\def\@nomath#1{\relax\ifmmode
   \@font@warning{Command \noexpand#1invalid in math mode}\fi}
%    \end{macrocode}
% \end{macro}
%
% \changes{v1.2c}{1990/01/23}{Macro \cs{no@alphabet@help} added}
% \changes{v2.1i}{1994/04/18}{Macro \cs{no@alphabet@help}
%                             removed again}
%
% \begin{macro}{\no@alphabet@error}
% \changes{v1.2c}{1990/01/23}{Changed to error call}
% \changes{v2.1i}{1994/04/18}{Use std LaTeX error macro}
%    The macro |\no@alphabet@error| is called whenever
%    the user requests a math \emph{alphabet} that is not
%    available in the current \emph{version}.
%    In math mode an error message is produced otherwise the command
%    keeps silent.
%    The argument is the name of the control sequence that identifies
%    the math \emph{alphabet}.
%    The |\relax| at the beginning is necessary to prevent
%    \TeX{} from scanning too far in certain situations.
%    \begin{macrocode}
\gdef\no@alphabet@error#1{\relax \ifmmode
    \@latex@error{Math\space alphabet\space identifier\space
          \noexpand#1is\space undefined\space in\space math\space
           version\space `\math@version'}%
        {Your\space requested\space math\space alphabet\space
         is\space undefined\space in\space the\space current\space
          math\space version.^^JCheck\space the\space spelling\space
          or\space use\space the\space \noexpand\SetMathAlphabet\space
          command.}
     \fi}
%    \end{macrocode}
% \end{macro}
%
%
% \begin{macro}{\new@mathgroup}
% \begin{macro}{\mathgroup}
%    We also give a new name to |\newfam| and |\fam|
%    to avoid verbal confusion
%    (see the introduction).\footnote{For the same reason
%                           it seems advisable to
%                           {\ttfamily\bslash let\bslash fam} and
%                           {\ttfamily\bslash newfam}
%                           equal to {\ttfamily\bslash relax},
%                           but this is commented out to retain
%                           compatibility to existing style files.}
% \changes{v1.2e}{1990/01/28}{\cs{newfam} let to \cs{new@mathgroup}.}
% \changes{v3.0a}{1995/05/24}
%         {(DPC) No need to redefine \cs{newfam} as not outer}
% \changes{v3.0b}{1995/05/27}
%         {(FMi) But a need to define \cs{new@mathgroup}}
% \changes{v3.0y}{2014/12/30}
%         {move allocation to ltplain.}
%    \begin{macrocode}
%\def\new@mathgroup{\alloc@8\mathgroup\chardef\sixt@@n}
\let\mathgroup\fam
%\let\newfam\new@mathgroup
\@onlypreamble\new@mathgroup
%    \end{macrocode}
% \end{macro}
% \end{macro}
%
%
% \section{Macros for setting up the tables}
%
%  \begin{macro}{\DeclareFontShape}
%
% \changes{v1.9a}{1992/07/26}{Introduced \cs{DeclareFontShape}}
%    The macro |\DeclareFontShape| takes $6$ arguments:
%    \begin{macrocode}
\def\DeclareFontShape{\begingroup
%    \end{macrocode}
%    First we restore the catcodes of all characters used in the syntax.
% \changes{v2.1e}{1994/02/24}
%     {Separate restoration of catcodes for fd cmds}
%    \begin{macrocode}
   \nfss@catcodes
%    \end{macrocode}
%    We use |\expandafter| |\endgroup| to restore catcode in case
%    something goes wrong with the argument parsing (suggested by Tim
%    Van Zandt)
%    \begin{macrocode}
   \expandafter\endgroup
   \DeclareFontShape@}
%    \end{macrocode}
%  \end{macro}
%
%
%
%  \begin{macro}{\DeclareFontShape@}
% \changes{v2.1c}{1994/02/07}{revert catcode settings earlier}
% \changes{v3.0m}{1995/11/01}
%      {(DPC) Test for \cs{relax} not \cs{undefined}, internal/1933}
%    \begin{macrocode}
%</2ekernel>
%<*2ekernel|latexrelease>
%<latexrelease>\IncludeInRelease{2020/02/02}%
%<latexrelease>                 {\DeclareFontShape@}{Maybe drop one m}%
\def\DeclareFontShape@#1#2#3#4#5#6{%
   \expandafter\ifx\csname #1+#2\endcsname\relax
     \@latex@error{Font family `#1+#2' unknown}\@eha
   \else
%    \end{macrocode}
%    If the series value is incorrectly specified with an extra ``m'',
%    e.g., ``mc'' instead of just ``c'', drop the surplus ``m'' but
%    keep the ``m'' if it is by its own. In that case also issue a
%    warning that the declaration needs correction.
% \changes{v3.2f}{2020/02/24}{Drop surplus ``m'' in series when
%                             defining fontshape (gh/289)}
% \changes{v3.2g}{2020/02/27}{Only ``m'' if the series value is a member
%     of a fixed list  and issue warning if doing it (gh/293)}
%
%    For this we compare the given value \verb=#3= with one where we
%    may have dropped an ``m''. If nothing has changes,
%    fine. Otherwise there was a wrong value which is now corrected in
%    \cs{reserved@b} so we use that and also issue a warning.
% \changes{v3.2j}{2021/06/09}{Improve information message}
%    \begin{macrocode}
     \edef\reserved@a{#3}%
     \series@maybe@drop@one@m\reserved@a\reserved@b
     \ifx\reserved@a\reserved@b\else
       \@latex@note{Font shape #1/#2/#3/#4 has incorrect series
         value `#3'.\MessageBreak It should not contain an `m'!
         Please correct it.\MessageBreak Found}%
     \fi
     \expandafter
       \xdef\csname#1/#2/\reserved@b/#4\endcsname
                 {\expandafter\noexpand\csname #5\endcsname}%
%
%    \end{macrocode}
%    Most of the time \verb=#6= is empty so using \cs{let} to
%    \cs{@empty} saves on space compared to using \cs{def}. That's
%    really one of the old space saving techniques and probably not
%    necessary these days.
%    \begin{macrocode}
     \def\reserved@a{#6}%
     \global
     \expandafter\let\csname#5\expandafter\endcsname
        \ifx\reserved@a\@empty
          \@empty
        \else
          \reserved@a
        \fi
   \fi
  }
%    \end{macrocode}
%
%    \begin{macrocode}
%</2ekernel|latexrelease>
%<latexrelease>\EndIncludeInRelease
%<latexrelease>\IncludeInRelease{0000/00/00}%
%<latexrelease>                 {\DeclareFontShape@}{Maybe drop one m}%
%<latexrelease>
%<latexrelease>\def\DeclareFontShape@#1#2#3#4#5#6{%
%<latexrelease>   \expandafter\ifx\csname #1+#2\endcsname\relax
%<latexrelease>     \@latex@error{Font family `#1+#2' unknown}\@eha
%<latexrelease>   \else
%<latexrelease>     \expandafter
%<latexrelease>       \xdef\csname#1/#2/#3/#4\endcsname{\expandafter\noexpand
%<latexrelease>                                   \csname #5\endcsname}%
%<latexrelease>     \def\reserved@a{#6}%
%<latexrelease>     \global
%<latexrelease>     \expandafter\let\csname#5\expandafter\endcsname
%<latexrelease>        \ifx\reserved@a\@empty
%<latexrelease>          \@empty
%<latexrelease>        \else
%<latexrelease>          \reserved@a
%<latexrelease>        \fi
%<latexrelease>   \fi
%<latexrelease>  }
%<latexrelease>\EndIncludeInRelease
%<*2ekernel>
%    \end{macrocode}
%  \end{macro}
%
%
%  \begin{macro}{\DeclareFixedFont}
%    Define a direct font switch that avoids all overhead.
% \changes{v2.1u}{1994/11/06}{Renamed
%                        \cs{every@size} to \cs{every@math@size}.}
%    \begin{macrocode}
\def\DeclareFixedFont#1#2#3#4#5#6{%
   \begingroup
      \math@fontsfalse
      \every@math@size{}%
      \fontsize{#6}\z@
      \usefont{#2}{#3}{#4}{#5}%
      \global\expandafter\let\expandafter#1\the\font
   \endgroup
  }
%    \end{macrocode}
%  \end{macro}
%
%
%
%
%  \begin{macro}{\do@subst@correction}
%
%    \begin{macrocode}
\def\do@subst@correction{%
       \xdef\subst@correction{%
          \font@name
          \global\expandafter\font
            \csname \curr@fontshape/\f@size\endcsname
            \noexpand\fontname\font
           \relax}%
%    \end{macrocode}
%    Calling |\subst@correction| after the current group means calling
%    it after we have loaded the substitution font which is done
%    inside a group.
%    \begin{macrocode}
       \aftergroup\subst@correction
}
%    \end{macrocode}
%  \end{macro}
%
% \begin{macro}{\DeclareFontFamily}
%    \begin{macrocode}
\def\DeclareFontFamily#1#2#3{%
%    \end{macrocode}
%    If we want fast checking for the encoding scheme we can just
%    check for |\T@..| being defined.
%    \begin{macrocode}
% \@tempswafalse
% \def\reserved@b{#1}%
% \def\cdp@elt##1##2##3##4{\def\reserved@c{##1}%
%      \ifx\reserved@b\reserved@c \@tempswatrue\fi}%
% \cdp@list
% \if@tempswa
 \@ifundefined{T@#1}%
    {%
     \@latex@error{Encoding scheme  `#1' unknown}\@eha
    }%
    {%
%    \end{macrocode}
%    Now we have to define the macro |\|\meta{\#1}|+|\meta{\#2}
%    to contain |#3|. But since most of the time |#3| will be empty
%    we use |\let| in a tricky way rather than a simple |\def| since
%    this will save internal memory.
%    We store the argument |#3| in a temporary macro
%    |\reserved@a|.
%    \begin{macrocode}
     \def\reserved@a{#3}%
%    \end{macrocode}
%    We compare |\reserved@a| with |\@empty|
%    If these two are the same we |\let| the `extra'
%    macro equal to |\@empty| which is not the same a doing a |\let|
%    to |\reserved@a| --- the latter would blow one extra memory
%    location rather  then reusing the one from |\@empty|.
%    \begin{macrocode}
     \global
     \expandafter\let\csname #1+#2\expandafter\endcsname
            \ifx \reserved@a\@empty
              \@empty
            \else \reserved@a
            \fi
    }%
}
%    \end{macrocode}
% \end{macro}
%
%
% \begin{macro}{\cdp@list}
%    We initialize the code page list to be empty.
%    \begin{macrocode}
\let\cdp@list\@empty
\@onlypreamble\cdp@list
%    \end{macrocode}
% \end{macro}
%
% \begin{macro}{\cdp@elt}
%    \begin{macrocode}
\let\cdp@elt\relax
\@onlypreamble\cdp@elt
%    \end{macrocode}
% \end{macro}
%
%
%
%
% \begin{macro}{\DeclareFontEncoding}
%    \begin{macrocode}
\def\DeclareFontEncoding{%
%    \end{macrocode}
%    First we start with ignoring all blanks and newlines since every
%    surplus space in the second or third argument will come out in
%    a weird place in the document.
% \changes{v2.1c}{1994/02/07}{revert catcode settings earlier}
% \changes{v2.1t}{1994/10/19}{Add missing \cs{relax}.}
% \changes{v2.1z}{1994/12/06}{use \cs{nfss@catcodes}}
%    \begin{macrocode}
   \begingroup
   \nfss@catcodes
   \expandafter\endgroup
   \DeclareFontEncoding@}
\@onlypreamble\DeclareFontEncoding
%    \end{macrocode}
% \changes{v2.0g}{1993/09/15}
%      {Corrected: \cs{default@T} to \cs{default@M}.}
%    \begin{macrocode}
\def\DeclareFontEncoding@#1#2#3{%
  \expandafter
  \ifx\csname T@#1\endcsname\relax
     \def\cdp@elt{\noexpand\cdp@elt}%
     \xdef\cdp@list{\cdp@list\cdp@elt{#1}%
                    {\default@family}{\default@series}%
                    {\default@shape}}%
%    \end{macrocode}
%    To support encoding dependent commands (like accents) we
%    initialise the command
% |\|\meta{encoding}|-cmd| to be |\@changed@cmd|.
%    (See \texttt{ltoutenc.dtx} for details.)
% \changes{v2.1l}{1994/05/13}{Init encoding change command}
% \changes{v2.1n}{1994/05/14}{Only init enc change cmd when
%                             new encoding}
% \changes{v2.1n}{1994/05/14}{Log if encoding is redeclared}
%    \begin{macrocode}
     \expandafter\let\csname#1-cmd\endcsname\@changed@cmd
  \else
     \@font@info{Redeclaring font encoding #1}%
  \fi
%    \end{macrocode}
%
%    \begin{macrocode}
  \global\@namedef{T@#1}{#2}%
  \global\@namedef{M@#1}{\default@M#3}%
%    \end{macrocode}
%    Keep a record of the last encoding being declared:
% \changes{v3.0w}{1999/01/06}{Added \cs{LastDeclaredEncoding} to
%   support cyrillic integration (pr/2988)}
%    \begin{macrocode}
  \xdef\LastDeclaredEncoding{#1}%
  }
\@onlypreamble\DeclareFontEncoding@
%    \end{macrocode}
% \end{macro}
%
%
% \begin{macro}{\LastDeclaredEncoding}
% \changes{v3.0w}{1999/01/06}{Added \cs{LastDeclaredEncoding} to
%   support cyrillic integration (pr/2988)}
%    The last encoding being declared by |\DeclareFontEncoding|.
%    \begin{macrocode}
\def\LastDeclaredEncoding{}
%    \end{macrocode}
% \end{macro}
%
% \begin{macro}{\DeclareFontSubstitution}
%    \begin{macrocode}
\def\DeclareFontSubstitution#1#2#3#4{%
  \expandafter
  \ifx\csname T@#1\endcsname\relax
    \@latex@error{Encoding scheme  `#1' unknown}\@eha
  \else
    \begingroup
%    \end{macrocode}
% \changes{v3.0x}{2002/10/02}{Adding \cs{LastDeclaredEncoding}
%   introduced a bug as on some occasions that macro name was stored
%   in the internal lists instead of the actual encoding. (pr/3459)}
%    We loop through the |\cdp@list| and rebuild it anew in |\toks@|
%    thereby replacing the defaults for the encoding in question with
%    the new defaults. It is important to store the encoding to test
%    against expanded in |\reserved@a| since it might just be
%    |\LastDeclaredEncoding| that is passed as |#1|.
%    \begin{macrocode}
       \edef\reserved@a{#1}%
       \toks@{}%
       \def\cdp@elt##1##2##3##4{%
          \def\reserved@b{##1}%
          \ifx\reserved@a\reserved@b
%    \end{macrocode}
%    Here we use the new defaults but we use |##1| (i.e., the encoding
%    name already stored previously) since we know that it is expanded.
%    \begin{macrocode}
             \addto@hook\toks@{\cdp@elt{##1}{#2}{#3}{#4}}%
          \else
%    \end{macrocode}
%    If |\reserved@a| and |\reserved@b| differ then we simply copy
%    from the old list to the new.
%    \begin{macrocode}
             \addto@hook\toks@{\cdp@elt{##1}{##2}{##3}{##4}}%
          \fi}%
        \cdp@list
        \xdef\cdp@list{\the\toks@}%
    \endgroup
    \global
    \@namedef{D@#1}{%
           \def\default@family{#2}%
           \def\default@series{#3}%
           \def\default@shape{#4}%
           }%
  \fi
 }
\@onlypreamble\DeclareFontSubstitution
%    \end{macrocode}
% \end{macro}
%
%
% \begin{macro}{\DeclareFontEncodingDefaults}
%    \begin{macrocode}
\def\DeclareFontEncodingDefaults#1#2{%
  \ifx\relax#1\else
    \ifx\default@T\@empty\else
      \@font@info{Overwriting encoding scheme text defaults}%
    \fi
    \gdef\default@T{#1}%
  \fi
  \ifx\relax#2\else
    \ifx\default@M\@empty\else
      \@font@info{Overwriting encoding scheme math defaults}%
    \fi
    \gdef\default@M{#2}%
  \fi
}
\@onlypreamble\DeclareFontEncodingDefaults
%    \end{macrocode}
%  \end{macro}
%
%
% \begin{macro}{\default@T}
% \begin{macro}{\default@M}
%    \begin{macrocode}
\let\default@T\@empty
\let\default@M\@empty
%    \end{macrocode}
% \end{macro}
% \end{macro}
%
%
% \begin{macro}{\DeclarePreloadSizes}
%    \begin{macrocode}
\def\DeclarePreloadSizes#1#2#3#4#5{%
 \@ifundefined{T@#1}%
   {\@latex@error{Encoding scheme  `#1' unknown}\@eha}%
   {%
%    \end{macrocode}
%    Don't know at the moment what this group here does!
%    \begin{macrocode}
   \begingroup
%    \end{macrocode}
%    We define a macro |\reserved@f|\footnote{We cannot use
%                                      {\ttfamily\bslash @tempa}
%                                      since it is needed in
%                                     {\ttfamily\bslash pickup@font}.}
%    that grabs the next \emph{size} and loads the corresponding
%    font.
%    This is done by delimiting |\reserved@f|'s only argument by the
%    \textsf{token} |,| (comma).
%    \begin{macrocode}
    \def\reserved@f##1,{%
%    \end{macrocode}
%    The end of the list will be detected when there are no more
%    elements, i.e.\ when |\reserved@f|'s argument is empty.
%    The trick used here is explained in Appendix~D of the
%    \TeX{}book: if the argument is empty the |\if|
%    will select the first clause and |\let| |\reserved@f|
%    equal to |\relax|.
%    (We use the |>| character here since it cannot appear
%    in font file names.)
% \changes{v1.2j}{1990/06/24}{Missing percent added.}
%    \begin{macrocode}
        \if>##1>%
          \let\reserved@f\relax
        \else
%    \end{macrocode}
%    Otherwise, we define |\font@name| appropriately and
%    call |\pickup@font| to do the work.
%    Note that the requested |\curr@fontshape|
%    combination must have been defined, or you will get an error.
%    The definition of |\font@name| is carried out globally
%    to be consistent with the rest of the code in this file.
%    \begin{macrocode}
          \xdef\font@name{\csname#1/#2/#3/#4/##1\endcsname}%
          \pickup@font
%    \end{macrocode}
%    Now we forget the name of the font just loaded.
%    More precisely, we set the corresponding control sequence
%    to |\relax|.  This means that later on, when the font
%    is first used, the macro |\define@newfont| is called
%    again to execute the `extra' macro for this font.
% \changes{v1.2d}{1990/01/27}{Font identifier set to \cs{relax}.}
%    \begin{macrocode}
          \global\expandafter\let\font@name\relax
        \fi
%    \end{macrocode}
%    Finally we call |\reserved@f| again to process the next
%    \emph{size}. If |\reserved@f| was |\let| equal to |\relax|
%    this will end the macro.
%    \begin{macrocode}
        \reserved@f}%
%    \end{macrocode}
%    We finish with reinserting the list of sizes after the
%    |\reserved@f|
%    macro and appending an empty element so that the end of the list
%    is recognized properly.
%    \begin{macrocode}
     \reserved@f#5,,%
   \endgroup
   }%
}
\@onlypreamble\DeclarePreloadSizes
%    \end{macrocode}
% \end{macro}
%
%
%
% \begin{macro}{\ifmath@fonts}
%    We need a switch to decide if we have to switch math fonts.
%    For this purpose we provide |\ifmath@fonts|
%    that can be set to true or false by the |\S@...| macros
%    depending on if math fonts
%    are provided for this size or not.
%    The default is of course to switch all fonts.
%    \begin{macrocode}
\newif\ifmath@fonts \math@fontstrue
%    \end{macrocode}
% \end{macro}
%
%
%
% \begin{macro}{\DeclareMathSizes}
% \begin{macro}{\DeclareMathSizes*}
%    |\DeclareMathSizes| takes the text size, math text size, math
%    script size, and math scriptscript size as arguments and defines
%    the right |\S@|\dots{} macro.
%
%    \begin{macrocode}
\def\DeclareMathSizes{%
  \@ifstar{\@DeclareMathSizes\math@fontsfalse}%
          {\@DeclareMathSizes{}}}
\@onlypreamble\DeclareMathSizes
%    \end{macrocode}
% \end{macro}
% \end{macro}
%
% \begin{macro}{\@DeclareMathSizes}
% \changes{v3.0p}{1996/07/26}{use faster \cs{if} test}
% This modification by Michael J. Downes on comp.text.tex on 2002/10/17
% allows the user to have settings such as\\
%  |\DeclareMathSizes{9.5dd}{9.5dd}{7.4dd}{6.6dd}|.
% \changes{v3.0y}{2015/01/11}{Allow arbitrary units (latexrelease)}
%    \begin{macrocode}
%</2ekernel>
%<latexrelease>\IncludeInRelease{2015/01/01}{\@DeclareMathSizes}%
%<latexrelease>                 {Arbitrary units in \DeclareMathSizes}%
%<*2ekernel|latexrelease>
\def\@DeclareMathSizes #1#2#3#4#5{%
  \@defaultunits\dimen@ #2pt\relax\@nnil
  \if $#3$%
    \expandafter\let\csname S@\strip@pt\dimen@\endcsname\math@fontsfalse
  \else
    \@defaultunits\dimen@ii #3pt\relax\@nnil
    \@defaultunits\@tempdima #4pt\relax\@nnil
    \@defaultunits\@tempdimb #5pt\relax\@nnil
    \toks@{#1}%
    \expandafter\xdef\csname S@\strip@pt\dimen@\endcsname{%
      \gdef\noexpand\tf@size{\strip@pt\dimen@ii}%
      \gdef\noexpand\sf@size{\strip@pt\@tempdima}%
      \gdef\noexpand\ssf@size{\strip@pt\@tempdimb}%
      \the\toks@
    }%
  \fi
}%
%</2ekernel|latexrelease>
%<latexrelease>\EndIncludeInRelease
%<latexrelease>\IncludeInRelease{0000/00/00}{\@DeclareMathSizes}%
%<latexrelease>                 {Arbitrary units in \DeclareMathSizes}%
%<latexrelease>\def\@DeclareMathSizes#1#2#3#4#5{%
%<latexrelease>    \@defaultunits\dimen@#2pt\relax\@nnil
%<latexrelease>    \if$#3$%
%<latexrelease>      \expandafter \let
%<latexrelease>        \csname S@\strip@pt\dimen@\endcsname
%<latexrelease>        \math@fontsfalse
%<latexrelease>    \else
%<latexrelease>      \expandafter \gdef
%<latexrelease>      \csname S@\strip@pt\dimen@\endcsname
%<latexrelease>            {\gdef\tf@size{#3}\gdef\sf@size{#4}%
%<latexrelease>                             \gdef\ssf@size{#5}%
%<latexrelease>             #1%
%<latexrelease>                             }%
%<latexrelease>    \fi}%
%<latexrelease>\EndIncludeInRelease
%<*2ekernel>
%    \end{macrocode}
%
%    \begin{macrocode}
\@onlypreamble\@DeclareMathSizes
%    \end{macrocode}
% \end{macro}
%
%
%
% \section{Selecting a new font}
%
% \subsection{Macros for the user}
%
%
% \begin{macro}{\fontencoding}
% \changes{v2.1j}{1994/05/12}{Use \cs{DeclareRobustCommand}.}
% \begin{macro}{\f@encoding}
% As we said in the introduction a font is described by four parameters.
% We first define macros to specify the wanted \emph{family},
% \emph{series}, or \emph{shape}.
% These are simply recorded in internal macros
% |\f@family|, |\f@series|, and |\f@shape|, resp.
% We use |\edef|'s so that the arguments can also be macros.
%    \begin{macrocode}
\DeclareRobustCommand\fontencoding[1]{%
    \expandafter\ifx\csname T@#1\endcsname\relax
      \@latex@error{Encoding scheme `#1' unknown}\@eha
    \else
      \edef\f@encoding{#1}%
      \ifx\cf@encoding\f@encoding
%    \end{macrocode}
%    If the new encoding is the same as the old
%    encoding we have nothing to do.
%    However, in case we had a sequence of several encoding changes
%    without a |\selectfont| in-between we can save processing by
%    making sure that |\enc@update| is |\relax|.
%    \begin{macrocode}
        \let\enc@update\relax
      \else
%    \end{macrocode}
%    If current and new encoding differ we define the macro
%    |\enc@update|
%    to contain all updates necessary at |\selectfont| time.
%    \begin{macrocode}
        \let\enc@update\@@enc@update
      \fi
    \fi
}
%    \end{macrocode}
% \end{macro}
% \end{macro}
%
% \begin{macro}{\@@enc@update}
%    \begin{macrocode}
\def\@@enc@update{%
%    \end{macrocode}
%    When |\@@enc@update| is executed |\f@encoding| holds the encoding
%    name
%    for the new encoding and |\cf@encoding| the name of the last active
%    encoding.
%
%    We start by setting the init command for encoding dependent
%    macros to |\@changed@cmd|.
%    \begin{macrocode}
          \expandafter
          \let
            \csname\cf@encoding -cmd\endcsname
            \@changed@cmd
%    \end{macrocode}
%    Then we turn the one for the new encoding to |\@current@cmd| (see
%    \texttt{ltoutenc.dtx} for further explanations).
%    \begin{macrocode}
          \expandafter
          \let
            \csname\f@encoding-cmd\endcsname
            \@current@cmd
%    \end{macrocode}
%    We execute the default settings |\default@T|, followed by the one
%    for the new encoding.
%    \begin{macrocode}
        \default@T
        \csname T@\f@encoding\endcsname
%    \end{macrocode}
%    Finally we change the default substitution values, disable
%    |\enc@update| and make |\f@encoding| officially the current
%    encoding.
%    \begin{macrocode}
        \csname D@\f@encoding\endcsname
        \let\enc@update\relax
        \let\cf@encoding\f@encoding
}
%    \end{macrocode}
% \end{macro}
%
%
%  \begin{macro}{\enc@update}
% \changes{v2.1m}{1994/05/14}{Macro added}
%    The default action in |\selectfont| is to do nothing.
%    \begin{macrocode}
\let\enc@update\relax
%    \end{macrocode}
%  \end{macro}
%
% \begin{macro}{\fontfamily}
% \changes{v2.1j}{1994/05/12}{Use \cs{DeclareRobustCommand}.}
% \begin{macro}{\f@family}
% \begin{macro}{\fontseries}
% \changes{v2.1j}{1994/05/12}{Use \cs{DeclareRobustCommand}.}
% \begin{macro}{\f@series}
% \begin{macro}{\fontshape}
% \changes{v2.1j}{1994/05/12}{Use \cs{DeclareRobustCommand}.}
% \changes{v2.1y}{1994/11/30}{Use \cs{@current@cmd} in
%    \cs{@@enc@update}.  ASAJ.}
% \begin{macro}{\f@shape}
%    \begin{macrocode}
\DeclareRobustCommand\fontfamily[1]{\edef\f@family{#1}}
%    \end{macrocode}
%    There are now defined later (and differently).
%    \begin{macrocode}
%\DeclareRobustCommand\fontseries[1]{\edef\f@series{#1}}
%\DeclareRobustCommand\fontshape [1]{\edef\f@shape{#1}}
%    \end{macrocode}
% \end{macro}
% \end{macro}
% \end{macro}
% \end{macro}
% \end{macro}
% \end{macro}
%
%
%
%  \begin{macro}{\usefont}
%    Some handy abbreviation if you want to get some particular font
%    in the current size. If also the size should change one has to
%    issue a |\fontsize| command first.
%
%    \cs{fontencoding} needs to do some setup work so we call that,
%    but instead of calling \cs{fontfamily}, \cs{fontseries} and
%    \cs{fontshape} it earlier versions of this code did, we now set
%    \cs{f@family}, etc.\ directly.  If we would call \cs{fontseries}
%    or \cs{fontshape} as it
%    was done in the past, they would now interact with the existing
%    series and shape which is not desired if we intend to use an
%    explicit font shape!
% \changes{v3.2e}{2019/12/17}{Don't call \cs{fontseries} or \cs{fontshape}}
% \changes{v3.2h}{2020/12/10}{Drop ``m'' if the series value is a member
%     of a fixed list  and issue warning if doing it (gh/453)}
%    \begin{macrocode}
%</2ekernel>
%<*2ekernel|latexrelease>
%<latexrelease>\IncludeInRelease{2021/06/01}%
%<latexrelease>                 {\usefont}{Force font face}%
\DeclareRobustCommand\usefont[4]{\fontencoding{#1}%
   \edef\f@family{#2}%
   \set@target@series{#3}%
   \edef\f@shape{#4}%
%    \end{macrocode}
%    Any earlier \cs{fontseries}, etc.\ should be canceled and we
%    should switch unconditionally to the requested font face so we
%    drop any  code that may have been stored in
%    \cs{delayed@f@adjustment}.
% \changes{v3.2i}{2021/04/26}{Unconditionally switch to the requested
%    font face (gh/444)}
%    \begin{macrocode}
   \let\delayed@f@adjustment\@empty
   \selectfont
   \ignorespaces}
%</2ekernel|latexrelease>
%<latexrelease>\EndIncludeInRelease
%<latexrelease>\IncludeInRelease{2020/02/02}%
%<latexrelease>                 {\usefont}{Drop m in usefont}%
%<latexrelease>
%<latexrelease>\DeclareRobustCommand\usefont[4]{\fontencoding{#1}%
%<latexrelease>   \edef\f@family{#2}%
%<latexrelease>   \set@target@series{#3}%
%<latexrelease>   \edef\f@shape{#4}\selectfont
%<latexrelease>   \ignorespaces}
%<latexrelease>
%<latexrelease>\EndIncludeInRelease
%<latexrelease>\IncludeInRelease{0000/00/00}%
%<latexrelease>                 {\usefont}{Drop m in usefont}%
%<latexrelease>
%<latexrelease>\DeclareRobustCommand\usefont[4]{\fontencoding{#1}%
%<latexrelease>   \edef\f@family{#2}%
%<latexrelease>   \edef\f@series{#3}%
%<latexrelease>   \edef\f@shape{#4}\selectfont
%<latexrelease>   \ignorespaces}
%<latexrelease>
%<latexrelease>\EndIncludeInRelease
%<*2ekernel>
%    \end{macrocode}
%  \end{macro}
%
%
%  \begin{macro}{\linespread}
% \changes{v2.1j}{1994/05/12}{New macro}
% \changes{v2.1p}{1994/05/16}{Remove surplus braces}
%    The command |\linespread| changes the current |\baselinestretch|
%    by calling |\set@fontsize|. The values for |\f@size| and
%    |\f@baselineskip| will be left unchanged.
%    \begin{macrocode}
\DeclareRobustCommand\linespread[1]
   {\set@fontsize{#1}\f@size\f@baselineskip}
%    \end{macrocode}
%  \end{macro}
%
%
%  \begin{macro}{\fontsize}
% \changes{v2.1j}{1994/05/12}{Redefined to use \cs{set@fontsize}}
% \changes{v2.1p}{1994/05/16}{Pass \cs{baselinstretch} not
%    \cs{f@linespread}}
%    We also define a macro that allows to specify a size.  In this
%    case, however, we also need the value of |\baselineskip|. As the
%    first argument to |\set@fontsize| we pass the current value of
%    |\baselinestretch|. This will either match the internal value (in
%    which case nothing changes) or it will be an updated value due to
%    a user change of that macro using |\renewcommand|. If we would
%    pass the internal |\f@linespread| such a change would be
%    effectively overwritten by a size change.
%    \begin{macrocode}
\DeclareRobustCommand\fontsize[2]
   {\set@fontsize\baselinestretch{#1}{#2}}
%    \end{macrocode}
%  \end{macro}
%
%
% \changes{v2.1n}{1994/05/14}{Set defaults for all \cs{f@...}}
%
%  \begin{macro}{\f@linespread}
% \changes{v2.1j}{1994/05/12}{New macro}
%    This macro holds the current internal value for
%    |\baselinestretch|.
%    \begin{macrocode}
\let\f@family\@empty
\let\f@series\@empty
\let\f@shape\@empty
\let\f@size\@empty
\let\f@baselineskip\@empty
\let\f@linespread\@empty
%    \end{macrocode}
%  \end{macro}
%
%  \begin{macro}{\cf@encoding}
% \changes{v2.1u}{1994/11/06}{New macro}
%    \begin{macrocode}
\let\f@encoding\@empty
\let\cf@encoding\@empty
%    \end{macrocode}
%  \end{macro}
%
%
%
%  \begin{macro}{\@defaultunits}
%
%    The function |\@defaultunits| when wrapped around a dimen or skip
%    assignment supplies default units. Usage:
%
%      |\@defaultunits\dimen@=#1pt\relax\@nnil|
%
%    Note: the |\relax| is *important*.  Other units can be substituted
%    for the `pt' if desired.
%
%    We use |\remove@to@nnil| as an auxiliary macros for
%    |\@defaultunits|. It just has to gobble the supplied default unit
%    `pt' or whatever, if it wasn't used in the assignment.
%    \begin{macrocode}
\def\@defaultunits{\afterassignment\remove@to@nnil}
%    \end{macrocode}
%  \end{macro}
%
%  \begin{macro}{\strip@pt}
%  \begin{macro}{\rem@pt}
%  This macro strips the characters |pt| produced by using |\the|
%  on a dimen register.
%    \begin{macrocode}
\begingroup
  \catcode`P=12
  \catcode`T=12
  \lowercase{
    \def\x{\def\rem@pt##1.##2PT{##1\ifnum##2>\z@.##2\fi}}}
  \expandafter\endgroup\x
\def\strip@pt{\expandafter\rem@pt\the}
%    \end{macrocode}
%  \end{macro}
%  \end{macro}
%
%
% \begin{macro}{\mathversion}
% \changes{v1.0h}{1989/04/29}{Test if version defined added.}
% \changes{v1.0m}{1989/09/14}
%     {Corrected typo: \cs{endscname} to \cs{endcsname}.}
% \changes{v2.1j}{1994/05/12}{Use \cs{DeclareRobustCommand}.}
% \begin{macro}{\math@version}
%    |\mathversion| takes the math \emph{version} name as
%    argument, defines |\math@version| appropriately and switches
%    to the font selected
%    forcing a call to |\glb@settings| if the \emph{version} is
%    known to the system.
% \changes{v1.0p}{1989/11/14}{Math version prefix `mv@' added.}
% \changes{v1.0r}{1989/11/22}
%      {\cs{def} $\to$ \cs{edef} for \cs{math@version}.}
% \changes{v1.2g}{1990/02/16}{\cs{@nomath} added.}
% \changes{v2.1a}{1994/01/17}{New math font setup}
% \changes{v3.2l}{2023/04/01}{Reset frozen mathversion gh/1028}
%    \begin{macrocode}
\DeclareRobustCommand\mathversion[1]
         {\@nomath\mathversion
          \expandafter\ifx\csname mv@#1\endcsname\relax
          \@latex@error{Math version `#1' is not defined}\@eha\else
%    \end{macrocode}
%  If there has been a frozen math version reset locally. See GH 1028.
%    \begin{macrocode}
          \ifcsname mv@\math@version @frozen\endcsname
            \expandafter\let
            \csname mv@\math@version @frozen\expandafter\endcsname
            \csname mv@\math@version\endcsname
          \fi
%    \end{macrocode}
%    \begin{macrocode}
          \edef\math@version{#1}%
%    \end{macrocode}
%    We need to force a math font setup both now and at the point
%    where we return to the previous math version.
%    Forcing a math font setup can simply be done by setting
%    |\glb@currsize| to an invalid value since this will trigger the
%    setup when the formula starts.
%    \begin{macrocode}
          \gdef\glb@currsize{}%
%    \end{macrocode}
%    When the scope of the current |\mathversion| ends we need to
%    restore the old setup. However this time we need to force it
%    directly at least if we are inside math, otherwise we could wait.
%    Another way to enhance this code here is todo the setting only if
%    the version really has changed after all. This might be
%    interesting in case of \texttt{amstext} and \texttt{boldsymbol}.
% \changes{v2.1b}{1994/01/25}{Corrections for math setup}
%    \begin{macrocode}
          \aftergroup\glb@settings
          \fi}
%    \end{macrocode}
% \end{macro}
% \end{macro}
%
%
% If \TeX{} would support a hook just before the end of a formula
% (opposite of |\everymath| so to speak) the implementation of the
% algorithm would be much simpler because in that case we would set up
% the correct math fonts at this point without having to worry about
% incorrect settings due to nesting. The same would be true if in
% \LaTeX{} the use of |$| (as the primitive \TeX{} command) would be
% impossible and instead only a higher-level interface would be
% available. Note that this does not mean that a |$| couldn't be the
% short-hand for starting and stopping that higher-level interface, it
% only means that the direct \TeX{} function must be hidden.
%
% Anyway, since we don't have this and won't have it in \LaTeXe{} we
% need to implement it in a somewhat slower way.
%
%
% We test for the current math font setup on entry of a formula,
% i.e., on the
% hooks |\everymath| and |\everydisplay|. But since these hooks may
% contain user data we provide ourselves with an internal version of
% these hooks which stays frozen.
%
%
% \begin{macro}{\frozen@everymath}
% \changes{v2.1a}{1994/01/17}{New math font setup}
% \begin{macro}{\frozen@everydisplay}
%    New internal names for |\everymath| and |\everydisplay|.
%    \begin{macrocode}
\let\frozen@everymath\everymath
\let\frozen@everydisplay\everydisplay
%    \end{macrocode}
% \end{macro}
% \end{macro}
%
% \begin{macro}{\everymath}
% \changes{v2.1a}{1994/01/17}{New math font setup}
% \begin{macro}{\everydisplay}
% \changes{v2.1a}{1994/01/17}{New math font setup}
%    Now we provide now user hooks that will be called in the
%    frozen internals.
%    \begin{macrocode}
\newtoks\everymath
\newtoks\everydisplay
%    \end{macrocode}
% \end{macro}
% \end{macro}
%
%
%

%
% \begin{macro}{\frozen@everydisplay}
% \changes{v2.1a}{1994/01/17}{New math font setup}
%    Now we define the behaviour of the frozen hooks: first
%    check the math setup then call the user hook.
%
%    The check code may push tokens after the math formula with
%    \cs{aftergroup} and they would prevent a \verb=$$= from  dropping
%    following spaces. We therefore use a switch to be set as the
%    first thing after the group so that following code can determine
%    if there was a display or some inline math (in  the latter case
%    we better not drop spaces).
%    After setting the switch we also have to place \cs{ignorespaces}
%    because setting the switch may be the only thing that happens
%    after the display.
% \changes{v3.2k}{2022/07/04}{Ignore spaces if necessary (gh/886)}
%    The issue with handling of spaces was found in 2022, but it is
%    really a bug fix for the code added in 2021/11.
%    \begin{macrocode}
%</2ekernel>
%<latexrelease>\IncludeInRelease{2021/11/15}
%<latexrelease>  {\frozen@everydisplay}{Handle spaces after math}%
%<*2ekernel|latexrelease>
\frozen@everydisplay = {%
  \aftergroup\@ignoretrue  \aftergroup\ignorespaces
  \check@mathfonts
  \the\everydisplay}
%    \end{macrocode}
% \end{macro}


% \begin{macro}{\frozen@everymath}
% \changes{v2.1a}{1994/01/17}{New math font setup}
% \changes{v3.2k}{2022/07/04}{Ignore spaces if necessary (gh/886)}
%    The frozen code for inline math is similar, except that here we
%    do not want to drop following spaces.
%    \begin{macrocode}
\frozen@everymath = {%
  \aftergroup\@ignorefalse
  \check@mathfonts
  \the\everymath}
%    \end{macrocode}
% \end{macro}
%
%    \begin{macrocode}
%</2ekernel|latexrelease>
%<latexrelease>\EndIncludeInRelease
%<latexrelease>\IncludeInRelease{2020/10/01}
%<latexrelease>  {\frozen@everydisplay}{Handle spaces after math}%
%<latexrelease>
%<latexrelease>\frozen@everydisplay = {\check@mathfonts
%<latexrelease>                        \the\everydisplay}
%<latexrelease>\frozen@everymath = {\check@mathfonts
%<latexrelease>                     \the\everymath}
%<latexrelease>
%<latexrelease>\EndIncludeInRelease
%<*2ekernel>
%    \end{macrocode}
%
%
%
%
%
%
% \changes{v3.0q}{1996/07/27}{\cs{if@inmath} switch removed}
%
%  \begin{macro}{\curr@math@size}
% \changes{v2.1a}{1994/01/17}{New math font setup}
%    This holds locally the current math size.
%    \begin{macrocode}
\let\curr@math@size\@empty
%    \end{macrocode}
%  \end{macro}
%
%
%
%
% \subsection{Macros for loading fonts}
%
% \begin{macro}{\pickup@font}
%    The macro |\pickup@font| which is used in
%    |\selectfont| is very simple:
%    if the font name is undefined (i.e.\ not known yet) it calls
%    |\define@newfont| to load it.
%    \begin{macrocode}
\def\pickup@font{%
    \expandafter \ifx \font@name \relax
       \define@newfont
    \fi}
%    \end{macrocode}
% \end{macro}
%
% \begin{macro}{\split@name}
%    |\pickup@font| assumes that |\font@name| is set
%    but it is sometimes called when |\f@family|, |\f@series|,
%    |\f@shape|, or |\f@size| may have the wrong settings
%    (see, e.g., the definition of |\getanddefine@fonts|).
%    Therefore we need a macro to extract font \emph{family},
%    \emph{series}, \emph{shape}, and \emph{size} from the font name.
%    To this end we define |\split@name| which takes the font
%    name as a list of characters of |\catcode| 12 (without the
%    backslash at the beginning) delimited by the
%    special control sequence |\@nil|.
%    This is not very complicated: we first ensure that |/| has
%    the right |\catcode|
%    \begin{macrocode}
{\catcode`\/=12
%    \end{macrocode}
%    and define |\split@name| so that it will define our
%    private |\f@encoding|, |\f@family|, |\f@series|, |\f@shape|,
%    and |\f@size| macros.
% \changes{v1.9a}{1992/07/26}{Added splitting into \cs{f@encoding}.}
%    \begin{macrocode}
\gdef\split@name#1/#2/#3/#4/#5\@nil{\def\f@encoding{#1}%
                                    \def\f@family{#2}%
                                    \def\f@series{#3}%
                                    \def\f@shape{#4}%
                                    \def\f@size{#5}}}
%    \end{macrocode}
% \end{macro}
%
%
%  \begin{macro}{\curr@fontshape}
%    Abbreviation which may get removed again for speed.
% \changes{v1.9a}{1992/07/26}{}
%    \begin{macrocode}
\def\curr@fontshape{\f@encoding/\f@family/\f@series/\f@shape}
%    \end{macrocode}
%  \end{macro}
%
% \begin{macro}{\define@newfont}
%    Now we can tackle the problem of defining a new font.
% \changes{v1.9a}{1992/07/26}{}
%    \begin{macrocode}
\def\define@newfont{%
%    \end{macrocode}
%    We have already mentioned that the \textsf{token} list that
%    |\split@name| will get as argument must not start with
%    a backslash.
%    To reach this goal we will set the |\escapechar| to $-1$ so
%    that the |\string| primitive will not generate an
%    escape character.
%    To keep this change local we open a group.  We use
%    |\begingroup| for this purpose since |\define@newfont|
%    might be called in math mode, and an empty
%    |\bgroup|\ldots|\egroup| would add an empty Ord atom
%    to the math list and thus affect the spacing.
%
%    Also locally redefine |\typeout| so that `No file ...fd'
%    Warnings become Font Info message just sent to the log file.
% \changes{v1.0q}{1989/11/19}{Group added.}
% \changes{v3.0m}{1995/11/17}{Redefine \cs{typeout} latex/1676}
%    \begin{macrocode}
  \begingroup
    \let\typeout\@font@info
    \escapechar\m@ne
%    \end{macrocode}
%    Then we extract \emph{encoding scheme}, \emph{family},
%    \emph{series}, \emph{shape},
%    and \emph{size} from the font name.
%    Note the four |\expandafter|'s so that |\font@name| is
%    expanded first, then |\string|, and finally
%    |\split@name|.
%    \begin{macrocode}
    \expandafter\expandafter\expandafter
       \split@name\expandafter\string\font@name\@nil
%    \end{macrocode}
%    If the |\curr@fontshape| combination is not available,
%    (i.e.\ undefined) we call the macro |\wrong@fontshape| to take
%    care of this case.  Otherwise |\extract@font|
%    will load the external font for us.
% \changes{v1.2f}{1990/01/28}{Added call to \cs{curr@fontshape} macro
%      to allow substitution.}
%    \begin{macrocode}
%    \expandafter\ifx
%        \csname\curr@fontshape\endcsname \relax
      \try@load@fontshape % try always
%    \fi
    \expandafter\ifx
       \csname\curr@fontshape\endcsname \relax
      \wrong@fontshape\else
%    \end{macrocode}
%    To allow substitution we call the  |curr@fontshape| macro
%    which usually will expand to |\relax| but may hold code for
%    substitution (see |\subst@fontshape| definition).
%    \begin{macrocode}
%      \csname\curr@fontshape\endcsname
      \extract@font\fi
%    \end{macrocode}
%    We are nearly finished and must only restore the
%    |\escapechar| by closing the group.
%    \begin{macrocode}
  \endgroup}
%    \end{macrocode}
%
%
% \changes{v3.2b}{2019/02/07}{Changed wording of warning (github/107)}
%    \begin{macrocode}
\def\try@load@fontshape{%
   \expandafter
   \ifx\csname \f@encoding+\f@family\endcsname\relax
     \@font@info{Trying to load font information for
                   \f@encoding+\f@family}%
%    \end{macrocode}
%    We predefine this combination to be |\@empty| which means that
%    next time we don't try again unnecessary in case we don't find a
%    |.fd| file. If the file contains a |\DeclareFontFamily| command
%    than this setting will be overwritten.
% \changes{v2.1e}{1994/02/24}
%     {Separate restoration of catcodes for fd cmds}
% \changes{v2.1l}{1994/05/13}{Use \cs{@input@} for fd files}
%    \begin{macrocode}
    \global\expandafter\let
       \csname\f@encoding+\f@family\endcsname\@empty
%    \end{macrocode}
%    Set the catcodes used in the syntax, but do it only once (this
%    will be restored at the end of the font loading group).
% \changes{v3.0t}{1997/10/21}{Move \cs{makeatletter} to
%    \cs{nfss@catcodes}.}
%    \begin{macrocode}
     \nfss@catcodes
     \let\nfss@catcodes\relax
%    \end{macrocode}
%
% \changes{v3.0s}{1996/11/18}
%     {(DPC) lowercase fd file names. internal/1044}
%    For increased portability make the external filename
%    monocase, but look for the (old style) mixed case
%    filename if the first attempt fails.
%
%   On any monocase system this means that the file is looked for twice
%   which takes up time and string space, but at least for this release
%   Check for both names to give people time to re-install their private
%   fd files with lowercase names.
%    \begin{macrocode}
     \edef\reserved@a{%
       \lowercase{%
         \noexpand\InputIfFileExists{\f@encoding\f@family.fd}}}%
     \reserved@a\relax
          {\@input@{\f@encoding\f@family.fd}}%
   \fi}
%    \end{macrocode}
% \end{macro}
%
%
%  \begin{macro}{\nfss@catcodes}
% \changes{v2.1e}{1994/02/24}
%     {Separate restoration of catcodes for fd cmds}
%    This macro should contain the standard |\catcode| assignments to
%    all characters which are used in the commands found in an
%    \texttt{.fd} file and which might have special |\catcode|s in the
%    middle of a document. If necessary, this list can be extended in
%    a package file using a suitable number of |\expandafter|, i.e.,
%\begin{verbatim}
%  \expandafter\def\expandafter\nfss@catcodes
%        \expandafter{\nfss@catcodes <additional settings>}
%\end{verbatim}
%    Note, that this macro might get executed several times since it
%    is also called by |\DeclareFontShape|, thus it probably should
%    not be misused as a general purpose hook.
%    \begin{macrocode}
\def\nfss@catcodes{%
%    \end{macrocode}
%    We start by making |@| a letter and ignoring all blanks and newlines.
% \changes{v2.1z}{1994/12/06}{Added tab char as well}
% \changes{v3.0p}{1996/07/26}{omit \cs{relax} as not needed}
% \changes{v3.0t}{1997/10/21}{Moved \cs{makeatletter} from
%    \cs{try@load@font@shape}.}
%    \begin{macrocode}
     \makeatletter
     \catcode`\ 9%
     \catcode`\^^I9%
     \catcode`\^^M9%
%    \end{macrocode}
%    Then we set up |\|, |{|, |}|, |#| and |%| in case an \texttt{.fd}
%    file is loaded during a verbatim environment.
% \changes{v3.0n}{1995/11/27}{Reset hash, for definitions in fd files}
% \changes{v3.00}{1995/12/06}{Reset hat, for typeouts etc in fd files}
%    \begin{macrocode}
     \catcode`\\\z@
     \catcode`\{\@ne
     \catcode`\}\tw@
     \catcode`\#6%
     \catcode`\^7%
     \catcode`\%14%
%    \end{macrocode}
%    The we make sure that the important syntax parts have the right
%    |\catcode|.
% \changes{v2.1s}{1994/09/16}{Reset [ and ] as well, just in case}
% \changes{v3.0r}{1996/08/25}{Reset the acute, grave and double quote
%    chars as well}
%    \begin{macrocode}
   \@makeother\<%
   \@makeother\>%
   \@makeother\*%
   \@makeother\.%
   \@makeother\-%
   \@makeother\/%
   \@makeother\[%
   \@makeother\]%
   \@makeother\`%
   \@makeother\'%
   \@makeother\"%
}
%    \end{macrocode}
%  \end{macro}
%
%
%
%  \begin{macro}{\LoadFontDefinitionFile}
%    Load and \texttt{.fd} files for some encoding and family (if it exists).
%    \begin{macrocode}
%</2ekernel>
%<*2ekernel|latexrelease>
%<latexrelease>\IncludeInRelease{2020/02/02}%
%<latexrelease>                 {\LoadFontDefinitionFile}{Loading .fd files}%
\def\LoadFontDefinitionFile#1#2{%
  \begingroup
    \edef\f@encoding{#1}%
    \edef\f@family{#2}%
    \try@load@fontshape
  \endgroup
}
%</2ekernel|latexrelease>
%<latexrelease>\EndIncludeInRelease
%<latexrelease>\IncludeInRelease{0000/00/00}%
%<latexrelease>                 {\LoadFontDefinitionFile}{Loading .fd files}%
%<latexrelease>
%<latexrelease>\let\LoadFontDefinitionFile\@undefined
%<latexrelease>\EndIncludeInRelease
%<*2ekernel>
%    \end{macrocode}
%  \end{macro}
%
%
%

%  \begin{macro}{\DeclareFontFamilySubstitution}
%    The idea for this macro is stolen from the \texttt{substitutefont}
%    package by Günter Milde, with some modifications and a new name.
%
%    Its purpose is to provide characters in a special encoding that
%    are not available in the current font family to be taken from a
%    different family that is visually compatible (or not if you
%    choose badly). For example, you can match the GFS Didot Greek
%    characters with \TeX{} Gyre Pagella (Palatino) by specifying
%\begin{verbatim}
% \DeclareFontFamilySubstitution{LGR}{qpl}{udidot}
%\end{verbatim}
%    This way if you ask for the \texttt{LGR} encoding in for the
%    \texttt{qpl} family you get the characters from the
%    \texttt{udidot} family substituted.
%
%    We need to ensure that the macro is defined with
%    \cs{nfss@catcodes} in force (not quite sure why at the moment to
%    be honest).
%    \begin{macrocode}
%</2ekernel>
%<*2ekernel|latexrelease>
%<latexrelease>\IncludeInRelease{2020/02/02}%
%<latexrelease>       {\DeclareFontFamilySubstitution}{Provide family substitution}%
\begingroup
\nfss@catcodes
\gdef\DeclareFontFamilySubstitution#1#2#3{%
%    \end{macrocode}
%    We only provide a set of silent substitutions. The package also
%    (re)declared the family, but this is incorrect in my eyes and it
%    is better to handle that differently.
%
%    Of course the families  may still need loading at
%    this point and so we arrange for this. Otherwise we might run into
%    trouble because the necessary \cs{DeclareFontFamily} has not been
%    seen.
%    \begin{macrocode}
   \LoadFontDefinitionFile{#1}{#2}%
   \LoadFontDefinitionFile{#1}{#3}%
%    \end{macrocode}
%
%    \begin{macrocode}
   \DeclareFontShape{#1}{#2}{m}{it}{<->ssub * #3/m/it}{}%
   \DeclareFontShape{#1}{#2}{m}{n}{<->ssub * #3/m/n}{}%
   \DeclareFontShape{#1}{#2}{m}{sc}{<->ssub * #3/m/sc}{}%
   \DeclareFontShape{#1}{#2}{m}{sl}{<->ssub * #3/m/sl}{}%
%    \end{macrocode}
%    These days a few more shapes might be around, so we declare those
%    too. If they don't exist then after the first substitution normal
%    fallbacks will happen.
%    \begin{macrocode}
   \DeclareFontShape{#1}{#2}{m}{sw}{<->ssub * #3/m/sw}{}%
   \DeclareFontShape{#1}{#2}{m}{scit}{<->ssub * #3/m/scit}{}%
   \DeclareFontShape{#1}{#2}{m}{scsl}{<->ssub * #3/m/scsl}{}%
%    \end{macrocode}
%    Same game with \texttt{b} and \texttt{bx}, for other weights you
%    are on your own:
%    \begin{macrocode}
   \DeclareFontShape{#1}{#2}{b}{it}{<->ssub * #3/b/it}{}%
   \DeclareFontShape{#1}{#2}{b}{n}{<->ssub * #3/b/n}{}%
   \DeclareFontShape{#1}{#2}{b}{scit}{<->ssub * #3/b/scit}{}%
   \DeclareFontShape{#1}{#2}{b}{scsl}{<->ssub * #3/b/scsl}{}%
   \DeclareFontShape{#1}{#2}{b}{sc}{<->ssub * #3/b/sc}{}%
   \DeclareFontShape{#1}{#2}{b}{sl}{<->ssub * #3/b/sl}{}%
   \DeclareFontShape{#1}{#2}{b}{sw}{<->ssub * #3/b/sw}{}%
   \DeclareFontShape{#1}{#2}{bx}{it}{<->ssub * #3/bx/it}{}%
   \DeclareFontShape{#1}{#2}{bx}{n}{<->ssub * #3/bx/n}{}%
   \DeclareFontShape{#1}{#2}{bx}{scit}{<->ssub * #3/bx/scit}{}%
   \DeclareFontShape{#1}{#2}{bx}{scsl}{<->ssub * #3/bx/scsl}{}%
   \DeclareFontShape{#1}{#2}{bx}{sc}{<->ssub * #3/bx/sc}{}%
   \DeclareFontShape{#1}{#2}{bx}{sl}{<->ssub * #3/bx/sl}{}%
   \DeclareFontShape{#1}{#2}{bx}{sw}{<->ssub * #3/bx/sw}{}%
}
\endgroup
%</2ekernel|latexrelease>
%<latexrelease>\EndIncludeInRelease
%<latexrelease>\IncludeInRelease{0000/00/00}%
%<latexrelease>      {\DeclareFontFamilySubstitution}{Provide family substitution}%
%<latexrelease>
%<latexrelease>\let\DeclareFontFamilySubstitution\@undefined
%<latexrelease>\EndIncludeInRelease
%<*2ekernel>
%    \end{macrocode}
%  \end{macro}
%
%
%
%  \begin{macro}{\DeclareErrorFont}
%    Declare the last resort shape! We assume that in this fontshape
%    there is a 10pt font but it doesn't really matter. We only loose
%    one macro name if the assumption is false. But at least the font
%    should be there!
%    \begin{macrocode}
%</2ekernel>
%<*2ekernel|latexrelease>
%<latexrelease>\IncludeInRelease{2019/10/01}%
%<latexrelease>                 {\DeclareErrorFont}{No side effects please}%
\def\DeclareErrorFont#1#2#3#4#5{%
      \xdef\error@fontshape{%
          \noexpand\expandafter\noexpand\split@name\noexpand\string
          \expandafter\noexpand\csname#1/#2/#3/#4/#5\endcsname
          \noexpand\@nil}%
%    \end{macrocode}
%    Initialize all those internal variables which may or may not have
%    values in the first seconds of NFSS' bootstrapping process. Later
%    on such values will be updated when an encoding is selected, etc.
%
%    We definitely don't want to set |\f@encoding|; we can set all the
%    others since if they are left ``blank'' any selection would grab
%    ``error default values'' as well. However, this probably should
%    go also---and now it did.
% \changes{v2.1n}{1994/05/14}{Don't set \cs{f@encoding}}
% \changes{v3.2c}{2019/07/09}{Don't set any \cs{f@...} macros}
%    \begin{macrocode}
%      \gdef\f@encoding{#1}%
      \gdef\default@family{#2}%
      \gdef\default@series{#3}%
      \gdef\default@shape{#4}%
}
%</2ekernel|latexrelease>
%<latexrelease>\EndIncludeInRelease
%<latexrelease>\IncludeInRelease{0000/00/00}%
%<latexrelease>                 {\DeclareErrorFont}{No side effects please}%
%<latexrelease>
%<latexrelease>\def\DeclareErrorFont#1#2#3#4#5{%
%<latexrelease>      \xdef\error@fontshape{%
%<latexrelease>          \noexpand\expandafter\noexpand\split@name\noexpand\string
%<latexrelease>          \expandafter\noexpand\csname#1/#2/#3/#4/#5\endcsname
%<latexrelease>          \noexpand\@nil}%
%<latexrelease>      \gdef\default@family{#2}%
%<latexrelease>      \gdef\default@series{#3}%
%<latexrelease>      \gdef\default@shape{#4}%
%<latexrelease>      \global\let\f@family\default@family
%<latexrelease>      \global\let\f@series\default@series
%<latexrelease>      \global\let\f@shape\default@shape
%<latexrelease>      \gdef\f@size{#5}%
%<latexrelease>      \gdef\f@baselineskip{#5pt}%
%<latexrelease>}
%<latexrelease>\EndIncludeInRelease
%<*2ekernel>
%    \end{macrocode}
%
%    \begin{macrocode}
\@onlypreamble\DeclareErrorFont
%    \end{macrocode}
%  \end{macro}
%
%
% \begin{macro}{\wrong@fontshape}
%    Before we come to the macro |\extract@font| we have to take
%    care of unknown |\curr@fontshape| combinations.
%    The general strategy is to issue a warning and to try a default
%    \emph{shape}, then a default \emph{series},
%    and finally a default \emph{family}.
%    If this last one also fails \TeX{} will go into an infinite loop.
%    But if the defaults are set incorrectly one deserves nothing else!
% \changes{v1.9a}{1992/07/26}{}
%    \begin{macrocode}
%</2ekernel>
%<latexrelease>\IncludeInRelease{2015/01/01}{\wrong@fontshape}%
%<latexrelease>                 {Font substitution in preamble}%
%<*2ekernel|latexrelease>
\def\wrong@fontshape{%
    \csname D@\f@encoding\endcsname   % install defaults if in math
%    \end{macrocode}
%    We remember the wanted |\curr@fontshape| combination
%    which we will need in a moment.
%    \begin{macrocode}
    \edef\reserved@a{\csname\curr@fontshape\endcsname}%
  \ifx\last@fontshape\reserved@a
     \errmessage{Corrupted NFSS tables}%
     \error@fontshape
  \else
%    \end{macrocode}
%    Then we warn the user about the mess and set the shape to its
%    default.
% \changes{v1.0q}{1989/11/19}
%     {Instead of calling \cs{family}\cs{default@family},
%                           etc. we directly set \cs{f@family}, etc.}
% \changes{v1.2f}{1990/01/28}{Warning message slightly changed.}
%    \begin{macrocode}
    \let\f@shape\default@shape
%    \end{macrocode}
%    If the combination is not known, try the default \emph{series}.
%    \begin{macrocode}
    \expandafter\ifx\csname\curr@fontshape\endcsname\relax
       \let\f@series\default@series
%    \end{macrocode}
%    If this is still undefined, try the default \emph{family}.
%    Otherwise give up. We never try to change the encoding scheme!
% \changes{v1.9a}{1992/07/26}{}
% \changes{v3.1a}{2015/04/07}{Try loading fd file if family has changed}
%    \begin{macrocode}
        \expandafter
          \ifx\csname\curr@fontshape\endcsname\relax
           \let\f@family\default@family
%    \end{macrocode}
%    If we change the font family and we are in the preamble then the
%    corresponding \texttt{.fd} file may not been loaded
%    yet. Therefore we try this now. Otherwise equating the requested
%    font shape with the finally selected fontshape below will fail
%    and can result in ``NFSS tables corrupted''. After begin document
%    that will not happen as all \texttt{.fd} files involved in
%    substitution are loaded at |\begin{document}|.
%    \begin{macrocode}
           \begingroup
              \try@load@fontshape
           \endgroup
        \fi \fi
  \fi
%    \end{macrocode}
%    At this point a valid |\curr@fontshape| combination must
%    have been found.
%    We inform the user about this fact.
%\changes{v3.0n}{1995/11/02}
%    {(DPC) Remove extra space with \cs{string} for latex/1676}
% \changes{v3.0m}{1995/11/17}
%    {Support \cs{@wrong@font@char} latex/1676}
%
% The |\expandafter\string| here stops \TeX\ adding the space
% that it usually puts after command names in messages. The similar
% construction with |\@undefined| just produces `undefined', but saves
% a few tokens.
%
% |\@wrong@font@char|  is locally redefined in |\UseTextSymbol| from
% its normal (empty) definition, to report the symbol generating the
% font switch.
%    \begin{macrocode}
     \@font@warning{Font shape `\expandafter\string\reserved@a'
                     \expandafter\@gobble\string\@undefined\MessageBreak
                   using `\curr@fontshape' instead\@wrong@font@char}%
    \global\let\last@fontshape\reserved@a
%    \end{macrocode}
%    We change |\@defaultsubs| to produce a warning at the end of
%    the document.
% \changes{v3.0d}{1995/07/13}{Change a macro not a switch to flag
%         default font substitutions}
% \changes{v3.0k}{1995/10/24}{Make this code inline since it
%         happens only here}
%    The macro |\@defaultsubs| is initially |\relax| but gets changed
%    here if some default font substitution happens.
%    It is then executed in |\enddocument|.
%    \begin{macrocode}
    \gdef\@defaultsubs{%
      \@font@warning{Some font shapes were not available, defaults
                      substituted.\@gobbletwo}}%
%    \end{macrocode}
%    If we substitute a |\curr@fontshape| combination
%    by the default one we don't want the warning to be printed out
%    whenever this (unknown) combination is used.
%    Therefore we globally |\let| the macro corresponding to
%    the wanted combination equal to its substitution.
%    This requires the use of four |\expandafter|'s
%    since |\csname|\dots|\endcsname| has to be
%    expanded before |\reserved@a| (i.e.\  the requested
%    combination),
%    and this must happen before the |\let| is executed.
%    \begin{macrocode}
    \global\expandafter\expandafter\expandafter\let
       \expandafter\reserved@a
           \csname\curr@fontshape\endcsname
%    \end{macrocode}
%    Now we can redefine |\font@name| accordingly.
%    This \emph{must} be done globally since it might occur in the
%    group opened by |\define@newfont|.  If we would this
%    definition were local the closing |\endgroup| there
%    would restore the old meaning of |\font@name| and then
%    switch to the wrong font at the end of |\selectfont|
%    although the correct font was loaded.
%    \begin{macrocode}
    \xdef\font@name{%
      \csname\curr@fontshape/\f@size\endcsname}%
%    \end{macrocode}
%    The last thing this macro does is to call |\pickup@font|
%    again to load the font if it is not defined yet.
%    At this point this code will loop endlessly if
%    the defaults are not well defined.
%    \begin{macrocode}
    \pickup@font}
%</2ekernel|latexrelease>
%<latexrelease>\EndIncludeInRelease
%<latexrelease>\IncludeInRelease{0000/00/00}{\wrong@fontshape}%
%<latexrelease>                 {Font substitution in preamble}%
%<latexrelease>\def\wrong@fontshape{%
%<latexrelease>    \csname D@\f@encoding\endcsname
%<latexrelease>    \edef\reserved@a{\csname\curr@fontshape\endcsname}%
%<latexrelease>  \ifx\last@fontshape\reserved@a
%<latexrelease>     \errmessage{Corrupted NFSS tables}%
%<latexrelease>     \error@fontshape
%<latexrelease>  \else
%<latexrelease>    \let\f@shape\default@shape
%<latexrelease>    \expandafter\ifx\csname\curr@fontshape\endcsname\relax
%<latexrelease>       \let\f@series\default@series
%<latexrelease>        \expandafter
%<latexrelease>          \ifx\csname\curr@fontshape\endcsname\relax
%<latexrelease>           \let\f@family\default@family
%<latexrelease>        \fi \fi
%<latexrelease>  \fi
%<latexrelease>     \@font@warning{Font shape
%<latexrelease>            `\expandafter\string\reserved@a'
%<latexrelease>            \expandafter\@gobble\string\@undefined
%<latexrelease>            \MessageBreak
%<latexrelease>            using `\curr@fontshape' instead\@wrong@font@char}%
%<latexrelease>    \global\let\last@fontshape\reserved@a
%<latexrelease>    \gdef\@defaultsubs{%
%<latexrelease>      \@font@warning{Some font shapes were not available,
%<latexrelease>                       defaults substituted.\@gobbletwo}}%
%<latexrelease>    \global\expandafter\expandafter\expandafter\let
%<latexrelease>       \expandafter\reserved@a
%<latexrelease>           \csname\curr@fontshape\endcsname
%<latexrelease>    \xdef\font@name{%
%<latexrelease>      \csname\curr@fontshape/\f@size\endcsname}%
%<latexrelease>    \pickup@font}
%<latexrelease>\EndIncludeInRelease
%<*2ekernel>
%    \end{macrocode}
% \end{macro}
%
%  \begin{macro}{\@wrong@font@char}
% \changes{v3.0m}{1995/11/17}{(DPC) Macro added. latex/1676}
%    Normally empty but redefined in |\UseTextSymbol| so that the
%    Font shape undefined message can refer to the symbol causing the
%    problem.
%    \begin{macrocode}
\let\@wrong@font@char\@empty
%    \end{macrocode}
%  \end{macro}
%
%  \begin{macro}{\@@defaultsubs}
% \changes{v3.0d}{1995/07/13}{macro added}
% \changes{v3.0k}{1995/10/24}{macro removed}
%  \begin{macro}{\@defaultsubs}
% \changes{v3.0d}{1995/07/13}{macro added}
%    See above.
%    \begin{macrocode}
\let\@defaultsubs\relax
%    \end{macrocode}
%  \end{macro}
%  \end{macro}
%
%
% \begin{macro}{\strip@prefix}
%    In |\extract@font| we will need a way to recover the
%    replacement text of a macro.
%    This is done by the primitive |\meaning| together
%    with the macro |\strip@prefix| (for the details
%    see appendix~D of the \TeX{}book, p.\ 382).
%    \begin{macrocode}
\def\strip@prefix#1>{}
%    \end{macrocode}
% \end{macro}
%
% ^^A \extract@font
%
%
%
% \section{Assigning math fonts to \emph{versions}}
%
%
%  \begin{macro}{\install@mathalphabet}
%    This is just another name for |\gdef| but we can redefine it if
%    necessary later on.
%    \begin{macrocode}
\let\install@mathalphabet\gdef
%    \end{macrocode}
%  \end{macro}
%
%
%
%  \begin{macro}{\math@fonts}
%
% \changes{v1.9a}{1992/07/26}{}
%    \begin{macrocode}
\let\math@fonts\@empty
%    \end{macrocode}
%  \end{macro}
%
%
%
% \begin{macro}{\select@group}
% \changes{v1.0t}{1989/11/26}{\cs{bgroup}/\cs{egroup} changed to
%   \cs{begingroup}/\cs{endgroup} to avoid empty Ord atom on math list.}
%    |\select@group| has four arguments: the new
%    \meta{math alphabet identifier} (a control sequence), the
%    \meta{math group number}, the extra macro for math mode
%    and the |\curr@fontshape| definition macro name.
%    We first check if we are in math mode.
% \changes{v1.1a}{1989/12/16}{\cs{relax} in front added.}
% \changes{v1.1a}{1989/12/16}{Now four arguments.}
%    \begin{macrocode}
%\def\select@group#1#2#3{\relax\ifmmode
%    \end{macrocode}
%    We do these things locally using |\begingroup| instead
%    of |\bgroup| to avoid the appearance of an empty Ord
%    atom on the math list.
%    \begin{macrocode}
%  \begingroup
%    \end{macrocode}
%    We set the math fonts for the \emph{family} in question by calling
%    |\getanddefine@fonts| in the correct environment.
% \changes{v1.9a}{1992/07/26}{}
%    \begin{macrocode}
%     \escapechar\m@ne
%    \getanddefine@fonts{\csname c@mv@\math@version\endcsname}#3%
%    \end{macrocode}
%    We globally select the math fonts\dots
%    \begin{macrocode}
%    \globaldefs\@ne  \math@fonts
%    \end{macrocode}
%    \dots{} and close the group to
%    restore |\globaldefs| and |\escapechar|.
%    \begin{macrocode}
%  \endgroup
%    \end{macrocode}
%    As long as no \emph{size} or \emph{version} change occurs
%    the \meta{math alphabet identifier} should simply switch to the
%    installed \emph{math group} instead of calling
%    |\select@group|
%    unnecessarily. So we globally redefine the first argument
%    (the new \meta{math alphabet identifier})
%    to expand into a |\mathgroup| switch
%    and then select this \emph{alphabet}. Note that this redefinition
%    will be overwritten by the next call to a \emph{version} macro.
% \changes{v1.1a}{1989/12/16}{Usage of `\quotechar=' macro added.}
% \changes{v1.1a}{1989/12/16}{Redefinition of alphabet now simpler.}
% \changes{v1.2a}{1990/01/20}{Def for alph id changed.}
%    The original code for the end of |\select@group| was
%    \begin{verbatim}
%  \gdef#1{#3\mathgroup #2}#1\fi}
%\end{verbatim}
%    i.e.\ first redefining the \meta{math alphabet identifier} and
%    then calling the new definition to switch to the wanted
%    \meta{math group}. Now we define the \meta{math alphabet
%    identifier} as a call to the |\use@mathgroup| command.
% \changes{v1.2b}{1990/01/21}{Code moved to \cs{use@mathgroup}.}
%    \begin{macrocode}
%  \xdef#1{\noexpand\use@mathgroup\noexpand#2%
%          {\number\csname c@mv@\math@version\endcsname}}%
%    \end{macrocode}
%    \changes{v1.3c}{1992/05/12}{Added call to
%             \cs{extract@alph@from@version}.}
%    But this is not sufficient, as we learned the hard way.
%    The problem here is that the loading of the fonts that comprise
%    the alphabet identifier |#1|, as well as the necessary math font
%    assignments is deferred until it is used. This is OK so far, but
%    if the fonts are switched within the current formula (which may
%    happen if a sub-formula is a box that contains a math version
%    switch) the font assignments for |#1| are not restored unless
%    |#1| is used again. This is disastrous since TeX sees the wrong
%    fonts at the end of the math formula, when it converts the math
%    list into a horizontal list.
%
%    This is taken into account as follows: When a math alphabet
%    identifier is used for the first time in a certain version
%    it modifies the corresponding macro |\mv@|\meta{version}
%    so that it calls |\getanddefine@fonts| directly in future as well.
%    We use the macro |\extract@alph@from@version| to do this.
%    It takes the math alphabet identifier |#1| and the math version
%    macro as arguments.
% \changes{v1.9a}{1992/07/26}{}
%    \begin{macrocode}
%     \expandafter\extract@alph@from@version
%        \csname mv@\math@version\expandafter\endcsname
%       \expandafter{\number\csname c@mv@\math@version\endcsname}%
%       #1%
%     \stepcounter{mv@\math@version}%
%    \end{macrocode}
%    Finally, it is not possible to simply call the new definition
%    since we
%    have an argument (the third argument of |\use@mathgroup|
%    or more exactly the argument of |\math@egroup| if the {\ttfamily
%    margid} option is in force)
%    which would swallow our closing |\fi|.  So
%    we use the |\expandafter| technique to remove the |\fi|
%    before the |\use@mathgroup| is expanded.
%    \begin{macrocode}
%\expandafter #1\fi}
%    \end{macrocode}
% \end{macro}
%
%
% \begin{macro}{\extract@alph@from@version}
%    \changes{v1.3c}{1992/05/12}{Macro added.}
%    We proceed to the definition of the
%    macro |\extract@alph@from@version|. As stated above, it takes
%    a math alphabet identifier and a math version macro
%    (e.g.\ |\mv@normal|) as its arguments.
%    \begin{macrocode}
\def\extract@alph@from@version#1#2#3{%
%    \end{macrocode}
%    To extract and replace the definition of math alphabet identifier
%    |#3| in macro |#1| we have to recall how this definition looks
%    like: Somewhere in the replacement text of |#1| there is
%    the sequence\\[2pt]
%    |\install@mathalphabet|\meta{math alphabet identifier} |#3||{%| \\
%      \hspace*{\MacroIndent}\hspace*{5mm}
%     \meta{Definitions for }|#3}| \\[2pt]
%    Hence, the first thing we do is to extract the tokens preceding
%    this definitions, the definition itself, and the tokens following
%    it. To this end we define one auxiliary macro |\reserved@a|.
%    \begin{macrocode}
     \def\reserved@a##1\install@mathalphabet#3##2##3\@nil{%
%    \end{macrocode}
%    When |\reserved@a| is expanded, it will have the tokens preceding
%    the definition in question in its first argument (|##1|), the
%    following tokens in its third argument (|##3|), and the replacement
%    text for the math alphabet identifier |#3| in its second argument.
%    (|##2|). This is then recorded for later use in a temporary macro
%    |\reserved@b|.
%    \begin{macrocode}
         \def\reserved@b{##2}%
%    \end{macrocode}
%    Additionally, we define a macro |\reserved@c| to reconstruct the
%    definitions for the math version in question from the tokens that
%    will remain unchanged (|##1| and |##3|) and the yet to build new
%    definitions for the math alphabet identifier |#3|.
%    \begin{macrocode}
         \def\reserved@c####1{\gdef#1{##1####1##3}}}%
%    \end{macrocode}
%    Then we execute our auxiliary macro.
%    \begin{macrocode}
     \expandafter\reserved@a#1\@nil
%    \end{macrocode}
%    OK, so now we have to build the new definition for |#3|. To do so,
%    we first extract the interesting parts out of the old one.
%    The old definition looks like:\\[2pt]
%    |\select@group|\meta{math alphabet identifier} \\
%      \hspace*{\MacroIndent}\hspace*{5mm}
%      \meta{math group number}\meta{math extra part}\\
%      \meta{|curr@fontshape| definition} \\[2pt]
%    So we define a new temporary macro |\reserved@a| that
%     extracts these parts.
%    \begin{macrocode}
     \def\reserved@a\select@group#3##1##2\@nil{%
%    \end{macrocode}
%    This macro can now directly rebuild the math version definition
%    by calling |\reserved@c|:
%    \begin{macrocode}
        \reserved@c{%
           \getanddefine@fonts{#2}##2%
           \install@mathalphabet#3{%
              \relax\ifmmode \else \non@alpherr#3\fi
              \use@mathgroup##1{#2}}}%
%    \end{macrocode}
% \changes{v2.1t}{1994/10/15}{Warn if math alpha is used outside math}
%    In addition it defines the alphabet the way it should be used from
%    now on.
%    \begin{macrocode}
       \gdef#3{\relax\ifmmode \else \non@alpherr#3\fi
               \use@mathgroup##1{#2}}}%
%    \end{macrocode}
%    Finally, we only have to call this macro |\reserved@a| on the old
%    definitions recorded in |\reserved@b|:
%    \begin{macrocode}
     \expandafter\reserved@a\reserved@b\@nil
     }
%    \end{macrocode}
% \end{macro}
%
%
%
% \begin{macro}{\math@bgroup}
% \changes{v1.2a}{1990/01/20}{Def. placed in this file.}
% \begin{macro}{\math@egroup}
% \changes{v1.2a}{1990/01/20}{Def. placed in this file.}
% \changes{v1.2h}{1990/03/30}{Changed to have one arg.}
% \changes{v2.2f}{1994/03/10}{Changed \cs{begingroup}/\cs{endgroup} to
%                             \cs{bgroup}/\cs{egroup}.}
%   Here are the default definitions for |\math@bgroup| and
%   |\math@egroup|. We use |\bgroup| instead of |\begingroup|
%   to avoid `leaking out' of style changes. This has the side
%   effect of always producing mathord atoms.
%    \begin{macrocode}
\let\math@bgroup\bgroup
\def\math@egroup#1{#1\egroup}
%    \end{macrocode}
% \end{macro}
% \end{macro}
%
%
% \begin{macro}{\calculate@math@sizes}
% \changes{v2.1i}{1994/04/18}{Changed message to log only}
%    Here is the default definition for |\calculate@math@sizes|
%    a more elaborate interface
%    is under testing in mthscale.sty.
%    \begin{macrocode}
\gdef\calculate@math@sizes{%
  \@font@info{Calculating\space math\space sizes\space for\space
              size\space <\f@size>}%
  \dimen@\f@size \p@
  \@tempdimb \defaultscriptratio \dimen@
  \dimen@ \defaultscriptscriptratio \dimen@
  \expandafter\xdef\csname S@\f@size\endcsname{%
    \gdef\noexpand\tf@size{\f@size}%
    \gdef\noexpand\sf@size{\strip@pt\@tempdimb}%
    \gdef\noexpand\ssf@size{\strip@pt\dimen@}%
    \noexpand\math@fontstrue}}
%    \end{macrocode}
%  \end{macro}
%
% \begin{macro}{\defaultscriptratio}
% \changes{v2.1h}{1994/04/11}{Macro added}
% \begin{macro}{\defaultscriptscriptratio}
% \changes{v2.1h}{1994/04/11}{Macro added}
%    The default ratio for math
%    sizes is:\\
% 1 to |\defaultscriptratio| to |\defaultscriptscriptratio|.\\
%  By default this is 1 to .7 to .5.
%    \begin{macrocode}
\def\defaultscriptratio{.7}
\def\defaultscriptscriptratio{.5}
%    \end{macrocode}
%  \end{macro}
%  \end{macro}
%
%  \begin{macro}{\noaccents@}
%    If we don't have a definition for |\noaccents@| we provide a
%    dummy.
%    \begin{macrocode}
\ifx\noaccents@\@undefined
  \let\noaccents@\@empty
\fi
%    \end{macrocode}
% \end{macro}
%
% \begin{macro}{\showhyphens}
% \changes{v1.2l}{1990/06/30}{Macro added.}
% \changes{v2.0e}{1993/05/16}{Use \cs{reset@font}}
% \changes{v3.0h}{1995/10/10}
%    {Use \cs{normalfont} and make colour safe, and autoloadable}
% \changes{v3.0u}{1998/03/25}
%    {Suppress unnecessary error when used in preamble}
% \changes{v3.2a}{2017/01/10}
%    {Add version of \cs{showhyphens} that works with Xe\TeX.}
%    The |\showhyphens| command must be redefined since the version in
%    \texttt{plain.tex} uses |\tenrm|.  We have also made some further
%    adjustments for its use in \LaTeX.
%    \begin{macrocode}
%</2ekernel>
%<latexrelease>\IncludeInRelease{2017/01/01}{\showhyphens}%
%<latexrelease>                 {XeTeX support for \showhyphens}%
%<*2ekernel|latexrelease>
\ifx\XeTeXcharclass\@undefined
%    \end{macrocode}
% Version for engines other than Xe\TeX.
%    \begin{macrocode}
\DeclareRobustCommand\showhyphens[1]{%
  \setbox0\vbox{%
    \color@begingroup
    \everypar{}%
    \parfillskip\z@skip\hsize\maxdimen
    \normalfont
    \pretolerance\m@ne\tolerance\m@ne\hbadness\z@\showboxdepth\z@\ #1%
    \color@endgroup}}
%    \end{macrocode}
%    \begin{macrocode}
\else
%    \end{macrocode}
% Xe\TeX\ version. When using system fonts Xe\TeX\ reports consecutive
% runs of characters as a single item in box logging, which means the
% standard |\showhyphens| does not work.  This version typesets the
% text into a narrow box to force hyphenation and then reconstructs a
% horizontal list with explicit hyphens to generate the display. Note
% that the |lmr| OpenType font is forced, this works even if the
% characters are not in the font as hyphenation is attempted due to
% the width of the space and hyphen character. It may generate
% spurious Missing Character warnings in the log, these are however
% suppressed from the terminal output by ensuring that
% |\tracingonline| is locally zero.
%    \begin{macrocode}
\DeclareRobustCommand\showhyphens[1]{%
  \setbox0\vbox{%
    \usefont{TU}{lmr}{m}{n}%
    \hsize 1sp %
    \hbadness\@M
    \hfuzz\maxdimen
    \tracingonline\z@
    \everypar={}%
    \leftskip\z@skip
    \rightskip\z@skip
    \parfillskip\z@skip
    \hyphenpenalty=-\@M
    \pretolerance\m@ne
    \interlinepenalty\z@
    \clubpenalty\z@
    \widowpenalty\z@
    \brokenpenalty1127 %
    \setbox\z@\hbox{}%
    \noindent
    \hskip\z@skip
    #1%
    \par
%    \end{macrocode}
% Note here we stop the loop if made no progress, non-removable items
% may mean that we can not process the whole list (which would be
% testable as |\lastnodetype=-1|).
%    \begin{macrocode}
     \loop
     \@tempswafalse
     \ifnum\lastnodetype=11\unskip\@tempswatrue\fi
     \ifnum\lastnodetype=12\unkern\@tempswatrue\fi
     \ifnum\lastnodetype=13 %
      \count@\lastpenalty
      \unpenalty\@tempswatrue
    \fi
    \ifnum\lastnodetype=\@ne
     \setbox\tw@\lastbox\@tempswatrue
     \setbox0\hbox{\unhbox\tw@\unskip\unskip\unpenalty
                   \ifnum\count@=1127 \else\ \fi
                   \unhbox0}%
     \count@\z@
    \fi
    \if@tempswa
    \repeat
   \hbadness\z@
   \hsize\maxdimen
   \showboxdepth\z@
   \tolerance\m@ne
   \hyphenpenalty\z@
   \noindent\unhbox\z@
}}
%    \end{macrocode}
%
%    \begin{macrocode}
\fi
%    \end{macrocode}
%
%    \begin{macrocode}
%</2ekernel|latexrelease>
%<latexrelease>\EndIncludeInRelease
%<latexrelease>\IncludeInRelease{0000/00/00}{\showhyphens}%
%<latexrelease>                 {XeTeX support for \showhyphens}%
%<latexrelease>\gdef\showhyphens#1{%
%<latexrelease>  \setbox0\vbox{%
%<latexrelease>    \color@begingroup
%<latexrelease>    \everypar{}%
%<latexrelease>    \parfillskip\z@skip\hsize\maxdimen
%<latexrelease>    \normalfont
%<latexrelease>    \pretolerance\m@ne\tolerance\m@ne
%<latexrelease>    \hbadness\z@\showboxdepth\z@\ #1%
%<latexrelease>    \color@endgroup}}
%<latexrelease>\EndIncludeInRelease
%<*2ekernel>
%    \end{macrocode}
% \end{macro}
%
%
% \begin{macro}{\addto@hook}
% \changes{v1.0u}{1989/12/05}{\cs{addto@hook} added.}
% \changes{v2.1d}{1994/02/10}{Made \cs{addto@hook} long.}
%    We need a macro to add tokens to a hook.
%    \begin{macrocode}
\long\def\addto@hook#1#2{#1\expandafter{\the#1#2}}
%    \end{macrocode}
% \end{macro}
%
%
% \begin{macro}{\@vpt}
% \changes{v2.1v}{1994/11/09}{(DPC) macros added, from setsize.dtx}
% \changes{v2.1v}{1994/11/09}{(DPC) reduce save stack usage latex/1742}
%    \begin{macrocode}
 \def\@vpt{5}
%    \end{macrocode}
% \end{macro}
%
% \begin{macro}{\@vipt}
%    \begin{macrocode}
 \def\@vipt{6}
%    \end{macrocode}
% \end{macro}
%
% \begin{macro}{\@viipt}
%    \begin{macrocode}
 \def\@viipt{7}
%    \end{macrocode}
% \end{macro}
%
% \begin{macro}{\@viiipt}
%    \begin{macrocode}
 \def\@viiipt{8}
%    \end{macrocode}
% \end{macro}
%
% \begin{macro}{\@ixpt}
%    \begin{macrocode}
 \def\@ixpt{9}
%    \end{macrocode}
% \end{macro}
%
% \begin{macro}{\@xpt}
%    \begin{macrocode}
 \def\@xpt{10}
%    \end{macrocode}
% \end{macro}
%
% \begin{macro}{\@xipt}
%    \begin{macrocode}
 \def\@xipt{10.95}
%    \end{macrocode}
% \end{macro}
%
% \begin{macro}{\@xiipt}
%    \begin{macrocode}
 \def\@xiipt{12}
%    \end{macrocode}
% \end{macro}
%
% \begin{macro}{\@xivpt}
%    \begin{macrocode}
 \def\@xivpt{14.4}
%    \end{macrocode}
% \end{macro}
%
% \begin{macro}{\@xviipt}
%    \begin{macrocode}
 \def\@xviipt{17.28}
%    \end{macrocode}
% \end{macro}
%
% \begin{macro}{\@xxpt}
%    \begin{macrocode}
 \def\@xxpt{20.74}
%    \end{macrocode}
% \end{macro}
%
% \begin{macro}{\@xxvpt}
%    \begin{macrocode}
 \def\@xxvpt{24.88}
%    \end{macrocode}
% \end{macro}
%
%    \begin{macrocode}
%</2ekernel>
%    \end{macrocode}
%
% \Finale
%
%%%%%%%%%%%%%%%%%%%%%%%%%%%%%%%%%%%%%%%%%%%%%
\endinput

^^A  Needed for emacs
^^A
^^A  Local Variables: 
^^A  mode: latex
^^A  coding: utf-8-unix
^^A  End: 
 
