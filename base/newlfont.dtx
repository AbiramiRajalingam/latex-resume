% \iffalse meta-comment
%
% Copyright (C) 1993-2021
% The LaTeX3 Project and any individual authors listed elsewhere
% in this file.
%
% This file is part of the LaTeX base system.
% -------------------------------------------
%
% It may be distributed and/or modified under the
% conditions of the LaTeX Project Public License, either version 1.3c
% of this license or (at your option) any later version.
% The latest version of this license is in
%    https://www.latex-project.org/lppl.txt
% and version 1.3c or later is part of all distributions of LaTeX
% version 2008 or later.
%
% This file has the LPPL maintenance status "maintained".
%
% The list of all files belonging to the LaTeX base distribution is
% given in the file `manifest.txt'. See also `legal.txt' for additional
% information.
%
% The list of derived (unpacked) files belonging to the distribution
% and covered by LPPL is defined by the unpacking scripts (with
% extension .ins) which are part of the distribution.
%
% \fi
%\iffalse   % this is a METACOMMENT !
%
%% File `newlfont.dtx'.
%% Copyright (C) 1989-1998 Frank Mittelbach and Rainer Sch\"opf,
%% all rights reserved.
%
%<*dtx>
          \ProvidesFile{newlfont.dtx}
%</dtx>
%<package>\NeedsTeXFormat{LaTeX2e}
%<package>\ProvidesPackage{newlfont}
%<driver> \ProvidesFile{newlfont.drv}
% \fi
%         \ProvidesFile{newlfont.dtx}
          [1998/08/17 v2.2m Standard LaTeX package]
%
% \iffalse
%<*driver>
\documentclass{ltxdoc}
\begin{document}
\DocInput{newlfont.dtx}
\end{document}
%</driver>
% \fi
%
%
%
%
% \changes{v2.2a}{1994/01/17}{Added missing definitions to support
%                             newlfont in compat mode}
% \changes{v2.2b}{1994/01/27}{Removed if@compat test since
%           these settings need also be executed in 2e mode.}
% \changes{v2.2c}{1994/03/06}{Added def for \cs{cal} and \cs{mit}}
% \changes{v2.2l}{1995/11/29}{Remove duplicate driver code.}
% \changes{v2.2m}{1998/08/17}{(RmS) Minor documentation fixes.}
%
% \providecommand\dst{\expandafter{\normalfont\scshape docstrip}}
%
%
% \GetFileInfo{newlfont.dtx}
%
% \title{The file \texttt{newlfont.dtx} for use with
%        \LaTeXe.\thanks{This file has version
%        number \fileversion, dated \filedate.}\\[2pt]
%        It contains the code for \texttt{newlfont.sty}}
%
% \author{Frank Mittelbach}
%
% \MaintainedByLaTeXTeam{latex}
% \maketitle
%
% \section{Introduction}
%
%    This file contains the code for the \texttt{newlfont} package
%    which provides defines commands like |\rm| to behave as with
%    NFSS1, i.e., to change one font attribute without resetting the
%    other attributes.
%
% \StopEventually{}
%
%
% \section{The Code}
%
%    \begin{macrocode}
%<*package>
%    \end{macrocode}
%
%    We define the font commands for selecting the typeface. They are
%    probably defined by the document class/style but we want to force
%    the old meaning from NFSS1.
%
% \changes{v2.2g}{1994/05/11}{DPC Remove definitions of \cs{prm} etc.}
%    \begin{macrocode}
\let\rm\rmfamily
\let\sf\sffamily
\let\tt\ttfamily
\let\bf\bfseries
\let\sl\slshape
\let\sc\scshape
\let\it\itshape
%    \end{macrocode}
%     We also have to define the \emph{emphasize} font change command
%    (i.e.\ |\em|). This command will look whether the current font is
%    sloped (i.e.\ has a positive |\fontdimen1|) and will then select
%    either an upright or italic font.
% \changes{v2.2g}{1994/05/11}{DPC use \cs{DeclareProtectedCommand}}
% \changes{v2.2h}{1994/05/13}{DPC renamed to \cs{DeclareRobustCommand}}
%    \begin{macrocode}
\DeclareRobustCommand\em{%
  \@nomath\em
  \ifdim \fontdimen\@ne\font >\z@\upshape \else \itshape \fi}
%    \end{macrocode}
%
%    For compatibility with old sources we should define the following
%    commands although their use in new documents is discouraged.
%    \begin{macrocode}
\let\mediumseries\mdseries
\let\normalshape\upshape
%    \end{macrocode}
%
%    In case \texttt{newlfont} is used in compatibility mode, eg
%\begin{verbatim}
%  \documentstyle[newlfont]{article}
%\end{verbatim}
%    or with the \texttt{oldlfont} package already loaded
%    we have to undo a number of settings changed by the compatibility
%    mode for \LaTeX~2.09 documents.
%
%    |\@setfontsize| should not have a |\reset@font| included.
% \changes{v2.2j}{1994/11/06}{Use \cs{@typeset@protect}}
%    \begin{macrocode}
\def\@setfontsize#1#2#3{\@nomath#1%
    \ifx\protect\@typeset@protect
      \let\@currsize#1%
    \fi
    \fontsize{#2}{#3}\selectfont}
%    \end{macrocode}
%    Math alphabet identifiers should have an argument.
% \changes{v2.2d}{1994/03/10}{Changed \cs{begingroup}/\cs{endgroup} to
%                             \cs{bgroup}/\cs{egroup}.}
% \changes{v2.2f}{1994/05/05}{Added saved versions of the
% math-groupers, CAR}
%    \begin{macrocode}
\let\math@bgroup\bgroup
\def\math@egroup#1{#1\egroup}
\let \@@math@bgroup \math@bgroup
\let \@@math@egroup \math@egroup
%    \end{macrocode}
%    Warn again, if text font commands are used in math
%    (same macro as in \texttt{lfonts.dtx}).
% \changes{v2.2e}{1994/04/21}{Changed error message}
% \changes{v2.2k}{1995/04/02}{add \cs{noexpand} to second part of message}
%    \begin{macrocode}
\def\not@math@alphabet#1#2{%
   \relax
   \ifmmode
     \@latex@error{Command \noexpand#1invalid in math mode}%
        {%
         Please
         \ifx#2\relax
            define a new math alphabet^^J%
            if you want to use a special font in math mode%
          \else
%    \end{macrocode}
%    We have to add |\noexpand| below to prevent expansion of |#2|. In
%    case of |#1| we can omit this (due to the current definition of
%    robust commands since they do come out right there :-).
%    \begin{macrocode}
            use the math alphabet \noexpand#2instead of
            the #1command%
         \fi
         .
        }%
   \fi}
%    \end{macrocode}
%
%    In NFSS1 math alphabets had arguments so we have to change |\cal|
%    and |\mit| which by default are defined to behave like in
%    \LaTeX209.
%    \begin{macrocode}
\let\pcal\@undefined
\let\cal\mathcal
\let\pmit\@undefined
\let\mit\mathnormal
%    \end{macrocode}
%
%    The NFSS1 version of \texttt{newlfont} included the \LaTeX{}
%    symbols. And that is probably all there is.
%    \begin{macrocode}
\RequirePackage{latexsym}
%</package>
%    \end{macrocode}
%
%
% \Finale
\endinput
