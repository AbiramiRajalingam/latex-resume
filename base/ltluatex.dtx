% \iffalse meta-comment
%
% Copyright (C) 2015-2024
% The LaTeX Project and any individual authors listed elsewhere
% in this file.
%
% This file is part of the LaTeX base system.
% -------------------------------------------
%
% It may be distributed and/or modified under the
% conditions of the LaTeX Project Public License, either version 1.3c
% of this license or (at your option) any later version.
% The latest version of this license is in
%    https://www.latex-project.org/lppl.txt
% and version 1.3c or later is part of all distributions of LaTeX
% version 2008 or later.
%
% This file has the LPPL maintenance status "maintained".
%
%<2ekernel>%%% From File: ltluatex.dtx
%<plain>\ifx\newluafunction\undefined\else\expandafter\endinput\fi
%<tex>\ifx
%<tex>  \ProvidesFile\undefined\begingroup\def\ProvidesFile
%<tex>  #1#2[#3]{\endgroup\immediate\write-1{File: #1 #3}}
%<tex>\fi
%<*dtx>
\ProvidesFile{ltluatex.dtx}
%</dtx>
%<plain>\ProvidesFile{ltluatex.tex}
%<*plain>
% \fi
% \ProvidesFile{ltluatex.dtx}
[2024/02/11 v1.2c
% LaTeX Kernel (LuaTeX support)^^A
%\iffalse
%<plain>   LuaTeX support for plain TeX (core)%
%\fi
]
% \iffalse
%</plain>  
%<*tex>
\edef\etatcatcode{\the\catcode`\@}
\catcode`\@=11
%</tex>
%<*driver>
\documentclass{ltxdoc}

\providecommand\InternalDetectionOff{}
\providecommand\InternalDetectionOn{}

\GetFileInfo{ltluatex.dtx}
\begin{document}
\title{\filename\\(Lua\TeX{}-specific support)}
\author{David Carlisle and Joseph Wright\footnote{Significant portions
  of the code here are adapted/simplified from the packages \textsf{luatex} and
  \textsf{luatexbase} written by Heiko Oberdiek, \'{E}lie Roux,
  Manuel P\'{e}gouri\'{e}-Gonnar and Philipp Gesang.}}
\date{\filedate}
\maketitle
\setcounter{tocdepth}{2}
\tableofcontents
\DocInput{\filename}
\end{document}
%</driver>
% \fi
%
%
% \section{Overview}
%
% Lua\TeX{} adds a number of engine-specific functions to \TeX{}. Several of
% these require set up that is best done in the kernel or need related support
% functions. This file provides \emph{basic} support for Lua\TeX{} at the
% \LaTeXe{} kernel level plus as a loadable file which can be used with
% plain \TeX{} and \LaTeX{}.
%
% This file contains code for both \TeX{} (to be stored as part of the format)
% and Lua (to be loaded at the start of each job). In the Lua code, the kernel
% uses the namespace |luatexbase|.
%
% The following |\count| registers are used here for register allocation:
% \begin{itemize}
%  \item[\texttt{\string\e@alloc@attribute@count}] Attributes (default~258)
%  \item[\texttt{\string\e@alloc@ccodetable@count}] Category code tables
%    (default~259)
%  \item[\texttt{\string\e@alloc@luafunction@count}] Lua functions
%    (default~260)
%  \item[\texttt{\string\e@alloc@whatsit@count}] User whatsits (default~261)
%  \item[\texttt{\string\e@alloc@bytecode@count}] Lua bytecodes (default~262)
%  \item[\texttt{\string\e@alloc@luachunk@count}] Lua chunks (default~263)
% \end{itemize}
% (|\count 256| is used for |\newmarks| allocation and |\count 257|
% is used for\linebreak
% |\newXeTeXintercharclass| with Xe\TeX{}, with code defined in
% \texttt{ltfinal.dtx}).
% With any \LaTeXe{} kernel from 2015 onward these registers are part of
% the block in the extended area reserved by the kernel (prior to 2015 the
% \LaTeXe{} kernel did not provide any functionality for the extended
% allocation area).
%
% \section{Core \TeX{} functionality}
%
% The commands defined here are defined for
% possible inclusion in a future \LaTeX{} format, however also extracted
% to the file |ltluatex.tex| which may be used with older \LaTeX\
% formats, and with plain \TeX.
%
% \noindent
% \DescribeMacro{\newattribute}
% |\newattribute{|\meta{attribute}|}|\\
% Defines a named \cs{attribute}, indexed from~$1$
% (\emph{i.e.}~|\attribute0| is never defined). Attributes initially
% have the marker value |-"7FFFFFFF| (`unset') set by the engine.
%
% \noindent
% \DescribeMacro{\newcatcodetable}
% |\newcatcodetable{|\meta{catcodetable}|}|\\
% Defines a named \cs{catcodetable}, indexed from~$1$
% (|\catcodetable0| is never assigned). A new catcode table will be
% populated with exactly those values assigned by Ini\TeX{} (as described
% in the Lua\TeX{} manual).
%
% \noindent
% \DescribeMacro{\newluafunction}
% |\newluafunction{|\meta{function}|}|\\
% Defines a named \cs{luafunction}, indexed from~$1$. (Lua indexes
% tables from $1$ so |\luafunction0| is not available).
%
% \noindent
% \DescribeMacro{\newluacmd}
% |\newluadef{|\meta{function}|}|\\
% Like \cs{newluafunction}, but defines the command using \cs{luadef}
% instead of just assigning an integer.
%
% \noindent
% \DescribeMacro{\newprotectedluacmd}
% |\newluadef{|\meta{function}|}|\\
% Like \cs{newluacmd}, but the defined command is not expandable.
%
% \noindent
% \DescribeMacro{\newwhatsit}
% |\newwhatsit{|\meta{whatsit}|}|\\
% Defines a custom \cs{whatsit}, indexed from~$1$.
%
% \noindent
% \DescribeMacro{\newluabytecode}
% |\newluabytecode{|\meta{bytecode}|}|\\
% Allocates a number for Lua bytecode register, indexed from~$1$.
%
% \noindent
% \DescribeMacro{\newluachunkname}
% |newluachunkname{|\meta{chunkname}|}|\\
% Allocates a number for Lua chunk register, indexed from~$1$.
% Also enters the name of the register (without backslash) into the
% \verb|lua.name| table to be used in stack traces.
%
% \noindent
% \DescribeMacro{\catcodetable@initex}
% \DescribeMacro{\catcodetable@string}
% \DescribeMacro{\catcodetable@latex}
% \DescribeMacro{\catcodetable@atletter}
% Predefined category code tables with the obvious assignments. Note
% that the |latex| and |atletter| tables set the full Unicode range
% to the codes predefined by the kernel.
%
% \noindent
% \DescribeMacro{\setattribute}
% \DescribeMacro{\unsetattribute}
% |\setattribute{|\meta{attribute}|}{|\meta{value}|}|\\
% |\unsetattribute{|\meta{attribute}|}|\\
% Set and unset attributes in a manner analogous to |\setlength|. Note that
% attributes take a marker value when unset so this operation is distinct
% from setting the value to zero.
%
% \section{Plain \TeX\ interface}
%
% The \textsf{ltluatex} interface may be used with plain \TeX\ using
% |% \iffalse meta-comment
%
% Copyright (C) 2015-2021
% The LaTeX Project and any individual authors listed elsewhere
% in this file.
%
% This file is part of the LaTeX base system.
% -------------------------------------------
%
% It may be distributed and/or modified under the
% conditions of the LaTeX Project Public License, either version 1.3c
% of this license or (at your option) any later version.
% The latest version of this license is in
%    https://www.latex-project.org/lppl.txt
% and version 1.3c or later is part of all distributions of LaTeX
% version 2008 or later.
%
% This file has the LPPL maintenance status "maintained".
%
%<2ekernel>%%% From File: ltluatex.dtx
%<plain>\ifx\newluafunction\undefined\else\expandafter\endinput\fi
%<tex>\ifx
%<tex>  \ProvidesFile\undefined\begingroup\def\ProvidesFile
%<tex>  #1#2[#3]{\endgroup\immediate\write-1{File: #1 #3}}
%<tex>\fi
%<plain>\ProvidesFile{ltluatex.tex}%
%<*driver>
\ProvidesFile{ltluatex.dtx}
%</driver>
%<*tex>
[2021/11/17 v1.1w
%</tex>
%<plain>  LuaTeX support for plain TeX (core)
%<*tex>
]
\edef\etatcatcode{\the\catcode`\@}
\catcode`\@=11
%</tex>
%<*driver>
\documentclass{ltxdoc}

\providecommand\InternalDetectionOff{}
\providecommand\InternalDetectionOn{}

\GetFileInfo{ltluatex.dtx}
\begin{document}
\title{\filename\\(Lua\TeX{}-specific support)}
\author{David Carlisle and Joseph Wright\footnote{Significant portions
  of the code here are adapted/simplified from the packages \textsf{luatex} and
  \textsf{luatexbase} written by Heiko Oberdiek, \'{E}lie Roux,
  Manuel P\'{e}gouri\'{e}-Gonnar and Philipp Gesang.}}
\date{\filedate}
\maketitle
\setcounter{tocdepth}{2}
\tableofcontents
\DocInput{\filename}
\end{document}
%</driver>
% \fi
%
%
% \section{Overview}
%
% Lua\TeX{} adds a number of engine-specific functions to \TeX{}. Several of
% these require set up that is best done in the kernel or need related support
% functions. This file provides \emph{basic} support for Lua\TeX{} at the
% \LaTeXe{} kernel level plus as a loadable file which can be used with
% plain \TeX{} and \LaTeX{}.
%
% This file contains code for both \TeX{} (to be stored as part of the format)
% and Lua (to be loaded at the start of each job). In the Lua code, the kernel
% uses the namespace |luatexbase|.
%
% The following |\count| registers are used here for register allocation:
% \begin{itemize}
%  \item[\texttt{\string\e@alloc@attribute@count}] Attributes (default~258)
%  \item[\texttt{\string\e@alloc@ccodetable@count}] Category code tables
%    (default~259)
%  \item[\texttt{\string\e@alloc@luafunction@count}] Lua functions
%    (default~260)
%  \item[\texttt{\string\e@alloc@whatsit@count}] User whatsits (default~261)
%  \item[\texttt{\string\e@alloc@bytecode@count}] Lua bytecodes (default~262)
%  \item[\texttt{\string\e@alloc@luachunk@count}] Lua chunks (default~263)
% \end{itemize}
% (|\count 256| is used for |\newmarks| allocation and |\count 257|
% is used for\linebreak
% |\newXeTeXintercharclass| with Xe\TeX{}, with code defined in
% \texttt{ltfinal.dtx}).
% With any \LaTeXe{} kernel from 2015 onward these registers are part of
% the block in the extended area reserved by the kernel (prior to 2015 the
% \LaTeXe{} kernel did not provide any functionality for the extended
% allocation area).
%
% \section{Core \TeX{} functionality}
%
% The commands defined here are defined for
% possible inclusion in a future \LaTeX{} format, however also extracted
% to the file |ltluatex.tex| which may be used with older \LaTeX\
% formats, and with plain \TeX.
%
% \noindent
% \DescribeMacro{\newattribute}
% |\newattribute{|\meta{attribute}|}|\\
% Defines a named \cs{attribute}, indexed from~$1$
% (\emph{i.e.}~|\attribute0| is never defined). Attributes initially
% have the marker value |-"7FFFFFFF| (`unset') set by the engine.
%
% \noindent
% \DescribeMacro{\newcatcodetable}
% |\newcatcodetable{|\meta{catcodetable}|}|\\
% Defines a named \cs{catcodetable}, indexed from~$1$
% (|\catcodetable0| is never assigned). A new catcode table will be
% populated with exactly those values assigned by Ini\TeX{} (as described
% in the Lua\TeX{} manual).
%
% \noindent
% \DescribeMacro{\newluafunction}
% |\newluafunction{|\meta{function}|}|\\
% Defines a named \cs{luafunction}, indexed from~$1$. (Lua indexes
% tables from $1$ so |\luafunction0| is not available).
%
% \noindent
% \DescribeMacro{\newwhatsit}
% |\newwhatsit{|\meta{whatsit}|}|\\
% Defines a custom \cs{whatsit}, indexed from~$1$.
%
% \noindent
% \DescribeMacro{\newluabytecode}
% |\newluabytecode{|\meta{bytecode}|}|\\
% Allocates a number for Lua bytecode register, indexed from~$1$.
%
% \noindent
% \DescribeMacro{\newluachunkname}
% |newluachunkname{|\meta{chunkname}|}|\\
% Allocates a number for Lua chunk register, indexed from~$1$.
% Also enters the name of the register (without backslash) into the
% \verb|lua.name| table to be used in stack traces.
%
% \noindent
% \DescribeMacro{\catcodetable@initex}
% \DescribeMacro{\catcodetable@string}
% \DescribeMacro{\catcodetable@latex}
% \DescribeMacro{\catcodetable@atletter}
% Predefined category code tables with the obvious assignments. Note
% that the |latex| and |atletter| tables set the full Unicode range
% to the codes predefined by the kernel.
%
% \noindent
% \DescribeMacro{\setattribute}
% \DescribeMacro{\unsetattribute}
% |\setattribute{|\meta{attribute}|}{|\meta{value}|}|\\
% |\unsetattribute{|\meta{attribute}|}|\\
% Set and unset attributes in a manner analogous to |\setlength|. Note that
% attributes take a marker value when unset so this operation is distinct
% from setting the value to zero.
%
% \section{Plain \TeX\ interface}
%
% The \textsf{ltluatex} interface may be used with plain \TeX\ using
% |% \iffalse meta-comment
%
% Copyright (C) 2015-2021
% The LaTeX Project and any individual authors listed elsewhere
% in this file.
%
% This file is part of the LaTeX base system.
% -------------------------------------------
%
% It may be distributed and/or modified under the
% conditions of the LaTeX Project Public License, either version 1.3c
% of this license or (at your option) any later version.
% The latest version of this license is in
%    https://www.latex-project.org/lppl.txt
% and version 1.3c or later is part of all distributions of LaTeX
% version 2008 or later.
%
% This file has the LPPL maintenance status "maintained".
%
%<2ekernel>%%% From File: ltluatex.dtx
%<plain>\ifx\newluafunction\undefined\else\expandafter\endinput\fi
%<tex>\ifx
%<tex>  \ProvidesFile\undefined\begingroup\def\ProvidesFile
%<tex>  #1#2[#3]{\endgroup\immediate\write-1{File: #1 #3}}
%<tex>\fi
%<plain>\ProvidesFile{ltluatex.tex}%
%<*driver>
\ProvidesFile{ltluatex.dtx}
%</driver>
%<*tex>
[2021/11/17 v1.1w
%</tex>
%<plain>  LuaTeX support for plain TeX (core)
%<*tex>
]
\edef\etatcatcode{\the\catcode`\@}
\catcode`\@=11
%</tex>
%<*driver>
\documentclass{ltxdoc}

\providecommand\InternalDetectionOff{}
\providecommand\InternalDetectionOn{}

\GetFileInfo{ltluatex.dtx}
\begin{document}
\title{\filename\\(Lua\TeX{}-specific support)}
\author{David Carlisle and Joseph Wright\footnote{Significant portions
  of the code here are adapted/simplified from the packages \textsf{luatex} and
  \textsf{luatexbase} written by Heiko Oberdiek, \'{E}lie Roux,
  Manuel P\'{e}gouri\'{e}-Gonnar and Philipp Gesang.}}
\date{\filedate}
\maketitle
\setcounter{tocdepth}{2}
\tableofcontents
\DocInput{\filename}
\end{document}
%</driver>
% \fi
%
%
% \section{Overview}
%
% Lua\TeX{} adds a number of engine-specific functions to \TeX{}. Several of
% these require set up that is best done in the kernel or need related support
% functions. This file provides \emph{basic} support for Lua\TeX{} at the
% \LaTeXe{} kernel level plus as a loadable file which can be used with
% plain \TeX{} and \LaTeX{}.
%
% This file contains code for both \TeX{} (to be stored as part of the format)
% and Lua (to be loaded at the start of each job). In the Lua code, the kernel
% uses the namespace |luatexbase|.
%
% The following |\count| registers are used here for register allocation:
% \begin{itemize}
%  \item[\texttt{\string\e@alloc@attribute@count}] Attributes (default~258)
%  \item[\texttt{\string\e@alloc@ccodetable@count}] Category code tables
%    (default~259)
%  \item[\texttt{\string\e@alloc@luafunction@count}] Lua functions
%    (default~260)
%  \item[\texttt{\string\e@alloc@whatsit@count}] User whatsits (default~261)
%  \item[\texttt{\string\e@alloc@bytecode@count}] Lua bytecodes (default~262)
%  \item[\texttt{\string\e@alloc@luachunk@count}] Lua chunks (default~263)
% \end{itemize}
% (|\count 256| is used for |\newmarks| allocation and |\count 257|
% is used for\linebreak
% |\newXeTeXintercharclass| with Xe\TeX{}, with code defined in
% \texttt{ltfinal.dtx}).
% With any \LaTeXe{} kernel from 2015 onward these registers are part of
% the block in the extended area reserved by the kernel (prior to 2015 the
% \LaTeXe{} kernel did not provide any functionality for the extended
% allocation area).
%
% \section{Core \TeX{} functionality}
%
% The commands defined here are defined for
% possible inclusion in a future \LaTeX{} format, however also extracted
% to the file |ltluatex.tex| which may be used with older \LaTeX\
% formats, and with plain \TeX.
%
% \noindent
% \DescribeMacro{\newattribute}
% |\newattribute{|\meta{attribute}|}|\\
% Defines a named \cs{attribute}, indexed from~$1$
% (\emph{i.e.}~|\attribute0| is never defined). Attributes initially
% have the marker value |-"7FFFFFFF| (`unset') set by the engine.
%
% \noindent
% \DescribeMacro{\newcatcodetable}
% |\newcatcodetable{|\meta{catcodetable}|}|\\
% Defines a named \cs{catcodetable}, indexed from~$1$
% (|\catcodetable0| is never assigned). A new catcode table will be
% populated with exactly those values assigned by Ini\TeX{} (as described
% in the Lua\TeX{} manual).
%
% \noindent
% \DescribeMacro{\newluafunction}
% |\newluafunction{|\meta{function}|}|\\
% Defines a named \cs{luafunction}, indexed from~$1$. (Lua indexes
% tables from $1$ so |\luafunction0| is not available).
%
% \noindent
% \DescribeMacro{\newwhatsit}
% |\newwhatsit{|\meta{whatsit}|}|\\
% Defines a custom \cs{whatsit}, indexed from~$1$.
%
% \noindent
% \DescribeMacro{\newluabytecode}
% |\newluabytecode{|\meta{bytecode}|}|\\
% Allocates a number for Lua bytecode register, indexed from~$1$.
%
% \noindent
% \DescribeMacro{\newluachunkname}
% |newluachunkname{|\meta{chunkname}|}|\\
% Allocates a number for Lua chunk register, indexed from~$1$.
% Also enters the name of the register (without backslash) into the
% \verb|lua.name| table to be used in stack traces.
%
% \noindent
% \DescribeMacro{\catcodetable@initex}
% \DescribeMacro{\catcodetable@string}
% \DescribeMacro{\catcodetable@latex}
% \DescribeMacro{\catcodetable@atletter}
% Predefined category code tables with the obvious assignments. Note
% that the |latex| and |atletter| tables set the full Unicode range
% to the codes predefined by the kernel.
%
% \noindent
% \DescribeMacro{\setattribute}
% \DescribeMacro{\unsetattribute}
% |\setattribute{|\meta{attribute}|}{|\meta{value}|}|\\
% |\unsetattribute{|\meta{attribute}|}|\\
% Set and unset attributes in a manner analogous to |\setlength|. Note that
% attributes take a marker value when unset so this operation is distinct
% from setting the value to zero.
%
% \section{Plain \TeX\ interface}
%
% The \textsf{ltluatex} interface may be used with plain \TeX\ using
% |% \iffalse meta-comment
%
% Copyright (C) 2015-2021
% The LaTeX Project and any individual authors listed elsewhere
% in this file.
%
% This file is part of the LaTeX base system.
% -------------------------------------------
%
% It may be distributed and/or modified under the
% conditions of the LaTeX Project Public License, either version 1.3c
% of this license or (at your option) any later version.
% The latest version of this license is in
%    https://www.latex-project.org/lppl.txt
% and version 1.3c or later is part of all distributions of LaTeX
% version 2008 or later.
%
% This file has the LPPL maintenance status "maintained".
%
%<2ekernel>%%% From File: ltluatex.dtx
%<plain>\ifx\newluafunction\undefined\else\expandafter\endinput\fi
%<tex>\ifx
%<tex>  \ProvidesFile\undefined\begingroup\def\ProvidesFile
%<tex>  #1#2[#3]{\endgroup\immediate\write-1{File: #1 #3}}
%<tex>\fi
%<plain>\ProvidesFile{ltluatex.tex}%
%<*driver>
\ProvidesFile{ltluatex.dtx}
%</driver>
%<*tex>
[2021/11/17 v1.1w
%</tex>
%<plain>  LuaTeX support for plain TeX (core)
%<*tex>
]
\edef\etatcatcode{\the\catcode`\@}
\catcode`\@=11
%</tex>
%<*driver>
\documentclass{ltxdoc}

\providecommand\InternalDetectionOff{}
\providecommand\InternalDetectionOn{}

\GetFileInfo{ltluatex.dtx}
\begin{document}
\title{\filename\\(Lua\TeX{}-specific support)}
\author{David Carlisle and Joseph Wright\footnote{Significant portions
  of the code here are adapted/simplified from the packages \textsf{luatex} and
  \textsf{luatexbase} written by Heiko Oberdiek, \'{E}lie Roux,
  Manuel P\'{e}gouri\'{e}-Gonnar and Philipp Gesang.}}
\date{\filedate}
\maketitle
\setcounter{tocdepth}{2}
\tableofcontents
\DocInput{\filename}
\end{document}
%</driver>
% \fi
%
%
% \section{Overview}
%
% Lua\TeX{} adds a number of engine-specific functions to \TeX{}. Several of
% these require set up that is best done in the kernel or need related support
% functions. This file provides \emph{basic} support for Lua\TeX{} at the
% \LaTeXe{} kernel level plus as a loadable file which can be used with
% plain \TeX{} and \LaTeX{}.
%
% This file contains code for both \TeX{} (to be stored as part of the format)
% and Lua (to be loaded at the start of each job). In the Lua code, the kernel
% uses the namespace |luatexbase|.
%
% The following |\count| registers are used here for register allocation:
% \begin{itemize}
%  \item[\texttt{\string\e@alloc@attribute@count}] Attributes (default~258)
%  \item[\texttt{\string\e@alloc@ccodetable@count}] Category code tables
%    (default~259)
%  \item[\texttt{\string\e@alloc@luafunction@count}] Lua functions
%    (default~260)
%  \item[\texttt{\string\e@alloc@whatsit@count}] User whatsits (default~261)
%  \item[\texttt{\string\e@alloc@bytecode@count}] Lua bytecodes (default~262)
%  \item[\texttt{\string\e@alloc@luachunk@count}] Lua chunks (default~263)
% \end{itemize}
% (|\count 256| is used for |\newmarks| allocation and |\count 257|
% is used for\linebreak
% |\newXeTeXintercharclass| with Xe\TeX{}, with code defined in
% \texttt{ltfinal.dtx}).
% With any \LaTeXe{} kernel from 2015 onward these registers are part of
% the block in the extended area reserved by the kernel (prior to 2015 the
% \LaTeXe{} kernel did not provide any functionality for the extended
% allocation area).
%
% \section{Core \TeX{} functionality}
%
% The commands defined here are defined for
% possible inclusion in a future \LaTeX{} format, however also extracted
% to the file |ltluatex.tex| which may be used with older \LaTeX\
% formats, and with plain \TeX.
%
% \noindent
% \DescribeMacro{\newattribute}
% |\newattribute{|\meta{attribute}|}|\\
% Defines a named \cs{attribute}, indexed from~$1$
% (\emph{i.e.}~|\attribute0| is never defined). Attributes initially
% have the marker value |-"7FFFFFFF| (`unset') set by the engine.
%
% \noindent
% \DescribeMacro{\newcatcodetable}
% |\newcatcodetable{|\meta{catcodetable}|}|\\
% Defines a named \cs{catcodetable}, indexed from~$1$
% (|\catcodetable0| is never assigned). A new catcode table will be
% populated with exactly those values assigned by Ini\TeX{} (as described
% in the Lua\TeX{} manual).
%
% \noindent
% \DescribeMacro{\newluafunction}
% |\newluafunction{|\meta{function}|}|\\
% Defines a named \cs{luafunction}, indexed from~$1$. (Lua indexes
% tables from $1$ so |\luafunction0| is not available).
%
% \noindent
% \DescribeMacro{\newwhatsit}
% |\newwhatsit{|\meta{whatsit}|}|\\
% Defines a custom \cs{whatsit}, indexed from~$1$.
%
% \noindent
% \DescribeMacro{\newluabytecode}
% |\newluabytecode{|\meta{bytecode}|}|\\
% Allocates a number for Lua bytecode register, indexed from~$1$.
%
% \noindent
% \DescribeMacro{\newluachunkname}
% |newluachunkname{|\meta{chunkname}|}|\\
% Allocates a number for Lua chunk register, indexed from~$1$.
% Also enters the name of the register (without backslash) into the
% \verb|lua.name| table to be used in stack traces.
%
% \noindent
% \DescribeMacro{\catcodetable@initex}
% \DescribeMacro{\catcodetable@string}
% \DescribeMacro{\catcodetable@latex}
% \DescribeMacro{\catcodetable@atletter}
% Predefined category code tables with the obvious assignments. Note
% that the |latex| and |atletter| tables set the full Unicode range
% to the codes predefined by the kernel.
%
% \noindent
% \DescribeMacro{\setattribute}
% \DescribeMacro{\unsetattribute}
% |\setattribute{|\meta{attribute}|}{|\meta{value}|}|\\
% |\unsetattribute{|\meta{attribute}|}|\\
% Set and unset attributes in a manner analogous to |\setlength|. Note that
% attributes take a marker value when unset so this operation is distinct
% from setting the value to zero.
%
% \section{Plain \TeX\ interface}
%
% The \textsf{ltluatex} interface may be used with plain \TeX\ using
% |\input{ltluatex}|. This inputs |ltluatex.tex| which inputs
% |etex.src| (or |etex.sty| if used with \LaTeX)
% if it is not already input, and then defines some internal commands to
% allow the \textsf{ltluatex} interface to be defined.
%
% The \textsf{luatexbase} package interface may also be used in plain \TeX,
% as before, by inputting the package |\input luatexbase.sty|. The new
% version of \textsf{luatexbase} is based on this \textsf{ltluatex}
% code but implements a compatibility layer providing the interface
% of the original package.
%
% \section{Lua functionality}
%
% \begingroup
%
% \begingroup\lccode`~=`_
% \lowercase{\endgroup\let~}_
% \catcode`_=12
%
% \subsection{Allocators in Lua}
%
% \DescribeMacro{new_attribute}
% |luatexbase.new_attribute(|\meta{attribute}|)|\\
% Returns an allocation number for the \meta{attribute}, indexed from~$1$.
% The attribute will be initialised with the marker value |-"7FFFFFFF|
% (`unset'). The attribute allocation sequence is shared with the \TeX{}
% code but this function does \emph{not} define a token using
% |\attributedef|.
% The attribute name is recorded in the |attributes| table. A
% metatable is provided so that the table syntax can be used
% consistently for attributes declared in \TeX\ or Lua.
%
% \noindent
% \DescribeMacro{new_whatsit}
% |luatexbase.new_whatsit(|\meta{whatsit}|)|\\
% Returns an allocation number for the custom \meta{whatsit}, indexed from~$1$.
%
% \noindent
% \DescribeMacro{new_bytecode}
% |luatexbase.new_bytecode(|\meta{bytecode}|)|\\
% Returns an allocation number for a bytecode register, indexed from~$1$.
% The optional \meta{name} argument is just used for logging.
%
% \noindent
% \DescribeMacro{new_chunkname}
% |luatexbase.new_chunkname(|\meta{chunkname}|)|\\
% Returns an allocation number for a Lua chunk name for use with
% |\directlua| and |\latelua|, indexed from~$1$.
% The number is returned and also \meta{name} argument is added to the
% |lua.name| array at that index.
%
% \begin{sloppypar}
% \noindent
% \DescribeMacro{new_luafunction}
% |luatexbase.new_luafunction(|\meta{functionname}|)|\\
% Returns an allocation number for a lua function for use
% with |\luafunction|, |\lateluafunction|, and |\luadef|,
% indexed from~$1$. The optional \meta{functionname} argument
% is just used for logging.
% \end{sloppypar}
%
% These functions all require access to a named \TeX{} count register
% to manage their allocations. The standard names are those defined
% above for access from \TeX{}, \emph{e.g.}~\string\e@alloc@attribute@count,
% but these can be adjusted by defining the variable
% \texttt{\meta{type}\_count\_name} before loading |ltluatex.lua|, for example
% \begin{verbatim}
% local attribute_count_name = "attributetracker"
% require("ltluatex")
% \end{verbatim}
% would use a \TeX{} |\count| (|\countdef|'d token) called |attributetracker|
% in place of \string\e@alloc@attribute@count.
%
% \subsection{Lua access to \TeX{} register numbers}
%
% \DescribeMacro{registernumber}
% |luatexbase.registernumer(|\meta{name}|)|\\
% Sometimes (notably in the case of Lua attributes) it is necessary to
% access a register \emph{by number} that has been allocated by \TeX{}.
% This package provides a function to look up the relevant number
% using Lua\TeX{}'s internal tables. After for example
% |\newattribute\myattrib|, |\myattrib| would be defined by (say)
% |\myattrib=\attribute15|.  |luatexbase.registernumer("myattrib")|
% would then return the register number, $15$ in this case. If the string passed
% as argument does not correspond to a token defined by |\attributedef|,
% |\countdef| or similar commands, the Lua value |false| is returned.
%
% As an example, consider the input:
%\begin{verbatim}
% \newcommand\test[1]{%
% \typeout{#1: \expandafter\meaning\csname#1\endcsname^^J
% \space\space\space\space
% \directlua{tex.write(luatexbase.registernumber("#1") or "bad input")}%
% }}
%
% \test{undefinedrubbish}
%
% \test{space}
%
% \test{hbox}
%
% \test{@MM}
%
% \test{@tempdima}
% \test{@tempdimb}
%
% \test{strutbox}
%
% \test{sixt@@n}
%
% \attrbutedef\myattr=12
% \myattr=200
% \test{myattr}
%
%\end{verbatim}
%
% If the demonstration code is processed with Lua\LaTeX{} then the following
% would be produced in the log and terminal output.
%\begin{verbatim}
% undefinedrubbish: \relax
%      bad input
% space: macro:->
%      bad input
% hbox: \hbox
%      bad input
% @MM: \mathchar"4E20
%      20000
% @tempdima: \dimen14
%      14
% @tempdimb: \dimen15
%      15
% strutbox: \char"B
%      11
% sixt@@n: \char"10
%      16
% myattr: \attribute12
%      12
%\end{verbatim}
%
% Notice how undefined commands, or commands unrelated to registers
% do not produce an error, just return |false| and so print
% |bad input| here. Note also that commands defined by |\newbox| work and
% return the number of the box register even though the actual command
% holding this number is a |\chardef| defined token (there is no
% |\boxdef|).
%
% \subsection{Module utilities}
%
% \DescribeMacro{provides_module}
% |luatexbase.provides_module(|\meta{info}|)|\\
% This function is used by modules to identify themselves; the |info| should be
% a table containing information about the module. The required field
% |name| must contain the name of the module. It is recommended to provide a
% field |date| in the usual \LaTeX{} format |yyyy/mm/dd|. Optional fields
% |version| (a string) and |description| may be used if present. This
% information will be recorded in the log. Other fields are ignored.
%
% \noindent
% \DescribeMacro{module_info}
% \DescribeMacro{module_warning}
% \DescribeMacro{module_error}
% |luatexbase.module_info(|\meta{module}, \meta{text}|)|\\
% |luatexbase.module_warning(|\meta{module}, \meta{text}|)|\\
% |luatexbase.module_error(|\meta{module}, \meta{text}|)|\\
% These functions are similar to \LaTeX{}'s |\PackageError|, |\PackageWarning|
% and |\PackageInfo| in the way they format the output.  No automatic line
% breaking is done, you may still use |\n| as usual for that, and the name of
% the package will be prepended to each output line.
%
% Note that |luatexbase.module_error| raises an actual Lua error with |error()|,
% which currently means a call stack will be dumped. While this may not
% look pretty, at least it provides useful information for tracking the
% error down.
%
% \subsection{Callback management}
%
% \noindent
% \DescribeMacro{add_to_callback}
% |luatexbase.add_to_callback(|^^A
% \meta{callback}, \meta{function}, \meta{description}|)|
% Registers the \meta{function} into the \meta{callback} with a textual
% \meta{description} of the function. Functions are inserted into the callback
% in the order loaded.
%
% \noindent
% \DescribeMacro{remove_from_callback}
% |luatexbase.remove_from_callback(|\meta{callback}, \meta{description}|)|
% Removes the callback function with \meta{description} from the \meta{callback}.
% The removed function and its description
% are returned as the results of this function.
%
% \noindent
% \DescribeMacro{in_callback}
% |luatexbase.in_callback(|\meta{callback}, \meta{description}|)|
% Checks if the \meta{description} matches one of the functions added
% to the list for the \meta{callback}, returning a boolean value.
%
% \noindent
% \DescribeMacro{disable_callback}
% |luatexbase.disable_callback(|\meta{callback}|)|
% Sets the \meta{callback} to \texttt{false} as described in the Lua\TeX{}
% manual for the underlying \texttt{callback.register} built-in. Callbacks
% will only be set to false (and thus be skipped entirely) if there are
% no functions registered using the callback.
%
% \noindent
% \DescribeMacro{callback_descriptions}
% A list of the descriptions of functions registered to the specified
% callback is returned. |{}| is returned if there are no functions registered.
%
% \noindent
% \DescribeMacro{create_callback}
% |luatexbase.create_callback(|\meta{name},meta{type},\meta{default}|)|
% Defines a user defined callback. The last argument is a default
% function or |false|.
%
% \noindent
% \DescribeMacro{call_callback}
% |luatexbase.call_callback(|\meta{name},\ldots|)|
% Calls a user defined callback with the supplied arguments.
%
% \endgroup
%
% \StopEventually{}
%
% \section{Implementation}
%
%    \begin{macrocode}
%<*2ekernel|tex|latexrelease>
%<2ekernel|latexrelease>\ifx\directlua\@undefined\else
%    \end{macrocode}
%
%
% \changes{v1.0j}{2015/12/02}{Remove nonlocal iteration variables (PHG)}
% \changes{v1.0j}{2015/12/02}{Assorted typos fixed (PHG)}
% \changes{v1.0j}{2015/12/02}{Remove unreachable code after calls to error() (PHG)}
% \subsection{Minimum Lua\TeX{} version}
%
% Lua\TeX{} has changed a lot over time. In the kernel support for ancient
% versions is not provided: trying to build a format with a very old binary
% therefore gives some information in the log and loading stops. The cut-off
% selected here relates to the tree-searching behaviour of |require()|:
% from version~0.60, Lua\TeX{} will correctly find Lua files in the |texmf|
% tree without `help'.
%    \begin{macrocode}
%<latexrelease>\IncludeInRelease{2015/10/01}
%<latexrelease>                 {\newluafunction}{LuaTeX}%
\ifnum\luatexversion<60 %
  \wlog{***************************************************}
  \wlog{* LuaTeX version too old for ltluatex support *}
  \wlog{***************************************************}
  \expandafter\endinput
\fi
%    \end{macrocode}
%
% \changes{v1.1n}{2020/06/10}{Define \cs{@gobble}/\cs{@firstofone} even for \LaTeX\ to allow early loading.}
% Two simple \LaTeX\ macros from |ltdefns.dtx| have to be defined here
% because ltdefns.dtx is not loaded yet when ltluatex.dtx is executed.
%    \begin{macrocode}
\long\def\@gobble#1{}
\long\def\@firstofone#1{#1}
%    \end{macrocode}
%
% \subsection{Older \LaTeX{}/Plain \TeX\ setup}
%
%    \begin{macrocode}
%<*tex>
%    \end{macrocode}
%
% Older \LaTeX{} formats don't have the primitives with `native' names:
% sort that out. If they already exist this will still be safe.
%    \begin{macrocode}
\directlua{tex.enableprimitives("",tex.extraprimitives("luatex"))}
%    \end{macrocode}
%
%    \begin{macrocode}
\ifx\e@alloc\@undefined
%    \end{macrocode}
%
% In pre-2014 \LaTeX{}, or plain \TeX{}, load |etex.{sty,src}|.
%    \begin{macrocode}
  \ifx\documentclass\@undefined
    \ifx\loccount\@undefined
      \input{etex.src}%
    \fi
    \catcode`\@=11 %
    \outer\expandafter\def\csname newfam\endcsname
                          {\alloc@8\fam\chardef\et@xmaxfam}
  \else
    \RequirePackage{etex}
    \expandafter\def\csname newfam\endcsname
                    {\alloc@8\fam\chardef\et@xmaxfam}
    \expandafter\let\expandafter\new@mathgroup\csname newfam\endcsname
  \fi
%    \end{macrocode}
%
% \subsubsection{Fixes to \texttt{etex.src}/\texttt{etex.sty}}
%
% These could and probably should be made directly in an
% update to |etex.src| which already has some Lua\TeX-specific
% code, but does not define the correct range for Lua\TeX.
%
% 2015-07-13 higher range in luatex.
%    \begin{macrocode}
\edef \et@xmaxregs {\ifx\directlua\@undefined 32768\else 65536\fi}
%    \end{macrocode}
% luatex/xetex also allow more math fam.
%    \begin{macrocode}
\edef \et@xmaxfam {\ifx\Umathcode\@undefined\sixt@@n\else\@cclvi\fi}
%    \end{macrocode}
%
%    \begin{macrocode}
\count 270=\et@xmaxregs % locally allocates \count registers
\count 271=\et@xmaxregs % ditto for \dimen registers
\count 272=\et@xmaxregs % ditto for \skip registers
\count 273=\et@xmaxregs % ditto for \muskip registers
\count 274=\et@xmaxregs % ditto for \box registers
\count 275=\et@xmaxregs % ditto for \toks registers
\count 276=\et@xmaxregs % ditto for \marks classes
%    \end{macrocode}
%
% and 256 or 16 fam. (Done above due to plain/\LaTeX\ differences in
% \textsf{ltluatex}.)
%    \begin{macrocode}
% \outer\def\newfam{\alloc@8\fam\chardef\et@xmaxfam}
%    \end{macrocode}
%
% End of proposed changes to \texttt{etex.src}
%
% \subsubsection{luatex specific settings}
%
% Switch to global cf |luatex.sty| to leave room for inserts
% not really needed for luatex but possibly most compatible
% with existing use.
%    \begin{macrocode}
\expandafter\let\csname newcount\expandafter\expandafter\endcsname
                \csname globcount\endcsname
\expandafter\let\csname newdimen\expandafter\expandafter\endcsname
                \csname globdimen\endcsname
\expandafter\let\csname newskip\expandafter\expandafter\endcsname
                \csname globskip\endcsname
\expandafter\let\csname newbox\expandafter\expandafter\endcsname
                \csname globbox\endcsname
%    \end{macrocode}
%
% Define|\e@alloc| as in latex (the existing macros in |etex.src|
% hard to extend to further register types as they assume specific
% 26x and 27x count range. For compatibility the existing register
% allocation is not changed.
%
%    \begin{macrocode}
\chardef\e@alloc@top=65535
\let\e@alloc@chardef\chardef
%    \end{macrocode}
%
%    \begin{macrocode}
\def\e@alloc#1#2#3#4#5#6{%
  \global\advance#3\@ne
  \e@ch@ck{#3}{#4}{#5}#1%
  \allocationnumber#3\relax
  \global#2#6\allocationnumber
  \wlog{\string#6=\string#1\the\allocationnumber}}%
%    \end{macrocode}
%
%    \begin{macrocode}
\gdef\e@ch@ck#1#2#3#4{%
  \ifnum#1<#2\else
    \ifnum#1=#2\relax
      #1\@cclvi
      \ifx\count#4\advance#1 10 \fi
    \fi
    \ifnum#1<#3\relax
    \else
      \errmessage{No room for a new \string#4}%
    \fi
  \fi}%
%    \end{macrocode}
%
% Fix up allocations not to clash with |etex.src|.
%
%    \begin{macrocode}
\expandafter\csname newcount\endcsname\e@alloc@attribute@count
\expandafter\csname newcount\endcsname\e@alloc@ccodetable@count
\expandafter\csname newcount\endcsname\e@alloc@luafunction@count
\expandafter\csname newcount\endcsname\e@alloc@whatsit@count
\expandafter\csname newcount\endcsname\e@alloc@bytecode@count
\expandafter\csname newcount\endcsname\e@alloc@luachunk@count
%    \end{macrocode}
%
% End of conditional setup for plain \TeX\ / old \LaTeX.
%    \begin{macrocode}
\fi
%</tex>
%    \end{macrocode}
%
% \subsection{Attributes}
%
% \begin{macro}{\newattribute}
% \changes{v1.0a}{2015/09/24}{Macro added}
% \changes{v1.1q}{2020/08/02}{Move reset to $0$ inside conditional}
%   As is generally the case for the Lua\TeX{} registers we start here
%   from~$1$. Notably, some code assumes that |\attribute0| is never used so
%   this is important in this case.
%    \begin{macrocode}
\ifx\e@alloc@attribute@count\@undefined
  \countdef\e@alloc@attribute@count=258
  \e@alloc@attribute@count=\z@
\fi
\def\newattribute#1{%
  \e@alloc\attribute\attributedef
    \e@alloc@attribute@count\m@ne\e@alloc@top#1%
}
%    \end{macrocode}
% \end{macro}
%
% \begin{macro}{\setattribute}
% \begin{macro}{\unsetattribute}
%   Handy utilities.
%    \begin{macrocode}
\def\setattribute#1#2{#1=\numexpr#2\relax}
\def\unsetattribute#1{#1=-"7FFFFFFF\relax}
%    \end{macrocode}
% \end{macro}
% \end{macro}
%
% \subsection{Category code tables}
%
% \begin{macro}{\newcatcodetable}
% \changes{v1.0a}{2015/09/24}{Macro added}
%   Category code tables are allocated with a limit half of that used by Lua\TeX{}
%   for everything else. At the end of allocation there needs to be an
%   initialization step. Table~$0$ is already taken (it's the global one for
%   current use) so the allocation starts at~$1$.
%    \begin{macrocode}
\ifx\e@alloc@ccodetable@count\@undefined
  \countdef\e@alloc@ccodetable@count=259
  \e@alloc@ccodetable@count=\z@
\fi
\def\newcatcodetable#1{%
  \e@alloc\catcodetable\chardef
    \e@alloc@ccodetable@count\m@ne{"8000}#1%
  \initcatcodetable\allocationnumber
}
%    \end{macrocode}
% \end{macro}
%
% \changes{v1.0l}{2015/12/18}{Load Unicode data from source}
% \begin{macro}{\catcodetable@initex}
% \changes{v1.0a}{2015/09/24}{Macro added}
% \begin{macro}{\catcodetable@string}
% \changes{v1.0a}{2015/09/24}{Macro added}
% \begin{macro}{\catcodetable@latex}
% \changes{v1.0a}{2015/09/24}{Macro added}
% \begin{macro}{\catcodetable@atletter}
% \changes{v1.0a}{2015/09/24}{Macro added}
%   Save a small set of standard tables. The Unicode data is read
%   here in using a parser simplified from that in |load-unicode-data|:
%   only the nature of letters needs to be detected.
%    \begin{macrocode}
\newcatcodetable\catcodetable@initex
\newcatcodetable\catcodetable@string
\begingroup
  \def\setrangecatcode#1#2#3{%
    \ifnum#1>#2 %
      \expandafter\@gobble
    \else
      \expandafter\@firstofone
    \fi
      {%
        \catcode#1=#3 %
        \expandafter\setrangecatcode\expandafter
          {\number\numexpr#1 + 1\relax}{#2}{#3}
      }%
  }
  \@firstofone{%
    \catcodetable\catcodetable@initex
      \catcode0=12 %
      \catcode13=12 %
      \catcode37=12 %
      \setrangecatcode{65}{90}{12}%
      \setrangecatcode{97}{122}{12}%
      \catcode92=12 %
      \catcode127=12 %
      \savecatcodetable\catcodetable@string
    \endgroup
  }%
\newcatcodetable\catcodetable@latex
\newcatcodetable\catcodetable@atletter
\begingroup
  \def\parseunicodedataI#1;#2;#3;#4\relax{%
    \parseunicodedataII#1;#3;#2 First>\relax
  }%
  \def\parseunicodedataII#1;#2;#3 First>#4\relax{%
    \ifx\relax#4\relax
      \expandafter\parseunicodedataIII
    \else
      \expandafter\parseunicodedataIV
    \fi
      {#1}#2\relax%
  }%
  \def\parseunicodedataIII#1#2#3\relax{%
    \ifnum 0%
      \if L#21\fi
      \if M#21\fi
      >0 %
      \catcode"#1=11 %
    \fi
  }%
  \def\parseunicodedataIV#1#2#3\relax{%
    \read\unicoderead to \unicodedataline
    \if L#2%
      \count0="#1 %
      \expandafter\parseunicodedataV\unicodedataline\relax
    \fi
  }%
  \def\parseunicodedataV#1;#2\relax{%
    \loop
      \unless\ifnum\count0>"#1 %
        \catcode\count0=11 %
        \advance\count0 by 1 %
    \repeat
  }%
  \def\storedpar{\par}%
  \chardef\unicoderead=\numexpr\count16 + 1\relax
  \openin\unicoderead=UnicodeData.txt %
  \loop\unless\ifeof\unicoderead %
    \read\unicoderead to \unicodedataline
    \unless\ifx\unicodedataline\storedpar
      \expandafter\parseunicodedataI\unicodedataline\relax
    \fi
  \repeat
  \closein\unicoderead
  \@firstofone{%
    \catcode64=12 %
    \savecatcodetable\catcodetable@latex
    \catcode64=11 %
    \savecatcodetable\catcodetable@atletter
   }
\endgroup
%    \end{macrocode}
% \end{macro}
% \end{macro}
% \end{macro}
% \end{macro}
%
% \subsection{Named Lua functions}
%
% \begin{macro}{\newluafunction}
% \changes{v1.0a}{2015/09/24}{Macro added}
% \changes{v1.1q}{2020/08/02}{Move reset to $0$ inside conditional}
%   Much the same story for allocating Lua\TeX{} functions except here they are
%   just numbers so they are allocated in the same way as boxes.
%   Lua indexes from~$1$ so once again slot~$0$ is skipped.
%    \begin{macrocode}
\ifx\e@alloc@luafunction@count\@undefined
  \countdef\e@alloc@luafunction@count=260
  \e@alloc@luafunction@count=\z@
\fi
\def\newluafunction{%
  \e@alloc\luafunction\e@alloc@chardef
    \e@alloc@luafunction@count\m@ne\e@alloc@top
}
%    \end{macrocode}
% \end{macro}
%
% \subsection{Custom whatsits}
%
% \begin{macro}{\newwhatsit}
% \changes{v1.0a}{2015/09/24}{Macro added}
% \changes{v1.1q}{2020/08/02}{Move reset to $0$ inside conditional}
%   These are only settable from Lua but for consistency are definable
%   here.
%    \begin{macrocode}
\ifx\e@alloc@whatsit@count\@undefined
  \countdef\e@alloc@whatsit@count=261
  \e@alloc@whatsit@count=\z@
\fi
\def\newwhatsit#1{%
  \e@alloc\whatsit\e@alloc@chardef
    \e@alloc@whatsit@count\m@ne\e@alloc@top#1%
}
%    \end{macrocode}
% \end{macro}
%
% \subsection{Lua bytecode registers}
%
% \begin{macro}{\newluabytecode}
% \changes{v1.0a}{2015/09/24}{Macro added}
% \changes{v1.1q}{2020/08/02}{Move reset to $0$ inside conditional}
%   These are only settable from Lua but for consistency are definable
%   here.
%    \begin{macrocode}
\ifx\e@alloc@bytecode@count\@undefined
  \countdef\e@alloc@bytecode@count=262
  \e@alloc@bytecode@count=\z@
\fi
\def\newluabytecode#1{%
  \e@alloc\luabytecode\e@alloc@chardef
    \e@alloc@bytecode@count\m@ne\e@alloc@top#1%
}
%    \end{macrocode}
% \end{macro}
%
% \subsection{Lua chunk registers}

% \begin{macro}{\newluachunkname}
% \changes{v1.0a}{2015/09/24}{Macro added}
% \changes{v1.1q}{2020/08/02}{Move reset to $0$ inside conditional}
% As for bytecode registers, but in addition we need to add a string
% to the \verb|lua.name| table to use in stack tracing. We use the
% name of the command passed to the allocator, with no backslash.
%    \begin{macrocode}
\ifx\e@alloc@luachunk@count\@undefined
  \countdef\e@alloc@luachunk@count=263
  \e@alloc@luachunk@count=\z@
\fi
\def\newluachunkname#1{%
  \e@alloc\luachunk\e@alloc@chardef
    \e@alloc@luachunk@count\m@ne\e@alloc@top#1%
    {\escapechar\m@ne
    \directlua{lua.name[\the\allocationnumber]="\string#1"}}%
}
%    \end{macrocode}
% \end{macro}
%
% \subsection{Lua loader}
% \changes{v1.1r}{2020/08/10}{Load ltluatex Lua module during format building}
%
% Lua code loaded in the format often has to be loaded again at the
% beginning of every job, so we define a helper which allows us to avoid
% duplicated code:
%
%    \begin{macrocode}
\def\now@and@everyjob#1{%
  \everyjob\expandafter{\the\everyjob
    #1%
  }%
  #1%
}
%    \end{macrocode}
%
% Load the Lua code at the start of every job.
% For the conversion of \TeX{} into numbers at the Lua side we need some
% known registers: for convenience we use a set of systematic names, which
% means using a group around the Lua loader.
%    \begin{macrocode}
%<2ekernel>\now@and@everyjob{%
  \begingroup
    \attributedef\attributezero=0 %
    \chardef     \charzero     =0 %
%    \end{macrocode}
% Note name change required on older luatex, for hash table access.
%    \begin{macrocode}
    \countdef    \CountZero    =0 %
    \dimendef    \dimenzero    =0 %
    \mathchardef \mathcharzero =0 %
    \muskipdef   \muskipzero   =0 %
    \skipdef     \skipzero     =0 %
    \toksdef     \tokszero     =0 %
    \directlua{require("ltluatex")}
  \endgroup
%<2ekernel>}
%<latexrelease>\EndIncludeInRelease
%    \end{macrocode}
%
% \changes{v1.0b}{2015/10/02}{Fix backing out of \TeX{} code}
% \changes{v1.0c}{2015/10/02}{Allow backing out of Lua code}
%    \begin{macrocode}
%<latexrelease>\IncludeInRelease{0000/00/00}
%<latexrelease>                 {\newluafunction}{LuaTeX}%
%<latexrelease>\let\e@alloc@attribute@count\@undefined
%<latexrelease>\let\newattribute\@undefined
%<latexrelease>\let\setattribute\@undefined
%<latexrelease>\let\unsetattribute\@undefined
%<latexrelease>\let\e@alloc@ccodetable@count\@undefined
%<latexrelease>\let\newcatcodetable\@undefined
%<latexrelease>\let\catcodetable@initex\@undefined
%<latexrelease>\let\catcodetable@string\@undefined
%<latexrelease>\let\catcodetable@latex\@undefined
%<latexrelease>\let\catcodetable@atletter\@undefined
%<latexrelease>\let\e@alloc@luafunction@count\@undefined
%<latexrelease>\let\newluafunction\@undefined
%<latexrelease>\let\e@alloc@luafunction@count\@undefined
%<latexrelease>\let\newwhatsit\@undefined
%<latexrelease>\let\e@alloc@whatsit@count\@undefined
%<latexrelease>\let\newluabytecode\@undefined
%<latexrelease>\let\e@alloc@bytecode@count\@undefined
%<latexrelease>\let\newluachunkname\@undefined
%<latexrelease>\let\e@alloc@luachunk@count\@undefined
%<latexrelease>\directlua{luatexbase.uninstall()}
%<latexrelease>\EndIncludeInRelease
%    \end{macrocode}
%
% In \verb|\everyjob|, if luaotfload is available, load it and switch to TU.
%    \begin{macrocode}
%<latexrelease>\IncludeInRelease{2017/01/01}%
%<latexrelease>                 {\fontencoding}{TU in everyjob}%
%<latexrelease>\fontencoding{TU}\let\encodingdefault\f@encoding
%<latexrelease>\ifx\directlua\@undefined\else
%<2ekernel>\everyjob\expandafter{%
%<2ekernel>  \the\everyjob
%<*2ekernel,latexrelease>
  \directlua{%
  if xpcall(function ()%
             require('luaotfload-main')%
            end,texio.write_nl) then %
  local _void = luaotfload.main ()%
  else %
  texio.write_nl('Error in luaotfload: reverting to OT1')%
  tex.print('\string\\def\string\\encodingdefault{OT1}')%
  end %
  }%
  \let\f@encoding\encodingdefault
  \expandafter\let\csname ver@luaotfload.sty\endcsname\fmtversion
%</2ekernel,latexrelease>
%<latexrelease>\fi
%<2ekernel>  }
%<latexrelease>\EndIncludeInRelease
%<latexrelease>\IncludeInRelease{0000/00/00}%
%<latexrelease>                 {\fontencoding}{TU in everyjob}%
%<latexrelease>\fontencoding{OT1}\let\encodingdefault\f@encoding
%<latexrelease>\EndIncludeInRelease
%    \end{macrocode}
%
%    \begin{macrocode}
%<2ekernel|latexrelease>\fi
%</2ekernel|tex|latexrelease>
%    \end{macrocode}
%
% \subsection{Lua module preliminaries}
%
% \begingroup
%
%  \begingroup\lccode`~=`_
%  \lowercase{\endgroup\let~}_
%  \catcode`_=12
%
%    \begin{macrocode}
%<*lua>
%    \end{macrocode}
%
% Some set up for the Lua module which is needed for all of the Lua
% functionality added here.
%
% \begin{macro}{luatexbase}
% \changes{v1.0a}{2015/09/24}{Table added}
%   Set up the table for the returned functions. This is used to expose
%   all of the public functions.
%    \begin{macrocode}
luatexbase       = luatexbase or { }
local luatexbase = luatexbase
%    \end{macrocode}
% \end{macro}
%
% Some Lua best practice: use local versions of functions where possible.
% \changes{v1.1u}{2021/08/11}{Define missing local function}
%    \begin{macrocode}
local string_gsub      = string.gsub
local tex_count        = tex.count
local tex_setattribute = tex.setattribute
local tex_setcount     = tex.setcount
local texio_write_nl   = texio.write_nl
local flush_list       = node.flush_list
%    \end{macrocode}
% \changes{v1.0i}{2015/11/29}{Declare this as local before used in the module error definitions (PHG)}
%    \begin{macrocode}
local luatexbase_warning
local luatexbase_error
%    \end{macrocode}
%
% \subsection{Lua module utilities}
%
% \subsubsection{Module tracking}
%
% \begin{macro}{modules}
% \changes{v1.0a}{2015/09/24}{Function modified}
%   To allow tracking of module usage, a structure is provided to store
%   information and to return it.
%    \begin{macrocode}
local modules = modules or { }
%    \end{macrocode}
% \end{macro}
%
% \begin{macro}{provides_module}
% \changes{v1.0a}{2015/09/24}{Function added}
% \changes{v1.0f}{2015/10/03}{use luatexbase\_log}
% Local function to write to the log.
%    \begin{macrocode}
local function luatexbase_log(text)
  texio_write_nl("log", text)
end
%    \end{macrocode}
%
%   Modelled on |\ProvidesPackage|, we store much the same information but
%   with a little more structure.
%    \begin{macrocode}
local function provides_module(info)
  if not (info and info.name) then
    luatexbase_error("Missing module name for provides_module")
  end
  local function spaced(text)
    return text and (" " .. text) or ""
  end
  luatexbase_log(
    "Lua module: " .. info.name
      .. spaced(info.date)
      .. spaced(info.version)
      .. spaced(info.description)
  )
  modules[info.name] = info
end
luatexbase.provides_module = provides_module
%    \end{macrocode}
% \end{macro}
%
% \subsubsection{Module messages}
%
% There are various warnings and errors that need to be given. For warnings
% we can get exactly the same formatting as from \TeX{}. For errors we have to
% make some changes. Here we give the text of the error in the \LaTeX{} format
% then force an error from Lua to halt the run. Splitting the message text is
% done using |\n| which takes the place of |\MessageBreak|.
%
% First an auxiliary for the formatting: this measures up the message
% leader so we always get the correct indent.
% \changes{v1.0j}{2015/12/02}{Declaration/use of first\_head fixed (PHG)}
%    \begin{macrocode}
local function msg_format(mod, msg_type, text)
  local leader = ""
  local cont
  local first_head
  if mod == "LaTeX" then
    cont = string_gsub(leader, ".", " ")
    first_head = leader .. "LaTeX: "
  else
    first_head = leader .. "Module "  .. msg_type
    cont = "(" .. mod .. ")"
      .. string_gsub(first_head, ".", " ")
    first_head =  leader .. "Module "  .. mod .. " " .. msg_type  .. ":"
  end
  if msg_type == "Error" then
    first_head = "\n" .. first_head
  end
  if string.sub(text,-1) ~= "\n" then
    text = text .. " "
  end
  return first_head .. " "
    .. string_gsub(
         text
	 .. "on input line "
         .. tex.inputlineno, "\n", "\n" .. cont .. " "
      )
   .. "\n"
end
%    \end{macrocode}
%
% \begin{macro}{module_info}
% \changes{v1.0a}{2015/09/24}{Function added}
% \begin{macro}{module_warning}
% \changes{v1.0a}{2015/09/24}{Function added}
% \begin{macro}{module_error}
% \changes{v1.0a}{2015/09/24}{Function added}
%   Write messages.
%    \begin{macrocode}
local function module_info(mod, text)
  texio_write_nl("log", msg_format(mod, "Info", text))
end
luatexbase.module_info = module_info
local function module_warning(mod, text)
  texio_write_nl("term and log",msg_format(mod, "Warning", text))
end
luatexbase.module_warning = module_warning
local function module_error(mod, text)
  error(msg_format(mod, "Error", text))
end
luatexbase.module_error = module_error
%    \end{macrocode}
% \end{macro}
% \end{macro}
% \end{macro}
%
% Dedicated versions for the rest of the code here.
%    \begin{macrocode}
function luatexbase_warning(text)
  module_warning("luatexbase", text)
end
function luatexbase_error(text)
  module_error("luatexbase", text)
end
%    \end{macrocode}
%
%
% \subsection{Accessing register numbers from Lua}
%
% \changes{v1.0g}{2015/11/14}{Track Lua\TeX{} changes for
%   \texttt{(new)token.create}}
% Collect up the data from the \TeX{} level into a Lua table: from
% version~0.80, Lua\TeX{} makes that easy.
% \changes{v1.0j}{2015/12/02}{Adjust hashtokens to store the result of tex.hashtokens()), not the function (PHG)}
%    \begin{macrocode}
local luaregisterbasetable = { }
local registermap = {
  attributezero = "assign_attr"    ,
  charzero      = "char_given"     ,
  CountZero     = "assign_int"     ,
  dimenzero     = "assign_dimen"   ,
  mathcharzero  = "math_given"     ,
  muskipzero    = "assign_mu_skip" ,
  skipzero      = "assign_skip"    ,
  tokszero      = "assign_toks"    ,
}
local createtoken
if tex.luatexversion > 81 then
  createtoken = token.create
elseif tex.luatexversion > 79 then
  createtoken = newtoken.create
end
local hashtokens    = tex.hashtokens()
local luatexversion = tex.luatexversion
for i,j in pairs (registermap) do
  if luatexversion < 80 then
    luaregisterbasetable[hashtokens[i][1]] =
      hashtokens[i][2]
  else
    luaregisterbasetable[j] = createtoken(i).mode
  end
end
%    \end{macrocode}
%
% \begin{macro}{registernumber}
%   Working out the correct return value can be done in two ways. For older
%   Lua\TeX{} releases it has to be extracted from the |hashtokens|. On the
%   other hand, newer Lua\TeX{}'s have |newtoken|, and whilst |.mode| isn't
%   currently documented, Hans Hagen pointed to this approach so we should be
%   OK.
%    \begin{macrocode}
local registernumber
if luatexversion < 80 then
  function registernumber(name)
    local nt = hashtokens[name]
    if(nt and luaregisterbasetable[nt[1]]) then
      return nt[2] - luaregisterbasetable[nt[1]]
    else
      return false
    end
  end
else
  function registernumber(name)
    local nt = createtoken(name)
    if(luaregisterbasetable[nt.cmdname]) then
      return nt.mode - luaregisterbasetable[nt.cmdname]
    else
      return false
    end
  end
end
luatexbase.registernumber = registernumber
%    \end{macrocode}
% \end{macro}
%
% \subsection{Attribute allocation}
%
% \begin{macro}{new_attribute}
% \changes{v1.0a}{2015/09/24}{Function added}
% \changes{v1.1c}{2017/02/18}{Parameterize count used in tracking}
%   As attributes are used for Lua manipulations its useful to be able
%   to assign from this end.
% \InternalDetectionOff
%    \begin{macrocode}
local attributes=setmetatable(
{},
{
__index = function(t,key)
return registernumber(key) or nil
end}
)
luatexbase.attributes = attributes
%    \end{macrocode}
%
%    \begin{macrocode}
local attribute_count_name =
                     attribute_count_name or "e@alloc@attribute@count"
local function new_attribute(name)
  tex_setcount("global", attribute_count_name,
                          tex_count[attribute_count_name] + 1)
  if tex_count[attribute_count_name] > 65534 then
    luatexbase_error("No room for a new \\attribute")
  end
  attributes[name]= tex_count[attribute_count_name]
  luatexbase_log("Lua-only attribute " .. name .. " = " ..
                 tex_count[attribute_count_name])
  return tex_count[attribute_count_name]
end
luatexbase.new_attribute = new_attribute
%    \end{macrocode}
% \InternalDetectionOn
% \end{macro}
%
% \subsection{Custom whatsit allocation}
%
% \begin{macro}{new_whatsit}
% \changes{v1.1c}{2017/02/18}{Parameterize count used in tracking}
% Much the same as for attribute allocation in Lua.
%    \begin{macrocode}
local whatsit_count_name = whatsit_count_name or "e@alloc@whatsit@count"
local function new_whatsit(name)
  tex_setcount("global", whatsit_count_name,
                         tex_count[whatsit_count_name] + 1)
  if tex_count[whatsit_count_name] > 65534 then
    luatexbase_error("No room for a new custom whatsit")
  end
  luatexbase_log("Custom whatsit " .. (name or "") .. " = " ..
                 tex_count[whatsit_count_name])
  return tex_count[whatsit_count_name]
end
luatexbase.new_whatsit = new_whatsit
%    \end{macrocode}
% \end{macro}
%
% \subsection{Bytecode register allocation}
%
% \begin{macro}{new_bytecode}
% \changes{v1.1c}{2017/02/18}{Parameterize count used in tracking}
% Much the same as for attribute allocation in Lua.
% The optional \meta{name} argument is used in the log if given.
%    \begin{macrocode}
local bytecode_count_name =
                         bytecode_count_name or "e@alloc@bytecode@count"
local function new_bytecode(name)
  tex_setcount("global", bytecode_count_name,
                         tex_count[bytecode_count_name] + 1)
  if tex_count[bytecode_count_name] > 65534 then
    luatexbase_error("No room for a new bytecode register")
  end
  luatexbase_log("Lua bytecode " .. (name or "") .. " = " ..
                 tex_count[bytecode_count_name])
  return tex_count[bytecode_count_name]
end
luatexbase.new_bytecode = new_bytecode
%    \end{macrocode}
% \end{macro}
%
% \subsection{Lua chunk name allocation}
%
% \begin{macro}{new_chunkname}
% \changes{v1.1c}{2017/02/18}{Parameterize count used in tracking}
% As for bytecode registers but also store the name in the
% |lua.name| table.
%    \begin{macrocode}
local chunkname_count_name =
                        chunkname_count_name or "e@alloc@luachunk@count"
local function new_chunkname(name)
  tex_setcount("global", chunkname_count_name,
                         tex_count[chunkname_count_name] + 1)
  local chunkname_count = tex_count[chunkname_count_name]
  chunkname_count = chunkname_count + 1
  if chunkname_count > 65534 then
    luatexbase_error("No room for a new chunkname")
  end
  lua.name[chunkname_count]=name
  luatexbase_log("Lua chunkname " .. (name or "") .. " = " ..
                 chunkname_count .. "\n")
  return chunkname_count
end
luatexbase.new_chunkname = new_chunkname
%    \end{macrocode}
% \end{macro}
%
% \subsection{Lua function allocation}
%
% \begin{macro}{new_luafunction}
% \changes{v1.1i}{2018/10/21}{Function added}
% Much the same as for attribute allocation in Lua.
% The optional \meta{name} argument is used in the log if given.
%    \begin{macrocode}
local luafunction_count_name =
                         luafunction_count_name or "e@alloc@luafunction@count"
local function new_luafunction(name)
  tex_setcount("global", luafunction_count_name,
                         tex_count[luafunction_count_name] + 1)
  if tex_count[luafunction_count_name] > 65534 then
    luatexbase_error("No room for a new luafunction register")
  end
  luatexbase_log("Lua function " .. (name or "") .. " = " ..
                 tex_count[luafunction_count_name])
  return tex_count[luafunction_count_name]
end
luatexbase.new_luafunction = new_luafunction
%    \end{macrocode}
% \end{macro}
%
% \subsection{Lua callback management}
%
% The native mechanism for callbacks in Lua\TeX\ allows only one per function.
% That is extremely restrictive and so a mechanism is needed to add and
% remove callbacks from the appropriate hooks.
%
% \subsubsection{Housekeeping}
%
% The main table: keys are callback names, and values are the associated lists
% of functions. More precisely, the entries in the list are tables holding the
% actual function as |func| and the identifying description as |description|.
% Only callbacks with a non-empty list of functions have an entry in this
% list.
%    \begin{macrocode}
local callbacklist = callbacklist or { }
%    \end{macrocode}
%
% Numerical codes for callback types, and name-to-value association (the
% table keys are strings, the values are numbers).
%    \begin{macrocode}
local list, data, exclusive, simple, reverselist = 1, 2, 3, 4, 5
local types   = {
  list        = list,
  data        = data,
  exclusive   = exclusive,
  simple      = simple,
  reverselist = reverselist,
}
%    \end{macrocode}
%
% Now, list all predefined callbacks with their current type, based on the
% Lua\TeX{} manual version~1.01. A full list of the currently-available
% callbacks can be obtained using
%  \begin{verbatim}
%    \directlua{
%      for i,_ in pairs(callback.list()) do
%        texio.write_nl("- " .. i)
%      end
%    }
%    \bye
%  \end{verbatim}
% in plain Lua\TeX{}. (Some undocumented callbacks are omitted as they are
% to be removed.)
%    \begin{macrocode}
local callbacktypes = callbacktypes or {
%    \end{macrocode}
%   Section 8.2: file discovery callbacks.
% \changes{v1.1g}{2018/05/02}{find\_sfd\_file removed}
%    \begin{macrocode}
  find_read_file     = exclusive,
  find_write_file    = exclusive,
  find_font_file     = data,
  find_output_file   = data,
  find_format_file   = data,
  find_vf_file       = data,
  find_map_file      = data,
  find_enc_file      = data,
  find_pk_file       = data,
  find_data_file     = data,
  find_opentype_file = data,
  find_truetype_file = data,
  find_type1_file    = data,
  find_image_file    = data,
%    \end{macrocode}
% \changes{v1.1g}{2018/05/02}{read\_sfd\_file removed}
%    \begin{macrocode}
  open_read_file     = exclusive,
  read_font_file     = exclusive,
  read_vf_file       = exclusive,
  read_map_file      = exclusive,
  read_enc_file      = exclusive,
  read_pk_file       = exclusive,
  read_data_file     = exclusive,
  read_truetype_file = exclusive,
  read_type1_file    = exclusive,
  read_opentype_file = exclusive,
%    \end{macrocode}
% \changes{v1.0m}{2016/02/11}{read\_cidmap\_file added}
% Not currently used by luatex but included for completeness.
% may be used by a font handler.
%    \begin{macrocode}
  find_cidmap_file   = data,
  read_cidmap_file   = exclusive,
%    \end{macrocode}
% Section 8.3: data processing callbacks.
% \changes{v1.0m}{2016/02/11}{token\_filter removed}
%    \begin{macrocode}
  process_input_buffer  = data,
  process_output_buffer = data,
  process_jobname       = data,
%    \end{macrocode}
% Section 8.4: node list processing callbacks.
% \changes{v1.0m}{2016/02/11}
% {process\_rule, [hv]pack\_quality  append\_to\_vlist\_filter added}
% \changes{v1.0n}{2016/03/13}{insert\_local\_par added}
% \changes{v1.0n}{2016/03/13}{contribute\_filter added}
% \changes{v1.1h}{2018/08/18}{append\_to\_vlist\_filter is \texttt{exclusive}}
% \changes{v1.1j}{2019/06/18}{new\_graf added}
% \changes{v1.1k}{2019/10/02}{linebreak\_filter is \texttt{exclusive}}
% \changes{v1.1k}{2019/10/02}{process\_rule is \texttt{exclusive}}
% \changes{v1.1k}{2019/10/02}{mlist\_to\_hlist is \texttt{exclusive}}
% \changes{v1.1l}{2020/02/02}{post\_linebreak\_filter is \texttt{reverselist}}
% \changes{v1.1p}{2020/08/01}{new\_graf is \texttt{exclusive}}
% \changes{v1.1w}{2021/11/17}{hpack\_quality is \texttt{exclusive}}
% \changes{v1.1w}{2021/11/17}{vpack\_quality is \texttt{exclusive}}
%    \begin{macrocode}
  contribute_filter      = simple,
  buildpage_filter       = simple,
  build_page_insert      = exclusive,
  pre_linebreak_filter   = list,
  linebreak_filter       = exclusive,
  append_to_vlist_filter = exclusive,
  post_linebreak_filter  = reverselist,
  hpack_filter           = list,
  vpack_filter           = list,
  hpack_quality          = exclusive,
  vpack_quality          = exclusive,
  pre_output_filter      = list,
  process_rule           = exclusive,
  hyphenate              = simple,
  ligaturing             = simple,
  kerning                = simple,
  insert_local_par       = simple,
  pre_mlist_to_hlist_filter = list,
  mlist_to_hlist         = exclusive,
  post_mlist_to_hlist_filter = reverselist,
  new_graf               = exclusive,
%    \end{macrocode}
% Section 8.5: information reporting callbacks.
% \changes{v1.0m}{2016/02/11}{show\_warning\_message added}
% \changes{v1.0p}{2016/11/17}{call\_edit added}
% \changes{v1.1g}{2018/05/02}{finish\_synctex\_callback added}
% \changes{v1.1j}{2019/06/18}{finish\_synctex\_callback renamed finish\_synctex}
% \changes{v1.1j}{2019/06/18}{wrapup\_run added}
%    \begin{macrocode}
  pre_dump             = simple,
  start_run            = simple,
  stop_run             = simple,
  start_page_number    = simple,
  stop_page_number     = simple,
  show_error_hook      = simple,
  show_warning_message = simple,
  show_error_message   = simple,
  show_lua_error_hook  = simple,
  start_file           = simple,
  stop_file            = simple,
  call_edit            = simple,
  finish_synctex       = simple,
  wrapup_run           = simple,
%    \end{macrocode}
% Section 8.6: PDF-related callbacks.
% \changes{v1.1j}{2019/06/18}{page\_objnum\_provider added}
% \changes{v1.1j}{2019/06/18}{process\_pdf\_image\_content added}
% \changes{v1.1j}{2019/10/22}{page\_objnum\_provider and process\_pdf\_image\_content classified data}
% \changes{v1.1l}{2020/02/02}{page\_order\_index added}
%    \begin{macrocode}
  finish_pdffile            = data,
  finish_pdfpage            = data,
  page_objnum_provider      = data,
  page_order_index          = data,
  process_pdf_image_content = data,
%    \end{macrocode}
% Section 8.7: font-related callbacks.
% \changes{v1.1e}{2017/03/28}{glyph\_stream\_provider added}
% \changes{v1.1g}{2018/05/02}{glyph\_not\_found added}
% \changes{v1.1j}{2019/06/18}{make\_extensible added}
% \changes{v1.1j}{2019/06/18}{font\_descriptor\_objnum\_provider added}
% \changes{v1.1l}{2020/02/02}{glyph\_info added}
% \changes{v1.1t}{2021/04/18}{input\_level\_string added}
% \changes{v1.1v}{2021/10/15}{provide\_charproc\_data added}
%    \begin{macrocode}
  define_font                     = exclusive,
  glyph_info                      = exclusive,
  glyph_not_found                 = exclusive,
  glyph_stream_provider           = exclusive,
  make_extensible                 = exclusive,
  font_descriptor_objnum_provider = exclusive,
  input_level_string              = exclusive,
  provide_charproc_data           = exclusive,
%    \end{macrocode}
% \changes{v1.0m}{2016/02/11}{pdf\_stream\_filter\_callback removed}
%    \begin{macrocode}
}
luatexbase.callbacktypes=callbacktypes
%    \end{macrocode}
%
% \begin{macro}{callback.register}
% \changes{v1.0a}{2015/09/24}{Function modified}
%   Save the original function for registering callbacks and prevent the
%   original being used. The original is saved in a place that remains
%   available so other more sophisticated code can override the approach
%   taken by the kernel if desired.
%    \begin{macrocode}
local callback_register = callback_register or callback.register
function callback.register()
  luatexbase_error("Attempt to use callback.register() directly\n")
end
%    \end{macrocode}
% \end{macro}
%
% \subsubsection{Handlers}
%
% The handler function is registered into the callback when the
% first function is added to this callback's list. Then, when the callback
% is called, the handler takes care of running all functions in the list.
% When the last function is removed from the callback's list, the handler
% is unregistered.
%
% More precisely, the functions below are used to generate a specialized
% function (closure) for a given callback, which is the actual handler.
%
%
% The way the functions are combined together depends on
% the type of the callback. There are currently 4 types of callback, depending
% on the calling convention of the functions the callback can hold:
% \begin{description}
%   \item[simple] is for functions that don't return anything: they are called
%     in order, all with the same argument;
%   \item[data] is for functions receiving a piece of data of any type
%     except node list head (and possibly other arguments) and returning it
%     (possibly modified): the functions are called in order, and each is
%     passed the return value of the previous (and the other arguments
%     untouched, if any). The return value is that of the last function;
%   \item[list] is a specialized variant of \emph{data} for functions
%     filtering node lists. Such functions may return either the head of a
%     modified node list, or the boolean values |true| or |false|. The
%     functions are chained the same way as for \emph{data} except that for
%     the following. If
%     one function returns |false|, then |false| is immediately returned and
%     the following functions are \emph{not} called. If one function returns
%     |true|, then the same head is passed to the next function. If all
%     functions return |true|, then |true| is returned, otherwise the return
%     value of the last function not returning |true| is used.
%   \item[reverselist] is a specialized variant of \emph{list} which executes
%     functions in inverse order.
%   \item[exclusive] is for functions with more complex signatures; functions in
%     this type of callback are \emph{not} combined: An error is raised if
%     a second callback is registered.
% \end{description}
%
% Handler for |data| callbacks.
%    \begin{macrocode}
local function data_handler(name)
  return function(data, ...)
    for _,i in ipairs(callbacklist[name]) do
      data = i.func(data,...)
    end
    return data
  end
end
%    \end{macrocode}
% Default for user-defined |data| callbacks without explicit default.
%    \begin{macrocode}
local function data_handler_default(value)
  return value
end
%    \end{macrocode}
% Handler for |exclusive| callbacks. We can assume |callbacklist[name]| is not
% empty: otherwise, the function wouldn't be registered in the callback any
% more.
%    \begin{macrocode}
local function exclusive_handler(name)
  return function(...)
    return callbacklist[name][1].func(...)
  end
end
%    \end{macrocode}
% Handler for |list| callbacks.
% \changes{v1.0k}{2015/12/02}{resolve name and i.description (PHG)}
% \changes{v1.1s}{2020/12/02}{Fix return value of list callbacks}
% \changes{v1.1w}{2021/11/17}{Never pass on \texttt{true} return values for list callbacks}
%    \begin{macrocode}
local function list_handler(name)
  return function(head, ...)
    local ret
    for _,i in ipairs(callbacklist[name]) do
      ret = i.func(head, ...)
      if ret == false then
        luatexbase_warning(
          "Function `" .. i.description .. "' returned false\n"
            .. "in callback `" .. name .."'"
         )
        return false
      end
      if ret ~= true then
        head = ret
      end
    end
    return head
  end
end
%    \end{macrocode}
% Default for user-defined |list| and |reverselist| callbacks without explicit default.
%    \begin{macrocode}
local function list_handler_default(head)
return head
end
%    \end{macrocode}
% Handler for |reverselist| callbacks.
% \changes{v1.1l}{2020/02/02}{Add reverselist callback type}
%    \begin{macrocode}
local function reverselist_handler(name)
  return function(head, ...)
    local ret
    local callbacks = callbacklist[name]
    for i = #callbacks, 1, -1 do
      local cb = callbacks[i]
      ret = cb.func(head, ...)
      if ret == false then
        luatexbase_warning(
          "Function `" .. cb.description .. "' returned false\n"
            .. "in callback `" .. name .."'"
         )
        return false
      end
      if ret ~= true then
        head = ret
      end
    end
    return head
  end
end
%    \end{macrocode}
% Handler for |simple| callbacks.
%    \begin{macrocode}
local function simple_handler(name)
  return function(...)
    for _,i in ipairs(callbacklist[name]) do
      i.func(...)
    end
  end
end
%    \end{macrocode}
% Default for user-defined |simple| callbacks without explicit default.
%    \begin{macrocode}
local function simple_handler_default()
end
%    \end{macrocode}
%
% Keep a handlers table for indexed access and a table with the corresponding default functions.
%    \begin{macrocode}
local handlers  = {
  [data]        = data_handler,
  [exclusive]   = exclusive_handler,
  [list]        = list_handler,
  [reverselist] = reverselist_handler,
  [simple]      = simple_handler,
}
local defaults = {
  [data]        = data_handler_default,
  [exclusive]   = nil,
  [list]        = list_handler_default,
  [reverselist] = list_handler_default,
  [simple]      = simple_handler_default,
}
%    \end{macrocode}
%
% \subsubsection{Public functions for callback management}
%
% Defining user callbacks perhaps should be in package code,
% but impacts on |add_to_callback|.
% If a default function is not required, it may be declared as |false|.
% First we need a list of user callbacks.
%    \begin{macrocode}
local user_callbacks_defaults = {
  pre_mlist_to_hlist_filter = list_handler_default,
  mlist_to_hlist = node.mlist_to_hlist,
  post_mlist_to_hlist_filter = list_handler_default,
}
%    \end{macrocode}
%
% \begin{macro}{create_callback}
% \changes{v1.0a}{2015/09/24}{Function added}
% \changes{v1.0i}{2015/11/29}{Check name is not nil in error message (PHG)}
% \changes{v1.0k}{2015/12/02}{Give more specific error messages (PHG)}
% \changes{v1.1l}{2020/02/02}{Provide proper fallbacks for user-defined callbacks without user-provided default handler}
%   The allocator itself.
%    \begin{macrocode}
local function create_callback(name, ctype, default)
  local ctype_id = types[ctype]
  if not name  or name  == ""
  or not ctype_id
  then
    luatexbase_error("Unable to create callback:\n" ..
                     "valid callback name and type required")
  end
  if callbacktypes[name] then
    luatexbase_error("Unable to create callback `" .. name ..
                     "':\ncallback is already defined")
  end
  default = default or defaults[ctype_id]
  if not default then
    luatexbase_error("Unable to create callback `" .. name ..
                     "':\ndefault is required for `" .. ctype ..
                     "' callbacks")
  elseif type (default) ~= "function" then
    luatexbase_error("Unable to create callback `" .. name ..
                     "':\ndefault is not a function")
  end
  user_callbacks_defaults[name] = default
  callbacktypes[name] = ctype_id
end
luatexbase.create_callback = create_callback
%    \end{macrocode}
% \end{macro}
%
% \begin{macro}{call_callback}
% \changes{v1.0a}{2015/09/24}{Function added}
% \changes{v1.0i}{2015/11/29}{Check name is not nil in error message (PHG)}
% \changes{v1.0k}{2015/12/02}{Give more specific error messages (PHG)}
%  Call a user defined callback. First check arguments.
%    \begin{macrocode}
local function call_callback(name,...)
  if not name or name == "" then
    luatexbase_error("Unable to create callback:\n" ..
                     "valid callback name required")
  end
  if user_callbacks_defaults[name] == nil then
    luatexbase_error("Unable to call callback `" .. name
                     .. "':\nunknown or empty")
   end
  local l = callbacklist[name]
  local f
  if not l then
    f = user_callbacks_defaults[name]
  else
    f = handlers[callbacktypes[name]](name)
  end
  return f(...)
end
luatexbase.call_callback=call_callback
%    \end{macrocode}
% \end{macro}
%
% \begin{macro}{add_to_callback}
% \changes{v1.0a}{2015/09/24}{Function added}
%   Add a function to a callback. First check arguments.
% \changes{v1.0k}{2015/12/02}{Give more specific error messages (PHG)}
%    \begin{macrocode}
local function add_to_callback(name, func, description)
  if not name or name == "" then
    luatexbase_error("Unable to register callback:\n" ..
                     "valid callback name required")
  end
  if not callbacktypes[name] or
    type(func) ~= "function" or
    not description or
    description == "" then
    luatexbase_error(
      "Unable to register callback.\n\n"
        .. "Correct usage:\n"
        .. "add_to_callback(<callback>, <function>, <description>)"
    )
  end
%    \end{macrocode}
%   Then test if this callback is already in use. If not, initialise its list
%   and register the proper handler.
%    \begin{macrocode}
  local l = callbacklist[name]
  if l == nil then
    l = { }
    callbacklist[name] = l
%    \end{macrocode}
% If it is not a user defined callback use the primitive callback register.
%    \begin{macrocode}
    if user_callbacks_defaults[name] == nil then
      callback_register(name, handlers[callbacktypes[name]](name))
    end
  end
%    \end{macrocode}
%  Actually register the function and give an error if more than one
%  |exclusive| one is registered.
%    \begin{macrocode}
  local f = {
    func        = func,
    description = description,
  }
  local priority = #l + 1
  if callbacktypes[name] == exclusive then
    if #l == 1 then
      luatexbase_error(
        "Cannot add second callback to exclusive function\n`" ..
        name .. "'")
    end
  end
  table.insert(l, priority, f)
%    \end{macrocode}
%  Keep user informed.
%    \begin{macrocode}
  luatexbase_log(
    "Inserting `" .. description .. "' at position "
      .. priority .. " in `" .. name .. "'."
  )
end
luatexbase.add_to_callback = add_to_callback
%    \end{macrocode}
% \end{macro}
%
% \begin{macro}{remove_from_callback}
% \changes{v1.0a}{2015/09/24}{Function added}
% \changes{v1.0k}{2015/12/02}{adjust initialization of cb local (PHG)}
% \changes{v1.0k}{2015/12/02}{Give more specific error messages (PHG)}
% \changes{v1.1m}{2020/03/07}{Do not call callback.register for user-defined callbacks}
%   Remove a function from a callback. First check arguments.
%    \begin{macrocode}
local function remove_from_callback(name, description)
  if not name or name == "" then
    luatexbase_error("Unable to remove function from callback:\n" ..
                     "valid callback name required")
  end
  if not callbacktypes[name] or
    not description or
    description == "" then
    luatexbase_error(
      "Unable to remove function from callback.\n\n"
        .. "Correct usage:\n"
        .. "remove_from_callback(<callback>, <description>)"
    )
  end
  local l = callbacklist[name]
  if not l then
    luatexbase_error(
      "No callback list for `" .. name .. "'\n")
  end
%    \end{macrocode}
%  Loop over the callback's function list until we find a matching entry.
%  Remove it and check if the list is empty: if so, unregister the
%   callback handler.
%    \begin{macrocode}
  local index = false
  for i,j in ipairs(l) do
    if j.description == description then
      index = i
      break
    end
  end
  if not index then
    luatexbase_error(
      "No callback `" .. description .. "' registered for `" ..
      name .. "'\n")
  end
  local cb = l[index]
  table.remove(l, index)
  luatexbase_log(
    "Removing  `" .. description .. "' from `" .. name .. "'."
  )
  if #l == 0 then
    callbacklist[name] = nil
    if user_callbacks_defaults[name] == nil then
      callback_register(name, nil)
    end
  end
  return cb.func,cb.description
end
luatexbase.remove_from_callback = remove_from_callback
%    \end{macrocode}
% \end{macro}
%
% \begin{macro}{in_callback}
% \changes{v1.0a}{2015/09/24}{Function added}
% \changes{v1.0h}{2015/11/27}{Guard against undefined list latex/4445}
%   Look for a function description in a callback.
%    \begin{macrocode}
local function in_callback(name, description)
  if not name
    or name == ""
    or not callbacklist[name]
    or not callbacktypes[name]
    or not description then
      return false
  end
  for _, i in pairs(callbacklist[name]) do
    if i.description == description then
      return true
    end
  end
  return false
end
luatexbase.in_callback = in_callback
%    \end{macrocode}
% \end{macro}
%
% \begin{macro}{disable_callback}
% \changes{v1.0a}{2015/09/24}{Function added}
%   As we subvert the engine interface we need to provide a way to access
%   this functionality.
%    \begin{macrocode}
local function disable_callback(name)
  if(callbacklist[name] == nil) then
    callback_register(name, false)
  else
    luatexbase_error("Callback list for " .. name .. " not empty")
  end
end
luatexbase.disable_callback = disable_callback
%    \end{macrocode}
% \end{macro}
%
% \begin{macro}{callback_descriptions}
% \changes{v1.0a}{2015/09/24}{Function added}
% \changes{v1.0h}{2015/11/27}{Match test in in-callback latex/4445}
%   List the descriptions of functions registered for the given callback.
%    \begin{macrocode}
local function callback_descriptions (name)
  local d = {}
  if not name
    or name == ""
    or not callbacklist[name]
    or not callbacktypes[name]
    then
    return d
  else
  for k, i in pairs(callbacklist[name]) do
    d[k]= i.description
    end
  end
  return d
end
luatexbase.callback_descriptions =callback_descriptions
%    \end{macrocode}
% \end{macro}
%
% \begin{macro}{uninstall}
% \changes{v1.0e}{2015/10/02}{Function added}
%   Unlike at the \TeX{} level, we have to provide a back-out mechanism here
%   at the same time as the rest of the code. This is not meant for use by
%   anything other than \textsf{latexrelease}: as such this is
%   \emph{deliberately} not documented for users!
%    \begin{macrocode}
local function uninstall()
  module_info(
    "luatexbase",
    "Uninstalling kernel luatexbase code"
  )
  callback.register = callback_register
  luatexbase = nil
end
luatexbase.uninstall = uninstall
%    \end{macrocode}
% \end{macro}
% \begin{macro}{mlist_to_hlist}
% \changes{v1.1l}{2020/02/02}{|pre/post_mlist_to_hlist| added}
%   To emulate these callbacks, the ``real'' |mlist_to_hlist| is replaced by a
%   wrapper calling the wrappers before and after.
%    \begin{macrocode}
callback_register("mlist_to_hlist", function(head, display_type, need_penalties)
  local current = call_callback("pre_mlist_to_hlist_filter", head, display_type, need_penalties)
  if current == false then
    flush_list(head)
    return nil
  end
  current = call_callback("mlist_to_hlist", current, display_type, need_penalties)
  local post = call_callback("post_mlist_to_hlist_filter", current, display_type, need_penalties)
  if post == false then
    flush_list(current)
    return nil
  end
  return post
end)
%    \end{macrocode}
% \end{macro}
% \endgroup
%
%    \begin{macrocode}
%</lua>
%    \end{macrocode}
%
% Reset the catcode of |@|.
%    \begin{macrocode}
%<tex>\catcode`\@=\etatcatcode\relax
%    \end{macrocode}
%
%
% \Finale
|. This inputs |ltluatex.tex| which inputs
% |etex.src| (or |etex.sty| if used with \LaTeX)
% if it is not already input, and then defines some internal commands to
% allow the \textsf{ltluatex} interface to be defined.
%
% The \textsf{luatexbase} package interface may also be used in plain \TeX,
% as before, by inputting the package |\input luatexbase.sty|. The new
% version of \textsf{luatexbase} is based on this \textsf{ltluatex}
% code but implements a compatibility layer providing the interface
% of the original package.
%
% \section{Lua functionality}
%
% \begingroup
%
% \begingroup\lccode`~=`_
% \lowercase{\endgroup\let~}_
% \catcode`_=12
%
% \subsection{Allocators in Lua}
%
% \DescribeMacro{new_attribute}
% |luatexbase.new_attribute(|\meta{attribute}|)|\\
% Returns an allocation number for the \meta{attribute}, indexed from~$1$.
% The attribute will be initialised with the marker value |-"7FFFFFFF|
% (`unset'). The attribute allocation sequence is shared with the \TeX{}
% code but this function does \emph{not} define a token using
% |\attributedef|.
% The attribute name is recorded in the |attributes| table. A
% metatable is provided so that the table syntax can be used
% consistently for attributes declared in \TeX\ or Lua.
%
% \noindent
% \DescribeMacro{new_whatsit}
% |luatexbase.new_whatsit(|\meta{whatsit}|)|\\
% Returns an allocation number for the custom \meta{whatsit}, indexed from~$1$.
%
% \noindent
% \DescribeMacro{new_bytecode}
% |luatexbase.new_bytecode(|\meta{bytecode}|)|\\
% Returns an allocation number for a bytecode register, indexed from~$1$.
% The optional \meta{name} argument is just used for logging.
%
% \noindent
% \DescribeMacro{new_chunkname}
% |luatexbase.new_chunkname(|\meta{chunkname}|)|\\
% Returns an allocation number for a Lua chunk name for use with
% |\directlua| and |\latelua|, indexed from~$1$.
% The number is returned and also \meta{name} argument is added to the
% |lua.name| array at that index.
%
% \begin{sloppypar}
% \noindent
% \DescribeMacro{new_luafunction}
% |luatexbase.new_luafunction(|\meta{functionname}|)|\\
% Returns an allocation number for a lua function for use
% with |\luafunction|, |\lateluafunction|, and |\luadef|,
% indexed from~$1$. The optional \meta{functionname} argument
% is just used for logging.
% \end{sloppypar}
%
% These functions all require access to a named \TeX{} count register
% to manage their allocations. The standard names are those defined
% above for access from \TeX{}, \emph{e.g.}~\string\e@alloc@attribute@count,
% but these can be adjusted by defining the variable
% \texttt{\meta{type}\_count\_name} before loading |ltluatex.lua|, for example
% \begin{verbatim}
% local attribute_count_name = "attributetracker"
% require("ltluatex")
% \end{verbatim}
% would use a \TeX{} |\count| (|\countdef|'d token) called |attributetracker|
% in place of \string\e@alloc@attribute@count.
%
% \subsection{Lua access to \TeX{} register numbers}
%
% \DescribeMacro{registernumber}
% |luatexbase.registernumer(|\meta{name}|)|\\
% Sometimes (notably in the case of Lua attributes) it is necessary to
% access a register \emph{by number} that has been allocated by \TeX{}.
% This package provides a function to look up the relevant number
% using Lua\TeX{}'s internal tables. After for example
% |\newattribute\myattrib|, |\myattrib| would be defined by (say)
% |\myattrib=\attribute15|.  |luatexbase.registernumer("myattrib")|
% would then return the register number, $15$ in this case. If the string passed
% as argument does not correspond to a token defined by |\attributedef|,
% |\countdef| or similar commands, the Lua value |false| is returned.
%
% As an example, consider the input:
%\begin{verbatim}
% \newcommand\test[1]{%
% \typeout{#1: \expandafter\meaning\csname#1\endcsname^^J
% \space\space\space\space
% \directlua{tex.write(luatexbase.registernumber("#1") or "bad input")}%
% }}
%
% \test{undefinedrubbish}
%
% \test{space}
%
% \test{hbox}
%
% \test{@MM}
%
% \test{@tempdima}
% \test{@tempdimb}
%
% \test{strutbox}
%
% \test{sixt@@n}
%
% \attrbutedef\myattr=12
% \myattr=200
% \test{myattr}
%
%\end{verbatim}
%
% If the demonstration code is processed with Lua\LaTeX{} then the following
% would be produced in the log and terminal output.
%\begin{verbatim}
% undefinedrubbish: \relax
%      bad input
% space: macro:->
%      bad input
% hbox: \hbox
%      bad input
% @MM: \mathchar"4E20
%      20000
% @tempdima: \dimen14
%      14
% @tempdimb: \dimen15
%      15
% strutbox: \char"B
%      11
% sixt@@n: \char"10
%      16
% myattr: \attribute12
%      12
%\end{verbatim}
%
% Notice how undefined commands, or commands unrelated to registers
% do not produce an error, just return |false| and so print
% |bad input| here. Note also that commands defined by |\newbox| work and
% return the number of the box register even though the actual command
% holding this number is a |\chardef| defined token (there is no
% |\boxdef|).
%
% \subsection{Module utilities}
%
% \DescribeMacro{provides_module}
% |luatexbase.provides_module(|\meta{info}|)|\\
% This function is used by modules to identify themselves; the |info| should be
% a table containing information about the module. The required field
% |name| must contain the name of the module. It is recommended to provide a
% field |date| in the usual \LaTeX{} format |yyyy/mm/dd|. Optional fields
% |version| (a string) and |description| may be used if present. This
% information will be recorded in the log. Other fields are ignored.
%
% \noindent
% \DescribeMacro{module_info}
% \DescribeMacro{module_warning}
% \DescribeMacro{module_error}
% |luatexbase.module_info(|\meta{module}, \meta{text}|)|\\
% |luatexbase.module_warning(|\meta{module}, \meta{text}|)|\\
% |luatexbase.module_error(|\meta{module}, \meta{text}|)|\\
% These functions are similar to \LaTeX{}'s |\PackageError|, |\PackageWarning|
% and |\PackageInfo| in the way they format the output.  No automatic line
% breaking is done, you may still use |\n| as usual for that, and the name of
% the package will be prepended to each output line.
%
% Note that |luatexbase.module_error| raises an actual Lua error with |error()|,
% which currently means a call stack will be dumped. While this may not
% look pretty, at least it provides useful information for tracking the
% error down.
%
% \subsection{Callback management}
%
% \noindent
% \DescribeMacro{add_to_callback}
% |luatexbase.add_to_callback(|^^A
% \meta{callback}, \meta{function}, \meta{description}|)|
% Registers the \meta{function} into the \meta{callback} with a textual
% \meta{description} of the function. Functions are inserted into the callback
% in the order loaded.
%
% \noindent
% \DescribeMacro{remove_from_callback}
% |luatexbase.remove_from_callback(|\meta{callback}, \meta{description}|)|
% Removes the callback function with \meta{description} from the \meta{callback}.
% The removed function and its description
% are returned as the results of this function.
%
% \noindent
% \DescribeMacro{in_callback}
% |luatexbase.in_callback(|\meta{callback}, \meta{description}|)|
% Checks if the \meta{description} matches one of the functions added
% to the list for the \meta{callback}, returning a boolean value.
%
% \noindent
% \DescribeMacro{disable_callback}
% |luatexbase.disable_callback(|\meta{callback}|)|
% Sets the \meta{callback} to \texttt{false} as described in the Lua\TeX{}
% manual for the underlying \texttt{callback.register} built-in. Callbacks
% will only be set to false (and thus be skipped entirely) if there are
% no functions registered using the callback.
%
% \noindent
% \DescribeMacro{callback_descriptions}
% A list of the descriptions of functions registered to the specified
% callback is returned. |{}| is returned if there are no functions registered.
%
% \noindent
% \DescribeMacro{create_callback}
% |luatexbase.create_callback(|\meta{name},meta{type},\meta{default}|)|
% Defines a user defined callback. The last argument is a default
% function or |false|.
%
% \noindent
% \DescribeMacro{call_callback}
% |luatexbase.call_callback(|\meta{name},\ldots|)|
% Calls a user defined callback with the supplied arguments.
%
% \endgroup
%
% \StopEventually{}
%
% \section{Implementation}
%
%    \begin{macrocode}
%<*2ekernel|tex|latexrelease>
%<2ekernel|latexrelease>\ifx\directlua\@undefined\else
%    \end{macrocode}
%
%
% \changes{v1.0j}{2015/12/02}{Remove nonlocal iteration variables (PHG)}
% \changes{v1.0j}{2015/12/02}{Assorted typos fixed (PHG)}
% \changes{v1.0j}{2015/12/02}{Remove unreachable code after calls to error() (PHG)}
% \subsection{Minimum Lua\TeX{} version}
%
% Lua\TeX{} has changed a lot over time. In the kernel support for ancient
% versions is not provided: trying to build a format with a very old binary
% therefore gives some information in the log and loading stops. The cut-off
% selected here relates to the tree-searching behaviour of |require()|:
% from version~0.60, Lua\TeX{} will correctly find Lua files in the |texmf|
% tree without `help'.
%    \begin{macrocode}
%<latexrelease>\IncludeInRelease{2015/10/01}
%<latexrelease>                 {\newluafunction}{LuaTeX}%
\ifnum\luatexversion<60 %
  \wlog{***************************************************}
  \wlog{* LuaTeX version too old for ltluatex support *}
  \wlog{***************************************************}
  \expandafter\endinput
\fi
%    \end{macrocode}
%
% \changes{v1.1n}{2020/06/10}{Define \cs{@gobble}/\cs{@firstofone} even for \LaTeX\ to allow early loading.}
% Two simple \LaTeX\ macros from |ltdefns.dtx| have to be defined here
% because ltdefns.dtx is not loaded yet when ltluatex.dtx is executed.
%    \begin{macrocode}
\long\def\@gobble#1{}
\long\def\@firstofone#1{#1}
%    \end{macrocode}
%
% \subsection{Older \LaTeX{}/Plain \TeX\ setup}
%
%    \begin{macrocode}
%<*tex>
%    \end{macrocode}
%
% Older \LaTeX{} formats don't have the primitives with `native' names:
% sort that out. If they already exist this will still be safe.
%    \begin{macrocode}
\directlua{tex.enableprimitives("",tex.extraprimitives("luatex"))}
%    \end{macrocode}
%
%    \begin{macrocode}
\ifx\e@alloc\@undefined
%    \end{macrocode}
%
% In pre-2014 \LaTeX{}, or plain \TeX{}, load |etex.{sty,src}|.
%    \begin{macrocode}
  \ifx\documentclass\@undefined
    \ifx\loccount\@undefined
      % \iffalse meta-comment
%
% Copyright 1997, 1998, 2008 2015 2016 LaTeX Project and Peter Breitenlohner.
% 
% This file (etex.sty) may be distributed and/or modified under the
% conditions of the LaTeX Project Public License, either version 1.3 of
% this license or (at your option) any later version.  The latest
% version of this license is in
%   http://www.latex-project.org/lppl.txt
% and version 1.3 or later is part of all distributions of LaTeX
% version 2003/12/01 or later.
% 
% This work has the LPPL maintenance status "maintained".
% 
% The Current Maintainer of this work is David Carlisle.
% https://github.com/davidcarlisle/dpctex/issues
% \fi

\NeedsTeXFormat{LaTeX2e}
\ProvidesPackage{etex}
%        [1997/08/12 v0.1 eTeX basic definition package (DPC)]
%        [1998/03/26 v2.0 eTeX basic definition package (PEB)]
%        [2015/03/02 v2.1 eTeX basic definition package (PEB,DPC)]
%        [2015/07/06 v2.2 eTeX basic definition package (PEB,DPC)]
%        [2015/07/08 v2.3 eTeX basic definition package (PEB,DPC)]
%        [2015/09/02 v2.4 eTeX basic definition package (PEB,DPC)]
%        [2016/01/07 v2.5 eTeX basic definition package (PEB,DPC)]
%        [2016/01/11 v2.6 eTeX basic definition package (PEB,DPC)]
         [2016/08/01 v2.7 eTeX basic definition package (PEB,DPC)]

%%%%%%%%%%%%%%%%%%%%%%%%%%%%%%%%%%%%%%%%%%%%%%%%%%%%%%%%%%%%%%%%%%%%%%%%

%% A basic interface to some etex primitives, closely modeled on
%% etex.src and etexdefs.lib provided by the core etex team.

%% The etex.src `module' system is not copied here, the standard
%% LaTeX package option mechanism is used instead,
%% however the package options match the module names.
%% (Currently grouptypes, interactionmodes, nodetypes, iftypes.)
%% The individual type names are different too: We use, e.g.,
%%
%% `\bottomleveltype' and `\simplegrouptype' instead of
%% `\grouptypes{bottomlevel}' and `\grouptypes{simple}'.

%%%%%%%%%%%%%%%%%%%%%%%%%%%%%%%%%%%%%%%%%%%%%%%%%%%%%%%%%%%%%%%%%%%%%%%%

%% Other Comments...

%% The names of the `interactionmodes' are not too good.
%% In particular \scroll and \batch are likely to clash with existing
%% uses. These names have been changed into \batchinteractionmode,
%% \scrollinteractionmode etc.
%% Similarly, the names of the `groupetypes' have been changed, in
%% particular \mathgroup would conflict with the LaTeX kernel.

%% \etex logo could have the same trick as \LaTeXe to pick up a bold
%% epsilon when needed. (Not done here, I hate wasting tokens on logos.)
%% This version does have a \m@th not in the original.

%% The \globcountvector, \loccountvector, etc. allocation macros are
%% not (yet) implemented.

%% Currently if run on a standard TeX, the package generates an error.
%% Perhaps it should instead load some code to try to fake
%% the new etex primitives in that case???
%% Likewise, the package generates an error when used with e-TeX V 1

%% The etex.src language mechanism is not copied here. That facility
%% does not use any of the etex features. LaTeX should be customised
%% using the same hyphen.cfg mechanism as for a format built with a
%% standard TeX.

%% David Carlisle

%% Upgraded for e-TeX V 2.0
%% Peter Breitenlohner

%%%%%%%%%%%%%%%%%%%%%%%%%%%%%%%%%%%%%%%%%%%%%%%%%%%%%%%%%%%%%%%%%%%%%%%%



\ifx\eTeXversion\@undefined
  \PackageError{etex}
    {This package may only be run using an\MessageBreak
     etex in extended mode}
    {Perhaps you forgot the `*' when making the format with (e)initex.%
    }
\fi

\ifnum\eTeXversion<2
  \PackageError{etex}
    {This package requires e-TeX V 2}
    {You are probably using the obsolete e-TeX V 1.%
    }
\fi

% 2.2
% Check if the new latex 2015/01/01 allocation is already using
% extended reisters. If so it is too late to change allocation scheme.
% Older versions of LaTeX would have given an error when the classic
% TeX registers were all allocated, but newer formats allocate from
% the extended range, so usually this package is not needed.
\@tempswafalse
\ifnum\count10>\@cclv\@tempswatrue\else
\ifnum\count11>\@cclv\@tempswatrue\else
\ifnum\count12>\@cclv\@tempswatrue\else
\ifnum\count13>\@cclv\@tempswatrue\else
\ifnum\count14>\@cclv\@tempswatrue\else
\ifnum\count15>\@cclv\@tempswatrue
\fi\fi\fi\fi\fi\fi

\if@tempswa
\PackageWarningNoLine{etex}{%
Extended allocation already in use.\MessageBreak
etex.sty code will not be used.\MessageBreak
To force etex package to load, add\MessageBreak
\string\RequirePackage{etex}\MessageBreak
at the start of the document}

% 2.5 define the global allocation to be the standard ones
% as extended allocation is already in use. Helps with
% compatibility with some packages that use these commands 
% after loading etex.
% 2.6 avoid error from outer if used with (e)plain
\expandafter\let\csname globcount\expandafter\endcsname
                \csname newcount\endcsname
\expandafter\let\csname globdimen\expandafter\endcsname
                \csname newdimen\endcsname
\expandafter\let\csname globskip\expandafter\endcsname
                \csname newskip\endcsname
\expandafter\let\csname globmuskip\expandafter\endcsname
                \csname newmuskip\endcsname
\expandafter\let\csname globtoks\expandafter\endcsname
                \csname newtoks\endcsname
\expandafter\let\csname globmarks\expandafter\endcsname
                \csname newmarks\endcsname
% end of 2.5/2.6 change

\expandafter\endinput\fi

% End of 2.2 addition.

% 2.3 move option handling after the above error checks.
\DeclareOption{grouptypes}{\catcode`\G=9}
\DeclareOption{interactionmodes}{\catcode`\I=9}
\DeclareOption{nodetypes}{\catcode`\N=9}
\DeclareOption{iftypes}{\catcode`\C=9}
\DeclareOption{localalloclog}{\let\et@xwlog\wlog} % the default
\DeclareOption{localallocnolog}{\let\et@xwlog\@gobble} % be quiet
\DeclareOption{localallocshow}{\let\et@xwlog\typeout} % debugging
% End of 2.3 addition.

% v2.7
% \extrafloats does not work with this package
% but make it give a sensible error, not mis-parse \ifnum.
%
% Note that using \extrafloats earlier might not be safe as
% it could in principle clash with registers used for local allocation.
% However it probably works (as local allocation is used locally...).
% A better fix would be not to load this package with current LaTeX.
% This current etex package is just provided to force the old behaviour
% and such documents should not be using new features such as \extrafloats).
\ifdefined\extrafloats
\def\extrafloats#1{%
  \PackageError{etex}{%
    \noexpand\extrafloats is incompatible with etex.sty allocation.\MessageBreak
    Try using \noexpand\extrafloats before loading etex}%
    \@ehc}
\fi
% end of v2.7 change

\def\eTeX{%
  $\m@th\varepsilon$-\TeX}

\def\tracingall{%
  \tracingcommands\thr@@        % etex
  \tracingstats\tw@
  \tracingpages\@ne
  \tracinglostchars\tw@         % etex
  \tracingmacros\tw@
  \tracingparagraphs\@ne
  \tracingrestores\@ne
  \tracinggroups\@ne            % etex
  \tracingifs\@ne               % etex
  \tracingscantokens\@ne        % etex
  \tracingnesting\@ne           % etex
  \tracingassigns\@ne           % etex
  \errorcontextlines\maxdimen
  \showoutput}

\def\loggingall{%
  \tracingall
  \tracingonline\z@}

\def\tracingnone{%
  \tracingonline\z@
  \showboxdepth\m@ne
  \showboxbreadth\m@ne
  \tracingoutput\z@
  \errorcontextlines\m@ne
  \tracingassigns\z@
  \tracingnesting\z@
  \tracingscantokens\z@
  \tracingifs\z@
  \tracinggroups\z@
  \tracingrestores\z@
  \tracingparagraphs\z@
  \tracingmacros\z@
  \tracinglostchars\@ne
  \tracingpages\z@
  \tracingstats\z@
  \tracingcommands\z@}

%% Register allocation
%% We have to adjust the Plain TeX / LaTeX register allocation counts
%% for our slightly modified book-keeping, but first we allocate our
%% insertion counter \et@xins, because \insc@ount of Plain TeX / LaTeX
%% will be used differently.

\newcount\et@xins

\advance\count10 by 1 % \count10=23 % allocates \count registers 23, 24, ...
\advance\count11 by 1 % \count11=10 % allocates \dimen registers 10, 11, ...
\advance\count12 by 1 % \count12=10 % allocates \skip registers 10, 11, ...
\advance\count13 by 1 % \count13=10 % allocates \muskip registers 10, 11, ...
\advance\count14 by 1 % \count14=10 % allocates \box registers 10, 11, ...
\advance\count15 by 1 % \count15=10 % allocates \toks registers 10, 11, ...
\advance\count16 by 1 % \count16=0 % allocates input streams 0, 1, ...
\advance\count17 by 1 % \count17=0 % allocates output streams 0, 1, ...
\advance\count18 by 1 % \count18=4 % allocates math families 4, 5, ...
\advance\count19 by 1 % \count19=0 % allocates \language codes 0, 1, ...

\et@xins=\insc@unt % \et@xins=255 % allocates insertions 254, 253, ...


%% To ensure working in LaTeX 2015 release do define \newcount etc
%% with their pre 2015 LaTeX definitions
\def\newcount{\alloc@0\count\countdef\insc@unt}
\def\newdimen{\alloc@1\dimen\dimendef\insc@unt}
\def\newskip{\alloc@2\skip\skipdef\insc@unt}
\def\newmuskip{\alloc@3\muskip\muskipdef\@cclvi}
\def\newbox{\alloc@4\box\chardef\insc@unt}
\def\newtoks{\alloc@5\toks\toksdef\@cclvi}
\def\newread{\alloc@6\read\chardef\sixt@@n}
\def\newwrite{\alloc@7\write\chardef\sixt@@n}
\def\new@mathgroup{\alloc@8\mathgroup\chardef\sixt@@n}
\let\newfam\new@mathgroup
\def\newlanguage{\alloc@9\language\chardef\@cclvi}

%% When the normal register pool for \count, \dimen, \skip, \muskip,
%% \box, or \toks registers is exhausted, we switch to the extended pool.

\def\alloc@#1#2#3#4#5%
 {\ifnum\count1#1<#4% make sure there's still room
    \allocationnumber\count1#1
    \global\advance\count1#1\@ne
    \global#3#5\allocationnumber
    \wlog{\string#5=\string#2\the\allocationnumber}%
  \else\ifnum#1<6
    \begingroup \escapechar\m@ne
    \expandafter\alloc@@\expandafter{\string#2}#5%
  \else\errmessage{No room for a new #2}\fi\fi
 }

%% The \expandafter construction used here allows the generation of
%% \newcount and \globcount from #1=count.

\def\alloc@@#1#2%
 {\endgroup % restore \escapechar
  \wlog{Normal \csname#1\endcsname register pool exhausted,
    switching to extended pool.}%
  \global\expandafter\let
    \csname new#1\expandafter\endcsname
    \csname glob#1\endcsname
  \csname new#1\endcsname#2%
 }

%% We do change the LaTeX definition of \newinsert

\def\newinsert#1{% make sure there's still room for ...
  \ch@ck0\et@xins\count{% ... a \count, ...
    \ch@ck1\et@xins\dimen{% ... \dimen, ...
      \ch@ck2\et@xins\skip{% ... \skip, ...
        \ch@ck4\et@xins\box{% ... and \box register
  \global\advance\et@xins\m@ne
  \unless\ifnum\insc@unt<\et@xins \global\insc@unt\et@xins \fi
  \allocationnumber\et@xins
  \global\chardef#1\allocationnumber
  \wlog{\string#1=\string\insert\the\allocationnumber}}}}}}

\def\ch@ck#1#2#3#4%
 {\ifnum\count1#1<#2#4\else\errmessage{No room for a new #3}\fi}

%% And we define \reserveinserts, so that you can say \reserveinserts{17}
%% in order to reserve room for up to 17 additional insertion classes, that
%% will not be taken away by \newcount, \newdimen, \newskip, or \newbox.

% 2.4 Remove \outer to match LaTeX allocations
% which are never \outer unlike plain TeX.

%\outer
\def\reserveinserts#1%
 {\global\insc@unt\numexpr \et@xins \ifnum#1>\z@ -#1\fi \relax}

% Now, we define \globcount, \globbox, etc., so that you can say
% \globcount\foo and \foo will be defined (with \countdef) to be the
% next count register from the vastly larger but somewhat less efficient
% extended register pool. We also define \loccount, etc., but these
% register definitions are local to the current group.

\count260=277 % globally allocates \count registers 277, 278, ...
\count261=256 % globally allocates \dimen registers 256, 257, ...
\count262=256 % globally allocates \skip registers 256, 257, ...
\count263=256 % globally allocates \muskip registers 256, 257, ...
\count264=256 % globally allocates \box registers 256, 257, ...
\count265=256 % globally allocates \toks registers 256, 257, ...
\count266=1 % globally allocates \marks classes 1, 2, ...

\count270=32768 % locally allocates \count registers 32767, 32766, ...
\count271=32768 % ditto for \dimen registers
\count272=32768 % ditto for \skip registers
\count273=32768 % ditto for \muskip registers
\count274=32768 % ditto for \box registers
\count275=32768 % ditto for \toks registers
\count276=32768 % ditto for \marks classes

% \count registers 256-259 and 267-269 are not (yet) used

% \def \et@xglob #1#2#3#4% <offset>, <type>, <method>, <register>
% \def \et@xloc #1#2#3#4% <offset>, <type>, <method>, <register>

\def \globcount  {\et@xglob 0\count  \countdef}
\def \loccount   {\et@xloc  0\count  \countdef}
\def \globdimen  {\et@xglob 1\dimen  \dimendef}
\def \locdimen   {\et@xloc  1\dimen  \dimendef}
\def \globskip   {\et@xglob 2\skip   \skipdef}
\def \locskip    {\et@xloc  2\skip   \skipdef}
\def \globmuskip {\et@xglob 3\muskip \muskipdef}
\def \locmuskip  {\et@xloc  3\muskip \muskipdef}
\def \globbox    {\et@xglob 4\box    \mathchardef}
\def \locbox     {\et@xloc  4\box    \mathchardef}
\def \globtoks   {\et@xglob 5\toks   \toksdef}
\def \loctoks    {\et@xloc  5\toks   \toksdef}
\def \globmarks  {\et@xglob 6\marks  \mathchardef}
\def \locmarks   {\et@xloc  6\marks  \mathchardef}

\let\newmarks=\globmarks %% this used to be \newmark for e-TeX V 1.1

\def\et@xglob#1#2#3#4%
 {\et@xchk#1#2{% make sure there's still room
  \allocationnumber=\count26#1%
  \global\advance\count26#1\@ne
  \global#3#4\allocationnumber
  \wlog{\string#4=\string#2\the\allocationnumber}}%
 }

\def\et@xloc#1#2#3#4%
 {\et@xchk#1#2{% make sure there's still room
  \advance\count27#1by\m@ne
  \allocationnumber=\count27#1%
  #3#4=\allocationnumber
  \et@xwlog{\string#4=\string#2\the\allocationnumber\space(local)}}%
 }

%% The allocation messages for local allocations use \et@xwlog, such that
%% these messages can easily be switched on/off

\let\et@xwlog=\wlog

\def\et@xchk#1#2#3%
 {\ifnum\count26#1<\count27#1 #3\else\errmessage{No room for a new #2}\fi}

% Next we define \globcountblk, \loccountblk, etc., so that you can
% say \globcountblk\foo{17} and \foo will be defined (with \mathchardef)
% as the first (the zeroth?) of a block of 17 consecutive registers.
% Thus the user is intended to reference elements <\foo+0> to <\foo+n-1>,
% where n is the length of the block allocated.

% \def \et@xgblk #1#2#3#4% <offset>, <type>, <register>, <size>
% \def \et@xlblk #1#2#3#4% <offset>, <type>, <register>, <size>

\def\globcountblk  {\et@xgblk 0\count  }
\def\loccountblk   {\et@xlblk 0\count  }
\def\globdimenblk  {\et@xgblk 1\dimen  }
\def\locdimenblk   {\et@xlblk 1\dimen  }
\def\globskipblk   {\et@xgblk 2\skip   }
\def\locskipblk    {\et@xlblk 2\skip   }
\def\globmuskipblk {\et@xgblk 3\muskip }
\def\locmuskipblk  {\et@xlblk 3\muskip }
\def\globboxblk    {\et@xgblk 4\box    }
\def\locboxblk     {\et@xlblk 4\box    }
\def\globtoksblk   {\et@xgblk 5\toks   }
\def\loctoksblk    {\et@xlblk 5\toks   }
\def\globmarksblk  {\et@xgblk 6\marks  }
\def\locmarksblk   {\et@xlblk 6\marks  }

% \def\et@xchkblk#1#1#3#4% <offset>, <type>, <size>, <action>

\def\et@xgblk#1#2#3#4%
 {\et@xchkblk#1#2{#4}% make sure there's still room
   {\allocationnumber\count26#1%
    \global\advance\count26#1by#4%
    \global\mathchardef#3\allocationnumber
    \wlog{\string#3=\string#2blk{\number#4} at
      \the\allocationnumber}%
   }%
 }

\def\et@xlblk#1#2#3#4%
 {\et@xchkblk#1#2{#4}% make sure there's still room
   {\advance\count27#1-#4%
    \allocationnumber\count27#1%
    \mathchardef#3\allocationnumber
    \et@xwlog{\string#3=\string#2blk{\number#4} at
      \the\allocationnumber\space(local)}%
   }%
 }

\def\et@xchkblk#1#2#3#4%
 {\ifnum#3<\z@
    \errmessage{Negative register block size \number#3}%
  \else\ifnum\numexpr\count26#1+#3>\count27#1%
    \errmessage{No room for new #2block of size \number#3}%
  \else #4\fi \fi
 }

\catcode`\G=14
\catcode`\I=14
\catcode`\N=14
\catcode`\C=14

\ProcessOptions

%% Declare names for `grouptypes'

G \chardef \bottomleveltype       =  0 % for the outside world
G \chardef \simplegrouptype       =  1 % for local structure only
G \chardef \hboxgrouptype         =  2 % for `\hbox{}'
G \chardef \adjustedhboxgrouptype =  3 % for `\hbox{}' in vertical mode
G \chardef \vboxgrouptype         =  4 % for `\vbox{}'
G \chardef \vtopgrouptype         =  5 % for `\vtop{}'
G \chardef \aligngrouptype        =  6 % for `\halign{}', `\valign{}'
G \chardef \noaligngrouptype      =  7 % for `\noalign{}'
G \chardef \outputgrouptype       =  8 % for output routine
G \chardef \mathgrouptype         =  9 % for, e.g, `^{}'
G \chardef \discgrouptype         = 10 % for `\discretionary{}{}{}'
G \chardef \insertgrouptype       = 11 % for `\insert{}', `\vadjust{}'
G \chardef \vcentergrouptype      = 12 % for `\vcenter{}'
G \chardef \mathchoicegrouptype   = 13 % for `\mathchoice{}{}{}{}'
G \chardef \semisimplegrouptype   = 14 % for `\begingroup...\endgroup'
G \chardef \mathshiftgrouptype    = 15 % for `$...$'
G \chardef \mathleftgrouptype     = 16 % for `\left...\right'

%% Declare names for `interactionmodes'

I \chardef \batchinteractionmode     = 0 % omits all stops and omits terminal output
I \chardef \nonstopinteractionmode   = 1 % omits all stops
I \chardef \scrollinteractionmode    = 2 % omits error stops
I \chardef \errorstopinteractionmode = 3 % stops at every opportunity to interact

%% Declare names for `nodetypes'

N \chardef \charnode     =  0 % character nodes
N \chardef \hlistnode    =  1 % hlist nodes
N \chardef \vlistnode    =  2 % vlist nodes
N \chardef \rulenode     =  3 % rule nodes
N \chardef \insnode      =  4 % insertion nodes
N \chardef \marknode     =  5 % a mark node
N \chardef \adjustnode   =  6 % an adjust node
N \chardef \ligaturenode =  7 % a ligature node
N \chardef \discnode     =  8 % a discretionary node
N \chardef \whatsitnode  =  9 % special extension nodes
N \chardef \mathnode     = 10 % a math node
N \chardef \gluenode     = 11 % node that points to a glue specification
N \chardef \kernnode     = 12 % a kern node
N \chardef \penaltynode  = 13 % a penalty node
N \chardef \unsetnode    = 14 % an unset node
N \chardef \mathsnodes   = 15 % nodes that occur only in maths mode

%% Declare names for `iftypes'

C \chardef \charif     =  1 % \if
C \chardef \catif      =  2 % \ifcat
C \chardef \numif      =  3 % \ifnum
C \chardef \dimif      =  4 % \ifdim
C \chardef \oddif      =  5 % \ifodd
C \chardef \vmodeif    =  6 % \ifvmode
C \chardef \hmodeif    =  7 % \ifhmode
C \chardef \mmodeif    =  8 % \ifmmode
C \chardef \innerif    =  9 % \ifinner
C \chardef \voidif     = 10 % \ifvoid
C \chardef \hboxif     = 11 % \ifhbox
C \chardef \vboxif     = 12 % \ifvbox
C \chardef \xif        = 13 % \ifx
C \chardef \eofif      = 14 % \ifeof
C \chardef \trueif     = 15 % \iftrue
C \chardef \falseif    = 16 % \iffalse
C \chardef \caseif     = 17 % \ifcase
C \chardef \definedif  = 18 % \ifdefined
C \chardef \csnameif   = 19 % \ifcsname
C \chardef \fontcharif = 20 % \iffontchar

\catcode`\G=11
\catcode`\I=11
\catcode`\N=11
\catcode`\C=11

%
    \fi
    \catcode`\@=11 %
    \outer\expandafter\def\csname newfam\endcsname
                          {\alloc@8\fam\chardef\et@xmaxfam}
  \else
    \RequirePackage{etex}
    \expandafter\def\csname newfam\endcsname
                    {\alloc@8\fam\chardef\et@xmaxfam}
    \expandafter\let\expandafter\new@mathgroup\csname newfam\endcsname
  \fi
%    \end{macrocode}
%
% \subsubsection{Fixes to \texttt{etex.src}/\texttt{etex.sty}}
%
% These could and probably should be made directly in an
% update to |etex.src| which already has some Lua\TeX-specific
% code, but does not define the correct range for Lua\TeX.
%
% 2015-07-13 higher range in luatex.
%    \begin{macrocode}
\edef \et@xmaxregs {\ifx\directlua\@undefined 32768\else 65536\fi}
%    \end{macrocode}
% luatex/xetex also allow more math fam.
%    \begin{macrocode}
\edef \et@xmaxfam {\ifx\Umathcode\@undefined\sixt@@n\else\@cclvi\fi}
%    \end{macrocode}
%
%    \begin{macrocode}
\count 270=\et@xmaxregs % locally allocates \count registers
\count 271=\et@xmaxregs % ditto for \dimen registers
\count 272=\et@xmaxregs % ditto for \skip registers
\count 273=\et@xmaxregs % ditto for \muskip registers
\count 274=\et@xmaxregs % ditto for \box registers
\count 275=\et@xmaxregs % ditto for \toks registers
\count 276=\et@xmaxregs % ditto for \marks classes
%    \end{macrocode}
%
% and 256 or 16 fam. (Done above due to plain/\LaTeX\ differences in
% \textsf{ltluatex}.)
%    \begin{macrocode}
% \outer\def\newfam{\alloc@8\fam\chardef\et@xmaxfam}
%    \end{macrocode}
%
% End of proposed changes to \texttt{etex.src}
%
% \subsubsection{luatex specific settings}
%
% Switch to global cf |luatex.sty| to leave room for inserts
% not really needed for luatex but possibly most compatible
% with existing use.
%    \begin{macrocode}
\expandafter\let\csname newcount\expandafter\expandafter\endcsname
                \csname globcount\endcsname
\expandafter\let\csname newdimen\expandafter\expandafter\endcsname
                \csname globdimen\endcsname
\expandafter\let\csname newskip\expandafter\expandafter\endcsname
                \csname globskip\endcsname
\expandafter\let\csname newbox\expandafter\expandafter\endcsname
                \csname globbox\endcsname
%    \end{macrocode}
%
% Define|\e@alloc| as in latex (the existing macros in |etex.src|
% hard to extend to further register types as they assume specific
% 26x and 27x count range. For compatibility the existing register
% allocation is not changed.
%
%    \begin{macrocode}
\chardef\e@alloc@top=65535
\let\e@alloc@chardef\chardef
%    \end{macrocode}
%
%    \begin{macrocode}
\def\e@alloc#1#2#3#4#5#6{%
  \global\advance#3\@ne
  \e@ch@ck{#3}{#4}{#5}#1%
  \allocationnumber#3\relax
  \global#2#6\allocationnumber
  \wlog{\string#6=\string#1\the\allocationnumber}}%
%    \end{macrocode}
%
%    \begin{macrocode}
\gdef\e@ch@ck#1#2#3#4{%
  \ifnum#1<#2\else
    \ifnum#1=#2\relax
      #1\@cclvi
      \ifx\count#4\advance#1 10 \fi
    \fi
    \ifnum#1<#3\relax
    \else
      \errmessage{No room for a new \string#4}%
    \fi
  \fi}%
%    \end{macrocode}
%
% Fix up allocations not to clash with |etex.src|.
%
%    \begin{macrocode}
\expandafter\csname newcount\endcsname\e@alloc@attribute@count
\expandafter\csname newcount\endcsname\e@alloc@ccodetable@count
\expandafter\csname newcount\endcsname\e@alloc@luafunction@count
\expandafter\csname newcount\endcsname\e@alloc@whatsit@count
\expandafter\csname newcount\endcsname\e@alloc@bytecode@count
\expandafter\csname newcount\endcsname\e@alloc@luachunk@count
%    \end{macrocode}
%
% End of conditional setup for plain \TeX\ / old \LaTeX.
%    \begin{macrocode}
\fi
%</tex>
%    \end{macrocode}
%
% \subsection{Attributes}
%
% \begin{macro}{\newattribute}
% \changes{v1.0a}{2015/09/24}{Macro added}
% \changes{v1.1q}{2020/08/02}{Move reset to $0$ inside conditional}
%   As is generally the case for the Lua\TeX{} registers we start here
%   from~$1$. Notably, some code assumes that |\attribute0| is never used so
%   this is important in this case.
%    \begin{macrocode}
\ifx\e@alloc@attribute@count\@undefined
  \countdef\e@alloc@attribute@count=258
  \e@alloc@attribute@count=\z@
\fi
\def\newattribute#1{%
  \e@alloc\attribute\attributedef
    \e@alloc@attribute@count\m@ne\e@alloc@top#1%
}
%    \end{macrocode}
% \end{macro}
%
% \begin{macro}{\setattribute}
% \begin{macro}{\unsetattribute}
%   Handy utilities.
%    \begin{macrocode}
\def\setattribute#1#2{#1=\numexpr#2\relax}
\def\unsetattribute#1{#1=-"7FFFFFFF\relax}
%    \end{macrocode}
% \end{macro}
% \end{macro}
%
% \subsection{Category code tables}
%
% \begin{macro}{\newcatcodetable}
% \changes{v1.0a}{2015/09/24}{Macro added}
%   Category code tables are allocated with a limit half of that used by Lua\TeX{}
%   for everything else. At the end of allocation there needs to be an
%   initialization step. Table~$0$ is already taken (it's the global one for
%   current use) so the allocation starts at~$1$.
%    \begin{macrocode}
\ifx\e@alloc@ccodetable@count\@undefined
  \countdef\e@alloc@ccodetable@count=259
  \e@alloc@ccodetable@count=\z@
\fi
\def\newcatcodetable#1{%
  \e@alloc\catcodetable\chardef
    \e@alloc@ccodetable@count\m@ne{"8000}#1%
  \initcatcodetable\allocationnumber
}
%    \end{macrocode}
% \end{macro}
%
% \changes{v1.0l}{2015/12/18}{Load Unicode data from source}
% \begin{macro}{\catcodetable@initex}
% \changes{v1.0a}{2015/09/24}{Macro added}
% \begin{macro}{\catcodetable@string}
% \changes{v1.0a}{2015/09/24}{Macro added}
% \begin{macro}{\catcodetable@latex}
% \changes{v1.0a}{2015/09/24}{Macro added}
% \begin{macro}{\catcodetable@atletter}
% \changes{v1.0a}{2015/09/24}{Macro added}
%   Save a small set of standard tables. The Unicode data is read
%   here in using a parser simplified from that in |load-unicode-data|:
%   only the nature of letters needs to be detected.
%    \begin{macrocode}
\newcatcodetable\catcodetable@initex
\newcatcodetable\catcodetable@string
\begingroup
  \def\setrangecatcode#1#2#3{%
    \ifnum#1>#2 %
      \expandafter\@gobble
    \else
      \expandafter\@firstofone
    \fi
      {%
        \catcode#1=#3 %
        \expandafter\setrangecatcode\expandafter
          {\number\numexpr#1 + 1\relax}{#2}{#3}
      }%
  }
  \@firstofone{%
    \catcodetable\catcodetable@initex
      \catcode0=12 %
      \catcode13=12 %
      \catcode37=12 %
      \setrangecatcode{65}{90}{12}%
      \setrangecatcode{97}{122}{12}%
      \catcode92=12 %
      \catcode127=12 %
      \savecatcodetable\catcodetable@string
    \endgroup
  }%
\newcatcodetable\catcodetable@latex
\newcatcodetable\catcodetable@atletter
\begingroup
  \def\parseunicodedataI#1;#2;#3;#4\relax{%
    \parseunicodedataII#1;#3;#2 First>\relax
  }%
  \def\parseunicodedataII#1;#2;#3 First>#4\relax{%
    \ifx\relax#4\relax
      \expandafter\parseunicodedataIII
    \else
      \expandafter\parseunicodedataIV
    \fi
      {#1}#2\relax%
  }%
  \def\parseunicodedataIII#1#2#3\relax{%
    \ifnum 0%
      \if L#21\fi
      \if M#21\fi
      >0 %
      \catcode"#1=11 %
    \fi
  }%
  \def\parseunicodedataIV#1#2#3\relax{%
    \read\unicoderead to \unicodedataline
    \if L#2%
      \count0="#1 %
      \expandafter\parseunicodedataV\unicodedataline\relax
    \fi
  }%
  \def\parseunicodedataV#1;#2\relax{%
    \loop
      \unless\ifnum\count0>"#1 %
        \catcode\count0=11 %
        \advance\count0 by 1 %
    \repeat
  }%
  \def\storedpar{\par}%
  \chardef\unicoderead=\numexpr\count16 + 1\relax
  \openin\unicoderead=UnicodeData.txt %
  \loop\unless\ifeof\unicoderead %
    \read\unicoderead to \unicodedataline
    \unless\ifx\unicodedataline\storedpar
      \expandafter\parseunicodedataI\unicodedataline\relax
    \fi
  \repeat
  \closein\unicoderead
  \@firstofone{%
    \catcode64=12 %
    \savecatcodetable\catcodetable@latex
    \catcode64=11 %
    \savecatcodetable\catcodetable@atletter
   }
\endgroup
%    \end{macrocode}
% \end{macro}
% \end{macro}
% \end{macro}
% \end{macro}
%
% \subsection{Named Lua functions}
%
% \begin{macro}{\newluafunction}
% \changes{v1.0a}{2015/09/24}{Macro added}
% \changes{v1.1q}{2020/08/02}{Move reset to $0$ inside conditional}
%   Much the same story for allocating Lua\TeX{} functions except here they are
%   just numbers so they are allocated in the same way as boxes.
%   Lua indexes from~$1$ so once again slot~$0$ is skipped.
%    \begin{macrocode}
\ifx\e@alloc@luafunction@count\@undefined
  \countdef\e@alloc@luafunction@count=260
  \e@alloc@luafunction@count=\z@
\fi
\def\newluafunction{%
  \e@alloc\luafunction\e@alloc@chardef
    \e@alloc@luafunction@count\m@ne\e@alloc@top
}
%    \end{macrocode}
% \end{macro}
%
% \subsection{Custom whatsits}
%
% \begin{macro}{\newwhatsit}
% \changes{v1.0a}{2015/09/24}{Macro added}
% \changes{v1.1q}{2020/08/02}{Move reset to $0$ inside conditional}
%   These are only settable from Lua but for consistency are definable
%   here.
%    \begin{macrocode}
\ifx\e@alloc@whatsit@count\@undefined
  \countdef\e@alloc@whatsit@count=261
  \e@alloc@whatsit@count=\z@
\fi
\def\newwhatsit#1{%
  \e@alloc\whatsit\e@alloc@chardef
    \e@alloc@whatsit@count\m@ne\e@alloc@top#1%
}
%    \end{macrocode}
% \end{macro}
%
% \subsection{Lua bytecode registers}
%
% \begin{macro}{\newluabytecode}
% \changes{v1.0a}{2015/09/24}{Macro added}
% \changes{v1.1q}{2020/08/02}{Move reset to $0$ inside conditional}
%   These are only settable from Lua but for consistency are definable
%   here.
%    \begin{macrocode}
\ifx\e@alloc@bytecode@count\@undefined
  \countdef\e@alloc@bytecode@count=262
  \e@alloc@bytecode@count=\z@
\fi
\def\newluabytecode#1{%
  \e@alloc\luabytecode\e@alloc@chardef
    \e@alloc@bytecode@count\m@ne\e@alloc@top#1%
}
%    \end{macrocode}
% \end{macro}
%
% \subsection{Lua chunk registers}

% \begin{macro}{\newluachunkname}
% \changes{v1.0a}{2015/09/24}{Macro added}
% \changes{v1.1q}{2020/08/02}{Move reset to $0$ inside conditional}
% As for bytecode registers, but in addition we need to add a string
% to the \verb|lua.name| table to use in stack tracing. We use the
% name of the command passed to the allocator, with no backslash.
%    \begin{macrocode}
\ifx\e@alloc@luachunk@count\@undefined
  \countdef\e@alloc@luachunk@count=263
  \e@alloc@luachunk@count=\z@
\fi
\def\newluachunkname#1{%
  \e@alloc\luachunk\e@alloc@chardef
    \e@alloc@luachunk@count\m@ne\e@alloc@top#1%
    {\escapechar\m@ne
    \directlua{lua.name[\the\allocationnumber]="\string#1"}}%
}
%    \end{macrocode}
% \end{macro}
%
% \subsection{Lua loader}
% \changes{v1.1r}{2020/08/10}{Load ltluatex Lua module during format building}
%
% Lua code loaded in the format often has to be loaded again at the
% beginning of every job, so we define a helper which allows us to avoid
% duplicated code:
%
%    \begin{macrocode}
\def\now@and@everyjob#1{%
  \everyjob\expandafter{\the\everyjob
    #1%
  }%
  #1%
}
%    \end{macrocode}
%
% Load the Lua code at the start of every job.
% For the conversion of \TeX{} into numbers at the Lua side we need some
% known registers: for convenience we use a set of systematic names, which
% means using a group around the Lua loader.
%    \begin{macrocode}
%<2ekernel>\now@and@everyjob{%
  \begingroup
    \attributedef\attributezero=0 %
    \chardef     \charzero     =0 %
%    \end{macrocode}
% Note name change required on older luatex, for hash table access.
%    \begin{macrocode}
    \countdef    \CountZero    =0 %
    \dimendef    \dimenzero    =0 %
    \mathchardef \mathcharzero =0 %
    \muskipdef   \muskipzero   =0 %
    \skipdef     \skipzero     =0 %
    \toksdef     \tokszero     =0 %
    \directlua{require("ltluatex")}
  \endgroup
%<2ekernel>}
%<latexrelease>\EndIncludeInRelease
%    \end{macrocode}
%
% \changes{v1.0b}{2015/10/02}{Fix backing out of \TeX{} code}
% \changes{v1.0c}{2015/10/02}{Allow backing out of Lua code}
%    \begin{macrocode}
%<latexrelease>\IncludeInRelease{0000/00/00}
%<latexrelease>                 {\newluafunction}{LuaTeX}%
%<latexrelease>\let\e@alloc@attribute@count\@undefined
%<latexrelease>\let\newattribute\@undefined
%<latexrelease>\let\setattribute\@undefined
%<latexrelease>\let\unsetattribute\@undefined
%<latexrelease>\let\e@alloc@ccodetable@count\@undefined
%<latexrelease>\let\newcatcodetable\@undefined
%<latexrelease>\let\catcodetable@initex\@undefined
%<latexrelease>\let\catcodetable@string\@undefined
%<latexrelease>\let\catcodetable@latex\@undefined
%<latexrelease>\let\catcodetable@atletter\@undefined
%<latexrelease>\let\e@alloc@luafunction@count\@undefined
%<latexrelease>\let\newluafunction\@undefined
%<latexrelease>\let\e@alloc@luafunction@count\@undefined
%<latexrelease>\let\newwhatsit\@undefined
%<latexrelease>\let\e@alloc@whatsit@count\@undefined
%<latexrelease>\let\newluabytecode\@undefined
%<latexrelease>\let\e@alloc@bytecode@count\@undefined
%<latexrelease>\let\newluachunkname\@undefined
%<latexrelease>\let\e@alloc@luachunk@count\@undefined
%<latexrelease>\directlua{luatexbase.uninstall()}
%<latexrelease>\EndIncludeInRelease
%    \end{macrocode}
%
% In \verb|\everyjob|, if luaotfload is available, load it and switch to TU.
%    \begin{macrocode}
%<latexrelease>\IncludeInRelease{2017/01/01}%
%<latexrelease>                 {\fontencoding}{TU in everyjob}%
%<latexrelease>\fontencoding{TU}\let\encodingdefault\f@encoding
%<latexrelease>\ifx\directlua\@undefined\else
%<2ekernel>\everyjob\expandafter{%
%<2ekernel>  \the\everyjob
%<*2ekernel,latexrelease>
  \directlua{%
  if xpcall(function ()%
             require('luaotfload-main')%
            end,texio.write_nl) then %
  local _void = luaotfload.main ()%
  else %
  texio.write_nl('Error in luaotfload: reverting to OT1')%
  tex.print('\string\\def\string\\encodingdefault{OT1}')%
  end %
  }%
  \let\f@encoding\encodingdefault
  \expandafter\let\csname ver@luaotfload.sty\endcsname\fmtversion
%</2ekernel,latexrelease>
%<latexrelease>\fi
%<2ekernel>  }
%<latexrelease>\EndIncludeInRelease
%<latexrelease>\IncludeInRelease{0000/00/00}%
%<latexrelease>                 {\fontencoding}{TU in everyjob}%
%<latexrelease>\fontencoding{OT1}\let\encodingdefault\f@encoding
%<latexrelease>\EndIncludeInRelease
%    \end{macrocode}
%
%    \begin{macrocode}
%<2ekernel|latexrelease>\fi
%</2ekernel|tex|latexrelease>
%    \end{macrocode}
%
% \subsection{Lua module preliminaries}
%
% \begingroup
%
%  \begingroup\lccode`~=`_
%  \lowercase{\endgroup\let~}_
%  \catcode`_=12
%
%    \begin{macrocode}
%<*lua>
%    \end{macrocode}
%
% Some set up for the Lua module which is needed for all of the Lua
% functionality added here.
%
% \begin{macro}{luatexbase}
% \changes{v1.0a}{2015/09/24}{Table added}
%   Set up the table for the returned functions. This is used to expose
%   all of the public functions.
%    \begin{macrocode}
luatexbase       = luatexbase or { }
local luatexbase = luatexbase
%    \end{macrocode}
% \end{macro}
%
% Some Lua best practice: use local versions of functions where possible.
% \changes{v1.1u}{2021/08/11}{Define missing local function}
%    \begin{macrocode}
local string_gsub      = string.gsub
local tex_count        = tex.count
local tex_setattribute = tex.setattribute
local tex_setcount     = tex.setcount
local texio_write_nl   = texio.write_nl
local flush_list       = node.flush_list
%    \end{macrocode}
% \changes{v1.0i}{2015/11/29}{Declare this as local before used in the module error definitions (PHG)}
%    \begin{macrocode}
local luatexbase_warning
local luatexbase_error
%    \end{macrocode}
%
% \subsection{Lua module utilities}
%
% \subsubsection{Module tracking}
%
% \begin{macro}{modules}
% \changes{v1.0a}{2015/09/24}{Function modified}
%   To allow tracking of module usage, a structure is provided to store
%   information and to return it.
%    \begin{macrocode}
local modules = modules or { }
%    \end{macrocode}
% \end{macro}
%
% \begin{macro}{provides_module}
% \changes{v1.0a}{2015/09/24}{Function added}
% \changes{v1.0f}{2015/10/03}{use luatexbase\_log}
% Local function to write to the log.
%    \begin{macrocode}
local function luatexbase_log(text)
  texio_write_nl("log", text)
end
%    \end{macrocode}
%
%   Modelled on |\ProvidesPackage|, we store much the same information but
%   with a little more structure.
%    \begin{macrocode}
local function provides_module(info)
  if not (info and info.name) then
    luatexbase_error("Missing module name for provides_module")
  end
  local function spaced(text)
    return text and (" " .. text) or ""
  end
  luatexbase_log(
    "Lua module: " .. info.name
      .. spaced(info.date)
      .. spaced(info.version)
      .. spaced(info.description)
  )
  modules[info.name] = info
end
luatexbase.provides_module = provides_module
%    \end{macrocode}
% \end{macro}
%
% \subsubsection{Module messages}
%
% There are various warnings and errors that need to be given. For warnings
% we can get exactly the same formatting as from \TeX{}. For errors we have to
% make some changes. Here we give the text of the error in the \LaTeX{} format
% then force an error from Lua to halt the run. Splitting the message text is
% done using |\n| which takes the place of |\MessageBreak|.
%
% First an auxiliary for the formatting: this measures up the message
% leader so we always get the correct indent.
% \changes{v1.0j}{2015/12/02}{Declaration/use of first\_head fixed (PHG)}
%    \begin{macrocode}
local function msg_format(mod, msg_type, text)
  local leader = ""
  local cont
  local first_head
  if mod == "LaTeX" then
    cont = string_gsub(leader, ".", " ")
    first_head = leader .. "LaTeX: "
  else
    first_head = leader .. "Module "  .. msg_type
    cont = "(" .. mod .. ")"
      .. string_gsub(first_head, ".", " ")
    first_head =  leader .. "Module "  .. mod .. " " .. msg_type  .. ":"
  end
  if msg_type == "Error" then
    first_head = "\n" .. first_head
  end
  if string.sub(text,-1) ~= "\n" then
    text = text .. " "
  end
  return first_head .. " "
    .. string_gsub(
         text
	 .. "on input line "
         .. tex.inputlineno, "\n", "\n" .. cont .. " "
      )
   .. "\n"
end
%    \end{macrocode}
%
% \begin{macro}{module_info}
% \changes{v1.0a}{2015/09/24}{Function added}
% \begin{macro}{module_warning}
% \changes{v1.0a}{2015/09/24}{Function added}
% \begin{macro}{module_error}
% \changes{v1.0a}{2015/09/24}{Function added}
%   Write messages.
%    \begin{macrocode}
local function module_info(mod, text)
  texio_write_nl("log", msg_format(mod, "Info", text))
end
luatexbase.module_info = module_info
local function module_warning(mod, text)
  texio_write_nl("term and log",msg_format(mod, "Warning", text))
end
luatexbase.module_warning = module_warning
local function module_error(mod, text)
  error(msg_format(mod, "Error", text))
end
luatexbase.module_error = module_error
%    \end{macrocode}
% \end{macro}
% \end{macro}
% \end{macro}
%
% Dedicated versions for the rest of the code here.
%    \begin{macrocode}
function luatexbase_warning(text)
  module_warning("luatexbase", text)
end
function luatexbase_error(text)
  module_error("luatexbase", text)
end
%    \end{macrocode}
%
%
% \subsection{Accessing register numbers from Lua}
%
% \changes{v1.0g}{2015/11/14}{Track Lua\TeX{} changes for
%   \texttt{(new)token.create}}
% Collect up the data from the \TeX{} level into a Lua table: from
% version~0.80, Lua\TeX{} makes that easy.
% \changes{v1.0j}{2015/12/02}{Adjust hashtokens to store the result of tex.hashtokens()), not the function (PHG)}
%    \begin{macrocode}
local luaregisterbasetable = { }
local registermap = {
  attributezero = "assign_attr"    ,
  charzero      = "char_given"     ,
  CountZero     = "assign_int"     ,
  dimenzero     = "assign_dimen"   ,
  mathcharzero  = "math_given"     ,
  muskipzero    = "assign_mu_skip" ,
  skipzero      = "assign_skip"    ,
  tokszero      = "assign_toks"    ,
}
local createtoken
if tex.luatexversion > 81 then
  createtoken = token.create
elseif tex.luatexversion > 79 then
  createtoken = newtoken.create
end
local hashtokens    = tex.hashtokens()
local luatexversion = tex.luatexversion
for i,j in pairs (registermap) do
  if luatexversion < 80 then
    luaregisterbasetable[hashtokens[i][1]] =
      hashtokens[i][2]
  else
    luaregisterbasetable[j] = createtoken(i).mode
  end
end
%    \end{macrocode}
%
% \begin{macro}{registernumber}
%   Working out the correct return value can be done in two ways. For older
%   Lua\TeX{} releases it has to be extracted from the |hashtokens|. On the
%   other hand, newer Lua\TeX{}'s have |newtoken|, and whilst |.mode| isn't
%   currently documented, Hans Hagen pointed to this approach so we should be
%   OK.
%    \begin{macrocode}
local registernumber
if luatexversion < 80 then
  function registernumber(name)
    local nt = hashtokens[name]
    if(nt and luaregisterbasetable[nt[1]]) then
      return nt[2] - luaregisterbasetable[nt[1]]
    else
      return false
    end
  end
else
  function registernumber(name)
    local nt = createtoken(name)
    if(luaregisterbasetable[nt.cmdname]) then
      return nt.mode - luaregisterbasetable[nt.cmdname]
    else
      return false
    end
  end
end
luatexbase.registernumber = registernumber
%    \end{macrocode}
% \end{macro}
%
% \subsection{Attribute allocation}
%
% \begin{macro}{new_attribute}
% \changes{v1.0a}{2015/09/24}{Function added}
% \changes{v1.1c}{2017/02/18}{Parameterize count used in tracking}
%   As attributes are used for Lua manipulations its useful to be able
%   to assign from this end.
% \InternalDetectionOff
%    \begin{macrocode}
local attributes=setmetatable(
{},
{
__index = function(t,key)
return registernumber(key) or nil
end}
)
luatexbase.attributes = attributes
%    \end{macrocode}
%
%    \begin{macrocode}
local attribute_count_name =
                     attribute_count_name or "e@alloc@attribute@count"
local function new_attribute(name)
  tex_setcount("global", attribute_count_name,
                          tex_count[attribute_count_name] + 1)
  if tex_count[attribute_count_name] > 65534 then
    luatexbase_error("No room for a new \\attribute")
  end
  attributes[name]= tex_count[attribute_count_name]
  luatexbase_log("Lua-only attribute " .. name .. " = " ..
                 tex_count[attribute_count_name])
  return tex_count[attribute_count_name]
end
luatexbase.new_attribute = new_attribute
%    \end{macrocode}
% \InternalDetectionOn
% \end{macro}
%
% \subsection{Custom whatsit allocation}
%
% \begin{macro}{new_whatsit}
% \changes{v1.1c}{2017/02/18}{Parameterize count used in tracking}
% Much the same as for attribute allocation in Lua.
%    \begin{macrocode}
local whatsit_count_name = whatsit_count_name or "e@alloc@whatsit@count"
local function new_whatsit(name)
  tex_setcount("global", whatsit_count_name,
                         tex_count[whatsit_count_name] + 1)
  if tex_count[whatsit_count_name] > 65534 then
    luatexbase_error("No room for a new custom whatsit")
  end
  luatexbase_log("Custom whatsit " .. (name or "") .. " = " ..
                 tex_count[whatsit_count_name])
  return tex_count[whatsit_count_name]
end
luatexbase.new_whatsit = new_whatsit
%    \end{macrocode}
% \end{macro}
%
% \subsection{Bytecode register allocation}
%
% \begin{macro}{new_bytecode}
% \changes{v1.1c}{2017/02/18}{Parameterize count used in tracking}
% Much the same as for attribute allocation in Lua.
% The optional \meta{name} argument is used in the log if given.
%    \begin{macrocode}
local bytecode_count_name =
                         bytecode_count_name or "e@alloc@bytecode@count"
local function new_bytecode(name)
  tex_setcount("global", bytecode_count_name,
                         tex_count[bytecode_count_name] + 1)
  if tex_count[bytecode_count_name] > 65534 then
    luatexbase_error("No room for a new bytecode register")
  end
  luatexbase_log("Lua bytecode " .. (name or "") .. " = " ..
                 tex_count[bytecode_count_name])
  return tex_count[bytecode_count_name]
end
luatexbase.new_bytecode = new_bytecode
%    \end{macrocode}
% \end{macro}
%
% \subsection{Lua chunk name allocation}
%
% \begin{macro}{new_chunkname}
% \changes{v1.1c}{2017/02/18}{Parameterize count used in tracking}
% As for bytecode registers but also store the name in the
% |lua.name| table.
%    \begin{macrocode}
local chunkname_count_name =
                        chunkname_count_name or "e@alloc@luachunk@count"
local function new_chunkname(name)
  tex_setcount("global", chunkname_count_name,
                         tex_count[chunkname_count_name] + 1)
  local chunkname_count = tex_count[chunkname_count_name]
  chunkname_count = chunkname_count + 1
  if chunkname_count > 65534 then
    luatexbase_error("No room for a new chunkname")
  end
  lua.name[chunkname_count]=name
  luatexbase_log("Lua chunkname " .. (name or "") .. " = " ..
                 chunkname_count .. "\n")
  return chunkname_count
end
luatexbase.new_chunkname = new_chunkname
%    \end{macrocode}
% \end{macro}
%
% \subsection{Lua function allocation}
%
% \begin{macro}{new_luafunction}
% \changes{v1.1i}{2018/10/21}{Function added}
% Much the same as for attribute allocation in Lua.
% The optional \meta{name} argument is used in the log if given.
%    \begin{macrocode}
local luafunction_count_name =
                         luafunction_count_name or "e@alloc@luafunction@count"
local function new_luafunction(name)
  tex_setcount("global", luafunction_count_name,
                         tex_count[luafunction_count_name] + 1)
  if tex_count[luafunction_count_name] > 65534 then
    luatexbase_error("No room for a new luafunction register")
  end
  luatexbase_log("Lua function " .. (name or "") .. " = " ..
                 tex_count[luafunction_count_name])
  return tex_count[luafunction_count_name]
end
luatexbase.new_luafunction = new_luafunction
%    \end{macrocode}
% \end{macro}
%
% \subsection{Lua callback management}
%
% The native mechanism for callbacks in Lua\TeX\ allows only one per function.
% That is extremely restrictive and so a mechanism is needed to add and
% remove callbacks from the appropriate hooks.
%
% \subsubsection{Housekeeping}
%
% The main table: keys are callback names, and values are the associated lists
% of functions. More precisely, the entries in the list are tables holding the
% actual function as |func| and the identifying description as |description|.
% Only callbacks with a non-empty list of functions have an entry in this
% list.
%    \begin{macrocode}
local callbacklist = callbacklist or { }
%    \end{macrocode}
%
% Numerical codes for callback types, and name-to-value association (the
% table keys are strings, the values are numbers).
%    \begin{macrocode}
local list, data, exclusive, simple, reverselist = 1, 2, 3, 4, 5
local types   = {
  list        = list,
  data        = data,
  exclusive   = exclusive,
  simple      = simple,
  reverselist = reverselist,
}
%    \end{macrocode}
%
% Now, list all predefined callbacks with their current type, based on the
% Lua\TeX{} manual version~1.01. A full list of the currently-available
% callbacks can be obtained using
%  \begin{verbatim}
%    \directlua{
%      for i,_ in pairs(callback.list()) do
%        texio.write_nl("- " .. i)
%      end
%    }
%    \bye
%  \end{verbatim}
% in plain Lua\TeX{}. (Some undocumented callbacks are omitted as they are
% to be removed.)
%    \begin{macrocode}
local callbacktypes = callbacktypes or {
%    \end{macrocode}
%   Section 8.2: file discovery callbacks.
% \changes{v1.1g}{2018/05/02}{find\_sfd\_file removed}
%    \begin{macrocode}
  find_read_file     = exclusive,
  find_write_file    = exclusive,
  find_font_file     = data,
  find_output_file   = data,
  find_format_file   = data,
  find_vf_file       = data,
  find_map_file      = data,
  find_enc_file      = data,
  find_pk_file       = data,
  find_data_file     = data,
  find_opentype_file = data,
  find_truetype_file = data,
  find_type1_file    = data,
  find_image_file    = data,
%    \end{macrocode}
% \changes{v1.1g}{2018/05/02}{read\_sfd\_file removed}
%    \begin{macrocode}
  open_read_file     = exclusive,
  read_font_file     = exclusive,
  read_vf_file       = exclusive,
  read_map_file      = exclusive,
  read_enc_file      = exclusive,
  read_pk_file       = exclusive,
  read_data_file     = exclusive,
  read_truetype_file = exclusive,
  read_type1_file    = exclusive,
  read_opentype_file = exclusive,
%    \end{macrocode}
% \changes{v1.0m}{2016/02/11}{read\_cidmap\_file added}
% Not currently used by luatex but included for completeness.
% may be used by a font handler.
%    \begin{macrocode}
  find_cidmap_file   = data,
  read_cidmap_file   = exclusive,
%    \end{macrocode}
% Section 8.3: data processing callbacks.
% \changes{v1.0m}{2016/02/11}{token\_filter removed}
%    \begin{macrocode}
  process_input_buffer  = data,
  process_output_buffer = data,
  process_jobname       = data,
%    \end{macrocode}
% Section 8.4: node list processing callbacks.
% \changes{v1.0m}{2016/02/11}
% {process\_rule, [hv]pack\_quality  append\_to\_vlist\_filter added}
% \changes{v1.0n}{2016/03/13}{insert\_local\_par added}
% \changes{v1.0n}{2016/03/13}{contribute\_filter added}
% \changes{v1.1h}{2018/08/18}{append\_to\_vlist\_filter is \texttt{exclusive}}
% \changes{v1.1j}{2019/06/18}{new\_graf added}
% \changes{v1.1k}{2019/10/02}{linebreak\_filter is \texttt{exclusive}}
% \changes{v1.1k}{2019/10/02}{process\_rule is \texttt{exclusive}}
% \changes{v1.1k}{2019/10/02}{mlist\_to\_hlist is \texttt{exclusive}}
% \changes{v1.1l}{2020/02/02}{post\_linebreak\_filter is \texttt{reverselist}}
% \changes{v1.1p}{2020/08/01}{new\_graf is \texttt{exclusive}}
% \changes{v1.1w}{2021/11/17}{hpack\_quality is \texttt{exclusive}}
% \changes{v1.1w}{2021/11/17}{vpack\_quality is \texttt{exclusive}}
%    \begin{macrocode}
  contribute_filter      = simple,
  buildpage_filter       = simple,
  build_page_insert      = exclusive,
  pre_linebreak_filter   = list,
  linebreak_filter       = exclusive,
  append_to_vlist_filter = exclusive,
  post_linebreak_filter  = reverselist,
  hpack_filter           = list,
  vpack_filter           = list,
  hpack_quality          = exclusive,
  vpack_quality          = exclusive,
  pre_output_filter      = list,
  process_rule           = exclusive,
  hyphenate              = simple,
  ligaturing             = simple,
  kerning                = simple,
  insert_local_par       = simple,
  pre_mlist_to_hlist_filter = list,
  mlist_to_hlist         = exclusive,
  post_mlist_to_hlist_filter = reverselist,
  new_graf               = exclusive,
%    \end{macrocode}
% Section 8.5: information reporting callbacks.
% \changes{v1.0m}{2016/02/11}{show\_warning\_message added}
% \changes{v1.0p}{2016/11/17}{call\_edit added}
% \changes{v1.1g}{2018/05/02}{finish\_synctex\_callback added}
% \changes{v1.1j}{2019/06/18}{finish\_synctex\_callback renamed finish\_synctex}
% \changes{v1.1j}{2019/06/18}{wrapup\_run added}
%    \begin{macrocode}
  pre_dump             = simple,
  start_run            = simple,
  stop_run             = simple,
  start_page_number    = simple,
  stop_page_number     = simple,
  show_error_hook      = simple,
  show_warning_message = simple,
  show_error_message   = simple,
  show_lua_error_hook  = simple,
  start_file           = simple,
  stop_file            = simple,
  call_edit            = simple,
  finish_synctex       = simple,
  wrapup_run           = simple,
%    \end{macrocode}
% Section 8.6: PDF-related callbacks.
% \changes{v1.1j}{2019/06/18}{page\_objnum\_provider added}
% \changes{v1.1j}{2019/06/18}{process\_pdf\_image\_content added}
% \changes{v1.1j}{2019/10/22}{page\_objnum\_provider and process\_pdf\_image\_content classified data}
% \changes{v1.1l}{2020/02/02}{page\_order\_index added}
%    \begin{macrocode}
  finish_pdffile            = data,
  finish_pdfpage            = data,
  page_objnum_provider      = data,
  page_order_index          = data,
  process_pdf_image_content = data,
%    \end{macrocode}
% Section 8.7: font-related callbacks.
% \changes{v1.1e}{2017/03/28}{glyph\_stream\_provider added}
% \changes{v1.1g}{2018/05/02}{glyph\_not\_found added}
% \changes{v1.1j}{2019/06/18}{make\_extensible added}
% \changes{v1.1j}{2019/06/18}{font\_descriptor\_objnum\_provider added}
% \changes{v1.1l}{2020/02/02}{glyph\_info added}
% \changes{v1.1t}{2021/04/18}{input\_level\_string added}
% \changes{v1.1v}{2021/10/15}{provide\_charproc\_data added}
%    \begin{macrocode}
  define_font                     = exclusive,
  glyph_info                      = exclusive,
  glyph_not_found                 = exclusive,
  glyph_stream_provider           = exclusive,
  make_extensible                 = exclusive,
  font_descriptor_objnum_provider = exclusive,
  input_level_string              = exclusive,
  provide_charproc_data           = exclusive,
%    \end{macrocode}
% \changes{v1.0m}{2016/02/11}{pdf\_stream\_filter\_callback removed}
%    \begin{macrocode}
}
luatexbase.callbacktypes=callbacktypes
%    \end{macrocode}
%
% \begin{macro}{callback.register}
% \changes{v1.0a}{2015/09/24}{Function modified}
%   Save the original function for registering callbacks and prevent the
%   original being used. The original is saved in a place that remains
%   available so other more sophisticated code can override the approach
%   taken by the kernel if desired.
%    \begin{macrocode}
local callback_register = callback_register or callback.register
function callback.register()
  luatexbase_error("Attempt to use callback.register() directly\n")
end
%    \end{macrocode}
% \end{macro}
%
% \subsubsection{Handlers}
%
% The handler function is registered into the callback when the
% first function is added to this callback's list. Then, when the callback
% is called, the handler takes care of running all functions in the list.
% When the last function is removed from the callback's list, the handler
% is unregistered.
%
% More precisely, the functions below are used to generate a specialized
% function (closure) for a given callback, which is the actual handler.
%
%
% The way the functions are combined together depends on
% the type of the callback. There are currently 4 types of callback, depending
% on the calling convention of the functions the callback can hold:
% \begin{description}
%   \item[simple] is for functions that don't return anything: they are called
%     in order, all with the same argument;
%   \item[data] is for functions receiving a piece of data of any type
%     except node list head (and possibly other arguments) and returning it
%     (possibly modified): the functions are called in order, and each is
%     passed the return value of the previous (and the other arguments
%     untouched, if any). The return value is that of the last function;
%   \item[list] is a specialized variant of \emph{data} for functions
%     filtering node lists. Such functions may return either the head of a
%     modified node list, or the boolean values |true| or |false|. The
%     functions are chained the same way as for \emph{data} except that for
%     the following. If
%     one function returns |false|, then |false| is immediately returned and
%     the following functions are \emph{not} called. If one function returns
%     |true|, then the same head is passed to the next function. If all
%     functions return |true|, then |true| is returned, otherwise the return
%     value of the last function not returning |true| is used.
%   \item[reverselist] is a specialized variant of \emph{list} which executes
%     functions in inverse order.
%   \item[exclusive] is for functions with more complex signatures; functions in
%     this type of callback are \emph{not} combined: An error is raised if
%     a second callback is registered.
% \end{description}
%
% Handler for |data| callbacks.
%    \begin{macrocode}
local function data_handler(name)
  return function(data, ...)
    for _,i in ipairs(callbacklist[name]) do
      data = i.func(data,...)
    end
    return data
  end
end
%    \end{macrocode}
% Default for user-defined |data| callbacks without explicit default.
%    \begin{macrocode}
local function data_handler_default(value)
  return value
end
%    \end{macrocode}
% Handler for |exclusive| callbacks. We can assume |callbacklist[name]| is not
% empty: otherwise, the function wouldn't be registered in the callback any
% more.
%    \begin{macrocode}
local function exclusive_handler(name)
  return function(...)
    return callbacklist[name][1].func(...)
  end
end
%    \end{macrocode}
% Handler for |list| callbacks.
% \changes{v1.0k}{2015/12/02}{resolve name and i.description (PHG)}
% \changes{v1.1s}{2020/12/02}{Fix return value of list callbacks}
% \changes{v1.1w}{2021/11/17}{Never pass on \texttt{true} return values for list callbacks}
%    \begin{macrocode}
local function list_handler(name)
  return function(head, ...)
    local ret
    for _,i in ipairs(callbacklist[name]) do
      ret = i.func(head, ...)
      if ret == false then
        luatexbase_warning(
          "Function `" .. i.description .. "' returned false\n"
            .. "in callback `" .. name .."'"
         )
        return false
      end
      if ret ~= true then
        head = ret
      end
    end
    return head
  end
end
%    \end{macrocode}
% Default for user-defined |list| and |reverselist| callbacks without explicit default.
%    \begin{macrocode}
local function list_handler_default(head)
return head
end
%    \end{macrocode}
% Handler for |reverselist| callbacks.
% \changes{v1.1l}{2020/02/02}{Add reverselist callback type}
%    \begin{macrocode}
local function reverselist_handler(name)
  return function(head, ...)
    local ret
    local callbacks = callbacklist[name]
    for i = #callbacks, 1, -1 do
      local cb = callbacks[i]
      ret = cb.func(head, ...)
      if ret == false then
        luatexbase_warning(
          "Function `" .. cb.description .. "' returned false\n"
            .. "in callback `" .. name .."'"
         )
        return false
      end
      if ret ~= true then
        head = ret
      end
    end
    return head
  end
end
%    \end{macrocode}
% Handler for |simple| callbacks.
%    \begin{macrocode}
local function simple_handler(name)
  return function(...)
    for _,i in ipairs(callbacklist[name]) do
      i.func(...)
    end
  end
end
%    \end{macrocode}
% Default for user-defined |simple| callbacks without explicit default.
%    \begin{macrocode}
local function simple_handler_default()
end
%    \end{macrocode}
%
% Keep a handlers table for indexed access and a table with the corresponding default functions.
%    \begin{macrocode}
local handlers  = {
  [data]        = data_handler,
  [exclusive]   = exclusive_handler,
  [list]        = list_handler,
  [reverselist] = reverselist_handler,
  [simple]      = simple_handler,
}
local defaults = {
  [data]        = data_handler_default,
  [exclusive]   = nil,
  [list]        = list_handler_default,
  [reverselist] = list_handler_default,
  [simple]      = simple_handler_default,
}
%    \end{macrocode}
%
% \subsubsection{Public functions for callback management}
%
% Defining user callbacks perhaps should be in package code,
% but impacts on |add_to_callback|.
% If a default function is not required, it may be declared as |false|.
% First we need a list of user callbacks.
%    \begin{macrocode}
local user_callbacks_defaults = {
  pre_mlist_to_hlist_filter = list_handler_default,
  mlist_to_hlist = node.mlist_to_hlist,
  post_mlist_to_hlist_filter = list_handler_default,
}
%    \end{macrocode}
%
% \begin{macro}{create_callback}
% \changes{v1.0a}{2015/09/24}{Function added}
% \changes{v1.0i}{2015/11/29}{Check name is not nil in error message (PHG)}
% \changes{v1.0k}{2015/12/02}{Give more specific error messages (PHG)}
% \changes{v1.1l}{2020/02/02}{Provide proper fallbacks for user-defined callbacks without user-provided default handler}
%   The allocator itself.
%    \begin{macrocode}
local function create_callback(name, ctype, default)
  local ctype_id = types[ctype]
  if not name  or name  == ""
  or not ctype_id
  then
    luatexbase_error("Unable to create callback:\n" ..
                     "valid callback name and type required")
  end
  if callbacktypes[name] then
    luatexbase_error("Unable to create callback `" .. name ..
                     "':\ncallback is already defined")
  end
  default = default or defaults[ctype_id]
  if not default then
    luatexbase_error("Unable to create callback `" .. name ..
                     "':\ndefault is required for `" .. ctype ..
                     "' callbacks")
  elseif type (default) ~= "function" then
    luatexbase_error("Unable to create callback `" .. name ..
                     "':\ndefault is not a function")
  end
  user_callbacks_defaults[name] = default
  callbacktypes[name] = ctype_id
end
luatexbase.create_callback = create_callback
%    \end{macrocode}
% \end{macro}
%
% \begin{macro}{call_callback}
% \changes{v1.0a}{2015/09/24}{Function added}
% \changes{v1.0i}{2015/11/29}{Check name is not nil in error message (PHG)}
% \changes{v1.0k}{2015/12/02}{Give more specific error messages (PHG)}
%  Call a user defined callback. First check arguments.
%    \begin{macrocode}
local function call_callback(name,...)
  if not name or name == "" then
    luatexbase_error("Unable to create callback:\n" ..
                     "valid callback name required")
  end
  if user_callbacks_defaults[name] == nil then
    luatexbase_error("Unable to call callback `" .. name
                     .. "':\nunknown or empty")
   end
  local l = callbacklist[name]
  local f
  if not l then
    f = user_callbacks_defaults[name]
  else
    f = handlers[callbacktypes[name]](name)
  end
  return f(...)
end
luatexbase.call_callback=call_callback
%    \end{macrocode}
% \end{macro}
%
% \begin{macro}{add_to_callback}
% \changes{v1.0a}{2015/09/24}{Function added}
%   Add a function to a callback. First check arguments.
% \changes{v1.0k}{2015/12/02}{Give more specific error messages (PHG)}
%    \begin{macrocode}
local function add_to_callback(name, func, description)
  if not name or name == "" then
    luatexbase_error("Unable to register callback:\n" ..
                     "valid callback name required")
  end
  if not callbacktypes[name] or
    type(func) ~= "function" or
    not description or
    description == "" then
    luatexbase_error(
      "Unable to register callback.\n\n"
        .. "Correct usage:\n"
        .. "add_to_callback(<callback>, <function>, <description>)"
    )
  end
%    \end{macrocode}
%   Then test if this callback is already in use. If not, initialise its list
%   and register the proper handler.
%    \begin{macrocode}
  local l = callbacklist[name]
  if l == nil then
    l = { }
    callbacklist[name] = l
%    \end{macrocode}
% If it is not a user defined callback use the primitive callback register.
%    \begin{macrocode}
    if user_callbacks_defaults[name] == nil then
      callback_register(name, handlers[callbacktypes[name]](name))
    end
  end
%    \end{macrocode}
%  Actually register the function and give an error if more than one
%  |exclusive| one is registered.
%    \begin{macrocode}
  local f = {
    func        = func,
    description = description,
  }
  local priority = #l + 1
  if callbacktypes[name] == exclusive then
    if #l == 1 then
      luatexbase_error(
        "Cannot add second callback to exclusive function\n`" ..
        name .. "'")
    end
  end
  table.insert(l, priority, f)
%    \end{macrocode}
%  Keep user informed.
%    \begin{macrocode}
  luatexbase_log(
    "Inserting `" .. description .. "' at position "
      .. priority .. " in `" .. name .. "'."
  )
end
luatexbase.add_to_callback = add_to_callback
%    \end{macrocode}
% \end{macro}
%
% \begin{macro}{remove_from_callback}
% \changes{v1.0a}{2015/09/24}{Function added}
% \changes{v1.0k}{2015/12/02}{adjust initialization of cb local (PHG)}
% \changes{v1.0k}{2015/12/02}{Give more specific error messages (PHG)}
% \changes{v1.1m}{2020/03/07}{Do not call callback.register for user-defined callbacks}
%   Remove a function from a callback. First check arguments.
%    \begin{macrocode}
local function remove_from_callback(name, description)
  if not name or name == "" then
    luatexbase_error("Unable to remove function from callback:\n" ..
                     "valid callback name required")
  end
  if not callbacktypes[name] or
    not description or
    description == "" then
    luatexbase_error(
      "Unable to remove function from callback.\n\n"
        .. "Correct usage:\n"
        .. "remove_from_callback(<callback>, <description>)"
    )
  end
  local l = callbacklist[name]
  if not l then
    luatexbase_error(
      "No callback list for `" .. name .. "'\n")
  end
%    \end{macrocode}
%  Loop over the callback's function list until we find a matching entry.
%  Remove it and check if the list is empty: if so, unregister the
%   callback handler.
%    \begin{macrocode}
  local index = false
  for i,j in ipairs(l) do
    if j.description == description then
      index = i
      break
    end
  end
  if not index then
    luatexbase_error(
      "No callback `" .. description .. "' registered for `" ..
      name .. "'\n")
  end
  local cb = l[index]
  table.remove(l, index)
  luatexbase_log(
    "Removing  `" .. description .. "' from `" .. name .. "'."
  )
  if #l == 0 then
    callbacklist[name] = nil
    if user_callbacks_defaults[name] == nil then
      callback_register(name, nil)
    end
  end
  return cb.func,cb.description
end
luatexbase.remove_from_callback = remove_from_callback
%    \end{macrocode}
% \end{macro}
%
% \begin{macro}{in_callback}
% \changes{v1.0a}{2015/09/24}{Function added}
% \changes{v1.0h}{2015/11/27}{Guard against undefined list latex/4445}
%   Look for a function description in a callback.
%    \begin{macrocode}
local function in_callback(name, description)
  if not name
    or name == ""
    or not callbacklist[name]
    or not callbacktypes[name]
    or not description then
      return false
  end
  for _, i in pairs(callbacklist[name]) do
    if i.description == description then
      return true
    end
  end
  return false
end
luatexbase.in_callback = in_callback
%    \end{macrocode}
% \end{macro}
%
% \begin{macro}{disable_callback}
% \changes{v1.0a}{2015/09/24}{Function added}
%   As we subvert the engine interface we need to provide a way to access
%   this functionality.
%    \begin{macrocode}
local function disable_callback(name)
  if(callbacklist[name] == nil) then
    callback_register(name, false)
  else
    luatexbase_error("Callback list for " .. name .. " not empty")
  end
end
luatexbase.disable_callback = disable_callback
%    \end{macrocode}
% \end{macro}
%
% \begin{macro}{callback_descriptions}
% \changes{v1.0a}{2015/09/24}{Function added}
% \changes{v1.0h}{2015/11/27}{Match test in in-callback latex/4445}
%   List the descriptions of functions registered for the given callback.
%    \begin{macrocode}
local function callback_descriptions (name)
  local d = {}
  if not name
    or name == ""
    or not callbacklist[name]
    or not callbacktypes[name]
    then
    return d
  else
  for k, i in pairs(callbacklist[name]) do
    d[k]= i.description
    end
  end
  return d
end
luatexbase.callback_descriptions =callback_descriptions
%    \end{macrocode}
% \end{macro}
%
% \begin{macro}{uninstall}
% \changes{v1.0e}{2015/10/02}{Function added}
%   Unlike at the \TeX{} level, we have to provide a back-out mechanism here
%   at the same time as the rest of the code. This is not meant for use by
%   anything other than \textsf{latexrelease}: as such this is
%   \emph{deliberately} not documented for users!
%    \begin{macrocode}
local function uninstall()
  module_info(
    "luatexbase",
    "Uninstalling kernel luatexbase code"
  )
  callback.register = callback_register
  luatexbase = nil
end
luatexbase.uninstall = uninstall
%    \end{macrocode}
% \end{macro}
% \begin{macro}{mlist_to_hlist}
% \changes{v1.1l}{2020/02/02}{|pre/post_mlist_to_hlist| added}
%   To emulate these callbacks, the ``real'' |mlist_to_hlist| is replaced by a
%   wrapper calling the wrappers before and after.
%    \begin{macrocode}
callback_register("mlist_to_hlist", function(head, display_type, need_penalties)
  local current = call_callback("pre_mlist_to_hlist_filter", head, display_type, need_penalties)
  if current == false then
    flush_list(head)
    return nil
  end
  current = call_callback("mlist_to_hlist", current, display_type, need_penalties)
  local post = call_callback("post_mlist_to_hlist_filter", current, display_type, need_penalties)
  if post == false then
    flush_list(current)
    return nil
  end
  return post
end)
%    \end{macrocode}
% \end{macro}
% \endgroup
%
%    \begin{macrocode}
%</lua>
%    \end{macrocode}
%
% Reset the catcode of |@|.
%    \begin{macrocode}
%<tex>\catcode`\@=\etatcatcode\relax
%    \end{macrocode}
%
%
% \Finale
|. This inputs |ltluatex.tex| which inputs
% |etex.src| (or |etex.sty| if used with \LaTeX)
% if it is not already input, and then defines some internal commands to
% allow the \textsf{ltluatex} interface to be defined.
%
% The \textsf{luatexbase} package interface may also be used in plain \TeX,
% as before, by inputting the package |\input luatexbase.sty|. The new
% version of \textsf{luatexbase} is based on this \textsf{ltluatex}
% code but implements a compatibility layer providing the interface
% of the original package.
%
% \section{Lua functionality}
%
% \begingroup
%
% \begingroup\lccode`~=`_
% \lowercase{\endgroup\let~}_
% \catcode`_=12
%
% \subsection{Allocators in Lua}
%
% \DescribeMacro{new_attribute}
% |luatexbase.new_attribute(|\meta{attribute}|)|\\
% Returns an allocation number for the \meta{attribute}, indexed from~$1$.
% The attribute will be initialised with the marker value |-"7FFFFFFF|
% (`unset'). The attribute allocation sequence is shared with the \TeX{}
% code but this function does \emph{not} define a token using
% |\attributedef|.
% The attribute name is recorded in the |attributes| table. A
% metatable is provided so that the table syntax can be used
% consistently for attributes declared in \TeX\ or Lua.
%
% \noindent
% \DescribeMacro{new_whatsit}
% |luatexbase.new_whatsit(|\meta{whatsit}|)|\\
% Returns an allocation number for the custom \meta{whatsit}, indexed from~$1$.
%
% \noindent
% \DescribeMacro{new_bytecode}
% |luatexbase.new_bytecode(|\meta{bytecode}|)|\\
% Returns an allocation number for a bytecode register, indexed from~$1$.
% The optional \meta{name} argument is just used for logging.
%
% \noindent
% \DescribeMacro{new_chunkname}
% |luatexbase.new_chunkname(|\meta{chunkname}|)|\\
% Returns an allocation number for a Lua chunk name for use with
% |\directlua| and |\latelua|, indexed from~$1$.
% The number is returned and also \meta{name} argument is added to the
% |lua.name| array at that index.
%
% \begin{sloppypar}
% \noindent
% \DescribeMacro{new_luafunction}
% |luatexbase.new_luafunction(|\meta{functionname}|)|\\
% Returns an allocation number for a lua function for use
% with |\luafunction|, |\lateluafunction|, and |\luadef|,
% indexed from~$1$. The optional \meta{functionname} argument
% is just used for logging.
% \end{sloppypar}
%
% These functions all require access to a named \TeX{} count register
% to manage their allocations. The standard names are those defined
% above for access from \TeX{}, \emph{e.g.}~\string\e@alloc@attribute@count,
% but these can be adjusted by defining the variable
% \texttt{\meta{type}\_count\_name} before loading |ltluatex.lua|, for example
% \begin{verbatim}
% local attribute_count_name = "attributetracker"
% require("ltluatex")
% \end{verbatim}
% would use a \TeX{} |\count| (|\countdef|'d token) called |attributetracker|
% in place of \string\e@alloc@attribute@count.
%
% \subsection{Lua access to \TeX{} register numbers}
%
% \DescribeMacro{registernumber}
% |luatexbase.registernumer(|\meta{name}|)|\\
% Sometimes (notably in the case of Lua attributes) it is necessary to
% access a register \emph{by number} that has been allocated by \TeX{}.
% This package provides a function to look up the relevant number
% using Lua\TeX{}'s internal tables. After for example
% |\newattribute\myattrib|, |\myattrib| would be defined by (say)
% |\myattrib=\attribute15|.  |luatexbase.registernumer("myattrib")|
% would then return the register number, $15$ in this case. If the string passed
% as argument does not correspond to a token defined by |\attributedef|,
% |\countdef| or similar commands, the Lua value |false| is returned.
%
% As an example, consider the input:
%\begin{verbatim}
% \newcommand\test[1]{%
% \typeout{#1: \expandafter\meaning\csname#1\endcsname^^J
% \space\space\space\space
% \directlua{tex.write(luatexbase.registernumber("#1") or "bad input")}%
% }}
%
% \test{undefinedrubbish}
%
% \test{space}
%
% \test{hbox}
%
% \test{@MM}
%
% \test{@tempdima}
% \test{@tempdimb}
%
% \test{strutbox}
%
% \test{sixt@@n}
%
% \attrbutedef\myattr=12
% \myattr=200
% \test{myattr}
%
%\end{verbatim}
%
% If the demonstration code is processed with Lua\LaTeX{} then the following
% would be produced in the log and terminal output.
%\begin{verbatim}
% undefinedrubbish: \relax
%      bad input
% space: macro:->
%      bad input
% hbox: \hbox
%      bad input
% @MM: \mathchar"4E20
%      20000
% @tempdima: \dimen14
%      14
% @tempdimb: \dimen15
%      15
% strutbox: \char"B
%      11
% sixt@@n: \char"10
%      16
% myattr: \attribute12
%      12
%\end{verbatim}
%
% Notice how undefined commands, or commands unrelated to registers
% do not produce an error, just return |false| and so print
% |bad input| here. Note also that commands defined by |\newbox| work and
% return the number of the box register even though the actual command
% holding this number is a |\chardef| defined token (there is no
% |\boxdef|).
%
% \subsection{Module utilities}
%
% \DescribeMacro{provides_module}
% |luatexbase.provides_module(|\meta{info}|)|\\
% This function is used by modules to identify themselves; the |info| should be
% a table containing information about the module. The required field
% |name| must contain the name of the module. It is recommended to provide a
% field |date| in the usual \LaTeX{} format |yyyy/mm/dd|. Optional fields
% |version| (a string) and |description| may be used if present. This
% information will be recorded in the log. Other fields are ignored.
%
% \noindent
% \DescribeMacro{module_info}
% \DescribeMacro{module_warning}
% \DescribeMacro{module_error}
% |luatexbase.module_info(|\meta{module}, \meta{text}|)|\\
% |luatexbase.module_warning(|\meta{module}, \meta{text}|)|\\
% |luatexbase.module_error(|\meta{module}, \meta{text}|)|\\
% These functions are similar to \LaTeX{}'s |\PackageError|, |\PackageWarning|
% and |\PackageInfo| in the way they format the output.  No automatic line
% breaking is done, you may still use |\n| as usual for that, and the name of
% the package will be prepended to each output line.
%
% Note that |luatexbase.module_error| raises an actual Lua error with |error()|,
% which currently means a call stack will be dumped. While this may not
% look pretty, at least it provides useful information for tracking the
% error down.
%
% \subsection{Callback management}
%
% \noindent
% \DescribeMacro{add_to_callback}
% |luatexbase.add_to_callback(|^^A
% \meta{callback}, \meta{function}, \meta{description}|)|
% Registers the \meta{function} into the \meta{callback} with a textual
% \meta{description} of the function. Functions are inserted into the callback
% in the order loaded.
%
% \noindent
% \DescribeMacro{remove_from_callback}
% |luatexbase.remove_from_callback(|\meta{callback}, \meta{description}|)|
% Removes the callback function with \meta{description} from the \meta{callback}.
% The removed function and its description
% are returned as the results of this function.
%
% \noindent
% \DescribeMacro{in_callback}
% |luatexbase.in_callback(|\meta{callback}, \meta{description}|)|
% Checks if the \meta{description} matches one of the functions added
% to the list for the \meta{callback}, returning a boolean value.
%
% \noindent
% \DescribeMacro{disable_callback}
% |luatexbase.disable_callback(|\meta{callback}|)|
% Sets the \meta{callback} to \texttt{false} as described in the Lua\TeX{}
% manual for the underlying \texttt{callback.register} built-in. Callbacks
% will only be set to false (and thus be skipped entirely) if there are
% no functions registered using the callback.
%
% \noindent
% \DescribeMacro{callback_descriptions}
% A list of the descriptions of functions registered to the specified
% callback is returned. |{}| is returned if there are no functions registered.
%
% \noindent
% \DescribeMacro{create_callback}
% |luatexbase.create_callback(|\meta{name},meta{type},\meta{default}|)|
% Defines a user defined callback. The last argument is a default
% function or |false|.
%
% \noindent
% \DescribeMacro{call_callback}
% |luatexbase.call_callback(|\meta{name},\ldots|)|
% Calls a user defined callback with the supplied arguments.
%
% \endgroup
%
% \StopEventually{}
%
% \section{Implementation}
%
%    \begin{macrocode}
%<*2ekernel|tex|latexrelease>
%<2ekernel|latexrelease>\ifx\directlua\@undefined\else
%    \end{macrocode}
%
%
% \changes{v1.0j}{2015/12/02}{Remove nonlocal iteration variables (PHG)}
% \changes{v1.0j}{2015/12/02}{Assorted typos fixed (PHG)}
% \changes{v1.0j}{2015/12/02}{Remove unreachable code after calls to error() (PHG)}
% \subsection{Minimum Lua\TeX{} version}
%
% Lua\TeX{} has changed a lot over time. In the kernel support for ancient
% versions is not provided: trying to build a format with a very old binary
% therefore gives some information in the log and loading stops. The cut-off
% selected here relates to the tree-searching behaviour of |require()|:
% from version~0.60, Lua\TeX{} will correctly find Lua files in the |texmf|
% tree without `help'.
%    \begin{macrocode}
%<latexrelease>\IncludeInRelease{2015/10/01}
%<latexrelease>                 {\newluafunction}{LuaTeX}%
\ifnum\luatexversion<60 %
  \wlog{***************************************************}
  \wlog{* LuaTeX version too old for ltluatex support *}
  \wlog{***************************************************}
  \expandafter\endinput
\fi
%    \end{macrocode}
%
% \changes{v1.1n}{2020/06/10}{Define \cs{@gobble}/\cs{@firstofone} even for \LaTeX\ to allow early loading.}
% Two simple \LaTeX\ macros from |ltdefns.dtx| have to be defined here
% because ltdefns.dtx is not loaded yet when ltluatex.dtx is executed.
%    \begin{macrocode}
\long\def\@gobble#1{}
\long\def\@firstofone#1{#1}
%    \end{macrocode}
%
% \subsection{Older \LaTeX{}/Plain \TeX\ setup}
%
%    \begin{macrocode}
%<*tex>
%    \end{macrocode}
%
% Older \LaTeX{} formats don't have the primitives with `native' names:
% sort that out. If they already exist this will still be safe.
%    \begin{macrocode}
\directlua{tex.enableprimitives("",tex.extraprimitives("luatex"))}
%    \end{macrocode}
%
%    \begin{macrocode}
\ifx\e@alloc\@undefined
%    \end{macrocode}
%
% In pre-2014 \LaTeX{}, or plain \TeX{}, load |etex.{sty,src}|.
%    \begin{macrocode}
  \ifx\documentclass\@undefined
    \ifx\loccount\@undefined
      % \iffalse meta-comment
%
% Copyright 1997, 1998, 2008 2015 2016 LaTeX Project and Peter Breitenlohner.
% 
% This file (etex.sty) may be distributed and/or modified under the
% conditions of the LaTeX Project Public License, either version 1.3 of
% this license or (at your option) any later version.  The latest
% version of this license is in
%   http://www.latex-project.org/lppl.txt
% and version 1.3 or later is part of all distributions of LaTeX
% version 2003/12/01 or later.
% 
% This work has the LPPL maintenance status "maintained".
% 
% The Current Maintainer of this work is David Carlisle.
% https://github.com/davidcarlisle/dpctex/issues
% \fi

\NeedsTeXFormat{LaTeX2e}
\ProvidesPackage{etex}
%        [1997/08/12 v0.1 eTeX basic definition package (DPC)]
%        [1998/03/26 v2.0 eTeX basic definition package (PEB)]
%        [2015/03/02 v2.1 eTeX basic definition package (PEB,DPC)]
%        [2015/07/06 v2.2 eTeX basic definition package (PEB,DPC)]
%        [2015/07/08 v2.3 eTeX basic definition package (PEB,DPC)]
%        [2015/09/02 v2.4 eTeX basic definition package (PEB,DPC)]
%        [2016/01/07 v2.5 eTeX basic definition package (PEB,DPC)]
%        [2016/01/11 v2.6 eTeX basic definition package (PEB,DPC)]
         [2016/08/01 v2.7 eTeX basic definition package (PEB,DPC)]

%%%%%%%%%%%%%%%%%%%%%%%%%%%%%%%%%%%%%%%%%%%%%%%%%%%%%%%%%%%%%%%%%%%%%%%%

%% A basic interface to some etex primitives, closely modeled on
%% etex.src and etexdefs.lib provided by the core etex team.

%% The etex.src `module' system is not copied here, the standard
%% LaTeX package option mechanism is used instead,
%% however the package options match the module names.
%% (Currently grouptypes, interactionmodes, nodetypes, iftypes.)
%% The individual type names are different too: We use, e.g.,
%%
%% `\bottomleveltype' and `\simplegrouptype' instead of
%% `\grouptypes{bottomlevel}' and `\grouptypes{simple}'.

%%%%%%%%%%%%%%%%%%%%%%%%%%%%%%%%%%%%%%%%%%%%%%%%%%%%%%%%%%%%%%%%%%%%%%%%

%% Other Comments...

%% The names of the `interactionmodes' are not too good.
%% In particular \scroll and \batch are likely to clash with existing
%% uses. These names have been changed into \batchinteractionmode,
%% \scrollinteractionmode etc.
%% Similarly, the names of the `groupetypes' have been changed, in
%% particular \mathgroup would conflict with the LaTeX kernel.

%% \etex logo could have the same trick as \LaTeXe to pick up a bold
%% epsilon when needed. (Not done here, I hate wasting tokens on logos.)
%% This version does have a \m@th not in the original.

%% The \globcountvector, \loccountvector, etc. allocation macros are
%% not (yet) implemented.

%% Currently if run on a standard TeX, the package generates an error.
%% Perhaps it should instead load some code to try to fake
%% the new etex primitives in that case???
%% Likewise, the package generates an error when used with e-TeX V 1

%% The etex.src language mechanism is not copied here. That facility
%% does not use any of the etex features. LaTeX should be customised
%% using the same hyphen.cfg mechanism as for a format built with a
%% standard TeX.

%% David Carlisle

%% Upgraded for e-TeX V 2.0
%% Peter Breitenlohner

%%%%%%%%%%%%%%%%%%%%%%%%%%%%%%%%%%%%%%%%%%%%%%%%%%%%%%%%%%%%%%%%%%%%%%%%



\ifx\eTeXversion\@undefined
  \PackageError{etex}
    {This package may only be run using an\MessageBreak
     etex in extended mode}
    {Perhaps you forgot the `*' when making the format with (e)initex.%
    }
\fi

\ifnum\eTeXversion<2
  \PackageError{etex}
    {This package requires e-TeX V 2}
    {You are probably using the obsolete e-TeX V 1.%
    }
\fi

% 2.2
% Check if the new latex 2015/01/01 allocation is already using
% extended reisters. If so it is too late to change allocation scheme.
% Older versions of LaTeX would have given an error when the classic
% TeX registers were all allocated, but newer formats allocate from
% the extended range, so usually this package is not needed.
\@tempswafalse
\ifnum\count10>\@cclv\@tempswatrue\else
\ifnum\count11>\@cclv\@tempswatrue\else
\ifnum\count12>\@cclv\@tempswatrue\else
\ifnum\count13>\@cclv\@tempswatrue\else
\ifnum\count14>\@cclv\@tempswatrue\else
\ifnum\count15>\@cclv\@tempswatrue
\fi\fi\fi\fi\fi\fi

\if@tempswa
\PackageWarningNoLine{etex}{%
Extended allocation already in use.\MessageBreak
etex.sty code will not be used.\MessageBreak
To force etex package to load, add\MessageBreak
\string\RequirePackage{etex}\MessageBreak
at the start of the document}

% 2.5 define the global allocation to be the standard ones
% as extended allocation is already in use. Helps with
% compatibility with some packages that use these commands 
% after loading etex.
% 2.6 avoid error from outer if used with (e)plain
\expandafter\let\csname globcount\expandafter\endcsname
                \csname newcount\endcsname
\expandafter\let\csname globdimen\expandafter\endcsname
                \csname newdimen\endcsname
\expandafter\let\csname globskip\expandafter\endcsname
                \csname newskip\endcsname
\expandafter\let\csname globmuskip\expandafter\endcsname
                \csname newmuskip\endcsname
\expandafter\let\csname globtoks\expandafter\endcsname
                \csname newtoks\endcsname
\expandafter\let\csname globmarks\expandafter\endcsname
                \csname newmarks\endcsname
% end of 2.5/2.6 change

\expandafter\endinput\fi

% End of 2.2 addition.

% 2.3 move option handling after the above error checks.
\DeclareOption{grouptypes}{\catcode`\G=9}
\DeclareOption{interactionmodes}{\catcode`\I=9}
\DeclareOption{nodetypes}{\catcode`\N=9}
\DeclareOption{iftypes}{\catcode`\C=9}
\DeclareOption{localalloclog}{\let\et@xwlog\wlog} % the default
\DeclareOption{localallocnolog}{\let\et@xwlog\@gobble} % be quiet
\DeclareOption{localallocshow}{\let\et@xwlog\typeout} % debugging
% End of 2.3 addition.

% v2.7
% \extrafloats does not work with this package
% but make it give a sensible error, not mis-parse \ifnum.
%
% Note that using \extrafloats earlier might not be safe as
% it could in principle clash with registers used for local allocation.
% However it probably works (as local allocation is used locally...).
% A better fix would be not to load this package with current LaTeX.
% This current etex package is just provided to force the old behaviour
% and such documents should not be using new features such as \extrafloats).
\ifdefined\extrafloats
\def\extrafloats#1{%
  \PackageError{etex}{%
    \noexpand\extrafloats is incompatible with etex.sty allocation.\MessageBreak
    Try using \noexpand\extrafloats before loading etex}%
    \@ehc}
\fi
% end of v2.7 change

\def\eTeX{%
  $\m@th\varepsilon$-\TeX}

\def\tracingall{%
  \tracingcommands\thr@@        % etex
  \tracingstats\tw@
  \tracingpages\@ne
  \tracinglostchars\tw@         % etex
  \tracingmacros\tw@
  \tracingparagraphs\@ne
  \tracingrestores\@ne
  \tracinggroups\@ne            % etex
  \tracingifs\@ne               % etex
  \tracingscantokens\@ne        % etex
  \tracingnesting\@ne           % etex
  \tracingassigns\@ne           % etex
  \errorcontextlines\maxdimen
  \showoutput}

\def\loggingall{%
  \tracingall
  \tracingonline\z@}

\def\tracingnone{%
  \tracingonline\z@
  \showboxdepth\m@ne
  \showboxbreadth\m@ne
  \tracingoutput\z@
  \errorcontextlines\m@ne
  \tracingassigns\z@
  \tracingnesting\z@
  \tracingscantokens\z@
  \tracingifs\z@
  \tracinggroups\z@
  \tracingrestores\z@
  \tracingparagraphs\z@
  \tracingmacros\z@
  \tracinglostchars\@ne
  \tracingpages\z@
  \tracingstats\z@
  \tracingcommands\z@}

%% Register allocation
%% We have to adjust the Plain TeX / LaTeX register allocation counts
%% for our slightly modified book-keeping, but first we allocate our
%% insertion counter \et@xins, because \insc@ount of Plain TeX / LaTeX
%% will be used differently.

\newcount\et@xins

\advance\count10 by 1 % \count10=23 % allocates \count registers 23, 24, ...
\advance\count11 by 1 % \count11=10 % allocates \dimen registers 10, 11, ...
\advance\count12 by 1 % \count12=10 % allocates \skip registers 10, 11, ...
\advance\count13 by 1 % \count13=10 % allocates \muskip registers 10, 11, ...
\advance\count14 by 1 % \count14=10 % allocates \box registers 10, 11, ...
\advance\count15 by 1 % \count15=10 % allocates \toks registers 10, 11, ...
\advance\count16 by 1 % \count16=0 % allocates input streams 0, 1, ...
\advance\count17 by 1 % \count17=0 % allocates output streams 0, 1, ...
\advance\count18 by 1 % \count18=4 % allocates math families 4, 5, ...
\advance\count19 by 1 % \count19=0 % allocates \language codes 0, 1, ...

\et@xins=\insc@unt % \et@xins=255 % allocates insertions 254, 253, ...


%% To ensure working in LaTeX 2015 release do define \newcount etc
%% with their pre 2015 LaTeX definitions
\def\newcount{\alloc@0\count\countdef\insc@unt}
\def\newdimen{\alloc@1\dimen\dimendef\insc@unt}
\def\newskip{\alloc@2\skip\skipdef\insc@unt}
\def\newmuskip{\alloc@3\muskip\muskipdef\@cclvi}
\def\newbox{\alloc@4\box\chardef\insc@unt}
\def\newtoks{\alloc@5\toks\toksdef\@cclvi}
\def\newread{\alloc@6\read\chardef\sixt@@n}
\def\newwrite{\alloc@7\write\chardef\sixt@@n}
\def\new@mathgroup{\alloc@8\mathgroup\chardef\sixt@@n}
\let\newfam\new@mathgroup
\def\newlanguage{\alloc@9\language\chardef\@cclvi}

%% When the normal register pool for \count, \dimen, \skip, \muskip,
%% \box, or \toks registers is exhausted, we switch to the extended pool.

\def\alloc@#1#2#3#4#5%
 {\ifnum\count1#1<#4% make sure there's still room
    \allocationnumber\count1#1
    \global\advance\count1#1\@ne
    \global#3#5\allocationnumber
    \wlog{\string#5=\string#2\the\allocationnumber}%
  \else\ifnum#1<6
    \begingroup \escapechar\m@ne
    \expandafter\alloc@@\expandafter{\string#2}#5%
  \else\errmessage{No room for a new #2}\fi\fi
 }

%% The \expandafter construction used here allows the generation of
%% \newcount and \globcount from #1=count.

\def\alloc@@#1#2%
 {\endgroup % restore \escapechar
  \wlog{Normal \csname#1\endcsname register pool exhausted,
    switching to extended pool.}%
  \global\expandafter\let
    \csname new#1\expandafter\endcsname
    \csname glob#1\endcsname
  \csname new#1\endcsname#2%
 }

%% We do change the LaTeX definition of \newinsert

\def\newinsert#1{% make sure there's still room for ...
  \ch@ck0\et@xins\count{% ... a \count, ...
    \ch@ck1\et@xins\dimen{% ... \dimen, ...
      \ch@ck2\et@xins\skip{% ... \skip, ...
        \ch@ck4\et@xins\box{% ... and \box register
  \global\advance\et@xins\m@ne
  \unless\ifnum\insc@unt<\et@xins \global\insc@unt\et@xins \fi
  \allocationnumber\et@xins
  \global\chardef#1\allocationnumber
  \wlog{\string#1=\string\insert\the\allocationnumber}}}}}}

\def\ch@ck#1#2#3#4%
 {\ifnum\count1#1<#2#4\else\errmessage{No room for a new #3}\fi}

%% And we define \reserveinserts, so that you can say \reserveinserts{17}
%% in order to reserve room for up to 17 additional insertion classes, that
%% will not be taken away by \newcount, \newdimen, \newskip, or \newbox.

% 2.4 Remove \outer to match LaTeX allocations
% which are never \outer unlike plain TeX.

%\outer
\def\reserveinserts#1%
 {\global\insc@unt\numexpr \et@xins \ifnum#1>\z@ -#1\fi \relax}

% Now, we define \globcount, \globbox, etc., so that you can say
% \globcount\foo and \foo will be defined (with \countdef) to be the
% next count register from the vastly larger but somewhat less efficient
% extended register pool. We also define \loccount, etc., but these
% register definitions are local to the current group.

\count260=277 % globally allocates \count registers 277, 278, ...
\count261=256 % globally allocates \dimen registers 256, 257, ...
\count262=256 % globally allocates \skip registers 256, 257, ...
\count263=256 % globally allocates \muskip registers 256, 257, ...
\count264=256 % globally allocates \box registers 256, 257, ...
\count265=256 % globally allocates \toks registers 256, 257, ...
\count266=1 % globally allocates \marks classes 1, 2, ...

\count270=32768 % locally allocates \count registers 32767, 32766, ...
\count271=32768 % ditto for \dimen registers
\count272=32768 % ditto for \skip registers
\count273=32768 % ditto for \muskip registers
\count274=32768 % ditto for \box registers
\count275=32768 % ditto for \toks registers
\count276=32768 % ditto for \marks classes

% \count registers 256-259 and 267-269 are not (yet) used

% \def \et@xglob #1#2#3#4% <offset>, <type>, <method>, <register>
% \def \et@xloc #1#2#3#4% <offset>, <type>, <method>, <register>

\def \globcount  {\et@xglob 0\count  \countdef}
\def \loccount   {\et@xloc  0\count  \countdef}
\def \globdimen  {\et@xglob 1\dimen  \dimendef}
\def \locdimen   {\et@xloc  1\dimen  \dimendef}
\def \globskip   {\et@xglob 2\skip   \skipdef}
\def \locskip    {\et@xloc  2\skip   \skipdef}
\def \globmuskip {\et@xglob 3\muskip \muskipdef}
\def \locmuskip  {\et@xloc  3\muskip \muskipdef}
\def \globbox    {\et@xglob 4\box    \mathchardef}
\def \locbox     {\et@xloc  4\box    \mathchardef}
\def \globtoks   {\et@xglob 5\toks   \toksdef}
\def \loctoks    {\et@xloc  5\toks   \toksdef}
\def \globmarks  {\et@xglob 6\marks  \mathchardef}
\def \locmarks   {\et@xloc  6\marks  \mathchardef}

\let\newmarks=\globmarks %% this used to be \newmark for e-TeX V 1.1

\def\et@xglob#1#2#3#4%
 {\et@xchk#1#2{% make sure there's still room
  \allocationnumber=\count26#1%
  \global\advance\count26#1\@ne
  \global#3#4\allocationnumber
  \wlog{\string#4=\string#2\the\allocationnumber}}%
 }

\def\et@xloc#1#2#3#4%
 {\et@xchk#1#2{% make sure there's still room
  \advance\count27#1by\m@ne
  \allocationnumber=\count27#1%
  #3#4=\allocationnumber
  \et@xwlog{\string#4=\string#2\the\allocationnumber\space(local)}}%
 }

%% The allocation messages for local allocations use \et@xwlog, such that
%% these messages can easily be switched on/off

\let\et@xwlog=\wlog

\def\et@xchk#1#2#3%
 {\ifnum\count26#1<\count27#1 #3\else\errmessage{No room for a new #2}\fi}

% Next we define \globcountblk, \loccountblk, etc., so that you can
% say \globcountblk\foo{17} and \foo will be defined (with \mathchardef)
% as the first (the zeroth?) of a block of 17 consecutive registers.
% Thus the user is intended to reference elements <\foo+0> to <\foo+n-1>,
% where n is the length of the block allocated.

% \def \et@xgblk #1#2#3#4% <offset>, <type>, <register>, <size>
% \def \et@xlblk #1#2#3#4% <offset>, <type>, <register>, <size>

\def\globcountblk  {\et@xgblk 0\count  }
\def\loccountblk   {\et@xlblk 0\count  }
\def\globdimenblk  {\et@xgblk 1\dimen  }
\def\locdimenblk   {\et@xlblk 1\dimen  }
\def\globskipblk   {\et@xgblk 2\skip   }
\def\locskipblk    {\et@xlblk 2\skip   }
\def\globmuskipblk {\et@xgblk 3\muskip }
\def\locmuskipblk  {\et@xlblk 3\muskip }
\def\globboxblk    {\et@xgblk 4\box    }
\def\locboxblk     {\et@xlblk 4\box    }
\def\globtoksblk   {\et@xgblk 5\toks   }
\def\loctoksblk    {\et@xlblk 5\toks   }
\def\globmarksblk  {\et@xgblk 6\marks  }
\def\locmarksblk   {\et@xlblk 6\marks  }

% \def\et@xchkblk#1#1#3#4% <offset>, <type>, <size>, <action>

\def\et@xgblk#1#2#3#4%
 {\et@xchkblk#1#2{#4}% make sure there's still room
   {\allocationnumber\count26#1%
    \global\advance\count26#1by#4%
    \global\mathchardef#3\allocationnumber
    \wlog{\string#3=\string#2blk{\number#4} at
      \the\allocationnumber}%
   }%
 }

\def\et@xlblk#1#2#3#4%
 {\et@xchkblk#1#2{#4}% make sure there's still room
   {\advance\count27#1-#4%
    \allocationnumber\count27#1%
    \mathchardef#3\allocationnumber
    \et@xwlog{\string#3=\string#2blk{\number#4} at
      \the\allocationnumber\space(local)}%
   }%
 }

\def\et@xchkblk#1#2#3#4%
 {\ifnum#3<\z@
    \errmessage{Negative register block size \number#3}%
  \else\ifnum\numexpr\count26#1+#3>\count27#1%
    \errmessage{No room for new #2block of size \number#3}%
  \else #4\fi \fi
 }

\catcode`\G=14
\catcode`\I=14
\catcode`\N=14
\catcode`\C=14

\ProcessOptions

%% Declare names for `grouptypes'

G \chardef \bottomleveltype       =  0 % for the outside world
G \chardef \simplegrouptype       =  1 % for local structure only
G \chardef \hboxgrouptype         =  2 % for `\hbox{}'
G \chardef \adjustedhboxgrouptype =  3 % for `\hbox{}' in vertical mode
G \chardef \vboxgrouptype         =  4 % for `\vbox{}'
G \chardef \vtopgrouptype         =  5 % for `\vtop{}'
G \chardef \aligngrouptype        =  6 % for `\halign{}', `\valign{}'
G \chardef \noaligngrouptype      =  7 % for `\noalign{}'
G \chardef \outputgrouptype       =  8 % for output routine
G \chardef \mathgrouptype         =  9 % for, e.g, `^{}'
G \chardef \discgrouptype         = 10 % for `\discretionary{}{}{}'
G \chardef \insertgrouptype       = 11 % for `\insert{}', `\vadjust{}'
G \chardef \vcentergrouptype      = 12 % for `\vcenter{}'
G \chardef \mathchoicegrouptype   = 13 % for `\mathchoice{}{}{}{}'
G \chardef \semisimplegrouptype   = 14 % for `\begingroup...\endgroup'
G \chardef \mathshiftgrouptype    = 15 % for `$...$'
G \chardef \mathleftgrouptype     = 16 % for `\left...\right'

%% Declare names for `interactionmodes'

I \chardef \batchinteractionmode     = 0 % omits all stops and omits terminal output
I \chardef \nonstopinteractionmode   = 1 % omits all stops
I \chardef \scrollinteractionmode    = 2 % omits error stops
I \chardef \errorstopinteractionmode = 3 % stops at every opportunity to interact

%% Declare names for `nodetypes'

N \chardef \charnode     =  0 % character nodes
N \chardef \hlistnode    =  1 % hlist nodes
N \chardef \vlistnode    =  2 % vlist nodes
N \chardef \rulenode     =  3 % rule nodes
N \chardef \insnode      =  4 % insertion nodes
N \chardef \marknode     =  5 % a mark node
N \chardef \adjustnode   =  6 % an adjust node
N \chardef \ligaturenode =  7 % a ligature node
N \chardef \discnode     =  8 % a discretionary node
N \chardef \whatsitnode  =  9 % special extension nodes
N \chardef \mathnode     = 10 % a math node
N \chardef \gluenode     = 11 % node that points to a glue specification
N \chardef \kernnode     = 12 % a kern node
N \chardef \penaltynode  = 13 % a penalty node
N \chardef \unsetnode    = 14 % an unset node
N \chardef \mathsnodes   = 15 % nodes that occur only in maths mode

%% Declare names for `iftypes'

C \chardef \charif     =  1 % \if
C \chardef \catif      =  2 % \ifcat
C \chardef \numif      =  3 % \ifnum
C \chardef \dimif      =  4 % \ifdim
C \chardef \oddif      =  5 % \ifodd
C \chardef \vmodeif    =  6 % \ifvmode
C \chardef \hmodeif    =  7 % \ifhmode
C \chardef \mmodeif    =  8 % \ifmmode
C \chardef \innerif    =  9 % \ifinner
C \chardef \voidif     = 10 % \ifvoid
C \chardef \hboxif     = 11 % \ifhbox
C \chardef \vboxif     = 12 % \ifvbox
C \chardef \xif        = 13 % \ifx
C \chardef \eofif      = 14 % \ifeof
C \chardef \trueif     = 15 % \iftrue
C \chardef \falseif    = 16 % \iffalse
C \chardef \caseif     = 17 % \ifcase
C \chardef \definedif  = 18 % \ifdefined
C \chardef \csnameif   = 19 % \ifcsname
C \chardef \fontcharif = 20 % \iffontchar

\catcode`\G=11
\catcode`\I=11
\catcode`\N=11
\catcode`\C=11

%
    \fi
    \catcode`\@=11 %
    \outer\expandafter\def\csname newfam\endcsname
                          {\alloc@8\fam\chardef\et@xmaxfam}
  \else
    \RequirePackage{etex}
    \expandafter\def\csname newfam\endcsname
                    {\alloc@8\fam\chardef\et@xmaxfam}
    \expandafter\let\expandafter\new@mathgroup\csname newfam\endcsname
  \fi
%    \end{macrocode}
%
% \subsubsection{Fixes to \texttt{etex.src}/\texttt{etex.sty}}
%
% These could and probably should be made directly in an
% update to |etex.src| which already has some Lua\TeX-specific
% code, but does not define the correct range for Lua\TeX.
%
% 2015-07-13 higher range in luatex.
%    \begin{macrocode}
\edef \et@xmaxregs {\ifx\directlua\@undefined 32768\else 65536\fi}
%    \end{macrocode}
% luatex/xetex also allow more math fam.
%    \begin{macrocode}
\edef \et@xmaxfam {\ifx\Umathcode\@undefined\sixt@@n\else\@cclvi\fi}
%    \end{macrocode}
%
%    \begin{macrocode}
\count 270=\et@xmaxregs % locally allocates \count registers
\count 271=\et@xmaxregs % ditto for \dimen registers
\count 272=\et@xmaxregs % ditto for \skip registers
\count 273=\et@xmaxregs % ditto for \muskip registers
\count 274=\et@xmaxregs % ditto for \box registers
\count 275=\et@xmaxregs % ditto for \toks registers
\count 276=\et@xmaxregs % ditto for \marks classes
%    \end{macrocode}
%
% and 256 or 16 fam. (Done above due to plain/\LaTeX\ differences in
% \textsf{ltluatex}.)
%    \begin{macrocode}
% \outer\def\newfam{\alloc@8\fam\chardef\et@xmaxfam}
%    \end{macrocode}
%
% End of proposed changes to \texttt{etex.src}
%
% \subsubsection{luatex specific settings}
%
% Switch to global cf |luatex.sty| to leave room for inserts
% not really needed for luatex but possibly most compatible
% with existing use.
%    \begin{macrocode}
\expandafter\let\csname newcount\expandafter\expandafter\endcsname
                \csname globcount\endcsname
\expandafter\let\csname newdimen\expandafter\expandafter\endcsname
                \csname globdimen\endcsname
\expandafter\let\csname newskip\expandafter\expandafter\endcsname
                \csname globskip\endcsname
\expandafter\let\csname newbox\expandafter\expandafter\endcsname
                \csname globbox\endcsname
%    \end{macrocode}
%
% Define|\e@alloc| as in latex (the existing macros in |etex.src|
% hard to extend to further register types as they assume specific
% 26x and 27x count range. For compatibility the existing register
% allocation is not changed.
%
%    \begin{macrocode}
\chardef\e@alloc@top=65535
\let\e@alloc@chardef\chardef
%    \end{macrocode}
%
%    \begin{macrocode}
\def\e@alloc#1#2#3#4#5#6{%
  \global\advance#3\@ne
  \e@ch@ck{#3}{#4}{#5}#1%
  \allocationnumber#3\relax
  \global#2#6\allocationnumber
  \wlog{\string#6=\string#1\the\allocationnumber}}%
%    \end{macrocode}
%
%    \begin{macrocode}
\gdef\e@ch@ck#1#2#3#4{%
  \ifnum#1<#2\else
    \ifnum#1=#2\relax
      #1\@cclvi
      \ifx\count#4\advance#1 10 \fi
    \fi
    \ifnum#1<#3\relax
    \else
      \errmessage{No room for a new \string#4}%
    \fi
  \fi}%
%    \end{macrocode}
%
% Fix up allocations not to clash with |etex.src|.
%
%    \begin{macrocode}
\expandafter\csname newcount\endcsname\e@alloc@attribute@count
\expandafter\csname newcount\endcsname\e@alloc@ccodetable@count
\expandafter\csname newcount\endcsname\e@alloc@luafunction@count
\expandafter\csname newcount\endcsname\e@alloc@whatsit@count
\expandafter\csname newcount\endcsname\e@alloc@bytecode@count
\expandafter\csname newcount\endcsname\e@alloc@luachunk@count
%    \end{macrocode}
%
% End of conditional setup for plain \TeX\ / old \LaTeX.
%    \begin{macrocode}
\fi
%</tex>
%    \end{macrocode}
%
% \subsection{Attributes}
%
% \begin{macro}{\newattribute}
% \changes{v1.0a}{2015/09/24}{Macro added}
% \changes{v1.1q}{2020/08/02}{Move reset to $0$ inside conditional}
%   As is generally the case for the Lua\TeX{} registers we start here
%   from~$1$. Notably, some code assumes that |\attribute0| is never used so
%   this is important in this case.
%    \begin{macrocode}
\ifx\e@alloc@attribute@count\@undefined
  \countdef\e@alloc@attribute@count=258
  \e@alloc@attribute@count=\z@
\fi
\def\newattribute#1{%
  \e@alloc\attribute\attributedef
    \e@alloc@attribute@count\m@ne\e@alloc@top#1%
}
%    \end{macrocode}
% \end{macro}
%
% \begin{macro}{\setattribute}
% \begin{macro}{\unsetattribute}
%   Handy utilities.
%    \begin{macrocode}
\def\setattribute#1#2{#1=\numexpr#2\relax}
\def\unsetattribute#1{#1=-"7FFFFFFF\relax}
%    \end{macrocode}
% \end{macro}
% \end{macro}
%
% \subsection{Category code tables}
%
% \begin{macro}{\newcatcodetable}
% \changes{v1.0a}{2015/09/24}{Macro added}
%   Category code tables are allocated with a limit half of that used by Lua\TeX{}
%   for everything else. At the end of allocation there needs to be an
%   initialization step. Table~$0$ is already taken (it's the global one for
%   current use) so the allocation starts at~$1$.
%    \begin{macrocode}
\ifx\e@alloc@ccodetable@count\@undefined
  \countdef\e@alloc@ccodetable@count=259
  \e@alloc@ccodetable@count=\z@
\fi
\def\newcatcodetable#1{%
  \e@alloc\catcodetable\chardef
    \e@alloc@ccodetable@count\m@ne{"8000}#1%
  \initcatcodetable\allocationnumber
}
%    \end{macrocode}
% \end{macro}
%
% \changes{v1.0l}{2015/12/18}{Load Unicode data from source}
% \begin{macro}{\catcodetable@initex}
% \changes{v1.0a}{2015/09/24}{Macro added}
% \begin{macro}{\catcodetable@string}
% \changes{v1.0a}{2015/09/24}{Macro added}
% \begin{macro}{\catcodetable@latex}
% \changes{v1.0a}{2015/09/24}{Macro added}
% \begin{macro}{\catcodetable@atletter}
% \changes{v1.0a}{2015/09/24}{Macro added}
%   Save a small set of standard tables. The Unicode data is read
%   here in using a parser simplified from that in |load-unicode-data|:
%   only the nature of letters needs to be detected.
%    \begin{macrocode}
\newcatcodetable\catcodetable@initex
\newcatcodetable\catcodetable@string
\begingroup
  \def\setrangecatcode#1#2#3{%
    \ifnum#1>#2 %
      \expandafter\@gobble
    \else
      \expandafter\@firstofone
    \fi
      {%
        \catcode#1=#3 %
        \expandafter\setrangecatcode\expandafter
          {\number\numexpr#1 + 1\relax}{#2}{#3}
      }%
  }
  \@firstofone{%
    \catcodetable\catcodetable@initex
      \catcode0=12 %
      \catcode13=12 %
      \catcode37=12 %
      \setrangecatcode{65}{90}{12}%
      \setrangecatcode{97}{122}{12}%
      \catcode92=12 %
      \catcode127=12 %
      \savecatcodetable\catcodetable@string
    \endgroup
  }%
\newcatcodetable\catcodetable@latex
\newcatcodetable\catcodetable@atletter
\begingroup
  \def\parseunicodedataI#1;#2;#3;#4\relax{%
    \parseunicodedataII#1;#3;#2 First>\relax
  }%
  \def\parseunicodedataII#1;#2;#3 First>#4\relax{%
    \ifx\relax#4\relax
      \expandafter\parseunicodedataIII
    \else
      \expandafter\parseunicodedataIV
    \fi
      {#1}#2\relax%
  }%
  \def\parseunicodedataIII#1#2#3\relax{%
    \ifnum 0%
      \if L#21\fi
      \if M#21\fi
      >0 %
      \catcode"#1=11 %
    \fi
  }%
  \def\parseunicodedataIV#1#2#3\relax{%
    \read\unicoderead to \unicodedataline
    \if L#2%
      \count0="#1 %
      \expandafter\parseunicodedataV\unicodedataline\relax
    \fi
  }%
  \def\parseunicodedataV#1;#2\relax{%
    \loop
      \unless\ifnum\count0>"#1 %
        \catcode\count0=11 %
        \advance\count0 by 1 %
    \repeat
  }%
  \def\storedpar{\par}%
  \chardef\unicoderead=\numexpr\count16 + 1\relax
  \openin\unicoderead=UnicodeData.txt %
  \loop\unless\ifeof\unicoderead %
    \read\unicoderead to \unicodedataline
    \unless\ifx\unicodedataline\storedpar
      \expandafter\parseunicodedataI\unicodedataline\relax
    \fi
  \repeat
  \closein\unicoderead
  \@firstofone{%
    \catcode64=12 %
    \savecatcodetable\catcodetable@latex
    \catcode64=11 %
    \savecatcodetable\catcodetable@atletter
   }
\endgroup
%    \end{macrocode}
% \end{macro}
% \end{macro}
% \end{macro}
% \end{macro}
%
% \subsection{Named Lua functions}
%
% \begin{macro}{\newluafunction}
% \changes{v1.0a}{2015/09/24}{Macro added}
% \changes{v1.1q}{2020/08/02}{Move reset to $0$ inside conditional}
%   Much the same story for allocating Lua\TeX{} functions except here they are
%   just numbers so they are allocated in the same way as boxes.
%   Lua indexes from~$1$ so once again slot~$0$ is skipped.
%    \begin{macrocode}
\ifx\e@alloc@luafunction@count\@undefined
  \countdef\e@alloc@luafunction@count=260
  \e@alloc@luafunction@count=\z@
\fi
\def\newluafunction{%
  \e@alloc\luafunction\e@alloc@chardef
    \e@alloc@luafunction@count\m@ne\e@alloc@top
}
%    \end{macrocode}
% \end{macro}
%
% \subsection{Custom whatsits}
%
% \begin{macro}{\newwhatsit}
% \changes{v1.0a}{2015/09/24}{Macro added}
% \changes{v1.1q}{2020/08/02}{Move reset to $0$ inside conditional}
%   These are only settable from Lua but for consistency are definable
%   here.
%    \begin{macrocode}
\ifx\e@alloc@whatsit@count\@undefined
  \countdef\e@alloc@whatsit@count=261
  \e@alloc@whatsit@count=\z@
\fi
\def\newwhatsit#1{%
  \e@alloc\whatsit\e@alloc@chardef
    \e@alloc@whatsit@count\m@ne\e@alloc@top#1%
}
%    \end{macrocode}
% \end{macro}
%
% \subsection{Lua bytecode registers}
%
% \begin{macro}{\newluabytecode}
% \changes{v1.0a}{2015/09/24}{Macro added}
% \changes{v1.1q}{2020/08/02}{Move reset to $0$ inside conditional}
%   These are only settable from Lua but for consistency are definable
%   here.
%    \begin{macrocode}
\ifx\e@alloc@bytecode@count\@undefined
  \countdef\e@alloc@bytecode@count=262
  \e@alloc@bytecode@count=\z@
\fi
\def\newluabytecode#1{%
  \e@alloc\luabytecode\e@alloc@chardef
    \e@alloc@bytecode@count\m@ne\e@alloc@top#1%
}
%    \end{macrocode}
% \end{macro}
%
% \subsection{Lua chunk registers}

% \begin{macro}{\newluachunkname}
% \changes{v1.0a}{2015/09/24}{Macro added}
% \changes{v1.1q}{2020/08/02}{Move reset to $0$ inside conditional}
% As for bytecode registers, but in addition we need to add a string
% to the \verb|lua.name| table to use in stack tracing. We use the
% name of the command passed to the allocator, with no backslash.
%    \begin{macrocode}
\ifx\e@alloc@luachunk@count\@undefined
  \countdef\e@alloc@luachunk@count=263
  \e@alloc@luachunk@count=\z@
\fi
\def\newluachunkname#1{%
  \e@alloc\luachunk\e@alloc@chardef
    \e@alloc@luachunk@count\m@ne\e@alloc@top#1%
    {\escapechar\m@ne
    \directlua{lua.name[\the\allocationnumber]="\string#1"}}%
}
%    \end{macrocode}
% \end{macro}
%
% \subsection{Lua loader}
% \changes{v1.1r}{2020/08/10}{Load ltluatex Lua module during format building}
%
% Lua code loaded in the format often has to be loaded again at the
% beginning of every job, so we define a helper which allows us to avoid
% duplicated code:
%
%    \begin{macrocode}
\def\now@and@everyjob#1{%
  \everyjob\expandafter{\the\everyjob
    #1%
  }%
  #1%
}
%    \end{macrocode}
%
% Load the Lua code at the start of every job.
% For the conversion of \TeX{} into numbers at the Lua side we need some
% known registers: for convenience we use a set of systematic names, which
% means using a group around the Lua loader.
%    \begin{macrocode}
%<2ekernel>\now@and@everyjob{%
  \begingroup
    \attributedef\attributezero=0 %
    \chardef     \charzero     =0 %
%    \end{macrocode}
% Note name change required on older luatex, for hash table access.
%    \begin{macrocode}
    \countdef    \CountZero    =0 %
    \dimendef    \dimenzero    =0 %
    \mathchardef \mathcharzero =0 %
    \muskipdef   \muskipzero   =0 %
    \skipdef     \skipzero     =0 %
    \toksdef     \tokszero     =0 %
    \directlua{require("ltluatex")}
  \endgroup
%<2ekernel>}
%<latexrelease>\EndIncludeInRelease
%    \end{macrocode}
%
% \changes{v1.0b}{2015/10/02}{Fix backing out of \TeX{} code}
% \changes{v1.0c}{2015/10/02}{Allow backing out of Lua code}
%    \begin{macrocode}
%<latexrelease>\IncludeInRelease{0000/00/00}
%<latexrelease>                 {\newluafunction}{LuaTeX}%
%<latexrelease>\let\e@alloc@attribute@count\@undefined
%<latexrelease>\let\newattribute\@undefined
%<latexrelease>\let\setattribute\@undefined
%<latexrelease>\let\unsetattribute\@undefined
%<latexrelease>\let\e@alloc@ccodetable@count\@undefined
%<latexrelease>\let\newcatcodetable\@undefined
%<latexrelease>\let\catcodetable@initex\@undefined
%<latexrelease>\let\catcodetable@string\@undefined
%<latexrelease>\let\catcodetable@latex\@undefined
%<latexrelease>\let\catcodetable@atletter\@undefined
%<latexrelease>\let\e@alloc@luafunction@count\@undefined
%<latexrelease>\let\newluafunction\@undefined
%<latexrelease>\let\e@alloc@luafunction@count\@undefined
%<latexrelease>\let\newwhatsit\@undefined
%<latexrelease>\let\e@alloc@whatsit@count\@undefined
%<latexrelease>\let\newluabytecode\@undefined
%<latexrelease>\let\e@alloc@bytecode@count\@undefined
%<latexrelease>\let\newluachunkname\@undefined
%<latexrelease>\let\e@alloc@luachunk@count\@undefined
%<latexrelease>\directlua{luatexbase.uninstall()}
%<latexrelease>\EndIncludeInRelease
%    \end{macrocode}
%
% In \verb|\everyjob|, if luaotfload is available, load it and switch to TU.
%    \begin{macrocode}
%<latexrelease>\IncludeInRelease{2017/01/01}%
%<latexrelease>                 {\fontencoding}{TU in everyjob}%
%<latexrelease>\fontencoding{TU}\let\encodingdefault\f@encoding
%<latexrelease>\ifx\directlua\@undefined\else
%<2ekernel>\everyjob\expandafter{%
%<2ekernel>  \the\everyjob
%<*2ekernel,latexrelease>
  \directlua{%
  if xpcall(function ()%
             require('luaotfload-main')%
            end,texio.write_nl) then %
  local _void = luaotfload.main ()%
  else %
  texio.write_nl('Error in luaotfload: reverting to OT1')%
  tex.print('\string\\def\string\\encodingdefault{OT1}')%
  end %
  }%
  \let\f@encoding\encodingdefault
  \expandafter\let\csname ver@luaotfload.sty\endcsname\fmtversion
%</2ekernel,latexrelease>
%<latexrelease>\fi
%<2ekernel>  }
%<latexrelease>\EndIncludeInRelease
%<latexrelease>\IncludeInRelease{0000/00/00}%
%<latexrelease>                 {\fontencoding}{TU in everyjob}%
%<latexrelease>\fontencoding{OT1}\let\encodingdefault\f@encoding
%<latexrelease>\EndIncludeInRelease
%    \end{macrocode}
%
%    \begin{macrocode}
%<2ekernel|latexrelease>\fi
%</2ekernel|tex|latexrelease>
%    \end{macrocode}
%
% \subsection{Lua module preliminaries}
%
% \begingroup
%
%  \begingroup\lccode`~=`_
%  \lowercase{\endgroup\let~}_
%  \catcode`_=12
%
%    \begin{macrocode}
%<*lua>
%    \end{macrocode}
%
% Some set up for the Lua module which is needed for all of the Lua
% functionality added here.
%
% \begin{macro}{luatexbase}
% \changes{v1.0a}{2015/09/24}{Table added}
%   Set up the table for the returned functions. This is used to expose
%   all of the public functions.
%    \begin{macrocode}
luatexbase       = luatexbase or { }
local luatexbase = luatexbase
%    \end{macrocode}
% \end{macro}
%
% Some Lua best practice: use local versions of functions where possible.
% \changes{v1.1u}{2021/08/11}{Define missing local function}
%    \begin{macrocode}
local string_gsub      = string.gsub
local tex_count        = tex.count
local tex_setattribute = tex.setattribute
local tex_setcount     = tex.setcount
local texio_write_nl   = texio.write_nl
local flush_list       = node.flush_list
%    \end{macrocode}
% \changes{v1.0i}{2015/11/29}{Declare this as local before used in the module error definitions (PHG)}
%    \begin{macrocode}
local luatexbase_warning
local luatexbase_error
%    \end{macrocode}
%
% \subsection{Lua module utilities}
%
% \subsubsection{Module tracking}
%
% \begin{macro}{modules}
% \changes{v1.0a}{2015/09/24}{Function modified}
%   To allow tracking of module usage, a structure is provided to store
%   information and to return it.
%    \begin{macrocode}
local modules = modules or { }
%    \end{macrocode}
% \end{macro}
%
% \begin{macro}{provides_module}
% \changes{v1.0a}{2015/09/24}{Function added}
% \changes{v1.0f}{2015/10/03}{use luatexbase\_log}
% Local function to write to the log.
%    \begin{macrocode}
local function luatexbase_log(text)
  texio_write_nl("log", text)
end
%    \end{macrocode}
%
%   Modelled on |\ProvidesPackage|, we store much the same information but
%   with a little more structure.
%    \begin{macrocode}
local function provides_module(info)
  if not (info and info.name) then
    luatexbase_error("Missing module name for provides_module")
  end
  local function spaced(text)
    return text and (" " .. text) or ""
  end
  luatexbase_log(
    "Lua module: " .. info.name
      .. spaced(info.date)
      .. spaced(info.version)
      .. spaced(info.description)
  )
  modules[info.name] = info
end
luatexbase.provides_module = provides_module
%    \end{macrocode}
% \end{macro}
%
% \subsubsection{Module messages}
%
% There are various warnings and errors that need to be given. For warnings
% we can get exactly the same formatting as from \TeX{}. For errors we have to
% make some changes. Here we give the text of the error in the \LaTeX{} format
% then force an error from Lua to halt the run. Splitting the message text is
% done using |\n| which takes the place of |\MessageBreak|.
%
% First an auxiliary for the formatting: this measures up the message
% leader so we always get the correct indent.
% \changes{v1.0j}{2015/12/02}{Declaration/use of first\_head fixed (PHG)}
%    \begin{macrocode}
local function msg_format(mod, msg_type, text)
  local leader = ""
  local cont
  local first_head
  if mod == "LaTeX" then
    cont = string_gsub(leader, ".", " ")
    first_head = leader .. "LaTeX: "
  else
    first_head = leader .. "Module "  .. msg_type
    cont = "(" .. mod .. ")"
      .. string_gsub(first_head, ".", " ")
    first_head =  leader .. "Module "  .. mod .. " " .. msg_type  .. ":"
  end
  if msg_type == "Error" then
    first_head = "\n" .. first_head
  end
  if string.sub(text,-1) ~= "\n" then
    text = text .. " "
  end
  return first_head .. " "
    .. string_gsub(
         text
	 .. "on input line "
         .. tex.inputlineno, "\n", "\n" .. cont .. " "
      )
   .. "\n"
end
%    \end{macrocode}
%
% \begin{macro}{module_info}
% \changes{v1.0a}{2015/09/24}{Function added}
% \begin{macro}{module_warning}
% \changes{v1.0a}{2015/09/24}{Function added}
% \begin{macro}{module_error}
% \changes{v1.0a}{2015/09/24}{Function added}
%   Write messages.
%    \begin{macrocode}
local function module_info(mod, text)
  texio_write_nl("log", msg_format(mod, "Info", text))
end
luatexbase.module_info = module_info
local function module_warning(mod, text)
  texio_write_nl("term and log",msg_format(mod, "Warning", text))
end
luatexbase.module_warning = module_warning
local function module_error(mod, text)
  error(msg_format(mod, "Error", text))
end
luatexbase.module_error = module_error
%    \end{macrocode}
% \end{macro}
% \end{macro}
% \end{macro}
%
% Dedicated versions for the rest of the code here.
%    \begin{macrocode}
function luatexbase_warning(text)
  module_warning("luatexbase", text)
end
function luatexbase_error(text)
  module_error("luatexbase", text)
end
%    \end{macrocode}
%
%
% \subsection{Accessing register numbers from Lua}
%
% \changes{v1.0g}{2015/11/14}{Track Lua\TeX{} changes for
%   \texttt{(new)token.create}}
% Collect up the data from the \TeX{} level into a Lua table: from
% version~0.80, Lua\TeX{} makes that easy.
% \changes{v1.0j}{2015/12/02}{Adjust hashtokens to store the result of tex.hashtokens()), not the function (PHG)}
%    \begin{macrocode}
local luaregisterbasetable = { }
local registermap = {
  attributezero = "assign_attr"    ,
  charzero      = "char_given"     ,
  CountZero     = "assign_int"     ,
  dimenzero     = "assign_dimen"   ,
  mathcharzero  = "math_given"     ,
  muskipzero    = "assign_mu_skip" ,
  skipzero      = "assign_skip"    ,
  tokszero      = "assign_toks"    ,
}
local createtoken
if tex.luatexversion > 81 then
  createtoken = token.create
elseif tex.luatexversion > 79 then
  createtoken = newtoken.create
end
local hashtokens    = tex.hashtokens()
local luatexversion = tex.luatexversion
for i,j in pairs (registermap) do
  if luatexversion < 80 then
    luaregisterbasetable[hashtokens[i][1]] =
      hashtokens[i][2]
  else
    luaregisterbasetable[j] = createtoken(i).mode
  end
end
%    \end{macrocode}
%
% \begin{macro}{registernumber}
%   Working out the correct return value can be done in two ways. For older
%   Lua\TeX{} releases it has to be extracted from the |hashtokens|. On the
%   other hand, newer Lua\TeX{}'s have |newtoken|, and whilst |.mode| isn't
%   currently documented, Hans Hagen pointed to this approach so we should be
%   OK.
%    \begin{macrocode}
local registernumber
if luatexversion < 80 then
  function registernumber(name)
    local nt = hashtokens[name]
    if(nt and luaregisterbasetable[nt[1]]) then
      return nt[2] - luaregisterbasetable[nt[1]]
    else
      return false
    end
  end
else
  function registernumber(name)
    local nt = createtoken(name)
    if(luaregisterbasetable[nt.cmdname]) then
      return nt.mode - luaregisterbasetable[nt.cmdname]
    else
      return false
    end
  end
end
luatexbase.registernumber = registernumber
%    \end{macrocode}
% \end{macro}
%
% \subsection{Attribute allocation}
%
% \begin{macro}{new_attribute}
% \changes{v1.0a}{2015/09/24}{Function added}
% \changes{v1.1c}{2017/02/18}{Parameterize count used in tracking}
%   As attributes are used for Lua manipulations its useful to be able
%   to assign from this end.
% \InternalDetectionOff
%    \begin{macrocode}
local attributes=setmetatable(
{},
{
__index = function(t,key)
return registernumber(key) or nil
end}
)
luatexbase.attributes = attributes
%    \end{macrocode}
%
%    \begin{macrocode}
local attribute_count_name =
                     attribute_count_name or "e@alloc@attribute@count"
local function new_attribute(name)
  tex_setcount("global", attribute_count_name,
                          tex_count[attribute_count_name] + 1)
  if tex_count[attribute_count_name] > 65534 then
    luatexbase_error("No room for a new \\attribute")
  end
  attributes[name]= tex_count[attribute_count_name]
  luatexbase_log("Lua-only attribute " .. name .. " = " ..
                 tex_count[attribute_count_name])
  return tex_count[attribute_count_name]
end
luatexbase.new_attribute = new_attribute
%    \end{macrocode}
% \InternalDetectionOn
% \end{macro}
%
% \subsection{Custom whatsit allocation}
%
% \begin{macro}{new_whatsit}
% \changes{v1.1c}{2017/02/18}{Parameterize count used in tracking}
% Much the same as for attribute allocation in Lua.
%    \begin{macrocode}
local whatsit_count_name = whatsit_count_name or "e@alloc@whatsit@count"
local function new_whatsit(name)
  tex_setcount("global", whatsit_count_name,
                         tex_count[whatsit_count_name] + 1)
  if tex_count[whatsit_count_name] > 65534 then
    luatexbase_error("No room for a new custom whatsit")
  end
  luatexbase_log("Custom whatsit " .. (name or "") .. " = " ..
                 tex_count[whatsit_count_name])
  return tex_count[whatsit_count_name]
end
luatexbase.new_whatsit = new_whatsit
%    \end{macrocode}
% \end{macro}
%
% \subsection{Bytecode register allocation}
%
% \begin{macro}{new_bytecode}
% \changes{v1.1c}{2017/02/18}{Parameterize count used in tracking}
% Much the same as for attribute allocation in Lua.
% The optional \meta{name} argument is used in the log if given.
%    \begin{macrocode}
local bytecode_count_name =
                         bytecode_count_name or "e@alloc@bytecode@count"
local function new_bytecode(name)
  tex_setcount("global", bytecode_count_name,
                         tex_count[bytecode_count_name] + 1)
  if tex_count[bytecode_count_name] > 65534 then
    luatexbase_error("No room for a new bytecode register")
  end
  luatexbase_log("Lua bytecode " .. (name or "") .. " = " ..
                 tex_count[bytecode_count_name])
  return tex_count[bytecode_count_name]
end
luatexbase.new_bytecode = new_bytecode
%    \end{macrocode}
% \end{macro}
%
% \subsection{Lua chunk name allocation}
%
% \begin{macro}{new_chunkname}
% \changes{v1.1c}{2017/02/18}{Parameterize count used in tracking}
% As for bytecode registers but also store the name in the
% |lua.name| table.
%    \begin{macrocode}
local chunkname_count_name =
                        chunkname_count_name or "e@alloc@luachunk@count"
local function new_chunkname(name)
  tex_setcount("global", chunkname_count_name,
                         tex_count[chunkname_count_name] + 1)
  local chunkname_count = tex_count[chunkname_count_name]
  chunkname_count = chunkname_count + 1
  if chunkname_count > 65534 then
    luatexbase_error("No room for a new chunkname")
  end
  lua.name[chunkname_count]=name
  luatexbase_log("Lua chunkname " .. (name or "") .. " = " ..
                 chunkname_count .. "\n")
  return chunkname_count
end
luatexbase.new_chunkname = new_chunkname
%    \end{macrocode}
% \end{macro}
%
% \subsection{Lua function allocation}
%
% \begin{macro}{new_luafunction}
% \changes{v1.1i}{2018/10/21}{Function added}
% Much the same as for attribute allocation in Lua.
% The optional \meta{name} argument is used in the log if given.
%    \begin{macrocode}
local luafunction_count_name =
                         luafunction_count_name or "e@alloc@luafunction@count"
local function new_luafunction(name)
  tex_setcount("global", luafunction_count_name,
                         tex_count[luafunction_count_name] + 1)
  if tex_count[luafunction_count_name] > 65534 then
    luatexbase_error("No room for a new luafunction register")
  end
  luatexbase_log("Lua function " .. (name or "") .. " = " ..
                 tex_count[luafunction_count_name])
  return tex_count[luafunction_count_name]
end
luatexbase.new_luafunction = new_luafunction
%    \end{macrocode}
% \end{macro}
%
% \subsection{Lua callback management}
%
% The native mechanism for callbacks in Lua\TeX\ allows only one per function.
% That is extremely restrictive and so a mechanism is needed to add and
% remove callbacks from the appropriate hooks.
%
% \subsubsection{Housekeeping}
%
% The main table: keys are callback names, and values are the associated lists
% of functions. More precisely, the entries in the list are tables holding the
% actual function as |func| and the identifying description as |description|.
% Only callbacks with a non-empty list of functions have an entry in this
% list.
%    \begin{macrocode}
local callbacklist = callbacklist or { }
%    \end{macrocode}
%
% Numerical codes for callback types, and name-to-value association (the
% table keys are strings, the values are numbers).
%    \begin{macrocode}
local list, data, exclusive, simple, reverselist = 1, 2, 3, 4, 5
local types   = {
  list        = list,
  data        = data,
  exclusive   = exclusive,
  simple      = simple,
  reverselist = reverselist,
}
%    \end{macrocode}
%
% Now, list all predefined callbacks with their current type, based on the
% Lua\TeX{} manual version~1.01. A full list of the currently-available
% callbacks can be obtained using
%  \begin{verbatim}
%    \directlua{
%      for i,_ in pairs(callback.list()) do
%        texio.write_nl("- " .. i)
%      end
%    }
%    \bye
%  \end{verbatim}
% in plain Lua\TeX{}. (Some undocumented callbacks are omitted as they are
% to be removed.)
%    \begin{macrocode}
local callbacktypes = callbacktypes or {
%    \end{macrocode}
%   Section 8.2: file discovery callbacks.
% \changes{v1.1g}{2018/05/02}{find\_sfd\_file removed}
%    \begin{macrocode}
  find_read_file     = exclusive,
  find_write_file    = exclusive,
  find_font_file     = data,
  find_output_file   = data,
  find_format_file   = data,
  find_vf_file       = data,
  find_map_file      = data,
  find_enc_file      = data,
  find_pk_file       = data,
  find_data_file     = data,
  find_opentype_file = data,
  find_truetype_file = data,
  find_type1_file    = data,
  find_image_file    = data,
%    \end{macrocode}
% \changes{v1.1g}{2018/05/02}{read\_sfd\_file removed}
%    \begin{macrocode}
  open_read_file     = exclusive,
  read_font_file     = exclusive,
  read_vf_file       = exclusive,
  read_map_file      = exclusive,
  read_enc_file      = exclusive,
  read_pk_file       = exclusive,
  read_data_file     = exclusive,
  read_truetype_file = exclusive,
  read_type1_file    = exclusive,
  read_opentype_file = exclusive,
%    \end{macrocode}
% \changes{v1.0m}{2016/02/11}{read\_cidmap\_file added}
% Not currently used by luatex but included for completeness.
% may be used by a font handler.
%    \begin{macrocode}
  find_cidmap_file   = data,
  read_cidmap_file   = exclusive,
%    \end{macrocode}
% Section 8.3: data processing callbacks.
% \changes{v1.0m}{2016/02/11}{token\_filter removed}
%    \begin{macrocode}
  process_input_buffer  = data,
  process_output_buffer = data,
  process_jobname       = data,
%    \end{macrocode}
% Section 8.4: node list processing callbacks.
% \changes{v1.0m}{2016/02/11}
% {process\_rule, [hv]pack\_quality  append\_to\_vlist\_filter added}
% \changes{v1.0n}{2016/03/13}{insert\_local\_par added}
% \changes{v1.0n}{2016/03/13}{contribute\_filter added}
% \changes{v1.1h}{2018/08/18}{append\_to\_vlist\_filter is \texttt{exclusive}}
% \changes{v1.1j}{2019/06/18}{new\_graf added}
% \changes{v1.1k}{2019/10/02}{linebreak\_filter is \texttt{exclusive}}
% \changes{v1.1k}{2019/10/02}{process\_rule is \texttt{exclusive}}
% \changes{v1.1k}{2019/10/02}{mlist\_to\_hlist is \texttt{exclusive}}
% \changes{v1.1l}{2020/02/02}{post\_linebreak\_filter is \texttt{reverselist}}
% \changes{v1.1p}{2020/08/01}{new\_graf is \texttt{exclusive}}
% \changes{v1.1w}{2021/11/17}{hpack\_quality is \texttt{exclusive}}
% \changes{v1.1w}{2021/11/17}{vpack\_quality is \texttt{exclusive}}
%    \begin{macrocode}
  contribute_filter      = simple,
  buildpage_filter       = simple,
  build_page_insert      = exclusive,
  pre_linebreak_filter   = list,
  linebreak_filter       = exclusive,
  append_to_vlist_filter = exclusive,
  post_linebreak_filter  = reverselist,
  hpack_filter           = list,
  vpack_filter           = list,
  hpack_quality          = exclusive,
  vpack_quality          = exclusive,
  pre_output_filter      = list,
  process_rule           = exclusive,
  hyphenate              = simple,
  ligaturing             = simple,
  kerning                = simple,
  insert_local_par       = simple,
  pre_mlist_to_hlist_filter = list,
  mlist_to_hlist         = exclusive,
  post_mlist_to_hlist_filter = reverselist,
  new_graf               = exclusive,
%    \end{macrocode}
% Section 8.5: information reporting callbacks.
% \changes{v1.0m}{2016/02/11}{show\_warning\_message added}
% \changes{v1.0p}{2016/11/17}{call\_edit added}
% \changes{v1.1g}{2018/05/02}{finish\_synctex\_callback added}
% \changes{v1.1j}{2019/06/18}{finish\_synctex\_callback renamed finish\_synctex}
% \changes{v1.1j}{2019/06/18}{wrapup\_run added}
%    \begin{macrocode}
  pre_dump             = simple,
  start_run            = simple,
  stop_run             = simple,
  start_page_number    = simple,
  stop_page_number     = simple,
  show_error_hook      = simple,
  show_warning_message = simple,
  show_error_message   = simple,
  show_lua_error_hook  = simple,
  start_file           = simple,
  stop_file            = simple,
  call_edit            = simple,
  finish_synctex       = simple,
  wrapup_run           = simple,
%    \end{macrocode}
% Section 8.6: PDF-related callbacks.
% \changes{v1.1j}{2019/06/18}{page\_objnum\_provider added}
% \changes{v1.1j}{2019/06/18}{process\_pdf\_image\_content added}
% \changes{v1.1j}{2019/10/22}{page\_objnum\_provider and process\_pdf\_image\_content classified data}
% \changes{v1.1l}{2020/02/02}{page\_order\_index added}
%    \begin{macrocode}
  finish_pdffile            = data,
  finish_pdfpage            = data,
  page_objnum_provider      = data,
  page_order_index          = data,
  process_pdf_image_content = data,
%    \end{macrocode}
% Section 8.7: font-related callbacks.
% \changes{v1.1e}{2017/03/28}{glyph\_stream\_provider added}
% \changes{v1.1g}{2018/05/02}{glyph\_not\_found added}
% \changes{v1.1j}{2019/06/18}{make\_extensible added}
% \changes{v1.1j}{2019/06/18}{font\_descriptor\_objnum\_provider added}
% \changes{v1.1l}{2020/02/02}{glyph\_info added}
% \changes{v1.1t}{2021/04/18}{input\_level\_string added}
% \changes{v1.1v}{2021/10/15}{provide\_charproc\_data added}
%    \begin{macrocode}
  define_font                     = exclusive,
  glyph_info                      = exclusive,
  glyph_not_found                 = exclusive,
  glyph_stream_provider           = exclusive,
  make_extensible                 = exclusive,
  font_descriptor_objnum_provider = exclusive,
  input_level_string              = exclusive,
  provide_charproc_data           = exclusive,
%    \end{macrocode}
% \changes{v1.0m}{2016/02/11}{pdf\_stream\_filter\_callback removed}
%    \begin{macrocode}
}
luatexbase.callbacktypes=callbacktypes
%    \end{macrocode}
%
% \begin{macro}{callback.register}
% \changes{v1.0a}{2015/09/24}{Function modified}
%   Save the original function for registering callbacks and prevent the
%   original being used. The original is saved in a place that remains
%   available so other more sophisticated code can override the approach
%   taken by the kernel if desired.
%    \begin{macrocode}
local callback_register = callback_register or callback.register
function callback.register()
  luatexbase_error("Attempt to use callback.register() directly\n")
end
%    \end{macrocode}
% \end{macro}
%
% \subsubsection{Handlers}
%
% The handler function is registered into the callback when the
% first function is added to this callback's list. Then, when the callback
% is called, the handler takes care of running all functions in the list.
% When the last function is removed from the callback's list, the handler
% is unregistered.
%
% More precisely, the functions below are used to generate a specialized
% function (closure) for a given callback, which is the actual handler.
%
%
% The way the functions are combined together depends on
% the type of the callback. There are currently 4 types of callback, depending
% on the calling convention of the functions the callback can hold:
% \begin{description}
%   \item[simple] is for functions that don't return anything: they are called
%     in order, all with the same argument;
%   \item[data] is for functions receiving a piece of data of any type
%     except node list head (and possibly other arguments) and returning it
%     (possibly modified): the functions are called in order, and each is
%     passed the return value of the previous (and the other arguments
%     untouched, if any). The return value is that of the last function;
%   \item[list] is a specialized variant of \emph{data} for functions
%     filtering node lists. Such functions may return either the head of a
%     modified node list, or the boolean values |true| or |false|. The
%     functions are chained the same way as for \emph{data} except that for
%     the following. If
%     one function returns |false|, then |false| is immediately returned and
%     the following functions are \emph{not} called. If one function returns
%     |true|, then the same head is passed to the next function. If all
%     functions return |true|, then |true| is returned, otherwise the return
%     value of the last function not returning |true| is used.
%   \item[reverselist] is a specialized variant of \emph{list} which executes
%     functions in inverse order.
%   \item[exclusive] is for functions with more complex signatures; functions in
%     this type of callback are \emph{not} combined: An error is raised if
%     a second callback is registered.
% \end{description}
%
% Handler for |data| callbacks.
%    \begin{macrocode}
local function data_handler(name)
  return function(data, ...)
    for _,i in ipairs(callbacklist[name]) do
      data = i.func(data,...)
    end
    return data
  end
end
%    \end{macrocode}
% Default for user-defined |data| callbacks without explicit default.
%    \begin{macrocode}
local function data_handler_default(value)
  return value
end
%    \end{macrocode}
% Handler for |exclusive| callbacks. We can assume |callbacklist[name]| is not
% empty: otherwise, the function wouldn't be registered in the callback any
% more.
%    \begin{macrocode}
local function exclusive_handler(name)
  return function(...)
    return callbacklist[name][1].func(...)
  end
end
%    \end{macrocode}
% Handler for |list| callbacks.
% \changes{v1.0k}{2015/12/02}{resolve name and i.description (PHG)}
% \changes{v1.1s}{2020/12/02}{Fix return value of list callbacks}
% \changes{v1.1w}{2021/11/17}{Never pass on \texttt{true} return values for list callbacks}
%    \begin{macrocode}
local function list_handler(name)
  return function(head, ...)
    local ret
    for _,i in ipairs(callbacklist[name]) do
      ret = i.func(head, ...)
      if ret == false then
        luatexbase_warning(
          "Function `" .. i.description .. "' returned false\n"
            .. "in callback `" .. name .."'"
         )
        return false
      end
      if ret ~= true then
        head = ret
      end
    end
    return head
  end
end
%    \end{macrocode}
% Default for user-defined |list| and |reverselist| callbacks without explicit default.
%    \begin{macrocode}
local function list_handler_default(head)
return head
end
%    \end{macrocode}
% Handler for |reverselist| callbacks.
% \changes{v1.1l}{2020/02/02}{Add reverselist callback type}
%    \begin{macrocode}
local function reverselist_handler(name)
  return function(head, ...)
    local ret
    local callbacks = callbacklist[name]
    for i = #callbacks, 1, -1 do
      local cb = callbacks[i]
      ret = cb.func(head, ...)
      if ret == false then
        luatexbase_warning(
          "Function `" .. cb.description .. "' returned false\n"
            .. "in callback `" .. name .."'"
         )
        return false
      end
      if ret ~= true then
        head = ret
      end
    end
    return head
  end
end
%    \end{macrocode}
% Handler for |simple| callbacks.
%    \begin{macrocode}
local function simple_handler(name)
  return function(...)
    for _,i in ipairs(callbacklist[name]) do
      i.func(...)
    end
  end
end
%    \end{macrocode}
% Default for user-defined |simple| callbacks without explicit default.
%    \begin{macrocode}
local function simple_handler_default()
end
%    \end{macrocode}
%
% Keep a handlers table for indexed access and a table with the corresponding default functions.
%    \begin{macrocode}
local handlers  = {
  [data]        = data_handler,
  [exclusive]   = exclusive_handler,
  [list]        = list_handler,
  [reverselist] = reverselist_handler,
  [simple]      = simple_handler,
}
local defaults = {
  [data]        = data_handler_default,
  [exclusive]   = nil,
  [list]        = list_handler_default,
  [reverselist] = list_handler_default,
  [simple]      = simple_handler_default,
}
%    \end{macrocode}
%
% \subsubsection{Public functions for callback management}
%
% Defining user callbacks perhaps should be in package code,
% but impacts on |add_to_callback|.
% If a default function is not required, it may be declared as |false|.
% First we need a list of user callbacks.
%    \begin{macrocode}
local user_callbacks_defaults = {
  pre_mlist_to_hlist_filter = list_handler_default,
  mlist_to_hlist = node.mlist_to_hlist,
  post_mlist_to_hlist_filter = list_handler_default,
}
%    \end{macrocode}
%
% \begin{macro}{create_callback}
% \changes{v1.0a}{2015/09/24}{Function added}
% \changes{v1.0i}{2015/11/29}{Check name is not nil in error message (PHG)}
% \changes{v1.0k}{2015/12/02}{Give more specific error messages (PHG)}
% \changes{v1.1l}{2020/02/02}{Provide proper fallbacks for user-defined callbacks without user-provided default handler}
%   The allocator itself.
%    \begin{macrocode}
local function create_callback(name, ctype, default)
  local ctype_id = types[ctype]
  if not name  or name  == ""
  or not ctype_id
  then
    luatexbase_error("Unable to create callback:\n" ..
                     "valid callback name and type required")
  end
  if callbacktypes[name] then
    luatexbase_error("Unable to create callback `" .. name ..
                     "':\ncallback is already defined")
  end
  default = default or defaults[ctype_id]
  if not default then
    luatexbase_error("Unable to create callback `" .. name ..
                     "':\ndefault is required for `" .. ctype ..
                     "' callbacks")
  elseif type (default) ~= "function" then
    luatexbase_error("Unable to create callback `" .. name ..
                     "':\ndefault is not a function")
  end
  user_callbacks_defaults[name] = default
  callbacktypes[name] = ctype_id
end
luatexbase.create_callback = create_callback
%    \end{macrocode}
% \end{macro}
%
% \begin{macro}{call_callback}
% \changes{v1.0a}{2015/09/24}{Function added}
% \changes{v1.0i}{2015/11/29}{Check name is not nil in error message (PHG)}
% \changes{v1.0k}{2015/12/02}{Give more specific error messages (PHG)}
%  Call a user defined callback. First check arguments.
%    \begin{macrocode}
local function call_callback(name,...)
  if not name or name == "" then
    luatexbase_error("Unable to create callback:\n" ..
                     "valid callback name required")
  end
  if user_callbacks_defaults[name] == nil then
    luatexbase_error("Unable to call callback `" .. name
                     .. "':\nunknown or empty")
   end
  local l = callbacklist[name]
  local f
  if not l then
    f = user_callbacks_defaults[name]
  else
    f = handlers[callbacktypes[name]](name)
  end
  return f(...)
end
luatexbase.call_callback=call_callback
%    \end{macrocode}
% \end{macro}
%
% \begin{macro}{add_to_callback}
% \changes{v1.0a}{2015/09/24}{Function added}
%   Add a function to a callback. First check arguments.
% \changes{v1.0k}{2015/12/02}{Give more specific error messages (PHG)}
%    \begin{macrocode}
local function add_to_callback(name, func, description)
  if not name or name == "" then
    luatexbase_error("Unable to register callback:\n" ..
                     "valid callback name required")
  end
  if not callbacktypes[name] or
    type(func) ~= "function" or
    not description or
    description == "" then
    luatexbase_error(
      "Unable to register callback.\n\n"
        .. "Correct usage:\n"
        .. "add_to_callback(<callback>, <function>, <description>)"
    )
  end
%    \end{macrocode}
%   Then test if this callback is already in use. If not, initialise its list
%   and register the proper handler.
%    \begin{macrocode}
  local l = callbacklist[name]
  if l == nil then
    l = { }
    callbacklist[name] = l
%    \end{macrocode}
% If it is not a user defined callback use the primitive callback register.
%    \begin{macrocode}
    if user_callbacks_defaults[name] == nil then
      callback_register(name, handlers[callbacktypes[name]](name))
    end
  end
%    \end{macrocode}
%  Actually register the function and give an error if more than one
%  |exclusive| one is registered.
%    \begin{macrocode}
  local f = {
    func        = func,
    description = description,
  }
  local priority = #l + 1
  if callbacktypes[name] == exclusive then
    if #l == 1 then
      luatexbase_error(
        "Cannot add second callback to exclusive function\n`" ..
        name .. "'")
    end
  end
  table.insert(l, priority, f)
%    \end{macrocode}
%  Keep user informed.
%    \begin{macrocode}
  luatexbase_log(
    "Inserting `" .. description .. "' at position "
      .. priority .. " in `" .. name .. "'."
  )
end
luatexbase.add_to_callback = add_to_callback
%    \end{macrocode}
% \end{macro}
%
% \begin{macro}{remove_from_callback}
% \changes{v1.0a}{2015/09/24}{Function added}
% \changes{v1.0k}{2015/12/02}{adjust initialization of cb local (PHG)}
% \changes{v1.0k}{2015/12/02}{Give more specific error messages (PHG)}
% \changes{v1.1m}{2020/03/07}{Do not call callback.register for user-defined callbacks}
%   Remove a function from a callback. First check arguments.
%    \begin{macrocode}
local function remove_from_callback(name, description)
  if not name or name == "" then
    luatexbase_error("Unable to remove function from callback:\n" ..
                     "valid callback name required")
  end
  if not callbacktypes[name] or
    not description or
    description == "" then
    luatexbase_error(
      "Unable to remove function from callback.\n\n"
        .. "Correct usage:\n"
        .. "remove_from_callback(<callback>, <description>)"
    )
  end
  local l = callbacklist[name]
  if not l then
    luatexbase_error(
      "No callback list for `" .. name .. "'\n")
  end
%    \end{macrocode}
%  Loop over the callback's function list until we find a matching entry.
%  Remove it and check if the list is empty: if so, unregister the
%   callback handler.
%    \begin{macrocode}
  local index = false
  for i,j in ipairs(l) do
    if j.description == description then
      index = i
      break
    end
  end
  if not index then
    luatexbase_error(
      "No callback `" .. description .. "' registered for `" ..
      name .. "'\n")
  end
  local cb = l[index]
  table.remove(l, index)
  luatexbase_log(
    "Removing  `" .. description .. "' from `" .. name .. "'."
  )
  if #l == 0 then
    callbacklist[name] = nil
    if user_callbacks_defaults[name] == nil then
      callback_register(name, nil)
    end
  end
  return cb.func,cb.description
end
luatexbase.remove_from_callback = remove_from_callback
%    \end{macrocode}
% \end{macro}
%
% \begin{macro}{in_callback}
% \changes{v1.0a}{2015/09/24}{Function added}
% \changes{v1.0h}{2015/11/27}{Guard against undefined list latex/4445}
%   Look for a function description in a callback.
%    \begin{macrocode}
local function in_callback(name, description)
  if not name
    or name == ""
    or not callbacklist[name]
    or not callbacktypes[name]
    or not description then
      return false
  end
  for _, i in pairs(callbacklist[name]) do
    if i.description == description then
      return true
    end
  end
  return false
end
luatexbase.in_callback = in_callback
%    \end{macrocode}
% \end{macro}
%
% \begin{macro}{disable_callback}
% \changes{v1.0a}{2015/09/24}{Function added}
%   As we subvert the engine interface we need to provide a way to access
%   this functionality.
%    \begin{macrocode}
local function disable_callback(name)
  if(callbacklist[name] == nil) then
    callback_register(name, false)
  else
    luatexbase_error("Callback list for " .. name .. " not empty")
  end
end
luatexbase.disable_callback = disable_callback
%    \end{macrocode}
% \end{macro}
%
% \begin{macro}{callback_descriptions}
% \changes{v1.0a}{2015/09/24}{Function added}
% \changes{v1.0h}{2015/11/27}{Match test in in-callback latex/4445}
%   List the descriptions of functions registered for the given callback.
%    \begin{macrocode}
local function callback_descriptions (name)
  local d = {}
  if not name
    or name == ""
    or not callbacklist[name]
    or not callbacktypes[name]
    then
    return d
  else
  for k, i in pairs(callbacklist[name]) do
    d[k]= i.description
    end
  end
  return d
end
luatexbase.callback_descriptions =callback_descriptions
%    \end{macrocode}
% \end{macro}
%
% \begin{macro}{uninstall}
% \changes{v1.0e}{2015/10/02}{Function added}
%   Unlike at the \TeX{} level, we have to provide a back-out mechanism here
%   at the same time as the rest of the code. This is not meant for use by
%   anything other than \textsf{latexrelease}: as such this is
%   \emph{deliberately} not documented for users!
%    \begin{macrocode}
local function uninstall()
  module_info(
    "luatexbase",
    "Uninstalling kernel luatexbase code"
  )
  callback.register = callback_register
  luatexbase = nil
end
luatexbase.uninstall = uninstall
%    \end{macrocode}
% \end{macro}
% \begin{macro}{mlist_to_hlist}
% \changes{v1.1l}{2020/02/02}{|pre/post_mlist_to_hlist| added}
%   To emulate these callbacks, the ``real'' |mlist_to_hlist| is replaced by a
%   wrapper calling the wrappers before and after.
%    \begin{macrocode}
callback_register("mlist_to_hlist", function(head, display_type, need_penalties)
  local current = call_callback("pre_mlist_to_hlist_filter", head, display_type, need_penalties)
  if current == false then
    flush_list(head)
    return nil
  end
  current = call_callback("mlist_to_hlist", current, display_type, need_penalties)
  local post = call_callback("post_mlist_to_hlist_filter", current, display_type, need_penalties)
  if post == false then
    flush_list(current)
    return nil
  end
  return post
end)
%    \end{macrocode}
% \end{macro}
% \endgroup
%
%    \begin{macrocode}
%</lua>
%    \end{macrocode}
%
% Reset the catcode of |@|.
%    \begin{macrocode}
%<tex>\catcode`\@=\etatcatcode\relax
%    \end{macrocode}
%
%
% \Finale
|. This inputs |ltluatex.tex| which inputs
% |etex.src| (or |etex.sty| if used with \LaTeX)
% if it is not already input, and then defines some internal commands to
% allow the \textsf{ltluatex} interface to be defined.
%
% The \textsf{luatexbase} package interface may also be used in plain \TeX,
% as before, by inputting the package |\input luatexbase.sty|. The new
% version of \textsf{luatexbase} is based on this \textsf{ltluatex}
% code but implements a compatibility layer providing the interface
% of the original package.
%
% \section{Lua functionality}
%
% \begingroup
%
% \begingroup\lccode`~=`_
% \lowercase{\endgroup\let~}_
% \catcode`_=12
%
% \subsection{Allocators in Lua}
%
% \DescribeMacro{new_attribute}
% |luatexbase.new_attribute(|\meta{attribute}|)|\\
% Returns an allocation number for the \meta{attribute}, indexed from~$1$.
% The attribute will be initialised with the marker value |-"7FFFFFFF|
% (`unset'). The attribute allocation sequence is shared with the \TeX{}
% code but this function does \emph{not} define a token using
% |\attributedef|.
% The attribute name is recorded in the |attributes| table. A
% metatable is provided so that the table syntax can be used
% consistently for attributes declared in \TeX\ or Lua.
%
% \noindent
% \DescribeMacro{new_whatsit}
% |luatexbase.new_whatsit(|\meta{whatsit}|)|\\
% Returns an allocation number for the custom \meta{whatsit}, indexed from~$1$.
%
% \noindent
% \DescribeMacro{new_bytecode}
% |luatexbase.new_bytecode(|\meta{bytecode}|)|\\
% Returns an allocation number for a bytecode register, indexed from~$1$.
% The optional \meta{name} argument is just used for logging.
%
% \noindent
% \DescribeMacro{new_chunkname}
% |luatexbase.new_chunkname(|\meta{chunkname}|)|\\
% Returns an allocation number for a Lua chunk name for use with
% |\directlua| and |\latelua|, indexed from~$1$.
% The number is returned and also \meta{name} argument is added to the
% |lua.name| array at that index.
%
% \begin{sloppypar}
% \noindent
% \DescribeMacro{new_luafunction}
% |luatexbase.new_luafunction(|\meta{functionname}|)|\\
% Returns an allocation number for a lua function for use
% with |\luafunction|, |\lateluafunction|, and |\luadef|,
% indexed from~$1$. The optional \meta{functionname} argument
% is just used for logging.
% \end{sloppypar}
%
% These functions all require access to a named \TeX{} count register
% to manage their allocations. The standard names are those defined
% above for access from \TeX{}, \emph{e.g.}~\string\e@alloc@attribute@count,
% but these can be adjusted by defining the variable
% \texttt{\meta{type}\_count\_name} before loading |ltluatex.lua|, for example
% \begin{verbatim}
% local attribute_count_name = "attributetracker"
% require("ltluatex")
% \end{verbatim}
% would use a \TeX{} |\count| (|\countdef|'d token) called |attributetracker|
% in place of \string\e@alloc@attribute@count.
%
% \subsection{Lua access to \TeX{} register numbers}
%
% \DescribeMacro{registernumber}
% |luatexbase.registernumer(|\meta{name}|)|\\
% Sometimes (notably in the case of Lua attributes) it is necessary to
% access a register \emph{by number} that has been allocated by \TeX{}.
% This package provides a function to look up the relevant number
% using Lua\TeX{}'s internal tables. After for example
% |\newattribute\myattrib|, |\myattrib| would be defined by (say)
% |\myattrib=\attribute15|.  |luatexbase.registernumer("myattrib")|
% would then return the register number, $15$ in this case. If the string passed
% as argument does not correspond to a token defined by |\attributedef|,
% |\countdef| or similar commands, the Lua value |false| is returned.
%
% As an example, consider the input:
%\begin{verbatim}
% \newcommand\test[1]{%
% \typeout{#1: \expandafter\meaning\csname#1\endcsname^^J
% \space\space\space\space
% \directlua{tex.write(luatexbase.registernumber("#1") or "bad input")}%
% }}
%
% \test{undefinedrubbish}
%
% \test{space}
%
% \test{hbox}
%
% \test{@MM}
%
% \test{@tempdima}
% \test{@tempdimb}
%
% \test{strutbox}
%
% \test{sixt@@n}
%
% \attrbutedef\myattr=12
% \myattr=200
% \test{myattr}
%
%\end{verbatim}
%
% If the demonstration code is processed with Lua\LaTeX{} then the following
% would be produced in the log and terminal output.
%\begin{verbatim}
% undefinedrubbish: \relax
%      bad input
% space: macro:->
%      bad input
% hbox: \hbox
%      bad input
% @MM: \mathchar"4E20
%      20000
% @tempdima: \dimen14
%      14
% @tempdimb: \dimen15
%      15
% strutbox: \char"B
%      11
% sixt@@n: \char"10
%      16
% myattr: \attribute12
%      12
%\end{verbatim}
%
% Notice how undefined commands, or commands unrelated to registers
% do not produce an error, just return |false| and so print
% |bad input| here. Note also that commands defined by |\newbox| work and
% return the number of the box register even though the actual command
% holding this number is a |\chardef| defined token (there is no
% |\boxdef|).
%
% \subsection{Module utilities}
%
% \DescribeMacro{provides_module}
% |luatexbase.provides_module(|\meta{info}|)|\\
% This function is used by modules to identify themselves; the |info| should be
% a table containing information about the module. The required field
% |name| must contain the name of the module. It is recommended to provide a
% field |date| in the usual \LaTeX{} format |yyyy/mm/dd|. Optional fields
% |version| (a string) and |description| may be used if present. This
% information will be recorded in the log. Other fields are ignored.
%
% \noindent
% \DescribeMacro{module_info}
% \DescribeMacro{module_warning}
% \DescribeMacro{module_error}
% |luatexbase.module_info(|\meta{module}, \meta{text}|)|\\
% |luatexbase.module_warning(|\meta{module}, \meta{text}|)|\\
% |luatexbase.module_error(|\meta{module}, \meta{text}|)|\\
% These functions are similar to \LaTeX{}'s |\PackageError|, |\PackageWarning|
% and |\PackageInfo| in the way they format the output.  No automatic line
% breaking is done, you may still use |\n| as usual for that, and the name of
% the package will be prepended to each output line.
%
% Note that |luatexbase.module_error| raises an actual Lua error with |error()|,
% which currently means a call stack will be dumped. While this may not
% look pretty, at least it provides useful information for tracking the
% error down.
%
% \subsection{Callback management}
%
% \noindent
% \DescribeMacro{add_to_callback}
% |luatexbase.add_to_callback(|^^A
% \meta{callback}, \meta{function}, \meta{description}|)|
% Registers the \meta{function} into the \meta{callback} with a textual
% \meta{description} of the function. Functions are inserted into the callback
% in the order loaded.
%
% \noindent
% \DescribeMacro{remove_from_callback}
% |luatexbase.remove_from_callback(|\meta{callback}, \meta{description}|)|
% Removes the callback function with \meta{description} from the \meta{callback}.
% The removed function and its description
% are returned as the results of this function.
%
% \noindent
% \DescribeMacro{in_callback}
% |luatexbase.in_callback(|\meta{callback}, \meta{description}|)|
% Checks if the \meta{description} matches one of the functions added
% to the list for the \meta{callback}, returning a boolean value.
%
% \noindent
% \DescribeMacro{disable_callback}
% |luatexbase.disable_callback(|\meta{callback}|)|
% Sets the \meta{callback} to \texttt{false} as described in the Lua\TeX{}
% manual for the underlying \texttt{callback.register} built-in. Callbacks
% will only be set to false (and thus be skipped entirely) if there are
% no functions registered using the callback.
%
% \noindent
% \DescribeMacro{callback_descriptions}
% A list of the descriptions of functions registered to the specified
% callback is returned. |{}| is returned if there are no functions registered.
%
% \noindent
% \DescribeMacro{create_callback}
% |luatexbase.create_callback(|\meta{name},\meta{type},\meta{default}|)|
% Defines a user defined callback. The last argument is a default
% function or |false|.
%
% \noindent
% \DescribeMacro{call_callback}
% |luatexbase.call_callback(|\meta{name},\ldots|)|
% Calls a user defined callback with the supplied arguments.
%
% \noindent
% \DescribeMacro{declare_callback_rule}
% |luatexbase.declare_callback_rule(|\meta{name}, \meta{first}, \meta{relation}, \meta{second}|)|
% Adds an ordering constraint between two callback functions for callback \meta{name}.
%
% The kind of constraint added depends on \meta{relation}:
% \begin{description}
%   \item[before] The callback function with description \meta{first} will be
%     executed before the function with description \meta{second}.
%   \item[after] The callback function with description \meta{first} will be
%     executed after the function with description \meta{second}.
%   \item[incompatible-warning] When both a callback function with description \meta{first}
%     and with description \meta{second} is registered, then a warning is printed when
%     the callback is executed.
%   \item[incompatible-error] When both a callback function with description \meta{first}
%     and with description \meta{second} is registered, then an error is printed when
%     the callback is executed.
%   \item[unrelated] Any previously declared callback rule between \meta{first}
%     and \meta{second} gets disabled.
% \end{description}
% Every call to \texttt{declare_callback_rule} with a specific callback \meta{name}
% and descriptions \meta{first} and \meta{second} overwrites all previous calls with
% same callback and descriptions.
%
% The callback functions do not have to be registered yet when the functions is called.
% Ony the constraints for which both callback descriptions refer to callbacks
% registered at the time the callback is called will have an effect.
%
% \endgroup
%
% \MaybeStop{}
%
% \section{Implementation}
%
%    \begin{macrocode}
%<*2ekernel|tex|latexrelease>
%<2ekernel|latexrelease>\ifx\directlua\@undefined\else
%    \end{macrocode}
%
%
% \changes{v1.0j}{2015/12/02}{Remove nonlocal iteration variables (PHG)}
% \changes{v1.0j}{2015/12/02}{Assorted typos fixed (PHG)}
% \changes{v1.0j}{2015/12/02}{Remove unreachable code after calls to error() (PHG)}
% \subsection{Minimum Lua\TeX{} version}
%
% Lua\TeX{} has changed a lot over time. In the kernel support for ancient
% versions is not provided: trying to build a format with a very old binary
% therefore gives some information in the log and loading stops. The cut-off
% selected here relates to the tree-searching behaviour of |require()|:
% from version~0.60, Lua\TeX{} will correctly find Lua files in the |texmf|
% tree without `help'.
%    \begin{macrocode}
%<latexrelease>\IncludeInRelease{2015/10/01}
%<latexrelease>                 {\newluafunction}{LuaTeX}%
\ifnum\luatexversion<60 %
  \wlog{***************************************************}
  \wlog{* LuaTeX version too old for ltluatex support *}
  \wlog{***************************************************}
  \expandafter\endinput
\fi
%    \end{macrocode}
%
% \changes{v1.1n}{2020/06/10}{Define \cs{@gobble}/\cs{@firstofone} even for \LaTeX\ to allow early loading.}
% Two simple \LaTeX\ macros from |ltdefns.dtx| have to be defined here
% because ltdefns.dtx is not loaded yet when ltluatex.dtx is executed.
%    \begin{macrocode}
\long\def\@gobble#1{}
\long\def\@firstofone#1{#1}
%    \end{macrocode}
%
% \subsection{Older \LaTeX{}/Plain \TeX\ setup}
%
%    \begin{macrocode}
%<*tex>
%    \end{macrocode}
%
% Older \LaTeX{} formats don't have the primitives with `native' names:
% sort that out. If they already exist this will still be safe.
%    \begin{macrocode}
\directlua{tex.enableprimitives("",tex.extraprimitives("luatex"))}
%    \end{macrocode}
%
%    \begin{macrocode}
\ifx\e@alloc\@undefined
%    \end{macrocode}
%
% In pre-2014 \LaTeX{}, or plain \TeX{}, load |etex.{sty,src}|.
%    \begin{macrocode}
  \ifx\documentclass\@undefined
    \ifx\loccount\@undefined
      % \iffalse meta-comment
%
% Copyright 1997, 1998, 2008 2015 2016 LaTeX Project and Peter Breitenlohner.
% 
% This file (etex.sty) may be distributed and/or modified under the
% conditions of the LaTeX Project Public License, either version 1.3 of
% this license or (at your option) any later version.  The latest
% version of this license is in
%   http://www.latex-project.org/lppl.txt
% and version 1.3 or later is part of all distributions of LaTeX
% version 2003/12/01 or later.
% 
% This work has the LPPL maintenance status "maintained".
% 
% The Current Maintainer of this work is David Carlisle.
% https://github.com/davidcarlisle/dpctex/issues
% \fi

\NeedsTeXFormat{LaTeX2e}
\ProvidesPackage{etex}
%        [1997/08/12 v0.1 eTeX basic definition package (DPC)]
%        [1998/03/26 v2.0 eTeX basic definition package (PEB)]
%        [2015/03/02 v2.1 eTeX basic definition package (PEB,DPC)]
%        [2015/07/06 v2.2 eTeX basic definition package (PEB,DPC)]
%        [2015/07/08 v2.3 eTeX basic definition package (PEB,DPC)]
%        [2015/09/02 v2.4 eTeX basic definition package (PEB,DPC)]
%        [2016/01/07 v2.5 eTeX basic definition package (PEB,DPC)]
%        [2016/01/11 v2.6 eTeX basic definition package (PEB,DPC)]
         [2016/08/01 v2.7 eTeX basic definition package (PEB,DPC)]

%%%%%%%%%%%%%%%%%%%%%%%%%%%%%%%%%%%%%%%%%%%%%%%%%%%%%%%%%%%%%%%%%%%%%%%%

%% A basic interface to some etex primitives, closely modeled on
%% etex.src and etexdefs.lib provided by the core etex team.

%% The etex.src `module' system is not copied here, the standard
%% LaTeX package option mechanism is used instead,
%% however the package options match the module names.
%% (Currently grouptypes, interactionmodes, nodetypes, iftypes.)
%% The individual type names are different too: We use, e.g.,
%%
%% `\bottomleveltype' and `\simplegrouptype' instead of
%% `\grouptypes{bottomlevel}' and `\grouptypes{simple}'.

%%%%%%%%%%%%%%%%%%%%%%%%%%%%%%%%%%%%%%%%%%%%%%%%%%%%%%%%%%%%%%%%%%%%%%%%

%% Other Comments...

%% The names of the `interactionmodes' are not too good.
%% In particular \scroll and \batch are likely to clash with existing
%% uses. These names have been changed into \batchinteractionmode,
%% \scrollinteractionmode etc.
%% Similarly, the names of the `groupetypes' have been changed, in
%% particular \mathgroup would conflict with the LaTeX kernel.

%% \etex logo could have the same trick as \LaTeXe to pick up a bold
%% epsilon when needed. (Not done here, I hate wasting tokens on logos.)
%% This version does have a \m@th not in the original.

%% The \globcountvector, \loccountvector, etc. allocation macros are
%% not (yet) implemented.

%% Currently if run on a standard TeX, the package generates an error.
%% Perhaps it should instead load some code to try to fake
%% the new etex primitives in that case???
%% Likewise, the package generates an error when used with e-TeX V 1

%% The etex.src language mechanism is not copied here. That facility
%% does not use any of the etex features. LaTeX should be customised
%% using the same hyphen.cfg mechanism as for a format built with a
%% standard TeX.

%% David Carlisle

%% Upgraded for e-TeX V 2.0
%% Peter Breitenlohner

%%%%%%%%%%%%%%%%%%%%%%%%%%%%%%%%%%%%%%%%%%%%%%%%%%%%%%%%%%%%%%%%%%%%%%%%



\ifx\eTeXversion\@undefined
  \PackageError{etex}
    {This package may only be run using an\MessageBreak
     etex in extended mode}
    {Perhaps you forgot the `*' when making the format with (e)initex.%
    }
\fi

\ifnum\eTeXversion<2
  \PackageError{etex}
    {This package requires e-TeX V 2}
    {You are probably using the obsolete e-TeX V 1.%
    }
\fi

% 2.2
% Check if the new latex 2015/01/01 allocation is already using
% extended reisters. If so it is too late to change allocation scheme.
% Older versions of LaTeX would have given an error when the classic
% TeX registers were all allocated, but newer formats allocate from
% the extended range, so usually this package is not needed.
\@tempswafalse
\ifnum\count10>\@cclv\@tempswatrue\else
\ifnum\count11>\@cclv\@tempswatrue\else
\ifnum\count12>\@cclv\@tempswatrue\else
\ifnum\count13>\@cclv\@tempswatrue\else
\ifnum\count14>\@cclv\@tempswatrue\else
\ifnum\count15>\@cclv\@tempswatrue
\fi\fi\fi\fi\fi\fi

\if@tempswa
\PackageWarningNoLine{etex}{%
Extended allocation already in use.\MessageBreak
etex.sty code will not be used.\MessageBreak
To force etex package to load, add\MessageBreak
\string\RequirePackage{etex}\MessageBreak
at the start of the document}

% 2.5 define the global allocation to be the standard ones
% as extended allocation is already in use. Helps with
% compatibility with some packages that use these commands 
% after loading etex.
% 2.6 avoid error from outer if used with (e)plain
\expandafter\let\csname globcount\expandafter\endcsname
                \csname newcount\endcsname
\expandafter\let\csname globdimen\expandafter\endcsname
                \csname newdimen\endcsname
\expandafter\let\csname globskip\expandafter\endcsname
                \csname newskip\endcsname
\expandafter\let\csname globmuskip\expandafter\endcsname
                \csname newmuskip\endcsname
\expandafter\let\csname globtoks\expandafter\endcsname
                \csname newtoks\endcsname
\expandafter\let\csname globmarks\expandafter\endcsname
                \csname newmarks\endcsname
% end of 2.5/2.6 change

\expandafter\endinput\fi

% End of 2.2 addition.

% 2.3 move option handling after the above error checks.
\DeclareOption{grouptypes}{\catcode`\G=9}
\DeclareOption{interactionmodes}{\catcode`\I=9}
\DeclareOption{nodetypes}{\catcode`\N=9}
\DeclareOption{iftypes}{\catcode`\C=9}
\DeclareOption{localalloclog}{\let\et@xwlog\wlog} % the default
\DeclareOption{localallocnolog}{\let\et@xwlog\@gobble} % be quiet
\DeclareOption{localallocshow}{\let\et@xwlog\typeout} % debugging
% End of 2.3 addition.

% v2.7
% \extrafloats does not work with this package
% but make it give a sensible error, not mis-parse \ifnum.
%
% Note that using \extrafloats earlier might not be safe as
% it could in principle clash with registers used for local allocation.
% However it probably works (as local allocation is used locally...).
% A better fix would be not to load this package with current LaTeX.
% This current etex package is just provided to force the old behaviour
% and such documents should not be using new features such as \extrafloats).
\ifdefined\extrafloats
\def\extrafloats#1{%
  \PackageError{etex}{%
    \noexpand\extrafloats is incompatible with etex.sty allocation.\MessageBreak
    Try using \noexpand\extrafloats before loading etex}%
    \@ehc}
\fi
% end of v2.7 change

\def\eTeX{%
  $\m@th\varepsilon$-\TeX}

\def\tracingall{%
  \tracingcommands\thr@@        % etex
  \tracingstats\tw@
  \tracingpages\@ne
  \tracinglostchars\tw@         % etex
  \tracingmacros\tw@
  \tracingparagraphs\@ne
  \tracingrestores\@ne
  \tracinggroups\@ne            % etex
  \tracingifs\@ne               % etex
  \tracingscantokens\@ne        % etex
  \tracingnesting\@ne           % etex
  \tracingassigns\@ne           % etex
  \errorcontextlines\maxdimen
  \showoutput}

\def\loggingall{%
  \tracingall
  \tracingonline\z@}

\def\tracingnone{%
  \tracingonline\z@
  \showboxdepth\m@ne
  \showboxbreadth\m@ne
  \tracingoutput\z@
  \errorcontextlines\m@ne
  \tracingassigns\z@
  \tracingnesting\z@
  \tracingscantokens\z@
  \tracingifs\z@
  \tracinggroups\z@
  \tracingrestores\z@
  \tracingparagraphs\z@
  \tracingmacros\z@
  \tracinglostchars\@ne
  \tracingpages\z@
  \tracingstats\z@
  \tracingcommands\z@}

%% Register allocation
%% We have to adjust the Plain TeX / LaTeX register allocation counts
%% for our slightly modified book-keeping, but first we allocate our
%% insertion counter \et@xins, because \insc@ount of Plain TeX / LaTeX
%% will be used differently.

\newcount\et@xins

\advance\count10 by 1 % \count10=23 % allocates \count registers 23, 24, ...
\advance\count11 by 1 % \count11=10 % allocates \dimen registers 10, 11, ...
\advance\count12 by 1 % \count12=10 % allocates \skip registers 10, 11, ...
\advance\count13 by 1 % \count13=10 % allocates \muskip registers 10, 11, ...
\advance\count14 by 1 % \count14=10 % allocates \box registers 10, 11, ...
\advance\count15 by 1 % \count15=10 % allocates \toks registers 10, 11, ...
\advance\count16 by 1 % \count16=0 % allocates input streams 0, 1, ...
\advance\count17 by 1 % \count17=0 % allocates output streams 0, 1, ...
\advance\count18 by 1 % \count18=4 % allocates math families 4, 5, ...
\advance\count19 by 1 % \count19=0 % allocates \language codes 0, 1, ...

\et@xins=\insc@unt % \et@xins=255 % allocates insertions 254, 253, ...


%% To ensure working in LaTeX 2015 release do define \newcount etc
%% with their pre 2015 LaTeX definitions
\def\newcount{\alloc@0\count\countdef\insc@unt}
\def\newdimen{\alloc@1\dimen\dimendef\insc@unt}
\def\newskip{\alloc@2\skip\skipdef\insc@unt}
\def\newmuskip{\alloc@3\muskip\muskipdef\@cclvi}
\def\newbox{\alloc@4\box\chardef\insc@unt}
\def\newtoks{\alloc@5\toks\toksdef\@cclvi}
\def\newread{\alloc@6\read\chardef\sixt@@n}
\def\newwrite{\alloc@7\write\chardef\sixt@@n}
\def\new@mathgroup{\alloc@8\mathgroup\chardef\sixt@@n}
\let\newfam\new@mathgroup
\def\newlanguage{\alloc@9\language\chardef\@cclvi}

%% When the normal register pool for \count, \dimen, \skip, \muskip,
%% \box, or \toks registers is exhausted, we switch to the extended pool.

\def\alloc@#1#2#3#4#5%
 {\ifnum\count1#1<#4% make sure there's still room
    \allocationnumber\count1#1
    \global\advance\count1#1\@ne
    \global#3#5\allocationnumber
    \wlog{\string#5=\string#2\the\allocationnumber}%
  \else\ifnum#1<6
    \begingroup \escapechar\m@ne
    \expandafter\alloc@@\expandafter{\string#2}#5%
  \else\errmessage{No room for a new #2}\fi\fi
 }

%% The \expandafter construction used here allows the generation of
%% \newcount and \globcount from #1=count.

\def\alloc@@#1#2%
 {\endgroup % restore \escapechar
  \wlog{Normal \csname#1\endcsname register pool exhausted,
    switching to extended pool.}%
  \global\expandafter\let
    \csname new#1\expandafter\endcsname
    \csname glob#1\endcsname
  \csname new#1\endcsname#2%
 }

%% We do change the LaTeX definition of \newinsert

\def\newinsert#1{% make sure there's still room for ...
  \ch@ck0\et@xins\count{% ... a \count, ...
    \ch@ck1\et@xins\dimen{% ... \dimen, ...
      \ch@ck2\et@xins\skip{% ... \skip, ...
        \ch@ck4\et@xins\box{% ... and \box register
  \global\advance\et@xins\m@ne
  \unless\ifnum\insc@unt<\et@xins \global\insc@unt\et@xins \fi
  \allocationnumber\et@xins
  \global\chardef#1\allocationnumber
  \wlog{\string#1=\string\insert\the\allocationnumber}}}}}}

\def\ch@ck#1#2#3#4%
 {\ifnum\count1#1<#2#4\else\errmessage{No room for a new #3}\fi}

%% And we define \reserveinserts, so that you can say \reserveinserts{17}
%% in order to reserve room for up to 17 additional insertion classes, that
%% will not be taken away by \newcount, \newdimen, \newskip, or \newbox.

% 2.4 Remove \outer to match LaTeX allocations
% which are never \outer unlike plain TeX.

%\outer
\def\reserveinserts#1%
 {\global\insc@unt\numexpr \et@xins \ifnum#1>\z@ -#1\fi \relax}

% Now, we define \globcount, \globbox, etc., so that you can say
% \globcount\foo and \foo will be defined (with \countdef) to be the
% next count register from the vastly larger but somewhat less efficient
% extended register pool. We also define \loccount, etc., but these
% register definitions are local to the current group.

\count260=277 % globally allocates \count registers 277, 278, ...
\count261=256 % globally allocates \dimen registers 256, 257, ...
\count262=256 % globally allocates \skip registers 256, 257, ...
\count263=256 % globally allocates \muskip registers 256, 257, ...
\count264=256 % globally allocates \box registers 256, 257, ...
\count265=256 % globally allocates \toks registers 256, 257, ...
\count266=1 % globally allocates \marks classes 1, 2, ...

\count270=32768 % locally allocates \count registers 32767, 32766, ...
\count271=32768 % ditto for \dimen registers
\count272=32768 % ditto for \skip registers
\count273=32768 % ditto for \muskip registers
\count274=32768 % ditto for \box registers
\count275=32768 % ditto for \toks registers
\count276=32768 % ditto for \marks classes

% \count registers 256-259 and 267-269 are not (yet) used

% \def \et@xglob #1#2#3#4% <offset>, <type>, <method>, <register>
% \def \et@xloc #1#2#3#4% <offset>, <type>, <method>, <register>

\def \globcount  {\et@xglob 0\count  \countdef}
\def \loccount   {\et@xloc  0\count  \countdef}
\def \globdimen  {\et@xglob 1\dimen  \dimendef}
\def \locdimen   {\et@xloc  1\dimen  \dimendef}
\def \globskip   {\et@xglob 2\skip   \skipdef}
\def \locskip    {\et@xloc  2\skip   \skipdef}
\def \globmuskip {\et@xglob 3\muskip \muskipdef}
\def \locmuskip  {\et@xloc  3\muskip \muskipdef}
\def \globbox    {\et@xglob 4\box    \mathchardef}
\def \locbox     {\et@xloc  4\box    \mathchardef}
\def \globtoks   {\et@xglob 5\toks   \toksdef}
\def \loctoks    {\et@xloc  5\toks   \toksdef}
\def \globmarks  {\et@xglob 6\marks  \mathchardef}
\def \locmarks   {\et@xloc  6\marks  \mathchardef}

\let\newmarks=\globmarks %% this used to be \newmark for e-TeX V 1.1

\def\et@xglob#1#2#3#4%
 {\et@xchk#1#2{% make sure there's still room
  \allocationnumber=\count26#1%
  \global\advance\count26#1\@ne
  \global#3#4\allocationnumber
  \wlog{\string#4=\string#2\the\allocationnumber}}%
 }

\def\et@xloc#1#2#3#4%
 {\et@xchk#1#2{% make sure there's still room
  \advance\count27#1by\m@ne
  \allocationnumber=\count27#1%
  #3#4=\allocationnumber
  \et@xwlog{\string#4=\string#2\the\allocationnumber\space(local)}}%
 }

%% The allocation messages for local allocations use \et@xwlog, such that
%% these messages can easily be switched on/off

\let\et@xwlog=\wlog

\def\et@xchk#1#2#3%
 {\ifnum\count26#1<\count27#1 #3\else\errmessage{No room for a new #2}\fi}

% Next we define \globcountblk, \loccountblk, etc., so that you can
% say \globcountblk\foo{17} and \foo will be defined (with \mathchardef)
% as the first (the zeroth?) of a block of 17 consecutive registers.
% Thus the user is intended to reference elements <\foo+0> to <\foo+n-1>,
% where n is the length of the block allocated.

% \def \et@xgblk #1#2#3#4% <offset>, <type>, <register>, <size>
% \def \et@xlblk #1#2#3#4% <offset>, <type>, <register>, <size>

\def\globcountblk  {\et@xgblk 0\count  }
\def\loccountblk   {\et@xlblk 0\count  }
\def\globdimenblk  {\et@xgblk 1\dimen  }
\def\locdimenblk   {\et@xlblk 1\dimen  }
\def\globskipblk   {\et@xgblk 2\skip   }
\def\locskipblk    {\et@xlblk 2\skip   }
\def\globmuskipblk {\et@xgblk 3\muskip }
\def\locmuskipblk  {\et@xlblk 3\muskip }
\def\globboxblk    {\et@xgblk 4\box    }
\def\locboxblk     {\et@xlblk 4\box    }
\def\globtoksblk   {\et@xgblk 5\toks   }
\def\loctoksblk    {\et@xlblk 5\toks   }
\def\globmarksblk  {\et@xgblk 6\marks  }
\def\locmarksblk   {\et@xlblk 6\marks  }

% \def\et@xchkblk#1#1#3#4% <offset>, <type>, <size>, <action>

\def\et@xgblk#1#2#3#4%
 {\et@xchkblk#1#2{#4}% make sure there's still room
   {\allocationnumber\count26#1%
    \global\advance\count26#1by#4%
    \global\mathchardef#3\allocationnumber
    \wlog{\string#3=\string#2blk{\number#4} at
      \the\allocationnumber}%
   }%
 }

\def\et@xlblk#1#2#3#4%
 {\et@xchkblk#1#2{#4}% make sure there's still room
   {\advance\count27#1-#4%
    \allocationnumber\count27#1%
    \mathchardef#3\allocationnumber
    \et@xwlog{\string#3=\string#2blk{\number#4} at
      \the\allocationnumber\space(local)}%
   }%
 }

\def\et@xchkblk#1#2#3#4%
 {\ifnum#3<\z@
    \errmessage{Negative register block size \number#3}%
  \else\ifnum\numexpr\count26#1+#3>\count27#1%
    \errmessage{No room for new #2block of size \number#3}%
  \else #4\fi \fi
 }

\catcode`\G=14
\catcode`\I=14
\catcode`\N=14
\catcode`\C=14

\ProcessOptions

%% Declare names for `grouptypes'

G \chardef \bottomleveltype       =  0 % for the outside world
G \chardef \simplegrouptype       =  1 % for local structure only
G \chardef \hboxgrouptype         =  2 % for `\hbox{}'
G \chardef \adjustedhboxgrouptype =  3 % for `\hbox{}' in vertical mode
G \chardef \vboxgrouptype         =  4 % for `\vbox{}'
G \chardef \vtopgrouptype         =  5 % for `\vtop{}'
G \chardef \aligngrouptype        =  6 % for `\halign{}', `\valign{}'
G \chardef \noaligngrouptype      =  7 % for `\noalign{}'
G \chardef \outputgrouptype       =  8 % for output routine
G \chardef \mathgrouptype         =  9 % for, e.g, `^{}'
G \chardef \discgrouptype         = 10 % for `\discretionary{}{}{}'
G \chardef \insertgrouptype       = 11 % for `\insert{}', `\vadjust{}'
G \chardef \vcentergrouptype      = 12 % for `\vcenter{}'
G \chardef \mathchoicegrouptype   = 13 % for `\mathchoice{}{}{}{}'
G \chardef \semisimplegrouptype   = 14 % for `\begingroup...\endgroup'
G \chardef \mathshiftgrouptype    = 15 % for `$...$'
G \chardef \mathleftgrouptype     = 16 % for `\left...\right'

%% Declare names for `interactionmodes'

I \chardef \batchinteractionmode     = 0 % omits all stops and omits terminal output
I \chardef \nonstopinteractionmode   = 1 % omits all stops
I \chardef \scrollinteractionmode    = 2 % omits error stops
I \chardef \errorstopinteractionmode = 3 % stops at every opportunity to interact

%% Declare names for `nodetypes'

N \chardef \charnode     =  0 % character nodes
N \chardef \hlistnode    =  1 % hlist nodes
N \chardef \vlistnode    =  2 % vlist nodes
N \chardef \rulenode     =  3 % rule nodes
N \chardef \insnode      =  4 % insertion nodes
N \chardef \marknode     =  5 % a mark node
N \chardef \adjustnode   =  6 % an adjust node
N \chardef \ligaturenode =  7 % a ligature node
N \chardef \discnode     =  8 % a discretionary node
N \chardef \whatsitnode  =  9 % special extension nodes
N \chardef \mathnode     = 10 % a math node
N \chardef \gluenode     = 11 % node that points to a glue specification
N \chardef \kernnode     = 12 % a kern node
N \chardef \penaltynode  = 13 % a penalty node
N \chardef \unsetnode    = 14 % an unset node
N \chardef \mathsnodes   = 15 % nodes that occur only in maths mode

%% Declare names for `iftypes'

C \chardef \charif     =  1 % \if
C \chardef \catif      =  2 % \ifcat
C \chardef \numif      =  3 % \ifnum
C \chardef \dimif      =  4 % \ifdim
C \chardef \oddif      =  5 % \ifodd
C \chardef \vmodeif    =  6 % \ifvmode
C \chardef \hmodeif    =  7 % \ifhmode
C \chardef \mmodeif    =  8 % \ifmmode
C \chardef \innerif    =  9 % \ifinner
C \chardef \voidif     = 10 % \ifvoid
C \chardef \hboxif     = 11 % \ifhbox
C \chardef \vboxif     = 12 % \ifvbox
C \chardef \xif        = 13 % \ifx
C \chardef \eofif      = 14 % \ifeof
C \chardef \trueif     = 15 % \iftrue
C \chardef \falseif    = 16 % \iffalse
C \chardef \caseif     = 17 % \ifcase
C \chardef \definedif  = 18 % \ifdefined
C \chardef \csnameif   = 19 % \ifcsname
C \chardef \fontcharif = 20 % \iffontchar

\catcode`\G=11
\catcode`\I=11
\catcode`\N=11
\catcode`\C=11

%
    \fi
    \catcode`\@=11 %
    \outer\expandafter\def\csname newfam\endcsname
                          {\alloc@8\fam\chardef\et@xmaxfam}
  \else
    \RequirePackage{etex}
    \expandafter\def\csname newfam\endcsname
                    {\alloc@8\fam\chardef\et@xmaxfam}
    \expandafter\let\expandafter\new@mathgroup\csname newfam\endcsname
  \fi
%    \end{macrocode}
%
% \subsubsection{Fixes to \texttt{etex.src}/\texttt{etex.sty}}
%
% These could and probably should be made directly in an
% update to |etex.src| which already has some Lua\TeX-specific
% code, but does not define the correct range for Lua\TeX.
%
% 2015-07-13 higher range in luatex.
%    \begin{macrocode}
\edef \et@xmaxregs {\ifx\directlua\@undefined 32768\else 65536\fi}
%    \end{macrocode}
% luatex/xetex also allow more math fam.
%    \begin{macrocode}
\edef \et@xmaxfam {\ifx\Umathcode\@undefined\sixt@@n\else\@cclvi\fi}
%    \end{macrocode}
%
%    \begin{macrocode}
\count 270=\et@xmaxregs % locally allocates \count registers
\count 271=\et@xmaxregs % ditto for \dimen registers
\count 272=\et@xmaxregs % ditto for \skip registers
\count 273=\et@xmaxregs % ditto for \muskip registers
\count 274=\et@xmaxregs % ditto for \box registers
\count 275=\et@xmaxregs % ditto for \toks registers
\count 276=\et@xmaxregs % ditto for \marks classes
%    \end{macrocode}
%
% and 256 or 16 fam. (Done above due to plain/\LaTeX\ differences in
% \textsf{ltluatex}.)
%    \begin{macrocode}
% \outer\def\newfam{\alloc@8\fam\chardef\et@xmaxfam}
%    \end{macrocode}
%
% End of proposed changes to \texttt{etex.src}
%
% \subsubsection{luatex specific settings}
%
% Switch to global cf |luatex.sty| to leave room for inserts
% not really needed for luatex but possibly most compatible
% with existing use.
%    \begin{macrocode}
\expandafter\let\csname newcount\expandafter\expandafter\endcsname
                \csname globcount\endcsname
\expandafter\let\csname newdimen\expandafter\expandafter\endcsname
                \csname globdimen\endcsname
\expandafter\let\csname newskip\expandafter\expandafter\endcsname
                \csname globskip\endcsname
\expandafter\let\csname newbox\expandafter\expandafter\endcsname
                \csname globbox\endcsname
%    \end{macrocode}
%
% Define|\e@alloc| as in \LaTeX{} (the existing macros in |etex.src| are
% hard to extend to further register types as they assume specific
% 26x and 27x count range). For compatibility the existing register
% allocation is not changed.
%
%    \begin{macrocode}
\chardef\e@alloc@top=65535
\let\e@alloc@chardef\chardef
%    \end{macrocode}
%
%    \begin{macrocode}
\def\e@alloc#1#2#3#4#5#6{%
  \global\advance#3\@ne
  \e@ch@ck{#3}{#4}{#5}#1%
  \allocationnumber#3\relax
  \global#2#6\allocationnumber
  \wlog{\string#6=\string#1\the\allocationnumber}}%
%    \end{macrocode}
%
%    \begin{macrocode}
\gdef\e@ch@ck#1#2#3#4{%
  \ifnum#1<#2\else
    \ifnum#1=#2\relax
      #1\@cclvi
      \ifx\count#4\advance#1 10 \fi
    \fi
    \ifnum#1<#3\relax
    \else
      \errmessage{No room for a new \string#4}%
    \fi
  \fi}%
%    \end{macrocode}
%
% Fix up allocations not to clash with |etex.src|.
%
%    \begin{macrocode}
\expandafter\csname newcount\endcsname\e@alloc@attribute@count
\expandafter\csname newcount\endcsname\e@alloc@ccodetable@count
\expandafter\csname newcount\endcsname\e@alloc@luafunction@count
\expandafter\csname newcount\endcsname\e@alloc@whatsit@count
\expandafter\csname newcount\endcsname\e@alloc@bytecode@count
\expandafter\csname newcount\endcsname\e@alloc@luachunk@count
%    \end{macrocode}
%
% End of conditional setup for plain \TeX\ / old \LaTeX.
%    \begin{macrocode}
\fi
%</tex>
%    \end{macrocode}
%
% \subsection{Attributes}
%
% \begin{macro}{\newattribute}
% \changes{v1.0a}{2015/09/24}{Macro added}
% \changes{v1.1q}{2020/08/02}{Move reset to $0$ inside conditional}
%   As is generally the case for the Lua\TeX{} registers we start here
%   from~$1$. Notably, some code assumes that |\attribute0| is never used so
%   this is important in this case.
%    \begin{macrocode}
\ifx\e@alloc@attribute@count\@undefined
  \countdef\e@alloc@attribute@count=258
  \e@alloc@attribute@count=\z@
\fi
\def\newattribute#1{%
  \e@alloc\attribute\attributedef
    \e@alloc@attribute@count\m@ne\e@alloc@top#1%
}
%    \end{macrocode}
% \end{macro}
%
% \begin{macro}{\setattribute}
% \begin{macro}{\unsetattribute}
%   Handy utilities.
%    \begin{macrocode}
\def\setattribute#1#2{#1=\numexpr#2\relax}
\def\unsetattribute#1{#1=-"7FFFFFFF\relax}
%    \end{macrocode}
% \end{macro}
% \end{macro}
%
% \subsection{Category code tables}
%
% \begin{macro}{\newcatcodetable}
% \changes{v1.0a}{2015/09/24}{Macro added}
%   Category code tables are allocated with a limit half of that used by Lua\TeX{}
%   for everything else. At the end of allocation there needs to be an
%   initialization step. Table~$0$ is already taken (it's the global one for
%   current use) so the allocation starts at~$1$.
%    \begin{macrocode}
\ifx\e@alloc@ccodetable@count\@undefined
  \countdef\e@alloc@ccodetable@count=259
  \e@alloc@ccodetable@count=\z@
\fi
\def\newcatcodetable#1{%
  \e@alloc\catcodetable\chardef
    \e@alloc@ccodetable@count\m@ne{"8000}#1%
  \initcatcodetable\allocationnumber
}
%    \end{macrocode}
% \end{macro}
%
% \changes{v1.0l}{2015/12/18}{Load Unicode data from source}
% \begin{macro}{\catcodetable@initex}
% \changes{v1.0a}{2015/09/24}{Macro added}
% \begin{macro}{\catcodetable@string}
% \changes{v1.0a}{2015/09/24}{Macro added}
% \begin{macro}{\catcodetable@latex}
% \changes{v1.0a}{2015/09/24}{Macro added}
% \begin{macro}{\catcodetable@atletter}
% \changes{v1.0a}{2015/09/24}{Macro added}
%   Save a small set of standard tables. The Unicode data is read
%   here in using a parser simplified from that in |load-unicode-data|:
%   only the nature of letters needs to be detected.
%    \begin{macrocode}
\newcatcodetable\catcodetable@initex
\newcatcodetable\catcodetable@string
\begingroup
  \def\setrangecatcode#1#2#3{%
    \ifnum#1>#2 %
      \expandafter\@gobble
    \else
      \expandafter\@firstofone
    \fi
      {%
        \catcode#1=#3 %
        \expandafter\setrangecatcode\expandafter
          {\number\numexpr#1 + 1\relax}{#2}{#3}
      }%
  }
  \@firstofone{%
    \catcodetable\catcodetable@initex
      \catcode0=12 %
      \catcode13=12 %
      \catcode37=12 %
      \setrangecatcode{65}{90}{12}%
      \setrangecatcode{97}{122}{12}%
      \catcode92=12 %
      \catcode127=12 %
      \savecatcodetable\catcodetable@string
    \endgroup
  }%
\newcatcodetable\catcodetable@latex
\newcatcodetable\catcodetable@atletter
\begingroup
  \def\parseunicodedataI#1;#2;#3;#4\relax{%
    \parseunicodedataII#1;#3;#2 First>\relax
  }%
  \def\parseunicodedataII#1;#2;#3 First>#4\relax{%
    \ifx\relax#4\relax
      \expandafter\parseunicodedataIII
    \else
      \expandafter\parseunicodedataIV
    \fi
      {#1}#2\relax%
  }%
  \def\parseunicodedataIII#1#2#3\relax{%
    \ifnum 0%
      \if L#21\fi
      \if M#21\fi
      >0 %
      \catcode"#1=11 %
    \fi
  }%
  \def\parseunicodedataIV#1#2#3\relax{%
    \read\unicoderead to \unicodedataline
    \if L#2%
      \count0="#1 %
      \expandafter\parseunicodedataV\unicodedataline\relax
    \fi
  }%
  \def\parseunicodedataV#1;#2\relax{%
    \loop
      \unless\ifnum\count0>"#1 %
        \catcode\count0=11 %
        \advance\count0 by 1 %
    \repeat
  }%
  \def\storedpar{\par}%
  \chardef\unicoderead=\numexpr\count16 + 1\relax
  \openin\unicoderead=UnicodeData.txt %
  \loop\unless\ifeof\unicoderead %
    \read\unicoderead to \unicodedataline
    \unless\ifx\unicodedataline\storedpar
      \expandafter\parseunicodedataI\unicodedataline\relax
    \fi
  \repeat
  \closein\unicoderead
  \@firstofone{%
    \catcode64=12 %
    \savecatcodetable\catcodetable@latex
    \catcode64=11 %
    \savecatcodetable\catcodetable@atletter
   }
\endgroup
%    \end{macrocode}
% \end{macro}
% \end{macro}
% \end{macro}
% \end{macro}
%
% \subsection{Named Lua functions}
%
% \begin{macro}{\newluafunction}
% \changes{v1.0a}{2015/09/24}{Macro added}
% \changes{v1.1q}{2020/08/02}{Move reset to $0$ inside conditional}
%   Much the same story for allocating Lua\TeX{} functions except here they are
%   just numbers so they are allocated in the same way as boxes.
%   Lua indexes from~$1$ so once again slot~$0$ is skipped.
%    \begin{macrocode}
\ifx\e@alloc@luafunction@count\@undefined
  \countdef\e@alloc@luafunction@count=260
  \e@alloc@luafunction@count=\z@
\fi
\def\newluafunction{%
  \e@alloc\luafunction\e@alloc@chardef
    \e@alloc@luafunction@count\m@ne\e@alloc@top
}
%    \end{macrocode}
% \end{macro}
%
% \begin{macro}{\newluacmd, \newprotectedluacmd}
% \changes{v1.1x}{2021/12/27}{Macros added}
%   Additionally two variants are provided to make the passed control
%   sequence call the function directly.
%    \begin{macrocode}
\def\newluacmd{%
  \e@alloc\luafunction\luadef
    \e@alloc@luafunction@count\m@ne\e@alloc@top
}
\def\newprotectedluacmd{%
  \e@alloc\luafunction{\protected\luadef}
    \e@alloc@luafunction@count\m@ne\e@alloc@top
}
%    \end{macrocode}
% \end{macro}
%
% \subsection{Custom whatsits}
%
% \begin{macro}{\newwhatsit}
% \changes{v1.0a}{2015/09/24}{Macro added}
% \changes{v1.1q}{2020/08/02}{Move reset to $0$ inside conditional}
%   These are only settable from Lua but for consistency are definable
%   here.
%    \begin{macrocode}
\ifx\e@alloc@whatsit@count\@undefined
  \countdef\e@alloc@whatsit@count=261
  \e@alloc@whatsit@count=\z@
\fi
\def\newwhatsit#1{%
  \e@alloc\whatsit\e@alloc@chardef
    \e@alloc@whatsit@count\m@ne\e@alloc@top#1%
}
%    \end{macrocode}
% \end{macro}
%
% \subsection{Lua bytecode registers}
%
% \begin{macro}{\newluabytecode}
% \changes{v1.0a}{2015/09/24}{Macro added}
% \changes{v1.1q}{2020/08/02}{Move reset to $0$ inside conditional}
%   These are only settable from Lua but for consistency are definable
%   here.
%    \begin{macrocode}
\ifx\e@alloc@bytecode@count\@undefined
  \countdef\e@alloc@bytecode@count=262
  \e@alloc@bytecode@count=\z@
\fi
\def\newluabytecode#1{%
  \e@alloc\luabytecode\e@alloc@chardef
    \e@alloc@bytecode@count\m@ne\e@alloc@top#1%
}
%    \end{macrocode}
% \end{macro}
%
% \subsection{Lua chunk registers}

% \begin{macro}{\newluachunkname}
% \changes{v1.0a}{2015/09/24}{Macro added}
% \changes{v1.1q}{2020/08/02}{Move reset to $0$ inside conditional}
% As for bytecode registers, but in addition we need to add a string
% to the \verb|lua.name| table to use in stack tracing. We use the
% name of the command passed to the allocator, with no backslash.
%    \begin{macrocode}
\ifx\e@alloc@luachunk@count\@undefined
  \countdef\e@alloc@luachunk@count=263
  \e@alloc@luachunk@count=\z@
\fi
\def\newluachunkname#1{%
  \e@alloc\luachunk\e@alloc@chardef
    \e@alloc@luachunk@count\m@ne\e@alloc@top#1%
    {\escapechar\m@ne
    \directlua{lua.name[\the\allocationnumber]="\string#1"}}%
}
%    \end{macrocode}
% \end{macro}
%
% \subsection{Lua loader}
% \changes{v1.1r}{2020/08/10}{Load ltluatex Lua module during format building}
%
% Lua code loaded in the format often has to be loaded again at the
% beginning of every job, so we define a helper which allows us to avoid
% duplicated code:
%
%    \begin{macrocode}
\def\now@and@everyjob#1{%
  \everyjob\expandafter{\the\everyjob
    #1%
  }%
  #1%
}
%    \end{macrocode}
%
% Load the Lua code at the start of every job.
% For the conversion of \TeX{} into numbers at the Lua side we need some
% known registers: for convenience we use a set of systematic names, which
% means using a group around the Lua loader.
%    \begin{macrocode}
%<2ekernel>\now@and@everyjob{%
  \begingroup
    \attributedef\attributezero=0 %
    \chardef     \charzero     =0 %
%    \end{macrocode}
% Note name change required on older luatex, for hash table access.
%    \begin{macrocode}
    \countdef    \CountZero    =0 %
    \dimendef    \dimenzero    =0 %
    \mathchardef \mathcharzero =0 %
    \muskipdef   \muskipzero   =0 %
    \skipdef     \skipzero     =0 %
    \toksdef     \tokszero     =0 %
    \directlua{require("ltluatex")}
  \endgroup
%<2ekernel>}
%<latexrelease>\EndIncludeInRelease
%    \end{macrocode}
%
% \changes{v1.0b}{2015/10/02}{Fix backing out of \TeX{} code}
% \changes{v1.0c}{2015/10/02}{Allow backing out of Lua code}
%    \begin{macrocode}
%<latexrelease>\IncludeInRelease{0000/00/00}
%<latexrelease>                 {\newluafunction}{LuaTeX}%
%<latexrelease>\let\e@alloc@attribute@count\@undefined
%<latexrelease>\let\newattribute\@undefined
%<latexrelease>\let\setattribute\@undefined
%<latexrelease>\let\unsetattribute\@undefined
%<latexrelease>\let\e@alloc@ccodetable@count\@undefined
%<latexrelease>\let\newcatcodetable\@undefined
%<latexrelease>\let\catcodetable@initex\@undefined
%<latexrelease>\let\catcodetable@string\@undefined
%<latexrelease>\let\catcodetable@latex\@undefined
%<latexrelease>\let\catcodetable@atletter\@undefined
%<latexrelease>\let\e@alloc@luafunction@count\@undefined
%<latexrelease>\let\newluafunction\@undefined
%<latexrelease>\let\e@alloc@luafunction@count\@undefined
%<latexrelease>\let\newwhatsit\@undefined
%<latexrelease>\let\e@alloc@whatsit@count\@undefined
%<latexrelease>\let\newluabytecode\@undefined
%<latexrelease>\let\e@alloc@bytecode@count\@undefined
%<latexrelease>\let\newluachunkname\@undefined
%<latexrelease>\let\e@alloc@luachunk@count\@undefined
%<latexrelease>\directlua{luatexbase.uninstall()}
%<latexrelease>\EndIncludeInRelease
%    \end{macrocode}
%
% In \verb|\everyjob|, if luaotfload is available, load it and switch to TU.
%    \begin{macrocode}
%<latexrelease>\IncludeInRelease{2017/01/01}%
%<latexrelease>                 {\fontencoding}{TU in everyjob}%
%<latexrelease>\fontencoding{TU}\let\encodingdefault\f@encoding
%<latexrelease>\ifx\directlua\@undefined\else
%<2ekernel>\everyjob\expandafter{%
%<2ekernel>  \the\everyjob
%<*2ekernel,latexrelease>
  \directlua{%
  if xpcall(function ()%
             require('luaotfload-main')%
            end,texio.write_nl) then %
  local _void = luaotfload.main ()%
  else %
  texio.write_nl('Error in luaotfload: reverting to OT1')%
  tex.print('\string\\def\string\\encodingdefault{OT1}')%
  end %
  }%
  \let\f@encoding\encodingdefault
  \expandafter\let\csname ver@luaotfload.sty\endcsname\fmtversion
%</2ekernel,latexrelease>
%<latexrelease>\fi
%<2ekernel>  }
%<latexrelease>\EndIncludeInRelease
%<latexrelease>\IncludeInRelease{0000/00/00}%
%<latexrelease>                 {\fontencoding}{TU in everyjob}%
%<latexrelease>\fontencoding{OT1}\let\encodingdefault\f@encoding
%<latexrelease>\EndIncludeInRelease
%    \end{macrocode}
%
%    \begin{macrocode}
%<2ekernel|latexrelease>\fi
%</2ekernel|tex|latexrelease>
%    \end{macrocode}
%
% \subsection{Lua module preliminaries}
%
% \begingroup
%
%  \begingroup\lccode`~=`_
%  \lowercase{\endgroup\let~}_
%  \catcode`_=12
%
%    \begin{macrocode}
%<*lua>
%    \end{macrocode}
%
% Some set up for the Lua module which is needed for all of the Lua
% functionality added here.
%
% \begin{macro}{luatexbase}
% \changes{v1.0a}{2015/09/24}{Table added}
%   Set up the table for the returned functions. This is used to expose
%   all of the public functions.
%    \begin{macrocode}
luatexbase       = luatexbase or { }
local luatexbase = luatexbase
%    \end{macrocode}
% \end{macro}
%
% Some Lua best practice: use local versions of functions where possible.
% \changes{v1.1u}{2021/08/11}{Define missing local function}
% \changes{v1.2b}{2023/01/19}{Remove unused local variable tex_setattribute}
%    \begin{macrocode}
local string_gsub      = string.gsub
local tex_count        = tex.count
local tex_setcount     = tex.setcount
local texio_write_nl   = texio.write_nl
local flush_list       = node.flush_list
%    \end{macrocode}
% \changes{v1.0i}{2015/11/29}{Declare this as local before used in the module error definitions (PHG)}
%    \begin{macrocode}
local luatexbase_warning
local luatexbase_error
%    \end{macrocode}
%
% \subsection{Lua module utilities}
%
% \subsubsection{Module tracking}
%
% \begin{macro}{modules}
% \changes{v1.0a}{2015/09/24}{Function modified}
%   To allow tracking of module usage, a structure is provided to store
%   information and to return it.
%    \begin{macrocode}
local modules = modules or { }
%    \end{macrocode}
% \end{macro}
%
% \begin{macro}{provides_module}
% \changes{v1.0a}{2015/09/24}{Function added}
% \changes{v1.0f}{2015/10/03}{use luatexbase\_log}
% Local function to write to the log.
%    \begin{macrocode}
local function luatexbase_log(text)
  texio_write_nl("log", text)
end
%    \end{macrocode}
%
%   Modelled on |\ProvidesPackage|, we store much the same information but
%   with a little more structure.
%    \begin{macrocode}
local function provides_module(info)
  if not (info and info.name) then
    luatexbase_error("Missing module name for provides_module")
  end
  local function spaced(text)
    return text and (" " .. text) or ""
  end
  luatexbase_log(
    "Lua module: " .. info.name
      .. spaced(info.date)
      .. spaced(info.version)
      .. spaced(info.description)
  )
  modules[info.name] = info
end
luatexbase.provides_module = provides_module
%    \end{macrocode}
% \end{macro}
%
% \subsubsection{Module messages}
%
% There are various warnings and errors that need to be given. For warnings
% we can get exactly the same formatting as from \TeX{}. For errors we have to
% make some changes. Here we give the text of the error in the \LaTeX{} format
% then force an error from Lua to halt the run. Splitting the message text is
% done using |\n| which takes the place of |\MessageBreak|.
%
% First an auxiliary for the formatting: this measures up the message
% leader so we always get the correct indent.
% \changes{v1.0j}{2015/12/02}{Declaration/use of first\_head fixed (PHG)}
%    \begin{macrocode}
local function msg_format(mod, msg_type, text)
  local leader = ""
  local cont
  local first_head
  if mod == "LaTeX" then
    cont = string_gsub(leader, ".", " ")
    first_head = leader .. "LaTeX: "
  else
    first_head = leader .. "Module "  .. msg_type
    cont = "(" .. mod .. ")"
      .. string_gsub(first_head, ".", " ")
    first_head =  leader .. "Module "  .. mod .. " " .. msg_type  .. ":"
  end
  if msg_type == "Error" then
    first_head = "\n" .. first_head
  end
  if string.sub(text,-1) ~= "\n" then
    text = text .. " "
  end
  return first_head .. " "
    .. string_gsub(
         text
	 .. "on input line "
         .. tex.inputlineno, "\n", "\n" .. cont .. " "
      )
   .. "\n"
end
%    \end{macrocode}
%
% \begin{macro}{module_info}
% \changes{v1.0a}{2015/09/24}{Function added}
% \begin{macro}{module_warning}
% \changes{v1.0a}{2015/09/24}{Function added}
% \begin{macro}{module_error}
% \changes{v1.0a}{2015/09/24}{Function added}
%   Write messages.
%    \begin{macrocode}
local function module_info(mod, text)
  texio_write_nl("log", msg_format(mod, "Info", text))
end
luatexbase.module_info = module_info
local function module_warning(mod, text)
  texio_write_nl("term and log",msg_format(mod, "Warning", text))
end
luatexbase.module_warning = module_warning
local function module_error(mod, text)
  error(msg_format(mod, "Error", text))
end
luatexbase.module_error = module_error
%    \end{macrocode}
% \end{macro}
% \end{macro}
% \end{macro}
%
% Dedicated versions for the rest of the code here.
%    \begin{macrocode}
function luatexbase_warning(text)
  module_warning("luatexbase", text)
end
function luatexbase_error(text)
  module_error("luatexbase", text)
end
%    \end{macrocode}
%
%
% \subsection{Accessing register numbers from Lua}
%
% \changes{v1.0g}{2015/11/14}{Track Lua\TeX{} changes for
%   \texttt{(new)token.create}}
% Collect up the data from the \TeX{} level into a Lua table: from
% version~0.80, Lua\TeX{} makes that easy.
% \changes{v1.0j}{2015/12/02}{Adjust hashtokens to store the result of tex.hashtokens()), not the function (PHG)}
%    \begin{macrocode}
local luaregisterbasetable = { }
local registermap = {
  attributezero = "assign_attr"    ,
  charzero      = "char_given"     ,
  CountZero     = "assign_int"     ,
  dimenzero     = "assign_dimen"   ,
  mathcharzero  = "math_given"     ,
  muskipzero    = "assign_mu_skip" ,
  skipzero      = "assign_skip"    ,
  tokszero      = "assign_toks"    ,
}
local createtoken
if tex.luatexversion > 81 then
  createtoken = token.create
elseif tex.luatexversion > 79 then
  createtoken = newtoken.create
end
local hashtokens    = tex.hashtokens()
local luatexversion = tex.luatexversion
for i,j in pairs (registermap) do
  if luatexversion < 80 then
    luaregisterbasetable[hashtokens[i][1]] =
      hashtokens[i][2]
  else
    luaregisterbasetable[j] = createtoken(i).mode
  end
end
%    \end{macrocode}
%
% \begin{macro}{registernumber}
%   Working out the correct return value can be done in two ways. For older
%   Lua\TeX{} releases it has to be extracted from the |hashtokens|. On the
%   other hand, newer Lua\TeX{}'s have |newtoken|, and whilst |.mode| isn't
%   currently documented, Hans Hagen pointed to this approach so we should be
%   OK.
%    \begin{macrocode}
local registernumber
if luatexversion < 80 then
  function registernumber(name)
    local nt = hashtokens[name]
    if(nt and luaregisterbasetable[nt[1]]) then
      return nt[2] - luaregisterbasetable[nt[1]]
    else
      return false
    end
  end
else
  function registernumber(name)
    local nt = createtoken(name)
    if(luaregisterbasetable[nt.cmdname]) then
      return nt.mode - luaregisterbasetable[nt.cmdname]
    else
      return false
    end
  end
end
luatexbase.registernumber = registernumber
%    \end{macrocode}
% \end{macro}
%
% \subsection{Attribute allocation}
%
% \begin{macro}{new_attribute}
% \changes{v1.0a}{2015/09/24}{Function added}
% \changes{v1.1c}{2017/02/18}{Parameterize count used in tracking}
%   As attributes are used for Lua manipulations its useful to be able
%   to assign from this end.
% \InternalDetectionOff
%    \begin{macrocode}
local attributes=setmetatable(
{},
{
__index = function(t,key)
return registernumber(key) or nil
end}
)
luatexbase.attributes = attributes
%    \end{macrocode}
%
%    \begin{macrocode}
local attribute_count_name =
                     attribute_count_name or "e@alloc@attribute@count"
local function new_attribute(name)
  tex_setcount("global", attribute_count_name,
                          tex_count[attribute_count_name] + 1)
  if tex_count[attribute_count_name] > 65534 then
    luatexbase_error("No room for a new \\attribute")
  end
  attributes[name]= tex_count[attribute_count_name]
  luatexbase_log("Lua-only attribute " .. name .. " = " ..
                 tex_count[attribute_count_name])
  return tex_count[attribute_count_name]
end
luatexbase.new_attribute = new_attribute
%    \end{macrocode}
% \InternalDetectionOn
% \end{macro}
%
% \subsection{Custom whatsit allocation}
%
% \begin{macro}{new_whatsit}
% \changes{v1.1c}{2017/02/18}{Parameterize count used in tracking}
% Much the same as for attribute allocation in Lua.
%    \begin{macrocode}
local whatsit_count_name = whatsit_count_name or "e@alloc@whatsit@count"
local function new_whatsit(name)
  tex_setcount("global", whatsit_count_name,
                         tex_count[whatsit_count_name] + 1)
  if tex_count[whatsit_count_name] > 65534 then
    luatexbase_error("No room for a new custom whatsit")
  end
  luatexbase_log("Custom whatsit " .. (name or "") .. " = " ..
                 tex_count[whatsit_count_name])
  return tex_count[whatsit_count_name]
end
luatexbase.new_whatsit = new_whatsit
%    \end{macrocode}
% \end{macro}
%
% \subsection{Bytecode register allocation}
%
% \begin{macro}{new_bytecode}
% \changes{v1.1c}{2017/02/18}{Parameterize count used in tracking}
% Much the same as for attribute allocation in Lua.
% The optional \meta{name} argument is used in the log if given.
%    \begin{macrocode}
local bytecode_count_name =
                         bytecode_count_name or "e@alloc@bytecode@count"
local function new_bytecode(name)
  tex_setcount("global", bytecode_count_name,
                         tex_count[bytecode_count_name] + 1)
  if tex_count[bytecode_count_name] > 65534 then
    luatexbase_error("No room for a new bytecode register")
  end
  luatexbase_log("Lua bytecode " .. (name or "") .. " = " ..
                 tex_count[bytecode_count_name])
  return tex_count[bytecode_count_name]
end
luatexbase.new_bytecode = new_bytecode
%    \end{macrocode}
% \end{macro}
%
% \subsection{Lua chunk name allocation}
%
% \begin{macro}{new_chunkname}
% \changes{v1.1c}{2017/02/18}{Parameterize count used in tracking}
% As for bytecode registers but also store the name in the
% |lua.name| table.
%    \begin{macrocode}
local chunkname_count_name =
                        chunkname_count_name or "e@alloc@luachunk@count"
local function new_chunkname(name)
  tex_setcount("global", chunkname_count_name,
                         tex_count[chunkname_count_name] + 1)
  local chunkname_count = tex_count[chunkname_count_name]
  chunkname_count = chunkname_count + 1
  if chunkname_count > 65534 then
    luatexbase_error("No room for a new chunkname")
  end
  lua.name[chunkname_count]=name
  luatexbase_log("Lua chunkname " .. (name or "") .. " = " ..
                 chunkname_count .. "\n")
  return chunkname_count
end
luatexbase.new_chunkname = new_chunkname
%    \end{macrocode}
% \end{macro}
%
% \subsection{Lua function allocation}
%
% \begin{macro}{new_luafunction}
% \changes{v1.1i}{2018/10/21}{Function added}
% \changes{v1.2c}{2023/07/02}{Ensure existing table entries not overwritten gh/1100}
% Much the same as for attribute allocation in Lua.
% The optional \meta{name} argument is used in the log if given.
%    \begin{macrocode}
local luafunction_count_name =
                         luafunction_count_name or "e@alloc@luafunction@count"
local function new_luafunction(name)
  tex_setcount("global", luafunction_count_name,
                         math.max(
                           #(lua.get_functions_table()),
                           tex_count[luafunction_count_name])
                          + 1)
  lua.get_functions_table()[tex_count[luafunction_count_name]] = false
  if tex_count[luafunction_count_name] > 65534 then
    luatexbase_error("No room for a new luafunction register")
  end
  luatexbase_log("Lua function " .. (name or "") .. " = " ..
                 tex_count[luafunction_count_name])
  return tex_count[luafunction_count_name]
end
luatexbase.new_luafunction = new_luafunction
%    \end{macrocode}
% \end{macro}
%
% \subsection{Lua callback management}
%
% The native mechanism for callbacks in Lua\TeX\ allows only one per function.
% That is extremely restrictive and so a mechanism is needed to add and
% remove callbacks from the appropriate hooks.
%
% \subsubsection{Housekeeping}
%
% The main table: keys are callback names, and values are the associated lists
% of functions. More precisely, the entries in the list are tables holding the
% actual function as |func| and the identifying description as |description|.
% Only callbacks with a non-empty list of functions have an entry in this
% list.
%
% Actually there are two tables: |realcallbacklist| directly contains the entries
% as described above while |callbacklist| only directly contains the already sorted
% entries. Other entries can be queried through |callbacklist| too which triggers a
% resort.
%
% Additionally |callbackrules| describes the ordering constraints: It contains two
% element tables with the descriptions of the constrained callback implementations.
% It can additionally contain a |type| entry indicating the kind of rule. A missing
% value indicates a normal ordering contraint.
%
% \changes{v1.2a}{2022/10/03}{Add rules for callback ordering}
%    \begin{macrocode}
local realcallbacklist = {}
local callbackrules = {}
local callbacklist = setmetatable({}, {
  __index = function(t, name)
    local list = realcallbacklist[name]
    local rules = callbackrules[name]
    if list and rules then
      local meta = {}
      for i, entry in ipairs(list) do
        local t = {value = entry, count = 0, pos = i}
        meta[entry.description], list[i] = t, t
      end
      local count = #list
      local pos = count
      for i, rule in ipairs(rules) do
        local rule = rules[i]
        local pre, post = meta[rule[1]], meta[rule[2]]
        if pre and post then
          if rule.type then
            if not rule.hidden then
              assert(rule.type == 'incompatible-warning' and luatexbase_warning
                or rule.type == 'incompatible-error' and luatexbase_error)(
                  "Incompatible functions \"" .. rule[1] .. "\" and \"" .. rule[2]
                  .. "\" specified for callback \"" .. name .. "\".")
              rule.hidden = true
            end
          else
            local post_count = post.count
            post.count = post_count+1
            if post_count == 0 then
              local post_pos = post.pos
              if post_pos ~= pos then
                local new_post_pos = list[pos]
                new_post_pos.pos = post_pos
                list[post_pos] = new_post_pos
              end
              list[pos] = nil
              pos = pos - 1
            end
            pre[#pre+1] = post
          end
        end
      end
      for i=1, count do -- The actual sort begins
        local current = list[i]
        if current then
          meta[current.value.description] = nil
          for j, cur in ipairs(current) do
            local count = cur.count
            if count == 1 then
              pos = pos + 1
              list[pos] = cur
            else
              cur.count = count - 1
            end
          end
          list[i] = current.value
        else
          -- Cycle occured. TODO: Show cycle for debugging
          -- list[i] = ...
          local remaining = {}
          for name, entry in next, meta do
            local value = entry.value
            list[#list + 1] = entry.value
            remaining[#remaining + 1] = name
          end
          table.sort(remaining)
          local first_name = remaining[1]
          for j, name in ipairs(remaining) do
            local entry = meta[name]
            list[i + j - 1] = entry.value
            for _, post_entry in ipairs(entry) do
              local post_name = post_entry.value.description
              if not remaining[post_name] then
                remaining[post_name] = name
              end
            end
          end
          local cycle = {first_name}
          local index = 1
          local last_name = first_name
          repeat
            cycle[last_name] = index
            last_name = remaining[last_name]
            index = index + 1
            cycle[index] = last_name
          until cycle[last_name]
          local length = index - cycle[last_name] + 1
          table.move(cycle, cycle[last_name], index, 1)
          for i=2, length//2 do
            cycle[i], cycle[length + 1 - i] = cycle[length + 1 - i], cycle[i]
          end
          error('Cycle occured at ' .. table.concat(cycle, ' -> ', 1, length))
        end
      end
    end
    realcallbacklist[name] = list
    t[name] = list
    return list
  end
})
%    \end{macrocode}
%
% Numerical codes for callback types, and name-to-value association (the
% table keys are strings, the values are numbers).
%    \begin{macrocode}
local list, data, exclusive, simple, reverselist = 1, 2, 3, 4, 5
local types   = {
  list        = list,
  data        = data,
  exclusive   = exclusive,
  simple      = simple,
  reverselist = reverselist,
}
%    \end{macrocode}
%
% Now, list all predefined callbacks with their current type, based on the
% Lua\TeX{} manual version~1.01. A full list of the currently-available
% callbacks can be obtained using
%  \begin{verbatim}
%    \directlua{
%      for i,_ in pairs(callback.list()) do
%        texio.write_nl("- " .. i)
%      end
%    }
%    \bye
%  \end{verbatim}
% in plain Lua\TeX{}. (Some undocumented callbacks are omitted as they are
% to be removed.)
%    \begin{macrocode}
local callbacktypes = callbacktypes or {
%    \end{macrocode}
%   Section 8.2: file discovery callbacks.
% \changes{v1.1g}{2018/05/02}{find\_sfd\_file removed}
%    \begin{macrocode}
  find_read_file     = exclusive,
  find_write_file    = exclusive,
  find_font_file     = data,
  find_output_file   = data,
  find_format_file   = data,
  find_vf_file       = data,
  find_map_file      = data,
  find_enc_file      = data,
  find_pk_file       = data,
  find_data_file     = data,
  find_opentype_file = data,
  find_truetype_file = data,
  find_type1_file    = data,
  find_image_file    = data,
%    \end{macrocode}
% \changes{v1.1g}{2018/05/02}{read\_sfd\_file removed}
%    \begin{macrocode}
  open_read_file     = exclusive,
  read_font_file     = exclusive,
  read_vf_file       = exclusive,
  read_map_file      = exclusive,
  read_enc_file      = exclusive,
  read_pk_file       = exclusive,
  read_data_file     = exclusive,
  read_truetype_file = exclusive,
  read_type1_file    = exclusive,
  read_opentype_file = exclusive,
%    \end{macrocode}
% \changes{v1.0m}{2016/02/11}{read\_cidmap\_file added}
% Not currently used by luatex but included for completeness.
% may be used by a font handler.
%    \begin{macrocode}
  find_cidmap_file   = data,
  read_cidmap_file   = exclusive,
%    \end{macrocode}
% Section 8.3: data processing callbacks.
% \changes{v1.0m}{2016/02/11}{token\_filter removed}
%    \begin{macrocode}
  process_input_buffer  = data,
  process_output_buffer = data,
  process_jobname       = data,
%    \end{macrocode}
% Section 8.4: node list processing callbacks.
% \changes{v1.0m}{2016/02/11}
% {process\_rule, [hv]pack\_quality  append\_to\_vlist\_filter added}
% \changes{v1.0n}{2016/03/13}{insert\_local\_par added}
% \changes{v1.0n}{2016/03/13}{contribute\_filter added}
% \changes{v1.1h}{2018/08/18}{append\_to\_vlist\_filter is \texttt{exclusive}}
% \changes{v1.1j}{2019/06/18}{new\_graf added}
% \changes{v1.1k}{2019/10/02}{linebreak\_filter is \texttt{exclusive}}
% \changes{v1.1k}{2019/10/02}{process\_rule is \texttt{exclusive}}
% \changes{v1.1k}{2019/10/02}{mlist\_to\_hlist is \texttt{exclusive}}
% \changes{v1.1l}{2020/02/02}{post\_linebreak\_filter is \texttt{reverselist}}
% \changes{v1.1p}{2020/08/01}{new\_graf is \texttt{exclusive}}
% \changes{v1.1w}{2021/11/17}{hpack\_quality is \texttt{exclusive}}
% \changes{v1.1w}{2021/11/17}{vpack\_quality is \texttt{exclusive}}
%    \begin{macrocode}
  contribute_filter      = simple,
  buildpage_filter       = simple,
  build_page_insert      = exclusive,
  pre_linebreak_filter   = list,
  linebreak_filter       = exclusive,
  append_to_vlist_filter = exclusive,
  post_linebreak_filter  = reverselist,
  hpack_filter           = list,
  vpack_filter           = list,
  hpack_quality          = exclusive,
  vpack_quality          = exclusive,
  pre_output_filter      = list,
  process_rule           = exclusive,
  hyphenate              = simple,
  ligaturing             = simple,
  kerning                = simple,
  insert_local_par       = simple,
% mlist_to_hlist         = exclusive,
  new_graf               = exclusive,
%    \end{macrocode}
% Section 8.5: information reporting callbacks.
% \changes{v1.0m}{2016/02/11}{show\_warning\_message added}
% \changes{v1.0p}{2016/11/17}{call\_edit added}
% \changes{v1.1g}{2018/05/02}{finish\_synctex\_callback added}
% \changes{v1.1j}{2019/06/18}{finish\_synctex\_callback renamed finish\_synctex}
% \changes{v1.1j}{2019/06/18}{wrapup\_run added}
%    \begin{macrocode}
  pre_dump             = simple,
  start_run            = simple,
  stop_run             = simple,
  start_page_number    = simple,
  stop_page_number     = simple,
  show_error_hook      = simple,
  show_warning_message = simple,
  show_error_message   = simple,
  show_lua_error_hook  = simple,
  start_file           = simple,
  stop_file            = simple,
  call_edit            = simple,
  finish_synctex       = simple,
  wrapup_run           = simple,
%    \end{macrocode}
% Section 8.6: PDF-related callbacks.
% \changes{v1.1j}{2019/06/18}{page\_objnum\_provider added}
% \changes{v1.1j}{2019/06/18}{process\_pdf\_image\_content added}
% \changes{v1.1j}{2019/10/22}{page\_objnum\_provider and process\_pdf\_image\_content classified data}
% \changes{v1.1l}{2020/02/02}{page\_order\_index added}
%    \begin{macrocode}
  finish_pdffile            = data,
  finish_pdfpage            = data,
  page_objnum_provider      = data,
  page_order_index          = data,
  process_pdf_image_content = data,
%    \end{macrocode}
% Section 8.7: font-related callbacks.
% \changes{v1.1e}{2017/03/28}{glyph\_stream\_provider added}
% \changes{v1.1g}{2018/05/02}{glyph\_not\_found added}
% \changes{v1.1j}{2019/06/18}{make\_extensible added}
% \changes{v1.1j}{2019/06/18}{font\_descriptor\_objnum\_provider added}
% \changes{v1.1l}{2020/02/02}{glyph\_info added}
% \changes{v1.1t}{2021/04/18}{input\_level\_string added}
% \changes{v1.1v}{2021/10/15}{provide\_charproc\_data added}
%    \begin{macrocode}
  define_font                     = exclusive,
  glyph_info                      = exclusive,
  glyph_not_found                 = exclusive,
  glyph_stream_provider           = exclusive,
  make_extensible                 = exclusive,
  font_descriptor_objnum_provider = exclusive,
  input_level_string              = exclusive,
  provide_charproc_data           = exclusive,
%    \end{macrocode}
% \changes{v1.0m}{2016/02/11}{pdf\_stream\_filter\_callback removed}
%    \begin{macrocode}
}
luatexbase.callbacktypes=callbacktypes
%    \end{macrocode}
%
% \changes{v1.1y}{2022/08/13}{shared\_callbacks added}
% Sometimes multiple callbacks correspond to a single underlying engine level callback.
% Then the engine level callback should be registered as long as at least one of these
% callbacks is in use. This is implemented though a shared table which counts how many
% of the involved callbacks are currently in use. The enging level callback is registered
% iff this count is not 0.
%
% We add |mlist_to_hlist| directly to the list to demonstrate this, but the handler gets
% added later when it is actually defined.
%
% All callbacks in this list are treated as user defined callbacks.
%
%    \begin{macrocode}
local shared_callbacks = {
  mlist_to_hlist = {
    callback = "mlist_to_hlist",
    count = 0,
    handler = nil,
  },
}
shared_callbacks.pre_mlist_to_hlist_filter = shared_callbacks.mlist_to_hlist
shared_callbacks.post_mlist_to_hlist_filter = shared_callbacks.mlist_to_hlist
%    \end{macrocode}
%
% \begin{macro}{callback.register}
% \changes{v1.0a}{2015/09/24}{Function modified}
%   Save the original function for registering callbacks and prevent the
%   original being used. The original is saved in a place that remains
%   available so other more sophisticated code can override the approach
%   taken by the kernel if desired.
%    \begin{macrocode}
local callback_register = callback_register or callback.register
function callback.register()
  luatexbase_error("Attempt to use callback.register() directly\n")
end
%    \end{macrocode}
% \end{macro}
%
% \subsubsection{Handlers}
%
% The handler function is registered into the callback when the
% first function is added to this callback's list. Then, when the callback
% is called, the handler takes care of running all functions in the list.
% When the last function is removed from the callback's list, the handler
% is unregistered.
%
% More precisely, the functions below are used to generate a specialized
% function (closure) for a given callback, which is the actual handler.
%
%
% The way the functions are combined together depends on
% the type of the callback. There are currently 4 types of callback, depending
% on the calling convention of the functions the callback can hold:
% \begin{description}
%   \item[simple] is for functions that don't return anything: they are called
%     in order, all with the same argument;
%   \item[data] is for functions receiving a piece of data of any type
%     except node list head (and possibly other arguments) and returning it
%     (possibly modified): the functions are called in order, and each is
%     passed the return value of the previous (and the other arguments
%     untouched, if any). The return value is that of the last function;
%   \item[list] is a specialized variant of \emph{data} for functions
%     filtering node lists. Such functions may return either the head of a
%     modified node list, or the boolean values |true| or |false|. The
%     functions are chained the same way as for \emph{data} except that for
%     the following. If
%     one function returns |false|, then |false| is immediately returned and
%     the following functions are \emph{not} called. If one function returns
%     |true|, then the same head is passed to the next function. If all
%     functions return |true|, then |true| is returned, otherwise the return
%     value of the last function not returning |true| is used.
%   \item[reverselist] is a specialized variant of \emph{list} which executes
%     functions in inverse order.
%   \item[exclusive] is for functions with more complex signatures; functions in
%     this type of callback are \emph{not} combined: An error is raised if
%     a second callback is registered.
% \end{description}
%
% Handler for |data| callbacks.
%    \begin{macrocode}
local function data_handler(name)
  return function(data, ...)
    for _,i in ipairs(callbacklist[name]) do
      data = i.func(data,...)
    end
    return data
  end
end
%    \end{macrocode}
% Default for user-defined |data| callbacks without explicit default.
%    \begin{macrocode}
local function data_handler_default(value)
  return value
end
%    \end{macrocode}
% Handler for |exclusive| callbacks. We can assume |callbacklist[name]| is not
% empty: otherwise, the function wouldn't be registered in the callback any
% more.
%    \begin{macrocode}
local function exclusive_handler(name)
  return function(...)
    return callbacklist[name][1].func(...)
  end
end
%    \end{macrocode}
% Handler for |list| callbacks.
% \changes{v1.0k}{2015/12/02}{resolve name and i.description (PHG)}
% \changes{v1.1s}{2020/12/02}{Fix return value of list callbacks}
% \changes{v1.1w}{2021/11/17}{Never pass on \texttt{true} return values for list callbacks}
%    \begin{macrocode}
local function list_handler(name)
  return function(head, ...)
    local ret
    for _,i in ipairs(callbacklist[name]) do
      ret = i.func(head, ...)
      if ret == false then
        luatexbase_warning(
          "Function `" .. i.description .. "' returned false\n"
            .. "in callback `" .. name .."'"
         )
        return false
      end
      if ret ~= true then
        head = ret
      end
    end
    return head
  end
end
%    \end{macrocode}
% Default for user-defined |list| and |reverselist| callbacks without explicit default.
%    \begin{macrocode}
local function list_handler_default(head)
return head
end
%    \end{macrocode}
% Handler for |reverselist| callbacks.
% \changes{v1.1l}{2020/02/02}{Add reverselist callback type}
%    \begin{macrocode}
local function reverselist_handler(name)
  return function(head, ...)
    local ret
    local callbacks = callbacklist[name]
    for i = #callbacks, 1, -1 do
      local cb = callbacks[i]
      ret = cb.func(head, ...)
      if ret == false then
        luatexbase_warning(
          "Function `" .. cb.description .. "' returned false\n"
            .. "in callback `" .. name .."'"
         )
        return false
      end
      if ret ~= true then
        head = ret
      end
    end
    return head
  end
end
%    \end{macrocode}
% Handler for |simple| callbacks.
%    \begin{macrocode}
local function simple_handler(name)
  return function(...)
    for _,i in ipairs(callbacklist[name]) do
      i.func(...)
    end
  end
end
%    \end{macrocode}
% Default for user-defined |simple| callbacks without explicit default.
%    \begin{macrocode}
local function simple_handler_default()
end
%    \end{macrocode}
%
% Keep a handlers table for indexed access and a table with the corresponding default functions.
%    \begin{macrocode}
local handlers  = {
  [data]        = data_handler,
  [exclusive]   = exclusive_handler,
  [list]        = list_handler,
  [reverselist] = reverselist_handler,
  [simple]      = simple_handler,
}
local defaults = {
  [data]        = data_handler_default,
  [exclusive]   = nil,
  [list]        = list_handler_default,
  [reverselist] = list_handler_default,
  [simple]      = simple_handler_default,
}
%    \end{macrocode}
%
% \subsubsection{Public functions for callback management}
%
% Defining user callbacks perhaps should be in package code,
% but impacts on |add_to_callback|.
% If a default function is not required, it may be declared as |false|.
% First we need a list of user callbacks.
%    \begin{macrocode}
local user_callbacks_defaults = {}
%    \end{macrocode}
%
% \begin{macro}{create_callback}
% \changes{v1.0a}{2015/09/24}{Function added}
% \changes{v1.0i}{2015/11/29}{Check name is not nil in error message (PHG)}
% \changes{v1.0k}{2015/12/02}{Give more specific error messages (PHG)}
% \changes{v1.1l}{2020/02/02}{Provide proper fallbacks for user-defined callbacks without user-provided default handler}
%   The allocator itself.
%    \begin{macrocode}
local function create_callback(name, ctype, default)
  local ctype_id = types[ctype]
  if not name  or name  == ""
  or not ctype_id
  then
    luatexbase_error("Unable to create callback:\n" ..
                     "valid callback name and type required")
  end
  if callbacktypes[name] then
    luatexbase_error("Unable to create callback `" .. name ..
                     "':\ncallback is already defined")
  end
  default = default or defaults[ctype_id]
  if not default then
    luatexbase_error("Unable to create callback `" .. name ..
                     "':\ndefault is required for `" .. ctype ..
                     "' callbacks")
  elseif type (default) ~= "function" then
    luatexbase_error("Unable to create callback `" .. name ..
                     "':\ndefault is not a function")
  end
  user_callbacks_defaults[name] = default
  callbacktypes[name] = ctype_id
end
luatexbase.create_callback = create_callback
%    \end{macrocode}
% \end{macro}
%
% \begin{macro}{call_callback}
% \changes{v1.0a}{2015/09/24}{Function added}
% \changes{v1.0i}{2015/11/29}{Check name is not nil in error message (PHG)}
% \changes{v1.0k}{2015/12/02}{Give more specific error messages (PHG)}
%  Call a user defined callback. First check arguments.
%    \begin{macrocode}
local function call_callback(name,...)
  if not name or name == "" then
    luatexbase_error("Unable to create callback:\n" ..
                     "valid callback name required")
  end
  if user_callbacks_defaults[name] == nil then
    luatexbase_error("Unable to call callback `" .. name
                     .. "':\nunknown or empty")
   end
  local l = callbacklist[name]
  local f
  if not l then
    f = user_callbacks_defaults[name]
  else
    f = handlers[callbacktypes[name]](name)
  end
  return f(...)
end
luatexbase.call_callback=call_callback
%    \end{macrocode}
% \end{macro}
%
% \begin{macro}{add_to_callback}
% \changes{v1.0a}{2015/09/24}{Function added}
%   Add a function to a callback. First check arguments.
% \changes{v1.0k}{2015/12/02}{Give more specific error messages (PHG)}
% \changes{v1.2a}{2022/10/03}{Add rules for callback ordering}
%    \begin{macrocode}
local function add_to_callback(name, func, description)
  if not name or name == "" then
    luatexbase_error("Unable to register callback:\n" ..
                     "valid callback name required")
  end
  if not callbacktypes[name] or
    type(func) ~= "function" or
    not description or
    description == "" then
    luatexbase_error(
      "Unable to register callback.\n\n"
        .. "Correct usage:\n"
        .. "add_to_callback(<callback>, <function>, <description>)"
    )
  end
%    \end{macrocode}
%   Then test if this callback is already in use. If not, initialise its list
%   and register the proper handler.
%    \begin{macrocode}
  local l = realcallbacklist[name]
  if l == nil then
    l = { }
    realcallbacklist[name] = l
%    \end{macrocode}
% \changes{v1.1y}{2022/08/13}{Adapted code for shared\_callbacks}
% Handle count for shared engine callbacks.
%    \begin{macrocode}
    local shared = shared_callbacks[name]
    if shared then
      shared.count = shared.count + 1
      if shared.count == 1 then
        callback_register(shared.callback, shared.handler)
      end
%    \end{macrocode}
% If it is not a user defined callback use the primitive callback register.
%    \begin{macrocode}
    elseif user_callbacks_defaults[name] == nil then
      callback_register(name, handlers[callbacktypes[name]](name))
    end
  end
%    \end{macrocode}
% \changes{v1.2a}{2022/10/03}{Add rules for callback ordering}
%  Actually register the function and give an error if more than one
%  |exclusive| one is registered.
%    \begin{macrocode}
  local f = {
    func        = func,
    description = description,
  }
  if callbacktypes[name] == exclusive then
    if #l == 1 then
      luatexbase_error(
        "Cannot add second callback to exclusive function\n`" ..
        name .. "'")
    end
  end
  table.insert(l, f)
  callbacklist[name] = nil
%    \end{macrocode}
%  Keep user informed.
%    \begin{macrocode}
  luatexbase_log(
    "Inserting `" .. description .. "' in `" .. name .. "'."
  )
end
luatexbase.add_to_callback = add_to_callback
%    \end{macrocode}
% \end{macro}
%
% \begin{macro}{declare_callback_rule}
%   \changes{v1.2a}{2022/10/03}{Add function}
%   Add an ordering constraint between two callback implementations
%    \begin{macrocode}
local function declare_callback_rule(name, desc1, relation, desc2)
  if not callbacktypes[name] or
    not desc1 or not desc2 or
    desc1 == "" or desc2 == "" then
    luatexbase_error(
      "Unable to create ordering constraint. "
        .. "Correct usage:\n"
        .. "declare_callback_rule(<callback>, <description_a>, <description_b>)"
    )
  end
  if relation == 'before' then
    relation = nil
  elseif relation == 'after' then
    desc2, desc1 = desc1, desc2
    relation = nil
  elseif relation == 'incompatible-warning' or relation == 'incompatible-error' then
  elseif relation == 'unrelated' then
  else
    luatexbase_error(
      "Unknown relation type in declare_callback_rule"
    )
  end
  callbacklist[name] = nil
  local rules = callbackrules[name]
  if rules then
    for i, rule in ipairs(rules) do
      if rule[1] == desc1 and rule[2] == desc2 or rule[1] == desc2 and rule[2] == desc1 then
        if relation == 'unrelated' then
          table.remove(rules, i)
        else
          rule[1], rule[2], rule.type = desc1, desc2, relation
        end
        return
      end
    end
    if relation ~= 'unrelated' then
      rules[#rules + 1] = {desc1, desc2, type = relation}
    end
  elseif relation ~= 'unrelated' then
    callbackrules[name] = {{desc1, desc2, type = relation}}
  end
end
luatexbase.declare_callback_rule = declare_callback_rule
%    \end{macrocode}
% \end{macro}
%
% \begin{macro}{remove_from_callback}
% \changes{v1.0a}{2015/09/24}{Function added}
% \changes{v1.0k}{2015/12/02}{adjust initialization of cb local (PHG)}
% \changes{v1.0k}{2015/12/02}{Give more specific error messages (PHG)}
% \changes{v1.1m}{2020/03/07}{Do not call callback.register for user-defined callbacks}
% \changes{v1.1y}{2022/08/13}{Adapted code for shared\_callbacks}
% \changes{v1.2a}{2022/10/03}{Add rules for callback ordering}
%   Remove a function from a callback. First check arguments.
%    \begin{macrocode}
local function remove_from_callback(name, description)
  if not name or name == "" then
    luatexbase_error("Unable to remove function from callback:\n" ..
                     "valid callback name required")
  end
  if not callbacktypes[name] or
    not description or
    description == "" then
    luatexbase_error(
      "Unable to remove function from callback.\n\n"
        .. "Correct usage:\n"
        .. "remove_from_callback(<callback>, <description>)"
    )
  end
  local l = realcallbacklist[name]
  if not l then
    luatexbase_error(
      "No callback list for `" .. name .. "'\n")
  end
%    \end{macrocode}
%  Loop over the callback's function list until we find a matching entry.
%  Remove it and check if the list is empty: if so, unregister the
%   callback handler.
%    \begin{macrocode}
  local index = false
  for i,j in ipairs(l) do
    if j.description == description then
      index = i
      break
    end
  end
  if not index then
    luatexbase_error(
      "No callback `" .. description .. "' registered for `" ..
      name .. "'\n")
  end
  local cb = l[index]
  table.remove(l, index)
  luatexbase_log(
    "Removing  `" .. description .. "' from `" .. name .. "'."
  )
  if #l == 0 then
    realcallbacklist[name] = nil
    callbacklist[name] = nil
    local shared = shared_callbacks[name]
    if shared then
      shared.count = shared.count - 1
      if shared.count == 0 then
        callback_register(shared.callback, nil)
      end
    elseif user_callbacks_defaults[name] == nil then
      callback_register(name, nil)
    end
  end
  return cb.func,cb.description
end
luatexbase.remove_from_callback = remove_from_callback
%    \end{macrocode}
% \end{macro}
%
% \begin{macro}{in_callback}
% \changes{v1.0a}{2015/09/24}{Function added}
% \changes{v1.0h}{2015/11/27}{Guard against undefined list latex/4445}
%   Look for a function description in a callback.
%    \begin{macrocode}
local function in_callback(name, description)
  if not name
    or name == ""
    or not realcallbacklist[name]
    or not callbacktypes[name]
    or not description then
      return false
  end
  for _, i in pairs(realcallbacklist[name]) do
    if i.description == description then
      return true
    end
  end
  return false
end
luatexbase.in_callback = in_callback
%    \end{macrocode}
% \end{macro}
%
% \begin{macro}{disable_callback}
% \changes{v1.0a}{2015/09/24}{Function added}
%   As we subvert the engine interface we need to provide a way to access
%   this functionality.
%    \begin{macrocode}
local function disable_callback(name)
  if(realcallbacklist[name] == nil) then
    callback_register(name, false)
  else
    luatexbase_error("Callback list for " .. name .. " not empty")
  end
end
luatexbase.disable_callback = disable_callback
%    \end{macrocode}
% \end{macro}
%
% \begin{macro}{callback_descriptions}
% \changes{v1.0a}{2015/09/24}{Function added}
% \changes{v1.0h}{2015/11/27}{Match test in in-callback latex/4445}
%   List the descriptions of functions registered for the given callback.
%   This will sort the list if necessary.
%    \begin{macrocode}
local function callback_descriptions (name)
  local d = {}
  if not name
    or name == ""
    or not realcallbacklist[name]
    or not callbacktypes[name]
    then
    return d
  else
  for k, i in pairs(callbacklist[name]) do
    d[k]= i.description
    end
  end
  return d
end
luatexbase.callback_descriptions =callback_descriptions
%    \end{macrocode}
% \end{macro}
%
% \begin{macro}{uninstall}
% \changes{v1.0e}{2015/10/02}{Function added}
%   Unlike at the \TeX{} level, we have to provide a back-out mechanism here
%   at the same time as the rest of the code. This is not meant for use by
%   anything other than \textsf{latexrelease}: as such this is
%   \emph{deliberately} not documented for users!
%    \begin{macrocode}
local function uninstall()
  module_info(
    "luatexbase",
    "Uninstalling kernel luatexbase code"
  )
  callback.register = callback_register
  luatexbase = nil
end
luatexbase.uninstall = uninstall
%    \end{macrocode}
% \end{macro}
% \begin{macro}{mlist_to_hlist}
% \changes{v1.1l}{2020/02/02}{|pre/post_mlist_to_hlist| added}
% \changes{v1.1y}{2022/08/13}{Use shared\_callback system for pre/post/mlist_to_hlist}
% \changes{v1.2c}{2023/08/03}{Fix callback type of post_mlist_to_hlist_callback}
%   To emulate these callbacks, the ``real'' |mlist_to_hlist| is replaced by a
%   wrapper calling the wrappers before and after.
%    \begin{macrocode}
create_callback('pre_mlist_to_hlist_filter', 'list')
create_callback('mlist_to_hlist', 'exclusive', node.mlist_to_hlist)
create_callback('post_mlist_to_hlist_filter', 'reverselist')
function shared_callbacks.mlist_to_hlist.handler(head, display_type, need_penalties)
  local current = call_callback("pre_mlist_to_hlist_filter", head, display_type, need_penalties)
  if current == false then
    flush_list(head)
    return nil
  end
  current = call_callback("mlist_to_hlist", current, display_type, need_penalties)
  local post = call_callback("post_mlist_to_hlist_filter", current, display_type, need_penalties)
  if post == false then
    flush_list(current)
    return nil
  end
  return post
end
%    \end{macrocode}
% \end{macro}
% \endgroup
%
%    \begin{macrocode}
%</lua>
%    \end{macrocode}
%
% Reset the catcode of |@|.
%    \begin{macrocode}
%<tex>\catcode`\@=\etatcatcode\relax
%    \end{macrocode}
%
%
% \Finale
