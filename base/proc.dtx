% \iffalse meta-comment
%
% Copyright (C) 1993-2021
% The LaTeX Project and any individual authors listed elsewhere
% in this file.
%
% This file is part of the LaTeX base system.
% -------------------------------------------
%
% It may be distributed and/or modified under the
% conditions of the LaTeX Project Public License, either version 1.3c
% of this license or (at your option) any later version.
% The latest version of this license is in
%    https://www.latex-project.org/lppl.txt
% and version 1.3c or later is part of all distributions of LaTeX
% version 2008 or later.
%
% This file has the LPPL maintenance status "maintained".
%
% The list of all files belonging to the LaTeX base distribution is
% given in the file `manifest.txt'. See also `legal.txt' for additional
% information.
%
% The list of derived (unpacked) files belonging to the distribution
% and covered by LPPL is defined by the unpacking scripts (with
% extension .ins) which are part of the distribution.
%
% \fi
%
% \iffalse
% Copyright (C) 1991,1992 by Leslie Lamport and
% Copyright (C) 1994-2020 by the LaTeX Project and Johannes Braams
%
% \changes{v1.1}{1994/01/25}{Made `proc' a document class and provide
%    a .sty file for compatibility mode}
% \changes{v1.2}{1994/01/26}{Removed the setting of a number of
%    variables that became superfluous by passing the twocolumn option
%    to the article class}
% \changes{v1.3}{1994/03/01}{Moved driver further up; added overview
%    of docstrip modules; added \cs{CheckSum}, \cs{StopEventually} and
%    \cs{Finale}}
% \changes{v1.3e}{1994/05/22}{Use new warning and error commands}
% \changes{v1.3f}{1994/05/26}{Inserted missing 'v's in \cs{changes}
%    entries}
% \changes{v1.3f}{1994/05/26}{Moved identification and driver to the
%    front of the file; these sections now no longer print}
%
% \subsection{Identification}
%
%    This document class can only be used with \LaTeXe, so it is made
%    sure that an appropriate message is displayed when another \TeX{}
%    format is used.
% \changes{v1.3l}{1995/11/30}{Added date of \LaTeX\ format to argument
%    of \cs{NeedsTeXFormat}}
%    \begin{macrocode}
%<+class|style>\NeedsTeXFormat{LaTeX2e}[1995/12/01]
%    \end{macrocode}
%
%    Announce the document class and its version.
% \changes{v1.3c}{1994/04/15}{Added missing / guard}
% \changes{v1.3d}{1994/05/01}{Removed the use of variables such as
%    \cs{fileversion}}
% \changes{v1.3b}{1994/03/16}{Removed use of \cs{wlog}, no longer
%    needed}
% \changes{v1.3f}{1994/05/26}{Removed \cs{typeout} from .sty file;
%    changed description}
% \changes{v1.3h}{1994/05/26}{The second argument to \cs{ProvidesFile}
%    wasn't closed in the driver and the style file}
%
%    \begin{macrocode}
%<*driver>
\ProvidesFile{proc.drv}
%</driver>
%<+class>\ProvidesClass{proc}
%<+style>\ProvidesFile{proc.sty}
              [2020/11/23 v1.3l
%<+class>               Standard LaTeX document class]
%<+style>               LaTeX 2.09 compatibility style option `proc']
%    \end{macrocode}
%
% \subsection{The documentation driver file}
%
%    This bit of code contains the documentation driver file for
%    \TeX{}, i.e., the file that will produce the documentation you
%    are currently reading. It can be extracted from this file by the
%    {\sc docstrip} program.
%    \begin{macrocode}
%<*driver>
]
\documentclass{ltxdoc}
\GetFileInfo{proc.drv}
\begin{document}
\newcommand\Lopt[1]{\textsf {#1}}
\DocInput{proc.dtx}
\end{document}
%</driver>
%    \end{macrocode}
% \fi
%
% \title{Producing proceedings articles with \LaTeXe\thanks
%    {File version \fileversion, dated \filedate}}
%
% \author{%
%  Johannes Braams\and
%  David Carlisle\and
%  Alan Jeffrey\and
%  Leslie Lamport\and
%  Frank Mittelbach\and
%  Chris Rowley\and
%  Rainer Sch\"opf}
%
% \date{\filedate}
%
% \MaintainedByLaTeXTeam{latex}
% \maketitle
%
% \section{Description}
%
%    This \LaTeXe\ document class is designed to produce camera-ready
%    conference proceedings. It is based upon the document class
%    \texttt{article}, but modifies some of its layout decisions.
%    It also adds one command.
%
%    For compatibility with \LaTeX2.09 a \texttt{.sty} version is
%    provided because \texttt{proc} used to be a style file. The
%    \texttt{.sty} file can \emph{not} be used as a package however.
%
% \DescribeMacro\copyrightspace
%    The |\copyrightspace| command is used to produce a blank space in
%    the first column where a copyright notice belongs.  Note that it
%    should appear after any |\footnote| commands that produce
%    footnotes for the first column.
%
% \MaybeStop{}
%
% \section{The {\sc docstrip} modules}
%
% The following modules are used in the implementation to direct
% {\sc docstrip} in generating the external files:
% \begin{center}
% \begin{tabular}{ll}
%   class   & produce the proc document class\\
%   sty     & produce proc.sty for compatibility mode\\
%   driver  & produce a documentation driver file \\
% \end{tabular}
% \end{center}
%
% \section{Implementation}
%
%    In this case a check is needed to see whether \LaTeXe\ runs in
%    compatibility mode.
% \changes{v1.3i}{1994/05/27}{Added missing docstrip guard}
%    \begin{macrocode}
%<*style>
\if@compatibility
%    \end{macrocode}
%    In this case processing silently continues, but in the |\else|
%    case a warning is issued and reading of this file stops.
%    \begin{macrocode}
\else
  \ClassWarningNoLine{proc}{^^J\@spaces%
    You requested the use of `proc' as a package^^J\@spaces%
    but it has been turned into a document class.^^J\@spaces%
    Please change your file to use `proc' as a class.^^J}
  \expandafter\endinput
\fi
%</style>
%    \end{macrocode}
%
%    Paper sizes smaller than A4 are not supported.
%    \begin{macrocode}
%<*class>
\DeclareOption{a5paper}
    {\ClassError{proc}{Option `a5paper' not supported}{}}
\DeclareOption{b5paper}
    {\ClassError{proc}{Option `b5paper' not supported}{}}
%    \end{macrocode}
%
%    Also the \Lopt{onecolumn} option is disabled in this document
%    class.
%    \begin{macrocode}
\DeclareOption{onecolumn}%
    {\ClassError{proc}{Option `onecolumn' not supported}{}}
%    \end{macrocode}
%
%    Proceedings articles never have separate titlepages, therefore
%    this option is disabled.
%    \begin{macrocode}
\DeclareOption{titlepage}%
    {\ClassError{proc}{Option `titlepage' not supported}{}}
%    \end{macrocode}
%
%    Always print in two column mode.
%    \begin{macrocode}
\PassOptionsToClass{twocolumn}{article}
%    \end{macrocode}
%
%    Any other options are passed on to the article class on which
%    this document class is based.
%    \begin{macrocode}
\DeclareOption*{\PassOptionsToClass{\CurrentOption}{article}}
%    \end{macrocode}
%
%    Now we can process the options
%    \begin{macrocode}
\ProcessOptions
%    \end{macrocode}
%
%    and load the article document class.
%    \begin{macrocode}
\LoadClass{article}
%</class>
%    \end{macrocode}
%
% \section{Document Layout}
%
%    Proceedings articles have a different layout than normal
%    articles. Therefore some of the layout parameters are given
%    different values here.
%
% \subsection{Page Layout}
%
% \subsubsection{Vertical spacing}
%
% \begin{macro}{\footskip}
%    The distance from the baseline of the box which contains the
%    running footer to the baseline of last line of text is controlled
%    by the |\footskip|.
%    \begin{macrocode}
%<*class|style>
\setlength\footskip{75\p@}
%    \end{macrocode}
% \end{macro}
%
% \subsubsection{The dimension of text}
%
% \begin{macro}{\textwidth}
%    When \LaTeXe\ runs in compatibility mode it has to be made sure
%    that the dimensions of the printed area are not different from
%    what the user was used to see.
%
%    \begin{macrocode}
\if@compatibility
  \setlength\textwidth{6.75in}
\else
  \setlength\textwidth{\paperwidth}
  \addtolength\textwidth{-126\p@}
  \@settopoint\textwidth
\fi
%    \end{macrocode}
%
%    Here the width of the text a little is modified to be a whole
%    number of points.
%    \begin{macrocode}
\if@compatibility
\else
  \@settopoint\textwidth
\fi
%    \end{macrocode}
% \end{macro}
%
% \begin{macro}{\textheight}
%    Now that we have computed the width of the text, we have to take
%    care of the height. The |\textheight| is the height of text
%    (including footnotes and figures, excluding running head and
%    foot).
%
%    \begin{macrocode}
\if@compatibility
  \setlength\textheight{9in}
\else
  \ifcase\@ptsize
    \setlength\textheight{53\baselineskip}
  \or
    \setlength\textheight{47\baselineskip}
  \or
    \setlength\textheight{44\baselineskip}
  \fi
\fi
%    \end{macrocode}
%
%    The first line on the page has a height of |\topskip|.
%    \begin{macrocode}
\advance\textheight by \topskip
%    \end{macrocode}
% \end{macro}
%
% \subsubsection{Margins}
%
% \begin{macro}{\oddsidemargin}
% \begin{macro}{\evensidemargin}
%    The width of the text is quite large, therefore the margins
%    have a negative width.
%    \begin{macrocode}
\setlength\oddsidemargin{-10\p@}
\setlength\evensidemargin{-10\p@}
%    \end{macrocode}
% \end{macro}
% \end{macro}
%
% \begin{macro}{\marginparwidth}
%    Because we have such a large |\textwidth| there is no real room
%    for notes in the margins.
%    \begin{macrocode}
\setlength\marginparwidth {\z@}
%    \end{macrocode}
% \end{macro}
%
% \begin{macro}{\topmargin}
%    The |\topmargin| is the distance between the top of `the
%    printable area' --which is 1 inch below the top of the paper--
%    and the top of the box which contains the running head.
%    \begin{macrocode}
\setlength\topmargin{-47\p@}
%    \end{macrocode}
% \end{macro}
%
% \begin{macro}{\columnsep}
%    Proceedings articles are always set in |\twocolumn| mode,
%    therefore the distance between the two columns  (|\columnsep|)is
%    set to a non zero value.
%    \begin{macrocode}
\setlength\columnsep{.375in}
%    \end{macrocode}
% \end{macro}
%
% \subsection{The page style}
%
%    The user can use |\markright| (or |\markboth|) to put an
%    identification in the lower left side of the page.  The page
%    number is printed in the lower right side of the page.
%
%    This works as long as he doesn't use either |\pagestyle| or
%    |\thispagestyle| commands.
% \changes{v1.3f}{1994/05/26}{changed \cs{rm} to
%    \cs{normalfont}\cs{rmfamily}}
% \changes{v1.3j}{1995/05/26}{Removed \cs{rmfamily} (PR 1578)}
%    \begin{macrocode}
\let\@oddhead\@empty
\let\@evenhead\@empty
\def\@oddfoot{\normalfont\rightmark \hfil
              \pagename{} \thepage}
\def\@evenfoot{\@oddfoot}
%    \end{macrocode}
%
%
% \section{Document Markup}
%
% \subsection{The title}
%
% \begin{macro}{\maketitle}
% \changes{v1.3k}{1995/07/18}{Modify footnote handling (PR 1735)}
%    The definition of |\maketitle| here is basically the same as the
%    one in the document class \texttt{article}, but the pagestyle is
%    \emph{not} set to \texttt{plain}.
%
%    \begin{macrocode}
\def\maketitle{\par
 \begingroup
   \renewcommand\thefootnote{\fnsymbol{footnote}}%
    \def\@makefnmark{\rlap{\@textsuperscript{\normalfont\@thefnmark}}}%
     \long\def\@makefntext##1{\parindent 1em\noindent
             \hb@xt@1.8em{%
                 \hss\@textsuperscript{\normalfont\@thefnmark}}##1}%
%    \end{macrocode}
%    This document class always uses two column mode but the title is
%    set to the full width of the text.
%    \begin{macrocode}
   \twocolumn[\@maketitle]%
   \@thanks
 \endgroup
 \setcounter{footnote}{0}%
 \let\maketitle\relax
 \let\@maketitle\relax
 \gdef\@thanks{}\gdef\@author{}\gdef\@title{}\let\thanks\relax}
%    \end{macrocode}
% \end{macro}
%
% \begin{macro}{\@maketitle}
%    This macro takes care of formatting the title information.
%
%    The information is set inside |\vbox| with a width of
%    |\textwidth| and a height of 2.25in. The text is vertically and
%    horizontally centered.
%    \begin{macrocode}
\def\@maketitle{%
  \vbox to 2.25in{%
    \hsize\textwidth
    \linewidth\hsize
    \vfil
    \centering
    {\LARGE \@title \par}
    \vskip 2em
    {\large \begin{tabular}[t]{c}\@author \end{tabular}\par}
    \vfil}}
%    \end{macrocode}
% \end{macro}
%
% \begin{macro}{\copyrightspace}
%    The |\copyrightspace| command is used to produce a blank space in
%    the first column where a copyright notice belongs.  It works by
%    producing a blank footnote of the appropriate size.
%
%    \begin{macrocode}
\def\copyrightspace{%
  \footnotetext[0]{\mbox{}\vrule \@height 97\p@ \@width \z@}}
%    \end{macrocode}
% \end{macro}
%
% \subsection{Modifying environments}
%
% \subsubsection{Abstract}
%
% \begin{macro}{\abstract}
%    For this document class the definition of the abstract
%    environment is quit simple.
%
%    \begin{macrocode}
\renewenvironment{abstract}%
  {\section*{\abstractname}}
  {\par}
%    \end{macrocode}
% \end{macro}
%
% \section{Initialization}
%
% \subsection{Words}
%
% \begin{macro}{\pagename}
%    This document class is for documents prepared in the English
%    language. To prepare a version for another language, one English
%    word must be replaced.  It is defined below in a command name.
%    \begin{macrocode}
\newcommand\pagename{Page}
%</class|style>
%    \end{macrocode}
% \end{macro}
%
% \Finale
%
\endinput
