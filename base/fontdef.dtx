% \iffalse meta-comment
%
% Copyright (C) 1993-2023
% The LaTeX Project and any individual authors listed elsewhere
% in this file.
%
% This file is part of the LaTeX base system.
% -------------------------------------------
%
% It may be distributed and/or modified under the
% conditions of the LaTeX Project Public License, either version 1.3c
% of this license or (at your option) any later version.
% The latest version of this license is in
%    https://www.latex-project.org/lppl.txt
% and version 1.3c or later is part of all distributions of LaTeX
% version 2008 or later.
%
% This file has the LPPL maintenance status "maintained".
%
% The list of all files belonging to the LaTeX base distribution is
% given in the file `manifest.txt'. See also `legal.txt' for additional
% information.
%
% The list of derived (unpacked) files belonging to the distribution
% and covered by LPPL is defined by the unpacking scripts (with
% extension .ins) which are part of the distribution.
%
% \fi
% \iffalse
%%%   From File: fontdef.dtx
%<*!latexrelease>
%<*dtx>
           \ProvidesFile{fontdef.dtx}
%</dtx>
%<text,   >\ProvidesFile{fonttext.ltx}
%<math,   >\ProvidesFile{fontmath.ltx}
%<+cfgtext>\ProvidesFile{fonttext.cfg}
%<+cfgmath>\ProvidesFile{fontmath.cfg}
%<+cfgprel>\ProvidesFile{preload.cfg}
%<driver, >\ProvidesFile{fontdef.drv}
% \fi
% \ProvidesFile{fontdef.dtx}
           [2024/02/09 v3.0i LaTeX Kernel
% \iffalse
%<text,   >   (Text font setup)
%<math,   >   (Math font setup)
%<+cfgtext>   (Uncustomized text font setup)
%<+cfgmath>   (Uncustomized math font setup)
%<+cfgprel>   (Uncustomized preload font setup)
%<*dtx>
% \fi
(font setup)
%\iffalse
%</dtx>
%\fi
]
% \iffalse
%</!latexrelease>
%\fi
%
%\iffalse        This is a META comment
%
% File `fontdef.dtx'.
% Copyright (C) 1989-1999 Frank Mittelbach and Rainer Sch\"opf,
% All rights reserved.
%
%\fi
%
% \changes{v2.1a}{1993/12/01}{Update for LaTeX2e}
% \changes{v2.2a}{1994/10/14}{New coding}
% \changes{v2.2i}{1994/12/02}{Commented out \cs{ldots}. ASAJ.}
% \changes{v2.2j}{1995/05/11}{Updates to some plain macros}
% \changes{v2.2l}{1995/10/03}{\cs{@@sqrt} from patch file for /1701}
% \changes{v2.2o}{1996/05/17}{\cs{@@sqrt} removed, at last}
% \changes{v2.2p}{1996/11/20}{lowercase fd and enc.def file names /1044}
% \changes{v2.2x}{1999/01/05}{Need special protection for character
%     \texttt{\char62} in \cs{changes} entry.}
%
% \title{The \texttt{fontdef.dtx} file\thanks
%         {This file has version number \fileversion, dated \filedate}}
% \author{Frank Mittelbach \and Rainer Sch\"opf}
%
% \def\dst{{\normalfont\scshape docstrip}}
% \setcounter{StandardModuleDepth}{1}
%
%
% \MaintainedByLaTeXTeam{latex}
% \maketitle
%
% \section{Introduction}
%
% This file is used to generate the files \texttt{fonttext.ltx} (text
% font declarations) and \texttt{fontmath.ltx} (math font
% declarations), which are used during the format generation.  It
% contains the declaration of the standard text encodings used at the
% site as well as a minimal subset of font shape groups that NFSS will
% look at to ensure that the specified encodings are valid.
%
% The math part contains the setup for math encodings as well as the
% default math symbol declarations that belong to the encoding.
%
% It is possible to change this setup (by using other fonts, or
% defaults) without losing the ability to
% process documents written at other sites. Portability in this sense
% means that a document will compile without errors. It does not mean,
% however, that identical output will be produced. For this it is
% necessary that the distributed setup is used at both installations.
%
% \section{Customization}
%
% You are not allowed to change this source file!  If you want to
% change the default encodings and/or the font shape groups preloaded
% you should create a copy of \texttt{fonttext.ltx}
% under the name \texttt{fonttext.cfg} and change this copy. If
% \LaTeXe{} finds a file of this name it will use it, otherwise it
% uses the standard file which is \texttt{fontdef.ltx}.
%
% If you don't plan to use Computer Modern much or at all, it might
% (!)  be a good idea to make your own \texttt{fonttext.cfg}. Look at
% the comments below (docstrip module `text') to see what should
% go into such a file.
%
% To change the math font setup use a copy of \texttt{fontmath.ltx}
% under the name \texttt{fontmath.cfg} and change this copy. However,
% dealing with this interface is even more a job for an expert than
% changing the text font setup --- in short, we don't encourage either.
%
% \begin{quote}
%   \textbf{Warning:} please note that we don't support customised
%   \LaTeX{} versions. Thus, before sending in a bug report please try
%   your test file with a \LaTeX{} format which is not customised and
%   send in the log from that version (unless the problem goes away).
% \end{quote}
%
% Please note: the following standard encodings  have to
% be defined in all local variants of \texttt{font....cfg} to guarantee
% that all \LaTeX{} installations behave in the same way.
% \begin{center}
% \begin{tabular}{ll}
%   |T1|      &    Cork \TeX{} text encoding \\
%   |OT1|      &   old \TeX{} text encoding \\
%   |U|        &   unknown encoding \\
%   |OML|      &   old \TeX{} math letters encoding \\
%   |OMS|      &   old \TeX{} math symbols encoding \\
%   |OMX|      &   old \TeX{} math extension symbols encoding\\
%   |TU|      &   Unicode
% \end{tabular}
% \end{center}
% Notice that some of these encodings are `old' in the sense that we
% hope that they will be superseded soon by encoding standards defined
% by the \TeX{} user community. Therefore this set of default encodings
% may change in the future.
%
% The first candidate is |OT1| which will soon be replaced by |T1|, the
% official \TeX{} text encoding.
%
% \begin{quote}\textbf{Warning:}
% If you add additional encodings to this file there is no guarantee
% any longer that files processable at your installation will also be
% processable at other installations. Thus, if you make use of
% such an encoding in your document, e.g.~if you intend to typeset in
% Cyrillic (|OT2| encoding), you need to specify this encoding in the
% preamble of your document prior to sending it to another
% installation. Once the encoding is specified in that place in your
% document, the document is processable at all \LaTeX{} installations
% (provided they have suitable fonts installed).
%
% For this reason we suggest that you define a short package file that
% sets up an additional encoding used at your site (rather than
% putting the encoding into this file) since this package can easily
% be shipped with your document.
% \end{quote}
%
%
% \MaybeStop{}
%
% \section{The \texttt{docstrip} modules}
%
% The following modules are used to direct \texttt{docstrip} in
% generating external files:
% \begin{center}
% \begin{tabular}{ll}
%   driver & produce a documentation driver file \\
%   text   & produce the file \texttt{fonttext.ltx}\\
%   math   & produce the file \texttt{fontmath.ltx}\\
%   cfgtext   & produce a dummy \texttt{fonttext.cfg} file\\
%   cfgmath   & produce a dummy \texttt{fontmath.cfg} file\\
% \end{tabular}
% \end{center}
% A typical \texttt{docstrip} command file would then have entries like:
% \begin{verbatim}
%\generateFile{fonttext.ltx}{t}{\from{fontdef.dtx}{text}}
%\end{verbatim}
%
%
% \section{A driver for this document}
%
% The next bit of code contains the documentation driver file for
% \TeX{}, i.e.~the file that will produce the documentation you are
% currently reading. It will be extracted from this file by the
% \dst{} program.
%    \begin{macrocode}
%<*driver>
\documentclass{ltxdoc}
\GetFileInfo{fontdef.dtx}
\begin{document}
   \DocInput{fontdef.dtx}
\end{document}
%</driver>
%    \end{macrocode}
%
%
%
% \section{The \texttt{fonttext.ltx} file}
%
%    The identification is done earlier on with a |\ProvidesFile|
%    declaration.
%    \begin{macrocode}
%<*text>
\typeout{=== Don't modify this file, use a .cfg file instead ===^^J}
%    \end{macrocode}
%
%  \subsection{Encodings}
%
%    This file declares the standard encodings for text and math
%    fonts. All others should be declared in packages or in the
%    documents directly.
%
%    For every text encoding there are normally a number of encoding
%    specific commands, e.g.~accents, special characters, etc.  (The
%    definition for such a command might have to change when the
%    encoding is changed, because the character is in a different
%    position, or not available at all, or the accent is produced in a
%    different way.)  This is handled by a general mechanism which is
%    described in \texttt{ltoutenc.dtx}.
%
%    By convention, text  encoding specific declarations, including the
%    |\DeclareFontEncoding| declaration, are kept in separate file of
%    the form \meta{enc}\texttt{enc.def}, e.g.~\texttt{ot1enc.def}. This
%    allows other applications to make use of the declarations as
%    well.
%
%    Similar to the default encoding, the loading of the encoding
%    files for the two major text encodings shouldn't be changed.
%    In particular, the \texttt{inputenc} package depends on this.
% \changes{v2.2s}{1997/12/20}{Added documentation}
%
% \changes{v2.1d}{1994/01/05}{Removed nf prefix from file names.}
% \changes{v2.1f}{1994/05/14}{Removed .def files.}
% \changes{v2.1g}{1994/05/16}{Removed \cs{DeclareFontEncoding} for ot1
%                             and t1 and input .def files instead}
% \changes{v2.2c}{1994/10/25}{Added OMSenc.def}
% \changes{v2.2d}{1994/10/31}{Added OMLenc.def ...}
% \changes{v2.2e}{1994/10/31}{... and moved further down}
% \changes{v2.2f}{1994/11/07}{(DPC) Updated to use \cs{ProvidesFile}}
% \changes{v2.2h}{1994/11/16}{(DPC) Removed \cmd\{ and \cmd\}}
% \changes{v3.0a}{2016/12/03}{(DPC) Default to TU encoding for Unicode TeX engines}
%    \begin{macrocode}
\input {omlenc.def}
\input {omsenc.def}
%    \end{macrocode}
%    Documents containing a lot of accented characters should really
%    be using T1 fonts. We therefore load this last so that T1 encoding
%    specific commands are executed as fast as possible (encoding
%    files are no longer reloaded in \texttt{fontenc}.
% \changes{v3.0f}{2020/01/25}{Load t1enc.def last (gh/255)}
%    \begin{macrocode}
\input {ot1enc.def}
\input  {t1enc.def}
%    \end{macrocode}
%
%    \begin{macrocode}
\input{ts1enc.def}
%    \end{macrocode}
%    
%
%
% \changes{v3.0a}{2016/12/03}{(DPC) Default to TU encoding for Unicode TeX engines}
%    \begin{macrocode}
\ifx\Umathcode\@undefined
%    \end{macrocode}
%
%    We then set the default text font encoding. This will
%    hopefully change some day to |T1|. This setting should \emph{not}
%    be changed to produce a portable format.
%    \begin{macrocode}
\fontencoding{OT1}
%    \end{macrocode}
%
%    The initial \texttt{fontenc} package load list if an 8-bit \TeX{}
%    engine is used:
% \changes{v3.0g}{2020/02/11}{Provide value for \cs{@fontenc@load@list} (gh/273)}
%    \begin{macrocode}
\def\@fontenc@load@list{\@elt{T1,OT1}}
%    \end{macrocode}
%
%
%
%
%    \begin{macrocode}
\def\rmsubstdefault{cmr}
\def\sfsubstdefault{cmss}
\def\ttsubstdefault{cmtt}
\LoadFontDefinitionFile{TS1}{cmr}
%    \end{macrocode}
%
%    \begin{macrocode}
\else
%    \end{macrocode}
% Unicode.
%    \begin{macrocode}
\input {tuenc.def}
\fontencoding{TU}
%    \end{macrocode}
%
%    The initial \texttt{fontenc} package load list if a Unicode
%    engine is used:
% \changes{v3.0g}{2020/02/11}{Provide value for \cs{@fontenc@load@list} (gh/273)}
%    \begin{macrocode}
\def\@fontenc@load@list{\@elt{TU}}
%    \end{macrocode}
%
%    \begin{macrocode}
\DeclareFontSubstitution{TU}{lmr}{m}{n}
\LoadFontDefinitionFile{TU}{lmr}
\LoadFontDefinitionFile{TU}{lmss}
\LoadFontDefinitionFile{TU}{lmtt}
%    \end{macrocode}
%
%    \begin{macrocode}
\def\rmsubstdefault{lmr}
\def\sfsubstdefault{lmss}
\def\ttsubstdefault{lmtt}
\LoadFontDefinitionFile{TS1}{lmr}
%    \end{macrocode}
%
%    \begin{macrocode}
\DeclareFontSubstitution{TU}{lmr}{m}{n}
%    \end{macrocode}
% End of Unicode branch.
%    \begin{macrocode}
\fi
%    \end{macrocode}
%
%    If different encodings for text fonts are in use one could put
%    the common setup into |\DeclareFontEncodingDefaults|. There is
%    now a better mechanism so using this interface is discouraged!
%    \begin{macrocode}
\DeclareFontEncodingDefaults{}{}
%    \end{macrocode}
%
%    Then we define the default substitution for every encoding.
%    This release of \LaTeXe{} assumes that the ec fonts are
%    available. It is possible to change this to point to some other
%    font family (e.g., Times with the appropriate encoding if it is
%    available) without making documents non-portable. However, in
%    such a case documents will produce different page breaks at other
%    sites. The substitution defaults can all be changed without
%    losing portability as long as there are font shape definitions
%    for the selected substitutions.
%    \begin{macrocode}
\DeclareFontSubstitution{T1}{cmr}{m}{n}
\DeclareFontSubstitution{OT1}{cmr}{m}{n}
%    \end{macrocode}
%
%    For every encoding declaration, \LaTeXe{} will try to verify that
%    the given substitution information makes sense, i.e.~that it is
%    impossible to go into an endless loop if font substitution
%    happens. This is done at the moment the |\begin{document}| is
%    encountered. \LaTeXe{} will then check that for every encoding the
%    substitution defaults form a valid font shape group, which means
%    that it will check if there is a |\DeclareFontShape| declaration
%    for this combination. We will therefore load the corresponding
%    |.fd| files now. If we don't do this they would be loaded at
%    verification time (i.e.~at |\begin{document}| which would delay
%    processing unnecessarily.
%
%    \begin{quote}
%       \textbf{Warning:} Please note that this means that you have to
%       regenerate the format whenever you change any of these
%       \texttt{.fd} files since \LaTeXe{} will not read \texttt{.fd}
%       files if it already knows about the encoding/family
%       combination.
%    \end{quote}
%
% \changes{v2.2m}{1995/11/01}{add \cs{nfss@catcodes} for internal/1932}
% The |\nfss@catcodes| ensures that white space is ignored in any
% definitions made in the fd files.
%    \begin{macrocode}
\begingroup
\nfss@catcodes
\input  {t1cmr.fd}
\input  {ot1cmr.fd}
\endgroup
%    \end{macrocode}
%
%    We also load some other font definition files which are normally
%    needed in a document. This is only done for processing speed and
%    you can comment the next two lines out to save some memory. If
%    necessary these files are then loaded when your document is
%    processed. (Loading |.fd| files is a less drastic step compared
%    to preloading fonts because the number of fonts is limited 255 at
%    (nearly) every \TeX{} installation, while the amount of main memory
%    is not a limiting factor at most installations.)
%
%    \begin{macrocode}
\begingroup
\nfss@catcodes
\input {ot1cmss.fd}
\input {ot1cmtt.fd}
\endgroup
%    \end{macrocode}
%
%    Even with all the precautions it is still possible that NFSS will
%    run into problems, for example, when a |.fd| file contains
%    corrupted data. To guard against such cases NFSS has a very
%    low-level fallback font that is installed with the following line.
%    \begin{macrocode}
\DeclareErrorFont{OT1}{cmr}{m}{n}{10}
%    \end{macrocode}
%    This means, ``if everything else fails use Computer Modern Roman
%    normal shape at 10pt in the old text encoding''.
%    You can change the font used but the encoding should be the same
%    as the one specified with |\fontencoding| above.
%
%
% \subsection{Defaults}
%
%    To allow the use of |\rmfamily|, |\sffamily|, etc.\ in documents
%    even if non-standard families are used we provide nine macros
%    which hold the name of the corresponding families, series, and so
%    on. This makes it easy to use other font families (like Times
%    Roman, etc.). One simply has to redefine these defaults.
%
%    All these hooks have to be defined in this file but you can
%    change their meaning (except for |\encodingdefault|) without
%    making documents non-portable.
%
%
% \begin{macro}{\encodingdefault}
% \begin{macro}{\rmdefault}
% \begin{macro}{\sfdefault}
% \begin{macro}{\ttdefault}
%    The following three definitions set up the meaning for
%    |\rmfamily|, |\sffamily|, and |\ttfamily|.
%    \begin{macrocode}
\ifx\Umathcode\@undefined
\newcommand\encodingdefault{OT1}
\newcommand\rmdefault{cmr}
\newcommand\sfdefault{cmss}
\newcommand\ttdefault{cmtt}
\else
\newcommand\encodingdefault{TU}
\newcommand\rmdefault{lmr}
\fontfamily{\rmdefault}
\newcommand\sfdefault{lmss}
\newcommand\ttdefault{lmtt}
\fi
%</text>
%<latexrelease>\IncludeInRelease{2017/01/01}%
%<latexrelease>                 {\encodingdefault}{TU encoding default}%
%<latexrelease>\ifx\Umathcode\@undefined
%<latexrelease>\renewcommand\encodingdefault{OT1}
%<latexrelease>\fontencoding{\encodingdefault}
%<latexrelease>\renewcommand\rmdefault{cmr}
%<latexrelease>\fontfamily{\rmdefault}
%<latexrelease>\renewcommand\sfdefault{cmss}
%<latexrelease>\renewcommand\ttdefault{cmtt}
%<latexrelease>\else
%<latexrelease>\renewcommand\encodingdefault{TU}
%<latexrelease>%done in everyjob\fontencoding{\encodingdefault}
%<latexrelease>\renewcommand\rmdefault{lmr}
%<latexrelease>\fontfamily{\rmdefault}
%<latexrelease>\renewcommand\sfdefault{lmss}
%<latexrelease>\renewcommand\ttdefault{lmtt}
%<latexrelease>\fi
%<latexrelease>\EndIncludeInRelease
%<latexrelease>\IncludeInRelease{0000/00/00}%
%<latexrelease>                 {\encodingdefault}{TU encoding default}%
%<latexrelease>\fontencoding{OT1}
%<latexrelease>\renewcommand\encodingdefault{OT1}
%<latexrelease>\fontencoding{\encodingdefault}
%<latexrelease>\renewcommand\rmdefault{cmr}
%<latexrelease>\fontfamily{\rmdefault}
%<latexrelease>\renewcommand\sfdefault{cmss}
%<latexrelease>\renewcommand\ttdefault{cmtt}
%<latexrelease>\EndIncludeInRelease
%<*text>
%    \end{macrocode}
% \end{macro}
% \end{macro}
% \end{macro}
% \end{macro}
%
% \begin{macro}{\bfdefault}
% \begin{macro}{\mddefault}
%    Series changing commands are influenced by the following hooks.
% \changes{v3.0e}{2019/12/17}{Set \cs{bfdefault} to ``b''}
%    \begin{macrocode}
\newcommand\bfdefault{b}  % overwritten below (for rollback)
\newcommand\mddefault{m}  % overwritten below (for rollback)
%    \end{macrocode}
% \end{macro}
% \end{macro}
%
% \begin{macro}{\itdefault}
% \begin{macro}{\sldefault}
% \begin{macro}{\scdefault}
% \begin{macro}{\updefault}
%
%    Shape changing commands use the following hooks.
% \changes{v3.0e}{2019/12/17}{Set \cs{updefault} to ``up''}
%    \begin{macrocode}
\newcommand\itdefault{it}
\newcommand\sldefault{sl}
\newcommand\scdefault{sc}
\newcommand\updefault{up}  % overwritten below (for rollback)
%    \end{macrocode}
% \end{macro}
% \end{macro}
% \end{macro}
% \end{macro}
%

%    \begin{macrocode}
%</text>
%<*text|latexrelease>
%<latexrelease>\IncludeInRelease{2020/02/02}%
%<latexrelease>                 {\updefault}{font defaults change}%
%    \begin{macrocode}
\renewcommand\updefault{up}
%    \end{macrocode}
%    We append \cs{@empty} to the series value so that we can detect
%    if it got changed via \cs{def} or \cs{renewcommand} later.
% \changes{v3.0h}{2020/03/19}{Support legacy use of \cs{bfdefault}
%        and \cs{mddefault} (gh/306)}
%    \begin{macrocode}
\renewcommand\bfdefault{b\@empty}
\renewcommand\mddefault{m\@empty}
%    \end{macrocode}
%    
%    \begin{macrocode}
\let\bfdefault@previous\bfdefault
\let\mddefault@previous\mddefault
%</text|latexrelease>
%<latexrelease>\EndIncludeInRelease
%<latexrelease>\IncludeInRelease{0000/00/00}%
%<latexrelease>                 {\updefault}{font defaults change}%
%<latexrelease>
%<latexrelease>\renewcommand\updefault{n}
%<latexrelease>\renewcommand\bfdefault{bx}
%<latexrelease>
%<latexrelease>\let\bfdefault@previous\undefined
%<latexrelease>\let\mddefault@previous\undefined
%<latexrelease>\EndIncludeInRelease
%<*text>
%    \end{macrocode}
%
% \begin{macro}{\familydefault}
% \begin{macro}{\seriesdefault}
% \begin{macro}{\shapedefault}
%    Finally we have the hooks that describe the behaviour of
%    the |\normalfont| command. To stay portable, the definition of
%    |\encodingdefault| should \emph{not} be changed and should match
%    the setting above for |\fontencoding|. All other values can be
%    set according to your taste.
% \changes{v3.0a}{2016/12/03}{(DPC) Default to TU encoding for Unicode TeX engines}
%    \begin{macrocode}
\newcommand\familydefault{\rmdefault}
\newcommand\seriesdefault{\mddefault}
%    \end{macrocode}
%    In previous releases \cs{shapedefault} pointed to \cs{updefault}
%    which resolved to \texttt{n}, but these days that is no longer
%    the case (and \texttt{up} is wrong when you want to do a
%    reset. So we now use \texttt{n} explicitly.
% \changes{v3.0e}{2019/12/17}{Set \cs{shapedefault} explicitly to ``n''}
%    \begin{macrocode}
\newcommand\shapedefault{n}
%    \end{macrocode}
% \end{macro}
% \end{macro}
% \end{macro}
%
%
%    This finishes the low-level setup in \texttt{fonttext.ltx}.
%    \begin{macrocode}
%</text>
%    \end{macrocode}
%
%
%
%
% \section{The \texttt{fontmath.ltx} file}
%
%    The identification is done earlier on with a |\ProvidesFile|
%    declaration.
%    \begin{macrocode}
%<*math>
\typeout{=== Don't modify this file, use a .cfg file instead ===^^J}
%    \end{macrocode}
%
% \subsection{The font encodings used}
%
%    \begin{macrocode}
\DeclareFontEncoding{OML}{}{}
\DeclareFontEncoding{OMS}{}{}
\DeclareFontEncoding{OMX}{}{}
%    \end{macrocode}
%    Finally a declaration for |U| encoding which serves for all fonts
%    that do not fit standard encodings. For math this sets up
%    |\noaccents@| providing for AMS-\LaTeX{}. This macro is used
%    therein to handle accented characters if they are not supported
%    by the font. In other words, if fonts with |U| encoding are used
%    in math, all accents (like from |\breve|) are obtained from some
%    other font that has them.
%    \begin{macrocode}
\DeclareFontEncoding{U}{}{\noaccents@}
%    \end{macrocode}
%    The encodings for math are next:
%    \begin{macrocode}
\DeclareFontSubstitution{OML}{cmm}{m}{it}
\DeclareFontSubstitution{OMS}{cmsy}{m}{n}
\DeclareFontSubstitution{OMX}{cmex}{m}{n}
\DeclareFontSubstitution{U}{cmr}{m}{n}
%    \end{macrocode}
%
%    \begin{macrocode}
\begingroup
\nfss@catcodes
\input  {omlcmm.fd}
\input  {omscmsy.fd}
\input  {omxcmex.fd}
\input  {ucmr.fd}
\endgroup
%    \end{macrocode}
%
%  \subsubsection{Symbolfont and Alphabet declarations}
%
%    We now define the basic symbol fonts used by \LaTeX{}.
%    These four symbol fonts must be defined by this file.
%
%    It is possible to make the symbol fonts point to other external
%    fonts without losing the ability to process  documents written
%    at other sites, as long as one defines the same symbol font names
%    with the same encodings, e.g.~|operators| with |OT1| etc.
%    If other encodings are used documents become non-portable.
%    Such a change should therefore be done in a package file.
%
% \changes{v2.1e}{1994/01/19}{Added missing setting for symbols in
%                             bold version.}
%    \begin{macrocode}
\DeclareSymbolFont{operators}   {OT1}{cmr} {m}{n}
\DeclareSymbolFont{letters}     {OML}{cmm} {m}{it}
\DeclareSymbolFont{symbols}     {OMS}{cmsy}{m}{n}
\DeclareSymbolFont{largesymbols}{OMX}{cmex}{m}{n}
%    \end{macrocode}
%
%    \begin{macrocode}
\SetSymbolFont{operators}{bold}{OT1}{cmr} {bx}{n}
\SetSymbolFont{letters}  {bold}{OML}{cmm} {b}{it}
\SetSymbolFont{symbols}  {bold}{OMS}{cmsy}{b}{n}
%    \end{macrocode}
%
%    Below are the seven math alphabets which are defined by NFSS.
%    Again they must be defined by this file.
%    However, as before you can change the fonts used without losing
%    portability, but you should be careful when changing the encoding
%    since that may make documents come out wrong.
%    \begin{macrocode}
\DeclareSymbolFontAlphabet{\mathrm}    {operators}
\DeclareSymbolFontAlphabet{\mathnormal}{letters}
\DeclareSymbolFontAlphabet{\mathcal}   {symbols}
\DeclareMathAlphabet      {\mathbf}{OT1}{cmr}{bx}{n}
\DeclareMathAlphabet      {\mathsf}{OT1}{cmss}{m}{n}
\DeclareMathAlphabet      {\mathit}{OT1}{cmr}{m}{it}
\DeclareMathAlphabet      {\mathtt}{OT1}{cmtt}{m}{n}
%    \end{macrocode}
%    Given the currently available fonts we cannot bold-en |\mathbf|
%    and |\mathtt| but in principle one could use `ultra bold' or
%    something. The alphabets defined via |\DeclareSymbolFontAlphabet|
%    will change automatically in a new math version if the
%    corresponding symbol font changes.
%    \begin{macrocode}
\SetMathAlphabet\mathsf{bold}{OT1}{cmss}{bx}{n}
\SetMathAlphabet\mathit{bold}{OT1}{cmr}{bx}{it}
%    \end{macrocode}
%
%
% \subsection{Math font sizes}
% \changes{v2.2f}{1994/11/07}
%     {(DPC) Add \cs{DeclareMathSizes} declarations}
%
%    The declarations below declare the text, script and scriptscript
%    size to be used for each text font size.
%
%    All occurrences of sizes longer than a single character are replaced
%    with the macro name that holds them, saving a number of
%    tokens (but losing a bit of speed, so this may not stay this way).
%    \begin{macrocode}
 \DeclareMathSizes{5}{5}{5}{5}
 \DeclareMathSizes{6}{6}{5}{5}
 \DeclareMathSizes{7}{7}{5}{5}
 \DeclareMathSizes{8}{8}{6}{5}
 \DeclareMathSizes{9}{9}{6}{5}
 \DeclareMathSizes{\@xpt}{\@xpt}{7}{5}
 \DeclareMathSizes{\@xipt}{\@xipt}{8}{6}
 \DeclareMathSizes{\@xiipt}{\@xiipt}{8}{6}
 \DeclareMathSizes{\@xivpt}{\@xivpt}{\@xpt}{7}
 \DeclareMathSizes{\@xviipt}{\@xviipt}{\@xiipt}{\@xpt}
 \DeclareMathSizes{\@xxpt}{\@xxpt}{\@xivpt}{\@xiipt}
 \DeclareMathSizes{\@xxvpt}{\@xxvpt}{\@xxpt}{\@xviipt}
%    \end{macrocode}
%
% \subsection{The math symbol assignments}
%
%    We start by setting up math codes for most of the characters
%    typed in directly from the keyboard. Most of them are normally
%    already setup up in the same way by Ini\TeX{}. However, we repeat
%    them here to have a complete setup which can be exchanged with
%    another if desired.
%
% \subsubsection{The letters}
%    \begin{macrocode}
\DeclareMathSymbol{a}{\mathalpha}{letters}{`a}
\DeclareMathSymbol{b}{\mathalpha}{letters}{`b}
\DeclareMathSymbol{c}{\mathalpha}{letters}{`c}
\DeclareMathSymbol{d}{\mathalpha}{letters}{`d}
\DeclareMathSymbol{e}{\mathalpha}{letters}{`e}
\DeclareMathSymbol{f}{\mathalpha}{letters}{`f}
\DeclareMathSymbol{g}{\mathalpha}{letters}{`g}
\DeclareMathSymbol{h}{\mathalpha}{letters}{`h}
\DeclareMathSymbol{i}{\mathalpha}{letters}{`i}
\DeclareMathSymbol{j}{\mathalpha}{letters}{`j}
\DeclareMathSymbol{k}{\mathalpha}{letters}{`k}
\DeclareMathSymbol{l}{\mathalpha}{letters}{`l}
\DeclareMathSymbol{m}{\mathalpha}{letters}{`m}
\DeclareMathSymbol{n}{\mathalpha}{letters}{`n}
\DeclareMathSymbol{o}{\mathalpha}{letters}{`o}
\DeclareMathSymbol{p}{\mathalpha}{letters}{`p}
\DeclareMathSymbol{q}{\mathalpha}{letters}{`q}
\DeclareMathSymbol{r}{\mathalpha}{letters}{`r}
\DeclareMathSymbol{s}{\mathalpha}{letters}{`s}
\DeclareMathSymbol{t}{\mathalpha}{letters}{`t}
\DeclareMathSymbol{u}{\mathalpha}{letters}{`u}
\DeclareMathSymbol{v}{\mathalpha}{letters}{`v}
\DeclareMathSymbol{w}{\mathalpha}{letters}{`w}
\DeclareMathSymbol{x}{\mathalpha}{letters}{`x}
\DeclareMathSymbol{y}{\mathalpha}{letters}{`y}
\DeclareMathSymbol{z}{\mathalpha}{letters}{`z}
%    \end{macrocode}
%
%    \begin{macrocode}
\DeclareMathSymbol{A}{\mathalpha}{letters}{`A}
\DeclareMathSymbol{B}{\mathalpha}{letters}{`B}
\DeclareMathSymbol{C}{\mathalpha}{letters}{`C}
\DeclareMathSymbol{D}{\mathalpha}{letters}{`D}
\DeclareMathSymbol{E}{\mathalpha}{letters}{`E}
\DeclareMathSymbol{F}{\mathalpha}{letters}{`F}
\DeclareMathSymbol{G}{\mathalpha}{letters}{`G}
\DeclareMathSymbol{H}{\mathalpha}{letters}{`H}
\DeclareMathSymbol{I}{\mathalpha}{letters}{`I}
\DeclareMathSymbol{J}{\mathalpha}{letters}{`J}
\DeclareMathSymbol{K}{\mathalpha}{letters}{`K}
\DeclareMathSymbol{L}{\mathalpha}{letters}{`L}
\DeclareMathSymbol{M}{\mathalpha}{letters}{`M}
\DeclareMathSymbol{N}{\mathalpha}{letters}{`N}
\DeclareMathSymbol{O}{\mathalpha}{letters}{`O}
\DeclareMathSymbol{P}{\mathalpha}{letters}{`P}
\DeclareMathSymbol{Q}{\mathalpha}{letters}{`Q}
\DeclareMathSymbol{R}{\mathalpha}{letters}{`R}
\DeclareMathSymbol{S}{\mathalpha}{letters}{`S}
\DeclareMathSymbol{T}{\mathalpha}{letters}{`T}
\DeclareMathSymbol{U}{\mathalpha}{letters}{`U}
\DeclareMathSymbol{V}{\mathalpha}{letters}{`V}
\DeclareMathSymbol{W}{\mathalpha}{letters}{`W}
\DeclareMathSymbol{X}{\mathalpha}{letters}{`X}
\DeclareMathSymbol{Y}{\mathalpha}{letters}{`Y}
\DeclareMathSymbol{Z}{\mathalpha}{letters}{`Z}
%    \end{macrocode}
%
% \subsubsection{The digits}
%
%    \begin{macrocode}
\DeclareMathSymbol{0}{\mathalpha}{operators}{`0}
\DeclareMathSymbol{1}{\mathalpha}{operators}{`1}
\DeclareMathSymbol{2}{\mathalpha}{operators}{`2}
\DeclareMathSymbol{3}{\mathalpha}{operators}{`3}
\DeclareMathSymbol{4}{\mathalpha}{operators}{`4}
\DeclareMathSymbol{5}{\mathalpha}{operators}{`5}
\DeclareMathSymbol{6}{\mathalpha}{operators}{`6}
\DeclareMathSymbol{7}{\mathalpha}{operators}{`7}
\DeclareMathSymbol{8}{\mathalpha}{operators}{`8}
\DeclareMathSymbol{9}{\mathalpha}{operators}{`9}
%    \end{macrocode}
%
%
% \subsubsection{Punctuation, brace, etc. keys}
%
%    \begin{macrocode}
\DeclareMathSymbol{!}{\mathclose}{operators}{"21}
\DeclareMathSymbol{*}{\mathbin}{symbols}{"03} % \ast
\DeclareMathSymbol{+}{\mathbin}{operators}{"2B}
\DeclareMathSymbol{,}{\mathpunct}{letters}{"3B}
\DeclareMathSymbol{-}{\mathbin}{symbols}{"00}
\DeclareMathSymbol{.}{\mathord}{letters}{"3A}
\DeclareMathSymbol{:}{\mathrel}{operators}{"3A}
\DeclareMathSymbol{;}{\mathpunct}{operators}{"3B}
\DeclareMathSymbol{=}{\mathrel}{operators}{"3D}
\DeclareMathSymbol{?}{\mathclose}{operators}{"3F}
%    \end{macrocode}
% The following symbols are defined as delimiters below
% which automatically defines them as math symbols.
%    \begin{macrocode}
%\DeclareMathSymbol{(}{\mathopen}{operators}{"28}
%\DeclareMathSymbol{)}{\mathclose}{operators}{"29}
%\DeclareMathSymbol{/}{\mathord}{letters}{"3D}
%\DeclareMathSymbol{[}{\mathopen}{operators}{"5B}
%\DeclareMathSymbol{]}{\mathclose}{operators}{"5D}
%\DeclareMathSymbol{|}{\mathord}{symbols}{"6A}
%\DeclareMathSymbol{<}{\mathrel}{letters}{"3C}
%\DeclareMathSymbol{>}{\mathrel}{letters}{"3E}
%    \end{macrocode}
%
%    Should all of the following being activated by default? Probably
%    not.
%    \begin{macrocode}
%\DeclareMathSymbol{`\{}{\mathopen}{symbols}{"66}
%\DeclareMathSymbol{`\}}{\mathclose}{symbols}{"67}
%\DeclareMathSymbol{`\\}{\mathord}{symbols}{"6E} % \backslash
\mathcode`\ ="8000 % \space
\mathcode`\'="8000 % ^\prime
\mathcode`\_="8000 % \_
%    \end{macrocode}
%
%
% \subsubsection{Delimitercodes for characters}
% \changes{v2.2q}{1997/01/08}
%     {Use \cs{DeclareMathDelimiter} to set delimiter codes}
% \changes{v2.2u}{1998/04/15}
%     {Use new syntax for \cs{DeclareMathDelimiter}}
%    [to be completed]
%
%    Finally, Ini\TeX{} sets all |\delcode| values to -1, except
%    |\delcode`.=0|
%    \begin{macrocode}
\DeclareMathDelimiter{(}{\mathopen} {operators}{"28}{largesymbols}{"00}
\DeclareMathDelimiter{)}{\mathclose}{operators}{"29}{largesymbols}{"01}
\DeclareMathDelimiter{[}{\mathopen} {operators}{"5B}{largesymbols}{"02}
\DeclareMathDelimiter{]}{\mathclose}{operators}{"5D}{largesymbols}{"03}
%    \end{macrocode}
%
% The next two are considered to be relations when not used in the context
% of a delimiter! And worse, they do even represent different glyphs when
% being used as delimiter and not as delimiter. This is a user level syntax
% inherited from plain \TeX{}. Therefore we explicitly redefine the math
% symbol definitions for these symbols afterwards.
% \changes{v2.2v}{1998/04/17}
%     {Reinsert symbol defs for \texttt{<} and \texttt{\char62} chars.}
%    \begin{macrocode}
\DeclareMathDelimiter{<}{\mathopen}{symbols}{"68}{largesymbols}{"0A}
\DeclareMathDelimiter{>}{\mathclose}{symbols}{"69}{largesymbols}{"0B}
\DeclareMathSymbol{<}{\mathrel}{letters}{"3C}
\DeclareMathSymbol{>}{\mathrel}{letters}{"3E}
%    \end{macrocode}
% And here is another case where the non-delimiter version produces a
% glyph different from the delimiter version.
% \changes{v2.2w}{1998/04/18}
%     {Reinsert symbol def for \texttt{/} char.}
%    \begin{macrocode}
\DeclareMathDelimiter{/}{\mathord}{operators}{"2F}{largesymbols}{"0E}
\DeclareMathSymbol{/}{\mathord}{letters}{"3D}
%    \end{macrocode}
%
%    \begin{macrocode}
\DeclareMathDelimiter{|}{\mathord}{symbols}{"6A}{largesymbols}{"0C}
%    \end{macrocode}
%
%    \begin{macrocode}
\expandafter\DeclareMathDelimiter\@backslashchar
                        {\mathord}{symbols}{"6E}{largesymbols}{"0F}
%    \end{macrocode}
% N.B. |{| and |}| should NOT get delcodes;
% otherwise parameter grouping fails!
%
%
% \subsection{Symbols accessed via control sequences}
%
% \subsubsection{Greek letters}
%
%    \begin{macrocode}
\DeclareMathSymbol{\alpha}{\mathord}{letters}{"0B}
\DeclareMathSymbol{\beta}{\mathord}{letters}{"0C}
\DeclareMathSymbol{\gamma}{\mathord}{letters}{"0D}
\DeclareMathSymbol{\delta}{\mathord}{letters}{"0E}
\DeclareMathSymbol{\epsilon}{\mathord}{letters}{"0F}
\DeclareMathSymbol{\zeta}{\mathord}{letters}{"10}
\DeclareMathSymbol{\eta}{\mathord}{letters}{"11}
\DeclareMathSymbol{\theta}{\mathord}{letters}{"12}
\DeclareMathSymbol{\iota}{\mathord}{letters}{"13}
\DeclareMathSymbol{\kappa}{\mathord}{letters}{"14}
\DeclareMathSymbol{\lambda}{\mathord}{letters}{"15}
\DeclareMathSymbol{\mu}{\mathord}{letters}{"16}
\DeclareMathSymbol{\nu}{\mathord}{letters}{"17}
\DeclareMathSymbol{\xi}{\mathord}{letters}{"18}
\DeclareMathSymbol{\pi}{\mathord}{letters}{"19}
\DeclareMathSymbol{\rho}{\mathord}{letters}{"1A}
\DeclareMathSymbol{\sigma}{\mathord}{letters}{"1B}
\DeclareMathSymbol{\tau}{\mathord}{letters}{"1C}
\DeclareMathSymbol{\upsilon}{\mathord}{letters}{"1D}
\DeclareMathSymbol{\phi}{\mathord}{letters}{"1E}
\DeclareMathSymbol{\chi}{\mathord}{letters}{"1F}
\DeclareMathSymbol{\psi}{\mathord}{letters}{"20}
\DeclareMathSymbol{\omega}{\mathord}{letters}{"21}
\DeclareMathSymbol{\varepsilon}{\mathord}{letters}{"22}
\DeclareMathSymbol{\vartheta}{\mathord}{letters}{"23}
\DeclareMathSymbol{\varpi}{\mathord}{letters}{"24}
\DeclareMathSymbol{\varrho}{\mathord}{letters}{"25}
\DeclareMathSymbol{\varsigma}{\mathord}{letters}{"26}
\DeclareMathSymbol{\varphi}{\mathord}{letters}{"27}
\DeclareMathSymbol{\Gamma}{\mathalpha}{operators}{"00}
\DeclareMathSymbol{\Delta}{\mathalpha}{operators}{"01}
\DeclareMathSymbol{\Theta}{\mathalpha}{operators}{"02}
\DeclareMathSymbol{\Lambda}{\mathalpha}{operators}{"03}
\DeclareMathSymbol{\Xi}{\mathalpha}{operators}{"04}
\DeclareMathSymbol{\Pi}{\mathalpha}{operators}{"05}
\DeclareMathSymbol{\Sigma}{\mathalpha}{operators}{"06}
\DeclareMathSymbol{\Upsilon}{\mathalpha}{operators}{"07}
\DeclareMathSymbol{\Phi}{\mathalpha}{operators}{"08}
\DeclareMathSymbol{\Psi}{\mathalpha}{operators}{"09}
\DeclareMathSymbol{\Omega}{\mathalpha}{operators}{"0A}
%    \end{macrocode}
%
%
% \subsubsection{Ordinary symbols}
%
%    \begin{macrocode}
\DeclareMathSymbol{\aleph}{\mathord}{symbols}{"40}
\DeclareMathSymbol{\imath}{\mathord}{letters}{"7B}
\DeclareMathSymbol{\jmath}{\mathord}{letters}{"7C}
\DeclareMathSymbol{\ell}{\mathord}{letters}{"60}
\DeclareMathSymbol{\wp}{\mathord}{letters}{"7D}
\DeclareMathSymbol{\Re}{\mathord}{symbols}{"3C}
\DeclareMathSymbol{\Im}{\mathord}{symbols}{"3D}
\DeclareMathSymbol{\partial}{\mathord}{letters}{"40}
\DeclareMathSymbol{\infty}{\mathord}{symbols}{"31}
\DeclareMathSymbol{\prime}{\mathord}{symbols}{"30}
\DeclareMathSymbol{\emptyset}{\mathord}{symbols}{"3B}
\DeclareMathSymbol{\nabla}{\mathord}{symbols}{"72}
\DeclareMathSymbol{\top}{\mathord}{symbols}{"3E}
\DeclareMathSymbol{\bot}{\mathord}{symbols}{"3F}
\DeclareMathSymbol{\triangle}{\mathord}{symbols}{"34}
\DeclareMathSymbol{\forall}{\mathord}{symbols}{"38}
\DeclareMathSymbol{\exists}{\mathord}{symbols}{"39}
\DeclareMathSymbol{\neg}{\mathord}{symbols}{"3A}
%    \end{macrocode}
%    Alias:
% \changes{v3.0e}{2019/12/21}{Distangle alias (gh/184)}
%    \begin{macrocode}
%    \let\lnot=\neg
\DeclareMathSymbol{\lnot}{\mathord}{symbols}{"3A}
%    \end{macrocode}
%
%    \begin{macrocode}
\DeclareMathSymbol{\flat}{\mathord}{letters}{"5B}
\DeclareMathSymbol{\natural}{\mathord}{letters}{"5C}
\DeclareMathSymbol{\sharp}{\mathord}{letters}{"5D}
\DeclareMathSymbol{\clubsuit}{\mathord}{symbols}{"7C}
\DeclareMathSymbol{\diamondsuit}{\mathord}{symbols}{"7D}
\DeclareMathSymbol{\heartsuit}{\mathord}{symbols}{"7E}
\DeclareMathSymbol{\spadesuit}{\mathord}{symbols}{"7F}
%    \end{macrocode}
%
% \changes{v3.0c}{2019/08/27}{Various commands made robust throughout
%   the file}
%    \begin{macrocode}
\DeclareRobustCommand\hbar{{\mathchar'26\mkern-9muh}}
\DeclareRobustCommand\surd{{\mathchar"1270}}
\DeclareRobustCommand\angle{{\vbox{\ialign{$\m@th\scriptstyle##$\crcr
      \not\mathrel{\mkern14mu}\crcr
      \noalign{\nointerlineskip}
      \mkern2.5mu\leaders\hrule \@height.34pt\hfill\mkern2.5mu\crcr}}}}
%    \end{macrocode}
%
%
% \subsubsection{Large Operators}
%
%    \begin{macrocode}
\DeclareMathSymbol{\coprod}{\mathop}{largesymbols}{"60}
\DeclareMathSymbol{\bigvee}{\mathop}{largesymbols}{"57}
\DeclareMathSymbol{\bigwedge}{\mathop}{largesymbols}{"56}
\DeclareMathSymbol{\biguplus}{\mathop}{largesymbols}{"55}
\DeclareMathSymbol{\bigcap}{\mathop}{largesymbols}{"54}
\DeclareMathSymbol{\bigcup}{\mathop}{largesymbols}{"53}
\DeclareMathSymbol{\intop}{\mathop}{largesymbols}{"52}
    \DeclareRobustCommand\int{\intop\nolimits}
\DeclareMathSymbol{\prod}{\mathop}{largesymbols}{"51}
\DeclareMathSymbol{\sum}{\mathop}{largesymbols}{"50}
\DeclareMathSymbol{\bigotimes}{\mathop}{largesymbols}{"4E}
\DeclareMathSymbol{\bigoplus}{\mathop}{largesymbols}{"4C}
\DeclareMathSymbol{\bigodot}{\mathop}{largesymbols}{"4A}
\DeclareMathSymbol{\ointop}{\mathop}{largesymbols}{"48}
    \DeclareRobustCommand\oint{\ointop\nolimits}
\DeclareMathSymbol{\bigsqcup}{\mathop}{largesymbols}{"46}
\DeclareMathSymbol{\smallint}{\mathop}{symbols}{"73}
%    \end{macrocode}
%
%
% \subsubsection{Binary symbols}
%
% \changes{v2.3a}{2004/02/04}
%     {Added bigtriangle synonyms for stmaryrd}
%    \begin{macrocode}
\DeclareMathSymbol{\triangleleft}{\mathbin}{letters}{"2F}
\DeclareMathSymbol{\triangleright}{\mathbin}{letters}{"2E}
\DeclareMathSymbol{\bigtriangleup}{\mathbin}{symbols}{"34}
\DeclareMathSymbol{\bigtriangledown}{\mathbin}{symbols}{"35}
%    \end{macrocode}
%    Alias:
% \changes{v3.0e}{2019/12/21}{Distangle alias (gh/184)}
%    \begin{macrocode}
%   \let \varbigtriangledown \bigtriangledown
%   \let \varbigtriangleup \bigtriangleup
\DeclareMathSymbol{\varbigtriangleup}{\mathbin}{symbols}{"34}
\DeclareMathSymbol{\varbigtriangledown}{\mathbin}{symbols}{"35}
%    \end{macrocode}
%
% These last two synonyms are needed because the \textsf{stmaryrd}
% package redefines them as Operators.
%
%    \begin{macrocode}
\DeclareMathSymbol{\wedge}{\mathbin}{symbols}{"5E}
\DeclareMathSymbol{\vee}{\mathbin}{symbols}{"5F}
%    \end{macrocode}
%    Alias:
% \changes{v3.0e}{2019/12/21}{Distangle alias (gh/184)}
%    \begin{macrocode}
%   \let\land=\wedge
%   \let\lor=\vee
\DeclareMathSymbol{\land}{\mathbin}{symbols}{"5E}
\DeclareMathSymbol{\lor}{\mathbin}{symbols}{"5F}
%    \end{macrocode}
%
%    \begin{macrocode}
\DeclareMathSymbol{\cap}{\mathbin}{symbols}{"5C}
\DeclareMathSymbol{\cup}{\mathbin}{symbols}{"5B}
\DeclareMathSymbol{\ddagger}{\mathbin}{symbols}{"7A}
\DeclareMathSymbol{\dagger}{\mathbin}{symbols}{"79}
\DeclareMathSymbol{\sqcap}{\mathbin}{symbols}{"75}
\DeclareMathSymbol{\sqcup}{\mathbin}{symbols}{"74}
\DeclareMathSymbol{\uplus}{\mathbin}{symbols}{"5D}
\DeclareMathSymbol{\amalg}{\mathbin}{symbols}{"71}
\DeclareMathSymbol{\diamond}{\mathbin}{symbols}{"05}
\DeclareMathSymbol{\bullet}{\mathbin}{symbols}{"0F}
\DeclareMathSymbol{\wr}{\mathbin}{symbols}{"6F}
\DeclareMathSymbol{\div}{\mathbin}{symbols}{"04}
\DeclareMathSymbol{\odot}{\mathbin}{symbols}{"0C}
\DeclareMathSymbol{\oslash}{\mathbin}{symbols}{"0B}
\DeclareMathSymbol{\otimes}{\mathbin}{symbols}{"0A}
\DeclareMathSymbol{\ominus}{\mathbin}{symbols}{"09}
\DeclareMathSymbol{\oplus}{\mathbin}{symbols}{"08}
\DeclareMathSymbol{\mp}{\mathbin}{symbols}{"07}
\DeclareMathSymbol{\pm}{\mathbin}{symbols}{"06}
\DeclareMathSymbol{\circ}{\mathbin}{symbols}{"0E}
\DeclareMathSymbol{\bigcirc}{\mathbin}{symbols}{"0D}
\DeclareMathSymbol{\setminus}{\mathbin}{symbols}{"6E}
\DeclareMathSymbol{\cdot}{\mathbin}{symbols}{"01}
\DeclareMathSymbol{\ast}{\mathbin}{symbols}{"03}
\DeclareMathSymbol{\times}{\mathbin}{symbols}{"02}
\DeclareMathSymbol{\star}{\mathbin}{letters}{"3F}
%    \end{macrocode}
%
%
% \subsubsection{Relations}
%
%    \begin{macrocode}
\DeclareMathSymbol{\propto}{\mathrel}{symbols}{"2F}
\DeclareMathSymbol{\sqsubseteq}{\mathrel}{symbols}{"76}
\DeclareMathSymbol{\sqsupseteq}{\mathrel}{symbols}{"77}
\DeclareMathSymbol{\parallel}{\mathrel}{symbols}{"6B}
\DeclareMathSymbol{\mid}{\mathrel}{symbols}{"6A}
\DeclareMathSymbol{\dashv}{\mathrel}{symbols}{"61}
\DeclareMathSymbol{\vdash}{\mathrel}{symbols}{"60}
\DeclareMathSymbol{\nearrow}{\mathrel}{symbols}{"25}
\DeclareMathSymbol{\searrow}{\mathrel}{symbols}{"26}
\DeclareMathSymbol{\nwarrow}{\mathrel}{symbols}{"2D}
\DeclareMathSymbol{\swarrow}{\mathrel}{symbols}{"2E}
\DeclareMathSymbol{\Leftrightarrow}{\mathrel}{symbols}{"2C}
\DeclareMathSymbol{\Leftarrow}{\mathrel}{symbols}{"28}
\DeclareMathSymbol{\Rightarrow}{\mathrel}{symbols}{"29}
   \DeclareRobustCommand\neq{\not=}
%    \end{macrocode}
%    As \cs{neq} is robust we should not use \cs{let} to define
%    \cs{ne} as then it would change if \cs{neq} changes.
% \changes{v3.0d}{2019/09/21}{Distangle alias (gh/184)}
%    \begin{macrocode}
   \DeclareRobustCommand\ne{\not=}
%    \end{macrocode}
%    It would ok to use \cs{let} for those declared by
%    \cs{DeclareMathSymbol} but for a cleaner interface we avoid it
%    always (just in case the internals change).
%    \begin{macrocode}
\DeclareMathSymbol{\leq}{\mathrel}{symbols}{"14}
\DeclareMathSymbol{\geq}{\mathrel}{symbols}{"15}
%    \end{macrocode}
%    Alias:
% \changes{v3.0e}{2019/12/21}{Distangle alias (gh/184)}
%    \begin{macrocode}
%   \let\le=\leq
%   \let\ge=\geq
\DeclareMathSymbol{\le}{\mathrel}{symbols}{"14}
\DeclareMathSymbol{\ge}{\mathrel}{symbols}{"15}
%    \end{macrocode}
%
%    \begin{macrocode}
\DeclareMathSymbol{\succ}{\mathrel}{symbols}{"1F}
\DeclareMathSymbol{\prec}{\mathrel}{symbols}{"1E}
\DeclareMathSymbol{\approx}{\mathrel}{symbols}{"19}
\DeclareMathSymbol{\succeq}{\mathrel}{symbols}{"17}
\DeclareMathSymbol{\preceq}{\mathrel}{symbols}{"16}
\DeclareMathSymbol{\supset}{\mathrel}{symbols}{"1B}
\DeclareMathSymbol{\subset}{\mathrel}{symbols}{"1A}
\DeclareMathSymbol{\supseteq}{\mathrel}{symbols}{"13}
\DeclareMathSymbol{\subseteq}{\mathrel}{symbols}{"12}
\DeclareMathSymbol{\in}{\mathrel}{symbols}{"32}
\DeclareMathSymbol{\ni}{\mathrel}{symbols}{"33}
%    \end{macrocode}
%    Alias:
% \changes{v3.0e}{2019/12/21}{Distangle alias (gh/184)}
%    \begin{macrocode}
%    \let\owns=\ni
\DeclareMathSymbol{\owns}{\mathrel}{symbols}{"33}
%    \end{macrocode}
%
%    \begin{macrocode}
\DeclareMathSymbol{\gg}{\mathrel}{symbols}{"1D}
\DeclareMathSymbol{\ll}{\mathrel}{symbols}{"1C}
\DeclareMathSymbol{\not}{\mathrel}{symbols}{"36}
\DeclareMathSymbol{\leftrightarrow}{\mathrel}{symbols}{"24}
\DeclareMathSymbol{\leftarrow}{\mathrel}{symbols}{"20}
\DeclareMathSymbol{\rightarrow}{\mathrel}{symbols}{"21}
%    \end{macrocode}
%    Alias:
% \changes{v3.0e}{2019/12/21}{Distangle alias (gh/184)}
%    \begin{macrocode}
%   \let\gets=\leftarrow
%   \let\to=\rightarrow
\DeclareMathSymbol{\gets}{\mathrel}{symbols}{"20}
\DeclareMathSymbol{\to}{\mathrel}{symbols}{"21}
%    \end{macrocode}
%
%    \begin{macrocode}
\DeclareMathSymbol{\mapstochar}{\mathrel}{symbols}{"37}
   \DeclareRobustCommand\mapsto{\mapstochar\rightarrow}
\DeclareMathSymbol{\sim}{\mathrel}{symbols}{"18}
\DeclareMathSymbol{\simeq}{\mathrel}{symbols}{"27}
\DeclareMathSymbol{\perp}{\mathrel}{symbols}{"3F}
\DeclareMathSymbol{\equiv}{\mathrel}{symbols}{"11}
\DeclareMathSymbol{\asymp}{\mathrel}{symbols}{"10}
\DeclareMathSymbol{\smile}{\mathrel}{letters}{"5E}
\DeclareMathSymbol{\frown}{\mathrel}{letters}{"5F}
\DeclareMathSymbol{\leftharpoonup}{\mathrel}{letters}{"28}
\DeclareMathSymbol{\leftharpoondown}{\mathrel}{letters}{"29}
\DeclareMathSymbol{\rightharpoonup}{\mathrel}{letters}{"2A}
\DeclareMathSymbol{\rightharpoondown}{\mathrel}{letters}{"2B}
%    \end{macrocode}
%
%    Here cometh much profligate robustification of math constructs.
%    Warning: some of these commands may become non-robust if an
%    AMS package is loaded.
%
%    Further potential problems: some math font packages may make
%    unfortunate assumptions about some of these definitions that are
%    not true of the robust versions we need.
% \changes{v2.3}{2004/02/02}
%     {Many things from here on made robust}
%    \begin{macrocode}
\DeclareRobustCommand
  \cong{\mathrel{\mathpalette\@vereq\sim}} % congruence sign
\def\@vereq#1#2{\lower.5\p@\vbox{\lineskiplimit\maxdimen\lineskip-.5\p@
    \ialign{$\m@th#1\hfil##\hfil$\crcr#2\crcr=\crcr}}}
\DeclareRobustCommand
  \notin{\mathrel{\m@th\mathpalette\c@ncel\in}}
\def\c@ncel#1#2{\m@th\ooalign{$\hfil#1\mkern1mu/\hfil$\crcr$#1#2$}}
\DeclareRobustCommand
  \rightleftharpoons{\mathrel{\mathpalette\rlh@{}}}
\def\rlh@#1{\vcenter{\m@th\hbox{\ooalign{\raise2pt
          \hbox{$#1\rightharpoonup$}\crcr
        $#1\leftharpoondown$}}}}
\DeclareRobustCommand
  \doteq{\buildrel\textstyle.\over=}
%    \end{macrocode}
%
% \subsubsection{Arrows}
%
%    \begin{macrocode}
\DeclareRobustCommand
  \joinrel{\mathrel{\mkern-3mu}}
\DeclareRobustCommand
  \relbar{\mathrel{\smash-}} % \smash, because -
                               % has the same height as +
%    \end{macrocode}
%    In contrast to \texttt{plain.tex} |\Relbar| got braces around the
%    equal sign to guard against it being ``math active'' expanding to
%    |\futurelet...|. This might be the case when packages are
%    implementing shorthands for math, e.g. |=>| meaning |\Rightarrow|
%    etc. It would actually be better not to use |=| in such
%    definitions but instead define something like |\mathequalsign|
%    and use this. However we can't do this now as it would break
%    other math layouts where characters are in different places
%    (since those wouldn't know about the need for a new command name).
% \changes{v2.2z}{2001/06/04}{Guard against math active equal sign in
%    \cs{Relbar} (pr/3333)}
%    \begin{macrocode}
\DeclareRobustCommand
  \Relbar{\mathrel{=}}
\DeclareMathSymbol{\lhook}{\mathrel}{letters}{"2C}
   \DeclareRobustCommand\hookrightarrow{\lhook\joinrel\rightarrow}
\DeclareMathSymbol{\rhook}{\mathrel}{letters}{"2D}
   \DeclareRobustCommand\hookleftarrow{\leftarrow\joinrel\rhook}
\DeclareRobustCommand
  \bowtie{\mathrel\triangleright\joinrel\mathrel\triangleleft}
%    \end{macrocode}
%
% \changes{v2.2z}{2001/06/04}{Guard against math active equal and pipe
%    sign in \cs{models} (pr/3333)}
%    \begin{macrocode}
\DeclareRobustCommand
  \models{\mathrel{|}\joinrel\Relbar}
\DeclareRobustCommand
  \Longrightarrow{\Relbar\joinrel\Rightarrow}
%    \end{macrocode}
%
% LaTeX Change: |\longrightarrow| and |\longleftarrow| redefined to make
%   then robust.
%    \begin{macrocode}
\DeclareRobustCommand\longrightarrow
     {\relbar\joinrel\rightarrow}
\DeclareRobustCommand\longleftarrow
     {\leftarrow\joinrel\relbar}
%    \end{macrocode}
%
%    \begin{macrocode}
\DeclareRobustCommand
  \Longleftarrow{\Leftarrow\joinrel\Relbar}
\DeclareRobustCommand
  \longmapsto{\mapstochar\longrightarrow}
\DeclareRobustCommand
  \longleftrightarrow{\leftarrow\joinrel\rightarrow}
\DeclareRobustCommand
  \Longleftrightarrow{\Leftarrow\joinrel\Rightarrow}
\DeclareRobustCommand
  \iff{\;\Longleftrightarrow\;}
%    \end{macrocode}
%
%
% \subsubsection{Punctuation symbols}
%
%    \begin{macrocode}
\DeclareMathSymbol{\ldotp}{\mathpunct}{letters}{"3A}
\DeclareMathSymbol{\cdotp}{\mathpunct}{symbols}{"01}
\DeclareMathSymbol{\colon}{\mathpunct}{operators}{"3A}
%    \end{macrocode}
%
%
% This is commented out, since |\ldots| is now defined in ltoutenc.dtx.
%    \begin{macrocode}
%\def\@ldots{\mathinner{\ldotp\ldotp\ldotp}}
%\DeclareRobustCommand\ldots
%          {\relax\ifmmode\@ldots\else\mbox{$\m@th\@ldots\,$}\fi}
%    \end{macrocode}
%
%    \begin{macrocode}
\DeclareRobustCommand
  \cdots{\mathinner{\cdotp\cdotp\cdotp}}
\DeclareRobustCommand
  \vdots{\vbox{\baselineskip4\p@ \lineskiplimit\z@
    \kern6\p@\hbox{.}\hbox{.}\hbox{.}}}
\DeclareRobustCommand
  \ddots{\mathinner{\mkern1mu\raise7\p@
    \vbox{\kern7\p@\hbox{.}}\mkern2mu
    \raise4\p@\hbox{.}\mkern2mu\raise\p@\hbox{.}\mkern1mu}}
%    \end{macrocode}
%
%
% \subsubsection{Math accents}
%
%    \begin{macrocode}
\DeclareMathAccent{\acute}{\mathalpha}{operators}{"13}
\DeclareMathAccent{\grave}{\mathalpha}{operators}{"12}
\DeclareMathAccent{\ddot}{\mathalpha}{operators}{"7F}
\DeclareMathAccent{\tilde}{\mathalpha}{operators}{"7E}
\DeclareMathAccent{\bar}{\mathalpha}{operators}{"16}
\DeclareMathAccent{\breve}{\mathalpha}{operators}{"15}
\DeclareMathAccent{\check}{\mathalpha}{operators}{"14}
\DeclareMathAccent{\hat}{\mathalpha}{operators}{"5E}
\DeclareMathAccent{\vec}{\mathord}{letters}{"7E}
\DeclareMathAccent{\dot}{\mathalpha}{operators}{"5F}
\DeclareMathAccent{\widetilde}{\mathord}{largesymbols}{"65}
\DeclareMathAccent{\widehat}{\mathord}{largesymbols}{"62}
%    \end{macrocode}
%    For some reason plain \TeX{} never bothered to provide
%    a ring accent in math (although it is available in the fonts),
%    but since we got a request for it here we go:
% \changes{v2.2t}{1998/04/11}{Added \cs{mathring} accent (pr2785)}
%    \begin{macrocode}
\DeclareMathAccent{\mathring}{\mathalpha}{operators}{"17}
%    \end{macrocode}
%
%
% \subsubsection{Radicals}
%
% \changes{v2.2o}{1996/05/17}{\cs{@@sqrt} removed, at last}
%    \begin{macrocode}
\DeclareMathRadical{\sqrtsign}{symbols}{"70}{largesymbols}{"70}
%    \end{macrocode}
%
%
% \subsubsection{Over and under something, etc}
%
%    \begin{macrocode}
\DeclareRobustCommand\overrightarrow[1]{\vbox{\m@th\ialign{##\crcr
      \rightarrowfill\crcr\noalign{\kern-\p@\nointerlineskip}
      $\hfil\displaystyle{#1}\hfil$\crcr}}}
\DeclareRobustCommand\overleftarrow[1]{\vbox{\m@th\ialign{##\crcr
      \leftarrowfill\crcr\noalign{\kern-\p@\nointerlineskip}%
      $\hfil\displaystyle{#1}\hfil$\crcr}}}
\DeclareRobustCommand\overbrace[1]
     {\mathop{\vbox{\m@th\ialign{##\crcr\noalign{\kern3\p@}%
      \downbracefill\crcr\noalign{\kern3\p@\nointerlineskip}%
      $\hfil\displaystyle{#1}\hfil$\crcr}}}\limits}
\DeclareRobustCommand\underbrace[1]{\mathop{\vtop{\m@th\ialign{##\crcr
   $\hfil\displaystyle{#1}\hfil$\crcr
   \noalign{\kern3\p@\nointerlineskip}%
   \upbracefill\crcr\noalign{\kern3\p@}}}}\limits}
%    \end{macrocode}
%    (quite a waste of tokens, IMHO --- Frank)
%    \begin{macrocode}
\DeclareRobustCommand\skew[3]
  {{\muskip\z@#1mu\divide\muskip\z@\tw@ \mkern\muskip\z@
    #2{\mkern-\muskip\z@{#3}\mkern\muskip\z@}\mkern-\muskip\z@}{}}
%    \end{macrocode}
%
% \changes{v2.2n}{1995/11/21}{Incorporate changed figures,
%                              as in plain.tex}
%    \begin{macrocode}
\DeclareRobustCommand\rightarrowfill{$\m@th\smash-\mkern-7mu%
  \cleaders\hbox{$\mkern-2mu\smash-\mkern-2mu$}\hfill
  \mkern-7mu\mathord\rightarrow$}
\DeclareRobustCommand\leftarrowfill{$\m@th\mathord\leftarrow\mkern-7mu%
  \cleaders\hbox{$\mkern-2mu\smash-\mkern-2mu$}\hfill
  \mkern-7mu\smash-$}
\DeclareMathSymbol{\braceld}{\mathord}{largesymbols}{"7A}
\DeclareMathSymbol{\bracerd}{\mathord}{largesymbols}{"7B}
\DeclareMathSymbol{\bracelu}{\mathord}{largesymbols}{"7C}
\DeclareMathSymbol{\braceru}{\mathord}{largesymbols}{"7D}
\DeclareRobustCommand\downbracefill{$\m@th \setbox\z@\hbox{$\braceld$}%
  \braceld\leaders\vrule \@height\ht\z@ \@depth\z@\hfill\braceru
  \bracelu\leaders\vrule \@height\ht\z@ \@depth\z@\hfill\bracerd$}
\DeclareRobustCommand\upbracefill{$\m@th \setbox\z@\hbox{$\braceld$}%
  \bracelu\leaders\vrule \@height\ht\z@ \@depth\z@\hfill\bracerd
  \braceld\leaders\vrule \@height\ht\z@ \@depth\z@\hfill\braceru$}
%    \end{macrocode}
%
% \subsubsection{Delimiters}
%
%    \begin{macrocode}
\DeclareMathDelimiter{\lmoustache}   % top from (, bottom from )
   {\mathopen}{largesymbols}{"7A}{largesymbols}{"40}
\DeclareMathDelimiter{\rmoustache}   % top from ), bottom from (
   {\mathclose}{largesymbols}{"7B}{largesymbols}{"41}
\DeclareMathDelimiter{\arrowvert}    % arrow without arrowheads
   {\mathord}{symbols}{"6A}{largesymbols}{"3C}
\DeclareMathDelimiter{\Arrowvert}    % double arrow without arrowheads
   {\mathord}{symbols}{"6B}{largesymbols}{"3D}
\DeclareMathDelimiter{\Vert}
   {\mathord}{symbols}{"6B}{largesymbols}{"0D}
%    \end{macrocode}
%    \cs{DeclareMathDelimiter} produces a command that is robust (with
%    an internal macro containing the payload) so we should not use
%    \cs{let} for making an alias
% \changes{v3.0d}{2019/09/21}{Distangle alias (gh/184)}
%    \begin{macrocode}
%\let\|=\Vert
\DeclareMathDelimiter{\|}
   {\mathord}{symbols}{"6B}{largesymbols}{"0D}
%    \end{macrocode}
%
%    \begin{macrocode}
\DeclareMathDelimiter{\vert}
   {\mathord}{symbols}{"6A}{largesymbols}{"0C}
\DeclareMathDelimiter{\uparrow}
   {\mathrel}{symbols}{"22}{largesymbols}{"78}
\DeclareMathDelimiter{\downarrow}
   {\mathrel}{symbols}{"23}{largesymbols}{"79}
\DeclareMathDelimiter{\updownarrow}
   {\mathrel}{symbols}{"6C}{largesymbols}{"3F}
\DeclareMathDelimiter{\Uparrow}
   {\mathrel}{symbols}{"2A}{largesymbols}{"7E}
\DeclareMathDelimiter{\Downarrow}
   {\mathrel}{symbols}{"2B}{largesymbols}{"7F}
\DeclareMathDelimiter{\Updownarrow}
   {\mathrel}{symbols}{"6D}{largesymbols}{"77}
\DeclareMathDelimiter{\backslash}    % for double coset G\backslash H
   {\mathord}{symbols}{"6E}{largesymbols}{"0F}
\DeclareMathDelimiter{\rangle}
   {\mathclose}{symbols}{"69}{largesymbols}{"0B}
\DeclareMathDelimiter{\langle}
   {\mathopen}{symbols}{"68}{largesymbols}{"0A}
\DeclareMathDelimiter{\rbrace}
   {\mathclose}{symbols}{"67}{largesymbols}{"09}
\DeclareMathDelimiter{\lbrace}
   {\mathopen}{symbols}{"66}{largesymbols}{"08}
\DeclareMathDelimiter{\rceil}
   {\mathclose}{symbols}{"65}{largesymbols}{"07}
\DeclareMathDelimiter{\lceil}
   {\mathopen}{symbols}{"64}{largesymbols}{"06}
\DeclareMathDelimiter{\rfloor}
   {\mathclose}{symbols}{"63}{largesymbols}{"05}
\DeclareMathDelimiter{\lfloor}
   {\mathopen}{symbols}{"62}{largesymbols}{"04}
%    \end{macrocode}
%
%  \begin{macro}{\lgroup}
%  \begin{macro}{\rgroup}
%  \begin{macro}{\bracevert}
%    There are three plain \TeX{} delimiters which are not fully
%    supported by NFSS, since they partly point into a bold cmr font.
%    Allocating a full symbol font, just to have three delimiters
%    seems a bit too much given the limited space available.  For this
%    reason only the extensible sizes are supported.  If this is not
%    desired one can use, without losing portability, define |\mathbf|
%    and |\mathtt| as font symbol alphabet (setting up
%    \texttt{cmr/bx/n} and \texttt{cmtt/m/n} as symbol fonts first)
%    and modify the delimiter declarations to point with their
%    small variant to those symbol fonts. (This is done in
%    \texttt{oldlfont.dtx} so look there for examples.)
%    \begin{macrocode}
\DeclareMathDelimiter{\lgroup} % extensible ( with sharper tips
     {\mathopen}{largesymbols}{"3A}{largesymbols}{"3A}
\DeclareMathDelimiter{\rgroup} % extensible ) with sharper tips
     {\mathclose}{largesymbols}{"3B}{largesymbols}{"3B}
\DeclareMathDelimiter{\bracevert} % the vertical bar that extends braces
     {\mathord}{largesymbols}{"3E}{largesymbols}{"3E}
%    \end{macrocode}
%  \end{macro}
%  \end{macro}
%  \end{macro}
%
% \subsection{Math versions of text commands}
%
% \changes{v2.2k}{1995/06/05}{Moved math commands from ltoutenc.dtx.}
%
% The |\mathunderscore| here is really a text definition, so it has
% been put back into |ltoutenc.dtx| (by Chris, 30/04/97) and should
% be removed from here.
%
% These symbols are the math versions of text commands such as |\P|,
% |\$|, etc.
% \begin{macro}{\mathparagraph}
% \changes{v2.2q}{1997/01/08}
%     {Define using \cs{DeclareMathSymbol}}
% \begin{macro}{\mathsection}
% \begin{macro}{\mathdollar}
% \begin{macro}{\mathsterling}
% \begin{macro}{\mathunderscore}
%    These math symbols are not in plain \TeX.
%    \begin{macrocode}
\DeclareMathSymbol{\mathparagraph}{\mathord}{symbols}{"7B}
\DeclareMathSymbol{\mathsection}{\mathord}{symbols}{"78}
\DeclareMathSymbol{\mathdollar}{\mathord}{operators}{"24}
%    \end{macrocode}
%
%    \begin{macrocode}
\DeclareRobustCommand\mathsterling{\mathit{\mathchar"7024}}
\DeclareRobustCommand\mathunderscore{\kern.06em\vbox{\hrule\@width.3em}}
%    \end{macrocode}
% \end{macro}
% \end{macro}
% \end{macro}
% \end{macro}
% \end{macro}
%
% \begin{macro}{\mathellipsis}
%    This is plain \TeX's |\ldots|.
%    \begin{macrocode}
\DeclareRobustCommand\mathellipsis{\mathinner{\ldotp\ldotp\ldotp}}%
%    \end{macrocode}
% \end{macro}
%
% \subsection{Other special functions and parameters}
%
% \subsubsection{Biggggg}
%
% \changes{v3.0b}{2018/09/24}{Start LR-mode if necessary (git/49)}
%    \begin{macrocode}
%</math>
%<*math|latexrelease>
%<latexrelease>\IncludeInRelease{2018/12/01}%
%<latexrelease>                 {\Big}{Start LR-mode}%
\DeclareRobustCommand\big[1]{\leavevmode@ifvmode
   {\hbox{$\left#1\vbox to8.5\p@{}\right.\n@space$}}}
\DeclareRobustCommand\Big[1]{\leavevmode@ifvmode
   {\hbox{$\left#1\vbox to11.5\p@{}\right.\n@space$}}}
\DeclareRobustCommand\bigg[1]{\leavevmode@ifvmode
   {\hbox{$\left#1\vbox to14.5\p@{}\right.\n@space$}}}
\DeclareRobustCommand\Bigg[1]{\leavevmode@ifvmode
   {\hbox{$\left#1\vbox to17.5\p@{}\right.\n@space$}}}
%</math|latexrelease>
%<latexrelease>\EndIncludeInRelease
%<latexrelease>\IncludeInRelease{0000/00/00}%
%<latexrelease>                 {\Big}{Start LR-mode}%
%<latexrelease>\def\big#1{{\hbox{$\left#1\vbox to8.5\p@{}\right.\n@space$}}}
%<latexrelease>\def\Big#1{{\hbox{$\left#1\vbox to11.5\p@{}\right.\n@space$}}}
%<latexrelease>\def\bigg#1{{\hbox{$\left#1\vbox to14.5\p@{}\right.\n@space$}}}
%<latexrelease>\def\Bigg#1{{\hbox{$\left#1\vbox to17.5\p@{}\right.\n@space$}}}
%<latexrelease>\EndIncludeInRelease
%<*math>
%    \end{macrocode}
%
%    \begin{macrocode}
\def\n@space{\nulldelimiterspace\z@ \m@th}
%    \end{macrocode}
%
%
%
% \subsubsection{The log-like functions}
%
% \begin{macro}{\operator@font}
%    The |\operator@font| determines the symbol font used for log-like
%    functions.
%    \begin{macrocode}
\def\operator@font{\mathgroup\symoperators}
%    \end{macrocode}
%  \end{macro}
%
%
% \subsubsection{Parameters}
%
%    \begin{macrocode}
\thinmuskip=3mu
\medmuskip=4mu plus 2mu minus 4mu
\thickmuskip=5mu plus 5mu
%    \end{macrocode}
%
%
%    This finishes the low-level setup in \texttt{fontmath.ltx}.
%    \begin{macrocode}
%</math>
%    \end{macrocode}
%
%
% \section{Default cfg files}
%
%    We provide default \texttt{cfg} files here to ensure that
%    on installations that search large file trees we do not pick up
%    some strange customisation files from somewhere.
% \changes{v2.2y}{2001/06/02}{Provide default cfg files (pr/3264)}
%    \begin{macrocode}
%<*cfgtext|cfgmath|cfgprel>
%%
%%
%%
%% Load the standard setup:
%%
%<+cfgtext>

\let\SAVEDUmathchar\Umathchar
\let\Umathchar\undefined

\ifx\SAVEDUmathchar\undefined



\let\SAVEDUmathchar\Umathchar
\let\Umathchar\undefined

\ifx\SAVEDUmathchar\undefined



\let\SAVEDUmathchar\Umathchar
\let\Umathchar\undefined

\ifx\SAVEDUmathchar\undefined

\input{fonttext.ltx}

\else
\let\saved@cdp@list\cdp@list
\input {tuenc.def}

\DeclareFontFamily{TU}{cmr}{}
\DeclareFontShape{TU}{cmr}{m}{n} {<->sub * lmr/m/n}{}
\let\cdp@list\saved@cdp@list
\let\saved@cdp@list\@undefined

\input{fonttext.ltx}
\def\@fontenc@load@list{\@elt{TU}}  % set this one explciitly

\fi


\let\Umathchar\SAVEDUmathchar

% just so you can check this format is being used
\def\FONTTEXTCONFIG{OT1-testing}


\else
\let\saved@cdp@list\cdp@list
\input {tuenc.def}

\DeclareFontFamily{TU}{cmr}{}
\DeclareFontShape{TU}{cmr}{m}{n} {<->sub * lmr/m/n}{}
\let\cdp@list\saved@cdp@list
\let\saved@cdp@list\@undefined



\let\SAVEDUmathchar\Umathchar
\let\Umathchar\undefined

\ifx\SAVEDUmathchar\undefined

\input{fonttext.ltx}

\else
\let\saved@cdp@list\cdp@list
\input {tuenc.def}

\DeclareFontFamily{TU}{cmr}{}
\DeclareFontShape{TU}{cmr}{m}{n} {<->sub * lmr/m/n}{}
\let\cdp@list\saved@cdp@list
\let\saved@cdp@list\@undefined

\input{fonttext.ltx}
\def\@fontenc@load@list{\@elt{TU}}  % set this one explciitly

\fi


\let\Umathchar\SAVEDUmathchar

% just so you can check this format is being used
\def\FONTTEXTCONFIG{OT1-testing}

\def\@fontenc@load@list{\@elt{TU}}  % set this one explciitly

\fi


\let\Umathchar\SAVEDUmathchar

% just so you can check this format is being used
\def\FONTTEXTCONFIG{OT1-testing}


\else
\let\saved@cdp@list\cdp@list
\input {tuenc.def}

\DeclareFontFamily{TU}{cmr}{}
\DeclareFontShape{TU}{cmr}{m}{n} {<->sub * lmr/m/n}{}
\let\cdp@list\saved@cdp@list
\let\saved@cdp@list\@undefined



\let\SAVEDUmathchar\Umathchar
\let\Umathchar\undefined

\ifx\SAVEDUmathchar\undefined



\let\SAVEDUmathchar\Umathchar
\let\Umathchar\undefined

\ifx\SAVEDUmathchar\undefined

\input{fonttext.ltx}

\else
\let\saved@cdp@list\cdp@list
\input {tuenc.def}

\DeclareFontFamily{TU}{cmr}{}
\DeclareFontShape{TU}{cmr}{m}{n} {<->sub * lmr/m/n}{}
\let\cdp@list\saved@cdp@list
\let\saved@cdp@list\@undefined

\input{fonttext.ltx}
\def\@fontenc@load@list{\@elt{TU}}  % set this one explciitly

\fi


\let\Umathchar\SAVEDUmathchar

% just so you can check this format is being used
\def\FONTTEXTCONFIG{OT1-testing}


\else
\let\saved@cdp@list\cdp@list
\input {tuenc.def}

\DeclareFontFamily{TU}{cmr}{}
\DeclareFontShape{TU}{cmr}{m}{n} {<->sub * lmr/m/n}{}
\let\cdp@list\saved@cdp@list
\let\saved@cdp@list\@undefined



\let\SAVEDUmathchar\Umathchar
\let\Umathchar\undefined

\ifx\SAVEDUmathchar\undefined

\input{fonttext.ltx}

\else
\let\saved@cdp@list\cdp@list
\input {tuenc.def}

\DeclareFontFamily{TU}{cmr}{}
\DeclareFontShape{TU}{cmr}{m}{n} {<->sub * lmr/m/n}{}
\let\cdp@list\saved@cdp@list
\let\saved@cdp@list\@undefined

\input{fonttext.ltx}
\def\@fontenc@load@list{\@elt{TU}}  % set this one explciitly

\fi


\let\Umathchar\SAVEDUmathchar

% just so you can check this format is being used
\def\FONTTEXTCONFIG{OT1-testing}

\def\@fontenc@load@list{\@elt{TU}}  % set this one explciitly

\fi


\let\Umathchar\SAVEDUmathchar

% just so you can check this format is being used
\def\FONTTEXTCONFIG{OT1-testing}

\def\@fontenc@load@list{\@elt{TU}}  % set this one explciitly

\fi


\let\Umathchar\SAVEDUmathchar

% just so you can check this format is being used
\def\FONTTEXTCONFIG{OT1-testing}

%<+cfgmath>\input{fontmath.ltx}
%<+cfgprel>% \iffalse meta-comment
%
% Copyright (C) 1993-2019
% The LaTeX3 Project and any individual authors listed elsewhere
% in this file.
%
% This file is part of the LaTeX base system.
% -------------------------------------------
%
% It may be distributed and/or modified under the
% conditions of the LaTeX Project Public License, either version 1.3c
% of this license or (at your option) any later version.
% The latest version of this license is in
%    https://www.latex-project.org/lppl.txt
% and version 1.3c or later is part of all distributions of LaTeX
% version 2008 or later.
%
% This file has the LPPL maintenance status "maintained".
%
% The list of all files belonging to the LaTeX base distribution is
% given in the file `manifest.txt'. See also `legal.txt' for additional
% information.
%
% The list of derived (unpacked) files belonging to the distribution
% and covered by LPPL is defined by the unpacking scripts (with
% extension .ins) which are part of the distribution.
%
% \fi
%
% \iffalse
%%% From File: preload.dtx
%<*dtx>
           \ProvidesFile{preload.dtx}
%</dtx>
%<*preload>
%<*!tex>
%<+cm>  \ProvidesFile{cmpreloa.%
%<+dc>  \ProvidesFile{dcpreloa.%
%<+xpt>                         xpt}
%<+xipt>                        xip}
%<+xiipt>                       xii}
%<+min> \ProvidesFile{preload.min}
%<+ori> \ProvidesFile{preload.ori}
%</!tex>
%<+tex> \ProvidesFile{preload.ltx}
% \fi
%       \ProvidesFile{preload.dtx}
         [2014/09/29 v2.1g LaTeX Kernel (Font Preloading)]
%
%
%
%\iffalse       This is a META comment
%
% File `preload.dtx'.
% Copyright (C) 1989-1994 Frank Mittelbach and Rainer Sch\"opf,
% all rights reserved.
%
% \fi
%
% \GetFileInfo{preload.dtx}
% \title{The \texttt{preload.dtx} file\thanks {This file has version
%    number \fileversion, dated \filedate}\\ for use with \LaTeXe}
% \date{\filedate}
% \author{Frank Mittelbach \and Rainer Sch\"opf}
%
% \changes{v2.0b}{1993/03/08}{Added 12pt preloads}
% \changes{v2.1e}{1994/11/07}{(DPC) Updated to use \cs{ProvidesFile}}
% \changes{v2.1g}{1998/08/17}{(RmS) Minor documentation fixes.}
%
% \def\dst{\expandafter{\normalfont\scshape docstrip}}
%
% \setcounter{StandardModuleDepth}{1}
%
% \MaintainedByLaTeXTeam{latex}
% \maketitle
%
% \section{Overview}
%
%   This file contains an number of possible settings for preloading
%   fonts during installation of NFSS2 (which is used by \LaTeXe).  It
%   will be used to generate the following files:
%   \begin{center}
%   \begin{tabular}{ll}
%   preload.min   &  minimal subset of fonts necessary to run NFSS2 \\
%   preload.ori   &  preload of CM fonts similar to the old
%                        \texttt{lfonts.tex}                       \\
%   preload.ltx    &  The standard selection of preloads \\
%   cmpreloa.xpt   &  preload of CM fonts for 10pt document size\\
%   cmpreloa.xip   &  preload of CM fonts for 11pt document size\\
%   cmpreloa.xii   &  preload of CM fonts for 12pt document size\\
%   dcpreloa.xpt   &  preload of DC fonts for 10pt size \\
%   dcpreloa.xip   &  preload of DC fonts for 11pt size \\
%   dcpreloa.xii   &  preload of DC fonts for 12pt size \\
%   \end{tabular}
%   \end{center}
%
%    These files are for installations that make use of Computer
%    Modern fonts either old encoding (OT1) or Cork encoding (T1). The
%    Computer Modern fonts with Cork encoding are known as DC-fonts.
%
%    Most important is \texttt{preload.ltx} which is used during
%    format generation. You are \emph{not} allowed to change this file.
%
% \section{Customization}
%
%    You can customize the preloaded fonts in your \LaTeXe{} system by
%    installing a file with the name \texttt{preload.cfg}. If this
%    file exists it will be used in place of the system file
%    \texttt{preload.ltx}.  You can, for example, copy one of the
%    files mentioned above (that can be generated from this source) to
%    \texttt{preload.cfg}.
%
%    Or you can define completely other preloads. In that case start
%    from \texttt{preload.min} since that contains the fonts that have
%    to be preloaded by *all* \LaTeXe{} systems.
%
%    Avoid using \texttt{preload.ori}, it will load so many fonts that
%    on most installations it is nearly impossible to load other font
%    families afterwards. This file is only generated to show what
%    fonts have been preloaded by \LaTeX~2.09.
%
%    If you normally use other fonts than Computer Modern
%    \texttt{preload.min} might be best.
%
%    \begin{quote} \textbf{Warning:} If you preload fonts with
%    encodings other than the normally supported encodings you have to
%    declare that encoding in a \texttt{fontdef.cfg} configuration
%    file (see the documentation in the file \texttt{fontdef.dtx}).
%    Adding an extra encoding to the format might produce non-portable
%    documents, thus this should be avoided if possible.
%    \end{quote}
%
%
% \StopEventually{}
%
% \section{Module switches for the \dst{} program}
%
%  The \dst{} will generate the above file from this source using the
%  following module directives:
% \begin{center}
% \begin{tabular}{ll}
%   driver & produce a documentation driver file \\
%   preload& produce a preload\ldots file \\[2pt]
%   cm     & for OT1 encoded Computer Modern \\
%   dc     & for T1 encoded Computer Modern \\[2pt]
%   min    & produce minimal subset \\
%   xpt    & produce 10pt preloads \\
%   xipt   & produce 11pt preloads \\
%   xiipt  & produce 12pt preloads \\
%   ori    & produce preloads similar to old \texttt{lfonts.tex}\\
%   tex    & produce preload.ltx\\
% \end{tabular}
% \end{center}
% A typical \dst{} command file would then have entries like:
% \begin{verbatim}
%\generateFile{preload.min}{t}{\from{preload.dtx}{preload,min}}
%\end{verbatim}
% for generating preload files.
%
% \section{A driver for this document}
%
%    The next bit of code contains the documentation driver file for
%    \TeX{}, i.e., the file that will produce the documentation you
%    are currently reading. It will be extracted from this file by the
%    \dst{} program.
%    \begin{macrocode}
%<*driver>
\documentclass{ltxdoc}
%\OnlyDescription  % comment out for implementation details
\begin{document}
   \DocInput{preload.dtx}
\end{document}
%</driver>
%    \end{macrocode}
%
%
% \section{The code}
%
%    We begin by loading the math extension font (cmex10)
%    and the \LaTeX{} line and circle fonts.
%    It is necessary to do this explicitly since these are
%    used by \texttt{lplain.tex} and \texttt{latex.tex}.
%    Since the internal font name contains |/| characters
%    and digits we construct the name via |\csname|.
%    These are the only fonts (!) that must be loaded in this file.
%
%    All |\DeclarePreloadSizes| can be removed or others can be added,
%    they only influence the processing speed.
% \changes{v2.0c}{1993/08/13}{Added \cs{relax} at end of font names.}
%    \begin{macrocode}
\expandafter\font\csname OMX/cmex/m/n/10\endcsname=cmex10\relax
\font\tenln  =line10   \font\tenlnw  =linew10\relax
\font\tencirc=lcircle10 \font\tencircw=lcirclew10\relax
%    \end{macrocode}
%    The above fonts should not be touched but anything below this
%    point here in the preload suggestions can be modified without any
%    problems.
%    \begin{macrocode}
%<-tex>%*******************************************
%<-tex>% Start any modification below this point **
%<-tex>%*******************************************
%<-tex>
%%
%% Computer Modern Roman:
%%-----------------------
%<*ori>
\DeclarePreloadSizes{OT1}{cmr}{m}{n}
        {5,6,7,8,9,10,10.95,12,14.4,17.28,20.74,24.88}
\DeclarePreloadSizes{OT1}{cmr}{bx}{n}{9,10,10.95,12,14.4,17.28}
\DeclarePreloadSizes{OT1}{cmr}{m}{sl}{10,10.95,12}
\DeclarePreloadSizes{OT1}{cmr}{m}{it}{7,8,9,10,10.95,12}
%</ori>
%<+xpt&cm> \DeclarePreloadSizes{OT1}{cmr}{m}{n}{5,7,10}
%<+xpt&dc> \DeclarePreloadSizes{T1}{cmr}{m}{n}{5,7,10}
%<+xipt&cm> \DeclarePreloadSizes{OT1}{cmr}{m}{n}{6,8,10.95}
%<+xipt&dc> \DeclarePreloadSizes{T1}{cmr}{m}{n}{6,8,10.95}
%<+xiipt&cm> \DeclarePreloadSizes{OT1}{cmr}{m}{n}{6,8,12}
%<+xiipt&dc> \DeclarePreloadSizes{T1}{cmr}{m}{n}{6,8,12}
%%
%% Computer Modern Sans:
%%----------------------
%<+ori> \DeclarePreloadSizes{OT1}{cmss}{m}{n}{10,10.95,12}
%%
%% Computer Modern Typewriter:
%%----------------------------
%<+ori> \DeclarePreloadSizes{OT1}{cmtt}{m}{n}{9,10,10.95,12}
%%
%% Computer Modern Math:
%%----------------------
%<*ori>
\DeclarePreloadSizes{OML}{cmm}{m}{it}
         {5,6,7,8,9,10,10.95,12,14.4,17.28,20.74}
\DeclarePreloadSizes{OMS}{cmsy}{m}{n}
         {5,6,7,8,9,10,10.95,12,14.4,17.28,20.74}
%</ori>
%    \end{macrocode}
%
%    The math fonts are the same for both DC and CM fonts. So far
%    there isn't an agreed on standard.
% \changes{v2.4e}{1995/12/04}
%      {Ulrik Vieth. added 12pt OMS and OML preloads  /1989}
%    \begin{macrocode}
%<*xpt>
\DeclarePreloadSizes{OML}{cmm}{m}{it}{5,7,10}
\DeclarePreloadSizes{OMS}{cmsy}{m}{n}{5,7,10}
%</xpt>
%<*xipt>
\DeclarePreloadSizes{OML}{cmm}{m}{it}{6,8,10.95}
\DeclarePreloadSizes{OMS}{cmsy}{m}{n}{6,8,10.95}
%</xipt>
%<*xiipt>
\DeclarePreloadSizes{OML}{cmm}{m}{it}{6,8,12}
\DeclarePreloadSizes{OMS}{cmsy}{m}{n}{6,8,12}
%</xiipt>
%%
%% LaTeX symbol fonts:
%%--------------------
%<*ori>
\DeclarePreloadSizes{U}{lasy}{m}{n}
         {5,6,7,8,9,10,10.95,12,14.4,17.28,20.74}
%</ori>
%</preload>
%    \end{macrocode}
%
%
%
% \Finale
%
\endinput

%%
%% Small changes could go here; see documentation in cfgguide.tex for
%% allowed modifications.
%%
%% In particular it is not allowed to misuse this configuration file
%% to modify internal LaTeX commands!
%%
%% If you use this file as the basis for configuration please change
%% the \ProvidesFile lines to clearly identify your modification, e.g.,
%%
%<+cfgtext>%%  \ProvidesFile{fonttext.cfg}[2001/06/01
%<+cfgmath>%%  \ProvidesFile{fonttext.cfg}[2001/06/01
%<+cfgprel>%%   \ProvidesFile{preload.cfg}[2001/06/01
%%                              Customised local font setup]
%%
%%
%</cfgtext|cfgmath|cfgprel>
%    \end{macrocode}
%
% \Finale
%
\endinput
