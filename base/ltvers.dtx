% \iffalse meta-comment
%
% Copyright (C) 1993-2020
% The LaTeX3 Project and any individual authors listed elsewhere
% in this file.
%
% This file is part of the LaTeX base system.
% -------------------------------------------
%
% It may be distributed and/or modified under the
% conditions of the LaTeX Project Public License, either version 1.3c
% of this license or (at your option) any later version.
% The latest version of this license is in
%    https://www.latex-project.org/lppl.txt
% and version 1.3c or later is part of all distributions of LaTeX
% version 2008 or later.
%
% This file has the LPPL maintenance status "maintained".
%
% The list of all files belonging to the LaTeX base distribution is
% given in the file `manifest.txt'. See also `legal.txt' for additional
% information.
%
% The list of derived (unpacked) files belonging to the distribution
% and covered by LPPL is defined by the unpacking scripts (with
% extension .ins) which are part of the distribution.
%
% \fi
%
% \iffalse
%%% From File: ltvers.dtx
%
%<*driver>
% \fi
\ProvidesFile{ltvers.dtx}
             [2020/03/02 v1.1e LaTeX Kernel (Version Info)]
% \iffalse
\documentclass{ltxdoc}
\GetFileInfo{ltvers.dtx}
\title{\filename}
\date{\filedate}
 \author{%
  Johannes Braams\and
  David Carlisle\and
  Alan Jeffrey\and
  Leslie Lamport\and
  Frank Mittelbach\and
  Chris Rowley\and
  Rainer Sch\"opf}
\begin{document}
 \MaintainedByLaTeXTeam{latex}
 \maketitle
 \DocInput{\filename}
\end{document}
%</driver>
% \fi
%
%
% \section{Version Identification}
% First we identify the date and version number of this release of
% \LaTeX, and set |\everyjob| so that it is printed at the start of
% every \LaTeX\ run.
%
% \StopEventually{}
%
% \changes{v1.0g}{1996/11/28}
%     {Check for old format modified /2319}
% \changes{v1.0f}{1996/11/20}
%     {Check for old format modified /2319}
% \changes{v1.0e}{1995/05/12}
%     {Add autoload docstrip guards}
% \changes{v1.0e}{1995/05/12}
%     {Check for format older than 1 year}
% \changes{v1.0d}{1994/05/25}
%     {Remove PRELIMINARY TEST RELEASE from startup banner
%      (spring is here)}
% \changes{v1.0b}{1994/04/12}
%     {Have version info generated automatically.}
% \changes{v1.0a}{1994/03/04}
%     {Initial version, split from latex.dtx}
% \changes{v1.0r}{2015/02/21}{Removed autoload code}
% \changes{v1.0t}{2015/06/23}
%     {set \cs{patch@level} in ltvers rather than in ltfinal/ltpatch}
%
% \begin{macro}{\fmtname}
% \begin{macro}{\fmtversion}
% \begin{macro}{\latexreleaseversion}
% \begin{macro}{\patch@level}
%    A |\patch@level| of \texttt{0} or higher denotes an official
%    public release. A negative value indicates a candidate release
%    that is not distributed.
%
%    If we put code updates into the kernel that are supposed to go
%    into the next release we set the |\patch@level| to \texttt{-1}
%    and the |\fmtversion| / |\latexreleaseversion| to  the dated of
%    the next release (guessed, the real value is not so important and
%    will get corrected when we make the release official).
%
%    If the |\patch@level| is already at \texttt{-1} we do nothing
%    here and use the |\fmtversion| date for  any new|\IncludeInRelease|
%    line when we add further code.
%
%    Finally, if we do make a public release we either just set the
%    |\patch@level| to zero (if our initial guess was good) or we also
%    change the date and then have to additionally change to that date
%    on all the |\IncludeInRelease| statements that used the
%    ``guessed'' date.
%    \begin{macrocode}
%<*2ekernel>
\def\fmtname{LaTeX2e}
\edef\fmtversion
%</2ekernel>
%<latexrelease>\edef\latexreleaseversion
%<*2ekernel|latexrelease>
   {2020-10-01}
%</2ekernel|latexrelease>
%<*2ekernel>
\def\patch@level{-6}
%    \end{macrocode}
%
% \begin{macro}{\development@branch@name}
%    For more fine grain control there is the possibility to name the
%    current development branch. This is only used when the
%    \cs{patch@level} is negative (i.e., a pre-release format) and is
%    intended to help us internally when we locally install a
%    format out of some development branch.
%    \begin{macrocode}
\edef\development@branch@name{develop \the\year-\the\month-\the\day}
%    \end{macrocode}
% \end{macro}
% \end{macro}
% \end{macro}
% \end{macro}
% \end{macro}
%
% Check that the format being made is not too old.
% The error message complains about `more than 5 years'
% but in fact the error is not triggered until 65 months.
%
% This code is currently not activated as we don't know if we already
% got to the last official 2e version (due to staff shortage or due to
% a successor (think positive:-)).
% \changes{v1.0i}{2001/06/04}{Check for old format disabled}
% \changes{v1.0k}{2004/01/28}{Check for old format made 5 years (pr/3601)}
% \changes{v1.0l}{2009/09/24}{Stop checking for old format}
%    \begin{macrocode}
\iffalse
\def\reserved@a#1/#2/#3\@nil{%
  \count@\year
  \advance\count@-#1\relax
  \multiply\count@ by 12\relax
  \advance\count@\month
  \advance\count@-#2\relax}
\expandafter\reserved@a\fmtversion\@nil
%    \end{macrocode}
% |\count@| is now the age of this file in months. Take a generous
% definition of `year' so this message is not generated too often.
%    \begin{macrocode}
\ifnum\count@>65
  \typeout{^^J%
!!!!!!!!!!!!!!!!!!!!!!!!!!!!!!!!!!!!!!!!!!!!!!!!!!!!!!!!!!!!!!!!!!^^J%
!  You are attempting to make a LaTeX format from a source file^^J%
!  That is more than five years old.^^J%
!^^J%
!  If you enter <return> to scroll past this message then the format^^J%
!  will be built, but please consider obtaining newer source files^^J%
!  before continuing to build LaTeX.^^J%
!!!!!!!!!!!!!!!!!!!!!!!!!!!!!!!!!!!!!!!!!!!!!!!!!!!!!!!!!!!!!!!!!!^^J%
}
   \errhelp{To avoid this error message, obtain new LaTeX sources.}
   \errmessage{LaTeX source files more than 5 years old!}
\fi
\let\reserved@a\relax
\fi
%    \end{macrocode}
%
% \changes{v1.0p}{2015/01/22}{Preserve any \cs{everyjob} material inserted
%   by a loader (\texttt{.ini} file)}
% \changes{v1.0v}{2015/08/23}{Allow negative patchlevel for pre-release}
% \changes{v1.1e}{2020/01/31}{Allow for upcoming format as pre-release 0}
%    \begin{macrocode}
  \ifnum0\ifnum\patch@level=0 \ifx\development@branch@name\@empty 1\fi\fi>0 %
    \everyjob\expandafter{\the\everyjob
      \typeout{\fmtname \space<\fmtversion>}}
    \immediate
    \write16{\fmtname \space<\fmtversion>}
  \else\ifnum\patch@level>0
    \everyjob\expandafter{\the\everyjob
      \typeout{\fmtname \space<\fmtversion> patch level \patch@level}}
    \immediate
    \write16{\fmtname \space<\fmtversion> patch level \patch@level}
  \else
    \everyjob\expandafter{\the\everyjob
      \typeout{\fmtname \space<\fmtversion> pre-release-\number-\patch@level\space
        \ifx\development@branch@name\@undefined  \else
          \ifx\development@branch@name\@empty  \else
            \space (\development@branch@name\space branch)%
          \fi
        \fi
    }}
    \immediate
    \write16{\fmtname \space<\fmtversion> pre-release-\number-\patch@level\space
        \ifx\development@branch@name\@undefined  \else
          \ifx\development@branch@name\@empty  \else
            \space (\development@branch@name\space branch)%
          \fi
        \fi
    }
    \fi
  \fi
%</2ekernel>
%    \end{macrocode}
%
% \begin{macro}{\IncludeInRelease}
% \changes{v1.0w}{2015/02/21}
%         {set \cs{@currname} empty here (in case \cs{IncludeInRelease} input early)}
%    \begin{macrocode}
%<2ekernel>\let\@currname\@empty
%    \end{macrocode}
%
% \changes{v1.0n}{2015/01/07}{macro added}
% \changes{v1.0m}{2015/01/17}{modified with \cs{@currname}}
% \changes{v1.0o}{2015/01/19}{Optional argument}
% \changes{v1.0q}{2015/02/19}{Swap argument order}
%
%    \begin{macrocode}
%<*2ekernel|latexrelease>
%<latexrelease>\newif\if@includeinrelease
%<latexrelease>\@includeinreleasefalse
%    \end{macrocode}
%    \begin{macrocode}
\def\IncludeInRelease#1{%
 \if@includeinrelease
   \PackageError{latexrelease}{mis-matched IncludeInRelease}%
                {There is an \string\EndIncludeRelease\space missing}%
  \@includeinreleasefalse
  \fi
  \kernel@ifnextchar[%
  {\@IncludeInRelease{#1}}
  {\@IncludeInRelease{#1}[#1]}}
%    \end{macrocode}
%
% If a specific date has not been specified in |latexrelease|
% use `|#1|`.
%    \begin{macrocode}
\def\@IncludeInRelease#1[#2]{\@IncludeInRele@se{#2}}
%    \end{macrocode}
%
%    \begin{macrocode}
\def\@IncludeInRele@se#1#2#3{%
  \toks@{[#1] #3}%
  \expandafter\ifx\csname\string#2+\@currname+IIR\endcsname\relax
%    \end{macrocode}
%    If we roll back and the first patch already match then applying
%    that is actually reapplying what is already in the format, i.e.,
%    it is useless and possibly allocating new registers. However, it
%    makes the logic simpler so this is the way it is for now. In
%    theory we could always jump overthe first patch because that is
%    only really needed for rolling forward. So maybe one day \dots
%    \begin{macrocode}
    \ifnum\expandafter\@parse@version#1//00\@nil
          >\expandafter\@parse@version\fmtversion//00\@nil
      \GenericInfo{}{Skipping: \the\toks@}%
     \expandafter\expandafter\expandafter\@gobble@IncludeInRelease
    \else
      \GenericInfo{}{Applying: \the\toks@}%
      \@includeinreleasetrue
      \expandafter\let\csname\string#2+\@currname+IIR\endcsname\@empty
    \fi
  \else
    \GenericInfo{}{Already applied: \the\toks@}%
    \expandafter\@gobble@IncludeInRelease
  \fi
}
%    \end{macrocode}
%
%    \begin{macrocode}
\def\EndIncludeInRelease{%
\if@includeinrelease
  \@includeinreleasefalse
\else
  \PackageError{latexrelease}{mis-matched EndIncludeInRelease}{}%
\fi}
%    \end{macrocode}
%
%    \begin{macrocode}
\long\def\@gobble@IncludeInRelease#1\EndIncludeInRelease{%
  \@includeinreleasefalse
  \@check@IncludeInRelease#1\IncludeInRelease\@check@IncludeInRelease
  \@end@check@IncludeInRelease}
%    \end{macrocode}
%    \begin{macrocode}
\long\def\@check@IncludeInRelease#1\IncludeInRelease
                                   #2#3\@end@check@IncludeInRelease{%
  \ifx\@check@IncludeInRelease#2\else
    \PackageError{latexrelease}{skipped IncludeInRelease for tag \string#2}{}%
  \fi}
%    \end{macrocode}
%
%    \begin{macrocode}
%</2ekernel|latexrelease>
%    \end{macrocode}
% \end{macro}
% \Finale
%
\endinput
