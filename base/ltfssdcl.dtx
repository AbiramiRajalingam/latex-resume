% \iffalse meta-comment
%
% Copyright (C) 1993-2021
% The LaTeX Project and any individual authors listed elsewhere
% in this file.
%
% This file is part of the LaTeX base system.
% -------------------------------------------
%
% It may be distributed and/or modified under the
% conditions of the LaTeX Project Public License, either version 1.3c
% of this license or (at your option) any later version.
% The latest version of this license is in
%    https://www.latex-project.org/lppl.txt
% and version 1.3c or later is part of all distributions of LaTeX
% version 2008 or later.
%
% This file has the LPPL maintenance status "maintained".
%
% The list of all files belonging to the LaTeX base distribution is
% given in the file `manifest.txt'. See also `legal.txt' for additional
% information.
%
% The list of derived (unpacked) files belonging to the distribution
% and covered by LPPL is defined by the unpacking scripts (with
% extension .ins) which are part of the distribution.
%
% \fi
% \iffalse
%%% From File: ltfssdcl.dtx
%% Copyright (C) 1989-1998 Frank Mittelbach and Rainer Sch\"opf,
%% all rights reserved.
%
%<*driver>
% \fi
%
%
\ProvidesFile{ltfssdcl.dtx}
            [2020/12/05 v3.0v LaTeX Kernel (NFSS Declarative Interface)]
% \iffalse
\documentclass{ltxdoc}
\begin{document}
\DocInput{ltfssdcl.dtx}
\end{document}
%</driver>
% \fi
%
% \iffalse
%<+checkmem>\def\CHECKMEM{\tracingstats=2
%<+checkmem>  \newlinechar=`\^^J
%<+checkmem>  \message{^^JMemory usage: \filename}\shipout\hbox{}}
%<+checkmem>\CHECKMEM
% \fi
%
%
%
% \GetFileInfo{ltfssdcl.dtx}
% \title{A new font selection scheme for \TeX{} macro packages\\
%        (Declarative Interface)\thanks
%       {This file has version number
%       \fileversion\ dated \filedate}}
%
% \author{Frank Mittelbach \and Rainer Sch\"opf}
%
% \MaintainedByLaTeXTeam{latex}
% \maketitle
%
% This file contains the main implementation of the font selection
% scheme commands. See other parts of the \LaTeX\ distribution, or
% \emph{The \LaTeX\ Companion} for higher level documentation of these
% commands.
%
% \begin{quote}
%   \textbf{Warning:}
%  The macro documentation is still basically the documentation from the
%  first NFSS release and therefore in some cases probably not
%  completely accurate.
% \end{quote}
%
% \StopEventually{}
%
% \section{Interface Commands}
%
% \changes{v3.0i}{1998/08/17}
%      {(RmS) Corrected minor glitches in changes entries.}
% \changes{v3.0b}{1995/06/15}
%      {(DPC) minor documentation changes}
% \changes{v3.0a}{1995/05/24}
%      {(DPC) Make file from previous file, latint.dtx 1995/05/21 v2.1t}
% \changes{v3.0p}{2015/02/21}
%         {Removed autoload code}
%
%
% \changes{v2.1o}{1994/11/22}
%         {wrap long lines}
% \changes{v2.1m}{1994/11/17}
%         {\cs{@tempa} to \cs{reserved@a}}
% \changes{v2.1j}{1994/05/29}{Use new error commands}
% \changes{v2.1h}{1994/05/13}{Removed file identification typeout}
% \changes{v2.1g}{1994/05/12}{Allow \cs{relax}'ed cmds to be declared}
% \changes{v2.1g}{1994/05/12}{Allow \cs{relax} as undefined command}
% \changes{v2.1e}{1994/04/28}{Removed all \cs{uppercase} in hex num
%                             parsing macros}
% \changes{v2.1d}{1994/04/18}{Removed surplus \cs{no@alphabet@error}
%                             (see fam.dtx)}
%
%
%
% \changes{v2.1f}{1994/05/03}{Renamed \cs{@@DeclareMathDelimiter} to
%                             \cs{@DeclareMathDelimiter}}
% \changes{v2.1c}{1994/03/13}{add 2ekernel module to omit repeated code}
%
%
% \begin{macro}{\in@}
% \changes{v3.0m}{2009/10/20}{More robust thanks to Heiko.}
% \changes{v3.0n}{2011/05/08}{Simplified thanks to Bruno.}
% \begin{macro}{\ifin@}
%    |\@in| is a utility macro with two arguments.  It determines
%    whether its first argument occurs in its second and sets the
%    switch |\ifin@| accordingly. The first argument may not
%    contain braces nor |#| (more precisely, tokens of category
%    code 1,~2, or~6).
%    \begin{macrocode}
%<*2ekernel>
\def\in@#1#2%
 {%
   \begingroup
     \def\in@@##1#1{}%
     \toks@\expandafter{\in@@#2{}{}#1}%
     \edef\in@@{\the\toks@}%
   \expandafter\endgroup
   \ifx\in@@\@empty
     \in@false
   \else
     \in@true
   \fi
 }
\newif\ifin@
%    \end{macrocode}
% \end{macro}
% \end{macro}
%
% Before the |\begin{document}| command several
% \meta{math versions} and \meta{math alphabet identifiers} may
% be declared.
% In principle, there should be exactly one family/series/shape
% combination be declared for each version/alphabet pair.
% But we want to allow for defaults as well for automagical
% filling of holes.
%
% While building the tables for math alphabet identifiers and math
% versions we keep several lists:
% \begin{itemize}
%   \item the list of all math versions, |\version@list|,
%         each entry prefixed by the control sequence |\version@elt|,
%         i.e.\ this list has the following form
%         \begin{quote}
%           |\version@elt|\meta{version$_1$}^^A
%           |\version@elt|\meta{version$_2$}\ldots\\\hspace*{\fill}^^A
%           |\version@elt|\meta{version$_n$}
%         \end{quote}
%   \item the list of all math alphabet identifiers.  Here every
%         entry has the form:\\[2pt]
%         \hspace*{\MacroIndent}
%         |\group@elt|\meta{math group number}\\
%         \hspace*{\MacroIndent}
%         |{{|\meta{default family}^^A
%         |}{|\meta{default series}^^A
%         |}{|\meta{default shape}|}}|.
%   \item Each defined math alphabet identifier holds a list containing
%         Information about the {\em versions\/} for which it is
%         defined.  This list has a more complicated structure: it
%         looks as follows:
%         \begin{quote}
%           |\set@alpha|\meta{the alphabet identifier itself}\\
%           \hspace*{\MacroIndent}
%             |\reserved@c|\meta{math version}\meta{font info}\\
%           \hspace*{\MacroIndent}\ldots\\
%           |\@nil|
%         \end{quote}
%         where \meta{font info} is either |\reserved@e|
%         (if the combination is not defined yet) or
%         \begin{quote}
%           |{{|\meta{family}|}{|\meta{series}^^A
%           |}{|\meta{shape}|}}|
%         \end{quote}
% \end{itemize}
%
% \begin{macro}{\version@list}
%    We initialize the version list to be empty.
%    \begin{macrocode}
\let\version@list=\@empty
\@onlypreamble\version@list
%    \end{macrocode}
% \end{macro}
%
%
% \begin{macro}{\version@elt}
%    \begin{macrocode}
\let\version@elt\relax
\@onlypreamble\version@elt
%    \end{macrocode}
% \end{macro}
%
%
% \begin{macro}{\new@mathversion}
%    The macro |\new@mathversion| is called with the version control
%    sequence as its argument.
%    \begin{macrocode}
%\def\new@mathversion#1{%
%    \end{macrocode}
%    The first thing this macro does is to check if the version
%    identifier is already present in |\version@list|.
%    We enclose |\version@list| in braces since it might be empty
%    (if no {\em version\/} is defined yet).  But this means that
%    we need a suitable number of |\expandafter| primitives.
%    \begin{macrocode}
%  \expandafter\in@\expandafter#1\expandafter{\version@list}%
%  \ifin@
%    \end{macrocode}
%    If so it prints an error message.
%    The |\next| macro is used to get rid of the four characters
%    |\mv@| that would otherwise appear at the begin of the
%    version name in the error message.
%    \begin{macrocode}
%    \@latex@error{Math version
%               `\expandafter\@gobblefour\string#1'
%               already defined}\@eha
%    \end{macrocode}
%    Otherwise we have a new version, and we can proceed with
%    entering it into the tables.
%    We add it to |\version@list|.  This is very easy:
%    we define |\version@elt| (which is the delimiter in
%    |\version@list|) to protect itself and the following token
%    from being expanded and simply redefine |\version@list|.
%    \begin{macrocode}
%  \else
%     \global\expandafter\newcount\csname c@\expandafter
%                                 \@gobble\string#1\endcsname
%     \global\csname c@\expandafter
%                                 \@gobble\string#1\endcsname\@ne
%     \def\version@elt{\noexpand\version@elt\noexpand}%
%     \edef\version@list{\version@list\version@elt#1}%
%    \end{macrocode}
%    Then we prepare to enter the new version into all math
%    alphabet identifier lists.  Remember that these lists
%    use |\reserved@c| as delimiter, and that there appears the
%    control sequence |\reserved@e| that must not be expanded.
%    Therefore we take suitable precautions.
%    \begin{macrocode}
%     \def\reserved@c{\noexpand\reserved@c\noexpand}%
%     \let\reserved@e\relax
%    \end{macrocode}
%    We will now go through the |\alpha@list| to process every
%    \meta{math alphabet identifier} in turn.
%    Since this list has |\group@elt| as a delimiter we define
%    this control sequence.  It has three arguments as every
%    entry consists of three items (as explained above).
%    \begin{macrocode}
%     \def\group@elt##1##2##3{%
%    \end{macrocode}
%    The first of these arguments is the \meta{math alphabet
%    identifier}.  We redefine it by appending the information
%    about the new version at the end of the list contained in
%    it.  However, there is one subtlety: the definitions for
%    |\reserved@c| and |\reserved@e| made above prevent the
%    main part of the list from being expanded.  But we still have
%    to take care of the header and the trailer.  To do this we
%    remove the trailer by means of the macro |\remove@nil|
%    which also protect the header from being expanded.  Its
%    definition is given below.  Now we can prepare to add the
%    new version.
%    \begin{macrocode}
%          \edef##1{\expandafter\remove@nil##1%
%                   \reserved@c
%                   #1%
%                   \reserved@e
%                   \noexpand\@nil}}%
%    \end{macrocode}
%    Finally we call |\alpha@list| which will now execute
%    the macro |\group@elt| once for every defined \meta{math
%    alphabet identifier}.  And that's all for now.
%    \begin{macrocode}
%     \alpha@list
%  \fi}
%    \end{macrocode}
% \end{macro}
%
%
% \begin{macro}{\alpha@list}
%    As we explained above every entry in |\alpha@list| has
%    the form\\[2pt]
%    |\alpha@elt|\\\meta{alphabet identifier}\meta{internal
%    group number}\meta{default font assignments}\ldots\\[2pt]
%    We initialize it to |\@empty|.
%    \begin{macrocode}
\let\alpha@list\@empty
\@onlypreamble\alpha@list
%    \end{macrocode}
% \end{macro}
%
%
% \begin{macro}{\alpha@elt}
%    \begin{macrocode}
\let\alpha@elt\relax
\@onlypreamble\alpha@elt
%    \end{macrocode}
% \end{macro}
%
%
%
%  \begin{macro}{\newgroup}
% Start the group (fam) allocation at 0. (Doesn't belong here.)
%    \begin{macrocode}
\count18=-1
%    \end{macrocode}
% \end{macro}
%
% \begin{macro}{\stepcounter}
% \changes{v3.0f}{1997/11/13}
%      {(DPC) Remove as never used. (Re)defined in ltcounts}
% \end{macro}
%
% \begin{macro}{\select@group}
%    We surround |\select@group| with braces so that functions using it
%    can be used directly after |_| or |^|.
% \changes{v2.1p}{1994/12/10}{Surround with braces (add fourth arg)}
%    However, if we use oldstyle syntax where the math alphabet
%    doesn't have arguments (ie if |\math@bgroup| is not |\bgroup|) we
%    need to get rid of the extra group.
% \changes{v2.1q}{1995/04/02}{fix problem for pr/1275}
% \changes{v3.0g}{1997/11/20}
%      {(DPC) inline use of \cs{stepcounter} (faster, and saves a csname
%       per math version as no reset list)}
% \changes{v3.0q}{2015/03/18}{Introduce \cs{e@mathgroup@top}}
%    \begin{macrocode}
%</2ekernel>
%<latexrelease>\IncludeInRelease{2015/01/01}
%<latexrelease>                 {\select@group}{\select@group}%
%<*2ekernel|latexrelease>
\def\select@group#1#2#3#4{%
 \ifx\math@bgroup\bgroup\else\relax\expandafter\@firstofone\fi
 {%
 \ifmmode
  \ifnum\csname c@mv@\math@version\endcsname<\e@mathgroup@top
     \begingroup
       \escapechar\m@ne
       \getanddefine@fonts{\csname c@mv@\math@version\endcsname}#3%
       \globaldefs\@ne  \math@fonts
     \endgroup
     \init@restore@version
     \xdef#1{\noexpand\use@mathgroup\noexpand#2%
             {\number\csname c@mv@\math@version\endcsname}}%
     \global\advance\csname c@mv@\math@version\endcsname\@ne
   \else
     \let#1\relax
     \@latex@error{Too many math alphabets used in
                   version \math@version}%
        \@eha
   \fi
 \else \expandafter\non@alpherr\fi
 #1{#4}%
 }%
}
%</2ekernel|latexrelease>
%<latexrelease>\EndIncludeInRelease
%<latexrelease>\IncludeInRelease{0000/00/00}
%<latexrelease>                 {\select@group}{\select@group}%
%<latexrelease>\def\select@group#1#2#3#4{%
%<latexrelease> \ifx\math@bgroup\bgroup\else\relax\expandafter\@firstofone\fi
%<latexrelease> {%
%<latexrelease> \ifmmode
%<latexrelease>  \ifnum\csname c@mv@\math@version\endcsname<\sixt@@n
%<latexrelease>     \begingroup
%<latexrelease>       \escapechar\m@ne
%<latexrelease>       \getanddefine@fonts
%<latexrelease>         {\csname c@mv@\math@version\endcsname}#3%
%<latexrelease>       \globaldefs\@ne  \math@fonts
%<latexrelease>     \endgroup
%<latexrelease>     \init@restore@version
%<latexrelease>     \xdef#1{\noexpand\use@mathgroup\noexpand#2%
%<latexrelease>             {\number\csname c@mv@\math@version\endcsname}}%
%<latexrelease>     \global\advance\csname c@mv@\math@version\endcsname\@ne
%<latexrelease>   \else
%<latexrelease>     \let#1\relax
%<latexrelease>     \@latex@error{Too many math alphabets used in
%<latexrelease>                   version \math@version}%
%<latexrelease>        \@eha
%<latexrelease>   \fi
%<latexrelease> \else \expandafter\non@alpherr\fi
%<latexrelease> #1{#4}%
%<latexrelease> }%
%<latexrelease>}
%<latexrelease>\EndIncludeInRelease
%<*2ekernel>
%    \end{macrocode}
%
%    \begin{macrocode}
\@onlypreamble\restore@mathversion
%    \end{macrocode}
% \end{macro}
%
% \begin{macro}{\init@restore@version}
% \changes{v3.0e}{1996/07/26}{Removed \cs{ifrestore@version} switch
%          and replaced by \cs{init@restore@version}}
%    \begin{macrocode}
\def\init@restore@version{%
        \global\let\init@restore@version\relax
        \xdef\restore@mathversion
             {\expandafter\noexpand\csname mv@\math@version\endcsname
              \global\csname c@mv@\math@version\endcsname
              \number\csname c@mv@\math@version\endcsname\relax}%
        \aftergroup\dorestore@version
}
\@onlypreamble\init@restore@version
%    \end{macrocode}
% \end{macro}
%
% \begin{macro}{\non@alpherr}
% \changes{v3.0c}{1995/10/10}
%         {(DPC) autoload error message}
% \changes{v3.0j}{2005/07/27}
%         {(MH) Change because command is now properly robust}
%    \begin{macrocode}
\gdef\non@alpherr#1{\@latex@error{%
%    \end{macrocode}
% The command here will have a space at the end of its name, so we make
% sure not to insert an extra one.
%    \begin{macrocode}
    \string#1allowed only in math mode}\@ehd}
%    \end{macrocode}
% \end{macro}
%
% \begin{macro}{\dorestore@version}
%    \begin{macrocode}
\def\dorestore@version
 {\ifmmode
    \aftergroup\dorestore@version
  \else
    \gdef\init@restore@version{%
        \global\let\init@restore@version\relax
        \xdef\restore@mathversion
             {\expandafter\noexpand\csname mv@\math@version\endcsname
              \global\csname c@mv@\math@version\endcsname
              \number\csname c@mv@\math@version\endcsname\relax}%
        \aftergroup\dorestore@version
    }%
    \begingroup
      \let\getanddefine@fonts\@gobbletwo
      \restore@mathversion
    \endgroup
  \fi}%
\@onlypreamble\dorestore@version
%    \end{macrocode}
% \end{macro}
%
% \begin{macro}{\document@select@group}
%    We surround |\select@group| with braces so that functions using it
%    can be used directly after |_| or |^|.
% \changes{v2.1p}{1994/12/10}{Surround with braces (add fourth arg)}
% \changes{v2.1q}{1995/04/02}{fix problem for pr/1275}
% \changes{v3.0g}{1997/11/20}
%      {(DPC) inline use of \cs{stepcounter} (faster, and saves a csname
%       per math version as no reset list)}
% \changes{v3.0q}{2015/03/18}{Introduce \cs{e@mathgroup@top}}
%    \begin{macrocode}
%</2ekernel>
%<latexrelease>\IncludeInRelease{2020/10/01}
%<latexrelease>  {\document@select@group}{\document@select@group}%
%<*2ekernel|latexrelease>
\def\document@select@group#1#2#3#4{%
 \ifx\math@bgroup\bgroup\else\relax\expandafter\@firstofone\fi
 {%
 \ifmmode
   \ifnum\csname c@mv@\math@version\endcsname<\e@mathgroup@top
     \begingroup
       \escapechar\m@ne
       \getanddefine@fonts{\csname c@mv@\math@version\endcsname}#3%
       \globaldefs\@ne  \math@fonts
     \endgroup
     \expandafter\extract@alph@from@version
         \csname mv@\math@version\expandafter\endcsname
         \expandafter{\number\csname
                       c@mv@\math@version\endcsname}%
          #1%
     \global\advance\csname c@mv@\math@version\endcsname\@ne
   \else
     \let#1\relax
     \@latex@error{Too many math alphabets used
                   in version \math@version}%
        \@eha
  \fi
 \else \expandafter\non@alpherr\fi
%    \end{macrocode}
%    If the legacy interface is used, e.g., \verb=$\sf -1$= the math
%    alphabet \verb=#1= does not take an argument so we better do not
%    surround \verb=#4= with braces, because then we get
%    \verb={\relax}= into the formula and introduce an extra Ord
%    atom. The two different cases can be distinguished by looking at
%    the current value of \cs{math@bgroup}.
% \changes{v3.0u}{2020/03/19}{fix for (gnats/3357)}
%    \begin{macrocode}
 \expandafter#1\ifx\math@bgroup\bgroup{#4}\else#4\fi
 }%
}
%</2ekernel|latexrelease>
%<latexrelease>\EndIncludeInRelease
%<latexrelease>\IncludeInRelease{2015/01/01}
%<latexrelease>  {\document@select@group}{\document@select@group}%
%<latexrelease>
%<latexrelease>\def\document@select@group#1#2#3#4{%
%<latexrelease> \ifx\math@bgroup\bgroup\else\relax\expandafter\@firstofone\fi
%<latexrelease> {%
%<latexrelease> \ifmmode
%<latexrelease>   \ifnum\csname c@mv@\math@version\endcsname<\e@mathgroup@top
%<latexrelease>     \begingroup
%<latexrelease>       \escapechar\m@ne
%<latexrelease>       \getanddefine@fonts{\csname c@mv@\math@version\endcsname}#3%
%<latexrelease>       \globaldefs\@ne  \math@fonts
%<latexrelease>     \endgroup
%<latexrelease>     \expandafter\extract@alph@from@version
%<latexrelease>         \csname mv@\math@version\expandafter\endcsname
%<latexrelease>         \expandafter{\number\csname
%<latexrelease>                       c@mv@\math@version\endcsname}%
%<latexrelease>          #1%
%<latexrelease>     \global\advance\csname c@mv@\math@version\endcsname\@ne
%<latexrelease>   \else
%<latexrelease>     \let#1\relax
%<latexrelease>     \@latex@error{Too many math alphabets used
%<latexrelease>                   in version \math@version}%
%<latexrelease>        \@eha
%<latexrelease>  \fi
%<latexrelease> \else \expandafter\non@alpherr\fi
%<latexrelease> #1{#4}%
%<latexrelease> }%
%<latexrelease>}
%<latexrelease>\EndIncludeInRelease
%<latexrelease>
%<latexrelease>\IncludeInRelease{0000/00/00}
%<latexrelease>  {\document@select@group}{\document@select@group}%
%<latexrelease>
%<latexrelease>\def\document@select@group#1#2#3#4{%
%<latexrelease> \ifx\math@bgroup\bgroup\else\relax\expandafter\@firstofone\fi
%<latexrelease> {%
%<latexrelease> \ifmmode
%<latexrelease>   \ifnum\csname c@mv@\math@version\endcsname<\sixt@@n
%<latexrelease>     \begingroup
%<latexrelease>       \escapechar\m@ne
%<latexrelease>       \getanddefine@fonts
%<latexrelease>         {\csname c@mv@\math@version\endcsname}#3%
%<latexrelease>       \globaldefs\@ne  \math@fonts
%<latexrelease>     \endgroup
%<latexrelease>     \expandafter\extract@alph@from@version
%<latexrelease>         \csname mv@\math@version\expandafter\endcsname
%<latexrelease>         \expandafter{\number\csname
%<latexrelease>                       c@mv@\math@version\endcsname}%
%<latexrelease>          #1%
%<latexrelease>     \global\advance\csname c@mv@\math@version\endcsname\@ne
%<latexrelease>   \else
%<latexrelease>     \let#1\relax
%<latexrelease>     \@latex@error{Too many math alphabets used
%<latexrelease>                   in version \math@version}%
%<latexrelease>        \@eha
%<latexrelease>  \fi
%<latexrelease> \else \expandafter\non@alpherr\fi
%<latexrelease> #1{#4}%
%<latexrelease> }%
%<latexrelease>}
%<latexrelease>\EndIncludeInRelease
%<*2ekernel>
%    \end{macrocode}
% \end{macro}
%
% \begin{macro}{\process@table}
%    \begin{macrocode}
\def\process@table{%
    \def\cdp@elt##1##2##3##4{%
        \@font@info{Checking defaults for
                  ##1/##2/##3/##4}%
        \expandafter
        \ifx\csname##1/##2/##3/##4\endcsname\relax
%    \end{macrocode}
%    Grouping is important for two reasons, first |\cdp@elt| will get
%    redefined if |\Declare...| functions are executed within the
%    external |.fd| file and secondly |\try@load@fontshape| changes a
%    lot of catcodes without surrounding itself with a group.
%    \begin{macrocode}
          \begingroup
           \def\f@encoding{##1}\def\f@family{##2}%
           \try@load@fontshape
          \endgroup
        \fi
        \expandafter
        \ifx\csname##1/##2/##3/##4\endcsname\relax
             \@latex@error{This NFSS system isn't set up properly}%
                       {For encoding scheme ##1 the defaults
                        ##2/##3/##4 do not form a valid font shape}%
        \else
             \@font@info{... okay}%
        \fi}%
    \cdp@list
%    \end{macrocode}
%    Now we make sure that |\error@fontshape| is okay.
%    \begin{macrocode}
    \begingroup
       \escapechar\m@ne
       \error@fontshape
       \expandafter\ifx\csname \curr@fontshape\endcsname\relax
          \begingroup
            \try@load@fontshape
           \endgroup
       \fi
       \expandafter\ifx\csname \curr@fontshape\endcsname\relax
         \@latex@error{This NFSS system isn't set up properly}%
            {The system maintainer forgot to specify a suitable
             substitution
             font shape using the \noexpand\DeclareErrorFont
             command}%
       \fi
    \endgroup
%    \end{macrocode}
%    Set |\select@group| to its meaning used within the document body.
%    \begin{macrocode}
    \let\select@group\document@select@group
%    \end{macrocode}
%    Install the default font attributes as they are currently pointing
%    to error font face. We can speed up the process by just using
%    \cs{edef}, thereby avoiding all kind of extra processing.
%    Don't use |\reset@font| since that would trigger |\selectfont|.
% \changes{v3.0v}{2020/04/13}{Small update for speed.}
%    \begin{macrocode}
    \fontencoding\encodingdefault
    \edef\f@family{\familydefault}%
    \edef\f@series{\seriesdefault}%
    \edef\f@shape{\shapedefault}%
%    \end{macrocode}
%    Drop stuff not longer needed.
%    We need to add many more!!!!!!
%    \begin{macrocode}
 \everyjob{}%
}
\@onlypreamble\process@table
%    \end{macrocode}
% \end{macro}
%
%    \begin{macrocode}
%\@onlypreamble\set@mathradical
%    \end{macrocode}
%
% \begin{macro}{\DeclareMathVersion}
%    \begin{macrocode}
\def\DeclareMathVersion#1{%
  \expandafter\new@mathversion\csname mv@#1\endcsname}
\@onlypreamble\DeclareMathVersion
%    \end{macrocode}
% \end{macro}
%
% \begin{macro}{\new@mathversion}
% \changes{v3.0o}{2011/09/03}
%      {(Will) Remove \cs{global} before \cs{newcount} (unnecessary and caused etex bug).}
%    \begin{macrocode}
\def\new@mathversion#1{%
  \expandafter\in@\expandafter#1\expandafter{\version@list}%
  \ifin@
    \@font@info{Redeclaring math version
               `\expandafter\@gobblefour\string#1'}%
  \else
    \expandafter\newcount\csname c@\expandafter
                                \@gobble\string#1\endcsname
    \def\version@elt{\noexpand\version@elt\noexpand}%
    \edef\version@list{\version@list\version@elt#1}%
  \fi
%    \end{macrocode}
%    |\toks@| is used to gather all tokens for the math version.
%    |\count@| will be used to count the math groups we add to this
%    version.
%    \begin{macrocode}
  \toks@{}%
  \count@\z@
%    \end{macrocode}
%    Now we loop over |\group@list| to add all math groups defined so
%    far to the version and at the same time to count them.
%    \begin{macrocode}
  \def\group@elt##1##2{%
       \advance\count@\@ne
       \addto@hook\toks@{\getanddefine@fonts##1##2}%
       }%
  \group@list
%    \end{macrocode}
%    We set the counter for this math version to the number of math
%    groups found in |\group@list|.
%    \begin{macrocode}
  \global\csname c@\expandafter\@gobble\string#1\endcsname\count@
%    \end{macrocode}
%    Now we loop over |\alpha@list| to add all math alphabets known so
%    far. We have to distinguish the case that an alphabet by default
%    should produce an error in new versions.
%    \begin{macrocode}
  \def\alpha@elt##1##2##3{%
       \ifx##2\no@alphabet@error
         \toks@\expandafter{\the\toks@\install@mathalphabet##1%
             {\no@alphabet@error##1}}%
       \else
         \toks@\expandafter{\the\toks@\install@mathalphabet##1%
             {\select@group##1##2##3}}%
       \fi
          }%
  \alpha@list
%    \end{macrocode}
%    Finally we define the math version to expand to the contents of
%    |\toks@|.
% \changes{v2.0e}{1993/08/18}{Exchanged names of encodings in warning
%              message of \cs{SetSymbolFont}.}
%    \begin{macrocode}
  \xdef#1{\the\toks@}%
}
\@onlypreamble\new@mathversion
%    \end{macrocode}
% \end{macro}
%
% \begin{macro}{\DeclareSymbolFont}
%    \begin{macrocode}
\def\DeclareSymbolFont#1#2#3#4#5{%
 \@tempswafalse
 \edef\reserved@b{#2}%
 \def\cdp@elt##1##2##3##4{\def\reserved@c{##1}%
      \ifx\reserved@b\reserved@c \@tempswatrue\fi}%
 \cdp@list
 \if@tempswa
   \@ifundefined{sym#1}{%
%    \end{macrocode}
%
% \changes{v3.0q}{2015/03/18}{Restrict Symbol fonts to 0-15}
%    \begin{macrocode}
      \ifnum\count18<15 %
        \expandafter\new@mathgroup\csname sym#1\endcsname
        \expandafter\new@symbolfont\csname sym#1\endcsname
                        {#2}{#3}{#4}{#5}%
      \else
         \@latex@error{Too many symbol fonts declared}\@eha
      \fi
     }%
     {%
%    \end{macrocode}
%
%    \begin{macrocode}
      \@font@info{Redeclaring symbol font `#1'}%
%    \end{macrocode}
% \changes{v3.0f}{1997/11/13}
%      {(DPC) Really update \cs{group@list} don't
%         leave new version in \cs{toks@}. latex/2661}
% Update the group list.
%    \begin{macrocode}
      \def\group@elt##1##2{%
           \noexpand\group@elt\noexpand##1%
           \expandafter\ifx\csname sym#1\endcsname##1%
             \expandafter\noexpand\csname#2/#3/#4/#5\endcsname
           \else
               \noexpand##2%
           \fi}%
      \xdef\group@list{\group@list}%
%    \end{macrocode}
% Update the version list.
%    \begin{macrocode}
      \def\version@elt##1{%
          \expandafter
          \SetSymbolFont@\expandafter##1\csname#2/#3/#4/#5\expandafter
              \endcsname \csname sym#1\endcsname
          }%
      \version@list
     }%
  \else
    \@latex@error{Encoding scheme  `#2' unknown}\@eha
  \fi
  }
\@onlypreamble\DeclareSymbolFont
%    \end{macrocode}
% \end{macro}
%
% \begin{macro}{\group@list}
%    \begin{macrocode}
\let\group@list\@empty
\@onlypreamble\group@list
%    \end{macrocode}
% \end{macro}
%
% \begin{macro}{\group@elt}
%    \begin{macrocode}
\let\group@elt\relax
\@onlypreamble\group@elt
%    \end{macrocode}
% \end{macro}
%
% \begin{macro}{\new@symbolfont}
%    \begin{macrocode}
\def\new@symbolfont#1#2#3#4#5{%
    \toks@\expandafter{\group@list}%
    \edef\group@list{\the\toks@\noexpand\group@elt\noexpand#1%
                     \expandafter\noexpand\csname#2/#3/#4/#5\endcsname}%
    \def\version@elt##1{\toks@\expandafter{##1}%
                   \edef##1{\the\toks@\noexpand\getanddefine@fonts
                   #1\expandafter\noexpand\csname#2/#3/#4/#5\endcsname}%
                  \global\advance\csname c@\expandafter
                                 \@gobble\string##1\endcsname\@ne
                 }%
    \version@list
}
\@onlypreamble\new@symbolfont
%    \end{macrocode}
% \end{macro}
%
% \begin{macro}{\SetSymbolFont}
%    \begin{macrocode}
\def\SetSymbolFont#1#2#3#4#5#6{%
 \@tempswafalse
 \edef\reserved@b{#3}%
 \def\cdp@elt##1##2##3##4{\def\reserved@c{##1}%
      \ifx\reserved@b\reserved@c \@tempswatrue\fi}%
 \cdp@list
 \if@tempswa
  \expandafter\SetSymbolFont@
    \csname mv@#2\expandafter\endcsname\csname#3/#4/#5/#6\expandafter
    \endcsname \csname sym#1\endcsname
 \else
  \@latex@error{Encoding scheme  `#3' unknown}\@eha
 \fi
}
\@onlypreamble\SetSymbolFont
%    \end{macrocode}
% \end{macro}
%
% \begin{macro}{\SetSymbolFont@}
% \changes{v3.0l}{2007/08/31}{Font warning changed to info
%    for encoding change (pr/3975)}
%    \begin{macrocode}
\def\SetSymbolFont@#1#2#3{%
  \expandafter\in@\expandafter#1\expandafter{\version@list}%
  \ifin@
    \expandafter\in@\expandafter#3\expandafter{\group@list}%
    \ifin@
      \begingroup
        \expandafter\get@cdp\string#2\@nil\reserved@a
        \toks@{}%
        \def\install@mathalphabet##1##2{%
             \addto@hook\toks@{\install@mathalphabet##1{##2}}%
            }%
        \def\getanddefine@fonts##1##2{%
          \ifnum##1=#3%
             \addto@hook\toks@{\getanddefine@fonts#3#2}%
             \expandafter\get@cdp\string##2\@nil\reserved@b
             \ifx\reserved@a\reserved@b\else
                \@font@info{Encoding `\reserved@b' has changed
                    to `\reserved@a' for symbol font\MessageBreak
                   `\expandafter\@gobblefour\string#3' in the
                    math version `\expandafter
                    \@gobblefour\string#1'}%
             \fi
             \@font@info{%
                Overwriting symbol font
                `\expandafter\@gobblefour\string#3' in
                 version `\expandafter
                \@gobblefour\string#1'\MessageBreak
                \@spaces \expandafter\@gobble\string##2 -->
                         \expandafter\@gobble\string#2}%
          \else
             \addto@hook\toks@{\getanddefine@fonts##1##2}%
          \fi}%
         #1%
         \xdef#1{\the\toks@}%
      \endgroup
    \else
       \@latex@error{Symbol font `\expandafter\@gobblefour\string#3'
                  not defined}\@eha
    \fi
  \else
    \@latex@error{Math version `\expandafter\@gobblefour\string#1'
       is not
       defined}{You probably misspelled the name of the math
       version.^^JOr you have to specify an additional package.}%
  \fi
}
\@onlypreamble\SetSymbolFont@
%    \end{macrocode}
% \end{macro}
%
% \begin{macro}{\get@cdp}
%    \begin{macrocode}
\def\get@cdp#1#2/#3\@nil#4{\def#4{#2}}
\@onlypreamble\get@cdp
%    \end{macrocode}
% \end{macro}
%
% \begin{macro}{\DeclareMathAlphabet}
% \changes{v3.0j}{2005/07/27}
%         {(MH) Make document commands robust}
%    \begin{macrocode}
\def\DeclareMathAlphabet#1#2#3#4#5{%
 \@tempswafalse
 \edef\reserved@b{#2}%
 \def\cdp@elt##1##2##3##4{\def\reserved@c{##1}%
      \ifx\reserved@b\reserved@c \@tempswatrue\fi}%
 \cdp@list
 \if@tempswa
   \expandafter\ifx
   \csname\expandafter\@gobble\string#1\endcsname
   \relax
      \new@mathalphabet#1{#2}{#3}{#4}{#5}%
   \else
%    \end{macrocode}
%    Check if it is already a math alphabet.
%    \begin{macrocode}
     \edef\reserved@a{\noexpand\in@{\string\select@group}%
         {\expandafter\meaning\csname \expandafter
          \@gobble\string#1\space\endcsname}}%
     \reserved@a
     \ifin@
       \@font@info{Redeclaring math alphabet \string#1}%
       \def\version@elt##1{%
         \expandafter\SetMathAlphabet@\expandafter
            ##1\csname#2/#3/#4/#5\expandafter\endcsname
%    \end{macrocode}
% \changes{???}{1994/04/18}{Pass correct arg (2 not 3)}
%    \begin{macrocode}
            \csname M@#2\expandafter\endcsname
            \csname \expandafter\@gobble\string#1\space\endcsname#1}%
       \version@list
     \else
%    \end{macrocode}
%    Check if it is a math alphabet defined via
%    |\DeclareSymbolFontAlphabet|.
%    \begin{macrocode}
       \edef\reserved@a{\noexpand\in@{\string\use@mathgroup}%
         {\expandafter\meaning\csname \expandafter
          \@gobble\string#1\space\endcsname}}%
       \reserved@a
       \ifin@
%    \end{macrocode}
%    In that case overwriting is simple since there is nothing
%    inserted in the math version macros.
%    \begin{macrocode}
         \@font@info{Redeclaring math alphabet \string#1}%
         \new@mathalphabet#1{#2}{#3}{#4}{#5}%
%    \end{macrocode}
%    Otherwise panic.
%    \begin{macrocode}
       \else
         \@latex@error{Command `\string#1' already defined}\@eha
       \fi
     \fi
  \fi
 \else
  \@latex@error{Encoding scheme  `#2' unknown}\@eha
 \fi
  }
\@onlypreamble\DeclareMathAlphabet
%    \end{macrocode}
% \end{macro}
%
% \begin{macro}{\new@mathalphabet}
% \changes{v3.0j}{2005/07/27}
%         {(MH) Make document commands robust}
%    \begin{macrocode}
\def\new@mathalphabet#1#2#3#4#5{%
    \toks@\expandafter{\alpha@list}%
    \edef#1{\expandafter\noexpand\csname \expandafter
            \@gobble\string#1\space\endcsname
            \if/#5/%
               \noexpand\no@alphabet@error
               \noexpand\no@alphabet@error
            \else
               \expandafter\noexpand\csname M@#2\endcsname
               \expandafter\noexpand\csname#2/#3/#4/#5\endcsname
            \fi
           }%
    \toks2\expandafter{#1}%
    \edef\alpha@list{\the\toks@\noexpand\alpha@elt\the\toks2}%
    \def\version@elt##1{\toks@\expandafter{##1}%
                   \edef##1{\the\toks@\install@mathalphabet
                            \expandafter\noexpand
                            \csname \expandafter\@gobble
                               \string#1\space\endcsname
                           {\if/#5/%
                             \noexpand\no@alphabet@error
                             \noexpand#1%
                            \else
                             \noexpand\select@group\the\toks2
                            \fi}}%
                 }%
    \version@list
    \expandafter\edef\csname \expandafter\@gobble
                \string#1\space\endcsname{\if/#5/%
              \noexpand\no@alphabet@error
              \noexpand#1%
            \else
              \noexpand\select@group\the\toks2
            \fi}%
    \edef#1{\noexpand\protect
            \expandafter\noexpand\csname \expandafter
            \@gobble\string#1\space\endcsname}%
}
\@onlypreamble\new@mathalphabet
%    \end{macrocode}
% \end{macro}
%
% \begin{macro}{\SetMathAlphabet}
% \changes{v3.0j}{2005/07/27}
%         {(MH) Make document commands robust}
%    \begin{macrocode}
\def\SetMathAlphabet#1#2#3#4#5#6{%
 \@tempswafalse
 \edef\reserved@b{#3}%
 \def\cdp@elt##1##2##3##4{\def\reserved@c{##1}%
      \ifx\reserved@b\reserved@c \@tempswatrue\fi}%
 \cdp@list
 \if@tempswa
  \expandafter\SetMathAlphabet@
    \csname mv@#2\expandafter\endcsname\csname#3/#4/#5/#6\expandafter
    \endcsname \csname M@#3\expandafter\endcsname
    \csname \expandafter\@gobble\string#1\space\endcsname#1%
 \else
  \@latex@error{Encoding scheme  `#3' unknown}\@eha
 \fi
}
\@onlypreamble\SetMathAlphabet
%    \end{macrocode}
% \end{macro}
%
% \begin{macro}{\SetMathAlphabet@}
%    \begin{macrocode}
\def\SetMathAlphabet@#1#2#3#4#5{%
  \expandafter\in@\expandafter#1\expandafter{\version@list}%
  \ifin@
    \expandafter\in@\expandafter#4\expandafter{\alpha@list}%
    \ifin@
      \begingroup
        \toks@{}%
        \def\getanddefine@fonts##1##2{%
             \addto@hook\toks@{\getanddefine@fonts##1##2}%
            }%
        \def\reserved@c##1##2##3##4{%                % for message below
            \expandafter\@gobble\string##4}%
        \def\install@mathalphabet##1##2{%
          \ifx##1#4%
             \addto@hook\toks@
                {\install@mathalphabet#4{\select@group#4#3#2}}%
             \@font@info{Overwriting math alphabet
                `\string#5' in version `\expandafter
                 \@gobblefour\string#1'\MessageBreak
                 \@spaces \reserved@c##2 -->
                        \expandafter\@gobble\string#2}%
          \else
             \addto@hook\toks@{\install@mathalphabet##1{##2}}%
          \fi
          }%
        #1%
        \xdef#1{\the\toks@}%
      \endgroup
    \else
%    \end{macrocode}
%    If the math alphabet was defined via |\DeclareSymbolFontAlphabet|
%    we have remove its external definition and add it as a normal
%    math alphabet to every version before trying to change it in one
%    version.
% \changes{v2.1b}{1994/03/11}{Changed parameter template in temporary
%          macro to catch check add below.}
%    \begin{macrocode}
       \edef\reserved@a{%
         \noexpand\in@{\string\use@mathgroup}{\meaning#4}}%
       \reserved@a
       \ifin@
         \def\reserved@b##1\use@mathgroup##2##3{%
             \def\reserved@b{##3}\def\reserved@c{##2}}%
         \expandafter\reserved@b#4%
         \begingroup
           \def\install@mathalphabet##1##2{%
               \addto@hook\toks@{\install@mathalphabet##1{##2}}%
               }%
            \def\getanddefine@fonts##1##2{%
              \addto@hook\toks@{\getanddefine@fonts##1##2}%
              \ifnum##1=\reserved@b
                 \expandafter
                 \addto@hook\expandafter\toks@
                 \expandafter{\expandafter\install@mathalphabet
                 \expandafter#4\expandafter
                       {\expandafter\select@group\expandafter
                         #4\reserved@c##2}}%
              \fi
                      }%
           \def\version@elt##1{%
               \toks@{}%
               ##1%
               \xdef##1{\the\toks@}%
              }%
           \version@list
         \endgroup
%    \end{macrocode}
%    Put it into the |\alpha@list| with default `error'
%    \begin{macrocode}
         \expandafter\gdef\expandafter\alpha@list\expandafter
             {\alpha@list
              \alpha@elt #4\no@alphabet@error \no@alphabet@error}%
         \gdef#4{\no@alphabet@error #5}% fake things :-)
%    \end{macrocode}
%    Then call the internal setting routine again:
%    \begin{macrocode}
         \SetMathAlphabet@{#1}{#2}{#3}#4#5%
       \else
         \@latex@error{Command `\string#5' not defined as a
                       math alphabet}%
            {Use \noexpand\DeclareMathAlphabet to define it.}%
       \fi
    \fi
  \else
    \@latex@error{Math version `\expandafter\@gobblefour\string#1'
       is not
       defined}{You probably misspelled the name of the math
       version.^^JOr you have to specify an additional package.}%
  \fi
}
\@onlypreamble\SetMathAlphabet@
%    \end{macrocode}
% \end{macro}
%
% \begin{macro}{\DeclareMathAccent}
%    Could do with more checks like allowing single number in |#4|
%    lowercase in |#4| etc
% \changes{v3.0r}{2016/02/18}
%         {Check for mathaccent not \cs{mathaccemt}}
% \changes{v3.0s}{2019/08/27}{Make math accents robust}
%    \begin{macrocode}
%</2ekernel>
%<*2ekernel|latexrelease>
%<latexrelease>\IncludeInRelease{2019/10/01}%
%<latexrelease>                 {DeclareMathAccent}{Make math accents robust}%
\def\DeclareMathAccent#1#2#3#4{%
  \expandafter\in@\csname sym#3\expandafter\endcsname
     \expandafter{\group@list}%
  \ifin@
    \begingroup
      \count\z@=#4\relax
      \count\tw@\count\z@
      \divide\count\z@\sixt@@n
      \count@\count\z@
      \multiply\count@\sixt@@n
      \advance\count\tw@-\count@
      \if\relax\noexpand#1% is command?
        \edef\reserved@a{\noexpand\in@
          {\expandafter\@gobble\string\mathaccent}
          {\expandafter\meaning
           \csname\expandafter\@gobble\string#1\space\endcsname}}%
        \reserved@a
        \ifin@
          \expandafter\let
            \csname\expandafter\@gobble\string#1\space\endcsname
            \@undefined
          \expandafter\set@mathaccent
             \csname sym#3\endcsname#1#2%
             {\hexnumber@{\count\z@}\hexnumber@{\count\tw@}}%
          \@font@info{Redeclaring math accent \string#1}%
        \else
          \expandafter\ifx
          \csname\expandafter\@gobble\string#1\endcsname
          \relax
            \expandafter\set@mathaccent
               \csname sym#3\endcsname#1#2%
               {\hexnumber@{\count\z@}\hexnumber@{\count\tw@}}%
          \else
            \@latex@error{Command `\string#1' already defined}\@eha
          \fi
        \fi
      \else
       \@latex@error{Not a command name: `\noexpand#1'}\@eha
      \fi
    \endgroup
  \else
    \@latex@error{Symbol font `#3' is not defined}\@eha
  \fi
}
%</2ekernel|latexrelease>
%<latexrelease>\EndIncludeInRelease
%<latexrelease>\IncludeInRelease{0000/00/00}%
%<latexrelease>                 {DeclareMathAccent}{Make math accents robust}%
%<latexrelease>\def\DeclareMathAccent#1#2#3#4{%
%<latexrelease>  \expandafter\in@\csname sym#3\expandafter\endcsname
%<latexrelease>     \expandafter{\group@list}%
%<latexrelease>  \ifin@
%<latexrelease>    \begingroup
%<latexrelease>      \count\z@=#4\relax
%<latexrelease>      \count\tw@\count\z@
%<latexrelease>      \divide\count\z@\sixt@@n
%<latexrelease>      \count@\count\z@
%<latexrelease>      \multiply\count@\sixt@@n
%<latexrelease>      \advance\count\tw@-\count@
%<latexrelease>      \if\relax\noexpand#1% is command?
%<latexrelease>        \edef\reserved@a{\noexpand\in@
%<latexrelease>           {\expandafter\@gobble\string\mathaccent}{\meaning#1}}%
%<latexrelease>        \reserved@a
%<latexrelease>        \ifin@
%<latexrelease>          \expandafter\set@mathaccent
%<latexrelease>             \csname sym#3\endcsname#1#2%
%<latexrelease>             {\hexnumber@{\count\z@}\hexnumber@{\count\tw@}}%
%<latexrelease>          \@font@info{Redeclaring math accent \string#1}%
%<latexrelease>        \else
%<latexrelease>          \expandafter\ifx
%<latexrelease>          \csname\expandafter\@gobble\string#1\endcsname
%<latexrelease>          \relax
%<latexrelease>            \expandafter\set@mathaccent
%<latexrelease>               \csname sym#3\endcsname#1#2%
%<latexrelease>               {\hexnumber@{\count\z@}\hexnumber@{\count\tw@}}%
%<latexrelease>          \else
%<latexrelease>            \@latex@error{Command `\string#1' already defined}\@eha
%<latexrelease>          \fi
%<latexrelease>        \fi
%<latexrelease>      \else
%<latexrelease>       \@latex@error{Not a command name: `\noexpand#1'}\@eha
%<latexrelease>      \fi
%<latexrelease>    \endgroup
%<latexrelease>  \else
%<latexrelease>    \@latex@error{Symbol font `#3' is not defined}\@eha
%<latexrelease>  \fi
%<latexrelease>}
%<latexrelease>\EndIncludeInRelease
%<*2ekernel>
%    \end{macrocode}
%
%    \begin{macrocode}
\@onlypreamble\DeclareMathAccent
%    \end{macrocode}
% \end{macro}
%
% \begin{macro}{\set@mathaccent}
%    \begin{macrocode}
%</2ekernel>
%<*2ekernel|latexrelease>
%<latexrelease>\IncludeInRelease{2019/10/01}%
%<latexrelease>                 {\set@mathaccent}{makemath accents robust}%
\def\set@mathaccent#1#2#3#4{%
  \xdef#2{\mathaccent"\mathchar@type#3\hexnumber@#1#4\relax}%
  \MakeRobust#2%
}
\@onlypreamble\set@mathaccent
%</2ekernel|latexrelease>
%<latexrelease>\EndIncludeInRelease
%<latexrelease>\IncludeInRelease{0000/00/00}%
%<latexrelease>                 {\set@mathaccent}{makemath accents robust}%
%<latexrelease>
%<latexrelease>\def\set@mathaccent#1#2#3#4{%
%<latexrelease>  \xdef#2{\mathaccent"\mathchar@type#3\hexnumber@#1#4\relax}}
%<latexrelease>
%<latexrelease>\EndIncludeInRelease
%<*2ekernel>
%    \end{macrocode}
% \end{macro}
%
% \begin{macro}{\DeclareMathSymbol}
% \changes{v3.0r}{2016/02/18}
%         {Check for mathchar not \cs{mathchar}}
% \changes{v3.0s}{2019/09/09}{Allow definition if the math symbol was
%                             a command already robust}
%    \begin{macrocode}
\def\DeclareMathSymbol#1#2#3#4{%
  \expandafter\in@\csname sym#3\expandafter\endcsname
     \expandafter{\group@list}%
  \ifin@
    \begingroup
      \count\z@=#4\relax
      \count\tw@\count\z@
      \divide\count\z@\sixt@@n
      \count@\count\z@
      \multiply\count@\sixt@@n
      \advance\count\tw@-\count@
      \if\relax\noexpand#1% is command?
%    \end{macrocode}
%    Store the command name with a space attached inside
%    \cs{reserved@@b} in case we look at a robust definition.
%    \begin{macrocode}
        \edef\reserved@b{\expandafter\noexpand
                         \csname\expandafter\@gobble\string#1\space\endcsname}%
%    \end{macrocode}
%    Test both \verb=#1= and \verb*=#1 = for containing \texttt{mathchar}.
%    \begin{macrocode}
        \edef\reserved@a
          {\noexpand\in@{\expandafter\@gobble\string\mathchar}%
                        {\meaning#1\expandafter\meaning\reserved@b}}%
        \reserved@a
%    \end{macrocode}
%    Drop  \verb*=#1 = in case it was defined before.
%    \begin{macrocode}
        \global\expandafter\let\reserved@b\@undefined
        \ifin@
          \expandafter\set@mathsymbol
             \csname sym#3\endcsname#1#2%
             {\hexnumber@{\count\z@}\hexnumber@{\count\tw@}}%
          \@font@info{Redeclaring math symbol \string#1}%
        \else
          \expandafter\ifx
            \csname\expandafter\@gobble\string#1\endcsname
            \relax
            \expandafter\set@mathsymbol
               \csname sym#3\endcsname#1#2%
               {\hexnumber@{\count\z@}\hexnumber@{\count\tw@}}%
          \else
            \@latex@error{Command `\string#1' already defined}\@eha
          \fi
        \fi
      \else
        \expandafter\set@mathchar
          \csname sym#3\endcsname#1#2
          {\hexnumber@{\count\z@}\hexnumber@{\count\tw@}}%
      \fi
    \endgroup
  \else
    \@latex@error{Symbol font `#3' is not defined}\@eha
  \fi
}
\@onlypreamble\DeclareMathSymbol
%    \end{macrocode}
% \end{macro}
%
% \begin{macro}{\set@mathchar}
%    \begin{macrocode}
\def\set@mathchar#1#2#3#4{%
  \global\mathcode`#2="\mathchar@type#3\hexnumber@#1#4\relax}
\@onlypreamble\set@mathchar
%    \end{macrocode}
% \end{macro}
%
% \begin{macro}{\set@mathsymbol}
%    \begin{macrocode}
\def\set@mathsymbol#1#2#3#4{%
  \global\mathchardef#2"\mathchar@type#3\hexnumber@#1#4\relax}
\@onlypreamble\set@mathsymbol
%    \end{macrocode}
% \end{macro}
%
%    \begin{macrocode}
%\def\mathsymbol#1#2#3{%
%  \@tempcnta=#3\relax
%  \@tempcntb\@tempcnta
%  \divide\@tempcnta\sixt@@n
%  \count@\@tempcnta
%  \multiply\count@\sixt@@n
%  \advance\@tempcntb-\count@
%  \mathchar"\mathchar@type#1\hexnumber@#2%
%             \hexnumber@\@tempcnta\hexnumber@\@tempcntb\relax}
%
%\def\DeclareMathAlphabetCharacter#1#2#3{%
%  \DeclareMathSymbol{#1}7{#2}{#3}}
%    \end{macrocode}
%
% \begin{macro}{\DeclareMathDelimiter}
% \changes{v2.1m}{1994/11/18}
%         {(DPC) \cs{expandafter} instead of \cs{next}}
%    \begin{macrocode}
\def\DeclareMathDelimiter#1{%
  \if\relax\noexpand#1%
    \expandafter\@DeclareMathDelimiter
  \else
    \expandafter\@xxDeclareMathDelimiter
  \fi
  #1}
\@onlypreamble\DeclareMathDelimiter
%    \end{macrocode}
% \end{macro}
%
% \begin{macro}{\@xxDeclareMathDelimiter}
% \changes{v3.0h}{1998/04/15}{Macro added (pr/2662)}
%    This macro checks if the second arg is a ``math type'' such
%    as |\mathopen|. The undocumented original code didn't use math
%    types when the delimiter was a single letter.
%    For this reason the coding is a bit strange as it tries to
%    support the undocumented syntax for compatibility reasons.
%    \begin{macrocode}
\def\@xxDeclareMathDelimiter#1#2#3#4{%
%    \end{macrocode}
%    7 is the default value returned in the case that |\mathchar@type|
%    is passed something unexpected, like a math symbol font name.
%    We locally move |\mathalpha| out of the way so if you use that
%    the right branch is taken. This will still fail if an explicit
%    number |7| is used!
%    \begin{macrocode}
   \begingroup
    \let\mathalpha\mathord
    \ifnum7=\mathchar@type{#2}%
      \endgroup
%    \end{macrocode}
%    If this branch is taken we have old syntax (5 arguments).
%    \begin{macrocode}
      \expandafter\@firstofone
    \else
%    \end{macrocode}
%    If this branch is taken |\mathchar@type| is different from 7 so
%    we assume new syntax. In this case we also use the arguments to
%    set up the letter as a math symbol for the case where it is not
%    used as a delimiter.
%    \begin{macrocode}
      \endgroup
      \DeclareMathSymbol#1{#2}{#3}{#4}%
%    \end{macrocode}
%    Then we arrange that |\@xDeclareMathDelimiter| only gets |#1|,
%    |#3|, |#4| \ldots\ as it does not expect a math type as argument.
%    \begin{macrocode}
      \expandafter\@firstoftwo
    \fi
    {\@xDeclareMathDelimiter#1}{#2}{#3}{#4}}
\@onlypreamble\@xxDeclareMathDelimiter
%    \end{macrocode}
% \end{macro}
%
% \begin{macro}{\@DeclareMathDelimiter}
% \changes{v3.0r}{2016/02/18}
%         {Check for delimiter not \cs{delimiter}}
%    \begin{macrocode}
\def\@DeclareMathDelimiter#1#2#3#4#5#6{%
  \expandafter\in@\csname sym#3\expandafter\endcsname
     \expandafter{\group@list}%
  \ifin@
    \expandafter\in@\csname sym#5\expandafter\endcsname
       \expandafter{\group@list}%
    \ifin@
      \begingroup
        \count\z@=#4\relax
        \count\tw@\count\z@
        \divide\count\z@\sixt@@n
        \count@\count\z@
        \multiply\count@\sixt@@n
        \advance\count\tw@-\count@
        \edef\reserved@c{\hexnumber@{\count\z@}\hexnumber@{\count\tw@}}%
      %
        \count\z@=#6\relax
        \count\tw@\count\z@
        \divide\count\z@\sixt@@n
        \count@\count\z@
        \multiply\count@\sixt@@n
        \advance\count\tw@-\count@
        \edef\reserved@d{\hexnumber@{\count\z@}\hexnumber@{\count\tw@}}%
      %
        \edef\reserved@a{\noexpand\in@
            {\expandafter\@gobble\string\delimiter}{\meaning#1}}%
        \reserved@a
        \ifin@
          \expandafter\set@mathdelimiter
             \csname sym#3\expandafter\endcsname
             \csname sym#5\endcsname#1#2%
             \reserved@c\reserved@d
          \@font@info{Redeclaring math delimiter \string#1}%
        \else
            \expandafter\ifx
            \csname\expandafter\@gobble\string#1\endcsname
            \relax
            \expandafter\set@mathdelimiter
              \csname sym#3\expandafter\endcsname
              \csname sym#5\endcsname#1#2%
              \reserved@c\reserved@d
          \else
            \@latex@error{Command `\string#1' already defined}\@eha
          \fi
        \fi
      \endgroup
    \else
      \@latex@error{Symbol font `#5' is not defined}\@eha
    \fi
  \else
    \@latex@error{Symbol font `#3' is not defined}\@eha
  \fi
}
%    \end{macrocode}
%
%    \begin{macrocode}
\@onlypreamble\@DeclareMathDelimiter
%    \end{macrocode}
% \end{macro}
%
% \begin{macro}{\@xDeclareMathDelimiter}
%    \begin{macrocode}
\def\@xDeclareMathDelimiter#1#2#3#4#5{%
  \expandafter\in@\csname sym#2\expandafter\endcsname
     \expandafter{\group@list}%
  \ifin@
    \expandafter\in@\csname sym#4\expandafter\endcsname
       \expandafter{\group@list}%
    \ifin@
      \begingroup
        \count\z@=#3\relax
        \count\tw@\count\z@
        \divide\count\z@\sixt@@n
        \count@\count\z@
        \multiply\count@\sixt@@n
        \advance\count\tw@-\count@
        \edef\reserved@c{\hexnumber@{\count\z@}\hexnumber@{\count\tw@}}%
      %
        \count\z@=#5\relax
        \count\tw@\count\z@
        \divide\count\z@\sixt@@n
        \count@\count\z@
        \multiply\count@\sixt@@n
        \advance\count\tw@-\count@
        \edef\reserved@d{\hexnumber@{\count\z@}\hexnumber@{\count\tw@}}%
        \expandafter\set@@mathdelimiter
           \csname sym#2\expandafter\endcsname\csname sym#4\endcsname#1%
           \reserved@c\reserved@d
      \endgroup
    \else
      \@latex@error{Symbol font `#4' is not defined}\@eha
    \fi
  \else
    \@latex@error{Symbol font `#2' is not defined}\@eha
  \fi
}
\@onlypreamble\@xDeclareMathDelimiter
%    \end{macrocode}
% \end{macro}
%
% \begin{macro}{\set@mathdelimiter}
%    We have to end the definition of a math delimiter like |\lfloor|
%    with a space and not with |\relax| as we did before, because
%    otherwise constructs involving |\abovewithdelims| will prematurely
%    end (pr/1329)
%
% \changes{v2.1q}{1995/04/02}{fix pr/1329}
%    \begin{macrocode}
%</2ekernel>
%<*2ekernel|latexrelease>
%<latexrelease>\IncludeInRelease{2019/10/01}%
%<latexrelease>                 {\set@mathdelimiter}{make delimiters robust}%
\def\set@mathdelimiter#1#2#3#4#5#6{%
%    \end{macrocode}
%    We use \cs{protected} not \cs{MakeRobust} so that
%    \verb=\bigl\lfoor= etc.\ works inside \cs{protected@edef}.
% \changes{v3.0s}{2019/08/27}{Make math delimiters robust}
% \changes{v3.0t}{2020/01/20}{fix for gh/251}
%    \begin{macrocode}
  \protected
  \xdef#3{\delimiter"\mathchar@type#4\hexnumber@#1#5%
    \hexnumber@#2#6 }%
%  \MakeRobust#3%
}
\@onlypreamble\set@mathdelimiter
%</2ekernel|latexrelease>
%<latexrelease>\EndIncludeInRelease
%<latexrelease>\IncludeInRelease{0000/00/00}%
%<latexrelease>                 {\set@mathdelimiter}{make delimiters robust}%
%<latexrelease>
%<latexrelease>\def\set@mathdelimiter#1#2#3#4#5#6{%
%<latexrelease>  \xdef#3{\delimiter"\mathchar@type#4\hexnumber@#1#5%
%<latexrelease>    \hexnumber@#2#6 }}
%<latexrelease>
%<latexrelease>\EndIncludeInRelease
%<*2ekernel>
%    \end{macrocode}
% \end{macro}
%
% \begin{macro}{\set@@mathdelimiter}
%    \begin{macrocode}
\def\set@@mathdelimiter#1#2#3#4#5{%
  \global\delcode`#3="\hexnumber@#1#4\hexnumber@#2#5\relax}
\@onlypreamble\set@@mathdelimiter
%    \end{macrocode}
% \end{macro}
%
% \begin{macro}{\DeclareMathRadical}
%    \begin{macrocode}
\def\DeclareMathRadical#1#2#3#4#5{%
%    \end{macrocode}
%    Below is a crude fix to make this macro work if |#1| is undefined
%    or |\relax|.  Should be improved!
% \changes{v2.1t}{1995/05/21}{Allow for undefined cs names}
% \changes{v3.0r}{2016/02/18}
%         {Check for radical not \cs{radical}}
%    \begin{macrocode}
  \expandafter\ifx
       \csname\expandafter\@gobble\string#1\endcsname
       \relax
     \let#1\radical
  \fi
  \edef\reserved@a{\noexpand\in@
       {\expandafter\@gobble\string\radical}{\meaning#1}}%
  \reserved@a
  \ifin@
    \expandafter\in@\csname sym#2\expandafter\endcsname
       \expandafter{\group@list}%
    \ifin@
      \expandafter\in@\csname sym#4\expandafter\endcsname
         \expandafter{\group@list}%
      \ifin@
        \begingroup
          \count\z@=#3\relax
          \count\tw@\count\z@
          \divide\count\z@\sixt@@n
          \count@\count\z@
          \multiply\count@\sixt@@n
          \advance\count\tw@-\count@
          \edef\reserved@c{%
            \hexnumber@{\count\z@}\hexnumber@{\count\tw@}}%
          \count\z@=#5\relax
          \count\tw@\count\z@
          \divide\count\z@\sixt@@n
          \count@\count\z@
          \multiply\count@\sixt@@n
          \advance\count\tw@-\count@
          \edef\reserved@d{%
            \hexnumber@{\count\z@}\hexnumber@{\count\tw@}}%
%    \end{macrocode}
%    Coded inline instead of using |\set@mathradical|
%    \begin{macrocode}
%          \expandafter\set@mathradical
%             \csname sym#2\expandafter\endcsname
%             \csname sym#4\endcsname#1%
%             \reserved@c\reserved@d
          \xdef#1{\radical"\expandafter\hexnumber@
                                \csname sym#2\endcsname\reserved@c
                             \expandafter\hexnumber@
                                \csname sym#4\endcsname\reserved@d
                  \relax}%
        \endgroup
      \else
        \@latex@error{Symbol font `#4' is not defined}\@eha
      \fi
    \else
      \@latex@error{Symbol font `#2' is not defined}\@eha
    \fi
  \else
    \@latex@error{Command `\string#1' already defined}\@eha
  \fi
}
\@onlypreamble\DeclareMathRadical
%    \end{macrocode}
% \end{macro}
%
% Definition below was wrong it contained |\delimiter| !
%
%\begin{verbatim}
%\def\set@mathradical#1#2#3#4#5{%
%  \xdef#3{\radical"\hexnumber@#1#4\hexnumber@#2#5\relax}}
%\end{verbatim}
%
% \begin{macro}{\mathalpha}
% just a dummy currently
%    \begin{macrocode}
\let\mathalpha\relax
%    \end{macrocode}
% \end{macro}
%
% \begin{macro}{\mathchar@type}
%    \begin{macrocode}
\def\mathchar@type#1{%
  \ifodd 2#11 #1\else             % is this non-negative number?
    \ifx#1\mathord 0\else
     \ifx#1\mathop   1\else
       \ifx#1\mathbin 2\else
         \ifx#1\mathrel 3\else
           \ifx#1\mathopen 4\else
             \ifx#1\mathclose 5\else
               \ifx#1\mathpunct 6\else
                   7%             % anything else is variable ord
               \fi
             \fi
           \fi
         \fi
       \fi
     \fi
    \fi
  \fi}
\@onlypreamble\mathchar@type
%    \end{macrocode}
% \end{macro}
%
% \begin{macro}{\DeclareSymbolFontAlphabet}
% \changes{v3.0j}{2005/07/27}
%         {(MH) Make document commands robust}
%    \begin{macrocode}
\def\DeclareSymbolFontAlphabet#1#2{%
   \expandafter\DeclareSymbolFontAlphabet@
     \csname \expandafter\@gobble\string#1\space\endcsname{#2}#1}
\@onlypreamble\DeclareSymbolFontAlphabet
%    \end{macrocode}
% \end{macro}
%
% \begin{macro}{\DeclareSymbolFontAlphabet@}
%    \begin{macrocode}
\def\DeclareSymbolFontAlphabet@#1#2#3{%
%    \end{macrocode}
%    We use the switch |\if@tempswa| to decide if we can declare this
%    symbol font alphabet.
%    \begin{macrocode}
    \@tempswatrue
%    \end{macrocode}
%    First check if |#2| is known to be a symbol font
%    \begin{macrocode}
  \expandafter\in@\csname sym#2\expandafter\endcsname
     \expandafter{\group@list}%
  \ifin@
%    \end{macrocode}
%    Check if |#1| is defined as a math alphabet defined via
%    |\DeclareMathAlphabet|:
%    \begin{macrocode}
    \expandafter\in@\expandafter#1\expandafter{\alpha@list}%
    \ifin@
%    \end{macrocode}
%    If so remove it from the |\alpha@list| and from all math version
%    macros.
%    \begin{macrocode}
      \@font@info{Redeclaring math alphabet \string#3}%
      \toks@{}%
      \def\alpha@elt##1##2##3{%
          \ifx##1#1\else\addto@hook\toks@{\alpha@elt##1##2##3}\fi}%
      \alpha@list
      \xdef\alpha@list{\the\toks@}%
%    \end{macrocode}
%    Now we loop over all versions and remove the math alphabet:
%    \begin{macrocode}
      \def\version@elt##1{%
          \begingroup
            \toks@{}%
            \def\getanddefine@fonts####1####2{%
               \addto@hook\toks@{\getanddefine@fonts####1####2}}%
            \def\install@mathalphabet####1####2{%
               \ifx####1#1\else
                 \addto@hook\toks@{\install@mathalphabet
                                    ####1{####2}}\fi}%
            ##1%
            \xdef##1{\the\toks@}%
          \endgroup
          }%
      \version@list
    \else
%    \end{macrocode}
%    If |#3| is not defined as a math alphabet check if it is defined
%    at all:
%    \begin{macrocode}
      \expandafter\ifx
      \csname\expandafter\@gobble\string#1\space\endcsname
      \relax
%    \end{macrocode}
%    If it is undefined, fine otherwise check if it is a math alphabet
%    defined via |\DeclareSymbolFontAlphabet|:
%    \begin{macrocode}
      \else
        \edef\reserved@a{%
          \noexpand\in@{\string\use@mathgroup}{\meaning#1}}%
        \reserved@a
        \ifin@
          \@font@info{Redeclaring math alphabet \string#3}%
        \else
%    \end{macrocode}
%    Since the command |#3| is defined to be something which is not a
%    math alphabet we have to skip redefining it.
%    \begin{macrocode}
          \@tempswafalse
          \@latex@error{Command `\string#3' already defined}\@eha
        \fi
      \fi
    \fi
   \else
%    \end{macrocode}
%    Since the symbol font is not known we better skip defining this
%    alphabet.
%    \begin{macrocode}
     \@tempswafalse
     \@latex@error{Unknown symbol font `#2'}\@eha
   \fi
   \if@tempswa
%    \end{macrocode}
%    When we reach this point we are allowed to define |#1| to be a
%    symbol font math alphabet. This means that we have to set it to
%    \begin{quote}
%      |\use@mathgroup| \meta{math-settings} |\sym|\meta{name}
%    \end{quote}
%    The \meta{math-settings} are the one for the encoding that is
%    used in the font shape where |\sym|\meta{name} is pointing to.
%    This means that we have to get it from the information stored in
%    |\group@list|. Thus we loop through that list after defining
%    |\group@elt| in a suitable way.
% \changes{v2.1b}{1994/03/11}{Added check against use of alphabet
%              switch outside of math mode.}
%    \begin{macrocode}
     \def\group@elt##1##2{%
        \expandafter\ifx\csname sym#2\endcsname##1%
        \expandafter\reserved@a\string##2\@nil
        \fi}%
     \def\reserved@a##1##2/##3\@nil{%
        \def\reserved@a{##2}}%
     \group@list
     \toks@{\relax\ifmmode \else \non@alpherr#1\fi}%
     \edef#1{\the\toks@
             \noexpand\use@mathgroup
             \expandafter\noexpand\csname M@\reserved@a\endcsname
             \csname sym#2\endcsname}%
     \def#3{\protect#1}%
   \fi
}
\@onlypreamble\DeclareSymbolFontAlphabet@
%</2ekernel>
%    \end{macrocode}
%  \end{macro}
%
% \Finale
%
