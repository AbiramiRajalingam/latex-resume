% \iffalse meta-comment
%
% Copyright (C) 1993-2021
% The LaTeX Project and any individual authors listed elsewhere
% in this file.
%
% This file is part of the LaTeX base system.
% -------------------------------------------
%
% It may be distributed and/or modified under the
% conditions of the LaTeX Project Public License, either version 1.3c
% of this license or (at your option) any later version.
% The latest version of this license is in
%    https://www.latex-project.org/lppl.txt
% and version 1.3c or later is part of all distributions of LaTeX
% version 2008 or later.
%
% This file has the LPPL maintenance status "maintained".
%
% The list of all files belonging to the LaTeX base distribution is
% given in the file `manifest.txt'. See also `legal.txt' for additional
% information.
%
% The list of derived (unpacked) files belonging to the distribution
% and covered by LPPL is defined by the unpacking scripts (with
% extension .ins) which are part of the distribution.
%
% \fi
% \iffalse
%%
%% File `alltt.dtx'.
%% Copyright (C) 1987 by Leslie Lamport
%% Copyright (C) 1994-2021 LaTeX Project, Johannes Braams
%%                       all rights reserved.
%%
%<*dtx>
\ProvidesFile{alltt.dtx}
%</dtx>
%<package>\NeedsTeXFormat{LaTeX2e}
%<package>\ProvidesPackage{alltt}
%<driver>\ProvidesFile{alltt.drv}
%\ProvidesFile{alltt.dtx}
              [2021/01/29 v2.0g defines alltt environment]
%
%<*driver>
\documentclass{ltxdoc}
\usepackage{alltt}
\begin{document}
\providecommand{\Lenv}[1]{\textsf{#1}}
\providecommand{\pkg}[1]{\texttt{#1}}
\providecommand{\file}[1]{\texttt{#1}}
\DocInput{alltt.dtx}
\end{document}
%</driver>
% \fi
%
% \changes{v2.0g}{1997/06/16}{A few documentation fixes (PR 2520)}
% \changes{v2.0d}{1995/04/02}{A few documentation fixes (PR 1517)}
% \changes{v2.0g}{2021-01-29}{Added a warning about OT1 versus T1 font
% encoding}
%
%  \GetFileInfo{alltt.dtx}
%  \title{The \texttt{alltt} environment\thanks{This file
%        has version number \fileversion, last
%        revised \filedate.}}
%  \author{Johannes Braams}
%  \date{\filedate}
%  \MaintainedByLaTeXTeam{latex}
%  \maketitle
%
%    \begin{abstract}
%    This package defines the \Lenv{alltt} environment, which is like
%    the \Lenv{verbatim} environment except that |\|, |{|, and |}|
%    have their usual meanings.
%
%    Thus, other commands and environments can appear within an
%    \Lenv{alltt} environment.
%    \end{abstract}
%
%  \section{Introduction}
%
%  \DescribeEnv{alltt}
%   Here are some things you may want to do in an \Lenv{alltt}
%   environment:
%   \begin{itemize}
%   \item Change fonts--e.g., by typing |{\em emphasized text\/}|
%
%   \item Insert text from a file \file{foo.tex} by typing
%    |\input{foo}|. Beware that each |<return>| starts a new line, so
%    if \file{foo.tex} ends with a |<return>| you can wind up with an
%    extra blank line if you're not careful.
%
%   \item Insert a math formula.  Note that |$| just produces a dollar
%    sign, so you'll have to type |\(...\)| or |\[...\]|.  Also, |^|
%    and |_| just produce their characters; use |\sp| or |\sb| for
%    super- and subscripts, as in |\(x\sp{2}\)|.
%   \end{itemize}
%
%    \textbf{NB} When you are using \textsf{OT1} encoded fonts you might
%    be surprsied when you switch to italics, becuase those fonts have
%    a different set of glyphs:
%    \begin{alltt}
%  The glyph at the position of the $ in a slanted font: \textsl{$}.
%  The glyph at the position of the $ in an italic font: \textit{$}.
%    \end{alltt}
%
%  \StopEventually{}
%
%  \section{The Implementation}
%
%    \begin{macrocode}
%<*package>
%    \end{macrocode}
%
%  \begin{environment}{alltt}
%    The \Lenv{alltt} environment is similar to the \Lenv{verbatim}
%    environment, except that |\|, |{| and |}| have their usual
%    behaviour.
% \changes{v2.0b}{1994/10/29}{Added a missing \cs{dospecials}}
% \changes{v2.0c}{1995/01/27}{Reset \cs{dospecials} after its changed
%    version is executed}
% \changes{v2.0e}{1995/05/13}{brought definition up to date with the
%    verbatim environment}
% \changes{v2.0e}{1995/05/13}{Save and restore (in math mode) the
%    definition of the \texttt{'} character}
%    \begin{macrocode}
\begingroup
\lccode`\~=`\'
\lowercase{\endgroup
\newenvironment{alltt}{%
  \trivlist
  \item\relax
    \if@minipage
    \else
      \vskip\parskip
    \fi
    \leftskip\@totalleftmargin
    \rightskip\z@skip
    \parindent\z@
    \parfillskip\@flushglue
    \parskip\z@skip
    \@@par
    \@tempswafalse
    \def\par{%
      \if@tempswa
        \leavevmode\null\@@par\penalty\interlinepenalty
    \else
      \@tempswatrue
      \ifhmode\@@par\penalty\interlinepenalty\fi
    \fi}
    \obeylines
    \verbatim@font
    \let\org@prime~%
    \@noligs
%    \end{macrocode}
% \changes{v2.0f}{1995/07/24}{Added setting of the \cs{catcode} of the
%    \texttt{'} character}
%    \begin{macrocode}
    \everymath\expandafter{\the\everymath
      \catcode`\'=12 \let~\org@prime}
    \everydisplay\expandafter{\the\everydisplay
      \catcode`\'=12 \let~\org@prime}
    \let\org@dospecials\dospecials
    \g@remfrom@specials{\\}
    \g@remfrom@specials{\{}
    \g@remfrom@specials{\}}
    \let\do\@makeother
    \dospecials
    \let\dospecials\org@dospecials
    \frenchspacing\@vobeyspaces
    \everypar \expandafter{\the\everypar \unpenalty}}
{\endtrivlist}}
%    \end{macrocode}
%  \end{environment}
%
%  \begin{macro}{\g@remfrom@specials}
%    In the old implementation of the \pkg{alltt} package a fixed
%    |\dospecials| was used. However nowadays the |\dospecials|
%    command might contain more special characters at run-time then as
%    was defined in the format. Therefore we remove the necessary
%    special character from |\dospecials| at the start of the
%    \Lenv{alltt} environment. For this we need a macro. Remember that
%    the list of special characters in |\dospecials| contains the
%    control sequence |\do| between the characters. We use that to
%    check whether a character has to be removed.
%
%    The macro |\g@remfrom@specials| takes one argument, the character
%    to be removed from the list.
%    \begin{macrocode}
\def\g@remfrom@specials#1{%
%    \end{macrocode}
%    We build up a new list in |\@new@speicals|.
%    \begin{macrocode}
  \def\@new@specials{}
%    \end{macrocode}
%    The command |\@remove| compares its argument with the argument of\\
%    |\g@remfrom@specials|.
%    \begin{macrocode}
  \def\@remove##1{%
    \ifx##1#1\else
%    \end{macrocode}
%    When they are not the same the argument of |\@remove| is added
%    (together with |\do|) to the new list.
%    \begin{macrocode}
    \g@addto@macro\@new@specials{\do ##1}\fi}
%    \end{macrocode}
%    Now we |\let| |\do| be equal to |\@remove| and execute
%    |\dospecials|.
%    \begin{macrocode}
  \let\do\@remove\dospecials
%    \end{macrocode}
%    All that's left is to make |\dospecials| point to the new list.
%    \begin{macrocode}
  \let\dospecials\@new@specials
  }
%    \end{macrocode}
%  \end{macro}
%
%    \begin{macrocode}
%</package>
%    \end{macrocode}
%
% \Finale
\endinput
