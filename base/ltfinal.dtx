% \iffalse meta-comment
%
% Copyright (C) 1993-2020
% The LaTeX3 Project and any individual authors listed elsewhere
% in this file.
%
% This file is part of the LaTeX base system.
% -------------------------------------------
%
% It may be distributed and/or modified under the
% conditions of the LaTeX Project Public License, either version 1.3c
% of this license or (at your option) any later version.
% The latest version of this license is in
%    https://www.latex-project.org/lppl.txt
% and version 1.3c or later is part of all distributions of LaTeX
% version 2008 or later.
%
% This file has the LPPL maintenance status "maintained".
%
% The list of all files belonging to the LaTeX base distribution is
% given in the file `manifest.txt'. See also `legal.txt' for additional
% information.
%
% The list of derived (unpacked) files belonging to the distribution
% and covered by LPPL is defined by the unpacking scripts (with
% extension .ins) which are part of the distribution.
%
% \fi
%
% \iffalse
%%% From File: ltfinal.dtx
%
%<*driver>
% \fi
\ProvidesFile{ltfinal.dtx}
             [2020-08-21 v2.2i LaTeX Kernel (Final Settings)]
% \iffalse
\documentclass{ltxdoc}
\GetFileInfo{ltfinal.dtx}
\title{\filename}
\date{\filedate}
\author{%
  Johannes Braams\and
  David Carlisle\and
  Alan Jeffrey\and
  Leslie Lamport\and
  Frank Mittelbach\and
  Chris Rowley\and
  Rainer Sch\"opf}
\begin{document}
 \MaintainedByLaTeXTeam{latex}
 \maketitle
 \DocInput{ltfinal.dtx}
\end{document}
%</driver>
% \fi
%
%
% \section{Final settings}
% This section contains the final settings for \LaTeX.  It initialises
% some debugging and typesetting parameters, sets the default
% |\catcode|s and uc/lc codes, and inputs the hyphenation file.
%
% \StopEventually{}
%
% \changes{v0.1a}{1994/03/07}{Initial version, split from latex.dtx}
% \changes{v0.1a}{1994/03/07}{Remove oldcomments environment}
% \changes{v0.1c}{1994/04/21}{Added comments, set the catcodes of
%    128--255.}
% \changes{v0.1d}{1994/04/23}{Check that \cs{font@submax} is still zero}
% \changes{v0.1e}{1994/05/02}{Set all the catcodes}
% \changes{v0.1f}{1994/05/03}{Set the catcode of control-J to be
%    `other', for use in messages.}
% \changes{v0.1g}{1994/05/05}{Added empty errhelp.}
% \changes{v0.1h}{1994/05/13}{Added package ot1enc, and defined
%    \cs{@acci}, \cs{@accii} and \cs{@acciii}.}
% \changes{v0.1j}{1994/05/18}{Corrected the lccode for d-bar.}
% \changes{v0.1k}{1994/05/19}{Removed \cs{makeat...}}
% \changes{v1.0n}{1994/05/31}{Renamed lthyphen.* to lthyphen.*.}
% \changes{v1.0o}{1994/11/17}
%         {\cs{@tempa} to \cs{reserved@a}}
% \changes{v1.0p}{1994/12/01}
%         {Renamed lthyphen.* to hyphen.*.}
% \changes{v1.0r}{1995/06/05}
%         {Added \cs{MakeUppercase} and \cs{MakeLowercase}.}
% \changes{v1.0s}{1995/06/06}
%         {Made \cs{MakeUppercase} and \cs{MakeLowercase} brace their
%         argument.}
% \changes{v2.0r}{2016/10/15}{Require e\TeX{}}
% \changes{v2.0s}{2016/10/15}{Tidy up status of char 127}
% \changes{v2.2i}{2020/08/21}{Integration of new hook management interface}
%
% \subsection{Debugging}
%
% By default, \LaTeX{} shows statistics:
%    \begin{macrocode}
%<*2ekernel>
\tracingstats1
%    \end{macrocode}
%
% \subsection{Typesetting parameters}
%
% \begin{macro}{\@lowpenalty}
% \begin{macro}{\@medpenalty}
% \begin{macro}{\@highpenalty}
%    These are penalties used internally.
%    \begin{macrocode}
\newcount\@lowpenalty
\newcount\@medpenalty
\newcount\@highpenalty
%    \end{macrocode}
% \end{macro}
% \end{macro}
% \end{macro}
%
%
%\begin{macro}{\newmarks}
% \changes{v2.0a}{2014/12/30}{macro added}
% \changes{v2.0b}{2015/01/23}{use reserved count 256}
% \changes{v2.0g}{2015/06/19}{Use $-1$ for first range to get contiguous allocation}
% Allocate extended marks types if etex is active.
% Placed here at the end of the format
% to increase compatibility with count allocations
% in earlier releases.
%    \begin{macrocode}
%</2ekernel>
%<*2ekernel|latexrelease>
%<latexrelease>\IncludeInRelease{2015/01/01}%
%<latexrelease>                 {\newmarks}{Extended Allocation}%
%    \end{macrocode}
%
%    \begin{macrocode}
\ifx\marks\@undefined\else
\def\newmarks{%
  \e@alloc\marks \e@alloc@chardef{\count256}\m@ne\e@alloc@top}
\fi
%    \end{macrocode}
%
%    \begin{macrocode}
%</2ekernel|latexrelease>
%<latexrelease>\EndIncludeInRelease
%<latexrelease>\IncludeInRelease{0000/00/00}%
%<latexrelease>                 {\newmarks}{Extended Allocation}%
%<latexrelease>\let\newmarks\@undefined
%<latexrelease>\EndIncludeInRelease
%<*2ekernel>
%    \end{macrocode}
% \end{macro}
%
%\begin{macro}{\newXeTeXintercharclass}
% \changes{v2.0a}{2014/12/30}{macro added}
% \changes{v2.0b}{2015/01/23}{use reserved count 257}
% \changes{v2.0f}{2015/04/28}{define \cs{xe@alloc@intercharclass} for compatibility with older xelatex initilisation}
%\begin{macro}{\xe@alloc@intercharclass}
%\begin{macro}{\e@alloc@intercharclass@top}
% \changes{v2.0j}{2016/01/04}{Start allocation at one not three}
% \changes{v2.0k}{2016/01/05}{Remove duplicated code}
% Allocate |\XeTeXintercharclass|  types if xetex is active.
% previously defined in |xetex.ini|.
%
%    \begin{macrocode}
%</2ekernel>
%<*2ekernel|latexrelease>
%<latexrelease>\IncludeInRelease{2015/01/01}%
%<latexrelease>              {\newXeTeXintercharclass}{Extended Allocation}%
%    \end{macrocode}
%
% Classes allocated 1 to 4094 (or 254 on older xetex)
% (In earlier XeLaTeX versions 1, 2 and 3 were pre-set for CJK).
% \changes{v2.0g}{2015/06/19}{Use $-1$ for first range to get contiguous allocation}
% \changes{v2.0q}{2016/04/22}{XeTeX 0.99996 has 4096 char classes not 256}
%    \begin{macrocode}
\ifx\XeTeXcharclass\@undefined
\else
%    \end{macrocode}
%    \begin{macrocode}
\ifdim\the\XeTeXversion\XeTeXrevision\p@>0.99993\p@
  \chardef\e@alloc@intercharclass@top=4095
\else
  \chardef\e@alloc@intercharclass@top=255
\fi
%    \end{macrocode}
%    \begin{macrocode}
\def\newXeTeXintercharclass{%
 \e@alloc\XeTeXcharclass
   \chardef\xe@alloc@intercharclass\m@ne\e@alloc@intercharclass@top}
\fi
%    \end{macrocode}
%
%    \begin{macrocode}
%</2ekernel|latexrelease>
%<latexrelease>\EndIncludeInRelease
%<latexrelease>\IncludeInRelease{0000/00/00}%
%<latexrelease>              {\newXeTeXintercharclass}{Extended Allocation}%
%<latexrelease> \ifx\XeTeXcharclass\@undefined
%<latexrelease> \else
%<latexrelease>    \def\xe@alloc@#1#2#3#4#5{\global\advance#1\@ne
%<latexrelease>     \xe@ch@ck#1#4#2%
%<latexrelease>     \allocationnumber#1%
%<latexrelease>     \global#3#5\allocationnumber
%<latexrelease>     \wlog{\string#5=\string#2\the\allocationnumber}}
%<latexrelease>    \def\xe@ch@ck#1#2#3{%
%<latexrelease>     \ifnum#1<#2\else
%<latexrelease>      \errmessage{No room for a new #3}%
%<latexrelease>     \fi}
%<latexrelease>    \def\newXeTeXintercharclass{%
%<latexrelease>     \xe@alloc@\xe@alloc@intercharclass
%<latexrelease>                    \XeTeXcharclass\chardef\@cclv}
%<latexrelease> \fi
%<latexrelease>\EndIncludeInRelease
%<*2ekernel|latexrelease>
%<latexrelease>\IncludeInRelease{2016/02/01}%
%<latexrelease>  {\xe@alloc@intercharclass}{Start of XeTeX class allocator}%
\ifx\XeTeXcharclass\@undefined
\else
  \countdef\xe@alloc@intercharclass=257
  \xe@alloc@intercharclass=\z@
\fi
%</2ekernel|latexrelease>
%<latexrelease>\EndIncludeInRelease
%<latexrelease>\IncludeInRelease{2015/01/01}%
%<latexrelease>  {\xe@alloc@intercharclass}{Start of XeTeX class allocator}%
%<latexrelease> \ifx\XeTeXcharclass\@undefined
%<latexrelease> \else
%<latexrelease>   \xe@alloc@intercharclass=\thr@@
%<latexrelease> \fi
%<latexrelease>\EndIncludeInRelease
%<latexrelease>\IncludeInRelease{0000/00/00}%
%<latexrelease>  {\xe@alloc@intercharclass}{Start of XeTeX class allocator}%
%<latexrelease> \ifx\XeTeXcharclass\@undefined
%<latexrelease> \else
%<latexrelease>   \newcount\xe@alloc@intercharclass
%<latexrelease>   \xe@alloc@intercharclass=\thr@@
%<latexrelease> \fi
%<latexrelease>\EndIncludeInRelease
%<*2ekernel>
%    \end{macrocode}
% \end{macro}
% \end{macro}
% \end{macro}
%
%
%
% The default values of the picture and |\fbox| parameters:
%    \begin{macrocode}
\unitlength = 1pt
\fboxsep = 3pt
\fboxrule = .4pt
%    \end{macrocode}
% The saved value of \TeX's |\maxdepth|:
%    \begin{macrocode}
\@maxdepth       = \maxdepth
%    \end{macrocode}
% |\vsize| initialized because a |\clearpage| with |\vsize < \topskip|
%  causes trouble.
% |\@colroom| and |\@colht| also initialized because |\vsize| may be
%  set to them if a |\clearpage| is done before the |\begin{document}|
%
%    \begin{macrocode}
\vsize = 1000pt
\@colroom = \vsize
\@colht = \vsize
%    \end{macrocode}
% Initialise |\textheight| |\textwidth| and page style, to avoid
% internal errors if they are not set by the class.
% \changes{v0.1b}{1994/04/18}
%         {Initialise \cs{textheight}, \cs{textwidth} and page style}
%    \begin{macrocode}
\textheight=.5\maxdimen
\textwidth=\textheight
\ps@empty
%    \end{macrocode}
%
% \subsection{Lccodes for hyphenation}
%
% \changes{v2.0a}{2015/01/03}{Unicode data loading added}
% \changes{v2.0c}{2015/01/24}{Skip T1-code entirely with Unicode engines}
% \changes{v2.0d}{2015/03/26}{Use renamed
%   \texttt{unicode-letters.def}}
% \changes{v2.0i}{2015/12/10}{Use new common Unicode data loaders}
% \changes{v2.0j}{2016/01/04}{Do not set up inter character classes for
%   XeTeX}
%  \changes{v2.0l}{2016/01/05}{Correct \textsf{latexrelease} guards}
%  \changes{v2.0l}{2016/01/05}{Ensure old definitions for inter-character
%    class toks are available using \textsf{latexrelease}}
%  \changes{v2.0m}{2016/01/05}{Undefine XeTeX classes when using patching
%    an older kernel}
%  \changes{v2.0l}{2016/01/05}{Missing brace}
%  \changes{v2.0p}{2016/01/05}{Only apply XeTeX change if XeTeX is in use}
%  For $7$- and $8$-bit engines the assumption of T1 encodings is the
%  basis for the hyphenation patterns. That's not the case for the Unicode
%  engines, where the assumption is engine-native working. The common
%  loader system provides access to data from the Unicode Consortium
%  covering not only |\lccode| but also other related data. The
%  |\lccode| part of that at least needs to be loaded before hyphenation is
%  tackled: Xe\TeX{} follows the standard \TeX{} route of building patterns
%  into the format. Lua\TeX{} doesn't require this data be loaded \emph{here}
%  but it does need to be loaded somewhere. Rather than test for the Unicode
%  engines by name, the approach here is to look for the extended math mode
%  handling both provide: any other engine developed in this area will
%  presumably also provide |\Umathcode|.
%    \begin{macrocode}
\ifnum 0%
  \ifx\Umathcode\@undefined\else 1\fi
  \ifx\XeTeXmathcode\@undefined\else 1\fi
  >\z@
  \message{ Unicode character data,}
  % File load-unicode-data.tex
%
% Copyright 2015-2019 The LaTeX3 Project
%
% It may be distributed and/or modified under the conditions of
% the LaTeX Project Public License (LPPL), either version 1.3c of
% this license or (at your option) any later version. The latest
% version of this license is in the file
% http://www.latex-project.org/lppl.txt.
%
% Issues with this file should be reported at
% https://github.com/latex3/unicode-data
%
% This file parses a number of data files provided by the Unicode Consortium
% and when used with used Unicode-capable engine sets up a range of TeX-related
% parameters based on the extracted information.
%
% From the file UnicodeData.txt the following properties are set:
% - \catcode 11 for all letters (Unicode class "L")
% - \catcode 11 for all combining marks (Unicode class "M")
% - \sfcode 999 for all code points of class "Lu" (upper case letters)
% - \lccode for all of class "Ll" (lower case letters) to the code point
%   itself, and \uccode to the upper case mapping (or if not given
%   to the code point itself)
% - \uccode for all of class "Lu" (upper case letters) to the code point
%   itself, and \lccode to the lower case mapping (or if not given
%   to the code point itself)
% - \lccode and \uccode for all of class "Lt" (title case letters) to the
%   lower and upper case mappings (or if not given to the code point itself)
% - \lccode and \uccode for all other letter code points are set to
%   the code point itself
% - \lccode and/or \uccode for non-letter code points for which an upper
%   or lower case mapping is given
% - \sfcode 0 (ignored) for code points of Unicode classes "Pe" (closing
%   punctuation marks) and "Pf" (final quotation marks)
% - \Umathcode for all letters as math type 7 (var)
%
% =============================================================================
%
% The data can only be loaded by Unicode engines. Currently this is limited to
% XeTeX and LuaTeX, both of which define \Umathcode.
\ifx\Umathcode\undefined
  \expandafter\endinput
\fi
% Just in case, check for the e-TeX extensions.
\ifx\eTeXversion\undefined
  \expandafter\endinput
\fi
% This file can be loaded in IniTeX mode so the category codes of |{|, |}| and
% |#| may not be correct. Everything is done in a group so that only the
% settings we want to propagate are made available generally.
\begingroup
  \catcode`\{=1 %
  \catcode`\}=2 %
  \catcode`\#=6 %
% Write some basic information to the log.
  \catcode`\^=7 %
  \newlinechar=`\^^J %
  \message{^^J}%
  \message{load-unicode-data.tex v1.10 (2019-08-21)^^J}%
  \message{Reading Unicode data^^J}%
% The first stage of parsing is dealing with the fact that there are lots of
% data items separated by |;|. Of those, only a few are needed so they are
% picked out and everything else is dropped. There is one complication: there
% are a few cases in the data file of ranges which are marked by the descriptor
% |First| and a matching |Last|. A separate routine is used to handle these
% cases.
  \def\parseunicodedataI#1;#2;#3;#4;#5;#6;#7;#8;#9;{%
    \parseunicodedataII#1;#3;#2 First>\relax
  }%
  \def\parseunicodedataII#1;#2;#3 First>#4\relax{%
    \ifx\relax#4\relax
      \expandafter\parseunicodedataIII
    \else
      \expandafter\parseunicodedataVII
    \fi
    #1;#2;%
  }%
  \def\parseunicodedataIII#1;#2;#3;#4;#5;#6;#7;#8\relax{%
    \parseunicodedataIV{#1}{#2}{#6}{#7}%
  }%
% At this stage we have a `normal' data line with four pieces of information:
% the code point, the Unicode class and the (possibly empty) upper and lower
% case mappings. A few utility macros are defined, then we branch based on the
% Unicode class. Notice that for all letter-like code points we first set the
% |\lccode| and |\uccode| values to the code point itself then test for the
% classes where a different setting might be appropriate. For non-letters
% there is a check to see if any mappings are available, and also for trailing
% punctuation to set the appropriate |\sfcode|.
  \def\Ll{Ll}%
  \def\Lt{Lt}%
  \def\Lu{Lu}%
  \def\Pe{Pe}%
  \def\Pf{Pf}%
  \def\firsttoken#1#2\relax{#1}%
  \def\parseunicodedataIV#1#2#3#4{%
    \ifnum 0%
      \if L\firsttoken#2?\relax 1\fi
      \if M\firsttoken#2?\relax 1\fi
      >0 %
      \parseunicodedataV{"#1}%
      \def\temp{#2}%
      \ifx\Ll\temp
        \parseunicodedataVI\uccode{#1}{#3}%
      \fi
      \ifx\Lt\temp
        \parseunicodedataVI\uccode{#1}{#3}%
        \parseunicodedataVI\lccode{#1}{#4}%
      \fi
      \ifx\Lu\temp
        \parseunicodedataVI\lccode{#1}{#4}%
        \global\sfcode"#1=999 %
      \fi
% All letters in math mode should be variables.
      \global\Umathcode"#1="7"01"#1 %
    \else
      \def\temp{#2}%
      \ifnum 0\ifx\temp\Pe 1\fi\ifx\temp\Pf 1\fi>0 %
        \global\sfcode"#1=0 %
      \fi
      \ifx\relax#3\relax
      \else
        \global\uccode"#1="#3 %
      \fi
      \ifx\relax#4\relax
      \else
        \global\lccode"#1="#4 %
      \fi
    \fi
  }%
% A simple auxiliary for all letter-like code points: the |\lccode| and
% |\uccode| may get reset for cased letters but this means the initial
% setting can't be forgotten.
  \def\parseunicodedataV#1{%
    \global\catcode#1=11 %
    \global\lccode#1=#1 %
    \global\uccode#1=#1 %
  }%
% An auxiliary to deal with the fact that some cased letters don't actually
% have a case mapping available.
  \def\parseunicodedataVI#1#2#3{%
    \ifx\relax#3\relax
    \else
      \global#1"#2="#3 %
    \fi
  }%
% For lines that were the |First>| of a range, read the data source again for
% last line. Lines for letters then trigger a loop over the entire range. These
% are always non-cased letters.
  \def\parseunicodedataVII#1;#2;#3\relax{%
    \read0 to \unicodedataline
    \expandafter\parseunicodedataXII\unicodedataline\relax#1;#2\relax
  }%
  \def\parseunicodedataXII#1;#2\relax#3;#4\relax{%
    \if L\firsttoken#4?\relax
      \begingroup
        \count0="#3 %
        \loop
          \unless\ifnum\count0>"#1 %
            \parseunicodedataV{\count0 }%
            \advance\count0 by 1 %
        \repeat
      \endgroup
    \fi
  }%
% From plain: may not be defined (yet).
  \def\loop#1\repeat{\def\body{#1}\iterate}%
  \def\iterate{%
    \body
      \let\next\iterate
    \else
      \let\next\relax
    \fi
    \next
  }%
  \let\repeat\fi
% There is no version data in |UnicodeData.txt|: log that it is being used with
% a hard-coded date (the modification date from ftp.unicode.org). This obviously
% needs to be updated when a new download takes place!
  \message{\string# UnicodeData-12.1.0.txt^^J}%
  \message{\string# Modified 2019-04-01 05:08:00 GMT [MFK]^^J}%
% Actually loading the file requires an input stream, done directly.
% There is a blank line at the end of the data source so there is a check
% here for a |\par|.
  \def\storedpar{\par}%
  \openin0=UnicodeData.txt %
  \loop\unless\ifeof0 %
    \read0 to \unicodedataline  
    \unless\ifx\unicodedataline\storedpar
      \expandafter\parseunicodedataI\unicodedataline\relax
    \fi
  \repeat
  \closein0 %
\endgroup

%</2ekernel>
%<latexrelease>\IncludeInRelease{2016/02/01}%
%<latexrelease>  {\XeTeXintercharclasses}{XeTeX character classes}%
%<latexrelease>  \ifx\XeTeXinterchartoks\undefined
%<latexrelease>  \else
%<latexrelease>    \begingroup
%<latexrelease>      \chardef\XeTeXcharclassID = 0 %
%<latexrelease>      \chardef\XeTeXcharclassOP = 0 %
%<latexrelease>      \chardef\XeTeXcharclassCL = 0 %
%<latexrelease>      \chardef\XeTeXcharclassEX = 0 %
%<latexrelease>      \chardef\XeTeXcharclassIS = 0 %
%<latexrelease>      \chardef\XeTeXcharclassNS = 0 %
%<latexrelease>      \chardef\XeTeXcharclassCM = 0 %
%<latexrelease>      % File load-unicode-xetex-classes.tex
%
% Copyright 2015-2023 The LaTeX Project
%
% It may be distributed and/or modified under the conditions of
% the LaTeX Project Public License (LPPL), either version 1.3c of
% this license or (at your option) any later version. The latest
% version of this license is in the file
% http://www.latex-project.org/lppl.txt.
%
% Issues with this file should be reported at
% https://github.com/latex3/unicode-data
%
% This file parses EastAsianWidth.txt and LineBreak.txt, provided by the
% Unicode Consortium, and when used with XeTeX sets \XeTeXcharclass for
% the following classes of code point:
% - "ID" (ideographic)
% - "CJ" (conditional Japanese starter)
% - "OP" (opener)
% - "CL" (closer)
% - "NS" (non-starter)
% - "EX" (exclamation)
% - "IS" (infix separator)
% - "CM" (combining marks)
%
% All code points of classes "ID" and "CJ" are assigned to a \XeTeXcharclass,
% but for other classes this only occurs when they fall into east Asian width
% type "F", "H" or "W" (full-, half- and wide-width).
%
% The following mappings between Unicode and XeTeX classes occur
% - "ID" and "CJ" are class 1
% - "OP" is class 2
% - "CL", "NS", "EX", "IS" are class 3
% - "CM" is class 256 (ignored)
% as standard: these may be over-ridden by defining \XeTeXcharclass<class>
% as required. (If classes "ID" or "CL" are explicitly set, the other members
% of the same groups above will inherit these values.)
%
% This file does _not_ activate XeTeX's inter-character token mechanism
% (\XeTeXinterchartokenstate is not set) nor does it install any material in
% the inter-character token registers.
%
% Note that this file is separate from the main loader as the data structure
% here may need more refinement at the macro level.
%
% =============================================================================
%
% The data loaded here can currently only be used by XeTeX: check for the
% appropriate primitive.
\ifx\XeTeXcharclass\undefined
  \expandafter\endinput
\fi
% Just in case, check for the e-TeX extensions.
\ifx\eTeXversion\undefined
  \expandafter\endinput
\fi
% This file can be loaded in IniTeX mode so the category codes of |{|, |}| and
% |#| may not be correct. Everything is done in a group so that only the
% settings we want to propagate are made available generally.
\begingroup
  \catcode`\{=1 %
  \catcode`\}=2 %
% Write some basic information to the log.
  \catcode`\^=7 %
  \newlinechar=`\^^J %
  \message{^^J}%
  \message{load-unicode-xetex-classes.tex v1.17 (2023-09-18)^^J}%
  \message{Reading Unicode east Asian character class data^^J}%
% A string version of |#| will be needed to look for comment lines in the
% source. Once that is done proper parsing can begin.
  \catcode`\#=12 %
  \def\hash{#}%
  \catcode`\#=6 %
  \def\firsttoken#1#2\relax{#1}%
  \def\parseunicodedataI#1\relax{%
    \unless\if\hash\firsttoken#1?\relax
      \parseunicodedataII#1\relax
    \fi
  }%
% Both files to be parsed here have potential ranges of code points: find the
% first entry and search for the second.
  \def\parseunicodedataII#1; #2 #3\relax{%
    \parseunicodedataIII#1....\relax{#2}%
  }%
% From plain: may not be defined (yet).
  \def\loop#1\repeat{\def\body{#1}\iterate}%
  \def\iterate{%
    \body
      \let\next\iterate
    \else
      \let\next\relax
    \fi
    \next
  }%
  \let\repeat\fi
% A shared routine for reading the data files: only one part of the parser
% has to be altered.
  \def\storedpar{\par}%
  \def\readandparse#1{%
    \openin0=#1.txt %
% Read two lines from the source file to extract the version information
    \catcode`\#=12 %
    \read0 to \unicodedataline
    \message{\unicodedataline ^^J}%
    \read0 to \unicodedataline
    \message{\unicodedataline ^^J}%
    \loop\unless\ifeof0 %
      \read0 to \unicodedataline
      \unless\ifx\unicodedataline\storedpar
        \expandafter\parseunicodedataI\unicodedataline\relax
      \fi
    \repeat
    \catcode`\#=6 %
    \closein0 %
  }%
% Set up the different line break classes recognised.
  \ifdefined\XeTeXcharclassID
  \else
    \chardef\XeTeXcharclassID=1 %
  \fi
  \ifdefined\XeTeXcharclassCJ
  \else
    \let\XeTeXcharclassCJ\XeTeXcharclassID
  \fi
  \ifdefined\XeTeXcharclassOP
  \else
    \chardef\XeTeXcharclassOP=2 %
  \fi
  \ifdefined\XeTeXcharclassCL
  \else
    \chardef\XeTeXcharclassCL=3 %
  \fi
  \ifdefined\XeTeXcharclassEX
  \else
    \let\XeTeXcharclassEX\XeTeXcharclassCL
  \fi
  \ifdefined\XeTeXcharclassIS
  \else
    \let\XeTeXcharclassIS\XeTeXcharclassCL
  \fi
  \ifdefined\XeTeXcharclassNS
  \else
    \let\XeTeXcharclassNS\XeTeXcharclassCL
  \fi
  \ifdefined\XeTeXcharclassCM
  \else
    \chardef\XeTeXcharclassCM=256 %
  \fi
% Check the line break class and if necessary the east Asian width for the
% current code point. For code points of class |ID| or |CJ| there may be a
% range to set, and these are always recorded. In other cases if the code point
% is one of those we may need to set up then save it for checking against the
% list of east Asian widths.
  \def\ID{ID}%
  \def\CJ{CJ}%
  \def\parseunicodedataIII#1..#2..#3\relax#4{%
    \def\temp{#4}%
    \ifnum 0%
      \ifx\temp\ID 1\fi
      \ifx\temp\CJ 1\fi
      >0 %
      \ifx\relax#2\relax
        \parseunicodedataIV{#1}{#1}{#4}%
      \else
        \parseunicodedataIV{#1}{#2}{#4}%
      \fi
    \else
      \ifcsname XeTeXcharclass#4\endcsname
        \ifx\relax#2\relax
          \expandafter\def\csname LB@\number"#1\endcsname{#4}%
        \else
          \let\savedbody\body
          \count0="#1 %
          \loop
            \unless\ifnum\count0>"#2 %
            \expandafter\def\csname LB@\number\count0 \endcsname{#4}%
            \advance\count0 by 1 %
          \repeat
          \let\body\savedbody
        \fi
      \fi
    \fi
  }%
% As we are inside a loop already, there needs to be a group here to preserve
% the iterator.
  \def\parseunicodedataIV#1#2#3{%
    \begingroup
      \count0="#1 %
      \loop
        \unless\ifnum\count0>"#2 %
          \global\XeTeXcharclass\count0=\csname XeTeXcharclass#3\endcsname
          \advance\count0 by 1 %
      \repeat
    \endgroup
  }%
  \readandparse{LineBreak}%
% For |EastAsianWidth.txt|, action is only needed if the character has width
% |F|, |H| or |W|. Once again there may be a range of characters to handle.
  \def\parseunicodedataIII#1..#2..#3\relax#4{%
    \ifnum 0%
      \if F\firsttoken#4\relax 1\fi
      \if H\firsttoken#4\relax 1\fi
      \if W\firsttoken#4\relax 1\fi
       >0 %
      \ifx\relax#2\relax
        \parseunicodedataIV{"#1}%
      \else
        \begingroup
          \count0="#1 %
          \loop
            \unless\ifnum\count0>"#2 %
              \parseunicodedataIV{\count0}%
              \advance\count0 by 1 %
          \repeat
        \endgroup
      \fi
    \fi
  }%
% Only take action if a line breaking class was previously saved: that will
% map to the correct class number.
  \def\parseunicodedataIV#1{%
    \ifcsname LB@\number#1\endcsname
      \global\XeTeXcharclass#1=
        \csname XeTeXcharclass\csname LB@\number#1\endcsname\endcsname
    \fi
  }%
  \readandparse{EastAsianWidth}%
\endgroup

%<latexrelease>    \endgroup
%<latexrelease>    \global\let\xtxHanGlue\undefined
%<latexrelease>    \global\let\xtxHanSpace\undefined
%<latexrelease>    \global\XeTeXinterchartoks 0 1 = {}
%<latexrelease>    \global\XeTeXinterchartoks 0 2 = {}
%<latexrelease>    \global\XeTeXinterchartoks 0 3 = {}
%<latexrelease>    \global\XeTeXinterchartoks 1 0 = {}
%<latexrelease>    \global\XeTeXinterchartoks 2 0 = {}
%<latexrelease>    \global\XeTeXinterchartoks 3 0 = {}
%<latexrelease>    \global\XeTeXinterchartoks 1 1 = {}
%<latexrelease>    \global\XeTeXinterchartoks 1 2 = {}
%<latexrelease>    \global\XeTeXinterchartoks 1 3 = {}
%<latexrelease>    \global\XeTeXinterchartoks 2 1 = {}
%<latexrelease>    \global\XeTeXinterchartoks 2 2 = {}
%<latexrelease>    \global\XeTeXinterchartoks 2 3 = {}
%<latexrelease>    \global\XeTeXinterchartoks 3 1 = {}
%<latexrelease>    \global\XeTeXinterchartoks 3 2 = {}
%<latexrelease>    \global\XeTeXinterchartoks 3 3 = {}
%<latexrelease>  \fi
%<latexrelease>\EndIncludeInRelease
%<latexrelease>\IncludeInRelease{0000/00/00}%
%<latexrelease>  {\XeTeXintercharclasses}{XeTeX character classes}%
%<latexrelease>  \ifx\XeTeXinterchartoks\undefined
%<latexrelease>  \else
%<latexrelease>   % File load-unicode-xetex-classes.tex
%
% Copyright 2015-2023 The LaTeX Project
%
% It may be distributed and/or modified under the conditions of
% the LaTeX Project Public License (LPPL), either version 1.3c of
% this license or (at your option) any later version. The latest
% version of this license is in the file
% http://www.latex-project.org/lppl.txt.
%
% Issues with this file should be reported at
% https://github.com/latex3/unicode-data
%
% This file parses EastAsianWidth.txt and LineBreak.txt, provided by the
% Unicode Consortium, and when used with XeTeX sets \XeTeXcharclass for
% the following classes of code point:
% - "ID" (ideographic)
% - "CJ" (conditional Japanese starter)
% - "OP" (opener)
% - "CL" (closer)
% - "NS" (non-starter)
% - "EX" (exclamation)
% - "IS" (infix separator)
% - "CM" (combining marks)
%
% All code points of classes "ID" and "CJ" are assigned to a \XeTeXcharclass,
% but for other classes this only occurs when they fall into east Asian width
% type "F", "H" or "W" (full-, half- and wide-width).
%
% The following mappings between Unicode and XeTeX classes occur
% - "ID" and "CJ" are class 1
% - "OP" is class 2
% - "CL", "NS", "EX", "IS" are class 3
% - "CM" is class 256 (ignored)
% as standard: these may be over-ridden by defining \XeTeXcharclass<class>
% as required. (If classes "ID" or "CL" are explicitly set, the other members
% of the same groups above will inherit these values.)
%
% This file does _not_ activate XeTeX's inter-character token mechanism
% (\XeTeXinterchartokenstate is not set) nor does it install any material in
% the inter-character token registers.
%
% Note that this file is separate from the main loader as the data structure
% here may need more refinement at the macro level.
%
% =============================================================================
%
% The data loaded here can currently only be used by XeTeX: check for the
% appropriate primitive.
\ifx\XeTeXcharclass\undefined
  \expandafter\endinput
\fi
% Just in case, check for the e-TeX extensions.
\ifx\eTeXversion\undefined
  \expandafter\endinput
\fi
% This file can be loaded in IniTeX mode so the category codes of |{|, |}| and
% |#| may not be correct. Everything is done in a group so that only the
% settings we want to propagate are made available generally.
\begingroup
  \catcode`\{=1 %
  \catcode`\}=2 %
% Write some basic information to the log.
  \catcode`\^=7 %
  \newlinechar=`\^^J %
  \message{^^J}%
  \message{load-unicode-xetex-classes.tex v1.17 (2023-09-18)^^J}%
  \message{Reading Unicode east Asian character class data^^J}%
% A string version of |#| will be needed to look for comment lines in the
% source. Once that is done proper parsing can begin.
  \catcode`\#=12 %
  \def\hash{#}%
  \catcode`\#=6 %
  \def\firsttoken#1#2\relax{#1}%
  \def\parseunicodedataI#1\relax{%
    \unless\if\hash\firsttoken#1?\relax
      \parseunicodedataII#1\relax
    \fi
  }%
% Both files to be parsed here have potential ranges of code points: find the
% first entry and search for the second.
  \def\parseunicodedataII#1; #2 #3\relax{%
    \parseunicodedataIII#1....\relax{#2}%
  }%
% From plain: may not be defined (yet).
  \def\loop#1\repeat{\def\body{#1}\iterate}%
  \def\iterate{%
    \body
      \let\next\iterate
    \else
      \let\next\relax
    \fi
    \next
  }%
  \let\repeat\fi
% A shared routine for reading the data files: only one part of the parser
% has to be altered.
  \def\storedpar{\par}%
  \def\readandparse#1{%
    \openin0=#1.txt %
% Read two lines from the source file to extract the version information
    \catcode`\#=12 %
    \read0 to \unicodedataline
    \message{\unicodedataline ^^J}%
    \read0 to \unicodedataline
    \message{\unicodedataline ^^J}%
    \loop\unless\ifeof0 %
      \read0 to \unicodedataline
      \unless\ifx\unicodedataline\storedpar
        \expandafter\parseunicodedataI\unicodedataline\relax
      \fi
    \repeat
    \catcode`\#=6 %
    \closein0 %
  }%
% Set up the different line break classes recognised.
  \ifdefined\XeTeXcharclassID
  \else
    \chardef\XeTeXcharclassID=1 %
  \fi
  \ifdefined\XeTeXcharclassCJ
  \else
    \let\XeTeXcharclassCJ\XeTeXcharclassID
  \fi
  \ifdefined\XeTeXcharclassOP
  \else
    \chardef\XeTeXcharclassOP=2 %
  \fi
  \ifdefined\XeTeXcharclassCL
  \else
    \chardef\XeTeXcharclassCL=3 %
  \fi
  \ifdefined\XeTeXcharclassEX
  \else
    \let\XeTeXcharclassEX\XeTeXcharclassCL
  \fi
  \ifdefined\XeTeXcharclassIS
  \else
    \let\XeTeXcharclassIS\XeTeXcharclassCL
  \fi
  \ifdefined\XeTeXcharclassNS
  \else
    \let\XeTeXcharclassNS\XeTeXcharclassCL
  \fi
  \ifdefined\XeTeXcharclassCM
  \else
    \chardef\XeTeXcharclassCM=256 %
  \fi
% Check the line break class and if necessary the east Asian width for the
% current code point. For code points of class |ID| or |CJ| there may be a
% range to set, and these are always recorded. In other cases if the code point
% is one of those we may need to set up then save it for checking against the
% list of east Asian widths.
  \def\ID{ID}%
  \def\CJ{CJ}%
  \def\parseunicodedataIII#1..#2..#3\relax#4{%
    \def\temp{#4}%
    \ifnum 0%
      \ifx\temp\ID 1\fi
      \ifx\temp\CJ 1\fi
      >0 %
      \ifx\relax#2\relax
        \parseunicodedataIV{#1}{#1}{#4}%
      \else
        \parseunicodedataIV{#1}{#2}{#4}%
      \fi
    \else
      \ifcsname XeTeXcharclass#4\endcsname
        \ifx\relax#2\relax
          \expandafter\def\csname LB@\number"#1\endcsname{#4}%
        \else
          \let\savedbody\body
          \count0="#1 %
          \loop
            \unless\ifnum\count0>"#2 %
            \expandafter\def\csname LB@\number\count0 \endcsname{#4}%
            \advance\count0 by 1 %
          \repeat
          \let\body\savedbody
        \fi
      \fi
    \fi
  }%
% As we are inside a loop already, there needs to be a group here to preserve
% the iterator.
  \def\parseunicodedataIV#1#2#3{%
    \begingroup
      \count0="#1 %
      \loop
        \unless\ifnum\count0>"#2 %
          \global\XeTeXcharclass\count0=\csname XeTeXcharclass#3\endcsname
          \advance\count0 by 1 %
      \repeat
    \endgroup
  }%
  \readandparse{LineBreak}%
% For |EastAsianWidth.txt|, action is only needed if the character has width
% |F|, |H| or |W|. Once again there may be a range of characters to handle.
  \def\parseunicodedataIII#1..#2..#3\relax#4{%
    \ifnum 0%
      \if F\firsttoken#4\relax 1\fi
      \if H\firsttoken#4\relax 1\fi
      \if W\firsttoken#4\relax 1\fi
       >0 %
      \ifx\relax#2\relax
        \parseunicodedataIV{"#1}%
      \else
        \begingroup
          \count0="#1 %
          \loop
            \unless\ifnum\count0>"#2 %
              \parseunicodedataIV{\count0}%
              \advance\count0 by 1 %
          \repeat
        \endgroup
      \fi
    \fi
  }%
% Only take action if a line breaking class was previously saved: that will
% map to the correct class number.
  \def\parseunicodedataIV#1{%
    \ifcsname LB@\number#1\endcsname
      \global\XeTeXcharclass#1=
        \csname XeTeXcharclass\csname LB@\number#1\endcsname\endcsname
    \fi
  }%
  \readandparse{EastAsianWidth}%
\endgroup

%<latexrelease>   \gdef\xtxHanGlue{\hskip0pt plus 0.1em\relax}
%<latexrelease>   \gdef\xtxHanSpace{\hskip0.2em plus 0.2em minus 0.1em\relax}
%<latexrelease>   \global\XeTeXinterchartoks 0 1 = {\xtxHanSpace}
%<latexrelease>   \global\XeTeXinterchartoks 0 2 = {\xtxHanSpace}
%<latexrelease>   \global\XeTeXinterchartoks 0 3 = {\nobreak\xtxHanSpace}
%<latexrelease>   \global\XeTeXinterchartoks 1 0 = {\xtxHanSpace}
%<latexrelease>   \global\XeTeXinterchartoks 2 0 = {\nobreak\xtxHanSpace}
%<latexrelease>   \global\XeTeXinterchartoks 3 0 = {\xtxHanSpace}
%<latexrelease>   \global\XeTeXinterchartoks 1 1 = {\xtxHanGlue}
%<latexrelease>   \global\XeTeXinterchartoks 1 2 = {\xtxHanGlue}
%<latexrelease>   \global\XeTeXinterchartoks 1 3 = {\nobreak\xtxHanGlue}
%<latexrelease>   \global\XeTeXinterchartoks 2 1 = {\nobreak\xtxHanGlue}
%<latexrelease>   \global\XeTeXinterchartoks 2 2 = {\nobreak\xtxHanGlue}
%<latexrelease>   \global\XeTeXinterchartoks 2 3 = {\xtxHanGlue}
%<latexrelease>   \global\XeTeXinterchartoks 3 1 = {\xtxHanGlue}
%<latexrelease>   \global\XeTeXinterchartoks 3 2 = {\xtxHanGlue}
%<latexrelease>   \global\XeTeXinterchartoks 3 3 = {\nobreak\xtxHanGlue}
%<latexrelease>  \fi
%<latexrelease>\EndIncludeInRelease
%<*2ekernel>
%    \end{macrocode}
% \changes{v2.0d}{2015/02/03}{Set \cs{lccode} for \texttt{-} with Unicode
%   engines}
% There is one over-ride that makes sense here (see below for the same for
% $8$-bit engines): setting the lccode for |-| to itself.
%    \begin{macrocode}
  \lccode`\- =`\- % default hyphen char
%    \end{macrocode}
% The alternative is that a ``traditional'' engine is in use.
%    \begin{macrocode}
\else
%    \end{macrocode}
% \changes{v1.1b}{1998/05/20}{Set up lccodes before loading
%    hyphenation files: pr/2639}
%    We set things up so that hyphenation files can assume that the
%    default (T1) lccodes are in use (at present this also sets up the
%    uccodes).
%    We temporarily define |\reserved@a| to apply |\reserved@c| to
%    all the numbers in the range of its arguments.
%    \begin{macrocode}
\def\reserved@a#1#2{%
   \@tempcnta#1\relax
   \@tempcntb#2\relax
   \reserved@b
}
\def\reserved@b{%
   \ifnum\@tempcnta>\@tempcntb\else
      \reserved@c\@tempcnta
      \advance\@tempcnta\@ne
      \expandafter\reserved@b
   \fi
}
%    \end{macrocode}
%    Depending on the \TeX{} version, we might not be allowed to do
%    this for non-ASCII characters.
% \changes{v1.0n}{1994/06/09}{For \TeX2, do not set codes for higher
%                   half of character table.}
%    \begin{macrocode}
\def\reserved@c#1{%
   \count@=#1\advance\count@ by -"20
   \uccode#1=\count@
   \lccode#1=#1
}
\reserved@a{`\a}{`\z}
\reserved@a{"A0}{"BC}
\reserved@a{"E0}{"FF}
%    \end{macrocode}
% The upper case characters need their |\uccode| and |\lccode| values
% set, and their |\sfcode| set to 999.
%    \begin{macrocode}
\def\reserved@c#1{%
   \count@=#1\advance\count@ by "20
   \uccode#1=#1
   \lccode#1=\count@
   \sfcode#1=999
}
\reserved@a{`\A}{`\Z}
\reserved@a{"80}{"9C}
\reserved@a{"C0}{"DF}
%    \end{macrocode}
% Well, it would be nice if that were correct, but unfortunately, the
% Cork encoding contains some odd slots whose uccode or lccode isn't
% quite what you'd expect.
%    \begin{macrocode}
\uccode`\^^Y=`\I     % dotless i
\lccode`\^^Y=`\^^Y   % dotless i
\uccode`\^^Z=`\J     % dotless j, ae in OT1
\lccode`\^^Z=`\^^Z   % dotless j, ae in OT1
\lccode`\^^9d=`\i    % dotted I
\uccode`\^^9d=`\^^9d % dotted I
\lccode`\^^9e=`\^^9e % d-bar
\uccode`\^^9e=`\^^d0 % d-bar
%    \end{macrocode}
% Finally here is one that helps hyphenation in the OT1 encoding.
% \changes{v1.0z}{1996/10/31}
%    {Added extra \cs{lcode}, hoping it does no harm in T1 (pr/1969)}
%    \begin{macrocode}
\lccode`\^^[=`\^^[   % oe in OT1
%    \end{macrocode}
%
% And we also set the |\lccode| of |\-| and |\textcompwordmark| so
% that they do not prevent hyphenation in the remainder of the word
% (as suggested by Lars Helstr\"om).
% \changes{v1.1e}{2003/10/13}
%    {Added extra \cs{lccode} for \cs{-} and \cs{textcompwordmark}}
%    \begin{macrocode}
\lccode`\- =`\-   % default hyphen char
\lccode 127=127   % alternate hyphen char
\lccode 23 =23    % textcompwordmark in T1
%    \end{macrocode}
%
% End of the conditional to select either Unicode or T1 encoding defaults.
%    \begin{macrocode}
\fi
%    \end{macrocode}
%
% \changes{v2.2b}{2020/07/08}
%         {Add a last-minute hook for \textsf{expl3}}
% At this stage, we can install any last-minute \textsf{expl3} set-up.
%    \begin{macrocode}
\@expl@finalise@setup@@
\def\@expl@finalise@setup@@{}
%    \end{macrocode}
%
%  This is as good a place as any to active a few Xe\TeX{}-specific
%  settings
%    \begin{macrocode}
\ifx\XeTeXuseglyphmetrics\@undefined
\else
  \XeTeXuseglyphmetrics=1 %
  \XeTeXdashbreakstate=1 %
\fi
%    \end{macrocode}
%
% \subsection{Hyphenation}
%
% \changes{v0.1a}{1994/03/07}{move code here from lhyphen.dtx}
% \changes{v0.1a}{1994/03/07}
%         {use \cs{InputIfFileExists} not \cs{IfFileExists}}
% \changes{v1.0x}{1995/11/01}
%      {(DPC) Switch meaning of \cs{@addtofilelist} for cfg files}%
% The following code will be compiled into the format file. It checks
% for the existence of \texttt{hyphen.cfg} in inputs that file if
% found. Otherwise it inputs \texttt{hyphen.ltx}.  Note that these
% are loaded in \emph{before} the |\catcode|s are set, so local
% hyphenation files can use 8-bit input.
%
% We try to load the customized hyphenation description file.
%    \begin{macrocode}
\InputIfFileExists{hyphen.cfg}
           {\typeout{===========================================^^J%
                      Local configuration file hyphen.cfg used^^J%
                     ===========================================}%
             \def\@addtofilelist##1{\xdef\@filelist{\@filelist,##1}}%
           }
           {\InputIfFileExists{UShyphen.tex}%
   {\message{Loading hyphenation patterns for US english.}%
    \language=0
    \lefthyphenmin=2 \righthyphenmin=3 
    \def\languagename{english}%
    } % disallow x- or -xx breaks
   {\errhelp{The configuration for hyphenation is incorrectly
             installed.^^J%
             If you don't understand this error message you need
             to seek^^Jexpert advice.}%
    \errmessage{OOPS! I can't find any hyphenation patterns for
                US english.^^J \space Think of getting some or the
                latex2e setup will never succeed}\@@end}
\endinput
}
\let\@addtofilelist\@gobble
%    \end{macrocode}
%
% \begin{macro}{\l@nohyphenation}
% \changes{v2.0t}{2017/03/09}{ensure \cs{l@nohyphenation} is defined.}
%    \begin{macrocode}
\ifx\l@nohyphenation \@undefined
  \newlanguage\l@nohyphenation
\fi
%    \end{macrocode}
% \end{macro}
%
%
% \begin{macro}{\document@default@language}
% Default document language. -1 acts as language 0, but used as a flag in |\document|
% to see if it has been set in the preamble.
% \changes{v2.1f}{2018/08/24}{Add to latexrelease (github/68)}
%    \begin{macrocode}
%</2ekernel>
%<*2ekernel|latexrelease>
%<latexrelease>\IncludeInRelease{2017/04/15}%
%<latexrelease>                 {\document@default@language}{Save language for hyphenation}%
\let\document@default@language\m@ne
%</2ekernel|latexrelease>
%<latexrelease>\EndIncludeInRelease
%<latexrelease>\IncludeInRelease{0000/00/00}%
%<latexrelease>                 {\document@default@language}{Save language for hyphenation}%
%<latexrelease>\let\document@default@language\@undefined
%<latexrelease>\EndIncludeInRelease
%<*2ekernel>
%    \end{macrocode}
% \end{macro}
%
%
%
% \subsection{Font loading}
%    Fonts loaded during the formatting process might already have
%    changed the |\font@submax| from |0pt| to something higher.
%    If so, we put out a bold warning.
% \changes{v0.1l}{1994/05/20}{Use new font warning commands}
% \changes{v1.1c}{2000/08/23}{Fix typo in warning}
%    \begin{macrocode}
\ifdim \font@submax >\z@
   \@font@warning{Size substitutions with differences\MessageBreak
                 up to \font@submax\space have occurred.\MessageBreak
                \MessageBreak
                Please check the transcript file
                carefully\MessageBreak
                and redo the format generation if necessary!
                \@gobbletwo}%
   \errhelp{Only stopped, to give you time to
            read the above message.}
   \errmessage{}
%    \end{macrocode}
%    We reset the macro. Otherwise every user will get a warning on
%    every job.
%    \begin{macrocode}
\def\font@submax{0pt}
\fi
%    \end{macrocode}
%
% \subsection{Input encoding}
%
% \changes{v2.1a}{2018/03/25}{default to UTF-8}
% Starting with the 2018 \LaTeX\ release default the inputencoding
% to UTF-8. Unless the format is being used with luatex, xetex, enctex or mltex.
%
% This is done in a way largely compatible with older releases: |utf8.def| is input just as if\\
% |\usepackage[utf8]{inputenc}|\\
% had been used, however rather than input the whole package a minimal core part just enough to support
% loading the UTF-8 encoding files is defined here.
%
% If a document re-specifies UTF-8 this is silently ignored.
%
%    \begin{macrocode}
%</2ekernel>
%<*2ekernel|latexrelease>
%    \end{macrocode}
%
% Check that a classic 8-bit tex engine is being used (LaTeX or PDFLaTeX).
%
%    \begin{macrocode}
%<latexrelease>\IncludeInRelease{2018/04/01}%
%<latexrelease>                 {\UTFviii@invalid}{UTF-8 default}%
%    \end{macrocode}
%
%
% Skip this section in Unicode TeX, or if  MLTeX and EncTeX are enabled.
%    \begin{macrocode}
\ifnum0%
  \ifx\Umathcode\@undefined\else 1\fi
  \ifx\mubyte\@undefined\else 1\fi
  \ifx\charsubdef\@undefined\else 1\fi
  =\z@
%    \end{macrocode}
%
%    \begin{macrocode}
\def\saved@space@catcode{10}
\let\@inpenc@test\relax
\def\IeC{%
  \ifx\protect\@typeset@protect
    \expandafter\@firstofone
  \else
    \noexpand\IeC
  \fi
}
%    \end{macrocode}
%
% Make characters active for UTF-8 input formats
%    \begin{macrocode}
\@tempcnta=1
\loop
  \catcode\@tempcnta=13  %
  \advance\@tempcnta\@ne %
\ifnum\@tempcnta<32      %
\repeat                  %
\catcode0=15  % null
\catcode9=10  % tab
\catcode10=12 % ctrl J
\catcode12=13 % ctrl L
\catcode13=5  % newline
\@tempcnta=128
\loop
  \catcode\@tempcnta=13
  \advance\@tempcnta\@ne
\ifnum\@tempcnta<256
\repeat
%    \end{macrocode}
%
% \begin{macro}{\UseRawInputEncoding}
% \changes{v2.1a}{2018/03/25}{Macro added}
% Reset 8 bit characters to catcode 12 so the input endcoing matches the ``Raw''
% font encoding.
% Useful for special behaviours, or for compatibility with older \LaTeX\ formats.
% \changes{v2.1b}{2018/04/06}{Undo changes to \cs{DeclareFontEncoding@} and
%                             definition of \cs{DeclareUnicodeCharacter}}
% \changes{v2.1c}{2018/04/07}{Undefine \cs{inputencodingname}}
%    \begin{macrocode}
\def\UseRawInputEncoding{%
\let\inputencodingname\@undefined                     % revert
\let\DeclareFontEncoding@\DeclareFontEncoding@saved   % revert
\let\DeclareUnicodeCharacter\@undefined               % revert
\@tempcnta=1
\loop
  \catcode\@tempcnta=15  %
  \advance\@tempcnta\@ne %
\ifnum\@tempcnta<32      %
\repeat                  %
\catcode0=15  % null
\catcode9=10  % tab
\catcode10=12 % ctrl J
\catcode12=13 % ctrl L
\catcode13=5  % newline
\@tempcnta=128
\loop
  \catcode\@tempcnta=12
  \advance\@tempcnta\@ne
\ifnum\@tempcnta<256
\repeat
}
%    \end{macrocode}
% \end{macro}
%
%  \begin{macro}{\DeclareFontEncoding@saved}
%    Saved version of |\DeclareFontEncoding@| before \texttt{utf8.def}
%    modifies it for use in |\UseRawInputEncoding| above.
%    \begin{macrocode}
\let\DeclareFontEncoding@saved\DeclareFontEncoding@
%    \end{macrocode}
%  \end{macro}
%
% \changes{v2.1d}{2018/04/08}{Delay full UTF-8 handling to \cs{everyjob}}
% \changes{v2.18}{2018/05/11}{Make invalid UTF-8 also safe, for legacy filesystem encodings}
%    \begin{macrocode}
\edef\inputencodingname{utf8}%
\input{utf8.def}
\let\UTFviii@undefined@err@@\UTFviii@undefined@err
\let\UTFviii@invalid@err@@\UTFviii@invalid@err
\let\UTFviii@two@octets@@\UTFviii@two@octets
\let\UTFviii@three@octets@@\UTFviii@three@octets
\let\UTFviii@four@octets@@\UTFviii@four@octets
%<2ekernel>\def\UTFviii@undefined@err#1{\@gobble#1}%
%<2ekernel>\let\UTFviii@invalid@err\string
%<2ekernel>\let\UTFviii@two@octets\string
%<2ekernel>\let\UTFviii@three@octets\string
%<2ekernel>\let\UTFviii@four@octets\string
%<2ekernel>\everyjob\expandafter{\the\everyjob
%<2ekernel>\let\UTFviii@undefined@err\UTFviii@undefined@err@@
%<2ekernel>\let\UTFviii@invalid@err\UTFviii@invalid@err@@
%<2ekernel>\let\UTFviii@two@octets\UTFviii@two@octets@@
%<2ekernel>\let\UTFviii@three@octets\UTFviii@three@octets@@
%<2ekernel>\let\UTFviii@four@octets\UTFviii@four@octets@@
%<2ekernel>}
\let\@inpenc@test\@undefined
\let\saved@space@catcode\@undefined
%    \end{macrocode}
%
% For formats not set up for UTF-8 default, set the C0 controls to catcode 15.
%    \begin{macrocode}
\else
\@tempcnta=0
\loop
  \catcode\@tempcnta=15  %
  \advance\@tempcnta\@ne %
\ifnum\@tempcnta<32      %
\repeat                  %
\catcode0=15  % null
\catcode9=10  % tab
\catcode10=12 % ctrl J
\catcode12=13 % ctrl L
\catcode13=5  % newline
%    \end{macrocode}
%
%    \begin{macrocode}
\let\UseRawInputEncoding\relax
%    \end{macrocode}
%
%    This ends the skipped code in Unicode engines:
%    \begin{macrocode}
\fi
%</2ekernel|latexrelease>
%<latexrelease>\EndIncludeInRelease
%<latexrelease>\IncludeInRelease{0000/00/00}%
%<latexrelease>                 {\UTFviii@invalid}{UTF-8 default}%
%    \end{macrocode}
%
%    The first block of commands got only introduced in 2019 but we
%    revert all of Unicode support  in one go not jump to the
%    intermediate version.
%    \begin{macrocode}
%<latexrelease>  \let\UTFviii@two@octets@combine\@undefined
%<latexrelease>  \let\UTFviii@three@octets@combine\@undefined
%<latexrelease>  \let\UTFviii@four@octets@combine\@undefined
%<latexrelease>  \let\UTFviii@two@octets@string\@undefined
%<latexrelease>  \let\UTFviii@three@octets@string\@undefined
%<latexrelease>  \let\UTFviii@four@octets@string\@undefined
%<latexrelease>  \let\UTFviii@two@octets@noexpand\@undefined
%<latexrelease>  \let\UTFviii@three@octets@noexpand\@undefined
%<latexrelease>  \let\UTFviii@four@octets@noexpand\@undefined
%    \end{macrocode}
%
%    \begin{macrocode}
%<latexrelease>\@tempcnta=0
%<latexrelease>\loop
%<latexrelease>  \catcode\@tempcnta=15
%<latexrelease>  \advance\@tempcnta\@ne
%<latexrelease>\ifnum\@tempcnta<32
%<latexrelease>\repeat       %
%<latexrelease>\catcode9=10  % tab
%<latexrelease>\catcode10=12 % ctrl J
%<latexrelease>\catcode12=13 % ctrl L
%<latexrelease>\catcode13=5  % newline
%<latexrelease>\@tempcnta=128
%<latexrelease>\loop
%<latexrelease>\catcode\@tempcnta=12
%<latexrelease>\advance\@tempcnta\@ne
%<latexrelease>\ifnum\@tempcnta<256
%<latexrelease>\repeat
%<latexrelease>\let\IeC\@undefined
%<latexrelease>\def\DeclareFontEncoding@#1#2#3{%
%<latexrelease>  \expandafter
%<latexrelease>  \ifx\csname T@#1\endcsname\relax
%<latexrelease>     \def\cdp@elt{\noexpand\cdp@elt}%
%<latexrelease>     \xdef\cdp@list{\cdp@list\cdp@elt{#1}%
%<latexrelease>                    {\default@family}{\default@series}%
%<latexrelease>                    {\default@shape}}%
%<latexrelease>     \expandafter\let\csname#1-cmd\endcsname\@changed@cmd
%<latexrelease>  \else
%<latexrelease>     \@font@info{Redeclaring font encoding #1}%
%<latexrelease>  \fi
%<latexrelease>  \global\@namedef{T@#1}{#2}%
%<latexrelease>  \global\@namedef{M@#1}{\default@M#3}%
%<latexrelease>  \xdef\LastDeclaredEncoding{#1}%
%<latexrelease>  }
%<latexrelease>  \let\UseRawInputEncoding\@undefined
%<latexrelease>  \let\DeclareFontEncoding@saved\@undefined
%<latexrelease>  \let\inputencodingname\@undefined
%<latexrelease>\EndIncludeInRelease
%    \end{macrocode}
%
%    \begin{macrocode}
%<*2ekernel>
%    \begin{macrocode}
%
% We temporarily define |\reserved@a| to apply |\reserved@c| to all the
% numbers in the range of its arguments.
%    \begin{macrocode}
\def\reserved@a#1#2{%
   \@tempcnta#1\relax
   \@tempcntb#2\relax
   \reserved@b
}
\def\reserved@b{%
   \ifnum\@tempcnta>\@tempcntb\else
      \reserved@c\@tempcnta
      \advance\@tempcnta\@ne
      \expandafter\reserved@b
   \fi
}
%    \end{macrocode}
% \changes{v0.1e}{1994/05/02}{Added setting the special catcodes.}
% \changes{v0.1f}{1994/05/02}{Set the catcode of control-J.}
% Set the special catcodes (although some of these are useless, since an
% error will have occurred if the catcodes have changed).  Note that
% |^^J| has catcode `other' for use in warning messages.
%    \begin{macrocode}
\catcode`\ =10
\catcode`\#=6
\catcode`\$=3
\catcode`\%=14
\catcode`\&=4
\catcode`\\=0
\catcode`\^=7
\catcode`\_=8
\catcode`\{=1
\catcode`\}=2
\catcode`\~=13
\catcode`\@=11
\catcode`\^^I=10
\catcode`\^^J=12
\catcode`\^^L=13
\catcode`\^^M=5
%    \end{macrocode}
% \changes{v0.1e}{1994/05/02}{Added setting the `other' catcodes.}
% Set the `other' catcodes.
%    \begin{macrocode}
\def\reserved@c#1{\catcode#1=12\relax}
\reserved@c{`\!}
\reserved@c{`\"}
\reserved@a{`\'}{`\?}
\reserved@c{`\[}
\reserved@c{`\]}
\reserved@c{`\`}
\reserved@c{`\|}
%    \end{macrocode}
% \changes{v0.1e}{1994/05/02}{Added setting the `letter' catcodes.}
% Set the `letter' catcodes.
%    \begin{macrocode}
\def\reserved@c#1{\catcode#1=11\relax}
\reserved@a{`\A}{`\Z}
\reserved@a{`\a}{`\z}
%    \end{macrocode}
% \changes{v0.1e}{1994/05/02}{Made slot 127 illegal}
% \changes{v1.0n}{1994/11/18}
%         {re-allow slots 127--255}
% All the characters in the range 0--31 and 127--255 are illegal,
% \emph{except} tab (|^^I|), nl (|^^J|), ff (|^^L|) and cr (|^^M|).
%
%
% \subsection{Lccodes and uccodes}
%
% \changes{v1.1b}{1998/05/20}{Set up uc/lccodes after loading
%    hyphenation files: pr/2639}
%    We now again set up the default (T1) uc/lccodes.
%    The lower case characters need their |\uccode| and |\lccode| values
%    set. Some of this is a repeat of the set-up before loading
%    hyphenation files.
%    Depending on the \TeX{} version, we might not be allowed to do
%    this for non-ASCII characters.
% \changes{v1.0n}{1994/06/09}{For \TeX2, do not set codes for higher
%                   half of character table.}
% \changes{v2.0a}{2015/01/03}{Skip resetting codes with Unicode engines}
%   For the Unicode engines (Xe\TeX{} and Lua\TeX{}) there is no need to
%   do any of this: they use hyphenation data which does not alter any
%   of the set up and so this entire block is skipped.
%    \begin{macrocode}
\ifnum 0%
  \ifx\Umathcode\@undefined\else 1\fi
  \ifx\XeTeXmathcode\@undefined\else 1\fi
  >\z@
\else
\def\reserved@c#1{%
   \count@=#1\advance\count@ by -"20
   \uccode#1=\count@
   \lccode#1=#1
}
\reserved@a{`\a}{`\z}
\reserved@a{"A0}{"BC}
\reserved@a{"E0}{"FF}
%    \end{macrocode}
% The upper case characters need their |\uccode| and |\lccode| values
% set, and their |\sfcode| set to 999.
%    \begin{macrocode}
\def\reserved@c#1{%
   \count@=#1\advance\count@ by "20
   \uccode#1=#1
   \lccode#1=\count@
   \sfcode#1=999
}
\reserved@a{`\A}{`\Z}
\reserved@a{"80}{"9C}
\reserved@a{"C0}{"DF}
%    \end{macrocode}
% Well, it would be nice if that were correct, but unfortunately, the
% Cork encoding contains some odd slots whose uccode or lccode isn't
% quite what you'd expect.
%    \begin{macrocode}
\uccode`\^^Y=`\I     % dotless i
\lccode`\^^Y=`\^^Y   % dotless i
\uccode`\^^Z=`\J     % dotless j, ae in OT1
\lccode`\^^Z=`\^^Z   % dotless j, ae in OT1
\lccode`\^^9d=`\i    % dotted I
\uccode`\^^9d=`\^^9d % dotted I
\lccode`\^^9e=`\^^9e % d-bar
\uccode`\^^9e=`\^^d0 % d-bar
%    \end{macrocode}
% Finally here is one that helps hyphenation in the OT1 encoding.
% \changes{v1.0z}{1996/10/31}
%    {Added extra \cs{lcode}, hoping it does no harm in T1 (pr/1969)}
%    \begin{macrocode}
\lccode`\^^[=`\^^[   % oe in OT1
\fi % End of reset block for 8-bit engines
%    \end{macrocode}
%
% \begin{macro}{\MakeUppercase}
% \begin{macro}{\MakeUppercase}
% \begin{macro}{\@uclclist}
%
% \changes{v1.1a}{1997/10/20}{Removed \cs{aa} and \cs{AA} from
%    \cs{@uclclist} as these are macros.}
%
%    And whilst we're doing things with uc/lc tables, here are two
%    commands to upper- and lower-case a string.
%
%    \emph{Note} that this implementation is subject to change!  At
%    the moment we're not providing any way to extend the list of
%    uc/lc commands, since finding a good interface is difficult.
%    These commands have some nasty features, such as uppercasing
%    mathematics, environment names, labels, etc.  A much better
%    long-term solution is to use all-caps fonts, but these aren't
%    generally available.
%    \begin{macrocode}
\DeclareRobustCommand{\MakeUppercase}[1]{{%
      \def\i{I}\def\j{J}%
      \def\reserved@a##1##2{\let##1##2\reserved@a}%
      \expandafter\reserved@a\@uclclist\reserved@b{\reserved@b\@gobble}%
%    \end{macrocode}
%    Tell UTF-8 processing to process chars even though we are in an \cs{protected@edef}.
% \changes{v2.1h}{2019/09/14}{Expand UTF8 chrs when case changing (github/177)}
%    \begin{macrocode}
      \let\UTF@two@octets@noexpand\@empty
      \let\UTF@three@octets@noexpand\@empty
      \let\UTF@four@octets@noexpand\@empty
      \protected@edef\reserved@a{\uppercase{#1}}%
      \reserved@a
   }}
\DeclareRobustCommand{\MakeLowercase}[1]{{%
      \def\reserved@a##1##2{\let##2##1\reserved@a}%
      \expandafter\reserved@a\@uclclist\reserved@b{\reserved@b\@gobble}%
      \let\UTF@two@octets@noexpand\@empty
      \let\UTF@three@octets@noexpand\@empty
      \let\UTF@four@octets@noexpand\@empty
      \protected@edef\reserved@a{\lowercase{#1}}%
      \reserved@a
   }}
%    \end{macrocode}
%
%    \begin{macrocode}
\def\@uclclist{\oe\OE\o\O\ae\AE
      \dh\DH\dj\DJ\l\L\ng\NG\ss\SS\th\TH}
%    \end{macrocode}
%    The above code works, but has the nasty side-effect that if you
%    say something like:
%\begin{verbatim}
%    \markboth{\MakeUppercase\contentsname}
%             {\MakeUppercase\contentsname}
%\end{verbatim}
%    then the uppercasing is only done to the first letter of the
%    contents name, since the mark expands out to:
%\begin{verbatim}
%    \mark{\protect\MakeUppercase Table of Contents}
%         {\protect\MakeUppercase Table of Contents}
%\end{verbatim}
%    In order to get round this, we redefine |\MakeUppercase| and
%    |\MakeLowercase| to grab their argument and brace it.  This is a
%    very low-level hack, and is \emph{not} recommended practice!
%    This is an instance of a general problem that makes it unsafe to
%    grab arguments unbraced, and probably needs a more general
%    solution.  For the moment though, this hack will do:
%    \begin{macrocode}
\protected@edef\MakeUppercase#1{\MakeUppercase{#1}}
\protected@edef\MakeLowercase#1{\MakeLowercase{#1}}
%    \end{macrocode}
% \end{macro}
% \end{macro}
% \end{macro}
%
% \changes{v1.0h}{1994/05/13}{Added output enc stuff}
% \changes{v1.0i}{1994/05/16}{moved output enc stuff to lfonts}
%
% \changes{v0.1a}{1994/03/07}{Add code from the old dump.dtx}
%
% \subsection{Applying Patch files}
% Between major releases, small patches will be distributed in
% files |ltpatch.ltx| which must be added at this point.
% \changes{v1.0m}{1994/06/08}{Add patch file system}
% \changes{v2.0h}{2015/06/23}
%     {set \cs{patch@level} in ltvers rather than in ltfinal/ltpatch}
%
% Patch file code removed.
%    \begin{macrocode}
%\IfFileExists{ltpatch.ltx}
%  {\typeout{=================================^^J%
%             Applying patch file ltpatch.ltx^^J%
%            =================================}
%   \def\fmtversion@topatch{unknown}
%   \input{ltpatch.ltx}
%   \ifx\fmtversion\fmtversion@topatch
%      \ifx\patch@level\@undefined
%        \typeout{^^J^^J^^J%
%         !!!!!!!!!!!!!!!!!!!!!!!!!!!!!!!!!!!!!!!!!!!!!!!!!!!!!!^^J%
%         !! Patch file `ltpatch.ltx' not suitable for this^^J%
%         !! version of LaTeX.^^J^^J%
%         !! Please check if initex found an old patch file:^^J%
%         !! --- if so, rename it or delete it, and redo the^^J%
%         !! initex run.^^J%
%         !!!!!!!!!!!!!!!!!!!!!!!!!!!!!!!!!!!!!!!!!!!!!!!!!!!!!!^^J}%
%        \batchmode \@@end
%      \else
%    \end{macrocode}
% \changes{v1.0q}{1995/04/21}
%         {Allow initial patch level 0}
% \changes{v1.0t}{1995/06/13}
%         {Add patch level string more carefully}
% The code below adds the `patch level' string to the first |\typeout|
% in the startup banner.
%    \begin{macrocode}
%        \def\fmtversion@topatch{0}%
%        \ifx\fmtversion@topatch\patch@level\else
%          \def\reserved@a\typeout##1##2\reserved@a{%
%                 \typeout{##1 patch level \patch@level}##2}
%          \everyjob\expandafter\expandafter\expandafter{%
%             \expandafter\reserved@a\the\everyjob\reserved@a}
%          \let\reserved@a\relax
%          \the\everyjob
%        \fi
%      \fi
%   \else
%      \typeout{^^J^^J^^J%
%    !!!!!!!!!!!!!!!!!!!!!!!!!!!!!!!!!!!!!!!!!!!!!!!!!!!!!!^^J%
%    !! Patch file `ltpatch.ltx' (for version <\fmtversion@topatch>)^^J%
%    !! is not suitable for version <\fmtversion> of LaTeX.^^J^^J%
%    !! Please check if initex found an old patch file:^^J%
%    !! --- if so, rename it or delete it, and redo the^^J%
%    !!     initex run.^^J%
%    !!!!!!!!!!!!!!!!!!!!!!!!!!!!!!!!!!!!!!!!!!!!!!!!!!!!!!^^J}%
%       \batchmode \@@end
%   \fi
%   \let\fmtversion@topatch\relax
%  }{}
%    \end{macrocode}
%
% \changes{v2.2}{2019-10-02}{Load \textsf{ltexpl}}
% \changes{v2.2a}{2020-06-04}{Load \textsf{ltexpl} in \texttt{ltdefns}}
%
% \subsection{Freeing Memory}
%
% \begin{macro}{\reserved@a}
% \begin{macro}{\reserved@b}
% \changes{v1.0v}{1995/10/17}{reset here after the \cs{input} above}
% And just to make sure nobody relies on those definitions of
% |\reserved@b| and friends.
% These macros are reserved for use in the kernel. \emph{Do not use
% them as general scratch macros}.
%    \begin{macrocode}
\let\reserved@a\@filelist
\let\reserved@b=\@undefined
\let\reserved@c=\@undefined
\let\reserved@d=\@undefined
\let\reserved@e=\@undefined
\let\reserved@f=\@undefined
%    \end{macrocode}
% \end{macro}
% \end{macro}
%
% \begin{macro}{\toks}
% \changes{v1.0y}{1996/07/10}
%      {Free up memory from scratch registers /2213}
%    \begin{macrocode}
\toks0{}
\toks2{}
\toks4{}
\toks6{}
\toks8{}
%    \end{macrocode}
% \end{macro}
%
% \begin{macro}{\errhelp}
% \changes{v0.1g}{1994/05/05}{Set error help empty.}
% \changes{v1.1d}{2000/09/01}{Set error help empty at very end
%                             (pr/449 done correctly).}
% Empty the error help message, which may have some rubbish:
%    \begin{macrocode}
\errhelp{}
%    \end{macrocode}
% \end{macro}
%
% \subsection{Initialise file list}
%
% \begin{macro}{\@providesfile}
% \changes{v1.0v}{1995/10/17}{reset macro}
% Initialise for use in the document. During initex a modified version
% has been used which leaves debugging information for |latexbug.tex|.
%    \begin{macrocode}
\def\@providesfile#1[#2]{%
    \wlog{File: #1 #2}%
    \expandafter\xdef\csname ver@#1\endcsname{#2}%
  \endgroup}
%    \end{macrocode}
% \end{macro}
%
% \begin{macro}{\@filelist}
% \changes{v1.0w}{1995/10/19}{Move after \cs{reserved@a} setting:-)}
% \begin{macro}{\@addtofilelist}
% Reset |\@filelist| so files input while making the format are not
% listed. The list built up so far may take up a lot of memory and so
% it is moved to |\reserved@a| where it will be overwritten as soon
% as almost any \LaTeX\ command is issued in a class file.
% However the |latexbug.tex| program will be able to access this
% information and insert it into a bug report.
%    \begin{macrocode}
\let\@filelist\@gobble
\def\@addtofilelist#1{\xdef\@filelist{\@filelist,#1}}%
%    \end{macrocode}
% \end{macro}
% \end{macro}
%
%
%
% \subsection{Do some temporary work for pre-release}
%
%    This is a good place to load code that hasn't yet been
%    integrated into the other files \ldots
%    \begin{macrocode}
%    \end{macrocode}
%
%    \subsection{Some last minute initializations \ldots}
%
%    \begin{macrocode}
%    \end{macrocode}
%
%
% \subsection{Dumping the format}
%    Finally we make |@| into a letter, ensure the format will
% be in the `normal' error mode, and dump everything into the
%    format file.
% \changes{v1.0t}{1995/06/13}
%         {Call \cs{errorstopmode}}
%    \begin{macrocode}
\makeatother
\errorstopmode
\dump
%</2ekernel>
%    \end{macrocode}
%
% \Finale
%
