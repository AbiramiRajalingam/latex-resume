% \iffalse meta-comment
%
% Copyright (C) 1993-2019
% The LaTeX3 Project and any individual authors listed elsewhere
% in this file.
%
% This file is part of the LaTeX base system.
% -------------------------------------------
%
% It may be distributed and/or modified under the
% conditions of the LaTeX Project Public License, either version 1.3c
% of this license or (at your option) any later version.
% The latest version of this license is in
%    https://www.latex-project.org/lppl.txt
% and version 1.3c or later is part of all distributions of LaTeX
% version 2008 or later.
%
% This file has the LPPL maintenance status "maintained".
%
% The list of all files belonging to the LaTeX base distribution is
% given in the file `manifest.txt'. See also `legal.txt' for additional
% information.
%
% The list of derived (unpacked) files belonging to the distribution
% and covered by LPPL is defined by the unpacking scripts (with
% extension .ins) which are part of the distribution.
%
% \fi
%
% \iffalse
%%% From File: lttab.dtx
%<*driver>
% \fi
\ProvidesFile{lttab.dtx}[2018/12/30 v1.1p LaTeX Kernel (Columns)]
% \iffalse
\documentclass{ltxdoc}
\GetFileInfo{lttab.dtx}
\title{\filename}
\date{\filedate}
 \author{%
  Johannes Braams\and
  David Carlisle\and
  Alan Jeffrey\and
  Leslie Lamport\and
  Frank Mittelbach\and
  Chris Rowley\and
  Rainer Sch\"opf}

\begin{document}
 \MaintainedByLaTeXTeam{latex}
 \maketitle
 \DocInput{\filename}
\end{document}
%</driver>
% \fi
%
%
%
% \changes{v1.0l}{1995/05/07}{Use \cs{hb@xt@}}
% \changes{v1.1a}{1995/05/22}{Support autoloading feature}
% \changes{v1.1b}{1995/06/14}{Use \cs{ProvidesFile} in autoload}
% \changes{v1.1d}{1995/10/04}{Modify autoload support}
% \changes{v1.1k}{1998/06/18}{Small addition to documentation}
% \changes{v1.1l}{1998/07/06}{Small correction to documentation}
% \changes{v1.1n}{2015/02/21}{Removed autoload code}
%
% \section{Tabbing, Tabular and Array Environments}
% This section deals with `Lining It Up in Columns'. First the
% |tabbing| environment is defined, and then in second part, |tabular|
% together with its variants, |tabular*| and |array|.
%
% Note that the |tabular| defined here is essentially the original
% \LaTeX~2.09 version, not the extended version described in \emph{The
% \LaTeX\ Companion}. Use the |array| package to obtain the extended
% version.
%
% \StopEventually{}
%
%
% \changes{v1.0a}{1994/03/04}{Initial version, split from latex.dtx}
% \changes{v1.0a}{1994/03/07}{Long lines wrapped to 72 columns}
% \changes{v1.0b}{1994/03/28}{Improve documentation}
% \changes{v1.0c}{1994/05/07}{Removed definition of \cs{+}}
% \changes{v1.0c}{1994/05/07}{Removed surplus braces from
%                             \cs{@ifnextchar} constructs}
% \changes{v1.0f}{1994/05/21}{Use new error commands}
% \changes{v1.0j}{1994/11/17}
%         {\cs{@tempa} to \cs{reserved@a}}
%
% \subsection{tabbing}
%
% \begin{oldcomments}
%
%  \dimen(\@firsttab + i) = distance of tab stop i from left margin
%         0 <= i <= 15 (?).
%
%  \dimen\@firsttab is initialized to \@totalleftmargin, so it starts
%      at the prevailing left margin.
%
%  \@maxtab          = number of highest defined tab register
%                       probably = \@firsttab + 12
%  \@nxttabmar = tab stop number of next line's left margin
%  \@curtabmar = tab stop number of current line's left margin
%  \@curtab    = number of the current tab. At start of line,
%                      it equals \@curtabmar
%  \@hightab   = largest tab number currently defined.
%  \@tabpush   = depth of \pushtab's
%
%  \box\@curline     = contents of current line, excluding left margin
%                       skip, and excluding contents of current field
%  \box\@curfield    = contents of current field
%
%  @rjfield          = switch: T iff the last field of the line should
%                       be right-justified at the right margin.
%
%  \tabbingsep          = distance left by the \' command between the
%                       current position and the field that is
%                       ``left-shifted''.
%
%  UTILITY MACROS
%   \@stopfield  : closes the current field
%   \@addfield   : adds the current field to the current line.
%   \@contfield  : continues the current field
%   \@startfield : begins the next field
%   \@stopline   : closes the current line and outputs it
%   \@startline  : starts the next line
%   \@ifatmargin : an \if that is true iff the current line.
%                  has width zero
%
% \@startline ==
%  BEGIN
%   \@curtabmar :=G \@nxttabmar
%   \@curtab :=G \@curtabmar
%   \box\@curline :=G null
%   \@startfield
%   \strut
%  END
%
% \@stopline ==
%  BEGIN
%   \unskip
%   \@stopfield
%   if @rjfield = T
%     then  @rjfield :=G F
%           \@tempdima := \@totalleftmargin + \linewidth
%           \hb@xt@ \@tempdima{\@itemfudge
%                               \hskip \dimen\@curtabmar
%                               \box\@curline
%                               \hfil
%                               \box\@curfield}
%     else \@addfield
%          \hbox {\@itemfudge
%                 \hskip \dimen\@curtabmar
%                 \box\@curline}
%   fi
%  END
%
% \@startfield ==
%  BEGIN
%    \box\@curfield :=G \hbox {
%  END
%
% \@stopfield ==
%  BEGIN
%     }
%  END
%
% \@contfield ==
%  BEGIN
%   \box\@curfield :=G \hbox { \unhbox\@currfield  %%} brace matching
%  END
% \@addfield ==
%  BEGIN
%   \box\@curline :=G \unbox\@curline * \unbox\@curfield
%  END
%
% \@ifatmargin ==
%  BEGIN
%   if  dim of box\@curline = 0pt  then
%  END
%
%
% \tabbing ==
%  BEGIN
%   \lineskip :=L 0pt
%   \> == \@rtab
%   \< == \@ltab
%   \= == \@settab
%   \+ == \@tabplus
%   \- == \@tabminus
%   \` == \@tabrj
%   \' == \@tablab
%   \\ == BEGIN \@stopline \@startline END
%   \\[DIST] == BEGIN
%                \@stopline \vskip DIST \@startline\ignorespaces END
%   \\* == BEGIN \@stopline \penalty 10000 \@startline END
%   \\*[DIST] == BEGIN \@stopline \penalty 10000 \vskip DIST
%                      \@startline\ignorespaces               END
%   \@hightab :=  \@nxttabmar :=G \@firsttab
%   \@tabpush :=G 0
%   \dimen\@firsttab := \@totalleftmargin
%   @rjfield :=G F
%   \trivlist  \item\relax
%   if @minipage = F then \vskip \parskip fi
%   \box\@tabfbox = \rlap{\indent\the\everypar}
%                          % note: \the\everypar sets @inlabel :=G F
%   \@itemfudge == BEGIN \box\@tabfbox END
%   \@startline
%   \ignorespaces
%  END
%
% \@endtabbing ==
%  BEGIN
%   \@stopline
%   if \@tabpush > 0 then error message: ''unmatched \poptabs'' fi
%   \endtrivlist
%  END
%
% \@rtab ==
%  BEGIN
%   \@stopfield
%   \@addfield
%   if \@curtab < \@hightab
%     then \@curtab :=G \@curtab + 1
%     else error message ``Undefined Tab''   fi
%   \@tempdima := \dimen\@curtab - \dimen\@curtabmar
%                        - width of box \@curline
%   \box\@curline :=G \hbox{\unhbox\@curline + \hskip\@tempdima}
%   \@startfield
%  END
%
% \@settab ==
%  BEGIN
%   \@stopfield
%   \@addfield
%   if \@curtab < \@maxtab
%     then \@curtab :=G \@curtab+1
%     else error message: ``Too many tabs''    fi
%   if \@curtab > \@hightab
%     then \@hightab :=L \@curtab    fi
%   \dimen\@curtab :=L \dimen\@curtabmar + width of \box\@curline
%   \@startfield
%  END
%
% \@ltab ==
%  BEGIN
%   \@ifatmargin
%     then if \@curtabmar > \@firsttab
%            then \@curtab :=G \@curtab - 1
%                 \@curtabmar :=G \@curtabmar - 1
%            else error message ``Too many untabs''      fi
%     else error message ``Left tab in middle of line''
%   fi
%  END
%
% \@tabplus ==
%  BEGIN
%        if  \@nxttabmar < \@hightab
%           then \@nxttabmar :=G \@nxttabmar+1
%           else error message ``Undefined tab''
%        fi
%  END
%
% \@tabminus ==
%  BEGIN
%        if \@nxttabmar > \@firsttab
%           then \@nxttabmar :=G \@nxttabmar-1
%           else error message ``Too many untabs''
%        fi
%  END
%
% \@tabrj ==
%  BEGIN \@stopfield
%        \@addfield
%        @rjfield :=G T
%        \@startfield
%  END
%
% \@tablab ==
%  BEGIN \@stopfield
%      \box\@curline G:= \hbox{\box\@curline %% `G' added 17 Jun 86
%                              \hskip - width of \box\@curfield
%                              \hskip -\tabbingsep
%                              \box\@curfield
%                              \hskip \tabbingsep }
%        \@startfield
%  END
%
% \pushtabs ==
%   BEGIN
%     \@stopfield
%     \@tabpush :=G \@tabpush + 1
%     \begingroup
%     \@contfield
%   END
%
% \poptabs ==
%  BEGIN
%    \@stopfield
%    if \@tabpush > 0
%      then \endgroup
%           \@tabpush :=G \@tabpush - 1
%      else error message: ``Too many \poptabs''
%    fi
%    \@contfield
%  END
%
% \end{oldcomments}
%
% \begin{macro}{\a}
% The accents |\`| , |\'| , and |\=| that have been redefined inside a
% tabbing environment can be called by typing |\a`| , |\a'| , and |\a=|.
% The macro |\a| is defined in |ltoutenc.dtx|.
% \changes{v1.0d}{1994/05/13}
%         {moved to ltoutenc}
% \end{macro}
%
%
% The `2ekernel' code ensures that a |\usepackage{autotabg}| is
% essentially ignored if a `full' format is being used that has
% picture mode already in the format.
%    \begin{macrocode}
%<2ekernel>\expandafter\let\csname ver@autotabg.sty\endcsname\fmtversion
%    \end{macrocode}
%
% \begin{macro}{\@firsttab}
% \begin{macro}{\@maxtab}
% \changes{v1.0c}{1994/05/07}{Changed \cs{@firsttab} to \cs{chardef}}
% \changes{v1.0c}{1994/05/07}{Changed \cs{@maxtab} to \cs{chardef}}
%    \begin{macrocode}
%<*2ekernel>
\newdimen\@gtempa
\chardef\@firsttab=\the\allocationnumber
\newdimen\@gtempa\newdimen\@gtempa\newdimen\@gtempa\newdimen\@gtempa
\newdimen\@gtempa\newdimen\@gtempa\newdimen\@gtempa\newdimen\@gtempa
\newdimen\@gtempa\newdimen\@gtempa\newdimen\@gtempa\newdimen\@gtempa
\newdimen\@gtempa
\chardef\@maxtab=\the\allocationnumber
\dimen\@firsttab=0pt
%    \end{macrocode}
% \end{macro}
% \end{macro}
%
% \begin{macro}{\@nxttabmar}
% \begin{macro}{\@curtabmar}
% \begin{macro}{\@curtab}
% \begin{macro}{\@hightab}
% \begin{macro}{\@tabpush}
%    \begin{macrocode}
\newcount\@nxttabmar
\newcount\@curtabmar
\newcount\@curtab
\newcount\@hightab
\newcount\@tabpush
%    \end{macrocode}
% \end{macro}
% \end{macro}
% \end{macro}
% \end{macro}
% \end{macro}
%
% \begin{macro}{\@curline}
% \begin{macro}{\@curfield}
% \begin{macro}{\@tabfbox}
%    \begin{macrocode}
\newbox\@curline
\newbox\@curfield
\newbox\@tabfbox
%    \end{macrocode}
% \end{macro}
% \end{macro}
% \end{macro}
%
%
% \begin{macro}{\if@rjfield}
%    \begin{macrocode}
\newif\if@rjfield
%    \end{macrocode}
% \end{macro}
%
% \begin{macro}{\@startline}
% \changes{v1.1f}{1995/10/23}{(CAR)Ensure that \cs{@nxttabmar} is never
% larger than \cs{@hightab}}
%    It is, in some sense, an error if the current margin
%    tab setting is higher than the value of |\@hightab| (which is
%    a local variable).  That this is allowed is a fundamental design
%    flaw which is not going to be corrected now.
%    \begin{macrocode}
\def\@startline{%
     \ifnum \@nxttabmar >\@hightab
       \@badtab
       \global\@nxttabmar \@hightab
     \fi
     \global\@curtabmar \@nxttabmar
     \global\@curtab \@curtabmar
     \global\setbox\@curline \hbox {}%
     \@startfield
     \strut}
%    \end{macrocode}
% \end{macro}
%
% \begin{macro}{\@stopline}
%    \begin{macrocode}
\def\@stopline{%
  \unskip
  \@stopfield
  \if@rjfield
    \global\@rjfieldfalse
    \@tempdima\@totalleftmargin
    \advance\@tempdima\linewidth
    \hb@xt@\@tempdima{%
      \@itemfudge\hskip\dimen\@curtabmar
      \box\@curline
      \hfil
      \box\@curfield}%
  \else
    \@addfield
    \hbox{\@itemfudge\hskip\dimen\@curtabmar\box\@curline}%
  \fi}
%    \end{macrocode}
% \end{macro}
%
% \begin{macro}{\@startfield}
% \changes{v1.0d}{1994/05/13}
%         {Colour support}
%    \begin{macrocode}
\def\@startfield{%
  \global\setbox\@curfield\hbox\bgroup\color@begingroup}
%    \end{macrocode}
% \end{macro}
%
% \begin{macro}{\@stopfield}
% \changes{v1.0d}{1994/05/13}
%         {Colour support}
%    \begin{macrocode}
\def\@stopfield{%
  \color@endgroup\egroup}
%    \end{macrocode}
% \end{macro}
%
% \begin{macro}{\@contfield}
% \changes{v1.0d}{1994/05/13}
%         {Colour support}
%    \begin{macrocode}
\def\@contfield{%
  \global\setbox\@curfield\hbox\bgroup\color@begingroup
  \unhbox\@curfield}
%    \end{macrocode}
% \end{macro}
%
% \begin{macro}{\@addfield}
%    \begin{macrocode}
\def\@addfield{\global\setbox\@curline\hbox{\unhbox
     \@curline\unhbox\@curfield}}
%    \end{macrocode}
% \end{macro}
%
% \begin{macro}{\@ifatmargin}
%    \begin{macrocode}
\def\@ifatmargin{\ifdim \wd\@curline =\z@}
%    \end{macrocode}
% \end{macro}
%
% \begin{macro}{\@tabcr}
%    \begin{macrocode}
\def\@tabcr{\@stopline \@ifstar{\penalty \@M \@xtabcr}\@xtabcr}
%    \end{macrocode}
% \end{macro}
%
% \begin{macro}{\@xtabcr}
%    \begin{macrocode}
\def\@xtabcr{\@ifnextchar[\@itabcr{\@startline\ignorespaces}}
%    \end{macrocode}
% \end{macro}
%
% \begin{macro}{\@itabcr}
%    \begin{macrocode}
\def\@itabcr[#1]{\vskip #1\@startline\ignorespaces}
%    \end{macrocode}
%
%    \begin{macrocode}
\DeclareRobustCommand\kill{\@stopfield\@startline\ignorespaces}
%    \end{macrocode}
% \end{macro}
%
% \begin{macro}{\tabbing}
% \changes{v1.1f}{1995/10/23}{(CAR)Make \cs{@hightab} consistently a
% local variable}
% \changes{latex2e}{1993/12/13}{Removed optional argument of \cs{item}}
%    We use |\relax| to prevent |\item| from scanning too far.
%    \begin{macrocode}
\def\tabbing{\lineskip \z@skip\let\>\@rtab\let\<\@ltab\let\=\@settab
     \let\+\@tabplus\let\-\@tabminus\let\`\@tabrj\let\'\@tablab
     \let\\=\@tabcr
     \@hightab\@firsttab
     \global\@nxttabmar\@firsttab
     \dimen\@firsttab\@totalleftmargin
     \global\@tabpush\z@ \global\@rjfieldfalse
     \trivlist \item\relax
     \if@minipage\else\vskip\parskip\fi
%    \end{macrocode}
% \changes{v1.1i}{1996/10/21}{Moved the \cs{indent} so that the
% \cs{everypar} can remove it when necessary; this is needed because
% the code for items in lists has changed (see pr/22111)}
%    \begin{macrocode}
     \setbox\@tabfbox\hbox{%
       \rlap{\hskip\@totalleftmargin\indent\the\everypar}}%
     \def\@itemfudge{\box\@tabfbox}%
     \@startline\ignorespaces}
%    \end{macrocode}
% \end{macro}
%
% \begin{macro}{\endtabbing}
%    \begin{macrocode}
\def\endtabbing{%
  \@stopline\ifnum\@tabpush >\z@ \@badpoptabs \fi\endtrivlist}
%    \end{macrocode}
% \end{macro}
%
% \begin{macro}{\@rtab}
% Omitted |\global| added to |\@rtab| 17 Jun 86
%    \begin{macrocode}
\def\@rtab{\@stopfield\@addfield\ifnum \@curtab<\@hightab
      \global\advance\@curtab \@ne \else\@badtab\fi
      \@tempdima\dimen\@curtab
      \advance\@tempdima -\dimen\@curtabmar
      \advance\@tempdima -\wd\@curline
      \global\setbox\@curline\hbox{\unhbox\@curline\hskip\@tempdima}%
      \@startfield\ignorespaces}
%    \end{macrocode}
% \end{macro}
%
% \begin{macro}{\@settab}
% \changes{v1.1f}{1995/10/23}{(CAR)Ensure that \cs{@hightab} increases
% by at most one}
%    \begin{macrocode}
\def\@settab{\@stopfield\@addfield
  \ifnum \@curtab <\@maxtab
    \ifnum\@curtab =\@hightab
      \advance\@hightab \@ne
    \fi
    \global\advance\@curtab \@ne
  \else
    \@latex@error{Tab overflow}\@ehd
  \fi
  \dimen\@curtab \dimen\@curtabmar
  \advance\dimen\@curtab \wd\@curline
  \@startfield
  \ignorespaces}
%    \end{macrocode}
% \end{macro}
%
% \begin{macro}{\@ltab}
%    \begin{macrocode}
\def\@ltab{\@ifatmargin\ifnum\@curtabmar >\@firsttab
      \global\advance\@curtab \m@ne \global\advance\@curtabmar\m@ne\else
      \@badtab\fi\else
      \@latex@error{\string\<\space in mid line}\@ehd\fi\ignorespaces}
%    \end{macrocode}
% \end{macro}
%
% \begin{macro}{\@tabplus}
%    \begin{macrocode}
\def\@tabplus{%
  \ifnum\@nxttabmar<\@hightab
    \global\advance\@nxttabmar\@ne
  \else
    \@badtab
  \fi
  \ignorespaces}
%    \end{macrocode}
% \end{macro}
%
% \begin{macro}{\@tabminus}
%    \begin{macrocode}
\def\@tabminus{%
  \ifnum\@nxttabmar>\@firsttab
    \global\advance\@nxttabmar\m@ne
  \else
    \@badtab
  \fi
  \ignorespaces}
%    \end{macrocode}
% \end{macro}
%
% \begin{macro}{\@tabrj}
%    \begin{macrocode}
\def\@tabrj{%
  \@stopfield\@addfield\global\@rjfieldtrue\@startfield\ignorespaces}
%    \end{macrocode}
% \end{macro}
%
% \begin{macro}{\@tablab}
% |\setbox\@curline| made |\global| in |\@tablab|. 17 Jun 86
%    \begin{macrocode}
\def\@tablab{%
  \@stopfield
  \global\setbox\@curline\hbox{%
    \box\@curline
    \hskip-\wd\@curfield \hskip-\tabbingsep
    \box\@curfield
    \hskip\tabbingsep}%
  \@startfield
  \ignorespaces}
%    \end{macrocode}
% \end{macro}
%
% \begin{macro}{\pushtabs}
%    \begin{macrocode}
\DeclareRobustCommand\pushtabs{%
  \@stopfield\@addfield\global\advance\@tabpush \@ne \begingroup
       \@contfield}
%    \end{macrocode}
% \end{macro}
%
% \begin{macro}{\poptabs}
% \changes{v1.1f}{1995/10/23}{(CAR)Ensure that \cs{@curtab} is never
% larger than \cs{@hightab}}
%    It is, in some sense, an error if, after the endgroup, the current
%    tab setting is higher than the new value of |\@hightab| (which is
%    a local variable).  That this is allowed is a fundamental design
%    flaw which is not going to be corrected now.
%    \begin{macrocode}
\DeclareRobustCommand\poptabs{\@stopfield\@addfield
  \ifnum \@tabpush >\z@
    \endgroup
    \global\advance\@tabpush \m@ne
    \ifnum \@curtab >\@hightab
      \global \@curtab \@hightab
      \@badtab
    \fi
  \else
    \@badpoptabs
  \fi
  \@contfield}
%    \end{macrocode}
% \end{macro}
%
%
% \begin{macro}{\tabbingsep}
%    \begin{macrocode}
\newdimen\tabbingsep
%    \end{macrocode}
% \end{macro}
%
%
%
% \subsection{array and tabular environments}
%
% \begin{oldcomments}
%
% ARRAY PARAMETERS:
%  \arraycolsep
%       : half the width separating columns in an array environment
%  \tabcolsep
%       : half the width separating columns in a tabular environment
%  \arrayrulewidth
%       : width of rules
%  \doublerulesep
%       : space between adjacent rules in array or tabular
%  \arraystretch
%       : line spacing in array and tabular environments is done by
%         placing a strut in every row of height and depth
%         \arraystretch times the height and depth of the strut
%         produced by an ordinary \strut command.
%
% PREAMBLE:
%  The PREAMBLE argument of an array or tabular environment can
%  contain the following:
%    l,r,c  : indicate where entry is to be placed.
%    |      : for vertical rule
%    @{EXP} : inserts the text EXP in every column.
%              \arraycolsep or \tabcolsep  spacing is suppressed.
%    *{N}{PRE} : equivalent to writing N copies of PRE in the preamble.
%                PRE may contain *{N'}{EXP'} expressions.
%    p{LEN} : makes entry in parbox of width LEN.
%
% SPECIAL ARRAY COMMANDS:
%   \multicolumn{N}{FORMAT}{ITEM} : replaces the next N column
%    items by ITEM, formatted according to FORMAT.
%    FORMAT should contain at most one l,r or c.
%    If it contains none, then ITEM is ignored.
%
%   \vline : draws a vertical line the height of the current row.  May
%            appear in an array element entry.
%   \hline : draws a horizontal line between rows.  Must appear either
%            before the first entry (to appear above the first row) or
%             right after a \\ command.  If followed by another \hline,
%             then adds a \vskip of \doublerulesep.
%
%   \cline{i-j} : draws horizontal lines between rows covering columns
%                 i through j, inclusive.  Multiple commands may follow
%                 one another to provide lines covering several disjoint
%                 columns
%   \extracolsep{WIDTH} : for use inside an @ in the preamble.  Causes
%               a WIDTH space to be added between columns for the rest
%                of the columns.  This is in addition to the ordinary
%                intercolumn space.
%
%  \array ==
%    BEGIN
%      \@acol    == \@arrayacol
%      \@classz  == \@arrayclassz
%      \@classiv == \@arrayclassiv
%      \\        == \@arraycr
%      \@halignto == NULL
%      \@tabarray
%    END
%
%  \endarray{NAME} ==  BEGIN  \crcr }}  END
%
%  \tabular  ==
%    BEGIN
%      \@halignto == NULL
%      \@tabular
%    END
%
%  \tabular*{WIDTH} ==
%    BEGIN
%      \@halignto == to WIDTH
%      \@tabular
%    END
%
%  \@tabular ==
%    BEGIN
%      \leavevmode
%      \hbox { $
%         \@acol    == \@tabacol
%         \@classz  == \@tabclassz
%         \@classiv == \@tabclassiv
%         \\        == \@tabularcr
%         \@tabarray
%    END
%
%  \endtabular == BEGIN \crcr}} $} END
%
%  \@tabarray == if next char = [ then \@array else \@array[c] fi
%
%  \@array[POS]{PREAMBLE} ==
%    BEGIN
%      define \@arstrutbox to make \@arstrut produce strut of height
%        and depth \arraystretch times the height and
%        depth of a normal strut.
%      \@mkpream{PREAMBLE}
%      \@preamble == \halign \@halignto {\tabskip=0pt\@arstrut
%                              eval{\@preamble}\tabskip = 0pt\cr %%}
%      \@startpbox == \@@startpbox
%      \@endpbox == \@@endpbox
%      if POS = t then \vtop
%                 else if POS = b then \vbox
%                                 else \vcenter
%      fi              fi
%     {
%      \par          ==L {} % changed 92/09/18
%      \@sharp       == #
%      \protect      == \relax
%      \lineskip     :=L 0pt
%      \baselineskip :=L 0pt
%      \@preamble
%    END
%
%
%  \@arraycr ==
%   BEGIN
%     $              %% Prevents extra space at end of row's last entry.
%     if next char = [
%      then  \@argarraycr
%      else  $ \cr         %% Needed to balance $
%   END
%
%  \@argarraycr[LENGTH] ==
%   BEGIN
%     $                    %% Needed to balance $ of \@arraycr
%     if LENGTH > 0
%       then  \@tempdima := depth of \@arstrutbox + LENGTH
%             \vrule height 0pt width 0pt depth \@tempdima
%             \cr
%       else  \cr \noalign{\vskip LENGTH}
%   END
%
%  \@tabularcr and \@argtabularcr  same as \@arraycr and \@argarraycr
%  except without the extra $'s.
% \end{oldcomments}
%
%
% \begin{macro}{\extracolsep}
%    \begin{macrocode}
\DeclareRobustCommand\extracolsep[1]{\tabskip #1\relax}
%    \end{macrocode}
% \end{macro}
%
% \begin{macro}{\array}
%    \begin{macrocode}
\def\array{\let\@acol\@arrayacol \let\@classz\@arrayclassz
 \let\@classiv\@arrayclassiv
 \let\\\@arraycr\let\@halignto\@empty\@tabarray}
%    \end{macrocode}
% \end{macro}
%
% \begin{macro}{\endarray}
% \begin{macro}{\endtabular}
% \begin{macro}{\endtabular*}
%    \begin{macrocode}
\def\endarray{\crcr\egroup\egroup}
\def\endtabular{\crcr\egroup\egroup $\egroup}
\expandafter \let \csname endtabular*\endcsname = \endtabular
%    \end{macrocode}
% \end{macro}
% \end{macro}
% \end{macro}
%
% \begin{macro}{\tabular}
%    \begin{macrocode}
\def\tabular{\let\@halignto\@empty\@tabular}
%    \end{macrocode}
% \end{macro}
%
% \begin{macro}{\tabular*}
% \changes{v1.1j}{1998/05/18}{Use \cs{setlength}, so that
%           calc extensions apply.}
%    Note that the change to use |\setlength| slightly alters the
%    timing of the expansion and use of the length in |#1| but this is
%    very unlikely to have any practical effect.
% \changes{latex2e}{1993/08/05}{Replaced \cs{expandafter}\cs{def}
%             by \cs{@namedef}.}
%    \begin{macrocode}
\@namedef{tabular*}#1{%
 \setlength\dimen@{#1}%
   \edef\@halignto{to\the\dimen@}\@tabular}
%    \end{macrocode}
% \end{macro}
%
% \begin{macro}{\@tabular}
%    \begin{macrocode}
\def\@tabular{\leavevmode \hbox \bgroup $\let\@acol\@tabacol
   \let\@classz\@tabclassz
   \let\@classiv\@tabclassiv \let\\\@tabularcr\@tabarray}
%    \end{macrocode}
% \end{macro}
%
% \begin{macro}{\@tabarray}
% RmS 91/11/04 added |\m@th|.
%    \begin{macrocode}
\def\@tabarray{\m@th\@ifnextchar[\@array{\@array[c]}}
%    \end{macrocode}
% \end{macro}
%
% RmS 1993/11/03 changed |\halign| to |\ialign| and removed superfluous
%              |\tabskip| assignment
%
%
% \changes{v1.1i}{1996/10/21}{Moved the code associated with
% \cs{@mkpream} into the group provided by the box, for robustness
% (latex/2183)}
% \begin{macro}{\@array}
%    \begin{macrocode}
\def\@array[#1]#2{%
  \if #1t\vtop \else \if#1b\vbox \else \vcenter \fi\fi
%    \end{macrocode}
% \changes{LaTeX2.09}{1992/09/18}
%     {Changed \cs{par} to \cs{@empty} to avoid starting new row
%               e.g. after \cs{hline}}
%    \begin{macrocode}
  \bgroup
%    \end{macrocode}
% This next bit of code sets up the strut and then builds the halign
% and its preamble according to the specification in the second
% argument.
%
% This code has been moved inside the box.
% A side effect of this has been to expose what was a buglet in the
% previous version: since the |\@arstrut| below is expanded and
% contains an |\ifmmode| then it could produce an unnecessary extra
% box in every row, thus wasting `lots of' main memory.
%    \begin{macrocode}
  \setbox\@arstrutbox\hbox{%
    \vrule \@height\arraystretch\ht\strutbox
           \@depth\arraystretch \dp\strutbox
           \@width\z@}%
  \@mkpream{#2}%
  \edef\@preamble{%
    \ialign \noexpand\@halignto
      \bgroup \@arstrut \@preamble \tabskip\z@skip \cr}%
%    \end{macrocode}
% That is the end of setting up the preamble; now we reset
% things before executing the halign built-up in |\@preamble|.
% The restorations could be done by introducing an extra group,
% thus saving tokens.
% \changes{v1.0k}{1994/12/08}{Add \cs{tabularnewline}}
% \changes{v1.1i}{1996/10/21}{Use \cs{set@typeset@protect}}
%    \begin{macrocode}
  \let\@startpbox\@@startpbox \let\@endpbox\@@endpbox
  \let\tabularnewline\\%
    \let\par\@empty
    \let\@sharp##%
    \set@typeset@protect
    \lineskip\z@skip\baselineskip\z@skip
%    \end{macrocode}
%    If the parsing of the preamble goes wrong there my be some
%    characters left which \TeX{} then tries to typeset, i.e., we
%    would be in horizontal mode. That would produce an endless loop
%    because the |\halign| expects vertical mode thus issues a |\par|
%    but that is a no-op at this point. So we better test this case
%    issue some error message and make a crude recovery by ending that
%    horizontal mode with force.
%    A better fix would be to ensure that we never pick up more than a
%    single character token (not done).
% \changes{v1.1m}{1998/11/13}{Check for hmode to see if something
%     went wrong during parsing (pr/2884)}
%    \begin{macrocode}
    \ifhmode \@preamerr\z@ \@@par\fi
    \@preamble}
%    \end{macrocode}
% \end{macro}
%
%  \begin{macro}{\@arraycr}
% Array version of |\\|.
%    \begin{macrocode}
\def\@arraycr{%
  ${\ifnum0=`}\fi\@ifstar\@xarraycr\@xarraycr}
%    \end{macrocode}
% \end{macro}
%
%  \begin{macro}{\@arraycr}
%    \begin{macrocode}
\def\@xarraycr{\@ifnextchar[\@argarraycr{\ifnum0=`{\fi}${}\cr}}
%    \end{macrocode}
% \end{macro}
%
%  \begin{macro}{\@argarraycr}
%    \begin{macrocode}
\def\@argarraycr[#1]{%
  \ifnum0=`{\fi}${}\ifdim #1>\z@ \@xargarraycr{#1}\else
   \@yargarraycr{#1}\fi}
%    \end{macrocode}
% \end{macro}
%
%  \begin{macro}{\tabularnewline}
% \changes{v1.0i}{1994/11/14}{(DPC) Macro added}
% \changes{v1.0k}{1994/12/08}{(DPC) Made it \cs{relax}}
% Tabular version of |\\|.
%    \begin{macrocode}
\let\tabularnewline\relax
%    \end{macrocode}
% \end{macro}
%
%  \begin{macro}{\@tabularcr}
%    \begin{macrocode}
\def\@tabularcr{%
  {\ifnum0=`}\fi\@ifstar\@xtabularcr\@xtabularcr}
%    \end{macrocode}
% \end{macro}
%
%  \begin{macro}{\@xtabularcr}
%    \begin{macrocode}
\def\@xtabularcr{\@ifnextchar[\@argtabularcr{\ifnum0=`{\fi}\cr}}
%    \end{macrocode}
% \end{macro}
%
%  \begin{macro}{\@argtabularcr}
%    \begin{macrocode}
\def\@argtabularcr[#1]{%
  \ifnum0=`{\fi}%
    \ifdim #1>\z@
      \unskip\@xargarraycr{#1}%
    \else
      \@yargarraycr{#1}%
    \fi}
%    \end{macrocode}
% \end{macro}
%
%  \begin{macro}{\@xargarraycr}
%    \begin{macrocode}
\def\@xargarraycr#1{\@tempdima #1\advance\@tempdima \dp \@arstrutbox
   \vrule \@height\z@ \@depth\@tempdima \@width\z@ \cr}
%    \end{macrocode}
% \end{macro}
%
%  \begin{macro}{\@yargarraycr}
%    \begin{macrocode}
\def\@yargarraycr#1{\cr\noalign{\vskip #1}}
%    \end{macrocode}
% \end{macro}
%
%  \begin{macro}{\multicolumn}
% \begin{oldcomments}
% \multicolumn{NUMBER}{FORMAT}{ITEM} ==
%  BEGIN
%  \multispan{NUMBER}
%  \begingroup
%  \@addamp == null
%  \@mkpream{FORMAT}
%  \@sharp == ITEM
%  \protect == \relax
%  \@startpbox == \@@startpbox
%  \@endpbox == \@@endpbox
%  \@arstrut
%  \@preamble
%  \endgroup
%  END
% \end{oldcomments}
%
% The command |\def\@addamp{}| was removed from |\multicolumn| on
% 6 Dec 86 because it caused embedded array environments not to work.  I
% think that it was included originally to prevent an error message if
% the 2nd argument to the |\multicolumn| command had two column
% specifiers.
%
% 8 Feb 89 --- |\hbox{}| added after |\@preamble| to correct bug that
%            occurred if |\multicolumn| preceded |\\[D]| with |D > 0|,
%            caused by |\\[]| command doing an |\unskip|, which removed
%            |\tabcolsep| glue inserted by |\multicolumn|.
%
%
% \changes{v1.0h}{1994/11/04}{(ASAJ) added \cs{set@typeset@protect}.}
% \changes{v1.1i}{1996/10/21}{Make \cs{multicolumn} long (latex/2180)}
% This has been made long so that, for example, a |p|-column can
% contain multiple paragraphs; maybe the arguments of |@|-expressions
% should also be able to contain multiple paragraphs.
%    \begin{macrocode}
\long\def\multicolumn#1#2#3{\multispan{#1}\begingroup
  \@mkpream{#2}%
  \def\@sharp{#3}\set@typeset@protect
  \let\@startpbox\@@startpbox\let\@endpbox\@@endpbox
  \@arstrut \@preamble\hbox{}\endgroup\ignorespaces}
%    \end{macrocode}
% \end{macro}
%
% \begin{oldcomments}
% Codes for classes and character numbers of array, tabular and
% multicolumn arguments.
%
%    Character     Class       Number
%    ---------     -----       ------
%        c           0           0
%        l           0           1
%        r           0           2
%
%        |           1           -
%        @           2           -
%        p           3           -
%      {@-exp}       4           -
%      {p-arg}       5           -
%
% \@testpach \foo : expands \foo, which should be an array parameter
%           token, and sets \@chclass and \@chnum to its class and
%           number. Uses \@lastchclass to distinguish 4 and 5
%
% Preamble error codes
%    0: 'illegal character'
%    1: 'Missing @-exp'
%    2: 'Missing p-arg'
%
% \@addamp ==
%   BEGIN if @firstamp = true then @firstamp := false
%                             else &                     fi
%   END
%
% \@mkpream TOKENLIST ==
%   BEGIN
%    @firstamp     := T
%    \@lastchclass := 6
%    \@preamble    == null
%    \@sharp       == \relax
%    \protect      == BEGIN \noexpand\protect\noexpand END
%    \@startpbox   == \relax
%    \@endpbox     == \relax
%    \@expast{TOKENLIST}
%    for \@nextchar := expand(\reserved@a)
%      do  \@testpach{\@nextchar}
%          case of \@chclass
%            0 -> \@classz
%            1 -> \@classi
%              ...
%            5 -> \@classv
%          end case
%          \@lastchclass := \@chclass
%      od
%      case of \@lastchclass
%         0 -> \hskip \arraycolsep             % lrc
%         1 ->                                  % |
%         2 -> \@preamerr1 % 'Missing @-exp'    % @
%         3 -> \@preamerr2 % 'Missing p-arg'    % p
%         4 ->                                  % @-exp
%         5 -> \hskip \arraycolsep             % p-exp
%      end case
%   END
%
%  \@arrayclassz ==
%    BEGIN
%      \@preamble := \@preamble *
%               case of \@lastchclass
%                  0 -> \hskip \arraycolsep \@addamp \hskip \arraycolsep
%                  1 -> \@addamp \hskip \arraycolsep
%                  2 ->  % impossible
%                  3 ->  % impossible
%                  4 -> \@addamp
%                  5 -> \hskip \arraycolsep \@addamp \hskip \arraycolsep
%                  6 -> \@addamp \hskip \arraycolsep
%                end case
%              * case of \@chnum
%                   0 -> \hfil$\relax\@sharp$\hfil
%                   1 -> $\relax\@sharp$\hfil
%                   2 -> \hfil$\relax\@sharp$
%                end case
%    END
%
% \@tabclassz == similar to \@arrayclassz
%
% \@classi ==
%  BEGIN
%    \@preamble := \@preamble *
%                  case of \@lastchclass
%                     0 -> \hskip \arraycolsep \@arrayrule
%                     1 -> \hskip \doublerulesep \@arrayrule
%                     2 -> % impossible
%                     3 -> % impossible
%                     4 -> \@arrayrule
%                     5 -> \hskip \arraycolsep \@arrayrule
%                     6 -> \@arrayrule
%                  end case
%  END
%
% \@classii ==
%  BEGIN
%    \@preamble := \@preamble *
%                  case of \@lastchclass
%                     0    ->
%                     1    -> \hskip .5\arrayrulewidth
%                     2    -> % impossible
%                     else ->
%                  end case
%  END
%
% \@classiii ==
%  BEGIN
%    \@preamble := \@preamble *
%               case of \@lastchclass
%                  0 -> \hskip \arraycolsep \@addamp \hskip \arraycolsep
%                  1 -> \@addamp \hskip \arraycolsep
%                  2 -> % impossible
%                  3 -> % impossible
%                  4 -> \@addamp
%                  5 -> \hskip \arraycolsep \@addamp \hskip \arraycolsep
%                  6 -> \@addamp \hskip \arraycolsep
%                end case
%  END
%
% \@arrayclassiv  ==
%      BEGIN  \@preamble := \@preamble * $ \@nextchar$  END
%
% \@tabclassiv   == same as \@arrayclassv except without the $ ... $
%
% \@classv ==
%   BEGIN
%    \@preamble :=
%        \@preamble * \@startpbox{\@nextchar}\ignorespaces\@sharp
%                               \@endpbox
%   END
%
% \@expast{S}:
%  Sets \reserved@a := S with all instances of *{N}{STRING}
%  replaced by N copies of STRING, where N > 0.  An *
%  appearing inside braces is ignored, but *-expressions
%  inside STRING are expanded, so nested *-expressions are
%  handled properly.
%
% \@expast{S} == BEGIN  \@xexpast S *0x\@@  END
%
% \@xexpast S1 *{N}{S2} S3 \@@ ==
%  BEGIN
%    \reserved@a   := S1
%    \@tempcnta := N
%    if \@tempcnta > 0
%      then  while \@tempcnta > 0 do \reserved@a   := \reserved@a S2
%                                   \@tempcnta := \@tempcnta - 1 od
%            \reserved@b == \@xexpast
%      else  \reserved@b == \@xexnoop
%    fi
%    \expandafter \reserved@b \reserved@a S3 \@@
%  END
% \end{oldcomments}
%
%
%  \begin{macro}{\@xexnoop}
%    \begin{macrocode}
\def\@xexnoop #1\@@{}
%    \end{macrocode}
% \end{macro}
%
%  \begin{macro}{\@expast}
%    \begin{macrocode}
\def\@expast#1{\@xexpast #1*0x\@@}
%    \end{macrocode}
% \end{macro}
%
%  \begin{macro}{\@xexpast}
%    \begin{macrocode}
\def\@xexpast#1*#2#3#4\@@{%
  \edef\reserved@a{#1}%
  \@tempcnta#2\relax
  \ifnum\@tempcnta>\z@
    \@whilenum\@tempcnta>\z@\do
       {\edef\reserved@a{\reserved@a#3}\advance\@tempcnta \m@ne}%
    \let\reserved@b\@xexpast
  \else
    \let\reserved@b\@xexnoop
  \fi
  \expandafter\reserved@b\reserved@a #4\@@}
%    \end{macrocode}
% \end{macro}
%
%  \begin{macro}{\if@firstamp}
%  \begin{macro}{\@addamp}
%    \begin{macrocode}
\newif\if@firstamp
%    \end{macrocode}
%
%    \begin{macrocode}
\def\@addamp{%
  \if@firstamp
    \@firstampfalse
  \else
    \edef\@preamble{\@preamble &}%
  \fi}
%    \end{macrocode}
% \end{macro}
% \end{macro}
%
%  \begin{macro}{\@arrayacol}
%  \begin{macro}{\@tabacol}
%  \begin{macro}{\@ampacol}
%  \begin{macro}{\@acolampacol}
%    \begin{macrocode}
\def\@arrayacol{\edef\@preamble{\@preamble \hskip \arraycolsep}}
\def\@tabacol{\edef\@preamble{\@preamble \hskip \tabcolsep}}
\def\@ampacol{\@addamp \@acol}
\def\@acolampacol{\@acol\@addamp\@acol}
%    \end{macrocode}
% \end{macro}
% \end{macro}
% \end{macro}
% \end{macro}
%
%  \begin{macro}{\@mkpream}
% \changes{v1.0h}{1994/11/04}{(ASAJ) Added \cs{@unexpandable@protect}
%    to \cs{@mkpream}.}
%    \begin{macrocode}
\def\@mkpream#1{\@firstamptrue\@lastchclass6
  \let\@preamble\@empty
  \let\protect\@unexpandable@protect
  \let\@sharp\relax
  \let\@startpbox\relax\let\@endpbox\relax
  \@expast{#1}%
  \expandafter\@tfor \expandafter
    \@nextchar \expandafter:\expandafter=\reserved@a\do
       {\@testpach\@nextchar
    \ifcase \@chclass \@classz \or \@classi \or \@classii \or \@classiii
      \or \@classiv \or\@classv \fi\@lastchclass\@chclass}%
  \ifcase \@lastchclass \@acol
      \or \or \@preamerr \@ne\or \@preamerr \tw@\or \or \@acol \fi}
%    \end{macrocode}
% \end{macro}
%
%  \begin{macro}{\@arrayclassz}
%    \begin{macrocode}
\def\@arrayclassz{\ifcase \@lastchclass \@acolampacol \or \@ampacol \or
   \or \or \@addamp \or
   \@acolampacol \or \@firstampfalse \@acol \fi
\edef\@preamble{\@preamble
  \ifcase \@chnum
     \hfil$\relax\@sharp$\hfil \or $\relax\@sharp$\hfil
    \or \hfil$\relax\@sharp$\fi}}
%    \end{macrocode}
% \end{macro}
%
%  \begin{macro}{\@tabclassz}
% RmS 91/08/14 inserted extra braces around entry for NFSS
%    \begin{macrocode}
\def\@tabclassz{%
  \ifcase\@lastchclass
    \@acolampacol
  \or
    \@ampacol
  \or
  \or
  \or
    \@addamp
  \or
    \@acolampacol
  \or
    \@firstampfalse\@acol
  \fi
  \edef\@preamble{%
    \@preamble{%
      \ifcase\@chnum
        \hfil
%    \end{macrocode}
% \changes{v1.1p}{2018/12/30}{Add extra \cs{hskip} to guard against an
%   \cs{unskip} at the start of a c-column cell (gh/102)}
%    \begin{macrocode}
        \hskip1sp%
        \ignorespaces\@sharp\unskip\hfil
      \or
%    \end{macrocode}
% \changes{v1.1g}{1996/04/22}
%     {(DPC) Extra \cs{hskip} keeps tabcolsep in empty columns
%             internal/2122}
% \changes{v1.1h}{1996/06/14}
%     {(DPC) Change both\cs{z@skip} to 1sp for latex/2160}
%    \begin{macrocode}
        \hskip1sp\ignorespaces\@sharp\unskip\hfil
      \or
        \hfil\hskip1sp\ignorespaces\@sharp\unskip
      \fi}}}
%    \end{macrocode}
% \end{macro}
%
%  \begin{macro}{\@classi}
%    \begin{macrocode}
\def\@classi{%
  \ifcase\@lastchclass
    \@acol\@arrayrule
  \or
    \@addtopreamble{\hskip \doublerulesep}\@arrayrule
  \or
  \or
  \or
    \@arrayrule
  \or
    \@acol\@arrayrule
  \or
    \@arrayrule
  \fi}
%    \end{macrocode}
% \end{macro}
%
%  \begin{macro}{\@classii}
%    \begin{macrocode}
\def\@classii{%
  \ifcase\@lastchclass
  \or
    \@addtopreamble{\hskip .5\arrayrulewidth}%
  \fi}
%    \end{macrocode}
% \end{macro}
%
%  \begin{macro}{\@classiii}
%    \begin{macrocode}
\def\@classiii{\ifcase \@lastchclass \@acolampacol \or
   \@addamp\@acol \or
   \or \or \@addamp \or
   \@acolampacol \or \@ampacol \fi}
%    \end{macrocode}
% \end{macro}
%
%  \begin{macro}{\@tabclassiv}
%    \begin{macrocode}
\def\@tabclassiv{\@addtopreamble\@nextchar}
%    \end{macrocode}
% \end{macro}
%
%  \begin{macro}{\@arrayclassiv}
%    \begin{macrocode}
\def\@arrayclassiv{\@addtopreamble{$\@nextchar$}}
%    \end{macrocode}
% \end{macro}
%
%  \begin{macro}{\@classv}
%    \begin{macrocode}
\def\@classv{\@addtopreamble{\@startpbox{\@nextchar}\ignorespaces
\@sharp\@endpbox}}
%    \end{macrocode}
% \end{macro}
%
% \begin{macro}{\@addtopreamble}
%    \begin{macrocode}
\def\@addtopreamble#1{\edef\@preamble{\@preamble #1}}
%    \end{macrocode}
% \end{macro}
%
% \begin{macro}{\@chclass}
% \begin{macro}{\@lastchclass}
% \begin{macro}{\@chnum}
%    \begin{macrocode}
\newcount\@chclass
\newcount\@lastchclass
\newcount\@chnum
%    \end{macrocode}
% \end{macro}
% \end{macro}
% \end{macro}
%
% \begin{macro}{\arraycolsep}
% \begin{macro}{\tabcolsep}
% \begin{macro}{\arrayrulewidth}
% \begin{macro}{\doublerulesep}
%    \begin{macrocode}
\newdimen\arraycolsep
\newdimen\tabcolsep
\newdimen\arrayrulewidth
\newdimen\doublerulesep
%    \end{macrocode}
% \end{macro}
% \end{macro}
% \end{macro}
% \end{macro}
%
% \begin{macro}{\arraystretch}
%    \begin{macrocode}
\def\arraystretch{1}    % Default value.
%    \end{macrocode}
% \end{macro}
%
% \begin{macro}{\@arstrutbox}
% \begin{macro}{\@arstrut}
%    \begin{macrocode}
\newbox\@arstrutbox
%    \end{macrocode}
%
%    \begin{macrocode}
\def\@arstrut{%
  \relax\ifmmode\copy\@arstrutbox\else\unhcopy\@arstrutbox\fi}
%    \end{macrocode}
% \end{macro}
% \end{macro}
%
% \begin{macro}{\@arrayrule}
%    \begin{macrocode}
\def\@arrayrule{\@addtopreamble{\hskip -.5\arrayrulewidth
   \vrule \@width \arrayrulewidth\hskip -.5\arrayrulewidth}}
%    \end{macrocode}
% \end{macro}
%
% \begin{macro}{\@testpatch}
%    \begin{macrocode}
\def\@testpach#1{\@chclass \ifnum \@lastchclass=\tw@ 4 \else
    \ifnum \@lastchclass=3 5 \else
     \z@ \if #1c\@chnum \z@ \else
                              \if #1l\@chnum \@ne \else
                              \if #1r\@chnum \tw@ \else
          \@chclass \if #1|\@ne \else
                    \if #1@\tw@ \else
                    \if #1p3 \else \z@ \@preamerr 0\fi
  \fi  \fi  \fi  \fi  \fi  \fi
\fi}
%    \end{macrocode}
% \end{macro}
%
% \begin{macro}{\hline}
%    \begin{macrocode}
\def\hline{%
  \noalign{\ifnum0=`}\fi\hrule \@height \arrayrulewidth \futurelet
   \reserved@a\@xhline}
%    \end{macrocode}
% \end{macro}
%
% \begin{macro}{\@xhline}
%    \begin{macrocode}
\def\@xhline{\ifx\reserved@a\hline
               \vskip\doublerulesep
%    \end{macrocode}
%    Measure from the middle of the rules.
% \changes{latex2e}{1993/12/16}{Measure from middle of vertical rules}
%    \begin{macrocode}
               \vskip-\arrayrulewidth
             \fi
      \ifnum0=`{\fi}}
%    \end{macrocode}
% \end{macro}
%
% \begin{macro}{\vline}
%    \begin{macrocode}
\def\vline{\vrule \@width \arrayrulewidth}
%    \end{macrocode}
% \end{macro}
%
% \begin{macro}{\cline}
% \changes{v1.1c}{1995/09/14}
%     {(DPC) New implementation}
% \begin{macro}{\@cline}
% The old \LaTeX2.09 implementation of |\cline| used up quite
% a lot of memory and two precious count registers.
% This new (1995/09/14) implementation does not use any count registers.
% It is coded in a way that depends heavily on the definition of
% |\multispan| so that command has been moved here from the file
% |ltplain.dtx|.
%
% These counters are no longer declared.
%\begin{verbatim}
% \newcount\@cla
% \newcount\@clb
%\end{verbatim}
%
%    \begin{macrocode}
\def\cline#1{\@cline#1\@nil}
%    \end{macrocode}
%
% \changes{v1.1e}{1995/10/17}
%     {(DPC) Use \cs{@multicnt}}
%    \begin{macrocode}
\def\@cline#1-#2\@nil{%
  \omit
%    \end{macrocode}
% Use the counter from |\multispan|.
%    \begin{macrocode}
  \@multicnt#1%
  \advance\@multispan\m@ne
  \ifnum\@multicnt=\@ne\@firstofone{&\omit}\fi
  \@multicnt#2%
  \advance\@multicnt-#1%
  \advance\@multispan\@ne
%    \end{macrocode}
% The original had |\unskip| at this point,
% but how could a skip get here ???
%    \begin{macrocode}
  \leaders\hrule\@height\arrayrulewidth\hfill
  \cr
%    \end{macrocode}
% This is back spacing is fairly horrible,
% but it is what happened in the old version\ldots\
% An alternative would be to make |\cline| look ahead for a following
% |\cline| as does |\hline|. This would alter the spacing in existing
% documents so keep the old version in the kernel. Perhaps a package
% should do this differently.
%    \begin{macrocode}
  \noalign{\vskip-\arrayrulewidth}}
%    \end{macrocode}
% \end{macro}
% \end{macro}
%
% \begin{macro}{\mscount}
% The |\mscount| counter is no longer declared, saving a csname and a
% register. It is declared in compatibility mode.
% \end{macro}
%
% \begin{macro}{\multispan}
% \begin{macro}{\@multispan}
% \changes{v1.1e}{1995/10/17}
%     {(DPC) Macro added.}
% \begin{macro}{\sp@n}
% Modify |\multispan| slightly from its plain \TeX\ definition
% to allow more efficient code sharing with |\multicolumn|.
% Also share a count register with |\multiput|.
%    \begin{macrocode}
\def\multispan{\omit\@multispan}
%    \end{macrocode}
%
%    \begin{macrocode}
\def\@multispan#1{%
  \@multicnt#1\relax
  \loop\ifnum\@multicnt>\@ne \sp@n\repeat}
%    \end{macrocode}
%
%    \begin{macrocode}
\def\sp@n{\span\omit\advance\@multicnt\m@ne}
%    \end{macrocode}
%  \end{macro}
%  \end{macro}
%  \end{macro}
%
%
% \begin{macro}{\@startpbox}
% \begin{macro}{\@endpbox}
% Helper macros for `p' columns.
%
% |\@startpbox|\marg{width} \emph{text} |\egroup| is essentially
% |\parbox|\marg{width}\marg{text}
%
% |\@endpbox| is essentially  |\unskip \strut \par \egroup\hfil|
%  (Changed 14 Jan 89) (changed again 1994/05/13)
%
% \changes{v1.0d}{1994/05/03}
%     {Use \cs{@finalstrut} based on depth of \cs{@arstrutbox}}

% \changes{v1.1j}{1998/05/18}{Use \cs{setlength} to set \cs{hsize},
%      so that the changes in the calc package apply here.}
%    \begin{macrocode}
\def\@startpbox#1{\vtop\bgroup \setlength\hsize{#1}\@arrayparboxrestore}
%    \end{macrocode}
%
%    \begin{macrocode}
\def\@endpbox{\@finalstrut\@arstrutbox\par\egroup\hfil}
%    \end{macrocode}
%
% 14 Jan 89: Def of |\@endpbox| changed from\\
%    |\def\@endpbox{\par\vskip\dp\@arstrutbox\egroup\hfil}|\\
% so vertical spacing works out right if the last line of a `p' entry
% has a descender.
% \end{macro}
% \end{macro}
%
% \begin{macro}{\@@startpbox}
% \begin{macro}{\@@endpbox}
%    \begin{macrocode}
\let\@@startpbox=\@startpbox
\let\@@endpbox=\@endpbox
%    \end{macrocode}
% \end{macro}
% \end{macro}
%
%    \begin{macrocode}
%</2ekernel>
%    \end{macrocode}
%
% \Finale
%
