% \iffalse meta-comment
%
%% File: ltmeta.dtx (C) Copyright 2021, 2022
%       Frank Mittelbach, LaTeX Team
%
% It may be distributed and/or modified under the conditions of the
% LaTeX Project Public License (LPPL), either version 1.3c of this
% license or (at your option) any later version.  The latest version
% of this license is in the file
%
%    https://www.latex-project.org/lppl.txt
%
%
%%% From File: ltmeta.dtx
%
%    \begin{macrocode}
\def\ltmetaversion{v1.0b}
\def\ltmetadate{2022/05/18}
%    \end{macrocode}
%<*driver>
\documentclass{l3doc}

% Fixing footnotes in  functions and variables: this should be in l3doc!

\newcommand\fixfootnote[2]{\footnotemark
  \AddToHookNext{env/#1/after}{\footnotetext{#2}}}
\AddToHook{env/function/begin}{\def\footnote{\fixfootnote{function}}}
\AddToHook{env/variable/begin}{\def\footnote{\fixfootnote{variable}}}

\EnableCrossrefs
\CodelineIndex
\begin{document}
  \DocInput{ltmeta.dtx}
\end{document}
%</driver>
%
% \fi
%
% \providecommand\hook[1]{\texttt{#1}}
% \providecommand\env[1]{\texttt{#1}}
%
%
%
% \title{The \texttt{ltmeta.dtx} code\thanks{This file has version
%    \ltmetaversion\ dated \ltmetadate, \copyright\ \LaTeX\
%    Project.}}
% \author{Frank Mittelbach}
%
% \maketitle
%
%
% \begin{abstract}
%    This code defines the \cs{DocumentMetadata} interface.
% \end{abstract}
%
% \tableofcontents
%
%
% \section{Introduction}
%
%
%    In the past there was no dedicated location to declare settings
%    concerning a document as a whole. Settings are placed somewhere
%    in the preamble or with the class options or even with some
%    package options.  For some settings this can be too late, for
%    example the pdf version can no longer be changed if a package has
%    used code which already opened the PDF.
%
%    \cs{DocumentMetadata} as a new command unifies such settings in
%    one place.  It must be used before \cs{documentclass} but can be
%    issued more than once there.
%
%    At the moment most of the code run by \cs{DocumentMetadata} is
%    external to the format and subject to change. This includes the
%    supported key/values.
%
%    For that reason all that happens right now in the format is to
%    look for suitable support files and if found, to redirect the
%    processing to them.
%
% \subsection{\cs{DocumentMetadata}}
%
% \begin{function}{\DocumentMetadata}
% \begin{syntax}
%  \cs{DocumentMetadata}\Arg{key-value list}
% \end{syntax}
% \end{function}
%
%
%
%    The keys defined for \cs{DocumentMetadata}
%    currently allow to set the PDF version, to set the PDF \texttt{/Lang},
%    to uncompress a PDF, to set the language and to declare a few PDF standards
%    and some color profiles.
%
%    \cs{DocumentMetadata} is also used to
%    activate the new PDF management code and it loads
%    a number of required files for the PDF management code.
%    As this forces the loading of the backend files, a backend
%    which can't be detected automatically like |dvipdfmx|,
%    must be set in the first \cs{DocumentMetadata} call (if there is more than one).
%
%    The full set of keys currently supported is documented in \texttt{documentmetadata-support.pdf}
%    for now.
%
%
% \StopEventually{\setlength\IndexMin{200pt}  \PrintIndex  }
%
%
% \section{The Implementation}
%
%
%
%    \begin{macrocode}
%<*2ekernel|latexrelease>
%    \end{macrocode}
%    Not needed yet but \ldots
%    \begin{macrocode}
%\ExplSyntaxOn
%    \end{macrocode}
%
%    \begin{macrocode}
%<latexrelease>\NewModuleRelease{2022/06/01}{ltmeta}
%<latexrelease>                 {Document Metadata handling}
%    \end{macrocode}
%
%    \begin{macrocode}
\let \IfDocumentMetadataTF \@secondoftwo
\protected\def\DocumentMetadata{%
  \InputIfFileExists{documentmetadata-support.ltx}%
     {}%
%    \end{macrocode}
%    The above file is changing \cs{DocumentMetadata} to a suitable
%    definition (or so we hope), so that we can try again --- if not
%    tough.
%
%    If the file can't be found we say so and carry on without it.
%    \begin{macrocode}
     {%
       \@latex@error{No support files for
                    \noexpand\DocumentMetadata found}
         {Is the 'LaTeX-lab' bundle installed?%
         \MessageBreak
         Without it, the declaration is ignored.}%
%    \end{macrocode}
%    No point in trying this more than once if there are several calls
%    in the document.
%    \begin{macrocode}
      \let\DocumentMetadata\@gobble
     }%
  \let \IfDocumentMetadataTF \@firstoftwo
  \DocumentMetadata
}
%    \end{macrocode}
%
%   To allow package and class author to support for document links
%   we provide also the new interface commands of the hyperref package
%   for the creation of targets.
% \changes{v1.0b}{2022/05/17}{Default definition for targets added}
%  \begin{macro}{\MakeLinkTarget}
%  \begin{macro}{\LinkTargetOn}
%  \begin{macro}{\LinkTargetOff}
%  \begin{macro}{\NextLinkTarget}
%    \begin{macrocode}
\NewDocumentCommand\MakeLinkTarget{sO{}m}{%
  \ifvmode
    \special{}%
  \else
    \@savsf\spacefactor
    \smash{}%
    \spacefactor\@savsf
  \fi}
\NewDocumentCommand\LinkTargetOn{}{}
\NewDocumentCommand\LinkTargetOff{}{}
\NewDocumentCommand\NextLinkTarget{m}{}
%    \end{macrocode}
%  \end{macro}
%  \end{macro}
%  \end{macro}
%  \end{macro}
%
%
%    We do not undo \cs{MakeLinkTarget} and friends if we roll back, in
%    case they are used in packages that themselves do not offer
%    rollback. This way a roll forward adds them, but the dummies remain
%    if you roll back and you don't get missing csname errors if they
%    are used.
%    \begin{macrocode}
%<latexrelease>\IncludeInRelease{0000/00/00}{ltmeta}%
%<latexrelease>                 {Undo Document Metadata handling}
%<latexrelease>
%<latexrelease>\let\DocumentMetadata\@undefined
%<latexrelease>
%<latexrelease>\EndModuleRelease
%    \end{macrocode}
%    Again for the future \ldots
%    \begin{macrocode}
%\ExplSyntaxOff
%    \end{macrocode}
%    \begin{macrocode}
%</2ekernel|latexrelease>
%    \end{macrocode}
%
%    Restore module prefix (if any):
%    \begin{macrocode}
%<@@=>
%    \end{macrocode}
%
%
%
%
%
%%%%%%%%%%%%%%%%%%%%%%%%%%%%%%%%%%%%%%%%%%%%%
\endinput
%%%%%%%%%%%%%%%%%%%%%%%%%%%%%%%%%%%%%%%%%%%%%
%
