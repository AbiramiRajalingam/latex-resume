
% textcomp stuff to be added to the kernel (or tuenc.def)

\makeatletter

\input{ts1enc.def}

\ifx\Umathcode\@undefined

\def\textcompsubstdefault{cmr}
\input{ts1cmr.fd}

% in pdftex pick up capital accents from TS1 

\DeclareTextAccentDefault{\capitalcedilla}{TS1}
\DeclareTextAccentDefault{\capitalogonek}{TS1}
\DeclareTextAccentDefault{\capitalgrave}{TS1}
\DeclareTextAccentDefault{\capitalacute}{TS1}
\DeclareTextAccentDefault{\capitalcircumflex}{TS1}
\DeclareTextAccentDefault{\capitaltilde}{TS1}
\DeclareTextAccentDefault{\capitaldieresis}{TS1}
\DeclareTextAccentDefault{\capitalhungarumlaut}{TS1}
\DeclareTextAccentDefault{\capitalring}{TS1}
\DeclareTextAccentDefault{\capitalcaron}{TS1}
\DeclareTextAccentDefault{\capitalbreve}{TS1}
\DeclareTextAccentDefault{\capitalmacron}{TS1}
\DeclareTextAccentDefault{\capitaldotaccent}{TS1}

% use textcomp \oldstylenums if in pdftex

\DeclareRobustCommand\oldstylenums[1]{%
 \begingroup
 \ifmmode
   \mathgroup\symletters #1%
  \else
   \CheckEncodingSubset\@use@text@encoding{TS1}%
       {\PackageWarning{textcomp}%
          {Oldstyle digits unavailable for
           family \f@family.\MessageBreak
           Lining digits used instead}}%
       \tw@{#1}%
   \fi
 \endgroup
}


\else  % the unicode engine case

\def\textcompsubstdefault{lmr}
% This file belongs to the Latin Modern package. The work is released
% under the GUST Font License. See the MANIFEST-Latin-Modern.txt and
% README-Latin-Modern.txt files for the details. For the most recent version of
% this license see http://www.gust.org.pl/fonts/licenses/GUST-FONT-LICENSE.txt
% or http://tug.org/fonts/licenses/GUST-FONT-LICENSE.txt

\ProvidesFile{ts1lmr.fd}[2009/10/30 v1.6 Font defs for Latin Modern]
\DeclareFontFamily{TS1}{lmr}{}
\DeclareFontShape{TS1}{lmr}{m}{n}%
     {<-5.5>    ts1-lmr5     <5.5-6.5> ts1-lmr6
      <6.5-7.5> ts1-lmr7     <7.5-8.5> ts1-lmr8
      <8.5-9.5> ts1-lmr9     <9.5-11>  ts1-lmr10
      <11-15>   ts1-lmr12
      <15-> ts1-lmr17
      }{}
\DeclareFontShape{TS1}{lmr}{m}{sl}%
     {<-8.5>    ts1-lmro8    <8.5-9.5> ts1-lmro9
      <9.5-11>  ts1-lmro10   <11-15>   ts1-lmro12
      <15-> ts1-lmro17
      }{}
\DeclareFontShape{TS1}{lmr}{m}{it}%
     {<-7.5>    ts1-lmri7
      <7.5-8.5> ts1-lmri8    <8.5-9.5> ts1-lmri9
      <9.5-11>  ts1-lmri10   <11->   ts1-lmri12
      }{}
\DeclareFontShape{TS1}{lmr}{m}{sc}%
     {<-> ts1-lmcsc10}{}
\DeclareFontShape{TS1}{lmr}{m}{ui}%
     {<-> ts1-lmu10}{}
%
% Is this the right 'shape'?:
\DeclareFontShape{TS1}{lmr}{m}{scsl}%
     {<-> ts1-lmcsco10}{}
%%%%%%% bold series
\DeclareFontShape{TS1}{lmr}{b}{n}
     {<-> ts1-lmb10}{}
\DeclareFontShape{TS1}{lmr}{b}{sl}
     {<-> ts1-lmbo10}{}
%%%%%%% bold extended series
\DeclareFontShape{TS1}{lmr}{bx}{n}
     {<-5.5>   ts1-lmbx5      <5.5-6.5> ts1-lmbx6
      <6.5-7.5> ts1-lmbx7      <7.5-8.5> ts1-lmbx8
      <8.5-9.5> ts1-lmbx9      <9.5-11>  ts1-lmbx10
      <11->   ts1-lmbx12
      }{}
\DeclareFontShape{TS1}{lmr}{bx}{it}
     {<-> ts1-lmbxi10}{}
\DeclareFontShape{TS1}{lmr}{bx}{sl}
     {<-> ts1-lmbxo10}{}
%%%%%%% Font/shape undefined, therefore substituted
\DeclareFontShape{TS1}{lmr}{b}{it}
     {<->sub * lmr/b/sl}{}
\endinput
%%
%% End of file `ts1lmr.fd'.


% in unicode engines make capital accents normal accents

\def\newtie{\t}
\def\capitaltie{\t}
\def\capitalnewtie{\t}
\def\capitalcedilla{\c}
\def\capitalogonek{\k}
\def\capitalgrave{\`}
\def\capitalacute{\'}
\def\capitalcircumflex{\^}
\def\capitaltilde{\~}
\def\capitaldieresis{\"}
\def\capitalhungarumlaut{\H}
\def\capitalring{\r}
\def\capitalcaron{\v}
\def\capitalbreve{\u}
\def\capitalmacron{\=}
\def\capitaldotaccent{\.}

% alternatively we could use TS1 accents if the subset of the current font is known and
% complete (ie 0) but perhaps that overkill

%\def\tc@check@accentii#1{\CheckEncodingSubset
%    \UseTextAccent{TS1}{\expandafter#1\@gobble}}
%
%\DeclareTextCommandDefault{\capitalcedilla}%
%    {\tc@check@accentii\c1\capitalcedilla}


% in unicode engines \oldstylenum will have the default kernel def
% which then can be overwritten by fontspec

\fi

\def\CheckEncodingSubset#1#2#3#4#5{%
    \ifnum #4>%
      \expandafter\ifx\csname #2:\f@family\endcsname\relax
        0\csname #2:?\endcsname
      \else
        \csname #2:\f@family\endcsname
      \fi
   \relax
   \expandafter\@firstoftwo
  \else
   \expandafter\@secondoftwo
 \fi
  {#1{#2}}{#3}%
  #5%
}

\def\DeclareEncodingSubset#1#2#3{%
   \@ifundefined{#1:#2}%
     {\PackageInfo{textcomp}{Setting #2 sub-encoding to #1/#3}}%
     {\PackageInfo{textcomp}{Changing #2 sub-encoding to #1/#3}}%
     \@namedef{#1:#2}{#3}}

\@onlypreamble\DeclareEncodingSubset

% maybe that should be kernel error now ...
\def\tc@errorwarn{\PackageError}

\def\tc@error#1{%
   \tc@errorwarn{textcomp}%  % should be latex error if general
    {Accent \string#1 not provided by\MessageBreak
     font family \f@family\space
     in TS1 encoding}\@eha
}

\def\tc@subst#1{%
   \tc@errorwarn{textcomp}%  % should be latex error if general
    {Symbol \string#1 not provided by\MessageBreak
     font family \f@family\space
     in TS1 encoding.\MessageBreak Default family used instead}\@eha
  \bgroup\fontfamily\textcompsubstdefault\selectfont#1\egroup
}


\def\tc@fake@euro#1{%
   \leavevmode
   \PackageInfo{textcomp}{Faking \noexpand#1for font family
                          \f@family\MessageBreak in TS1 encoding}%
   \valign{##\cr
      \vfil\hbox to 0.07em{\dimen@\f@size\p@
                           \math@fontsfalse
                           \fontsize{.7\dimen@}\z@\selectfont=\hss}%
      \vfil\cr%
      \hbox{C}\crcr
   }%
}
\def\tc@check@symbol{\CheckEncodingSubset\UseTextSymbol{TS1}\tc@subst}
\def\tc@check@accent{\CheckEncodingSubset\UseTextAccent{TS1}\tc@error}

% for backwards compat we fetch smbols from OMS/OML in OT1/T1
%\def\tc@compatibility@fetch#1#2{%
%  \DeclareTextCommand{#1}{OT1}{\UseTextSymbol{#2}#1}%
%  \DeclareTextCommand{#1}{T1}{\UseTextSymbol{#2}#1}%
%}

\DeclareTextSymbolDefault{\textcapitalcompwordmark}{TS1}
\DeclareTextSymbolDefault{\textascendercompwordmark}{TS1}
\DeclareTextSymbolDefault{\textquotestraightbase}{TS1}
\DeclareTextSymbolDefault{\textquotestraightdblbase}{TS1}
\DeclareTextSymbolDefault{\texttwelveudash}{TS1}
\DeclareTextSymbolDefault{\textthreequartersemdash}{TS1}
\DeclareTextSymbolDefault{\textdollar}{TS1}
\UndeclareTextCommand{\textdollar}  {OT1}            % don't use the OT1 def any longer

\DeclareTextSymbolDefault{\textquotesingle}{TS1}
\DeclareTextSymbolDefault{\textasteriskcentered}{TS1}
%\tc@compatibility@fetch{\textasteriskcentered}{OMS}


\DeclareTextSymbolDefault{\textfractionsolidus}{TS1}
\DeclareTextSymbolDefault{\textminus}{TS1}
\DeclareTextSymbolDefault{\textlbrackdbl}{TS1}
\DeclareTextSymbolDefault{\textrbrackdbl}{TS1}
\DeclareTextSymbolDefault{\textasciigrave}{TS1}
\DeclareTextSymbolDefault{\texttildelow}{TS1}
\DeclareTextSymbolDefault{\textasciibreve}{TS1}
\DeclareTextSymbolDefault{\textasciicaron}{TS1}
\DeclareTextSymbolDefault{\textgravedbl}{TS1}
\DeclareTextSymbolDefault{\textacutedbl}{TS1}

\DeclareTextSymbolDefault{\textdagger}{TS1}
%\tc@compatibility@fetch{\textdagger}{OMS}


\DeclareTextSymbolDefault{\textdaggerdbl}{TS1}
%\tc@compatibility@fetch{\textdaggerdbl}{OMS}

\DeclareTextSymbolDefault{\textbardbl}{TS1}
\DeclareTextSymbolDefault{\textperthousand}{TS1}
\UndeclareTextCommand{\textperthousand}{T1}       % don't use the T1 def

\DeclareTextSymbolDefault{\textbullet}{TS1} 
%\tc@compatibility@fetch{\textbullet}{OMS}

\DeclareTextSymbolDefault{\textcelsius}{TS1}
\DeclareTextSymbolDefault{\textflorin}{TS1}
\DeclareTextSymbolDefault{\texttrademark}{TS1}
\DeclareTextSymbolDefault{\textcent}{TS1}

\DeclareTextSymbolDefault{\textsterling}{TS1}
\UndeclareTextCommand{\textsterling}{OT1}         % don't use the OT1 def any longer


\DeclareTextSymbolDefault{\textyen}{TS1}
\DeclareTextSymbolDefault{\textbrokenbar}{TS1}
\DeclareTextSymbolDefault{\textsection}{TS1}
%\tc@compatibility@fetch{\textsection}{OMS}

\DeclareTextSymbolDefault{\textasciidieresis}{TS1}
\DeclareTextSymbolDefault{\textcopyright}{TS1}
\DeclareTextSymbolDefault{\textordfeminine}{TS1}
\DeclareTextSymbolDefault{\textlnot}{TS1}
\DeclareTextSymbolDefault{\textregistered}{TS1}
\DeclareTextSymbolDefault{\textasciimacron}{TS1}
\DeclareTextSymbolDefault{\textdegree}{TS1}
\DeclareTextSymbolDefault{\textpm}{TS1}
\DeclareTextSymbolDefault{\texttwosuperior}{TS1}
\DeclareTextSymbolDefault{\textthreesuperior}{TS1}
\DeclareTextSymbolDefault{\textasciiacute}{TS1}
\DeclareTextSymbolDefault{\textmu}{TS1}
\DeclareTextSymbolDefault{\textparagraph}{TS1}
%\tc@compatibility@fetch{\textparagraph}{OMS}

\DeclareTextSymbolDefault{\textperiodcentered}{TS1}
%\tc@compatibility@fetch{\textperiodcentered}{OMS}

\DeclareTextSymbolDefault{\textonesuperior}{TS1}
\DeclareTextSymbolDefault{\textordmasculine}{TS1}
\DeclareTextSymbolDefault{\textonequarter}{TS1}
\DeclareTextSymbolDefault{\textonehalf}{TS1}
\DeclareTextSymbolDefault{\textthreequarters}{TS1}
\DeclareTextSymbolDefault{\texttimes}{TS1}
\DeclareTextSymbolDefault{\textdiv}{TS1}
\DeclareTextCommandDefault{\texteuro}
   {\CheckEncodingSubset\UseTextSymbol{TS1}\tc@fake@euro5\texteuro}
\DeclareTextCommandDefault{\textohm}{\tc@check@symbol4\textohm}
\DeclareTextCommandDefault{\textestimated}%
    {\tc@check@symbol3\textestimated}
\DeclareTextCommandDefault{\textcurrency}%
    {\tc@check@symbol3\textcurrency}
\DeclareTextCommandDefault{\capitaltie}%
    {\tc@check@accent2\capitaltie}
\DeclareTextCommandDefault{\newtie}%
    {\tc@check@accent2\newtie}
\DeclareTextCommandDefault{\capitalnewtie}%
    {\tc@check@accent2\capitalnewtie}
\DeclareTextCommandDefault{\textleftarrow}%
    {\tc@check@symbol2\textleftarrow}
\DeclareTextCommandDefault{\textrightarrow}%
    {\tc@check@symbol2\textrightarrow}
\DeclareTextCommandDefault{\textblank}%
    {\tc@check@symbol2\textblank}
\DeclareTextCommandDefault{\textdblhyphen}%
    {\tc@check@symbol2\textdblhyphen}
\DeclareTextCommandDefault{\textzerooldstyle}%
    {\tc@check@symbol2\textzerooldstyle}
\DeclareTextCommandDefault{\textoneoldstyle}%
    {\tc@check@symbol2\textoneoldstyle}
\DeclareTextCommandDefault{\texttwooldstyle}%
    {\tc@check@symbol2\texttwooldstyle}
\DeclareTextCommandDefault{\textthreeoldstyle}%
    {\tc@check@symbol2\textthreeoldstyle}
\DeclareTextCommandDefault{\textfouroldstyle}%
    {\tc@check@symbol2\textfouroldstyle}
\DeclareTextCommandDefault{\textfiveoldstyle}%
    {\tc@check@symbol2\textfiveoldstyle}
\DeclareTextCommandDefault{\textsixoldstyle}%
    {\tc@check@symbol2\textsixoldstyle}
\DeclareTextCommandDefault{\textsevenoldstyle}%
    {\tc@check@symbol2\textsevenoldstyle}
\DeclareTextCommandDefault{\texteightoldstyle}%
    {\tc@check@symbol2\texteightoldstyle}
\DeclareTextCommandDefault{\textnineoldstyle}%
    {\tc@check@symbol2\textnineoldstyle}
\DeclareTextCommandDefault{\textlangle}%
    {\tc@check@symbol2\textlangle}
\DeclareTextCommandDefault{\textrangle}%
    {\tc@check@symbol2\textrangle}
\DeclareTextCommandDefault{\textmho}%
    {\tc@check@symbol2\textmho}
\DeclareTextCommandDefault{\textbigcircle}%
    {\tc@check@symbol2\textbigcircle}
\DeclareTextCommandDefault{\textuparrow}%
    {\tc@check@symbol2\textuparrow}
\DeclareTextCommandDefault{\textdownarrow}%
    {\tc@check@symbol2\textdownarrow}
\DeclareTextCommandDefault{\textborn}%
    {\tc@check@symbol2\textborn}
\DeclareTextCommandDefault{\textdivorced}%
    {\tc@check@symbol2\textdivorced}
\DeclareTextCommandDefault{\textdied}%
    {\tc@check@symbol2\textdied}
\DeclareTextCommandDefault{\textleaf}%
    {\tc@check@symbol2\textleaf}
\DeclareTextCommandDefault{\textmarried}%
    {\tc@check@symbol2\textmarried}
\DeclareTextCommandDefault{\textmusicalnote}%
    {\tc@check@symbol2\textmusicalnote}
\DeclareTextCommandDefault{\textdblhyphenchar}%
    {\tc@check@symbol2\textdblhyphenchar}
\DeclareTextCommandDefault{\textdollaroldstyle}%
    {\tc@check@symbol2\textdollaroldstyle}
\DeclareTextCommandDefault{\textcentoldstyle}%
    {\tc@check@symbol2\textcentoldstyle}
\DeclareTextCommandDefault{\textcolonmonetary}%
    {\tc@check@symbol2\textcolonmonetary}
\DeclareTextCommandDefault{\textwon}%
    {\tc@check@symbol2\textwon}
\DeclareTextCommandDefault{\textnaira}%
    {\tc@check@symbol2\textnaira}
\DeclareTextCommandDefault{\textguarani}%  ₲
    {\tc@check@symbol2\textguarani}
\DeclareTextCommandDefault{\textpeso}%
    {\tc@check@symbol2\textpeso}
\DeclareTextCommandDefault{\textlira}%
    {\tc@check@symbol2\textlira}
\DeclareTextCommandDefault{\textrecipe}%
    {\tc@check@symbol2\textrecipe}
\DeclareTextCommandDefault{\textinterrobang}%
    {\tc@check@symbol2\textinterrobang}
\DeclareTextCommandDefault{\textinterrobangdown}%
    {\tc@check@symbol2\textinterrobangdown}
\DeclareTextCommandDefault{\textdong}%
    {\tc@check@symbol2\textdong}
\DeclareTextCommandDefault{\textpertenthousand}%
    {\tc@check@symbol2\textpertenthousand}
\DeclareTextCommandDefault{\textpilcrow}%
    {\tc@check@symbol2\textpilcrow}
\DeclareTextCommandDefault{\textbaht}%
    {\tc@check@symbol2\textbaht}
\DeclareTextCommandDefault{\textnumero}%
    {\tc@check@symbol2\textnumero}
\DeclareTextCommandDefault{\textdiscount}%
    {\tc@check@symbol2\textdiscount}
\DeclareTextCommandDefault{\textopenbullet}%
    {\tc@check@symbol2\textopenbullet}
\DeclareTextCommandDefault{\textservicemark}%
    {\tc@check@symbol2\textservicemark}
\DeclareTextCommandDefault{\textlquill}%
    {\tc@check@symbol2\textlquill}
\DeclareTextCommandDefault{\textrquill}%
    {\tc@check@symbol2\textrquill}
\DeclareTextCommandDefault{\textcopyleft}%
    {\tc@check@symbol2\textcopyleft}
\DeclareTextCommandDefault{\textcircledP}%
    {\tc@check@symbol2\textcircledP}
\DeclareTextCommandDefault{\textreferencemark}%
    {\tc@check@symbol2\textreferencemark}
\DeclareTextCommandDefault{\textsurd}%
    {\tc@check@symbol2\textsurd}
\DeclareTextCommandDefault{\textcircled}
  {\CheckEncodingSubset\UseTextAccent{TS1}%
    {\UseTextAccent{OMS}}1\textcircled}
\DeclareTextCommandDefault{\t}
  {\CheckEncodingSubset\UseTextAccent{TS1}%
    {\UseTextAccent{OML}}1\t}
    

    
% take these from TS1 still (so probably from LMR)

\DeclareTextSymbol{\textcopyleft}{TS1}{171}
\DeclareTextSymbol{\textdblhyphen}{TS1}{45}
\DeclareTextSymbol{\textdblhyphenchar}{TS1}{127}
\DeclareTextSymbol{\textquotestraightbase}{TS1}{13}
\DeclareTextSymbol{\textquotestraightdblbase}{TS1}{18}
\DeclareTextSymbol{\textleaf}{TS1}{108}
\DeclareTextSymbol{\texttwelveudash}{TS1}{21}
\DeclareTextSymbol{\textthreequartersemdash}{TS1}{22}


\ifx\Umathcode\@undefined
\else

 % not always set up at this point while not fully integrated
\providecommand\UnicodeEncodingName{TU}

\def\add@unicode@accent#1#2{%
  \if\relax\detokenize{#2}\relax^^a0\else#2\fi
  \char#1\relax}
\def\DeclareUnicodeAccent#1#2#3{%
  \DeclareTextCommand{#1}{#2}{\add@unicode@accent{#3}}%
}

% use \oldstylenums for these

\DeclareTextCommand{\textzerooldstyle} \UnicodeEncodingName{\oldstylenums{0}}
\DeclareTextCommand{\textoneoldstyle}  \UnicodeEncodingName{\oldstylenums{1}}
\DeclareTextCommand{\texttwooldstyle}  \UnicodeEncodingName{\oldstylenums{2}}
\DeclareTextCommand{\textthreeoldstyle}\UnicodeEncodingName{\oldstylenums{3}}
\DeclareTextCommand{\textfouroldstyle} \UnicodeEncodingName{\oldstylenums{4}}
\DeclareTextCommand{\textfiveoldstyle} \UnicodeEncodingName{\oldstylenums{5}}
\DeclareTextCommand{\textsixoldstyle}  \UnicodeEncodingName{\oldstylenums{6}}
\DeclareTextCommand{\textsevenoldstyle}\UnicodeEncodingName{\oldstylenums{7}}
\DeclareTextCommand{\texteightoldstyle}\UnicodeEncodingName{\oldstylenums{8}}
\DeclareTextCommand{\textnineoldstyle} \UnicodeEncodingName{\oldstylenums{9}}

% those have Unicode points

\DeclareTextSymbol{\textpilcrow}       \UnicodeEncodingName{"00B6}
\DeclareTextSymbol{\textborn}          \UnicodeEncodingName{"002A}
\DeclareTextSymbol{\textdied}          \UnicodeEncodingName{"2020}
\DeclareTextSymbol{\textlbrackdbl}     \UnicodeEncodingName{"27E6}
\DeclareTextSymbol{\textrbrackdbl}     \UnicodeEncodingName{"27E7}
\DeclareTextSymbol{\textguarani}       \UnicodeEncodingName{"20B2}

% this can't be set up if TU isn't loaded in the test suite config
%
% right now 20DD doesn't exist in LM, so maybe not a good idea at the moment
% maybe test for the slot and do the fallback oalign if it doesn't exist
%
  \expandafter\ifx\csname TU-cmd\endcsname\relax
  \else
    \DeclareUnicodeAccent{\textcircled}    \UnicodeEncodingName{"20DD}
  \fi

\fi

% we could make them point to dollar and cent glyphs in TU

%\DeclareTextSymbol{\textcentoldstyle}            \UnicodeEncodingName{"00A2}
%\DeclareTextSymbol{\textdollaroldstyle}          \UnicodeEncodingName{"0024}


% but I think it is better to pick them up from TS1 even if that usually means LMR fonts
\DeclareTextSymbol{\textdollaroldstyle}{TS1}{138}
\DeclareTextSymbol{\textcentoldstyle}  {TS1}{139}


% we declare the subsets of a good number of fonts in the kernel ...

% the default:
\DeclareEncodingSubset{TS1}{?}{1}

\DeclareEncodingSubset{TS1}{cmr}     {0}
\DeclareEncodingSubset{TS1}{cmss}    {0}
\DeclareEncodingSubset{TS1}{cmtt}    {0}
\DeclareEncodingSubset{TS1}{cmvtt}   {0}
\DeclareEncodingSubset{TS1}{cmbr}    {0}
\DeclareEncodingSubset{TS1}{cmtl}    {0}
\DeclareEncodingSubset{TS1}{ccr}     {0}
\DeclareEncodingSubset{TS1}{ptm}     {4}
\DeclareEncodingSubset{TS1}{pcr}     {4}
\DeclareEncodingSubset{TS1}{phv}     {4}
\DeclareEncodingSubset{TS1}{ppl}     {3}
\DeclareEncodingSubset{TS1}{pag}     {4}
\DeclareEncodingSubset{TS1}{pbk}     {4}
\DeclareEncodingSubset{TS1}{pnc}     {4}
\DeclareEncodingSubset{TS1}{pzc}     {4}
\DeclareEncodingSubset{TS1}{bch}     {4}
\DeclareEncodingSubset{TS1}{put}     {5}
\DeclareEncodingSubset{TS1}{uag}     {5}
\DeclareEncodingSubset{TS1}{ugq}     {5}
\DeclareEncodingSubset{TS1}{ul8}     {4}
\DeclareEncodingSubset{TS1}{ul9}     {4}  % (LuxiSans, one day)
\DeclareEncodingSubset{TS1}{augie}   {5}
\DeclareEncodingSubset{TS1}{dayrom}  {3}
\DeclareEncodingSubset{TS1}{dayroms} {3}
\DeclareEncodingSubset{TS1}{pxr}     {0}
\DeclareEncodingSubset{TS1}{pxss}    {0}
\DeclareEncodingSubset{TS1}{pxtt}    {0}
\DeclareEncodingSubset{TS1}{txr}     {0}
\DeclareEncodingSubset{TS1}{txss}    {0}
\DeclareEncodingSubset{TS1}{txtt}    {0}
\DeclareEncodingSubset{TS1}{lmr}     {0}
\DeclareEncodingSubset{TS1}{lmdh}    {0}
\DeclareEncodingSubset{TS1}{lmss}    {0}
\DeclareEncodingSubset{TS1}{lmssq}   {0}
\DeclareEncodingSubset{TS1}{lmvtt}   {0}
\DeclareEncodingSubset{TS1}{lmtt}    {0}
\DeclareEncodingSubset{TS1}{qhv}     {0}
\DeclareEncodingSubset{TS1}{qag}     {0}
\DeclareEncodingSubset{TS1}{qbk}     {0}
\DeclareEncodingSubset{TS1}{qcr}     {0}
\DeclareEncodingSubset{TS1}{qcs}     {0}
\DeclareEncodingSubset{TS1}{qpl}     {0}
\DeclareEncodingSubset{TS1}{qtm}     {0}
\DeclareEncodingSubset{TS1}{qzc}     {0}
\DeclareEncodingSubset{TS1}{qhvc}    {0}
\DeclareEncodingSubset{TS1}{futs}    {4}
\DeclareEncodingSubset{TS1}{futx}    {4}
\DeclareEncodingSubset{TS1}{futj}    {4}
\DeclareEncodingSubset{TS1}{hlh}     {3}
\DeclareEncodingSubset{TS1}{hls}     {3}
\DeclareEncodingSubset{TS1}{hlst}    {3}
\DeclareEncodingSubset{TS1}{hlct}    {5}
\DeclareEncodingSubset{TS1}{hlx}     {5}
\DeclareEncodingSubset{TS1}{hlce}    {5}
\DeclareEncodingSubset{TS1}{hlcn}    {5}
\DeclareEncodingSubset{TS1}{hlcw}    {5}
\DeclareEncodingSubset{TS1}{hlcf}    {5}
\DeclareEncodingSubset{TS1}{pplx}    {3}
\DeclareEncodingSubset{TS1}{pplj}    {3}
\DeclareEncodingSubset{TS1}{ptmx}    {4}
\DeclareEncodingSubset{TS1}{ptmj}    {4}


% for item labels provide:

\def\labelitemnormalfont{\fontseries\seriesdefault\fontshape\shapedefault\selectfont}

% or

\def\labelitemnormalfont{\normalfont}

% and use in \labelitem... in classes

% footnote symbols already use \normalfont hardwired (and that should
% probably stay like that).


% maybe this should also be available in Unicode engines. There a
% default is not enough since the symbols are defined in TU so one
% would need to undefine them there or perhaps better redefine them
% there.

\def\UseLegacyTextSymbols{%
  \DeclareTextSymbolDefault{\textasteriskcentered}{OMS}%
  \DeclareTextSymbolDefault{\textdagger}{OMS}%
  \DeclareTextSymbolDefault{\textdaggerdbl}{OMS}%
  \DeclareTextSymbolDefault{\textbullet}{OMS}%
  \DeclareTextSymbolDefault{\textsection}{OMS}%
  \DeclareTextSymbolDefault{\textparagraph}{OMS}%
  \DeclareTextSymbolDefault{\textperiodcentered}{OMS}%
}

\makeatother    

