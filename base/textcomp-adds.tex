
% textcomp stuff to be added to the kernel (or tuenc.def)

\makeatletter


% if we don't know what font to use for substituations we default to the ``rm'' family

\def\textcompsubstdefault{\rmsubstdefault}


\ifx\Umathcode\@undefined





% preserve the old definition of \cs{oldstylenums} under a different name

\DeclareRobustCommand\legacyoldstylenums[1]{%
   \begingroup
    \spaceskip\fontdimen\tw@\font
    \usefont{OML}{\rmdefault}{\f@series}{it}%
    \mathgroup\symletters #1%
   \endgroup
}


% and here is the improved one that adjusts depending on surroundings.

\DeclareRobustCommand\oldstylenums[1]{%
 \begingroup
 \ifmmode
   \mathgroup\symletters #1%
  \else
   \CheckEncodingSubset\@use@text@encoding{TS1}\tc@oldstyle2{{#1}}%
 \fi
 \endgroup
}


% this is already defined by nfssaxes, so we can reuse it (but since I
% currently run things individually I define it again). It is needed
% in case the user changed the defaults mid document.

\def\expand@font@defaults{%
  \edef\rmdef@ult{\rmdefault}%
  \edef\sfdef@ult{\sfdefault}%
  \edef\ttdef@ult{\ttdefault}%
  \edef\bfdef@ult{\bfdefault}%
  \edef\mddef@ult{\mddefault}%
  \edef\famdef@ult{\familydefault}%
}


\def\tc@oldstyle#1{%
   \tc@errorwarn
          {Oldstyle digits unavailable for
           family \f@family.\MessageBreak
           Default oldstyle digits used instead}\@eha
  \bgroup
      \expand@font@defaults
      \ifx\f@family\rmdef@ult
         \fontfamily\rmsubstdefault
      \else\ifx\f@family\sfdef@ult
         \fontfamily\sfsubstdefault
      \else\ifx\f@family\ttdef@ult
         \fontfamily\ttsubstdefault
      \else
        \fontfamily\substdefault
      \fi\fi\fi
      \fontencoding{TS1}\selectfont#1%
  \egroup
}



\else  % the unicode engine case


% in unicode engines make capital accents normal accents

\def\capitalacute{\'}
\def\capitalbreve{\u}
\def\capitalcaron{\v}
\def\capitalcedilla{\c}
\def\capitalcircumflex{\^}
\def\capitaldieresis{\"}
\def\capitaldotaccent{\.}
\def\capitalgrave{\`}
\def\capitalhungarumlaut{\H}
\def\capitalmacron{\=}
\def\capitalnewtie{\t}
\def\capitalogonek{\k}
\def\capitalring{\r}
\def\capitaltie{\t}
\def\capitaltilde{\~}
\def\newtie{\t}

% alternatively we could use TS1 accents if the subset of the current font is known and
% complete (ie 0) but perhaps that overkill

%\def\tc@check@accentii#1{\CheckEncodingSubset
%    \UseTextAccent{TS1}{\expandafter#1\@gobble}}
%
%\DeclareTextCommandDefault{\capitalcedilla}%
%    {\tc@check@accentii\c1\capitalcedilla}


% in unicode engines \oldstylenum will have the default kernel def
% which then can be overwritten by fontspec

\fi

\def\CheckEncodingSubset#1#2#3#4#5{%
    \ifnum #4>%
      \expandafter\ifx\csname #2:\f@family\endcsname\relax
        0\csname #2:?\endcsname
      \else
        \csname #2:\f@family\endcsname
      \fi
   \relax
   \expandafter\@firstoftwo    % use TS1 definition
  \else
   \expandafter\@secondoftwo   % use substitute
 \fi
  {#1{#2}}{#3}%
  #5%
}

% \DeclareEncodingSubeset{TS1}{Alegreya-*}{2} defines all std names, i.e.
%
% \DeclareEncodingSubeset{TS1}{Alegreya-LF}{2}
% \DeclareEncodingSubeset{TS1}{Alegreya-OsF}{2}
% \DeclareEncodingSubeset{TS1}{Alegreya-TLF}{2}
% \DeclareEncodingSubeset{TS1}{Alegreya-TOsF}{2}
%
% in one go. If only some are needed then one can define them individually.
% Below is a some primitive coding without error checking.
%

\def\DeclareEncodingSubset#1#2{%
  \DeclareEncodingSubset@aux{#1}#2*\DeclareEncodingSubset@aux
}  

\def\DeclareEncodingSubset@aux#1#2*#3\DeclareEncodingSubset@aux#4{%
  \expandafter\ifx\expandafter X\detokenize{#3}X%
    \@DeclareEncodingSubset{#1}{#2}{#4}%
  \else
    \@DeclareEncodingSubset{#1}{#2LF}{#4}%
    \@DeclareEncodingSubset{#1}{#2TLF}{#4}%
    \@DeclareEncodingSubset{#1}{#2OsF}{#4}%
    \@DeclareEncodingSubset{#1}{#2TOsF}{#4}%
  \fi
}

\def\@DeclareEncodingSubset#1#2#3{%
   \@ifundefined{#1:#2}%
     {\PackageInfo{textcomp}{Setting #2 sub-encoding to #1/#3}}%
     {\PackageInfo{textcomp}{Changing #2 sub-encoding to #1/#3}}%
     \@namedef{#1:#2}{#3}}

% any reason to allow those in the middle of documents?

\@onlypreamble\DeclareEncodingSubset
\@onlypreamble\DeclareEncodingSubset@aux
\@onlypreamble\@DeclareEncodingSubset

% maybe that should be kernel error now ...
%\def\tc@errorwarn{\PackageError}
% or maybe not ...
\def\tc@errorwarn#1#2{\@latex@info{#1}}


% next macro only works when nfssaxes are also loaded. Otherwise it
% will always select \substdefault

\def\tc@subst#1{%
   \tc@errorwarn
    {Symbol \string#1 not provided by\MessageBreak
     font family \f@family\space
     in TS1 encoding.\MessageBreak Default family used instead}\@eha
  \bgroup
      \expand@font@defaults
      \ifx\f@family\rmdef@ult
         \fontfamily\rmsubstdefault
      \else\ifx\f@family\sfdef@ult
         \fontfamily\sfsubstdefault
      \else\ifx\f@family\ttdef@ult
         \fontfamily\ttsubstdefault
      \else
        \fontfamily\substdefault
      \fi\fi\fi
% claim that the default can handle all slots (even if not true) to avoid looping      
      \@namedef{TS1:\f@family}{0}%
      \selectfont#1%
  \egroup
}


\def\tc@fake@euro#1{%
   \leavevmode
   \PackageInfo{textcomp}{Faking \noexpand#1for font family
                          \f@family\MessageBreak in TS1 encoding}%
   \valign{##\cr
      \vfil\hbox to 0.07em{\dimen@\f@size\p@
                           \math@fontsfalse
                           \fontsize{.7\dimen@}\z@\selectfont=\hss}%
      \vfil\cr%
      \hbox{C}\crcr
   }%
}
\def\tc@check@symbol{\CheckEncodingSubset\UseTextSymbol{TS1}\tc@subst}
\def\tc@check@accent#1{\CheckEncodingSubset\UseTextAccent{TS1}{\tc@swap@accent#1}}

\def\tc@swap@accent#1#2{\UseTextAccent{T1}#1}

% for backwards compat we fetch smbols from OMS/OML in OT1/T1
%\def\tc@compatibility@fetch#1#2{%
%  \DeclareTextCommand{#1}{OT1}{\UseTextSymbol{#2}#1}%
%  \DeclareTextCommand{#1}{T1}{\UseTextSymbol{#2}#1}%
%}


% \subsection{Sub-encodings}
%
% 0 is full the TS1 encoding


\DeclareTextSymbolDefault{\textcapitalcompwordmark}{TS1}
\DeclareTextSymbolDefault{\textascendercompwordmark}{TS1}




%------------------ sub-enc 1 (drop things that don't work in lmr)


% 1 is TS1 without \cs{textcircled} (but in LM) 


\DeclareTextCommandDefault{\textcircled}
  {\CheckEncodingSubset\UseTextAccent{TS1}{\UseTextAccent{OMS}}1\textcircled}


% but \t loks ok to me, not sure why that was 1 in textcomp

\DeclareTextCommandDefault{\t}
  {\CheckEncodingSubset\UseTextAccent{TS1}{\UseTextAccent{OML}}2\t}

% even if the captial accents are in a font they usually aren't really
% we mark only a few font 0 even if they have a full set of glyphs.
% instead we replace them explicitly with T1 normal accents


\DeclareTextCommandDefault{\capitalacute}       {\tc@check@accent{\'}2\capitalacute}
\DeclareTextCommandDefault{\capitalbreve}       {\tc@check@accent{\u}2\capitalbreve}
\DeclareTextCommandDefault{\capitalcaron}       {\tc@check@accent{\v}2\capitalcaron}
\DeclareTextCommandDefault{\capitalcircumflex}  {\tc@check@accent{\^}2\capitalcircumflex}
\DeclareTextCommandDefault{\capitaldieresis}    {\tc@check@accent{\"}2\capitaldieresis}
\DeclareTextCommandDefault{\capitaldotaccent}   {\tc@check@accent{\.}2\capitaldotaccent}
\DeclareTextCommandDefault{\capitalgrave}       {\tc@check@accent{\`}2\capitalgrave}
\DeclareTextCommandDefault{\capitalhungarumlaut}{\tc@check@accent{\H}2\capitalhungarumlaut}
\DeclareTextCommandDefault{\capitalmacron}      {\tc@check@accent{\=}2\capitalmacron}
\DeclareTextCommandDefault{\capitalring}        {\tc@check@accent{\r}2\capitalring}
\DeclareTextCommandDefault{\capitaltie}         {\tc@check@accent{\t}2\capitaltie}
\DeclareTextCommandDefault{\capitaltilde}       {\tc@check@accent{\~}2\capitaltilde}

% for \newtie and \capitalnewtie this is actually wrong, they should pick up
% the accent from the substituation font (maybe that should be coded)

\DeclareTextCommandDefault{\newtie}             {\tc@check@accent{\t}2\newtie}
\DeclareTextCommandDefault{\capitalnewtie}      {\tc@check@accent{\t}2\capitalnewtie}


%------------------ sub-enc 2 (majority of new OTF fonts via autoinst)

% next two exists in some fonts (faked?), but we ignore that to get linear subsets

\DeclareTextCommandDefault{\textlbrackdbl}      {\tc@check@symbol2\textlbrackdbl}
\DeclareTextCommandDefault{\textrbrackdbl}      {\tc@check@symbol2\textrbrackdbl}

% old style numerals are again in some fonts but using -OsF etc is the better approach
% to get them so we claim they aren in sub-encoding 2 as that's true for most fonts

\DeclareTextCommandDefault{\texteightoldstyle}  {\tc@check@symbol2\texteightoldstyle}
\DeclareTextCommandDefault{\textfiveoldstyle}   {\tc@check@symbol2\textfiveoldstyle}
\DeclareTextCommandDefault{\textfouroldstyle}   {\tc@check@symbol2\textfouroldstyle}
\DeclareTextCommandDefault{\textnineoldstyle}   {\tc@check@symbol2\textnineoldstyle}
\DeclareTextCommandDefault{\textoneoldstyle}    {\tc@check@symbol2\textoneoldstyle}
\DeclareTextCommandDefault{\textsevenoldstyle}  {\tc@check@symbol2\textsevenoldstyle}
\DeclareTextCommandDefault{\textsixoldstyle}    {\tc@check@symbol2\textsixoldstyle}
\DeclareTextCommandDefault{\textthreeoldstyle}  {\tc@check@symbol2\textthreeoldstyle}
\DeclareTextCommandDefault{\texttwooldstyle}    {\tc@check@symbol2\texttwooldstyle}
\DeclareTextCommandDefault{\textzerooldstyle}   {\tc@check@symbol2\textzerooldstyle}

% those are special to TeX fonts (and for a few older PS fonts
% supported in the virtual fonts), but not any longer in the majority
% of fonts provided through autoinst, so we pretend there aren't
% available in sub-encoding 2 and below.

\DeclareTextCommandDefault{\textacutedbl}       {\tc@check@symbol2\textacutedbl}
\DeclareTextCommandDefault{\textasciiacute}     {\tc@check@symbol2\textasciiacute}
\DeclareTextCommandDefault{\textasciibreve}     {\tc@check@symbol2\textasciibreve}
\DeclareTextCommandDefault{\textasciicaron}     {\tc@check@symbol2\textasciicaron}
\DeclareTextCommandDefault{\textasciidieresis}  {\tc@check@symbol2\textasciidieresis}
\DeclareTextCommandDefault{\textasciigrave}     {\tc@check@symbol2\textasciigrave}
\DeclareTextCommandDefault{\textasciimacron}    {\tc@check@symbol2\textasciimacron}
\DeclareTextCommandDefault{\textgravedbl}       {\tc@check@symbol2\textgravedbl}
\DeclareTextCommandDefault{\texttildelow}       {\tc@check@symbol2\texttildelow}

% none of those are arround other than in CM based fonts

\DeclareTextCommandDefault{\textbaht}           {\tc@check@symbol2\textbaht}
\DeclareTextCommandDefault{\textbigcircle}      {\tc@check@symbol2\textbigcircle}
\DeclareTextCommandDefault{\textborn}           {\tc@check@symbol2\textborn}
\DeclareTextCommandDefault{\textcentoldstyle}   {\tc@check@symbol2\textcentoldstyle}
\DeclareTextCommandDefault{\textcircledP}       {\tc@check@symbol2\textcircledP}
\DeclareTextCommandDefault{\textcopyleft}       {\tc@check@symbol2\textcopyleft}
\DeclareTextCommandDefault{\textdblhyphenchar}  {\tc@check@symbol2\textdblhyphenchar}
\DeclareTextCommandDefault{\textdblhyphen}      {\tc@check@symbol2\textdblhyphen}
\DeclareTextCommandDefault{\textdied}           {\tc@check@symbol2\textdied}
\DeclareTextCommandDefault{\textdiscount}       {\tc@check@symbol2\textdiscount}
\DeclareTextCommandDefault{\textdivorced}       {\tc@check@symbol2\textdivorced}
\DeclareTextCommandDefault{\textdollaroldstyle} {\tc@check@symbol2\textdollaroldstyle}
\DeclareTextCommandDefault{\textguarani}        {\tc@check@symbol2\textguarani}
\DeclareTextCommandDefault{\textleaf}           {\tc@check@symbol2\textleaf}
\DeclareTextCommandDefault{\textlquill}         {\tc@check@symbol2\textlquill}
\DeclareTextCommandDefault{\textmarried}        {\tc@check@symbol2\textmarried}
\DeclareTextCommandDefault{\textmho}            {\tc@check@symbol2\textmho}
\DeclareTextCommandDefault{\textmusicalnote}    {\tc@check@symbol2\textmusicalnote}
\DeclareTextCommandDefault{\textnaira}          {\tc@check@symbol2\textnaira}
\DeclareTextCommandDefault{\textopenbullet}     {\tc@check@symbol2\textopenbullet}
\DeclareTextCommandDefault{\textpeso}           {\tc@check@symbol2\textpeso}
\DeclareTextCommandDefault{\textpilcrow}        {\tc@check@symbol2\textpilcrow}
\DeclareTextCommandDefault{\textrecipe}         {\tc@check@symbol2\textrecipe}
\DeclareTextCommandDefault{\textreferencemark}  {\tc@check@symbol2\textreferencemark}
\DeclareTextCommandDefault{\textrquill}         {\tc@check@symbol2\textrquill}
\DeclareTextCommandDefault{\textservicemark}    {\tc@check@symbol2\textservicemark}
\DeclareTextCommandDefault{\textsurd}           {\tc@check@symbol2\textsurd}

% we have a choice here: in T1 there is definition
%  for \cs{textpertenthousand} making the symbol up from \%
%  and \verb=\char 24= (twice) but in many fonts that char doesn't
%  exist and the slot is reused for random ligatures. So better not
%  use it because often it is wrong.  But pointing to TS1 is also not
%  great as only a few fonts have it as a real symbol, so we get a
%  substitution to CM or LM.

%  Alternatively we could just state that the symbol is unavailable in
%  those fonts. For now I substitute.

\DeclareTextCommandDefault{\textpertenthousand} {\tc@check@symbol2\textpertenthousand}
\UndeclareTextCommand{\textpertenthousand}{T1}

%------------------ sub-enc 3 

\DeclareTextCommandDefault{\textlangle}         {\tc@check@symbol3\textlangle}
\DeclareTextCommandDefault{\textrangle}         {\tc@check@symbol3\textrangle}



%------------------ sub-enc 4 


\DeclareTextCommandDefault{\textcolonmonetary}  {\tc@check@symbol4\textcolonmonetary}
\DeclareTextCommandDefault{\textdong}           {\tc@check@symbol4\textdong}
\DeclareTextCommandDefault{\textdownarrow}      {\tc@check@symbol4\textdownarrow}
\DeclareTextCommandDefault{\textleftarrow}      {\tc@check@symbol4\textleftarrow}
\DeclareTextCommandDefault{\textlira}           {\tc@check@symbol4\textlira}
\DeclareTextCommandDefault{\textrightarrow}     {\tc@check@symbol4\textrightarrow}
\DeclareTextCommandDefault{\textuparrow}        {\tc@check@symbol4\textuparrow}
\DeclareTextCommandDefault{\textwon}            {\tc@check@symbol4\textwon}


%------------------ sub-enc 5 (most older PS fonts (they produce tofu if a symbol is missing)

\DeclareTextCommandDefault{\textestimated}      {\tc@check@symbol5\textestimated}
\DeclareTextCommandDefault{\textnumero}         {\tc@check@symbol5\textnumero}


%------------------ sub-enc 6

\DeclareTextCommandDefault{\textflorin}         {\tc@check@symbol6\textflorin}    
\DeclareTextCommandDefault{\textcurrency}       {\tc@check@symbol6\textcurrency}

%------------------ sub-enc 7

\DeclareTextCommandDefault{\textfractionsolidus}{\tc@check@symbol7\textfractionsolidus}
\DeclareTextCommandDefault{\textohm}            {\tc@check@symbol7\textohm}
\DeclareTextCommandDefault{\textmu}             {\tc@check@symbol7\textmu}
\DeclareTextCommandDefault{\textminus}          {\tc@check@symbol7\textminus}




%------------------ sub-enc 8 (faked euro plus others)

\DeclareTextCommandDefault{\textblank}          {\tc@check@symbol{8}\textblank}
\DeclareTextCommandDefault{\textinterrobangdown}{\tc@check@symbol{8}\textinterrobangdown}
\DeclareTextCommandDefault{\textinterrobang}    {\tc@check@symbol{8}\textinterrobang}

\DeclareTextCommandDefault{\texteuro}
            {\CheckEncodingSubset\UseTextSymbol{TS1}\tc@fake@euro{8}\texteuro}



%------------------ sub-enc 9 (really most stuff missing in cochineal and AlgolRevived)

\DeclareTextCommandDefault{\textcelsius}{\tc@check@symbol{9}\textcelsius}
\DeclareTextCommandDefault{\textonesuperior}{\tc@check@symbol{9}\textonesuperior}
\DeclareTextCommandDefault{\textthreequartersemdash}{\tc@check@symbol{9}\textthreequartersemdash}
\DeclareTextCommandDefault{\textthreesuperior}{\tc@check@symbol{9}\textthreesuperior}
\DeclareTextCommandDefault{\texttwelveudash}{\tc@check@symbol{9}\texttwelveudash}
\DeclareTextCommandDefault{\texttwosuperior}{\tc@check@symbol{9}\texttwosuperior}
\DeclareTextCommandDefault{\textbardbl}{\tc@check@symbol{9}\textbardbl}

%------------------ always available or so we hope

\DeclareTextSymbolDefault{\textdollar}{TS1}
\UndeclareTextCommand{\textdollar}  {OT1}         % don't use the OT1 def any longer

\DeclareTextSymbolDefault{\textsterling}{TS1}
\UndeclareTextCommand{\textsterling}{OT1}         % don't use the OT1 def any longer

\DeclareTextSymbolDefault{\textperthousand}{TS1}
\UndeclareTextCommand{\textperthousand}{T1}       % don't use the T1 def
                                                  % should actually not be declaed
                                                  % in the t1enc.def file instead
                                                  % otherwise fontenc would overwrite
                                                  % it again (legacy tests!)
                                                  


% these are the std itemize and footnote symbols originally taken from OMS

\DeclareTextSymbolDefault{\textasteriskcentered}{TS1} 
\DeclareTextSymbolDefault{\textbullet}{TS1}
\DeclareTextSymbolDefault{\textdaggerdbl}{TS1}
\DeclareTextSymbolDefault{\textdagger}{TS1}
\DeclareTextSymbolDefault{\textparagraph}{TS1}
\DeclareTextSymbolDefault{\textperiodcentered}{TS1}
\DeclareTextSymbolDefault{\textsection}{TS1}

%\tc@compatibility@fetch{\textasteriskcentered}{OMS}
%\tc@compatibility@fetch{\textbullet}{OMS}
%\tc@compatibility@fetch{\textdaggerdbl}{OMS}
%\tc@compatibility@fetch{\textdagger}{OMS}
%\tc@compatibility@fetch{\textparagraph}{OMS}
%\tc@compatibility@fetch{\textperiodcentered}{OMS}
%\tc@compatibility@fetch{\textsection}{OMS}


\DeclareTextAccentDefault{\capitalcedilla}{TS1}
\DeclareTextAccentDefault{\capitalogonek}{TS1}

%%\DeclareTextSymbolDefault{\textbardbl}{TS1} % subst in sub-enc 9 above
\DeclareTextSymbolDefault{\textbrokenbar}{TS1}
%%\DeclareTextSymbolDefault{\textcelsius}{TS1} % subst in sub-enc 9 above
\DeclareTextSymbolDefault{\textcent}{TS1} 
\DeclareTextSymbolDefault{\textcopyright}{TS1}
\DeclareTextSymbolDefault{\textdegree}{TS1}
\DeclareTextSymbolDefault{\textdiv}{TS1}
\DeclareTextSymbolDefault{\textlnot}{TS1}
\DeclareTextSymbolDefault{\textonehalf}{TS1}
\DeclareTextSymbolDefault{\textonequarter}{TS1}
%%\DeclareTextSymbolDefault{\textonesuperior}{TS1} % subst in sub-enc 9 above
\DeclareTextSymbolDefault{\textordfeminine}{TS1}
\DeclareTextSymbolDefault{\textordmasculine}{TS1}
\DeclareTextSymbolDefault{\textpm}{TS1}
\DeclareTextSymbolDefault{\textquotesingle}{TS1}
\DeclareTextSymbolDefault{\textquotestraightbase}{TS1}
\DeclareTextSymbolDefault{\textquotestraightdblbase}{TS1}
\DeclareTextSymbolDefault{\textregistered}{TS1}
%%\DeclareTextSymbolDefault{\textthreequartersemdash}{TS1} % subst in sub-enc 9 above
\DeclareTextSymbolDefault{\textthreequarters}{TS1}
%%\DeclareTextSymbolDefault{\textthreesuperior}{TS1} % subst in sub-enc 9 above
\DeclareTextSymbolDefault{\texttimes}{TS1}
\DeclareTextSymbolDefault{\texttrademark}{TS1}
%%\DeclareTextSymbolDefault{\texttwelveudash}{TS1} % subst in sub-enc 9 above
%%\DeclareTextSymbolDefault{\texttwosuperior}{TS1} % subst in sub-enc 9 above
\DeclareTextSymbolDefault{\textyen}{TS1}



%------------------ END of defaults (some get overwritten for Unicode engines)


    


% \subsection{Dealing with Unicode engines}




\ifx \Umathcode\@undefined  \else


% take these from TS1 still (so probably from LMR)

%\DeclareTextSymbol{\textcopyleft}{TS1}{171}
%\DeclareTextSymbol{\textdblhyphen}{TS1}{45}
%\DeclareTextSymbol{\textdblhyphenchar}{TS1}{127}
%\DeclareTextSymbol{\textquotestraightbase}{TS1}{13}
%\DeclareTextSymbol{\textquotestraightdblbase}{TS1}{18}
%\DeclareTextSymbol{\textleaf}{TS1}{108}
%\DeclareTextSymbol{\texttwelveudash}{TS1}{21}
%\DeclareTextSymbol{\textthreequartersemdash}{TS1}{22}




% not always set up at this point while not fully integrated
\providecommand\UnicodeEncodingName{TU}

\def\add@unicode@accent#1#2{%
  \if\relax\detokenize{#2}\relax^^a0\else#2\fi
  \char#1\relax}
\def\DeclareUnicodeAccent#1#2#3{%
  \DeclareTextCommand{#1}{#2}{\add@unicode@accent{#3}}%
}

% use \oldstylenums for these

\DeclareTextCommand{\textzerooldstyle} \UnicodeEncodingName{\oldstylenums{0}}
\DeclareTextCommand{\textoneoldstyle}  \UnicodeEncodingName{\oldstylenums{1}}
\DeclareTextCommand{\texttwooldstyle}  \UnicodeEncodingName{\oldstylenums{2}}
\DeclareTextCommand{\textthreeoldstyle}\UnicodeEncodingName{\oldstylenums{3}}
\DeclareTextCommand{\textfouroldstyle} \UnicodeEncodingName{\oldstylenums{4}}
\DeclareTextCommand{\textfiveoldstyle} \UnicodeEncodingName{\oldstylenums{5}}
\DeclareTextCommand{\textsixoldstyle}  \UnicodeEncodingName{\oldstylenums{6}}
\DeclareTextCommand{\textsevenoldstyle}\UnicodeEncodingName{\oldstylenums{7}}
\DeclareTextCommand{\texteightoldstyle}\UnicodeEncodingName{\oldstylenums{8}}
\DeclareTextCommand{\textnineoldstyle} \UnicodeEncodingName{\oldstylenums{9}}

% those have Unicode points

\DeclareTextSymbol{\textpilcrow}       \UnicodeEncodingName{"00B6}
\DeclareTextSymbol{\textborn}          \UnicodeEncodingName{"002A}
\DeclareTextSymbol{\textdied}          \UnicodeEncodingName{"2020}
\DeclareTextSymbol{\textlbrackdbl}     \UnicodeEncodingName{"27E6}
\DeclareTextSymbol{\textrbrackdbl}     \UnicodeEncodingName{"27E7}
\DeclareTextSymbol{\textguarani}       \UnicodeEncodingName{"20B2}

% this can't be set up if TU isn't loaded in the test suite config
%
% right now 20DD doesn't exist in LM, so maybe not a good idea at the moment
% maybe test for the slot and do the fallback oalign if it doesn't exist
%
  \expandafter\ifx\csname TU-cmd\endcsname\relax
  \else
    \DeclareUnicodeAccent{\textcircled}    \UnicodeEncodingName{"20DD}
  \fi

% we could make them point to dollar and cent glyphs in TU

%\DeclareTextSymbol{\textcentoldstyle}            \UnicodeEncodingName{"00A2}
%\DeclareTextSymbol{\textdollaroldstyle}          \UnicodeEncodingName{"0024}


% but I think it is better to pick them up from TS1 even if that usually means LMR fonts

\DeclareTextSymbol{\textdollaroldstyle}{TS1}{138}
\DeclareTextSymbol{\textcentoldstyle}  {TS1}{139}


\fi  % --- END of Unicode engines


% \subsection{Font sub-encodings setup

% we declare the subsets of a good number of fonts in the kernel ...

% the default:
\DeclareEncodingSubset{TS1}{?}{1}  %FMI not sure 0/1  instead?

\DeclareEncodingSubset{TS1}{ccr}     {0}
\DeclareEncodingSubset{TS1}{cmbr}    {0}
\DeclareEncodingSubset{TS1}{cmr}     {0}
\DeclareEncodingSubset{TS1}{cmss}    {0}
\DeclareEncodingSubset{TS1}{cmtl}    {0}
\DeclareEncodingSubset{TS1}{cmtt}    {0}
\DeclareEncodingSubset{TS1}{cmvtt}   {0}
\DeclareEncodingSubset{TS1}{pxr}     {0}
\DeclareEncodingSubset{TS1}{pxss}    {0}
\DeclareEncodingSubset{TS1}{pxtt}    {0}
\DeclareEncodingSubset{TS1}{qag}     {0}
\DeclareEncodingSubset{TS1}{qbk}     {0}
\DeclareEncodingSubset{TS1}{qcr}     {0}
\DeclareEncodingSubset{TS1}{qcs}     {0}
\DeclareEncodingSubset{TS1}{qhvc}    {0}
\DeclareEncodingSubset{TS1}{qhv}     {0}
\DeclareEncodingSubset{TS1}{qpl}     {0}
\DeclareEncodingSubset{TS1}{qtm}     {0}
\DeclareEncodingSubset{TS1}{qzc}     {0}
\DeclareEncodingSubset{TS1}{txr}     {0}
\DeclareEncodingSubset{TS1}{txss}    {0}
\DeclareEncodingSubset{TS1}{txtt}    {0}

\DeclareEncodingSubset{TS1}{lmr}     {1}
\DeclareEncodingSubset{TS1}{lmdh}    {1}
\DeclareEncodingSubset{TS1}{lmss}    {1}
\DeclareEncodingSubset{TS1}{lmssq}   {1}
\DeclareEncodingSubset{TS1}{lmvtt}   {1}
\DeclareEncodingSubset{TS1}{lmtt}    {1} % missing TM, SM, pertenthousand for some reason

\DeclareEncodingSubset{TS1}{ptmx}    {2}
\DeclareEncodingSubset{TS1}{ptmj}    {2}
\DeclareEncodingSubset{TS1}{ul8}     {2}


\DeclareEncodingSubset{TS1}{bch}     {5}  % tofu for blank, ohm 
\DeclareEncodingSubset{TS1}{futj}    {5}  % tofu for blank, interrobang/down, ohm 
\DeclareEncodingSubset{TS1}{futs}    {5}  % tofu for blank, ohm 
\DeclareEncodingSubset{TS1}{futx}    {5}  % probably (currently broken distrib)
\DeclareEncodingSubset{TS1}{pag}     {5}  % tofu for blank, interrobang/down, ohm 
\DeclareEncodingSubset{TS1}{pbk}     {5}  % tofu for blank, interrobang/down, ohm 
\DeclareEncodingSubset{TS1}{pcr}     {5}  % tofu for blank, interrobang/down, ohm 
\DeclareEncodingSubset{TS1}{phv}     {5}  % tofu for blank, interrobang/down, ohm 
\DeclareEncodingSubset{TS1}{pnc}     {5}  % tofu for blank, interrobang/down, ohm 
\DeclareEncodingSubset{TS1}{pplj}    {5}  % tofu for blank
\DeclareEncodingSubset{TS1}{pplx}    {5}  % tofu for blank
\DeclareEncodingSubset{TS1}{ppl}     {5}  % tofu for blank interrobang/down
\DeclareEncodingSubset{TS1}{ptm}     {5}  % tofu for blank, interrobang/down, ohm 
\DeclareEncodingSubset{TS1}{pzc}     {5}  % tofu for blank, interrobang/down, ohm 
\DeclareEncodingSubset{TS1}{ul9}     {5}  % tofu for blank, interrobang/down, ohm 

\DeclareEncodingSubset{TS1}{dayroms} {6}  % tofu for blank, interrobang/down, ohm 
\DeclareEncodingSubset{TS1}{dayrom}  {6}  % tofu for blank, interrobang/down, ohm 


\DeclareEncodingSubset{TS1}{augie}   {8}  % really only missing euro
\DeclareEncodingSubset{TS1}{put}     {8}
\DeclareEncodingSubset{TS1}{uag}     {8}  % probably (currently broken distrib)
\DeclareEncodingSubset{TS1}{ugq}     {8}  

\DeclareEncodingSubset{TS1}{zi4}     {9}


% LucidaBright (sold through TUG) probably not quite correct as I have the older fonts ...
%
\DeclareEncodingSubset{TS1}{hls}     {5}
\DeclareEncodingSubset{TS1}{hlst}    {5}
\DeclareEncodingSubset{TS1}{hlct}    {5}
\DeclareEncodingSubset{TS1}{hlh}     {5}
\DeclareEncodingSubset{TS1}{hlx}     {8}
\DeclareEncodingSubset{TS1}{hlce}    {8}
\DeclareEncodingSubset{TS1}{hlcn}    {8}
\DeclareEncodingSubset{TS1}{hlcw}    {8}
\DeclareEncodingSubset{TS1}{hlcf}    {8}




\DeclareEncodingSubset{TS1}{lato-*}                   {0}  % with a bunch of tofu inside
\DeclareEncodingSubset{TS1}{opensans-*}               {0}  % with a bunch of tofu inside
\DeclareEncodingSubset{TS1}{cantarell-*}              {0}  % with a bunch of tofu inside
\DeclareEncodingSubset{TS1}{fbb-*}                    {0}  % missing centoldstyle

\DeclareEncodingSubset{TS1}{tli}                      {1}  % with lots of tofu inside

\DeclareEncodingSubset{TS1}{Alegreya-*}               {2}
\DeclareEncodingSubset{TS1}{AlegreyaSans-*}           {2}
\DeclareEncodingSubset{TS1}{DejaVuSans-TLF}           {2}
\DeclareEncodingSubset{TS1}{DejaVuSansCondensed-TLF}  {2}
\DeclareEncodingSubset{TS1}{DejaVuSansMono-TLF}       {2}
\DeclareEncodingSubset{TS1}{EBGaramond-*}             {2}
\DeclareEncodingSubset{TS1}{Tempora-TLF}              {2}
\DeclareEncodingSubset{TS1}{Tempora-TOsF}             {2}

\DeclareEncodingSubset{TS1}{Arimo-TLF}                {3}
\DeclareEncodingSubset{TS1}{Carlito-*}                {3}
\DeclareEncodingSubset{TS1}{FiraSans-*}               {3}
\DeclareEncodingSubset{TS1}{IBMPlexSans-TLF}          {3}
\DeclareEncodingSubset{TS1}{Merriweather-OsF}         {3} 
\DeclareEncodingSubset{TS1}{Montserrat-*}             {3}
\DeclareEncodingSubset{TS1}{MontserratAlternates-*}   {3}
\DeclareEncodingSubset{TS1}{SourceCodePro-TLF}        {3}
\DeclareEncodingSubset{TS1}{SourceCodePro-TOsF}       {3}
\DeclareEncodingSubset{TS1}{SourceSansPro-*}          {3}
\DeclareEncodingSubset{TS1}{SourceSerifPro-*}         {3}
\DeclareEncodingSubset{TS1}{Tinos-TLF}                {3}


\DeclareEncodingSubset{TS1}{AccanthisADFStdNoThree-LF}{4}
\DeclareEncodingSubset{TS1}{Cabin-TLF}                {4}
\DeclareEncodingSubset{TS1}{Caladea-TLF}              {4}
\DeclareEncodingSubset{TS1}{Chivo-*}                  {4}
\DeclareEncodingSubset{TS1}{ClearSans-TLF}            {4}
\DeclareEncodingSubset{TS1}{Coelacanth-LF}            {4}
\DeclareEncodingSubset{TS1}{CrimsonPro-*}             {4}
\DeclareEncodingSubset{TS1}{FiraMono-TLF}             {4}
\DeclareEncodingSubset{TS1}{FiraMono-TOsF}            {4}
\DeclareEncodingSubset{TS1}{Go-TLF}                   {4}
\DeclareEncodingSubset{TS1}{GoMono-TLF}               {4}
\DeclareEncodingSubset{TS1}{InriaSans-*}              {4}
\DeclareEncodingSubset{TS1}{InriaSerif-*}             {4}
\DeclareEncodingSubset{TS1}{LibertinusSans-*}         {4}
\DeclareEncodingSubset{TS1}{LibertinusSerif-*}        {4}
\DeclareEncodingSubset{TS1}{LibreBodoni-TLF}          {4}
\DeclareEncodingSubset{TS1}{LibreFranklin-TLF}        {4}
\DeclareEncodingSubset{TS1}{LinguisticsPro-LF}        {4}
\DeclareEncodingSubset{TS1}{LinguisticsPro-OsF}       {4}
\DeclareEncodingSubset{TS1}{LinuxBiolinumT-*}         {4}
\DeclareEncodingSubset{TS1}{LinuxLibertineT-*}        {4}
\DeclareEncodingSubset{TS1}{MerriweatherSans-OsF}     {4} 
\DeclareEncodingSubset{TS1}{MintSpirit-*}             {4}
\DeclareEncodingSubset{TS1}{MintSpiritNoTwo-*}        {4}
\DeclareEncodingSubset{TS1}{PTMono-TLF}               {4}
\DeclareEncodingSubset{TS1}{PTSans-TLF}               {4}
\DeclareEncodingSubset{TS1}{PTSansCaption-TLF}        {4}
\DeclareEncodingSubset{TS1}{PTSansNarrow-TLF}         {4}
\DeclareEncodingSubset{TS1}{PTSerif-TLF}              {4}
\DeclareEncodingSubset{TS1}{PTSerifCaption-TLF}       {4}
\DeclareEncodingSubset{TS1}{Raleway-TLF}              {4}
\DeclareEncodingSubset{TS1}{Raleway-TOsF}             {4}
\DeclareEncodingSubset{TS1}{Roboto-*}                 {4}
\DeclareEncodingSubset{TS1}{RobotoMono-TLF}           {4}
\DeclareEncodingSubset{TS1}{RobotoSlab-TLF}           {4}
\DeclareEncodingSubset{TS1}{Rosario-*}                {4}
\DeclareEncodingSubset{TS1}{SticksTooText-*}          {4}
\DeclareEncodingSubset{TS1}{UniversalisADFStd-LF}     {4}


\DeclareEncodingSubset{TS1}{Almendra-OsF}             {5} 
\DeclareEncodingSubset{TS1}{Baskervaldx-*}            {5}
\DeclareEncodingSubset{TS1}{BaskervilleF-*}           {5}
\DeclareEncodingSubset{TS1}{Bitter-TLF}               {5}
\DeclareEncodingSubset{TS1}{Cinzel-LF}                {5}
\DeclareEncodingSubset{TS1}{CinzelDecorative-LF}      {5}
\DeclareEncodingSubset{TS1}{DejaVuSerif-TLF}          {5}
\DeclareEncodingSubset{TS1}{DejaVuSerifCondensed-TLF} {5}
\DeclareEncodingSubset{TS1}{GilliusADF-LF}            {5}
\DeclareEncodingSubset{TS1}{GilliusADFCond-LF}        {5}
\DeclareEncodingSubset{TS1}{GilliusADFNoTwo-LF}       {5}
\DeclareEncodingSubset{TS1}{GilliusADFNoTwoCond-LF}   {5}
\DeclareEncodingSubset{TS1}{LobsterTwo-LF}            {5}
\DeclareEncodingSubset{TS1}{OldStandard-TLF}          {5}
\DeclareEncodingSubset{TS1}{PlayfairDisplay-TLF}      {5}
\DeclareEncodingSubset{TS1}{PlayfairDisplay-TOsF}     {5}
\DeclareEncodingSubset{TS1}{TheanoDidot-TLF}          {5}
\DeclareEncodingSubset{TS1}{TheanoDidot-TOsF}         {5}
\DeclareEncodingSubset{TS1}{TheanoModern-TLF}         {5}
\DeclareEncodingSubset{TS1}{TheanoModern-TOsF}        {5}
\DeclareEncodingSubset{TS1}{TheanoOldStyle-TLF}       {5}
\DeclareEncodingSubset{TS1}{TheanoOldStyle-TOsF}      {5}

\DeclareEncodingSubset{TS1}{Crimson-TLF}              {6}
\DeclareEncodingSubset{TS1}{IBMPlexMono-TLF}          {6}
\DeclareEncodingSubset{TS1}{IBMPlexSerif-TLF}         {6}
\DeclareEncodingSubset{TS1}{LibertinusMono-TLF}       {6}
\DeclareEncodingSubset{TS1}{LibertinusSerifDisplay-LF}{6}
\DeclareEncodingSubset{TS1}{LinuxLibertineDisplayT-*} {6}
\DeclareEncodingSubset{TS1}{LinuxLibertineMonoT-LF}   {6}
\DeclareEncodingSubset{TS1}{LinuxLibertineMonoT-TLF}  {6}
\DeclareEncodingSubset{TS1}{Overlock-LF}              {6}


\DeclareEncodingSubset{TS1}{CormorantGaramond-*}      {7}
\DeclareEncodingSubset{TS1}{Heuristica-TLF}           {7}
\DeclareEncodingSubset{TS1}{Heuristica-TOsF}          {7}
\DeclareEncodingSubset{TS1}{IMFELLEnglish-TLF}        {7}
\DeclareEncodingSubset{TS1}{LibreBaskerville-TLF}     {7}
\DeclareEncodingSubset{TS1}{LibreCaslon-*}            {7}
\DeclareEncodingSubset{TS1}{Marcellus-LF}             {7}
\DeclareEncodingSubset{TS1}{NotoSans-*}               {7}
\DeclareEncodingSubset{TS1}{NotoSansMono-TLF}         {7}
\DeclareEncodingSubset{TS1}{NotoSansMono-TOsF}        {7}
\DeclareEncodingSubset{TS1}{NotoSerif-*}              {7}
\DeclareEncodingSubset{TS1}{Quattrocento-TLF}         {7}
\DeclareEncodingSubset{TS1}{QuattrocentoSans-TLF}     {7}
\DeclareEncodingSubset{TS1}{XCharter-TLF}             {7}
\DeclareEncodingSubset{TS1}{XCharter-TOsF}            {7}
\DeclareEncodingSubset{TS1}{erewhon-*}                {7} 
\DeclareEncodingSubset{TS1}{ComicNeue-TLF}            {7} 
\DeclareEncodingSubset{TS1}{ComicNeueAngular-TLF}     {7} 
\DeclareEncodingSubset{TS1}{Forum-LF}                 {7}  % the superiors are missing 

\DeclareEncodingSubset{TS1}{Cochineal-*}              {8}

\DeclareEncodingSubset{TS1}{AlgolRevived-TLF}         {9}






% footnote symbols already use \normalfont hardwired (and that should
% probably stay like that).


% maybe this should also be available in Unicode engines. There a
% default is not enough since the symbols are defined in TU so one
% would need to undefine them there or perhaps better redefine them
% there.

\def\UseLegacyTextSymbols{%
  \DeclareTextSymbolDefault{\textasteriskcentered}{OMS}%
  \DeclareTextSymbolDefault{\textbardbl}{OMS}%
  \DeclareTextSymbolDefault{\textbullet}{OMS}%
  \DeclareTextSymbolDefault{\textdaggerdbl}{OMS}%
  \DeclareTextSymbolDefault{\textdagger}{OMS}%
  \DeclareTextSymbolDefault{\textparagraph}{OMS}%
  \DeclareTextSymbolDefault{\textperiodcentered}{OMS}%
  \DeclareTextSymbolDefault{\textsection}{OMS}%
  \UndeclareTextCommand{\textsection}{T1}%
  \expandafter\let\csname oldstylenums \expandafter\endcsname
                  \csname legacyoldstylenums \endcsname
}


% we go the roundabout way via separate OMS declarations so that
%   \renewcommand\textbullet{\textlegacybullet}
% doesn't produce an endless loop

\DeclareTextSymbol{\textlegacyasteriskcentered}{OMS}{3}   % "03
\DeclareTextSymbol{\textlegacybardbl}{OMS}{107}           % "6B
\DeclareTextSymbol{\textlegacybullet}{OMS}{15}            % "0F
\DeclareTextSymbol{\textlegacydaggerdbl}{OMS}{122}        % "7A
\DeclareTextSymbol{\textlegacydagger}{OMS}{121}           % "79
\DeclareTextSymbol{\textlegacyparagraph}{OMS}{123}        % "7B
\DeclareTextSymbol{\textlegacyperiodcentered}{OMS}{1}     % "01
\DeclareTextSymbol{\textlegacysection}{OMS}{120}          % "78

\DeclareTextSymbolDefault{\textlegacyasteriskcentered}{OMS}
\DeclareTextSymbolDefault{\textlegacybardbl}{OMS}
\DeclareTextSymbolDefault{\textlegacybullet}{OMS}
\DeclareTextSymbolDefault{\textlegacydaggerdbl}{OMS}
\DeclareTextSymbolDefault{\textlegacydagger}{OMS}
\DeclareTextSymbolDefault{\textlegacyparagraph}{OMS}
\DeclareTextSymbolDefault{\textlegacyperiodcentered}{OMS}
\DeclareTextSymbolDefault{\textlegacysection}{OMS}


% kernel leftover .. this should be like this these days ... not faked

\DeclareTextSymbolDefault{\textcompwordmark}{T1}



\makeatother    

\endinput

