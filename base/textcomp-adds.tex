
% textcomp stuff to be added to the kernel (or tuenc.def)

\makeatletter

\input{ts1enc.def}

\ifx\Umathcode\@undefined

\def\textcompsubstdefault{cmr}
\input{ts1cmr.fd}

% in pdftex pick up capital accents from TS1 if available 

%\DeclareTextAccent{\capitalacute}{TS1}{1}
%\DeclareTextAccent{\capitalbreve}{TS1}{8}
%\DeclareTextAccent{\capitalcaron}{TS1}{7}
%\DeclareTextAccent{\capitalcircumflex}{TS1}{2}
%\DeclareTextAccent{\capitaldieresis}{TS1}{4}
%\DeclareTextAccent{\capitaldotaccent}{TS1}{10}
%\DeclareTextAccent{\capitalgrave}{TS1}{0}
%\DeclareTextAccent{\capitalhungarumlaut}{TS1}{5}
%\DeclareTextAccent{\capitalmacron}{TS1}{9}
%\DeclareTextAccent{\capitalring}{TS1}{6}
%\DeclareTextAccent{\capitaltilde}{TS1}{3}


% use textcomp \oldstylenums if in pdftex

\DeclareRobustCommand\oldstylenums[1]{%
 \begingroup
 \ifmmode
   \mathgroup\symletters #1%
  \else
   \CheckEncodingSubset\@use@text@encoding{TS1}%
       {\PackageWarning{textcomp}%
          {Oldstyle digits unavailable for
           family \f@family.\MessageBreak
           Lining digits used instead}}%
       \tw@{#1}%
   \fi
 \endgroup
}


\else  % the unicode engine case

\def\textcompsubstdefault{lmr}
% This file belongs to the Latin Modern package. The work is released
% under the GUST Font License. See the MANIFEST-Latin-Modern.txt and
% README-Latin-Modern.txt files for the details. For the most recent version of
% this license see http://www.gust.org.pl/fonts/licenses/GUST-FONT-LICENSE.txt
% or http://tug.org/fonts/licenses/GUST-FONT-LICENSE.txt

\ProvidesFile{ts1lmr.fd}[2009/10/30 v1.6 Font defs for Latin Modern]
\DeclareFontFamily{TS1}{lmr}{}
\DeclareFontShape{TS1}{lmr}{m}{n}%
     {<-5.5>    ts1-lmr5     <5.5-6.5> ts1-lmr6
      <6.5-7.5> ts1-lmr7     <7.5-8.5> ts1-lmr8
      <8.5-9.5> ts1-lmr9     <9.5-11>  ts1-lmr10
      <11-15>   ts1-lmr12
      <15-> ts1-lmr17
      }{}
\DeclareFontShape{TS1}{lmr}{m}{sl}%
     {<-8.5>    ts1-lmro8    <8.5-9.5> ts1-lmro9
      <9.5-11>  ts1-lmro10   <11-15>   ts1-lmro12
      <15-> ts1-lmro17
      }{}
\DeclareFontShape{TS1}{lmr}{m}{it}%
     {<-7.5>    ts1-lmri7
      <7.5-8.5> ts1-lmri8    <8.5-9.5> ts1-lmri9
      <9.5-11>  ts1-lmri10   <11->   ts1-lmri12
      }{}
\DeclareFontShape{TS1}{lmr}{m}{sc}%
     {<-> ts1-lmcsc10}{}
\DeclareFontShape{TS1}{lmr}{m}{ui}%
     {<-> ts1-lmu10}{}
%
% Is this the right 'shape'?:
\DeclareFontShape{TS1}{lmr}{m}{scsl}%
     {<-> ts1-lmcsco10}{}
%%%%%%% bold series
\DeclareFontShape{TS1}{lmr}{b}{n}
     {<-> ts1-lmb10}{}
\DeclareFontShape{TS1}{lmr}{b}{sl}
     {<-> ts1-lmbo10}{}
%%%%%%% bold extended series
\DeclareFontShape{TS1}{lmr}{bx}{n}
     {<-5.5>   ts1-lmbx5      <5.5-6.5> ts1-lmbx6
      <6.5-7.5> ts1-lmbx7      <7.5-8.5> ts1-lmbx8
      <8.5-9.5> ts1-lmbx9      <9.5-11>  ts1-lmbx10
      <11->   ts1-lmbx12
      }{}
\DeclareFontShape{TS1}{lmr}{bx}{it}
     {<-> ts1-lmbxi10}{}
\DeclareFontShape{TS1}{lmr}{bx}{sl}
     {<-> ts1-lmbxo10}{}
%%%%%%% Font/shape undefined, therefore substituted
\DeclareFontShape{TS1}{lmr}{b}{it}
     {<->sub * lmr/b/sl}{}
\endinput
%%
%% End of file `ts1lmr.fd'.


% in unicode engines make capital accents normal accents

\def\newtie{\t}
\def\capitaltie{\t}
\def\capitalnewtie{\t}
\def\capitalcedilla{\c}
\def\capitalogonek{\k}
\def\capitalgrave{\`}
\def\capitalacute{\'}
\def\capitalcircumflex{\^}
\def\capitaltilde{\~}
\def\capitaldieresis{\"}
\def\capitalhungarumlaut{\H}
\def\capitalring{\r}
\def\capitalcaron{\v}
\def\capitalbreve{\u}
\def\capitalmacron{\=}
\def\capitaldotaccent{\.}

% alternatively we could use TS1 accents if the subset of the current font is known and
% complete (ie 0) but perhaps that overkill

%\def\tc@check@accentii#1{\CheckEncodingSubset
%    \UseTextAccent{TS1}{\expandafter#1\@gobble}}
%
%\DeclareTextCommandDefault{\capitalcedilla}%
%    {\tc@check@accentii\c1\capitalcedilla}


% in unicode engines \oldstylenum will have the default kernel def
% which then can be overwritten by fontspec

\fi

\def\CheckEncodingSubset#1#2#3#4#5{%
    \ifnum #4>%
      \expandafter\ifx\csname #2:\f@family\endcsname\relax
        0\csname #2:?\endcsname
      \else
        \csname #2:\f@family\endcsname
      \fi
   \relax
   \expandafter\@firstoftwo    % use TS1 definition
  \else
   \expandafter\@secondoftwo   % use substitute
 \fi
  {#1{#2}}{#3}%
  #5%
}

\def\DeclareEncodingSubset#1#2#3{%
   \@ifundefined{#1:#2}%
     {\PackageInfo{textcomp}{Setting #2 sub-encoding to #1/#3}}%
     {\PackageInfo{textcomp}{Changing #2 sub-encoding to #1/#3}}%
     \@namedef{#1:#2}{#3}}

\@onlypreamble\DeclareEncodingSubset

% maybe that should be kernel error now ...
%\def\tc@errorwarn{\PackageError}
% or maybe not ...
\def\tc@errorwarn#1#2#3{\PackageWarning{#1}{#2}}

\def\tc@error#1{%
   \tc@errorwarn{textcomp}%  % should be latex error if general
    {Accent \string#1 not provided by\MessageBreak
     font family \f@family\space
     in TS1 encoding}\@eha
}

\def\tc@subst#1{%
   \tc@errorwarn{textcomp}%  % should be latex error if general
    {Symbol \string#1 not provided by\MessageBreak
     font family \f@family\space
     in TS1 encoding.\MessageBreak Default family used instead}\@eha
  \bgroup\fontfamily\textcompsubstdefault\selectfont#1\egroup
}


\def\tc@fake@euro#1{%
   \leavevmode
   \PackageInfo{textcomp}{Faking \noexpand#1for font family
                          \f@family\MessageBreak in TS1 encoding}%
   \valign{##\cr
      \vfil\hbox to 0.07em{\dimen@\f@size\p@
                           \math@fontsfalse
                           \fontsize{.7\dimen@}\z@\selectfont=\hss}%
      \vfil\cr%
      \hbox{C}\crcr
   }%
}
\def\tc@check@symbol{\CheckEncodingSubset\UseTextSymbol{TS1}\tc@subst}
\def\tc@check@accent{\CheckEncodingSubset\UseTextAccent{TS1}\tc@error}

% for backwards compat we fetch smbols from OMS/OML in OT1/T1
%\def\tc@compatibility@fetch#1#2{%
%  \DeclareTextCommand{#1}{OT1}{\UseTextSymbol{#2}#1}%
%  \DeclareTextCommand{#1}{T1}{\UseTextSymbol{#2}#1}%
%}


% \subsection{Sub-encodings}
%
% 0 is full the TS1 encoding


\DeclareTextSymbolDefault{\textcapitalcompwordmark}{TS1}
\DeclareTextSymbolDefault{\textascendercompwordmark}{TS1}


\DeclareTextSymbolDefault{\textbardbl}{TS1}


%------------------ sub-enc 1 (drop things that don't work in lmr)


% 1 is TS1 without \cs{textcircled}, \cs{t} (why?)


\DeclareTextCommandDefault{\textcircled}
  {\CheckEncodingSubset\UseTextAccent{TS1}{\UseTextAccent{OMS}}1\textcircled}
\DeclareTextCommandDefault{\t}
  {\CheckEncodingSubset\UseTextAccent{TS1}{\UseTextAccent{OML}}1\t}


%------------------ sub-enc 2 (majority of new OTF fonts via autoinst)

\DeclareTextCommandDefault{\capitalacute}       {\tc@check@accent2\capitalacute}
\DeclareTextCommandDefault{\capitalbreve}       {\tc@check@accent2\capitalbreve}
\DeclareTextCommandDefault{\capitalcaron}       {\tc@check@accent2\capitalcaron}
\DeclareTextCommandDefault{\capitalcircumflex}  {\tc@check@accent2\capitalcircumflex}
\DeclareTextCommandDefault{\capitaldieresis}    {\tc@check@accent2\capitaldieresis}
\DeclareTextCommandDefault{\capitaldotaccent}   {\tc@check@accent2\capitaldotaccent}
\DeclareTextCommandDefault{\capitalgrave}       {\tc@check@accent2\capitalgrave}
\DeclareTextCommandDefault{\capitalmacron}      {\tc@check@accent2\capitalmacron}
\DeclareTextCommandDefault{\capitalring}        {\tc@check@accent2\capitalring}
\DeclareTextCommandDefault{\capitaltilde}       {\tc@check@accent2\capitaltilde}
\DeclareTextCommandDefault{\capitalnewtie}      {\tc@check@accent2\capitalnewtie}
\DeclareTextCommandDefault{\capitaltie}         {\tc@check@accent2\capitaltie}
\DeclareTextCommandDefault{\capitalhungarumlaut}{\tc@check@accent2\capitalhungarumlaut}
\DeclareTextCommandDefault{\newtie}             {\tc@check@accent2\newtie}

% next two exists in some fonts (faked?), but we ignore that to get linear subsets

\DeclareTextCommandDefault{\textlbrackdbl}      {\tc@check@symbol2\textlbrackdbl}
\DeclareTextCommandDefault{\textrbrackdbl}      {\tc@check@symbol2\textrbrackdbl}

% old style numerals are again in some fonts but using -OsF etc is the better approach
% to get them so we claim they aren in sub-encoding 2 as that's true for most fonts

\DeclareTextCommandDefault{\texteightoldstyle}  {\tc@check@symbol2\texteightoldstyle}
\DeclareTextCommandDefault{\textfiveoldstyle}   {\tc@check@symbol2\textfiveoldstyle}
\DeclareTextCommandDefault{\textfouroldstyle}   {\tc@check@symbol2\textfouroldstyle}
\DeclareTextCommandDefault{\textnineoldstyle}   {\tc@check@symbol2\textnineoldstyle}
\DeclareTextCommandDefault{\textoneoldstyle}    {\tc@check@symbol2\textoneoldstyle}
\DeclareTextCommandDefault{\textsevenoldstyle}  {\tc@check@symbol2\textsevenoldstyle}
\DeclareTextCommandDefault{\textsixoldstyle}    {\tc@check@symbol2\textsixoldstyle}
\DeclareTextCommandDefault{\textthreeoldstyle}  {\tc@check@symbol2\textthreeoldstyle}
\DeclareTextCommandDefault{\texttwooldstyle}    {\tc@check@symbol2\texttwooldstyle}
\DeclareTextCommandDefault{\textzerooldstyle}   {\tc@check@symbol2\textzerooldstyle}

% those are special to TeX fonts (and for a few older PS fonts
% supported in the virtual fonts, but not any longer in the majority
% of fonts provided through autoinst, so we pretend there aren't
% available in sub-encoding 2 and below.

\DeclareTextCommandDefault{\textasciigrave}     {\tc@check@symbol2\textasciigrave}
\DeclareTextCommandDefault{\texttildelow}       {\tc@check@symbol2\texttildelow}
\DeclareTextCommandDefault{\textasciibreve}     {\tc@check@symbol2\textasciibreve}
\DeclareTextCommandDefault{\textasciicaron}     {\tc@check@symbol2\textasciicaron}
\DeclareTextCommandDefault{\textgravedbl}       {\tc@check@symbol2\textgravedbl}
\DeclareTextCommandDefault{\textacutedbl}       {\tc@check@symbol2\textacutedbl}
\DeclareTextCommandDefault{\textasciidieresis}  {\tc@check@symbol2\textasciidieresis}
\DeclareTextCommandDefault{\textasciimacron}    {\tc@check@symbol2\textasciimacron}
\DeclareTextCommandDefault{\textasciiacute}     {\tc@check@symbol2\textasciiacute}

% none of those are arround other than in CM based fonts

\DeclareTextCommandDefault{\textbaht}           {\tc@check@symbol2\textbaht}
\DeclareTextCommandDefault{\textbigcircle}      {\tc@check@symbol2\textbigcircle}
\DeclareTextCommandDefault{\textborn}           {\tc@check@symbol2\textborn}
\DeclareTextCommandDefault{\textcentoldstyle}   {\tc@check@symbol2\textcentoldstyle}
\DeclareTextCommandDefault{\textcircledP}       {\tc@check@symbol2\textcircledP}
\DeclareTextCommandDefault{\textcopyleft}       {\tc@check@symbol2\textcopyleft}
\DeclareTextCommandDefault{\textdblhyphenchar}  {\tc@check@symbol2\textdblhyphenchar}
\DeclareTextCommandDefault{\textdblhyphen}      {\tc@check@symbol2\textdblhyphen}
\DeclareTextCommandDefault{\textdied}           {\tc@check@symbol2\textdied}
\DeclareTextCommandDefault{\textdiscount}       {\tc@check@symbol2\textdiscount}
\DeclareTextCommandDefault{\textdivorced}       {\tc@check@symbol2\textdivorced}
\DeclareTextCommandDefault{\textdollaroldstyle} {\tc@check@symbol2\textdollaroldstyle}
\DeclareTextCommandDefault{\textguarani}        {\tc@check@symbol2\textguarani}
\DeclareTextCommandDefault{\textleaf}           {\tc@check@symbol2\textleaf}
\DeclareTextCommandDefault{\textlquill}         {\tc@check@symbol2\textlquill}
\DeclareTextCommandDefault{\textmarried}        {\tc@check@symbol2\textmarried}
\DeclareTextCommandDefault{\textmho}            {\tc@check@symbol2\textmho}
\DeclareTextCommandDefault{\textmusicalnote}    {\tc@check@symbol2\textmusicalnote}
\DeclareTextCommandDefault{\textnaira}          {\tc@check@symbol2\textnaira}
\DeclareTextCommandDefault{\textopenbullet}     {\tc@check@symbol2\textopenbullet}
\DeclareTextCommandDefault{\textpeso}           {\tc@check@symbol2\textpeso}
\DeclareTextCommandDefault{\textpilcrow}        {\tc@check@symbol2\textpilcrow}
\DeclareTextCommandDefault{\textrecipe}         {\tc@check@symbol2\textrecipe}
\DeclareTextCommandDefault{\textreferencemark}  {\tc@check@symbol2\textreferencemark}
\DeclareTextCommandDefault{\textrquill}         {\tc@check@symbol2\textrquill}
\DeclareTextCommandDefault{\textservicemark}    {\tc@check@symbol2\textservicemark}
\DeclareTextCommandDefault{\textsurd}           {\tc@check@symbol2\textsurd}

% we have a choice here: in T1 there is definition
%  for \cs{textpertenthousand} making the symbol up from \%
%  and \verb=\char 24= (twice) but in many fonts that char doesn't
%  exist and the slot is reused for random ligatures. So better not
%  use it because often it is wrong.  But pointing to TS1 is also not
%  great as only a few fonts have it as a real symbol, so we get a
%  substitution to CM or LM.

%  Alternatively we could just state that the symbol is unavailable in
%  those fonts. For now I substitute.


\DeclareTextCommandDefault{\textpertenthousand} {\tc@check@symbol2\textpertenthousand}
\UndeclareTextCommand{\textpertenthousand}{T1}

%------------------ sub-enc 3 

\DeclareTextCommandDefault{\textlangle}         {\tc@check@symbol3\textlangle}
\DeclareTextCommandDefault{\textrangle}         {\tc@check@symbol3\textrangle}



%------------------ sub-enc 4 


\DeclareTextCommandDefault{\textcolonmonetary}  {\tc@check@symbol4\textcolonmonetary}
\DeclareTextCommandDefault{\textdong}           {\tc@check@symbol4\textdong}
\DeclareTextCommandDefault{\textdownarrow}      {\tc@check@symbol4\textdownarrow}
\DeclareTextCommandDefault{\textleftarrow}      {\tc@check@symbol4\textleftarrow}
\DeclareTextCommandDefault{\textlira}           {\tc@check@symbol4\textlira}
\DeclareTextCommandDefault{\textrightarrow}     {\tc@check@symbol4\textrightarrow}
\DeclareTextCommandDefault{\textuparrow}        {\tc@check@symbol4\textuparrow}
\DeclareTextCommandDefault{\textwon}            {\tc@check@symbol4\textwon}


%------------------ sub-enc 5 ()

\DeclareTextCommandDefault{\textflorin}         {\tc@check@symbol5\textflorin}    
\DeclareTextCommandDefault{\textnumero}         {\tc@check@symbol5\textnumero}
\DeclareTextCommandDefault{\textestimated}      {\tc@check@symbol5\textestimated}
\DeclareTextCommandDefault{\textcurrency}       {\tc@check@symbol5\textcurrency}



%------------------ sub-enc 4 ()


\DeclareTextCommandDefault{\textfractionsolidus}{\tc@check@symbol3\textfractionsolidus}
\DeclareTextCommandDefault{\textblank}          {\tc@check@symbol4\textblank}
\DeclareTextCommandDefault{\textinterrobangdown}{\tc@check@symbol4\textinterrobangdown}
\DeclareTextCommandDefault{\textinterrobang}    {\tc@check@symbol4\textinterrobang}




%------------------ sub-enc 5 (older PS fonts)

\DeclareTextCommandDefault{\textohm}            {\tc@check@symbol5\textohm}
\DeclareTextCommandDefault{\textmu}             {\tc@check@symbol5\textmu}

%------------------ sub-enc 6 (conservative guess)

\DeclareTextCommandDefault{\texteuro}
   {\CheckEncodingSubset\UseTextSymbol{TS1}\tc@fake@euro6\texteuro}  %FMi fix


\DeclareTextCommandDefault{\textminus}          {\tc@check@symbol6\textminus} % FIX


%------------------ always available or so we hope

\DeclareTextSymbolDefault{\textdollar}{TS1}
\UndeclareTextCommand{\textdollar}  {OT1}         % don't use the OT1 def any longer

\DeclareTextSymbolDefault{\textsterling}{TS1}
\UndeclareTextCommand{\textsterling}{OT1}         % don't use the OT1 def any longer

\DeclareTextSymbolDefault{\textperthousand}{TS1}
\UndeclareTextCommand{\textperthousand}{T1}       % don't use the T1 def


% these are the std itemize and footnote symbols originally taken from OMS

\DeclareTextSymbolDefault{\textasteriskcentered}{TS1}
\DeclareTextSymbolDefault{\textbullet}{TS1} %0
\DeclareTextSymbolDefault{\textdaggerdbl}{TS1}
\DeclareTextSymbolDefault{\textdagger}{TS1}
\DeclareTextSymbolDefault{\textparagraph}{TS1}
\DeclareTextSymbolDefault{\textperiodcentered}{TS1}
\DeclareTextSymbolDefault{\textsection}{TS1}

%\tc@compatibility@fetch{\textasteriskcentered}{OMS}
%\tc@compatibility@fetch{\textbullet}{OMS}
%\tc@compatibility@fetch{\textdaggerdbl}{OMS}
%\tc@compatibility@fetch{\textdagger}{OMS}
%\tc@compatibility@fetch{\textparagraph}{OMS}
%\tc@compatibility@fetch{\textperiodcentered}{OMS}
%\tc@compatibility@fetch{\textsection}{OMS}


\DeclareTextAccentDefault{\capitalcedilla}{TS1}
\DeclareTextAccentDefault{\capitalogonek}{TS1}

\DeclareTextSymbolDefault{\textbrokenbar}{TS1} %0
\DeclareTextSymbolDefault{\textcelsius}{TS1}
\DeclareTextSymbolDefault{\textcent}{TS1}    %0
\DeclareTextSymbolDefault{\textcopyright}{TS1} %0
\DeclareTextSymbolDefault{\textdegree}{TS1}
\DeclareTextSymbolDefault{\textdiv}{TS1}
\DeclareTextSymbolDefault{\textlnot}{TS1}
\DeclareTextSymbolDefault{\textonehalf}{TS1}
\DeclareTextSymbolDefault{\textonequarter}{TS1}
\DeclareTextSymbolDefault{\textonesuperior}{TS1}
\DeclareTextSymbolDefault{\textordfeminine}{TS1}
\DeclareTextSymbolDefault{\textordmasculine}{TS1}
\DeclareTextSymbolDefault{\textpm}{TS1}
\DeclareTextSymbolDefault{\textquotesingle}{TS1}
\DeclareTextSymbolDefault{\textquotestraightbase}{TS1}
\DeclareTextSymbolDefault{\textquotestraightdblbase}{TS1}
\DeclareTextSymbolDefault{\textregistered}{TS1}
\DeclareTextSymbolDefault{\textthreequartersemdash}{TS1}
\DeclareTextSymbolDefault{\textthreequarters}{TS1}
\DeclareTextSymbolDefault{\textthreesuperior}{TS1}
\DeclareTextSymbolDefault{\texttimes}{TS1}
\DeclareTextSymbolDefault{\texttrademark}{TS1}
\DeclareTextSymbolDefault{\texttwelveudash}{TS1}
\DeclareTextSymbolDefault{\texttwosuperior}{TS1}
\DeclareTextSymbolDefault{\textyen}{TS1}



%------------------ END of defaults (some get overwritten for Unicode engines)


    


% \subsection{Dealing with Unicode engines}




\ifx \Umathcode\@undefined  \else


% take these from TS1 still (so probably from LMR)

%\DeclareTextSymbol{\textcopyleft}{TS1}{171}
%\DeclareTextSymbol{\textdblhyphen}{TS1}{45}
%\DeclareTextSymbol{\textdblhyphenchar}{TS1}{127}
%\DeclareTextSymbol{\textquotestraightbase}{TS1}{13}
%\DeclareTextSymbol{\textquotestraightdblbase}{TS1}{18}
%\DeclareTextSymbol{\textleaf}{TS1}{108}
%\DeclareTextSymbol{\texttwelveudash}{TS1}{21}
%\DeclareTextSymbol{\textthreequartersemdash}{TS1}{22}




% not always set up at this point while not fully integrated
\providecommand\UnicodeEncodingName{TU}

\def\add@unicode@accent#1#2{%
  \if\relax\detokenize{#2}\relax^^a0\else#2\fi
  \char#1\relax}
\def\DeclareUnicodeAccent#1#2#3{%
  \DeclareTextCommand{#1}{#2}{\add@unicode@accent{#3}}%
}

% use \oldstylenums for these

\DeclareTextCommand{\textzerooldstyle} \UnicodeEncodingName{\oldstylenums{0}}
\DeclareTextCommand{\textoneoldstyle}  \UnicodeEncodingName{\oldstylenums{1}}
\DeclareTextCommand{\texttwooldstyle}  \UnicodeEncodingName{\oldstylenums{2}}
\DeclareTextCommand{\textthreeoldstyle}\UnicodeEncodingName{\oldstylenums{3}}
\DeclareTextCommand{\textfouroldstyle} \UnicodeEncodingName{\oldstylenums{4}}
\DeclareTextCommand{\textfiveoldstyle} \UnicodeEncodingName{\oldstylenums{5}}
\DeclareTextCommand{\textsixoldstyle}  \UnicodeEncodingName{\oldstylenums{6}}
\DeclareTextCommand{\textsevenoldstyle}\UnicodeEncodingName{\oldstylenums{7}}
\DeclareTextCommand{\texteightoldstyle}\UnicodeEncodingName{\oldstylenums{8}}
\DeclareTextCommand{\textnineoldstyle} \UnicodeEncodingName{\oldstylenums{9}}

% those have Unicode points

\DeclareTextSymbol{\textpilcrow}       \UnicodeEncodingName{"00B6}
\DeclareTextSymbol{\textborn}          \UnicodeEncodingName{"002A}
\DeclareTextSymbol{\textdied}          \UnicodeEncodingName{"2020}
\DeclareTextSymbol{\textlbrackdbl}     \UnicodeEncodingName{"27E6}
\DeclareTextSymbol{\textrbrackdbl}     \UnicodeEncodingName{"27E7}
\DeclareTextSymbol{\textguarani}       \UnicodeEncodingName{"20B2}

% this can't be set up if TU isn't loaded in the test suite config
%
% right now 20DD doesn't exist in LM, so maybe not a good idea at the moment
% maybe test for the slot and do the fallback oalign if it doesn't exist
%
  \expandafter\ifx\csname TU-cmd\endcsname\relax
  \else
    \DeclareUnicodeAccent{\textcircled}    \UnicodeEncodingName{"20DD}
  \fi

% we could make them point to dollar and cent glyphs in TU

%\DeclareTextSymbol{\textcentoldstyle}            \UnicodeEncodingName{"00A2}
%\DeclareTextSymbol{\textdollaroldstyle}          \UnicodeEncodingName{"0024}


% but I think it is better to pick them up from TS1 even if that usually means LMR fonts

\DeclareTextSymbol{\textdollaroldstyle}{TS1}{138}
\DeclareTextSymbol{\textcentoldstyle}  {TS1}{139}


\fi  % --- END of Unicode engines


% \subsection{Font sub-encodings setup

% we declare the subsets of a good number of fonts in the kernel ...

% the default:
\DeclareEncodingSubset{TS1}{?}{4}  %FMI not sure 0/1  instead?

\DeclareEncodingSubset{TS1}{cmr}     {0}
\DeclareEncodingSubset{TS1}{cmss}    {0}
\DeclareEncodingSubset{TS1}{cmtt}    {0}
\DeclareEncodingSubset{TS1}{cmvtt}   {0}
\DeclareEncodingSubset{TS1}{cmbr}    {0}
\DeclareEncodingSubset{TS1}{cmtl}    {0}
\DeclareEncodingSubset{TS1}{ccr}     {0}
\DeclareEncodingSubset{TS1}{pxr}     {0}
\DeclareEncodingSubset{TS1}{pxss}    {0}
\DeclareEncodingSubset{TS1}{pxtt}    {0}
\DeclareEncodingSubset{TS1}{txr}     {0}
\DeclareEncodingSubset{TS1}{txss}    {0}
\DeclareEncodingSubset{TS1}{txtt}    {0}
\DeclareEncodingSubset{TS1}{lmr}     {0}
\DeclareEncodingSubset{TS1}{lmdh}    {0}
\DeclareEncodingSubset{TS1}{lmss}    {0}
\DeclareEncodingSubset{TS1}{lmssq}   {0}
\DeclareEncodingSubset{TS1}{lmvtt}   {0}
\DeclareEncodingSubset{TS1}{lmtt}    {0}
\DeclareEncodingSubset{TS1}{qhv}     {0}
\DeclareEncodingSubset{TS1}{qag}     {0}
\DeclareEncodingSubset{TS1}{qbk}     {0}
\DeclareEncodingSubset{TS1}{qcr}     {0}
\DeclareEncodingSubset{TS1}{qcs}     {0}
\DeclareEncodingSubset{TS1}{qpl}     {0}
\DeclareEncodingSubset{TS1}{qtm}     {0}
\DeclareEncodingSubset{TS1}{qzc}     {0}
\DeclareEncodingSubset{TS1}{qhvc}    {0}

\DeclareEncodingSubset{TS1}{ppl}     {8}
\DeclareEncodingSubset{TS1}{dayrom}  {8}
\DeclareEncodingSubset{TS1}{dayroms} {8}
\DeclareEncodingSubset{TS1}{hlh}     {8}
\DeclareEncodingSubset{TS1}{hls}     {8}
\DeclareEncodingSubset{TS1}{hlst}    {8}
\DeclareEncodingSubset{TS1}{pplx}    {8}
\DeclareEncodingSubset{TS1}{pplj}    {8}

\DeclareEncodingSubset{TS1}{ptmx}    {9}
\DeclareEncodingSubset{TS1}{ptmj}    {9}
\DeclareEncodingSubset{TS1}{ptm}     {9}
\DeclareEncodingSubset{TS1}{pcr}     {9}
\DeclareEncodingSubset{TS1}{phv}     {9}
\DeclareEncodingSubset{TS1}{pag}     {9}
\DeclareEncodingSubset{TS1}{ul8}     {9}
\DeclareEncodingSubset{TS1}{futs}    {9}
\DeclareEncodingSubset{TS1}{futx}    {9}
\DeclareEncodingSubset{TS1}{futj}    {9}
\DeclareEncodingSubset{TS1}{ul9}     {9}  % (LuxiSans, one day)
\DeclareEncodingSubset{TS1}{pbk}     {9}
\DeclareEncodingSubset{TS1}{pnc}     {9}
\DeclareEncodingSubset{TS1}{pzc}     {9}
\DeclareEncodingSubset{TS1}{bch}     {9}

\DeclareEncodingSubset{TS1}{put}     {10}
\DeclareEncodingSubset{TS1}{uag}     {10}
\DeclareEncodingSubset{TS1}{ugq}     {10}
\DeclareEncodingSubset{TS1}{augie}   {10}
\DeclareEncodingSubset{TS1}{hlct}    {10}
\DeclareEncodingSubset{TS1}{hlx}     {10}
\DeclareEncodingSubset{TS1}{hlce}    {10}
\DeclareEncodingSubset{TS1}{hlcn}    {10}
\DeclareEncodingSubset{TS1}{hlcw}    {10}
\DeclareEncodingSubset{TS1}{hlcf}    {10}



%? \DeclareEncodingSubset{TS1}{fbb-LF}  % centoldstyle
\DeclareEncodingSubset{TS1}{lato-LF}     {0}
\DeclareEncodingSubset{TS1}{opensans-LF} {0}
\DeclareEncodingSubset{TS1}{tli}         {0}
\DeclareEncodingSubset{TS1}{cantarell-LF}{0}


\DeclareEncodingSubset{TS1}{Alegreya-LF}              {2}
\DeclareEncodingSubset{TS1}{AlegreyaSans-LF}          {2}
\DeclareEncodingSubset{TS1}{Cinzel-LF}                {2}
\DeclareEncodingSubset{TS1}{CinzelDecorative-LF}      {2}
\DeclareEncodingSubset{TS1}{CormorantGaramond-LF}     {2}
\DeclareEncodingSubset{TS1}{DejaVuSans-TLF}           {2}
\DeclareEncodingSubset{TS1}{DejaVuSansCondensed-TLF}  {2}
\DeclareEncodingSubset{TS1}{DejaVuSansMono-TLF}       {2}
\DeclareEncodingSubset{TS1}{EBGaramond-LF}            {2}
\DeclareEncodingSubset{TS1}{Heuristica-TLF}           {2}
\DeclareEncodingSubset{TS1}{IMFELLEnglish-TLF}        {2}
\DeclareEncodingSubset{TS1}{LibreBaskerville-TLF}     {2}
\DeclareEncodingSubset{TS1}{LibreCaslon-LF}           {2}
\DeclareEncodingSubset{TS1}{Marcellus-LF}             {2}
\DeclareEncodingSubset{TS1}{NotoSans-LF}              {2}
\DeclareEncodingSubset{TS1}{NotoSansMono-TLF}         {2}
\DeclareEncodingSubset{TS1}{NotoSerif-LF}             {2}
\DeclareEncodingSubset{TS1}{Quattrocento-TLF}         {2}
\DeclareEncodingSubset{TS1}{QuattrocentoSans-TLF}     {2}
\DeclareEncodingSubset{TS1}{Tempora-TLF}              {2}
\DeclareEncodingSubset{TS1}{XCharter-TLF}             {2}


\DeclareEncodingSubset{TS1}{FiraSans-LF}              {3}
\DeclareEncodingSubset{TS1}{FiraSans-TLF}             {3}
\DeclareEncodingSubset{TS1}{Arimo-TLF}                {3}
\DeclareEncodingSubset{TS1}{Carlito-LF}               {3}
\DeclareEncodingSubset{TS1}{IBMPlexSans-TLF}          {3}
\DeclareEncodingSubset{TS1}{Montserrat-LF}            {3}
\DeclareEncodingSubset{TS1}{MontserratAlternates-LF}  {3}
\DeclareEncodingSubset{TS1}{SourceCodePro-TLF}        {3}
\DeclareEncodingSubset{TS1}{SourceSansPro-LF}         {3}
\DeclareEncodingSubset{TS1}{SourceSerifPro-LF}        {3}
\DeclareEncodingSubset{TS1}{Tinos-TLF}                {3}

\DeclareEncodingSubset{TS1}{AccanthisADFStdNoThree-LF}{4}
\DeclareEncodingSubset{TS1}{Cabin-TLF}                {4}
\DeclareEncodingSubset{TS1}{Caladea-TLF}              {4}
\DeclareEncodingSubset{TS1}{Chivo-LF}                 {4}
\DeclareEncodingSubset{TS1}{ClearSans-TLF}            {4}
\DeclareEncodingSubset{TS1}{Coelacanth-LF}            {4}
\DeclareEncodingSubset{TS1}{CrimsonPro-LF}            {4}
\DeclareEncodingSubset{TS1}{FiraMono-TLF}             {4}
\DeclareEncodingSubset{TS1}{Go-TLF}                   {4}
\DeclareEncodingSubset{TS1}{GoMono-TLF}               {4}
\DeclareEncodingSubset{TS1}{InriaSans-LF}             {4}
\DeclareEncodingSubset{TS1}{InriaSerif-LF}            {4}
\DeclareEncodingSubset{TS1}{LibertinusSans-LF}        {4}
\DeclareEncodingSubset{TS1}{LibertinusSerif-LF}       {4}
\DeclareEncodingSubset{TS1}{LibreBodoni-TLF}          {4}
\DeclareEncodingSubset{TS1}{LibreFranklin-TLF}        {4}
\DeclareEncodingSubset{TS1}{LinguisticsPro-LF}        {4}
\DeclareEncodingSubset{TS1}{LinuxBiolinumT-LF}        {4}
\DeclareEncodingSubset{TS1}{LinuxLibertineT-LF}       {4}
\DeclareEncodingSubset{TS1}{MintSpirit-LF}            {4}
\DeclareEncodingSubset{TS1}{MintSpiritNoTwo-LF}       {4}
\DeclareEncodingSubset{TS1}{PTMono-TLF}               {4}
\DeclareEncodingSubset{TS1}{PTSans-TLF}               {4}
\DeclareEncodingSubset{TS1}{PTSansCaption-TLF}        {4}
\DeclareEncodingSubset{TS1}{PTSansNarrow-TLF}         {4}
\DeclareEncodingSubset{TS1}{PTSerif-TLF}              {4}
\DeclareEncodingSubset{TS1}{PTSerifCaption-TLF}       {4}
\DeclareEncodingSubset{TS1}{Raleway-TLF}              {4}
\DeclareEncodingSubset{TS1}{Roboto-LF}                {4}
\DeclareEncodingSubset{TS1}{RobotoMono-TLF}           {4}
\DeclareEncodingSubset{TS1}{RobotoSlab-TLF}           {4}
\DeclareEncodingSubset{TS1}{Rosario-LF}               {4}
\DeclareEncodingSubset{TS1}{SticksTooText-LF}         {4}
\DeclareEncodingSubset{TS1}{UniversalisADFStd-LF}     {4}


\DeclareEncodingSubset{TS1}{Baskervaldx-LF}           {5}
\DeclareEncodingSubset{TS1}{BaskervilleF-LF}          {5}
\DeclareEncodingSubset{TS1}{Bitter-TLF}               {5}
\DeclareEncodingSubset{TS1}{Crimson-TLF}              {5}
\DeclareEncodingSubset{TS1}{DejaVuSerif-TLF}          {5}
\DeclareEncodingSubset{TS1}{DejaVuSerifCondensed-TLF} {5}
\DeclareEncodingSubset{TS1}{GilliusADF-LF}            {5}
\DeclareEncodingSubset{TS1}{GilliusADFCond-LF}        {5}
\DeclareEncodingSubset{TS1}{GilliusADFNoTwo-LF}       {5}
\DeclareEncodingSubset{TS1}{GilliusADFNoTwoCond-LF}   {5}
\DeclareEncodingSubset{TS1}{IBMPlexMono-TLF}          {5}
\DeclareEncodingSubset{TS1}{IBMPlexSerif-TLF}         {5}
\DeclareEncodingSubset{TS1}{LibertinusMono-TLF}       {5}
\DeclareEncodingSubset{TS1}{LibertinusSerifDisplay-LF}{5}
\DeclareEncodingSubset{TS1}{LinuxLibertineDisplayT-LF}{5}
\DeclareEncodingSubset{TS1}{LinuxLibertineMonoT-LF}   {5}
\DeclareEncodingSubset{TS1}{OldStandard-TLF}          {5}
\DeclareEncodingSubset{TS1}{Overlock-LF}              {5}
\DeclareEncodingSubset{TS1}{PlayfairDisplay-TLF}      {5}
\DeclareEncodingSubset{TS1}{TheanoDidot-TLF}          {5}
\DeclareEncodingSubset{TS1}{TheanoModern-TLF}         {5}
\DeclareEncodingSubset{TS1}{TheanoOldStyle-TLF}       {5}



% Fix those
\DeclareEncodingSubset{TS1}{erewhon-LF}               {5} 
\DeclareEncodingSubset{TS1}{AlgolRevived-TLF}         {7}
\DeclareEncodingSubset{TS1}{Cochineal-LF}             {8}



% for item labels provide:

\def\labelitemfont{\fontseries\seriesdefault\fontshape\shapedefault\selectfont}

% or

\def\labelitemfont{\normalfont}

% and use in \labelitem... in classes

% footnote symbols already use \normalfont hardwired (and that should
% probably stay like that).


% maybe this should also be available in Unicode engines. There a
% default is not enough since the symbols are defined in TU so one
% would need to undefine them there or perhaps better redefine them
% there.

\def\UseLegacyTextSymbols{%
  \DeclareTextSymbolDefault{\textasteriskcentered}{OMS}%
  \DeclareTextSymbolDefault{\textbardbl}{OMS}%
  \DeclareTextSymbolDefault{\textbullet}{OMS}%
  \DeclareTextSymbolDefault{\textdaggerdbl}{OMS}%
  \DeclareTextSymbolDefault{\textdagger}{OMS}%
  \DeclareTextSymbolDefault{\textparagraph}{OMS}%
  \DeclareTextSymbolDefault{\textperiodcentered}{OMS}%
  \DeclareTextSymbolDefault{\textsection}{OMS}%
  \UndeclareTextCommand{\textsection}{T1}%
}


% we go the roundabout way via separate OMS declarations so that
%   \renewcommand\textdagger{\textlegacydagger}
% doesn't prudce an endless loop

\DeclareTextSymbol{\textlegacyasteriskcentered}{OMS}{3}   % "03
\DeclareTextSymbol{\textlegacybardbl}{OMS}{107}           % "6B
\DeclareTextSymbol{\textlegacybullet}{OMS}{15}            % "0F
\DeclareTextSymbol{\textlegacydaggerdbl}{OMS}{122}        % "7A
\DeclareTextSymbol{\textlegacydagger}{OMS}{121}           % "79
\DeclareTextSymbol{\textlegacyparagraph}{OMS}{123}        % "7B
\DeclareTextSymbol{\textlegacyperiodcentered}{OMS}{1}     % "01
\DeclareTextSymbol{\textlegacysection}{OMS}{120}          % "78

\DeclareTextSymbolDefault{\textlegacyasteriskcentered}{OMS}
\DeclareTextSymbolDefault{\textlegacybardbl}{OMS}
\DeclareTextSymbolDefault{\textlegacybullet}{OMS}
\DeclareTextSymbolDefault{\textlegacydaggerdbl}{OMS}
\DeclareTextSymbolDefault{\textlegacydagger}{OMS}
\DeclareTextSymbolDefault{\textlegacyparagraph}{OMS}
\DeclareTextSymbolDefault{\textlegacyperiodcentered}{OMS}
\DeclareTextSymbolDefault{\textlegacysection}{OMS}


% kernel leftover .. this should be like this these days ... not faked

\DeclareTextSymbolDefault{\textcompwordmark}{T1}



\makeatother    

\endinput

