% \iffalse meta-comment
%
%% File: lthooks.dtx (C) Copyright 2020 Frank Mittelbach, LaTeX Team
%
% It may be distributed and/or modified under the conditions of the
% LaTeX Project Public License (LPPL), either version 1.3c of this
% license or (at your option) any later version.  The latest version
% of this license is in the file
%
%    https://www.latex-project.org/lppl.txt
%
%
% The development version of the bundle can be found below
%
%    https://github.com/FrankMittelbach/...
%
% for those people who are interested or want to report an issue.
%
%    \begin{macrocode}
\providecommand\lthooksversion{v0.8v}
\providecommand\lthooksdate{2020/07/13}
%    \end{macrocode}
%
%<*driver>
\RequirePackage[debug]{lthooks}

\documentclass{l3doc}

% bug fix fo l3doc.cls
\ExplSyntaxOn
\cs_set_protected:Npn \__codedoc_macro_typeset_one:nN #1#2
  {
    \vbox_set:Nn \l__codedoc_macro_box
      {
        \vbox_unpack_drop:N \l__codedoc_macro_box
        \hbox { \llap { \__codedoc_print_macroname:nN {#1} #2
            \MacroFont       % <----- without it the \ is in lmr10 if a link is made
            \      
        } }
      }
    \int_incr:N \l__codedoc_macro_int
  }
\ExplSyntaxOff

\EnableCrossrefs
\CodelineIndex
\begin{document}
  \DocInput{lthooks.dtx}
\end{document}
%</driver>
%
% \fi
%
%
% \long\def\fmi#1{\begin{quote}\itshape FMi: #1\end{quote}}
% \long\def\pho#1{\begin{quote}\itshape PhO: #1\end{quote}}
%
% \newcommand\hook[1]{\texttt{#1}}
%    
%
% \title{The \texttt{lthooks} package\thanks{This package has version
%    \lthooksversion\ dated \lthooksdate, \copyright\ \LaTeX\
%    Project.}}
%
% \author{Frank Mittelbach\thanks{Code improvements for speed and other goodies by Phelype Oleinik}}
%
% \maketitle
%
%
% \tableofcontents
%
% \section{Introduction}
%
%    Hooks are points in the code of commands or environments where it
%    is possible to add processing code into existing commands. This
%    can be done by different packages that do not know about each
%    other and to allow for hopefully safe processing it is necessary
%    to sort different chunks of code added by different packages into
%    a suitable processing order.
%
%    This is done by the packages adding chunks of code (via
%    \cs{AddToHook}) and labeling their code with some label by
%    default using the package name as a label.
%
%    At \verb=\begin{document}= all code for a hook is then sorted
%    according to some rules (given by \cs{DeclareHookRule}) for fast
%    execution without processing overhead. If the hook code is
%    modified afterwards (or the rules are changed),
%    a new version for fast processing is generated.
%
%    Some hooks are used already in the preamble of the document. If
%    that happens then the hook is prepared for execution (and sorted)
%    already at that point.
%
%
% \section{Package writer interface}
%
%    The hook management system is offered as a set of CamelCase
%    commands for traditional \LaTeXe{} packages (and for use in the
%    document preamble if needed) as well as \texttt{expl3} commands
%    for modern packages, that use the L3 programming layer of
%    \LaTeX{}. Behind the scenes, a single set of data structures is
%    accessed so that packages from both worlds can coexist and access
%    hooks in other packages.
%
%
%
% \subsection{\LaTeXe\ interfaces}
%
% \subsubsection{Declaring hooks and using them in code}
%
%    With two exceptions, hooks have to be declared before they can be
%    used. The exceptions are hooks in environments (i.e., executed at
%    \cs{begin} and \cs{end}) and hooks run when loading files,
%    e.g. before and after a package is loaded, etc. Their hook names
%    depend on the environment or the file name and so declaring them
%    beforehand is difficult.
%
%
% \begin{function}{\NewHook}
%   \begin{syntax}
%     \cs{NewHook} \Arg{hook}
%   \end{syntax}
%   Creates a new \meta{hook}.
%    If this is a hook provided as part of a package it is suggested
%    that the \meta{hook} name is always structured as follows:
%    \meta{package-name}\texttt{/}\meta{hook-name}. If necessary you
%    can further subdivide the name by adding more \texttt{/} parts.
%    If a hook name is already taken, an error is raised and the hook
%    is not created.
%
%    The \meta{hook} can be specified using the dot-syntax to denote
%    the current package name. See section~\ref{sec:default-label}.
% \end{function}
%
% \begin{function}{\NewReversedHook}
%   \begin{syntax}
%     \cs{NewReversedHook} \Arg{hook}
%   \end{syntax}
%     Like \cs{NewHook} declares a new \meta{hook}.
%     the difference is that the code chunks for this hook are in
%     reverse order by default (those added last are executed first).
%     Any rules for the hook are applied after the default ordering.
%     See sections~\ref{sec:order} and \ref{sec:reversed-order}
%    for further details.
%
%    The \meta{hook} can be specified using the dot-syntax to denote
%    the current package name. See section~\ref{sec:default-label}.
% \end{function}
%
%
% \begin{function}{\NewMirroredHookPair}
%   \begin{syntax}
%     \cs{NewMirroredHookPair} \Arg{hook-1} \Arg{hook-2}
%   \end{syntax}
%     A shorthand for
%    \cs{NewHook}\Arg{hook-1}\cs{NewReversedHook}\Arg{hook-2}.
%
%    The \meta{hooks} can be specified using the dot-syntax to denote
%    the current package name. See section~\ref{sec:default-label}.
% \end{function}
%



% \begin{function}{\UseHook}
%   \begin{syntax}
%     \cs{UseHook} \Arg{hook}
%   \end{syntax}
%    Execute the hook code inside a command or environment.\footnote{For
%     legacy hooks such as \hook{begindocument} it is also
%    possible to call \cs{@...hook}, e.g., \cs{@begindocumenthook},
%    but this syntax is discouraged.}
%
%    Before \verb=\begin{document}= the fast execution code for a hook
%    is not set up, so in order to use a hook there it is explicitly
%    initialized first. As that involves assignments using a hook at
%    those times is not 100\% the same as using it after
%    \verb=\begin{document}=.
%
%    The \meta{hook} \emph{cannot} be specified using the dot-syntax.
%    A leading |.| is treated literally.
% \end{function}
%
% \begin{function}{\UseOneTimeHook}
%   \begin{syntax}
%     \cs{UseOneTimeHook} \Arg{hook}
%   \end{syntax}
%    Some hooks are only used (and can be only used) in one place, for
%    example, those in \verb=\begin{document}= or
%    \verb=\end{document}=. Once we have passed that point adding to
%    the hook through a defined \cs{\meta{addto-cmd}} command (e.g.,
%    \cs{AddToHook} or \cs{AtBeginDocument}, etc.\@) would have no
%    effect (as would the use of such a command inside the hook code
%    itself). It is therefore customary to redefine
%    \cs{\meta{addto-cmd}} to simply  process its argument, i.e.,
%    essentially make it behave like \cs{@firstofone}.
%
%    \cs{UseOneTimeHook} does that: it records that the hook has been
%    consumed and any further attempt to add to it will result in
%    executing the code to be added immediately.
%
%    \fmi{Maybe add an error version as well?}
%
%    The \meta{hook} \emph{cannot} be specified using the dot-syntax.
%    A leading |.| is treated literally.
% \end{function}
%
%
% \subsubsection{Updating code for hooks}
%
% \begin{function}{\AddToHook}
%   \begin{syntax}
%     \cs{AddToHook} \Arg{hook}\oarg{label}\Arg{code}
%   \end{syntax}
%    Adds \meta{code} to the \meta{hook} labeled by \meta{label}. If
%    the optional argument \meta{label} is not provided, if \cs{AddToHook}
%    is used in a package/class, then the current
%    package/class name is used, otherwise \hook{top-level} is
%    used~(see section~\ref{sec:default-label}).
%
%    If there already exists code under the \meta{label} then the new
%    \meta{code} is appended to the existing one (even if this is a reversed hook).
%    If you want to replace existing code under the
%    \meta{label}, first apply \cs{RemoveFromHook}.
%
%    The hook doesn't have to exist for code to be added to
%    it. However, if it is not declared later then obviously the
%    added \meta{code} will never be executed.  This
%    allows for hooks to work regardless of package loading order and
%    enables packages to add to hook of other packages without
%    worrying whether they are actually used in the current document.
%    See section~\ref{sec:querying}.
%
%    The \meta{hook} and \meta{label} can be specified using the
%    dot-syntax to denote the current package name.
%    See section~\ref{sec:default-label}.
% \end{function}
%
% \begin{function}{\RemoveFromHook}
%   \begin{syntax}
%     \cs{RemoveFromHook} \Arg{hook}\oarg{label}
%   \end{syntax}
%    Removes any code labeled by \meta{label} from the \meta{hook}.
%    If the optional
%    argument \meta{label} is not provided, if \cs{AddToHook}
%    is used in a package/class, then the current
%    package/class name is used, otherwise \hook{top-level} is used.
%
%    If the optional argument is \texttt{*}, then all code chunks are
%    removed. This is rather dangerous as it drops code from other
%    packages one may not know about!
%
%    The \meta{hook} and \meta{label} can be specified using the
%    dot-syntax to denote the current package name.
%    See section~\ref{sec:default-label}.
% \end{function}
%
% \medskip
%
% A useful application for this declaration inside the document body
% is when one wants to temporarily add code to hooks and later remove
% it again, e.g.,
%\begin{verbatim}
%   \AddToHook{env/quote/before}{\small}
%   \begin{quote}
%     A quote set in a smaller typeface
%   \end{quote}
%   ...
%   \RemoveFromHook{env/quote/before}
%   ... now back to normal for further quotes
%\end{verbatim}
% Note that you can't cancel the setting with
%\begin{verbatim}
%   \AddToHook{env/quote/before}{}
%\end{verbatim}
% because that only \enquote{adds} a further empty chunk of code to
% the hook. Adding \cs{normalsize} would work but that means the hook
% then contained \cs{small}\cs{normalsize} which means to font size
% changes for no good reason.
%
% The above is only needed if one wants to typeset several quotes in a
% smaller typeface. If the hook is only needed once then
% \cs{AddToHookNext} is simpler, because it resets itself after one use.
%
%
% \begin{function}{\AddToHookNext}
%   \begin{syntax}
%     \cs{AddToHookNext} \Arg{hook}\Arg{code}
%   \end{syntax}
%    Adds \meta{code} to the next invocation of the \meta{hook}.
%    The code is executed after the normal hook code has finished and
%    it is executed only once, i.e. it is deleted after it was used.
%
%    Using the declaration is a global operation, i.e., the code is
%    not lost, even if the declaration is used inside a group and the
%    next invocation happens after the group. If the declaration is
%    used several times before the hook is executed then all code is
%    executed in the order in which it was declared.\footnotemark
%
%    The hook doesn't have to exist for code to be added to it.  This
%    allows for hooks to work regardless of package loading order.
%    See section~\ref{sec:querying}.
%
%    The \meta{hook} can be specified using the dot-syntax to denote
%    the current package name.  See section~\ref{sec:default-label}.
% \end{function}\footnotetext{There is
%    no mechanism to reorder such code chunks (or delete them).}
%
% \subsection{Hook names and default labels}
% \label{sec:default-label}
%
% It is best practice to use \cs{AddToHook} in packages or classes
% \emph{without specifying a \meta{label}} because then the package
% or class name is automatically used, which is helpful if rules are
% needed, and avoids mistyping the \meta{label}.
%
% Using an explicit \meta{label} is only necessary in very specific
% situations, e.g., if you want to add several chunks of code into a
% single hook and have them placed in different parts of the hook
% (by providing some rules).
%
% The other case is when you develop a larger package with several
% sub-packages. In that case you may want to use the same
% \meta{label} throughout the sub-packages in order to avoid
% that the labels change if you internally reorganize your code.
%
% It is not enforced, but highly recommended that the hooks defined by
% a package, and the \meta{labels} used to add code to other hooks
% contain the package name to easily identify the source of the code
% chunk and to prevent clashes.  This should be the standard practice,
% so this hook management code provides a shortcut to refer to the
% current package in the name of a \meta{hook} and in a \meta{label}.
% If \meta{hook} name or \meta{label} consist just of a single dot
% (|.|), or starts with a dot followed by a slash (|./|) then the dot
% denotes the \meta{default label} (usually the current package or class
% name---see~\cs{DeclareDefaultHookLabel}).
% A \enquote{|.|} or \enquote{|./|} anywhere else in a \meta{hook} or in
% \meta{label} is treated literally and is not replaced.

% For example,
% inside the package \texttt{mypackage.sty}, the default label is
% \texttt{mypackage}, so the instructions:
% \begin{verbatim}
%   \NewHook   {./hook}
%   \AddToHook {./hook}[.]{code}     % Same as \AddToHook{./hook}{code}
%   \AddToHook {./hook}[./sub]{code}
%   \DeclareHookRule{begindocument}{.}{<}{babel}
%   \AddToHook {file/after/foo.tex}{code}
% \end{verbatim}
%    are equivalent to:
% \begin{verbatim}
%   \NewHook   {mypackage/hook}[mypackage]{code}
%   \AddToHook {mypackage/hook}[mypackage]{code}
%   \AddToHook {mypackage/hook}[mypackage/sub]{code}
%   \DeclareHookRule{begindocument}{mypackage}{<}{babel}
%   \AddToHook {file/after/foo.tex}{code}                  % unchanged
% \end{verbatim}
%
% The \meta{default label} is automatically set to the name of the
% current package or class (using \cs{@currname}).  If \cs{@currname}
% is not set (because the hook command is used outside of a package, or
% the current file wasn't loaded with \cs{usepackage} or
% \cs{documentclass}), then the \texttt{top-level} is used as the
% \meta{default label}.
%
% This syntax is available in all \meta{label} arguments and most
% \meta{hook}, both in the \LaTeXe{} interface, and the \LaTeX3
% interface described in section~\ref{sec:l3hook-interface}.
%
% Note, however, that the replacement of |.| by the \meta{default label}
% takes place when the hook command is executed, so actions that are
% somehow executed after the package ends will have the wrong
% \meta{default label} if the dot-syntax is used.  For that reason,
% this syntax is not available in \cs{UseHook} (and \cs{hook_use:n})
% because the hook is most of the time used outside of the package file
% in which it was defined. This syntax is also not available in the hook
% conditionals \cs{IfHookEmptyTF} (and \cs{hook_if_empty:nTF}) and
% \cs{IfHookExistTF} (and \cs{hook_if_exist:nTF}) because these
% conditionals are used in some performance-critical parts of the hook
% management code, and because they are usually used to refer to other
% package's hooks, so the dot-syntax doesn't make much sense.
%
% In some cases, for example in large packages, one may want to separate
% it in logical parts, but still use the main package name as
% \meta{label}, then the \meta{default label} can be set using
% \cs{DeclareDefaultHookLabel}:
%
% \begin{function}{\DeclareDefaultHookLabel}
%   \begin{syntax}
%     \cs{DeclareDefaultHookLabel} \Arg{default label}
%   \end{syntax}
%   Sets the \meta{default label} to be used in \meta{label} arguments.
%   If \cs{DeclareDefaultHookLabel} is not used in the
%   current package, \cs{@currname} is used instead.  If \cs{@currname}
%   is not set, the code is assumed to be in the main document, in which
%   case \texttt{top-level} is used.
%
%   The effect of \cs{DeclareDefaultHookLabel} holds for the current
%   file, and is reset to the previous value when the file is closed.
% \end{function}
%
%
% \subsubsection{Defining relations between hook code}
%
% The default assumption is that code added to hooks by different
% packages is indepedent and the order in which it is executed is
% irrelevant. While this is true in many case it is  obviously false
% in many others.
%
% Before the hook management system was introduced
% packages had to take elaborate precaution to determine of some other
% package got loaded as well (before or after) and find some ways to
% alter its behavior accordingly. In addition is was often the user's
% responsibility to load packages in the right order so that code
% added to hooks got added in the right orde and some cases even
% altering the loading order wouldn't resolve the conflicts.
%
% With the new hook management system it is now possible to define
% rules (i.e., relationships) between code chunks added by different
% packages and explicitly describe in which order they should be
% processed.
%
% \begin{function}{\DeclareHookRule}
%   \begin{syntax}
%     \cs{DeclareHookRule} \Arg{hook}\Arg{label1}\Arg{relation}\Arg{label2}
%   \end{syntax}
%    Defines a relation between \meta{label1} and \meta{label2} for a
%    given \meta{hook}. If \meta{hook} is \texttt{??} this defines a
%    relation for all hooks that use the two labels, i.e., that have
%    chunks of code labeled with \meta{label1} and \meta{label2}.
%    Rules specific to a given hook take precedence over default
%    rules that use \texttt{??} as the \meta{hook}.
%
%    Currently, the supported relations are the following:
%    \begin{itemize}
%
%    \item[\texttt{before} or \texttt{\string<}]
%
%      Code for \meta{label1} comes before code for \meta{label2}.
%
%    \item[\texttt{after} or \texttt{\string>}]
%      Code for \meta{label1} comes after code for \meta{label2}.
%
%    \item[\texttt{incompatible-warning}]
%
%      Only code for either \meta{label1} or \meta{label2} can appear
%      for that hook (a way to say that two packages---or parts of
%      them---are incompatible). A warning is raised if both labels
%      appear in the same hook.
%
%    \item[\texttt{incompatible-error}]
%
%      Like \texttt{incompatible-error} but instead of a warning a
%      \LaTeX{} error is raised, and the code for both labels are
%      dropped from that hook until the conflict is resolved.
%
%    \item[\texttt{removes}]
%
%      Code for \meta{label1} overwrites code for \meta{label2}. More
%      precisely, code for \meta{label2} is dropped for that
%      hook. This can be used, for example if one package is a
%      superset in functionality of another one and therefore wants to
%      undo code in some hook and replace it with its own version.
%
%    \item[\texttt{unrelated}]
%
%       The order of code for \meta{label1} and \meta{label2} is
%      irrelevant. This rule is there to undo an incorrect rule
%      specified earlier.
%
%    \end{itemize}
%
%    The \meta{hook} and \meta{label} can be specified using the
%    dot-syntax to denote the current package name.
%    See section~\ref{sec:default-label}.
% \end{function}
%
%
% \begin{function}{\ClearHookRule}
%   \begin{syntax}
%     \cs{ClearHookRule}\Arg{hook}\Arg{label1}\Arg{label2}
%   \end{syntax}
%    Syntactic sugar for saying that \meta{label1} and \meta{label2}
%    are unrelated for the given \meta{hook}.
% \end{function}
%
%
%
% \begin{function}{\DeclareDefaultHookRule}
%   \begin{syntax}
%     \cs{DeclareDefaultHookRule}\Arg{label1}\Arg{relation}\Arg{label2}
%   \end{syntax}
%   This sets up a relation between \meta{label1} and \meta{label2}
%    for all hooks unless overwritten by a specific rule for a hook.
%    Useful for cases where one package has a specific relation to
%    some other package, e.g., is \texttt{incompatible} or always
%    needs a special ordering \texttt{before} or \texttt{after}.
%    (Technically it is just a shorthand for using \cs{DeclareHookRule}
%    with \texttt{??} as the hook name.)
%
%    Declaring default rules is only supported in the document
%    premable.\footnote{Trying to do so, e.g., via
%    \cs{DeclareHookRule} with \texttt{??}  has bad side-effects and
%    is not supported (though not explicitly caught for performance
%    reasons.}
%
%    The \meta{label} can be specified using the dot-syntax to denote
%    the current package name. See section~\ref{sec:default-label}.
% \end{function}
%
%
%
% \subsubsection{Querying hooks}
% \label{sec:querying}
%
% Simpler data types, like token lists, have three possible states; they
% can:
% \begin{itemize}
%   \item exist and be empty;
%   \item exist and be non-empty; and
%   \item not exist (in which case emptiness doesn't apply);
% \end{itemize}
% Hooks are a bit more complicated: they have four possible states.
% A hook may exist or not, and either way it may or may not be empty.
% This means that even a hook that doesn't exist may be non-empty.
%
% This seemingly strange state may happen when, for example, package~$A$
% defines hook \hook{A/foo}, and package $B$ adds some code to that
% hook.  However, a document may load package $B$ before package $A$, or
% may not load package $A$ at all.  In both cases some code is added to
% hook \hook{A/foo} without that hook being defined yet, thus that
% hook is said to be non-empty, whereas it doesn't exist.  Therefore,
% querying the existence of a hook doesn't imply its emptiness, neither
% does the other way around.
%
% A hook is said to be empty when no code was added to it, either to
% its permanent code pool, or to its ``next'' token list.  The hook
% doesn't need to be declared to have code added to its code pool.
% A hook is said to exist when it was declared with \cs{NewHook} or
% some variant thereof.
%
% \begin{function}[EXP]{\IfHookEmptyTF}
%   \begin{syntax}
%     \cs{IfHookEmptyTF} \Arg{hook} \Arg{true code} \Arg{false code}
%   \end{syntax}
%   Tests if the \meta{hook} is empty (\emph{i.e.}, no code was added to
%   it using either \cs{AddToHook} or \cs{AddToHookNext}), and
%   branches to either \meta{true code} or \meta{false code} depending
%   on the result.
%
%    The \meta{hook} \emph{cannot} be specified using the dot-syntax.
%    A leading |.| is treated literally.
% \end{function}
%
% \begin{function}[EXP]{\IfHookExistTF}
%   \begin{syntax}
%     \cs{IfHookExistTF} \Arg{hook} \Arg{true code} \Arg{false code}
%   \end{syntax}
%   Tests if the \meta{hook} exists (if it was created with either
%   \cs{NewHook}, \cs{NewReversedHook}, or \cs{NewMirroredHookPair}), and
%   branches to either \meta{true code} or \meta{false code} depending
%   on the result.
%
%   The existence of a hook usually doesn't mean much from the viewpoint
%   of code that tries to add/remove code from that hook, since package
%   loading order may vary, thus the creation of hooks is asynchronous
%   to adding and removing code from it, so this test should be used
%   sparingly.
%
%    The \meta{hook} \emph{cannot} be specified using the dot-syntax.
%    A leading |.| is treated literally.
% \end{function}
%
% \fmi{Would be helpful if we provide some use cases}
%
% \subsubsection{Displaying hook code}
%
%    If one has to adjust the code execution in a hook using a hook
%    rule it is helpful to get some information about the code
%    associated with a hook, its current order and the existing rules.
%
% \begin{function}{\ShowHook}
%   \begin{syntax}
%     \cs{ShowHook} \Arg{hook}
%   \end{syntax}
%   Displays information about the \meta{hook} such as
%   \begin{itemize}
%   \item
%      the code chunks (and their labels) added to it,
%   \item
%      any rules set up to order them,
%      \fmi{currently this is missing the default rules that apply,
%      guess that needs fixing}
%   \item
%      the computed order (if already defined),
%   \item
%      any code executed on the next invocation only.
%   \end{itemize}
%
%    The \meta{hook} can be specified using the dot-syntax to denote
%    the current package name. See section~\ref{sec:default-label}.
% \end{function}
%
%
% \subsubsection{Debugging hook code}
%
% \begin{function}{\DebugHookOn,\DebugHookOff}
%   \begin{syntax}
%     \cs{DebugHookOn}
%   \end{syntax}
%    Turn the debugging of hook code on or off. This displays changes
%    made to the hook data structures. The output is rather coarse and
%      not really intended for normal use.
% \end{function}
%
%
% \subsection{L3 programming layer (\texttt{expl3}) interfaces}
% \label{sec:l3hook-interface}
%
%
% This is a quick summary of the \LaTeX3 programming interfaces for
% use with packages written in \texttt{expl3}. In contrast to the
% \LaTeXe{} interfaces they always use mandatory arguments only, e.g.,
% you always have to specify the \meta{label} for a code chunk.  We
% therefore suggest to use the declarations discussed in the previous
% section even in \texttt{expl3} packages, but the choice is yours.
%
%
% \begin{function}
%   {\hook_new:n,\hook_new_reversed:n,\hook_new_pair:nn}
%   \begin{syntax}
%     \cs{hook_new:n}\Arg{hook}
%     \cs{hook_new_pair:nn}\Arg{hook-1}\Arg{hook-2}
%   \end{syntax}
%   Creates a new \meta{hook} with normal or reverse ordering of code
%    chunks. \cs{hook_new_pair:nn} creates a pair of such hooks with
%    \Arg{hook-2} being a reversed hook.
%    If a hook name is already taken, an error is raised and the hook
%    is not created.
%
%    The \meta{hook} can be specified using the dot-syntax to denote
%    the current package name. See section~\ref{sec:default-label}.
% \end{function}
%
%
%
% \begin{function}{\hook_use:n}
%   \begin{syntax}
%     \cs{hook_use:n} \Arg{hook}
%   \end{syntax}
%    Executes the \Arg{hook} code followed (if set up) by the code for next
%    invocation only, then empties that next invocation code.
%
%    The \meta{hook} \emph{cannot} be specified using the dot-syntax.
%    A leading |.| is treated literally.
% \end{function}
%
% \begin{function}{\hook_use_once:n}
%   \begin{syntax}
%     \cs{hook_use_once:n} \Arg{hook}
%   \end{syntax}
%     Changes the \Arg{hook} status so that from now on any addition to
%     the hook code is executed immediately. Then execute any
%     \Arg{hook} code already set up.
%    \fmi{better L3 name?}
%
%    The \meta{hook} \emph{cannot} be specified using the dot-syntax.
%    A leading |.| is treated literally.
% \end{function}
%
% \begin{function}{\hook_gput_code:nnn}
%   \begin{syntax}
%     \cs{hook_gput_code:nnn} \Arg{hook} \Arg{label} \Arg{code}
%   \end{syntax}
%    Adds a chunk of \meta{code} to the \meta{hook} labeled
%    \meta{label}. If the label already exists the \meta{code} is
%    appended to the already existing code.
%
%    If code is added to an external \meta{hook} (of the kernel or
%    another package) then the convention is to use the package name
%    as the \meta{label} not some internal module name or some other
%    arbitrary string.
%
%    The \meta{hook} and \meta{label} can be specified using the
%    dot-syntax to denote the current package name.
%    See section~\ref{sec:default-label}.
% \end{function}
%
% \begin{function}
%   {\hook_gput_next_code:nn}
%   \begin{syntax}
%     \cs{hook_gput_next_code:nn} \Arg{hook} \Arg{code}
%   \end{syntax}
%    Adds a chunk of \meta{code} for use only in the next invocation of the
%    \meta{hook}. Once used it is gone.
%
%    This is simpler than \cs{hook_gput_code:nnn}, the code is simply
%    appended to the hook in the order of declaration at the very end,
%    i.e., after all standard code for the hook got executed.
%
%    Thus if one needs to undo what the standard does one has to do
%    that as part of \meta{code}.
%
%    The \meta{hook} can be specified using the dot-syntax to denote
%    the current package name. See section~\ref{sec:default-label}.
% \end{function}
%


%  \begin{function}{\hook_gremove_code:nn}
%   \begin{syntax}
%     \cs{hook_gremove_code:nn} \Arg{hook} \Arg{label}
%   \end{syntax}
%    Removes any code for \meta{hook} labeled \meta{label}.
%
%    If the code for that \meta{label} wasn't yet added to the
%    \meta{hook}, an order is set so that when some code attempts to add
%    that label, the removal order takes action and the code is not
%    added.
%
%    If the second argument is \texttt{*}, then all code chunks are
%    removed. This is rather dangerous as it drops code from other
%    packages one may not know about, so think twice before using
%    that!
%
%    The \meta{hook} and \meta{label} can be specified using the
%    dot-syntax to denote the current package name.
%    See section~\ref{sec:default-label}.
% \end{function}


%  \begin{function}{\hook_gset_rule:nnnn}
%   \begin{syntax}
%     \cs{hook_gset_rule:nnnn} \Arg{hook} \Arg{label1} \Arg{relation} \Arg{label2}
%   \end{syntax}
%    Relate \meta{label1} with \meta{label2} when used in \meta{hook}.
%    See \cs{DeclareHookRule} for the allowed \meta{relation}s.
%    If \meta{hook} is \texttt{??} a default rule is specified.
%
%    The \meta{hook} and \meta{label} can be specified using the
%    dot-syntax to denote the current package name.
%    See section~\ref{sec:default-label}.
%    The dot-syntax is parsed in both \meta{label} arguments, but it
%    usually makes sense to be used in only one of them.
% \end{function}
%
% \begin{function}[pTF]{\hook_if_empty:n}
%   \begin{syntax}
%     \cs{hook_if_empty:n} \Arg{hook} \Arg{true code} \Arg{false code}
%   \end{syntax}
%   Tests if the \meta{hook} is empty (\emph{i.e.}, no code was added to
%   it using either \cs{AddToHook} or \cs{AddToHookNext}), and
%   branches to either \meta{true code} or \meta{false code} depending
%   on the result.
%
%    The \meta{hook} \emph{cannot} be specified using the dot-syntax.
%    A leading |.| is treated literally.
% \end{function}
%
% \begin{function}[pTF]{\hook_if_exist:n}
%   \begin{syntax}
%     \cs{hook_if_exist:n} \Arg{hook} \Arg{true code} \Arg{false code}
%   \end{syntax}
%   Tests if the \meta{hook} exists (if it was created with either
%   \cs{NewHook}, \cs{NewReversedHook}, or \cs{NewMirroredHookPair}), and
%   branches to either \meta{true code} or \meta{false code} depending
%   on the result.
%
%  \fmi{what are the results for generic hooks that do not need to be declared?}
%
%   The existence of a hook usually doesn't mean much from the viewpoint
%   of code that tries to add/remove code from that hook, since package
%   loading order may vary, thus the creation of hooks is asynchronous
%   to adding and removing code from it, so this test should be used
%   sparingly.
%
%    The \meta{hook} \emph{cannot} be specified using the dot-syntax.
%    A leading |.| is treated literally.
% \end{function}
%
%
% \begin{function}{\hook_debug_on:,\hook_debug_off:}
%   \begin{syntax}
%     \cs{hook_debug_on:}
%   \end{syntax}
%    Turns the debugging of hook code on or off. This displays changes
%    to the hook data.
% \end{function}
%
%
%
% \subsection{On the order of hook code execution} \label{sec:order}
%
%    Chunks of code for a \meta{hook} under different labels are supposed
%    to be independent if there are no special rules set up that
%    define a relation between the chunks. This means that you can't
%    make assumptions about the order of execution!
%
%    Suppose you have the following declarations:
%\begin{verbatim}
%    \NewHook{myhook}
%    \AddToHook{myhook}[packageA]{\typeout{A}}
%    \AddToHook{myhook}[packageB]{\typeout{B}}
%    \AddToHook{myhook}[packageC]{\typeout{C}}
%\end{verbatim}
%    then executing the hook with \cs{UseHook} will produce the
%    typeout \texttt{A} \texttt{B} \texttt{C} in that order.  In other
%    words, the execution order is computed to be \texttt{packageA},
%    \texttt{packageB}, \texttt{packageC} which you can verify with
%    \cs{ShowHook}\texttt{\{myhook\}}:
%\begin{verbatim}
%   The hook 'myhook':
%    Code chunks:
%       packageA -> \typeout {A}
%       packageB -> \typeout {B}
%       packageC -> \typeout {C}
%    Extra code next invocation:
%       ---
%    Rules:
%       ---
%    Execution order:
%       packageA, packageB, packageC
%\end{verbatim}
%    The reason is that the code chunks are internally saved in a property list
%    and the initial order of such a property list is the order in
%    which key-value pairs got added. However, that is only true if
%    nothing other than adding happens!
%
%    Suppose, or example, you want to replace the code chunk for
%    \texttt{packageA}, e.g.,
%\begin{verbatim}
%    \RemoveFromHook{myhook}[packageA]
%    \AddToHook{myhook}[packageA]{\typeout{A alt}}
%\end{verbatim}
%    then your order becomes  \texttt{packageB},
%    \texttt{packageC}, \texttt{packageA} because the label got removed
%    from the property list and then re-added (at its end).
%
%    While that may not be too surprising,  the execution order is
%    also sometimes altered if you add a redundant rule, e.g. if you specify
%\begin{verbatim}
%    \DeclareHookRule{myhook}{packageA}{before}{packageB}
%\end{verbatim}
%    instead of the previous lines we get
%\begin{verbatim}
%   The hook 'myhook':
%    Code chunks:
%       packageA -> \typeout {A}
%       packageB -> \typeout {B}
%       packageC -> \typeout {C}
%    Extra code next invocation:
%       ---
%    Rules:
%       packageA|packageB with relation before
%    Execution order (after applying rules):
%       packageA, packageC, packageB
%\end{verbatim}
%    As you can see the code chunks are still in the same order, but
%    in the execution order for the labels \texttt{packageB} and
%    \texttt{packageC} have
%    swapped places.
%    The reason is that, with the rule there are two orders that
%    satisfy it, and the algorithm for sorting happened to pick a
%    different one compared to the case without rules (where it
%    doesn't run at all as there is nothing to resolve).
%    Incidentally, if we had instead specified the redundant rule
%\begin{verbatim}
%    \DeclareHookRule{myhook}{packageB}{before}{label-3}
%\end{verbatim}
%    the execution order would not have changed.
%
%    In summary: it is not possible to rely on the order of execution
%    unless there are rules that partially or fully define the order
%    (in which you can rely on them being fulfilled).
%
%
% \subsection{The use of \enquote{reversed} hooks} \label{sec:reversed-order}
%
%    You may have wondered why you can declare a \enquote{reversed} hook
%    with \cs{NewReversedHook} and what that does exactly.
%
%    In short: the execution order of a reversed hook (without any
%    rules!) is exactly reversed to the order you would have gotten for
%    a hook declared with \cs{NewHook}.
%
%    This is helpful if you have a pair of hooks where you expect to see
%    code added that involves grouping, e.g., starting an environment
%    in the first and closing that environment in the second hook.
%    To give a somewhat contrived example\footnote{there are simpler
%    ways to achieve the same effect.}, suppose there is a package
%    adding the following:
%\begin{verbatim}
%    \AddToHook{env/quote/before}[package-1]{\begin{itshape}}
%    \AddToHook{env/quote/after} [package-1]{\end{itshape}}
%\end{verbatim}
%    As a result, all quotes will be in italics.
%    Now suppose further that the user wants the quotes also in blue
%    and therefore adds:
%\begin{verbatim}
%    \usepackage{color}
%    \AddToHook{env/quote/before}{\begin{color}{blue}}
%    \AddToHook{env/quote/after} {\end{color}}
%\end{verbatim}
%    Now if the \hook{env/quote/after} hook would be a normal hook we
%    would get the same execution order in  both hooks, namely:
%\begin{verbatim}
%    package-1, top-level
%\end{verbatim}
%    (or vice versa) and as a result, would get:
%\begin{verbatim}
%    \begin{itshape}\begin{color}{blue} ...
%    \end{itshape}\end{color}
%\end{verbatim}
%   and an error message that \verb=\begin{color}= ended by
%    \verb=\end{itshape}=.
%    With \hook{env/quote/after} declared as a reversed hook the
%    execution order is reversed and so all environments are closed in
%    the correct sequence and \cs{ShowHook} would give us the
%    following output:
%\begin{verbatim}
%   The hook 'env/quote/after':
%    Code chunks:
%       package-1 -> \end {itshape}
%       top-level -> \end {color}
%    Extra code next invocation:
%       ---
%    Rules:
%       ---
%    Execution order (after reversal):
%       top-level, package-1
%\end{verbatim}
%
%    The reversal of the execution order happens before applying any
%    rules, so if you alter the order you will probably have to alter
%    it in both hooks, not just in one, but that depends on the use case.
%
%
%
%
% \subsection{Private \LaTeX{} kernel hooks}
%
%    There are a few places where it is absolutely essential for
%    \LaTeX{} to function correctly that code is executed in a precisely
%    defined order. Even that could have been implemented with the
%    hook management (by adding various rules to ensure the
%    appropriate ordering with respect to other code added by
%    packages). However, this makes every document unnecessary
%    slow, because there has to be sorting even through the result is
%    predetermined. Furthermore it forces package writers to
%    unnecessarily add such rules if they add further code to the hook
%    (or break \LaTeX{}).
%
%    For that reason such code is not using the hook management, but
%    instead private kernel commands directly before or after a public
%    hook with the following naming
%    convention: \cs{@kernel@before@\meta{hookname}} or
%    \cs{@kernel@after@\meta{hookname}}. For example, in
%    \cs{enddocument} you find
%\begin{verbatim}
%   \UseHook{enddocument}%
%   \@kernel@after@enddocument
%\end{verbatim}
%    which means first the user/package-accessible \hook{enddocument}
%    hook is executed and then the internal kernel hook. As their name
%    indicates these kernel commands should not be altered by third-party
%    packages, so please refrain from that in the interest of
%    stability and instead use the public hook next to it.\footnote{As
%    with everything in \TeX{} there is not enforcement of this rule,
%    and by looking at the code it is easy to find out how the kernel
%    adds to them. The main reason of this section is therefore to say
%    \enquote{please don't do that, this is unconfigurable code!}}
%
%
%
% \subsection{Legacy \LaTeXe{} interfaces}
%
%  \LaTeXe{} offered a small number of hooks together with commands to
%    add to them. They are listed here and are retained for backwards
%    compatibility.
%
%  With the new hook management several additional hooks have been added
%    to \LaTeX\ and more will follow. See the next section for what
%    is already available.
%
%
% \begin{function}{\AtBeginDocument}
%   \begin{syntax}
%     \cs{AtBeginDocument} \oarg{label} \Arg{code}
%   \end{syntax}
%   If used without the optional argument \meta{label}, it works essentially
%    like before, i.e., it is adding \meta{code} to the hook
%    \hook{begindocument} 
%    (which is executed inside \verb=\begin{document}=).
%    However, all code added this way is labeled with the label
%    \hook{top-level} if done outside of a package or class or with the
%    package/class name if called inside such a file.
%
%    This way one can add further code to the hook using
%    \cs{AddToHook} or \cs{AtBeginDocument} using a different label
%    and explicitly order the code chunks as necessary, e.g., run some
%    code before or after the \hook{top-level} code.  When using the
%    optional argument the call is equivalent to running
%    \cs{AddToHook} \texttt{\{begindocument\}} \oarg{label}
%    \Arg{code}.
%
%    For important packages with known order requirement we may over
%    time add rules to the kernel (or to those packages) so that they
%    work regardless of the loading-order in the document.
% \end{function}
%
% \begin{function}{\AtEndDocument}
%   \begin{syntax}
%     \cs{AtEndDocument} \oarg{label} \Arg{code}
%   \end{syntax}
%   Like \cs{AtBeginDocument} but for the \hook{enddocument} hook.
% \end{function}
%
%
%
% \begin{function}{\AtBeginDvi}
%   \begin{syntax}
%     \cs{AtBeginDvi} \oarg{label} \Arg{code}
%   \end{syntax}
%   This hook is discussed in conjunction with the shipout hooks.
% \end{function}
%
%
%
% \subsection{\LaTeXe{} commands and environments augmented by
%    hooks}
%
%  \emph{intro to be written}
%
% \subsubsection{Generic hooks for all environments}
%
%    Every environment \meta{env} has now four associated hooks coming
%    with it:
%    \begin{description}
%    \item[\hook{env/\meta{env}/before}]
%
%       This hook is executed as part of \cs{begin} as the very first
%       action, in particular prior to starting the environment group.
%       Its scope is therefore not restricted by the environment.
%
%    \item[\hook{env/\meta{env}/begin}]
%
%       This hook is executed as part of \cs{begin} directly in front
%       of the code specific to the environment start (e.g., the
%       second argument of \cs{newenvironment}).  Its scope is the
%       environment body.
%
%    \item[\hook{env/\meta{env}/end}]
%
%       This is executed as part of \cs{end} directly in front of the
%       code specific to the end of the environment (e.g., the third
%       argument of \cs{newenvironment}).
%
%    \item[\hook{env/\meta{env}/after}]
%
%       This is executed as part of \cs{end} after the
%       code specific to the environment end and after the environment
%       group has ended.
%       Its scope is therefore not restricted by the environment.
%
%       This hook is implemented as a reversed hook so if two packages
%       add code to \hook{env/\meta{env}/before} and to
%       \hook{env/\meta{env}/after} they can add surrounding
%       environments and the order of closing them happens in the
%       right sequence.
%
%    \end{description}
%    In contrast to other hooks these hooks do not need to be declared
%    using \cs{NewHook}.
%
%    The hooks are only executed if \cs{begin}\Arg{env} and
%    \cs{end}\Arg{env} is used. If the environment code is executed
%    via low-level calls to \cs{\meta{env}} and \cs{end\meta{env}}
%    (e.g., to avoid the environment grouping) they are not
%    available. If you want them available in code using this method,
%    you would need to add them yourself, i.e., write something like
%\begin{verbatim}
%  \UseHook{env/quote/before}\quote
%      ...
%  \endquote\UseHook{env/quote/after}
%\end{verbatim}
%    to add the outer hooks, etc.
%
%
%
% \subsubsection{Hooks provided by \cs{begin}\texttt{\{document\}}}
%
%    Until 2020 \cs{begin}\texttt{\{document\}} offered exactly one
%    hook that one had to fill using \cs{AtBeginDocument}. Experience
%    has shown that this single hook in one place was not enough and
%    as part of adding the general hook management system a number of
%    additional hooks have been added at this point. The places for
%    hooks have been chosen to provide the same support as offered by
%    external packages, such as \pkg{etoolbox} and others that
%    augmented \cs{document} to gain better control.
%
%    Supported are now the following hooks:
%    \begin{description}
%
%    \item[\hook{env/document/before}]
%
%      This is the generic environment hook executed effectively
%      before \verb=\begin{document}= starts, i.e., one can think of
%      it as a hook for code at the end of the preamble section.
%
%    \item[\hook{env/document/begin}]
%
%      This is the second generic environment hook that is executed
%      after the environment has started its group. But given that for the
%      \texttt{document} environment this group is canceled there is
%      little difference to the previous one as the two are directly
%      executed one after another (the only difference is that in this
%      hook \cs{@currenvir} is now set to \texttt{document} but
%      anybody adding to this hook would know that already).
%
%
%    \item[\hook{begindocument}]
%
%      This hook is added to by \cs{AtBeginDocument} and is executed
%      after the \texttt{.aux} file as be read in and most
%      initialization are done, so they can be altered and inspeced by
%      the hook code. It is followed by a small number of further
%      initializations that shouldn't be altered and are therefore
%      coming later.
%
%    \item[\hook{begindocument/end}]
%
%      This hook is executed at the end of the \cs{document} code in
%      other words at the beginning of the document body. The only
%      command that follows it is \cs{ignorespaces}.
%
%    \end{description}
%
%
%
%
% \subsubsection{Hooks provided by \cs{end}\texttt{\{document\}}}
%
%    \LaTeXe{} always provided \cs{AtEndDocument} to add code to the
%    execution of \verb=\end{document}= just in front of the code that
%    is normally executed there. While this was a big improvement over
%    the situation in \LaTeX\,2.09 it was not flexible enough for a
%    number of use cases and so packages, such as \pkg{etoolbox},
%    \pkg{atveryend} and others patched \cs{enddocument} to add
%    additional points where code could be hooked into.
%
%    Patching using packages is always problematical as leads to
%    conflicts (code availability, ordering of patches, incompatible
%    patches, etc.).  For this reason a number of additional hooks
%    have been added to the \cs{enddocument} code to allow packages
%    to add code in various places in a controlled way without the
%    need for overwriting or patching the core code.
%
%    Supported are now the following hooks:
%    \begin{description}
%
%    \item[\hook{env/document/end}] The generic hook inside \cs{end}.
%
%    \item[\hook{enddocument}]

%      The hook associated with \cs{AtEndDocument}. It is immediately
%      called after the previous hook so there could be just
%      one.\footnote{We could make \cs{AtEndDocument} just fill the
%      \hook{env/document/end} but maybe that is a bit confusing.}
%
%      When this hook is executed there may be still unprocessed
%      material (e.g., floats on the deferlist) and the hook may add
%      further material to be typeset. After it, \cs{clearpage} is
%      called to ensure that all such material gets typeset. If there
%      is nothing waiting the \cs{clearpage} has no effect.
%
%
%    \item[\hook{enddocument/afterlastpage}]
%
%      As the name indicates this hook should not receive code that
%      generates material for further pages. It is the right place to
%      do some final housekeeping and possibly write out some
%      information to the \texttt{.aux} file (which is still open at
%      this point to receive data). It is also the correct place to
%      set up any testing code to be run when the \texttt{.aux} file
%      is re-read in the next step.
%
%
%      After this hook has been executed the \texttt{.aux} file is
%      closed for writing and then read back in to do some tests
%      (e.g., looking for missing references or duplicated labels, etc.).
%
%    \item[\hook{enddocument/afteraux}]
%
%      At this point, the \texttt{.aux} file has been reprocessed and so
%      this is a possible place for final checks and display of
%      information to the user. However, for the latter you might
%      prefer the next hook, so that your information is displayed after the
%      (possibly longish) list of files if that got requested via \cs{listfiles}.
%
%    \item[\hook{enddocument/info}]
%
%      This hook is meant to receive code that write final information
%      messages to the terminal. It follows immediately after the
%      previous hook (so both could have been combined, but then
%      packages adding further code would always need to also supply
%      an explicit rule to specify where it should go.
%
%      This hook already contains some code added by the kernel (under
%      the labels \texttt{kernel/filelist} and
%      \texttt{kernel/warnings}), namely the list of files when
%      \cs{listfiles} has been used and the warnings for duplicate
%      labels, missing references, font substitutions etc.
%
%    \item[\hook{enddocument/end}]
%
%      Finally, this hook is executed just in front of the final call
%      to \cs{@{}@end}.
%
%    \end{description}
%
%
%    There is also the hook \hook{shipout/lastpage}. This hook is
%    executed as part of the last \cs{shipout} in the document to
%    allow package to add final \cs{special}'s to that page. Where
%    this hook is executed in relation to those from the above list
%    can vary from document to document. Furthermore to determine correctly
%    which of the \cs{shipout}s is the last one, \LaTeX{} needs to be run
%    several times, so initially it might get executed on the wrong
%    page. See section~\ref{sec:shipout} for where to find the details.
%
%
% \subsubsection{Hooks provided \cs{shipout} operations}
% \label{sec:shipout}
%
%    There are several hooks and mechanisms added to \LaTeX{}'s
%    process of generating pages. These are documented in
%    \texttt{ltshipout-doc.pdf} or with code in
%    \texttt{ltshipout-code.pdf}.
%
%
% \subsubsection{Hooks provided file loading operations}
%
%    There are several hooks added to \LaTeX{}'s
%    process of loading file via its high-level interfaces such as
%    \cs{input}, \cs{include}, \cs{usepackage}, etc. These are documented in
%    \texttt{ltfilehook-doc.pdf} or with code in
%    \texttt{ltfilehook-code.pdf}.
%
%

% \StopEventually{\setlength\IndexMin{200pt}  \PrintIndex  }
%
%
% \section{The Implementation}
%    \begin{macrocode}
%<@@=hook>
%    \end{macrocode}
%
%    \begin{macrocode}
%<*2ekernel>
%    \end{macrocode}
%    
%    \begin{macrocode}
\ExplSyntaxOn
%    \end{macrocode}
%
%
%  \subsection{Debugging}
%
%  \begin{macro}{\g_@@_debug_bool}
%    Holds the current debugging state.
%    \begin{macrocode}
\bool_new:N \g_@@_debug_bool
%    \end{macrocode}
%  \end{macro}
%
%  \begin{macro}{\hook_debug_on:,\hook_debug_off:}
%  \begin{macro}{\@@_debug:n}
%  \begin{macro}{\@@_debug_gset:}
%    Turns debugging on and off by redefining \cs{@@_debug:n}.
%    \begin{macrocode}
\cs_new_eq:NN \@@_debug:n \use_none:n
\cs_new_protected:Npn \hook_debug_on:
  {
    \bool_gset_true:N \g_@@_debug_bool
    \@@_debug_gset:
  }
\cs_new_protected:Npn \hook_debug_off:
  {
    \bool_gset_false:N \g_@@_debug_bool
    \@@_debug_gset:
  }
\cs_new_protected:Npn \@@_debug_gset:
  {
    \cs_gset_protected:Npx \@@_debug:n ##1
      { \bool_if:NT \g_@@_debug_bool {##1} }
  }
%    \end{macrocode}
%  \end{macro}
%  \end{macro}
%  \end{macro}

%
%
%  \subsection{Borrowing from internals of other kernel modules}
%
%
% \begin{macro}[EXP]{\@@_str_compare:nn}
%   Private copy of \cs{__str_if_eq:nn}
%    \begin{macrocode}
\cs_new_eq:NN \@@_str_compare:nn \__str_if_eq:nn
%    \end{macrocode}
% \end{macro}
%
%  \subsection{Declarations}
%
%  \begin{macro}{\l_@@_return_tl,\l_@@_tmpa_tl,\l_@@_tmpb_tl}
%    Scratch variables used throughout the package.
%    \begin{macrocode}
\tl_new:N \l_@@_return_tl
\tl_new:N \l_@@_tmpa_tl
\tl_new:N \l_@@_tmpb_tl
%    \end{macrocode}
%  \end{macro}
%
%  \begin{macro}{\g_@@_all_seq}
%    In a few places we need a list of all hook names ever defined so
%    we keep track if them in this sequence.
%    \begin{macrocode}
\seq_new:N \g_@@_all_seq
%    \end{macrocode}
%  \end{macro}
%
% \begin{macro}{\g_@@_removal_list_prop}
%   A property list to hold delayed removals.
%    \begin{macrocode}
\tl_new:N \g_@@_removal_list_tl
%    \end{macrocode}
% \end{macro}
%
% \begin{macro}{\l_@@_cur_hook_tl}
%   Stores the name of the hook currently being sorted.
%    \begin{macrocode}
\tl_new:N \l_@@_cur_hook_tl
%    \end{macrocode}
% \end{macro}
%
% \begin{macro}{\g_@@_code_temp_prop}
%   A property list to temporarily save the original one so that it
%   isn't permanently changed during sorting.
%    \begin{macrocode}
\prop_new:N \g_@@_code_temp_prop
%    \end{macrocode}
% \end{macro}
%
% \begin{macro}{\g_@@_hook_curr_name_tl,\g_@@_name_stack_seq}
%   Default label used for hook commands, and a stack to keep track of
%   packages within packages.
%    \begin{macrocode}
\tl_new:N \g_@@_hook_curr_name_tl
\seq_new:N \g_@@_name_stack_seq
%    \end{macrocode}
% \end{macro}
%
% \begin{macro}{\@@_tmp:w}
%   Temporary macro for generic usage.
%    \begin{macrocode}
\cs_new_eq:NN \@@_tmp:w ?
%    \end{macrocode}
% \end{macro}
%
% \begin{macro}{\tl_gremove_once:Nx}
%   Some variants of \pkg{expl3} functions. \fmi{should be moved to expl3}
%    \begin{macrocode}
\cs_generate_variant:Nn \tl_gremove_once:Nn { Nx }
%    \end{macrocode}
% \end{macro}
%
% \begin{macro}{\s_@@_mark}
%   Scan mark used for delimited arguments.
%    \begin{macrocode}
\scan_new:N \s_@@_mark
%    \end{macrocode}
% \end{macro}
%
%
% \subsection{Providing new hooks}
%
%  \begin{macro}{\g_@@_..._code_prop,\g_@@_..._rules_prop,
%                \g_@@_..._code_tl,\g_@@_..._next_code_tl}
%
%    Hooks have a \meta{name} and for each hook we have to provide a number of
%    data structures. These are
%    \begin{description}
%    \item[\cs{g_@@_\meta{name}_code_prop}] A property list holding the code
%    for the hook in separate chunks. The keys are by default the
%    package names that add code to the hook, but it is possible
%    for packages to define other keys. 
%
%    \item[\cs{g_@@_\meta{name}_rules_prop}] A property listing holding
%    relation info how the code chunks should be ordered within a
%    hook. This is used for debugging only. The actual rule for a
%    \meta{hook} is stored in a separate token lists named
%    \cs[no-index]{g_@@_\meta{hook}_rule_\meta{label1}\string|\meta{label2}_tl}
%    for a pair of labels.
%
%    \item[\cs{g_@@_\meta{name}_code_tl}] The code that is actually executed
%    when the hook is called in the document is stored in this token
%    list. It is constructed from the code chunks applying the
%    information.
%
%    \item[\cs{g_@@_\meta{name}_next_code_tl}] Finally there is extra code
%    (normally empty) that is used on the next invocation of the hook
%    (and then deleted). This can be used to define some special
%    behavior for a single occasion from within the document.
%
%    \end{description}
%  \end{macro}


%
%
%  \begin{macro}{\hook_new:n}
%    The \cs{hook_new:n} declaration declare a new hook and expects
%    the hook \meta{name} as its argument, e.g.,
%    \hook{begindocument}.
%    \begin{macrocode}
\cs_new_protected:Npn \hook_new:n #1
  {
    \exp_args:Nx \@@_new:n
      { \@@_parse_label_default:nn {#1} { top-level } }
  }
\cs_new_protected:Npn \@@_new:n #1 {
%    \end{macrocode}
%    We check for one of the internal data structures and if it
%    already exists we complain.
%    \begin{macrocode}
  \hook_if_exist:nTF {#1}
     { \ErrorHookExists }
%    \end{macrocode}
%    Otherwise we add the hook name to the list of all hooks and
%    allocate the necessary data structures for the new hook.
%    \begin{macrocode}
     { \seq_gput_right:Nn \g_@@_all_seq {#1}
%    \end{macrocode}
%    This is only used by the actual code of the current hook, so
%    declare it normally:
%    \begin{macrocode}
       \tl_new:c { g_@@_#1_code_tl }
%    \end{macrocode}
%    Now ensure that the base data structure for the hook exists:
%    \begin{macrocode}
       \@@_declare:n {#1}
%    \end{macrocode}
%    The \cs{g_@@_\meta{hook}_labels_clist} holds the sorted list of
%    labels (once it got sorted). This is used only for debugging.
%    \begin{macrocode}
       \clist_new:c {g_@@_#1_labels_clist}
%    \end{macrocode}
%    Some hooks should reverse the default order of code chunks. To
%    signal this we have a token list which is empty for normal hooks
%    and contains a \verb=-= for reversed hooks.
%    \begin{macrocode}
       \tl_new:c { g_@@_#1_reversed_tl }
%    \end{macrocode}
%    The above is all in L3 convention, but we also provide an
%    interface to legacy \LaTeXe{} for use in the current kernel. This
%    is done in a separate macro.
%    \begin{macrocode}
       \@@_provide_legacy_interface:n {#1}
     }
}
%    \end{macrocode}
%  \end{macro}

%
%
% \begin{macro}{\@@_declare:n}
%   This function declares the basic data structures for a hook without
%   actually declaring the hook itself.  This is needed to allow adding
%   to undeclared hooks.  Here it is unnecessary to check whether both
%   variables exist, since both are declared at the same time (either
%   both exist, or neither).
%    \begin{macrocode}
\cs_new_protected:Npn \@@_declare:n #1
  {
    \@@_if_exist:nF {#1}
      {
        \prop_new:c { g_@@_#1_code_prop }
        \tl_new:c { g_@@_#1_next_code_tl }
        \prop_new:c { g_@@_#1_rules_prop } % only for debugging
      }
  }
%    \end{macrocode}
%  \end{macro}



%  \begin{macro}{\hook_new_reversed:n}
%
%    Declare a new hook. The default ordering of code chunks is
%    reversed, signaled by setting the token list to a minus sign.
%    \begin{macrocode}
\cs_new_protected:Npn \hook_new_reversed:n #1 {
  \hook_new:n {#1}
  \tl_gset:cn { g_@@_#1_reversed_tl } { - }
}
%    \end{macrocode}
%  \end{macro}

%  \begin{macro}{\hook_new_pair:nn}
%    A shorthand for declaring a normal and a (matching) reversed hook in one go.
%    \begin{macrocode}
\cs_new_protected:Npn \hook_new_pair:nn #1#2 {
  \hook_new:n {#1}  \hook_new_reversed:n {#2}
}
%    \end{macrocode}
%  \end{macro}


% \begin{macro}{\@@_provide_legacy_interface:n}
%    The \LaTeX{} legacy concept for hooks uses with hooks the
%    following naming scheme in the code: \cs{@...hook}.
%
%    We follow this convention and insert the hook code using this
%    naming scheme in \LaTeXe{}. At least as long as this code is in a
%    package, some such hooks are already filled with data when we move
%    them over to the new scheme. We therefore insert already existing
%    code under the label \texttt{legacy} into the hook management
%    machinery and then replace the \cs{@...hook} with its counterpart
%    which is \cs{g_@@_\#1_code_tl}.\footnote{This means one extra
%    unnecessary expansion on each invocation in the document but
%    keeps the \LaTeXe{} and the L3 coding side properly separated.}
%    \begin{macrocode}
\cs_new_protected:Npn \@@_provide_legacy_interface:n #1
  {
%    \end{macrocode}
%    If the \pkg{expl3} code is run with checking on then assigning or
%    using non L3 names such as \cs{@enddocumenthook} with \pkg{expl3}
%    functions will trigger warnings so we run this code with
%    debugging explicitly suspended.
%    \begin{macrocode}
    \debug_suspend:
    \tl_if_exist:cT { @#1hook }
%    \end{macrocode}
%    Of course if the hook exists but is still empty, there is no need
%    to add anything under \texttt{legacy} or the current package name.
%    \begin{macrocode}
      {
        \tl_if_empty:cF { @#1hook }
          {
            \@@_gput_code:nxv {#1}
              { \@@_parse_label_default:Vn \c_novalue_tl { legacy } }
              { @#1hook }
          }
      }
%    \end{macrocode}
%    We need a global definition in case the declaration is done
%    inside a group (which happens below at the end of the file).
%    This is another reason why need to suspend checking, otherwise
%    \cs{tl_gset:co} would complain about \cs{@...hook} not starting
%    with \cs{g_}.
%    \begin{macrocode}
    \tl_gset:co{@#1hook}{\cs:w g_@@_#1_code_tl\cs_end:}
    \debug_resume:
  }
%    \end{macrocode}
% \end{macro}
%
% \subsection{Parsing a label}
%
% \begin{macro}[EXP]{\@@_parse_label_default:nn,\@@_parse_label_default:Vn}
%   This macro checks if a label was given (not \cs{c_novalue_tl}), and
%   if so, tries to parse the label looking for a leading \verb|.| to
%   replace for \cs{@currname}.  Otherwise \cs{@@_currname_or_default:n}
%   is used to pick \cs{@currname} or the fallback value.
%    \begin{macrocode}
\cs_new:Npn \@@_parse_label_default:nn #1 #2
  {
    \tl_if_novalue:nTF {#1}
      { \@@_currname_or_default:n {#2} }
      { \tl_trim_spaces_apply:nN {#1} \@@_parse_dot_label:nn {#2} }
  }
\cs_generate_variant:Nn \@@_parse_label_default:nn { V }
%    \end{macrocode}
% \end{macro}
%
% \begin{macro}[EXP]{\@@_parse_dot_label:nn}
% \begin{macro}[EXP]{
%     \@@_parse_dot_label:nw,
%     \@@_parse_dot_label_cleanup:w,
%     \@@_parse_dot_label_aux:nw
%   }
%   Start by checking if the label is empty, which raises an error, and
%   uses the fallback value.  If not,
%   split the label at a \verb|./|, if any, and check if no tokens are
%   before the \verb|./|, or if the only character is a \verb|.|.
%   If these requirements are fulfilled, the leading
%   \verb|.| is replaced with \cs{@@_currname_or_default:n}.  Otherwise
%   the label is returned unchanged.
%    \begin{macrocode}
\cs_new:Npn \@@_parse_dot_label:nn #1 #2
  {
    \tl_if_empty:nTF {#1}
      {
        \msg_expandable_error:nnn { hooks } { empty-label } {#2}
        #2
      }
      {
        \str_if_eq:nnTF {#1} { . }
          { \@@_currname_or_default:n {#1} }
          { \@@_parse_dot_label:nw {#2} #1 ./ \s_@@_mark }
      }
  }
\cs_new:Npn \@@_parse_dot_label:nw #1 #2 ./ #3 \s_@@_mark
  {
    \tl_if_empty:nTF {#2}
      { \@@_parse_dot_label_aux:nw {#1} #3 \s_@@_mark }
      {
        \tl_if_empty:nTF {#3}
          {#2}
          { \@@_parse_dot_label_cleanup:w #2 ./ #3 \s_@@_mark }
      }
  }
\cs_new:Npn \@@_parse_dot_label_cleanup:w #1 ./ \s_@@_mark {#1}
\cs_new:Npn \@@_parse_dot_label_aux:nw #1 #2 ./ \s_@@_mark
  { \@@_currname_or_default:n {#1} / #2 }
%    \end{macrocode}
% \end{macro}
% \end{macro}
%
% \begin{macro}[EXP]{\@@_currname_or_default:n}
%   Uses \cs{g_@@_hook_curr_name_tl} if it is set, otherwise tries
%   \cs{@currname}.  If neither is set, uses the fallback value
%   \verb|#1| (usually \texttt{top-level}).
%    \begin{macrocode}
\cs_new:Npn \@@_currname_or_default:n #1
  {
    \tl_if_empty:NTF \g_@@_hook_curr_name_tl
      {
        \tl_if_empty:NTF \@currname
          {#1}
          { \@currname }
      }
      { \g_@@_hook_curr_name_tl }
  }
%    \end{macrocode}
% \end{macro}




% \begin{macro}{\hook_gput_code:nnn}
% \begin{macro}{\@@_gput_code:nnn,\@@_gput_code:nxv,\@@_hook_gput_code_do:nnn}
%
%    With \cs{hook_gput_code:nnn}\Arg{hook}\Arg{label}\Arg{code} a
%    chunk of \meta{code} is added to an existing \meta{hook} labeled
%    with \meta{label}.
%    \begin{macrocode}
\cs_new_protected:Npn \hook_gput_code:nnn #1 #2
  {
    \exp_args:Nxx \@@_gput_code:nnn
      { \@@_parse_label_default:nn {#1} { top-level } }
      { \@@_parse_label_default:nn {#2} { top-level } }
  }
\cs_new_protected:Npn \@@_gput_code:nnn #1 #2 #3
  {
%    \end{macrocode}
%    First we check if the hook exists.
%    \begin{macrocode}
    \@@_if_marked_removal:nnTF {#1} {#2}
      { \@@_unmark_removal:nn {#1} {#2} }
      {
%    \end{macrocode}
%    First we check if the hook exists.
%    \begin{macrocode}
        \hook_if_exist:nTF {#1}
%    \end{macrocode}
%    If so we simply add (or append) the new code to the property list
%    holding different chunks for the hook. At \verb=\begin{document}=
%    this is then sorted into a token list for fast execution.
%    \begin{macrocode}
          {
            \@@_hook_gput_code_do:nnn {#1} {#2} {#3}
%    \end{macrocode}
%    However, if there is an update within the document we need to alter
%    this execution code which is done by
%    \cs{@@_update_hook_code:n}. In the preamble this does nothing.
%    \begin{macrocode}
            \@@_update_hook_code:n {#1}
          }
%    \end{macrocode}
%    
%    \begin{macrocode}
          { \@@_try_declaring_generic_hook:nnn {#1} {#2} {#3} }
      }
  }
\cs_generate_variant:Nn \@@_gput_code:nnn { nxv }
%    \end{macrocode}
%
%   This macro will unconditionally add a chunk of code to the given hook.
%    \begin{macrocode}
\cs_new_protected:Npn \@@_hook_gput_code_do:nnn #1 #2 #3
  {
%   However, first some debugging info if debugging is enabled:
%    \begin{macrocode}
    \@@_debug:n{\iow_term:x{****~ Add~ to~
                      \hook_if_exist:nF {#1} { undeclared~ }
                      hook~ #1~ (#2)
                      \on@line\space <-~ \tl_to_str:n{#3}} }
%    \end{macrocode}
%   Then try to get the code chunk labeled \verb=#2= from the hook.
%   If there's code already there, then append \verb=#3= to that,
%   otherwise just put \verb=#3=.
%    \begin{macrocode}
    \prop_get:cnNTF { g_@@_#1_code_prop } {#2} \l_@@_return_tl
      {
        \prop_gput:cno { g_@@_#1_code_prop } {#2}
          { \l_@@_return_tl #3 }
      }
      { \prop_gput:cnn { g_@@_#1_code_prop } {#2} {#3} }
  }
%    \end{macrocode}
% \end{macro}
% \end{macro}
%
% \begin{macro}{\@@_gput_undeclared_hook:nnn}
%   Often it may happen that a package $A$ defines a hook \verb=foo=,
%   but package $B$, that adds code to that hook, is loaded before $A$.
%   In such case we need to add code to the hook before it's declared.
%    \begin{macrocode}
\cs_new_protected:Npn \@@_gput_undeclared_hook:nnn #1 #2 #3
  {
    \@@_declare:n {#1}
    \@@_hook_gput_code_do:nnn {#1} {#2} {#3}
  }
%    \end{macrocode}
% \end{macro}
%
% \begin{macro}{\@@_try_declaring_generic_hook:nnn}
% \begin{macro}{\@@_try_declaring_generic_next_hook:nn}
%   These entry-level macros just pass the arguments along to the
%   common \cs{@@_try_declaring_generic_hook:nNNnn} with the right
%   functions to execute when some action is to be taken.
%
%   The wrapper \cs{@@_try_declaring_generic_hook:nnn} then defers
%   \cs{hook_gput_code:nnn} if the generic hook was declared, or to
%   \cs{@@_gput_undeclared_hook:nnn} otherwse (the hook was tested for
%   existence before, so at this point if it isn't generic, it doesn't
%   exist).
%
%   The wrapper \cs{@@_try_declaring_generic_next_hook:nn} for
%   next-execution hooks does the same: it defers the code to
%   \cs{hook_gput_next_code:nn} if the generic hook was declared, or
%   to \cs{@@_gput_next_do:nn} otherwise.
%    \begin{macrocode}
\cs_new_protected:Npn \@@_try_declaring_generic_hook:nnn #1
  {
    \@@_try_declaring_generic_hook:nNNnn {#1}
      \hook_gput_code:nnn \@@_gput_undeclared_hook:nnn
  }
\cs_new_protected:Npn \@@_try_declaring_generic_next_hook:nn #1
  {
    \@@_try_declaring_generic_hook:nNNnn {#1}
      \hook_gput_next_code:nn \@@_gput_next_do:nn
  }
%    \end{macrocode}
%
% \begin{macro}{
%     \@@_try_declaring_generic_hook:nNNnn,
%     \@@_try_declaring_generic_hook_split:nNNnn
%   }
% \begin{macro}[TF]{\@@_try_declaring_generic_hook:wn}
%   \cs{@@_try_declaring_generic_hook:nNNnn} now splits the hook name
%   at the first \texttt{/} (if any) and first checks if it is a
%   file-specific hook (they require some normalisation) using
%   \cs{@@_if_file_hook:wTF}. If not then check it is one of a
%   predefined set for generic names. We also split off the second
%   component to see if we have to make a reversed hook.  In either case
%   the function returns \meta{true} for a generic hook and \meta{false}
%   in other cases.
%    \begin{macrocode}
\cs_new_protected:Npn \@@_try_declaring_generic_hook:nNNnn #1
  {
    \@@_if_file_hook:wTF #1 / / \s_@@_mark
      {
        \exp_args:Ne \@@_try_declaring_generic_hook_split:nNNnn
          { \exp_args:Ne \@@_file_hook_normalise:n {#1} }
      }
      { \@@_try_declaring_generic_hook_split:nNNnn {#1} }
  }
\cs_new_protected:Npn \@@_try_declaring_generic_hook_split:nNNnn #1 #2 #3
  {
    \@@_try_declaring_generic_hook:wnTF #1 / / / \scan_stop: {#1}
      { #2 }
      { #3 } {#1}
  }
\prg_new_protected_conditional:Npnn \@@_try_declaring_generic_hook:wn
    #1 / #2 / #3 / #4 \scan_stop: #5 { TF }
  {
    \tl_if_empty:nTF {#2}
      { \prg_return_false: }
      {
        \prop_if_in:NnTF \c_@@_generics_prop {#1}
          {
            \hook_if_exist:nF {#5} { \hook_new:n {#5} }
%    \end{macrocode}
%    After having declared the hook we check the second component (for
%    file hooks) or the third component for environment hooks) and
%    if it is on the list of components for which we should have declared
%    a reversed hook we alter the hook datastructure accordingly.
%    \begin{macrocode}
            \prop_if_in:NnTF \c_@@_generics_reversed_ii_prop {#2}
              { \tl_gset:cn { g_@@_#5_reversed_tl } { - } }
              {
                \prop_if_in:NnT \c_@@_generics_reversed_iii_prop {#3}
                  { \tl_gset:cn { g_@@_#5_reversed_tl } { - } }
              }
%    \end{macrocode}
%    Now that we know that the hook is declared we can add the code to it.
%    \begin{macrocode}
            \prg_return_true:
          }
          { \prg_return_false: }
      }
  }
%    \end{macrocode}
% \end{macro}
% \end{macro}
% \end{macro}
% \end{macro}
%
% \begin{macro}[pTF]{\@@_if_file_hook:w}
%   \cs{@@_if_file_hook:wTF} checks if the argument is a valid
%   file-specific hook (not, for example, |file/before|, but
%   |file/before/foo.tex|).  If it is a file-specific hook, then it
%   executes the \meta{true} branch, otherwise \meta{false}.
%
%   A file-specific hook is \texttt{file/\meta{position}/\meta{name}}.
%   If any of these parts don't exist, it is a general file hook or not
%   a file hook at all, so the conditional evaluates to \meta{false}.
%   Otherwise, it checks that the first part is |file| and that the
%   \meta{position} is in the \cs{c_@@_generics_file_prop}.
%
%   A property list is used here to avoid having to worry with catcodes,
%   because \pkg{expl3}'s file name parsin turns all characters into
%   catcode-12 tokens, which might differ from hand-input letters.
%    \begin{macrocode}
\prg_new_conditional:Npnn \@@_if_file_hook:w
    #1 / #2 / #3 \s_@@_mark { TF }
  {
    \str_if_eq:nnTF {#1} { file }
      {
        \bool_lazy_or:nnTF
            { \tl_if_empty_p:n {#3} }
            { \str_if_eq_p:nn {#3} { / } }
          { \prg_return_false: }
          {
            \prop_if_in:NnTF \c_@@_generics_file_prop {#2}
              { \prg_return_true: }
              { \prg_return_false: }
          }
      }
      { \prg_return_false: }
  }
%    \end{macrocode}
% \end{macro}
%
% \begin{macro}[EXP]{\@@_file_hook_normalise:n}
% \begin{macro}[EXP]{\@@_strip_double_slash:n,\@@_strip_double_slash:w}
%   When a file-specific hook is found, before being declared it is
%   lightly normalised by \cs{@@_file_hook_normalise:n}.  The current
%   implementation just replaces two consecutive slashes (|//|) by a
%   single one, to cope with simple cases where the user did something
%   like \verb|\def\input@path{{./mypath/}}|, in which case a hook would
%   have to be \verb|\AddToHook{file/after/./mypath//file.tex}|.
%    \begin{macrocode}
\cs_new:Npn \@@_file_hook_normalise:n #1
  { \@@_strip_double_slash:n {#1} }
\cs_new:Npn \@@_strip_double_slash:n #1
  { \@@_strip_double_slash:w #1 // \s_@@_mark }
\cs_new:Npn \@@_strip_double_slash:w #1 // #2 \s_@@_mark
  {
    \tl_if_empty:nTF {#2}
      {#1}
      { \@@_strip_double_slash:w #1 / #2 \s_@@_mark }
  }
%    \end{macrocode}
% \end{macro}
% \end{macro}





%  \begin{macro}{\c_@@_generics_prop}
%    Clist holding the generic names. We don't provide any user
%    interface to this as this is meant to be static.
%    \begin{description}
%    \item[\texttt{env}]
%      The generic hooks used in \cs{begin} and \cs{end}.
%    \item[\texttt{file}]
%      The generic hooks used when loading a file
%    \end{description}
%    \begin{macrocode}
\prop_const_from_keyval:Nn \c_@@_generics_prop
  {env=,file=,package=,class=,include=}
%    \end{macrocode}
%  \end{macro}
%
%  \begin{macro}{\c_@@_generics_reversed_ii_prop,
%                \c_@@_generics_reversed_iii_prop,
%                \c_@@_generics_file_prop}
%    Some of the generic hooks are supposed to use reverse ordering, these are
%    the following (only the second or third sub-component is checked):
%    \begin{macrocode}
\prop_const_from_keyval:Nn \c_@@_generics_reversed_ii_prop {after=,end=}
\prop_const_from_keyval:Nn \c_@@_generics_reversed_iii_prop {after=}
\prop_const_from_keyval:Nn \c_@@_generics_file_prop {before=,after=}
%    \end{macrocode}
%  \end{macro}


%  \begin{macro}{\@@_update_hook_code:n}
%    Before \verb=\begin{document}=  this does nothing, in the body it
%    reinitializes the hook code using the altered data.
%    \begin{macrocode}
\cs_new_eq:NN \@@_update_hook_code:n \use_none:n
%    \end{macrocode}
%  \end{macro}


% \begin{macro}{\hook_gremove_code:nn}
% \begin{macro}{\@@_gremove_code:nn}
%    
%    With \cs{hook_gremove_code:nn}\Arg{hook}\Arg{label} any code
%    for \meta{hook} stored under \meta{label} is removed.
%    \begin{macrocode}
\cs_new_protected:Npn \hook_gremove_code:nn #1 #2
  {
    \exp_args:Nxx \@@_gremove_code:nn
      { \@@_parse_label_default:nn {#1} { top-level } }
      { \@@_parse_label_default:nn {#2} { top-level } }
  }
\cs_new_protected:Npn \@@_gremove_code:nn #1 #2
  {
%    \end{macrocode}
%    First check that the hook code pool exists.  \cs{hook_if_exist:nTF}
%    isn't used here because it should be possible to remove code from a
%    hook before it's defined (see section~\ref{sec:querying}).
%    \begin{macrocode}
    \@@_if_exist:nTF {#1}
%    \end{macrocode}
%    Then remove the chunk and run \cs{@@_update_hook_code:n} so
%    that the execution token list reflects the change if we are after
%    \verb=\begin{document}=.
%    \begin{macrocode}
      {
        \str_if_eq:nnTF {#2} {*}
          {
            \prop_gclear:c { g_@@_#1_code_prop }
            \clist_gclear:c { g_@@_#1_labels_clist } % for debugging only
          }
          {
%    \end{macrocode}
%    Check if the label being removed exists in the code pool.  If it does,
%    just call \cs{@@_gremove_code_do:nn} to do the removal, otherwise mark it
%    to be removed.
%    \begin{macrocode}
            \prop_get:cnNTF { g_@@_#1_code_prop } {#2} \l_@@_return_tl
              { \@@_gremove_code_do:nn }
              { \@@_mark_removal:nn }
                  {#1} {#2}
          }
%    \end{macrocode}
%    Finally update the code, if the hook exists.
%    \begin{macrocode}
        \hook_if_exist:nT {#1}
          { \@@_update_hook_code:n {#1} }
      }
      { \@@_mark_removal:nn {#1} {#2} }
  }
%    \end{macrocode}
%
% \begin{macro}{\@@_gremove_code_do:nn}
%    \begin{macrocode}
\cs_new_protected:Npn \@@_gremove_code_do:nn #1 #2
  {
    \prop_gremove:cn { g_@@_#1_code_prop } {#2}
%    \end{macrocode}
%    Removing the dropped label from \verb=\g_@@_#1_labels_clist= is
%    rather tricky, because that clists holds the labels as strings
%    (i.e., not ordinary text which is what we have in \verb=#2=).
%    \begin{macrocode}
    \exp_args:Nco \clist_gremove_all:Nn
      { g_@@_#1_labels_clist } { \tl_to_str:n {#2} } % for debugging only
  }
%    \end{macrocode}
% \end{macro}
% \end{macro}
% \end{macro}


%
% \begin{macro}{\@@_mark_removal:nn}
%   Marks \meta{label} (\verb=#2=) to be removed from \meta{hook}
%   (\verb=#1=).
%    \begin{macrocode}
\cs_new_protected:Npn \@@_mark_removal:nn #1 #2
  {
    \tl_gput_right:Nx \g_@@_removal_list_tl
      { \@@_removal_tl:nn {#1} {#2} }
  }
%    \end{macrocode}
% \end{macro}
%
% \begin{macro}{\@@_unmark_removal:nn}
%   Unmarks \meta{label} (\verb=#2=) to be removed from \meta{hook}
%   (\verb=#1=).  \cs{tl_gremove_once:Nx} is used rather than
%   \cs{tl_gremove_all:Nx} so that two additions are needed to cancel
%   two marked removals, rather than only one.
%    \begin{macrocode}
\cs_new_protected:Npn \@@_unmark_removal:nn #1 #2
  {
    \tl_gremove_once:Nx \g_@@_removal_list_tl
      { \@@_removal_tl:nn {#1} {#2} }
  }
%    \end{macrocode}
% \end{macro}
%
% \begin{macro}[TF]{\@@_if_marked_removal:nn}
%   Checks if the \cs{g_@@_removal_list_tl} contains the current
%   \meta{label} (\verb=#2=) and \meta{hook} (\verb=#1=).
%    \begin{macrocode}
\prg_new_protected_conditional:Npnn \@@_if_marked_removal:nn #1 #2 { TF }
  {
    \exp_args:NNx \tl_if_in:NnTF \g_@@_removal_list_tl
      { \@@_removal_tl:nn {#1} {#2} }
      { \prg_return_true: } { \prg_return_false: }
  }
%    \end{macrocode}
% \end{macro}
%
% \begin{macro}[rEXP]{\@@_removal_tl:nn}
%   Builds a token list with \verb=#1= and \verb=#2= which can only be
%   matched by \verb=#1= and \verb=#2=.
%    \begin{macrocode}
\cs_new:Npn \@@_removal_tl:nn #1 #2
  { & \tl_to_str:n {#2} $ \tl_to_str:n {#1} $ }
%    \end{macrocode}
% \end{macro}




%
%
%
% \begin{macro}{
%     \g_@@_??_rules_prop,
%     \g_@@_??_code_prop,
%     \g_@@_??_code_tl,
%     \g_@@_??_reversed_tl,
%   }
%
%    Default rules applying to all hooks are stored in this property
%    list. Initially it simply used an empty ``label'' name (not two
%    question marks). This was a bit unfortunate, because then
%    \texttt{l3doc} complains about \verb=__= in the middle of a
%    command name when trying to typeset the documentation. However
%    using a ``normal'' name such as \texttt{default} has the
%    disadvantage of that being not really distinguishable from a real
%    hook name. I now have settled for \texttt{??} which needs some
%    gymnastics to get it into the csname, but since this is used a
%    lot things should be fast, so this is not done with \texttt{c}
%    expansion in the code later on.
%
%    \cs{g_@@_??_code_tl} isn't used, but it has to be defined to trick
%    the code into thinking that \verb=??= is actually a hook.
%    \begin{macrocode}
\prop_new:c {g_@@_??_rules_prop}
\prop_new:c {g_@@_??_code_prop}
\prop_new:c {g_@@_??_code_tl}
%    \end{macrocode}
%
%    Default rules are always given in normal ordering (never in
%    reversed ordering). If such a rule is applied to a reversed
%    hook it behaves as if the rule is reversed (e.g.,
%    \texttt{after} becomes \texttt{before})
%    because those rules are applied first and then the order is reversed.
%    \begin{macrocode}
\tl_new:c {g_@@_??_reversed_tl}
%    \end{macrocode}
%  \end{macro}



%  \begin{macro}{\@@_debug_gset_rule:nnnn}
%
%    \fmi{this needs cleanup and docu correction!}
%
%    With
%    \cs{@@_debug_gset_rule:nnnn}\Arg{hook}\Arg{label1}\Arg{relation}\Arg{label2}
%    a relation is defined between the two code labels for the given
%    \meta{hook}.  The special hook \texttt{??} stands for \emph{any}
%    hook describing an default rule.
%    \begin{macrocode}
\cs_new_protected:Npn \@@_debug_gset_rule:nnnn #1#2#3#4
  {
%    \end{macrocode}
%    If so we drop any existing rules with the two labels (in case
%    there are any).
%    \begin{macrocode}
    \prop_gremove:cn{g_@@_#1_rules_prop}{#2|#4}
    \prop_gremove:cn{g_@@_#1_rules_prop}{#4|#2}
%    \end{macrocode}
%    Then  we add the new one (normalizing the input a bit, e.g., we
%    always use \texttt{before} and not \texttt{after} and
%    instead reorder the labels):
%    \begin{macrocode}
    \str_case_e:nnF {#3}
       {
         {before} { \prop_gput:cnn {g_@@_#1_rules_prop}{#2|#4}{<} }
         {after}  { \prop_gput:cnn {g_@@_#1_rules_prop}{#4|#2}{<} }
%    \end{macrocode}
%    More special rule types \ldots
%    \begin{macrocode}
         {incompatible-error}   { \prop_gput:cnn {g_@@_#1_rules_prop}{#2|#4}{xE} }
         {incompatible-warning} { \prop_gput:cnn {g_@@_#1_rules_prop}{#2|#4}{xW} }
         {removes}      { \prop_gput:cnn {g_@@_#1_rules_prop}{#2|#4}{->} }
%    \end{macrocode}
%    Undo a setting:
%    \begin{macrocode}
         {unrelated}{ \prop_gremove:cn {g_@@_#1_rules_prop}{#2|#4} 
                      \prop_gremove:cn {g_@@_#1_rules_prop}{#4|#2} }
       }
       { \ERRORunknownrule }
  }
%    \end{macrocode}
%  \end{macro}
%
%
%  \subsection{Setting rules for hooks code}
%
%  \begin{macro}{\hook_gset_rule:nnnn}
%  \begin{macro}{\@@_gset_rule:nnnn}
%
%    \fmi{needs docu correction given new implementation}
%
%    With
%    \cs{hook_gset_rule:nnnn}\Arg{hook}\Arg{label1}\Arg{relation}\Arg{label2}
%    a relation is defined between the two code labels for the given
%    \meta{hook}.  The special hook \texttt{??} stands for \emph{any}
%    hook describing a default rule.
%    \begin{macrocode}
\cs_new_protected:Npn \hook_gset_rule:nnnn #1#2#3#4
  {
    \use:x
      {
        \@@_gset_rule:nnnn
          { \@@_parse_label_default:nn {#1} { top-level } }
          { \@@_parse_label_default:nn {#2} { top-level } }
          {#3}
          { \@@_parse_label_default:nn {#4} { top-level } }
      }
  }
%    \end{macrocode}
%    
%    \begin{macrocode}
\cs_new_protected:Npn \@@_gset_rule:nnnn #1#2#3#4
  {
%    \end{macrocode}
%    First we ensure the basic data structure of the hook exists:
%    \begin{macrocode}
    \@@_declare:n {#1}
%    \end{macrocode}
%    Then we clear any previous relationship between both labels.
%    \begin{macrocode}
    \@@_rule_gclear:nnn {#1} {#2} {#4}
%    \end{macrocode}
%    Then we call the function to handle the given rule. Throw an error if the
%    rule is invalid.
%    \begin{macrocode}
    \debug_suspend:
    \cs_if_exist_use:cTF { @@_rule_#3_gset:nnn }
      {
          {#1} {#2} {#4}
        \@@_update_hook_code:n {#1}
      }
      { \ERRORunknownrule }
    \debug_resume:
    \@@_debug_gset_rule:nnnn {#1} {#2} {#3} {#4} % for debugging
  }
%    \end{macrocode}
% \end{macro}
% \end{macro}
%
% \begin{macro}{\@@_rule_before_gset:nnn, \@@_rule_after_gset:nnn,
%               \@@_rule_<_gset:nnn, \@@_rule_>_gset:nnn}
%    Then we add the new rule.  We need to normalize the rules here to
%    allow for faster processing later.  Given a pair of labels
%    $l_A$ and $l_B$, the rule $l_A>l_B$ is the same as $l_B<l_A$\fmi{}
%    said differently.  But normalizing the
%    forms of the rule to a single representation, say, $l_B<l_A$, then
%    the time spent looking for the rules later is considerably reduced.
%
%    Here we do that normalization by using \cs[no-index]{(pdf)strcmp} to
%    lexically sort labels $l_A$ and $l_B$ to a fixed order.  This order
%    is then enforced every time these two labels are used together.
%
%    Here we use \cs{@@_label_pair:nn}~\Arg{hook}~\Arg{l_A}~\Arg{l_B}
%    to build a string \texttt{$l_B$\string|$l_A$} with a fixed order, and
%    use \cs{@@_label_ordered:nnTF} to apply the correct rule to the pair
%    of labels, depending if it was sorted or not.
%    \begin{macrocode}
\cs_new_protected:Npn \@@_rule_before_gset:nnn #1#2#3
  {
    \tl_gset:cx { g_@@_#1_rule_ \@@_label_pair:nn {#2} {#3} _tl }
      { \@@_label_ordered:nnTF {#2} {#3} { < } { > } }
  }
\cs_new_eq:cN { @@_rule_<_gset:nnn } \@@_rule_before_gset:nnn
%    \end{macrocode}
%
%    \begin{macrocode}
\cs_new_protected:Npn \@@_rule_after_gset:nnn #1#2#3
  {
    \tl_gset:cx { g_@@_#1_rule_ \@@_label_pair:nn {#3} {#2} _tl }
      { \@@_label_ordered:nnTF {#3} {#2} { < } { > } }
  }
\cs_new_eq:cN { @@_rule_>_gset:nnn } \@@_rule_after_gset:nnn
%    \end{macrocode}
%  \end{macro}
%
% \begin{macro}{\@@_rule_removes_gset:nnn}
%   This rule removes (clears, actually) the code from label |#3| if
%   label |#2| is in the hook |#1|.
%    \begin{macrocode}
\cs_new_protected:Npn \@@_rule_removes_gset:nnn #1#2#3
  {
    \tl_gset:cx { g_@@_#1_rule_ \@@_label_pair:nn {#2} {#3} _tl }
      { \@@_label_ordered:nnTF {#2} {#3} { -> } { <- } }
  }
%    \end{macrocode}
%  \end{macro}
%
% \begin{macro}{
%     \@@_rule_incompatible-error_gset:nnn,
%     \@@_rule_incompatible-warning_gset:nnn,
%   }
%   These relations make an error/warning if labels |#2| and |#3| appear
%   together in hook |#1|.
%    \begin{macrocode}
\cs_new_protected:cpn { @@_rule_incompatible-error_gset:nnn } #1#2#3
  { \tl_gset:cn { g_@@_#1_rule_ \@@_label_pair:nn {#2} {#3} _tl } { xE } }
\cs_new_protected:cpn { @@_rule_incompatible-warning_gset:nnn } #1#2#3
  { \tl_gset:cn { g_@@_#1_rule_ \@@_label_pair:nn {#2} {#3} _tl } { xW } }
%    \end{macrocode}
%  \end{macro}
%
% \begin{macro}{\@@_rule_unrelated_gset:nnn, \@@_rule_gclear:nnn}
%    Undo a setting. \cs{@@_rule_unrelated_gset:nnn} doesn't need to do anything,
%    since we use \cs{@@_rule_gclear:nnn} before setting any rule.
%    \begin{macrocode}
\cs_new_protected:Npn \@@_rule_unrelated_gset:nnn #1#2#3 { }
\cs_new_protected:Npn \@@_rule_gclear:nnn #1#2#3
  { \cs_undefine:c { g_@@_#1_rule_ \@@_label_pair:nn {#2} {#3} _tl } }
%    \end{macrocode}
%  \end{macro}
%
% \begin{macro}[EXP]{\@@_label_pair:nn}
%   Ensure that the lexically greater label comes first.
%    \begin{macrocode}
\cs_new:Npn \@@_label_pair:nn #1#2
  {
    \if_case:w \@@_str_compare:nn {#1} {#2} \exp_stop_f:
           #1 | #1 %  0
    \or:   #1 | #2 % +1
    \else: #2 | #1 % -1
    \fi:
  }
%    \end{macrocode}
%  \end{macro}
%
% \begin{macro}[pTF]{\@@_label_ordered:nn}
%   Check that labels |#1| and |#2| are in the correct order (as
%   returned by \cs{@@_label_pair:nn}) and if so return true, else
%   return false.
%    \begin{macrocode}
\prg_new_conditional:Npnn \@@_label_ordered:nn #1#2 { TF }
  {
    \if_int_compare:w \@@_str_compare:nn {#1} {#2} > 0 \exp_stop_f:
      \prg_return_true:
    \else
      \prg_return_false:
    \fi:
  }
%    \end{macrocode}
%  \end{macro}
%
% \begin{macro}[EXP]{\@@_if_label_case:nnnnn}
%   To avoid doing the string comparison twice in \cs{@@_initialize_single:NNNNn}
%   (once with \cs{str_if_eq:nn} and again with \cs{@@_label_ordered:nn}),
%   we use a three-way branching macro that will compare |#1| and |#2|
%   and expand to \cs{use_i:nnn} if they are equal, \cs{use_ii:nn} if
%   |#1| is lexically greater, and \cs{use_iii:nn} otherwise.
%    \begin{macrocode}
\cs_new:Npn \@@_if_label_case:nnnnn #1#2
   {
     \cs:w use_
       \if_case:w \@@_str_compare:nn {#1} {#2}
          i \or: ii \else: iii \fi: :nnn
     \cs_end:
   }
%    \end{macrocode}
%  \end{macro}


%  \begin{macro}{\@@_initialize_all:}
%    Initialize all known hooks (at \verb=\begin{document}=), i.e.,
%    update the fast execution token lists to hold the necessary code
%    in the right  order.
%    \begin{macrocode}
\cs_new_protected:Npn \@@_initialize_all: {
%    \end{macrocode}
%    First we change \cs{@@_update_hook_code:n} which so far was a
%    no-op to now initialize one hook. This way any later updates to
%    the hook will run that code and also update the execution token
%    list.
%    \begin{macrocode}
  \cs_gset_eq:NN \@@_update_hook_code:n \@@_initialize_hook_code:n
%    \end{macrocode}
%    Now we loop over all hooks that have been defined and update each
%    of them.
%    \begin{macrocode}
  \@@_debug:n { \prop_gclear:N \g_@@_used_prop }
  \seq_map_inline:Nn \g_@@_all_seq
      {
        \@@_update_hook_code:n {##1}
      }
%    \end{macrocode}
%    If we are debugging we show results hook by hook for all hooks
%    that have data. 
%    \begin{macrocode}
  \@@_debug:n
     { \iow_term:x{^^JAll~ initialized~ (non-empty)~ hooks:}
       \prop_map_inline:Nn \g_@@_used_prop
           { \iow_term:x{^^J~ ##1~ ->~
               \exp_not:v {g_@@_##1_code_tl}~ }
           }
     }
%    
%    \end{macrocode}
%    After all hooks are initialized we change the ``use'' to just
%    call the hook code and not initialize it (as it was done in the
%    preamble.
%    \begin{macrocode}
  \cs_gset_eq:NN \hook_use:n \@@_use_initialized:n
  \cs_gset_eq:NN \@@_preamble_hook:n \use_none:n
}
%    \end{macrocode}
%  \end{macro}



%  \begin{macro}{\@@_initialize_hook_code:n}
%    Initializing or reinitializing the fast execution hook code. In
%    the preamble this is selectively done in case a hook gets used
%    and at \verb=\begin{document}= this is done for all hooks and
%    afterwards only if the hook code changes.
%    \begin{macrocode}
\cs_new_protected:Npn \@@_initialize_hook_code:n #1 {
  \@@_debug:n{ \iow_term:x{^^JUpdate~ code~ for~ hook~
                                  '#1' \on@line :^^J} }
%    \end{macrocode}
%    This does the sorting and the updates. If there aren't any code
%    chunks for the current hook, there is no point in even starting
%    the sorting routine so we make a quick test for that and in that
%    case just update \cs{g_@@_\meta{hook}_code_tl} to hold the next
%    code. If there are code chunks we call
%    \cs{@@_initialize_single:NNNNn} and pass to it ready made csnames
%    as they are needed several times inside. This way we save a bit
%    on processing time if we do that up front.
%    \begin{macrocode}
  \hook_if_exist:nT {#1}
    {
      \prop_if_empty:cTF {g_@@_#1_code_prop}
        { \tl_gset:co {g_@@_#1_code_tl}
                      {\cs:w g_@@_#1_next_code_tl \cs_end: } }
        {
%    \end{macrocode}
%    By default the algorithm sorts the code chunks and then saves the
%    result in a token list for fast execution by adding the code one
%    after another using \cs{tl_gput_right:NV}. When we sort code for
%    a reversed hook, all we have to do is to add the code chunks in
%    the opposite order into the token list. So all we have to do
%    in preparation is to change two definitions used later on.
%    \begin{macrocode}
          \@@_if_reversed:nTF {#1}
            { \cs_set_eq:NN \@@_tl_gput:NV    \tl_gput_left:NV
              \cs_set_eq:NN \@@_clist_gput:NV \clist_gput_left:NV  }
            { \cs_set_eq:NN \@@_tl_gput:NV    \tl_gput_right:NV
              \cs_set_eq:NN \@@_clist_gput:NV \clist_gput_right:NV }
%    \end{macrocode}
%
%    When sorting, some relations (namely \verb|->| \verb|<-|) need to
%    act destructively on the code property lists to remove code that
%    shouldn't appear in the sorted hook token list.
%    \begin{macrocode}
          \prop_gset_eq:Nc \g_@@_code_temp_prop { g_@@_#1_code_prop }
          \@@_initialize_single:ccccn
            { g_@@_#1_code_prop } { g_@@_#1_code_tl }
            { g_@@_#1_next_code_tl } { g_@@_#1_labels_clist }
            {#1}
          \prop_gset_eq:cN { g_@@_#1_code_prop } \g_@@_code_temp_prop
%    \end{macrocode}
%    For debug display we want to keep track of those hooks that
%    actually got code added to them, so we record that in plist. We
%    use a plist to ensure that we record each hook name only once,
%    i.e., we are only interested in storing the keys and the value is arbitrary
%    \begin{macrocode}
          \@@_debug:n{ \exp_args:NNx \prop_gput:Nnn \g_@@_used_prop {#1}{} }
        }
    }
}
%    \end{macrocode}
%  \end{macro}


%  \begin{macro}{\g_@@_used_prop}
%    All hooks that receive code (for use in debugging display).
%    \begin{macrocode}
\prop_new:N\g_@@_used_prop
%    \end{macrocode}
%  \end{macro}


%
% \begin{macro}[EXP]{\@@_tl_csname:n,\@@_seq_csname:n}
%   It is faster to pass a single token and expand it when necessary
%   than to pass a bunch of character tokens around.
%   \fmi{note to myself: verify}
%    \begin{macrocode}
\cs_new:Npn \@@_tl_csname:n #1 { l_@@_label_#1_tl }
\cs_new:Npn \@@_seq_csname:n #1 { l_@@_label_#1_seq }
%    \end{macrocode}
% \end{macro}
%

%
%
%  \begin{macro}{\l_@@_labels_seq,\l_@@_labels_int,\l_@@_front_tl,
%      \l_@@_rear_tl,\l_@@_label_0_tl}
%
%    For the sorting I am basically implementing Knuth's algorithm for
%    topological sorting as given in TAOCP volume 1 pages 263--266.
%    For this algorithm we need a number of local variables:
%    \begin{itemize}
%    \item
%       List of labels used in the current hook to label code chunks:
%    \begin{macrocode}
\seq_new:N \l_@@_labels_seq
%    \end{macrocode}
%    \item
%      Number of labels used in the current hook. In Knuth's algorithm
%      this is called $N$:
%    \begin{macrocode}
\int_new:N \l_@@_labels_int
%    \end{macrocode}
%    \item
%      The sorted code list to be build is managed using two pointers
%      one to the front of the queue and one to the rear. We model this
%      using token list pointers. Knuth calls them $F$ and $R$:
%    \begin{macrocode}
\tl_new:N \l_@@_front_tl
\tl_new:N \l_@@_rear_tl
%    \end{macrocode}
%    \item
%      The data for the start of the queue is kept in this token list,
%      it corresponds to what Don calls \texttt{QLINK[0]} but since we
%      aren't manipulating individual words in memory it is slightly
%      differently done:
%    \begin{macrocode}
\tl_new:c { \@@_tl_csname:n { 0 } }
%    \end{macrocode}
%
%    \end{itemize}
%  \end{macro}


%  \begin{macro}{\@@_initialize_single:NNNNn,\@@_initialize_single:ccccn}
%
%    \cs{@@_initialize_single:NNNNn} implements the sorting of the code
%    chunks for a hook and saves the result in the token list for fast
%    execution (\verb=#3=). The arguments are \meta{hook-code-plist},
%    \meta{hook-code-tl}, \meta{hook-next-code-tl},
%    \meta{hook-ordered-labels-clist} and \meta{hook-name} (the latter
%    is only used for debugging---the \meta{hook-rule-plist} is accessed
%    using the \meta{hook-name}).
%
%    The additional complexity compared to Don's algorithm is that we
%    do not use simple positive integers but have arbitrary
%    alphanumeric labels. As usual Don's data structures are chosen in
%    a way that one can omit a lot of tests and I have mimicked that as
%    far as possible. The result is a restriction I do not test for at
%    the moment: a label can't be equal to the number 0!  \fmi{Needs
%    checking for, just in case}
%    \begin{macrocode}
\cs_new_protected:Npn \@@_initialize_single:NNNNn #1#2#3#4#5 {
%    \end{macrocode}
%    
%    \begin{macrocode}
  \debug_suspend:
%    \end{macrocode}
%    Step T1: Initialize the data structure \ldots
%    \begin{macrocode}
  \seq_clear:N \l_@@_labels_seq
  \int_zero:N  \l_@@_labels_int
%    \end{macrocode}
%
%    Store the name of the hook:
%    \begin{macrocode}
  \tl_set:Nn \l_@@_cur_hook_tl {#5}
%    \end{macrocode}
%    
%    We loop over the property list holding the code and record all
%    labels listed there. Only rules for those labels are of interest
%    to us. While we are at it we count them (which gives us the $N$
%    in Knuth's algorithm.  The prefix |label_| is added to the variables
%    to ensure that labels named |front|, |rear|, |labels|, or |return|
%    don't interact with our code.
%    \begin{macrocode}
  \prop_map_inline:Nn #1
     {
       \int_incr:N \l_@@_labels_int
       \seq_put_right:Nn \l_@@_labels_seq {##1}
       \tl_set:cn { \@@_tl_csname:n {##1} }{0}     % the counter k for number of
                                                   % j before k rules
       \seq_clear_new:c { \@@_seq_csname:n {##1} } % sequence of successors to k
                                                   % i.e., k before j rules (stores
                                                   % the names of the j's)
     }
%    \end{macrocode}
%    Steps T2 and T3: Sort the relevant rules into the data structure\ldots
%    
%    This loop constitutes a square matrix of the labels in |#1| in the
%    vertical and the horizontal directions.  However since the rule
%    $l_A\meta{rel}l_B$ is the same as $l_B\meta{rel}^{-1}l_A$ we can cut
%    the loop short at the diagonal of the matrix (\emph{i.e.}, when
%    both labels are equal), saving a good amount of time.  The way the
%    rules were set up (see the implementation of \cs{@@_rule_before_gset:nnn}
%    above) ensures that we have no rule in the ignored side of the
%    matrix, and all rules are seen.  The rules are applied in
%    \cs{@@_apply_label_pair:nnn}, which takes the properly-ordered pair
%    of labels as argument.
%    \begin{macrocode}
  \prop_map_inline:Nn #1
    {
      \prop_map_inline:Nn #1
        {
          \@@_if_label_case:nnnnn {##1} {####1}
            { \prop_map_break: }
            { \@@_apply_label_pair:nnn {##1} {####1} }
            { \@@_apply_label_pair:nnn {####1} {##1} }
                {#5}
        }
    }
%    \end{macrocode}
%    Take a breath and take a look at the data structures that have
%    been set up:
%    \begin{macrocode}
  \@@_debug:n { \@@_debug_label_data:N #1 }
%    \end{macrocode}
%    
%
%    Step T4:
%    \begin{macrocode}
  \tl_set:Nn \l_@@_rear_tl { 0 }
  \tl_set:cn { \@@_tl_csname:n { 0 } } { 0 } % really {l_@@_label_ \l_@@_rear_tl _tl}
  \seq_map_inline:Nn \l_@@_labels_seq
      {
        \int_compare:nNnT { \cs:w \@@_tl_csname:n {##1} \cs_end: } = 0
            {
              \tl_set:cn { \@@_tl_csname:n { \l_@@_rear_tl } }{##1}
              \tl_set:Nn \l_@@_rear_tl {##1}
            }
      }
  \tl_set_eq:Nc \l_@@_front_tl { \@@_tl_csname:n { 0 } }
%    \end{macrocode}
%    
%    \begin{macrocode}
  \tl_gclear:N #2
  \clist_gclear:N #4
%    \end{macrocode}
%
%    The whole loop combines steps T5--T7:
%    \begin{macrocode}
  \bool_while_do:nn { ! \str_if_eq_p:Vn \l_@@_front_tl { 0 } }
       {
%    \end{macrocode}
%    This part is step T5:
%    \begin{macrocode}
         \int_decr:N \l_@@_labels_int
         \prop_get:NVN #1 \l_@@_front_tl \l_@@_return_tl
         \@@_tl_gput:NV #2 \l_@@_return_tl
%    \end{macrocode}
%    
%    \begin{macrocode}
         \@@_clist_gput:NV #4 \l_@@_front_tl
         \@@_debug:n{ \iow_term:x{Handled~ code~ for~ \l_@@_front_tl} }
%    \end{macrocode}
%
%    This is step T6 except that we don't use a pointer $P$ to move
%    through the successors, but instead use \verb=##1= of the mapping
%    function.
%    \begin{macrocode}
         \seq_map_inline:cn { \@@_seq_csname:n { \l_@@_front_tl } }
             {
               \tl_set:cx { \@@_tl_csname:n {##1} }
                   { \int_eval:n { \cs:w \@@_tl_csname:n {##1} \cs_end: - 1 } }
               \int_compare:nNnT { \cs:w \@@_tl_csname:n {##1} \cs_end: } = 0
                   {
                     \tl_set:cn { \@@_tl_csname:n { \l_@@_rear_tl } } {##1}
                     \tl_set:Nn \l_@@_rear_tl            {##1}
                   }
             }
%    \end{macrocode}
%    and step T7:
%    \begin{macrocode}
         \tl_set_eq:Nc \l_@@_front_tl { \@@_tl_csname:n { \l_@@_front_tl } }
%    \end{macrocode}
%
%    This is step T8: If we haven't moved the code for all labels
%    (i.e., if \cs{l_@@_labels_int} is still greater than zero) we
%    have a loop and our partial order can't be flattened out.
%    \begin{macrocode}
       }
  \int_compare:nNnF \l_@@_labels_int = 0
      {
        \iow_term:x{====================}
        \iow_term:x{Error:~ label~ rules~ are~ incompatible:}
%    \end{macrocode}
%
%    This is not really the information one needs in the error case
%    but will do for now \ldots \fmi{fix}
%    \begin{macrocode}
        \@@_debug_label_data:N #1
        \iow_term:x{====================}
      }
%    \end{macrocode}
%    After we have added all hook code to \verb=#2= we finish it off
%    with adding extra code for a one time execution. That is stored
%    in \verb=#3= but is normally empty.
%    \begin{macrocode}
  \tl_gput_right:Nn #2 {#3}
%    \end{macrocode}
%    
%    \begin{macrocode}
  \debug_resume:
}
%    \end{macrocode}
%    
%    \begin{macrocode}
\cs_generate_variant:Nn \@@_initialize_single:NNNNn {cccc}
%    \end{macrocode}
%  \end{macro}



%  \begin{macro}{\@@_tl_gput:NV,\@@_clist_gput:NV}
%    These append either on the right (normal hook) or on the left
%    (reversed hook). This is setup up in
%    \cs{@@_initialize_hook_code:n}, elsewhere their behavior is undefined.
%    \begin{macrocode}
\cs_new:Npn \@@_tl_gput:NV     {\ERROR}
\cs_new:Npn \@@_clist_gput:NV  {\ERROR}
%    \end{macrocode}
%  \end{macro}


%
%  \begin{macro}{\@@_apply_label_pair:nnn,\@@_label_if_exist_apply:nnnF}
%
%    This is the payload of steps T2 and T3 executed in the loop described
%    above. This macro assumes |#1| and |#2| are ordered, which means that
%    any rule pertaining the pair |#1| and |#2| is
%    \cs{g_@@_\meta{hook}_rule_\#1\string|\#2_tl}, and not
%    \cs{g_@@_\meta{hook}_rule_\#2\string|\#1_tl}.  This also saves a great deal
%    of time since we only need to check the order of the labels once.
%
%    The arguments here are \meta{label1}, \meta{label2}, \meta{hook}, and
%    \meta{hook-code-plist}.  We are about to apply the next rule and
%    enter it into the data structure.  \cs{@@_apply_label_pair:nnn} will
%    just call \cs{@@_label_if_exist_apply:nnnF} for the \meta{hook}, and
%    if no rule is found, also try the \meta{hook} name \verb=??=
%    denoting a default hook rule.
%
%    \cs{@@_label_if_exist_apply:nnnF} will check if the rule exists for
%    the given hook, and if so call \cs{@@_apply_rule:nnn}.
%    \begin{macrocode}
\cs_new_protected:Npn \@@_apply_label_pair:nnn #1#2#3
  {
%    \end{macrocode}
%    Extra complication: as we use default rules and local hook specific
%    rules we first have to check if there is a local rule and if that
%    exist use it. Otherwise check if there is a default rule and use
%    that.
%    \begin{macrocode}
    \@@_label_if_exist_apply:nnnF {#1} {#2} {#3}
      {
%    \end{macrocode}
%    If there is no hook-specific rule we check for a default one and
%    use that if it exists.
%    \begin{macrocode}
        \@@_label_if_exist_apply:nnnF {#1} {#2} { ?? } { }
      }
  }
\cs_new_protected:Npn \@@_label_if_exist_apply:nnnF #1#2#3
  {
    \if_cs_exist:w g_@@_ #3 _rule_ #1 | #2 _tl \cs_end:
%    \end{macrocode}
%    What to do precisely depends on the type of rule we have
%    encountered. If it is a \texttt{before} rule it will be handled by the
%    algorithm but other types need to be managed differently. All
%    this is done in \cs{@@_apply_rule:nnnN}.
%    \begin{macrocode}
      \@@_apply_rule:nnn {#1} {#2} {#3}
      \exp_after:wN \use_none:n
    \else:
      \use:nn
    \fi:
  }
%    \end{macrocode}
%  \end{macro}




%  \begin{macro}{\@@_apply_rule:nnn}
%    This is the code executed in steps T2 and T3 while looping through
%    the matrix  This is part of step T3. We are about to apply the next
%    rule and enter it into the data structure. The arguments are
%    \meta{label1}, \meta{label2}, \meta{hook-name}, and \meta{hook-code-plist}.
%    \begin{macrocode}
\cs_new_protected:Npn \@@_apply_rule:nnn #1#2#3
  {
    \cs:w @@_apply_
      \cs:w g_@@_#3_reversed_tl \cs_end: rule_
        \cs:w g_@@_ #3 _rule_ #1 | #2 _tl \cs_end: :nnn \cs_end:
      {#1} {#2} {#3}
  }
%    \end{macrocode}
% \end{macro}
%
%  \begin{macro}{\@@_apply_rule_<:nnn,\@@_apply_rule_>:nnn}
%    The most common cases are \texttt{\string<} and \texttt{\string>} so we handle
%    that first.  They are relations $\prec$ and $\succ$ in TAOCP, and
%    they dictate sorting.
%    \begin{macrocode}
\cs_new_protected:cpn { @@_apply_rule_<:nnn } #1#2#3
  {
    \@@_debug:n { \@@_msg_pair_found:nnn {#1} {#2} {#3} }
    \tl_set:cx { \@@_tl_csname:n {#2} }
       { \int_eval:n{ \cs:w \@@_tl_csname:n {#2} \cs_end: + 1 } }
    \seq_put_right:cn{ \@@_seq_csname:n {#1} }{#2}
  }
\cs_new_protected:cpn { @@_apply_rule_>:nnn } #1#2#3
  {
    \@@_debug:n { \@@_msg_pair_found:nnn {#1} {#2} {#3} }
    \tl_set:cx { \@@_tl_csname:n {#1} }
       { \int_eval:n{ \cs:w \@@_tl_csname:n {#1} \cs_end: + 1 } }
    \seq_put_right:cn{ \@@_seq_csname:n {#2} }{#1}
  }
%    \end{macrocode}
% \end{macro}
%
% \begin{macro}{\@@_apply_rule_xE:nnn,\@@_apply_rule_xW:nnn}
%   These relations make two labels incompatible within a hook.
%   |xE| makes raises an error if the labels are found in the same
%   hook, and |xW| makes it a warning.
%    \begin{macrocode}
\cs_new_protected:cpn { @@_apply_rule_xE:nnn } #1#2#3
  {
    \@@_debug:n { \@@_msg_pair_found:nnn {#1} {#2} {#3} }
    \msg_error:nnnnnn { hooks } { labels-incompatible }
      {#1} {#2} {#3} { 1 }
    \use:c { @@_apply_rule_->:nnn } {#1} {#2} {#3}
    \use:c { @@_apply_rule_<-:nnn } {#1} {#2} {#3}
  }
\cs_new_protected:cpn { @@_apply_rule_xW:nnn } #1#2#3
  {
    \@@_debug:n { \@@_msg_pair_found:nnn {#1} {#2} {#3} }
    \msg_warning:nnnnnn { hooks } { labels-incompatible }
      {#1} {#2} {#3} { 0 }
  }
%    \end{macrocode}
% \end{macro}
%
%  \begin{macro}{\@@_apply_rule_->:nnn,\@@_apply_rule_<-:nnn}
%    If we see \texttt{\detokenize{->}} we have to drop code for label
%    \verb=#3= and carry on. We could do a little better and trop
%    everything for that label since it doesn't matter where we sort
%    in the empty code. However that would complicate the algorithm a
%    lot with little gain. So we still unnecessarily try to sort it in
%    and depending on the rules that might result in a loop that is
%    otherwise resolved. If that turns out to be a real issue, we can
%    improve the code.
%
%    Here the code is removed from \cs{l_@@_cur_hook_tl} rather than
%    \verb=#3= because the latter may be \verb=??=, and the default
%    hook doesn't store any code.  Removing from \cs{l_@@_cur_hook_tl}
%    makes default rules \verb=->= and  \verb=<-= work properly.
%    \begin{macrocode}
\cs_new_protected:cpn { @@_apply_rule_->:nnn } #1#2#3
  {
    \@@_debug:n
       {
         \@@_msg_pair_found:nnn {#1} {#2} {#3}
         \iow_term:x{--->~ Drop~ '#2'~ code~ from~
           \iow_char:N \\ g_@@_ \l_@@_cur_hook_tl _code_prop ~ because~ of~ '#1' }
       }
    \prop_gput:cnn { g_@@_ \l_@@_cur_hook_tl _code_prop } {#2} { }
  }
\cs_new_protected:cpn { @@_apply_rule_<-:nnn } #1#2#3
  {
    \@@_debug:n
       {
         \@@_msg_pair_found:nnn {#1} {#2} {#3}
         \iow_term:x{--->~ Drop~ '#1'~ code~ from~
           \iow_char:N \\ g_@@_ \l_@@_cur_hook_tl _code_prop ~ because~ of~ '#2' }
       }
    \prop_gput:cnn { g_@@_ \l_@@_cur_hook_tl _code_prop } {#1} { }
  }
%    \end{macrocode}
%  \end{macro}

% \begin{macro}{
%     \@@_apply_-rule_<:nnn,
%     \@@_apply_-rule_>:nnn,
%     \@@_apply_-rule_<-:nnn,
%     \@@_apply_-rule_->:nnn,
%     \@@_apply_-rule_x:nnn,
%   }
%   Reversed rules.
%    \begin{macrocode}
\cs_new_eq:cc { @@_apply_-rule_<:nnn  } { @@_apply_rule_>:nnn }
\cs_new_eq:cc { @@_apply_-rule_>:nnn  } { @@_apply_rule_<:nnn }
\cs_new_eq:cc { @@_apply_-rule_<-:nnn } { @@_apply_rule_<-:nnn }
\cs_new_eq:cc { @@_apply_-rule_->:nnn } { @@_apply_rule_->:nnn }
\cs_new_eq:cc { @@_apply_-rule_xE:nnn  } { @@_apply_rule_xE:nnn }
\cs_new_eq:cc { @@_apply_-rule_xW:nnn  } { @@_apply_rule_xW:nnn }
%    \end{macrocode}
% \end{macro}


% \begin{macro}{\@@_msg_pair_found:nnn}
%   A macro to avoid moving this many tokens around.
%    \begin{macrocode}
\cs_new_protected:Npn \@@_msg_pair_found:nnn #1#2#3
  {
    \iow_term:x{~ \str_if_eq:nnTF {#3} {??} {default} {~normal} ~
               rule~ \@@_label_pair:nn {#1} {#2}:~
             \use:c { g_@@_#3_rule_ \@@_label_pair:nn {#1} {#2} _tl } ~ found}
  }
%    \end{macrocode}
% \end{macro}


%  \begin{macro}{\@@_debug_label_data:N}
%    
%    \begin{macrocode}
\cs_new_protected:Npn \@@_debug_label_data:N #1 {
  \iow_term:x{Code~ labels~ for~ sorting:}
  \iow_term:x{~ \seq_use:Nnnn\l_@@_labels_seq {~and~}{,~}{~and~} }  % fix name!
  \iow_term:x{^^J Data~ structure~ for~ label~ rules:}
  \prop_map_inline:Nn #1
       {
         \iow_term:x{~ ##1~ =~ \tl_use:c{ \@@_tl_csname:n {##1} }~ ->~
           \seq_use:cnnn{ \@@_seq_csname:n {##1} }{~->~}{~->~}{~->~}
         }
       }
  \iow_term:x{}
}
%    \end{macrocode}
%  \end{macro}



%  \begin{macro}{\hook_log:n}
%    This writes out information about the hook given in its argument
%    onto the terminal and the \texttt{.log} file.
%    \begin{macrocode}
\cs_new_protected:Npn \hook_log:n #1
  {
    \exp_args:Nx \@@_log:n
      { \@@_parse_label_default:nn {#1} { top-level } }
  }
\cs_new_protected:Npn \@@_log:n #1
  {
    \iow_term:x{^^JThe~ hook~ '#1':}
%    \end{macrocode}
%    
%    \begin{macrocode}
    \hook_if_exist:nF {#1}
      { \iow_term:x {~Hook~ is~ not~ declared!} }
    \@@_if_exist:nTF {#1}
      {
        \iow_term:x{~Code~ chunks:}
        \prop_if_empty:cTF {g_@@_#1_code_prop}
          { \iow_term:x{\@spaces ---} }
          {
            \prop_map_inline:cn {g_@@_#1_code_prop}
              { \iow_term:x{\@spaces ##1~ ->~ \tl_to_str:n{##2} } }
          }
%    \end{macrocode}
%    
%    \begin{macrocode}
        \iow_term:x{~Extra~ code~ next~ invocation:}
        \iow_term:x{\@spaces
          \tl_if_empty:cTF { g_@@_#1_next_code_tl }
            {---} {->~ \str_use:c{g_@@_#1_next_code_tl} } }
%    \end{macrocode}
%
%    \fmi{This is currently only displaying the local rules, but it
%         should also show the matching global rules!}
%
%    \begin{macrocode}
        \iow_term:x{~Rules:}
        \prop_if_empty:cTF {g_@@_#1_rules_prop}
          { \iow_term:x{\@spaces ---} }
          { \prop_map_inline:cn {g_@@_#1_rules_prop}
              { \iow_term:x{\@spaces ##1~ with~ relation~ ##2} }
          }
%    \end{macrocode}
%    
%    \begin{macrocode}
        \hook_if_exist:nT {#1}
          { \iow_term:x { ~Execution~ order
               \prop_if_empty:cTF {g_@@_#1_rules_prop}
                 { \@@_if_reversed:nT {#1}
                        { ~ (after~ reversal) }
                 }
                 { ~ (after~
                   \@@_if_reversed:nT {#1} {reversal~ and~}
                   applying~ rules)
                 }
               :    
              }
            \iow_term:x { \@spaces
              \clist_if_empty:cTF{g_@@_#1_labels_clist}
                 {not~ set~ yet}
                 { \clist_use:cnnn {g_@@_#1_labels_clist}
                                   { ,~ } { ,~ } { ,~ }   }
            }
          }
      }
      { \iow_term:n { ~The~hook~is~empty. } }
    \iow_term:n { }
  }
%    \end{macrocode}
%    
%  \end{macro}



%  \subsection{Specifying code for next invocation}
%
%
%
%
%
%  \begin{macro}{\hook_gput_next_code:nn}
%    
%    \begin{macrocode}
\cs_new_protected:Npn \hook_gput_next_code:nn #1
  {
    \exp_args:Nx \@@_gput_next_code:nn
      { \@@_parse_label_default:nn {#1} { top-level } }
  }
\cs_new_protected:Npn \@@_gput_next_code:nn #1 #2
  {
    \@@_declare:n {#1}
    \hook_if_exist:nTF {#1}
      { \@@_gput_next_do:nn {#1} {#2} }
      { \@@_try_declaring_generic_next_hook:nn {#1} {#2} }
  }
\cs_new_protected:Npn \@@_gput_next_do:nn #1 #2
  {
    \tl_gput_right:cn { g_@@_#1_next_code_tl }
      { #2 \tl_gclear:c { g_@@_#1_next_code_tl } }
  }
%    \end{macrocode}
%  \end{macro}
%
%
% \subsection{Using the hook}
%
% \begin{macro}{\hook_use:n}
% \begin{macro}[EXP]{\@@_use_initialized:n}
% \begin{macro}{\@@_preamble_hook:n}
%   \cs{hook_use:n} as defined here is used in the preamble, where
%   hooks aren't initialised by default.  \cs{@@_use_initialized:n} is
%   also defined, which is the non-\tn{protected} version for use within
%   the document.  Their definition is identical, except for the
%   \cs{@@_preamble_hook:n} (which wouldn't hurt in the expandable
%   version, but it would be an unnecessary extra expansion).
%
%   \cs{@@_use_initialized:n} holds the expandable definition while in
%   the preamble. \cs{@@_preamble_hook:n} initialises the hook in the
%   preamble, and is redefined to \cs{use_none:n} at |\begin{document}|.
%
%   Both versions do the same internally:  check if the hook exist as
%   given, and if so use it as quickly as possible.  If it doesn't
%   exist, the a call to \cs{@@_use:wn} checks for file hooks.
%
%   At |\begin{document}|, all hooks are initialised, and any change in
%   them causes an update, so \cs{hook_use:n} can be made expandable.
%   This one is better not protected so that it can expand into nothing
%   if containing no code. Also important in case of generic hooks that
%   we do not generate a \cs{relax} as a side effect of checking for a
%   csname. In contrast to the \TeX{} low-level
%   \verb=\csname ...\endcsname= construct \cs{tl_if_exist:c} is
%   careful to avoid this.
%    \begin{macrocode}
\cs_new_protected:Npn \hook_use:n #1
  {
    \tl_if_exist:cTF { g_@@_#1_code_tl }
      {
        \@@_preamble_hook:n {#1}
        \cs:w g_@@_#1_code_tl \cs_end:
      }
      { \@@_use:wn #1 / \s_@@_mark {#1} }
  }
\cs_new:Npn \@@_use_initialized:n #1
  {
    \tl_if_exist:cTF { g_@@_#1_code_tl }
      { \cs:w g_@@_#1_code_tl \cs_end: }
      { \@@_use:wn #1 / \s_@@_mark {#1} }
  }
\cs_new_protected:Npn \@@_preamble_hook:n #1
  { \@@_initialize_hook_code:n {#1} }
%    \end{macrocode}
% \end{macro}
% \end{macro}
% \end{macro}
%
% \begin{macro}[EXP]{\@@_use:wn,\@@_try_file_hook:n,\@@_if_exist_use:n}
%   \cs{@@_use:wn} does a quick check to test if the current hook is a
%   file hook: those need a special treatment.  If it is not, the hook
%   does not exist.  If it is, then \cs{@@_try_file_hook:n} is called,
%   and checks that that the current hook is a file-specific hook using
%   \cs{@@_if_file_hook:wTF}.  If it's not, then it's a generic |file/|
%   hook and is used if it exist.
%
%   If it is a file-specific hook, it passes through the same
%   normalisation as during declaration, and then it is used if defined.
%
%   \cs{@@_if_exist_use:n} checks if the hook exist, and calls
%   \cs{@@_preamble_hook:n} if so, then uses the hook.
%    \begin{macrocode}
\cs_new:Npn \@@_use:wn #1 / #2 \s_@@_mark #3
  {
    \str_if_eq:nnTF {#1} { file }
      { \@@_try_file_hook:n {#3} }
      { } % Hook doesn't exist
  }
\cs_new:Npn \@@_try_file_hook:n #1
  {
    \@@_if_file_hook:wTF #1 / / \s_@@_mark
      {
        \exp_args:Ne \@@_if_exist_use:n
          { \exp_args:Ne \@@_file_hook_normalise:n {#1} }
      }
      { \@@_if_exist_use:n {#1} } % file/ generic hook (e.g. file/before)
  }
\cs_new:Npn \@@_if_exist_use:n #1
  {
    \tl_if_exist:cT { g_@@_#1_code_tl }
      {
        \@@_preamble_hook:n {#1}
        \cs:w g_@@_#1_code_tl \cs_end:
      }
  }
%    \end{macrocode}
% \end{macro}
%
%  \begin{macro}{\hook_use_once:n}
%    For hooks that can and should be used only once we have a special
%    use command that remembers the hook name in
%    \cs{g_@@_execute_immediately_clist}. This has the effect that any
%    further code added to the hook is executed immediately rather
%    than stored in the hook.
%    \begin{macrocode}
\cs_new_protected:Npn \hook_use_once:n #1
  {
    \tl_if_exist:cT { g_@@_#1_code_tl }
      {
        \clist_gput_left:Nn \g_@@_execute_immediately_clist {#1}
        \hook_use:n {#1}
      }
  }
%    \end{macrocode}
%  \end{macro}

% \subsection{Querying a hook}
%
% Simpler data types, like token lists, have three possible states; they
% can exist and be empty, exist and be non-empty, and they may not
% exist, in which case emptiness doesn't apply (though
% \cs{tl_if_empty:N} returns false in this case).
%
% Hooks are a bit more complicated: they have four possible states.
% A hook may exist or not, and either way it may or may not be empty
% (even a hook that doesn't exist may be non-empty).
%
% A hook is said to be empty when no code was added to it, either to
% its permanent code pool, or to its ``next'' token list.  The hook
% doesn't need to be declared to have code added to its code pool
% (it may happen that a package $A$ defines a hook \hook{foo}, but
% it's loaded after package $B$, which adds some code to that hook.
% In this case it is important that the code added by package $B$ is
% remembered until package $A$ is loaded).
%
% A hook is said to exist when it was declared with \cs{hook_new:n} or
% some variant thereof.
%
% \begin{macro}[pTF]{\hook_if_empty:n}
%   Test if a hook is empty (that is, no code was added to that hook).
%   A hook being empty means that \emph{both} its
%   \cs[no-index]{g_@@_\meta{hook}_code_prop} and its
%   \cs[no-index]{g_@@_\meta{hook}_next_code_tl} are empty.
%    \begin{macrocode}
\prg_new_conditional:Npnn \hook_if_empty:n #1 { p , T , F , TF }
  {
    \@@_if_exist:nTF {#1}
      {
        \bool_lazy_and:nnTF
            { \prop_if_empty_p:c { g_@@_#1_code_prop } }
            { \tl_if_empty_p:c { g_@@_#1_next_code_tl } }
          { \prg_return_true: }
          { \prg_return_false: }
      }
      { \prg_return_true: }
  }
%    \end{macrocode}
% \end{macro}

% \begin{macro}[pTF]{\hook_if_exist:n}
%   A canonical way to test if a hook exists.  A hook exists if the
%   token list that stores the sorted code for that hook,
%   \cs[no-index]{g_@@_\meta{hook}_code_tl}, exists.  The property list
%   \cs[no-index]{g_@@_\meta{hook}_code_prop} cannot be used here
%   because often it is necessary to add code to a hook without knowing
%   if such hook was already declared, or even if it will ever be
%   (for example, in case the package that defines it isn't loaded).
%    \begin{macrocode}
\prg_new_conditional:Npnn \hook_if_exist:n #1 { p , T , F , TF }
  {
    \tl_if_exist:cTF { g_@@_#1_code_tl }
      { \prg_return_true: }
      { \prg_return_false: }
  }
%    \end{macrocode}
% \end{macro}

% \begin{macro}[pTF]{\@@_if_exist:n}
%   An internal check if the hook has already been declared with
%   \cs{@@_declare:n}.  This means that the hook was already used somehow
%   (a code chunk or rule was added to it), but it still wasn't declared
%   with \cs{hook_new:n}.
%    \begin{macrocode}
\prg_new_conditional:Npnn \@@_if_exist:n #1 { p , T , F , TF }
  {
    \prop_if_exist:cTF { g_@@_#1_code_prop }
      { \prg_return_true: }
      { \prg_return_false: }
  }
%    \end{macrocode}
% \end{macro}
%
% \begin{macro}[pTF]{\@@_if_reversed:n}
%   An internal conditional that checks if a hook is reversed.
%    \begin{macrocode}
\prg_new_conditional:Npnn \@@_if_reversed:n #1 { p , T , F , TF }
  {
    \if_int_compare:w \cs:w g_@@_#1_reversed_tl \cs_end: 1 < 0 \exp_stop_f:
      \prg_return_true:
    \else:
      \prg_return_false:
    \fi:
  }
%    \end{macrocode}
% \end{macro}


%  \begin{macro}{\g_@@_execute_immediately_clist}
%    List of hooks that from no on should not longer receive code.
%    \begin{macrocode}
\clist_new:N \g_@@_execute_immediately_clist
%    \end{macrocode}
%  \end{macro}
%
%  \subsection{Messages}
%
%    \begin{macrocode}
\msg_new:nnnn { hooks } { labels-incompatible }
  {
    Labels~`#1'~and~`#2'~are~incompatible
    \str_if_eq:nnF {#3} {??} { ~in~hook~`#3' } .~
    \int_compare:nNnT {#4} = { 1 }
      { The~code~for~both~labels~will~be~dropped. }
  }
  {
    LaTeX~found~two~incompatible~labels~in~the~same~hook.~
    This~indicates~an~incompatibility~between~packages.
  }
\msg_new:nnn { hooks } { empty-label }
  { Empty~code~label~\msg_line_context:.~Using~`#1'~instead. }
%    \end{macrocode}
%
%  \subsection{\LaTeXe{} package interface commands}
%


%  \begin{macro}{\NewHook,\NewReversedHook,\NewMirroredHookPair}
%    Declaring new hooks \ldots
%    \begin{macrocode}
\NewDocumentCommand \NewHook             { m }{ \hook_new:n {#1} }
\NewDocumentCommand \NewReversedHook     { m }{ \hook_new_reversed:n {#1} }
\NewDocumentCommand \NewMirroredHookPair { mm }{ \hook_new_pair:nn {#1}{#2} }
%    \end{macrocode}
%  \end{macro}

%  \begin{macro}{\AddToHook}
%    
%    \begin{macrocode}
\NewDocumentCommand \AddToHook { m o +m }
  {
    \clist_if_in:NnTF \g_@@_execute_immediately_clist {#1}
      {#3}
      { \hook_gput_code:nnn {#1} {#2} {#3} }
  }
%    \end{macrocode}
%  \end{macro}

%  \begin{macro}{\AddToHookNext}
%    
%    \begin{macrocode}
\NewDocumentCommand \AddToHookNext { m +m }
  { \hook_gput_next_code:nn {#1} {#2} }
%    \end{macrocode}
%  \end{macro}


%  \begin{macro}{\RemoveFromHook}
%    
%    \begin{macrocode}
\NewDocumentCommand \RemoveFromHook { m o }
  { \hook_gremove_code:nn {#1} {#2} }
%    \end{macrocode}
%  \end{macro}
%
% \begin{macro}{\DeclareDefaultHookLabel}
% \begin{macro}{\@@_curr_name_push:n,\@@_curr_name_pop:}
%   The token list \cs{g_@@_hook_curr_name_tl} stores the name of the
%   current package/file to be used as label for hooks.
%   Providing a consistent interface is tricky, because packages can
%   be loaded within packages, and some packages may not use
%   \cs{DeclareDefaultHookLabel} to change the default label (in which case
%   \cs{@currname} is used, if set).
%
%   To pull that off, we keep a stack that contains the default label
%   for each level of input.  The bottom of the stack contains the
%   default label for the top-level.  Since the string \verb|top-level|
%   is hardcoded, here this item of the stack is empty.  Also, since
%   we're in an input level, add \verb|lthooks| to the stack as well.
%   This stack should never go empty, so we loop through \LaTeXe's
%   file name stack, and add empty entries to \cs{g_@@_name_stack_seq}
%   for each item in that stack.  The last item is the \verb|top-level|,
%   which also gets an empty entry.
%
%   Also check for the case we're loading \texttt{lthooks} in the
%   \LaTeXe{} kernel.  In that case, \cs{@currname} isn't \verb|lthooks|
%   and just the top-level is added to the stack as an empty entry.
%    \begin{macrocode}
\str_if_eq:VnTF \@currname { lthooks }
  {
    \seq_gpush:Nn \g_@@_name_stack_seq { lthooks }
    \cs_set_protected:Npn \@@_tmp:w #1 #2 #3
      {
        \quark_if_recursion_tail_stop:n {#1}
        \seq_gput_right:Nn \g_@@_name_stack_seq { }
        \@@_tmp:w
      }
    \exp_after:wN \@@_tmp:w
      \@currnamestack
      \q_recursion_tail \q_recursion_tail
      \q_recursion_tail \q_recursion_stop
  }
  { \seq_gpush:Nn \g_@@_name_stack_seq { } }
%    \end{macrocode}
%
%   Two commands keep track of the stack: when a file is input,
%   \cs{@@_curr_name_push:n} pushes an (empty by default) label to the
%   stack:
%    \begin{macrocode}
\cs_new_protected:Npn \@@_curr_name_push:n #1
  {
    \seq_gpush:Nn \g_@@_name_stack_seq {#1}
    \tl_gset:Nn \g_@@_hook_curr_name_tl {#1}
  }
%
%    \end{macrocode}
%   and when an input is over, the topmost item of the stack is popped,
%   since the label will not be used again, and \cs{g_@@_hook_curr_name_tl}
%   is updated to the now topmost item of the stack:
%    \begin{macrocode}
\cs_new_protected:Npn \@@_curr_name_pop:
  {
    \seq_gpop:NN \g_@@_name_stack_seq \l_@@_return_tl
    \seq_get:NNTF \g_@@_name_stack_seq \l_@@_return_tl
      { \tl_gset_eq:NN \g_@@_hook_curr_name_tl \l_@@_return_tl }
      { \ERROR_should_not_happen }
  }
%    \end{macrocode}
%
%   The token list \cs{g_@@_hook_curr_name_tl} is but a mirror of the top
%   of the stack.
%
%   Now define a wrapper that replaces the top of the stack with the
%   argument, and updates \cs{g_@@_hook_curr_name_tl} accordingly.
%    \begin{macrocode}
\NewDocumentCommand \DeclareDefaultHookLabel { m }
  {
    \seq_gpop:NN \g_@@_name_stack_seq \l_@@_return_tl
    \@@_curr_name_push:n {#1}
  }
%    \begin{macrocode}
%
%   The push and pop macros are injected in \cs{@pushfilename} and
%   \cs{@popfilename} so that they correctly keep track of the label.s
%    \begin{macrocode}
% TODO! \pho{Properly integrate in the kernel}
\tl_gput_left:Nn \@pushfilename { \@@_curr_name_push:n { } }
\tl_gput_left:Nn \@popfilename { \@@_curr_name_pop: }
% TODO! \pho{Properly integrate in the kernel}
%    \end{macrocode}
% \end{macro}
% \end{macro}
%
%
%
%
%  \begin{macro}{\UseHook}
%    Avoid the overhead of \pkg{xparse} and its protection that we
%    don't want here (since the hook should vanish without trace if empty)!
%    \begin{macrocode}
\newcommand \UseHook { \hook_use:n }
%    \end{macrocode}
%  \end{macro}
%
%  \begin{macro}{\UseOneTimeHook}
%
%    \begin{macrocode}
\cs_new_protected:Npn \UseOneTimeHook { \hook_use_once:n }
%    \end{macrocode}
%  \end{macro}



%  \begin{macro}{\ShowHook}
%    
%    \begin{macrocode}
\cs_new_protected:Npn \ShowHook { \hook_log:n }
%    \end{macrocode}
%  \end{macro}

%  \begin{macro}{\DebugHookOn,\DebugHookOff}
%    
%    \begin{macrocode}
\cs_new_protected:Npn \DebugHookOn { \hook_debug_on: }
\cs_new_protected:Npn \DebugHookOff { \hook_debug_off: }
%    \end{macrocode}
%  \end{macro}



%  \begin{macro}{\DeclareHookRule}
%    
%    \begin{macrocode}
\NewDocumentCommand \DeclareHookRule { m m m m }
{ \hook_gset_rule:nnnn {#1}{#2}{#3}{#4} }
%    \end{macrocode}
%  \end{macro}

%  \begin{macro}{\DeclareDefaultHookRule}
%    This declaration is only supported before \verb=\begin{document}=.
%    \begin{macrocode}
\NewDocumentCommand \DeclareDefaultHookRule { m m m }
                    { \hook_gset_rule:nnnn {??}{#1}{#2}{#3} }
\@onlypreamble\DeclareDefaultHookRule                    
%    \end{macrocode}
%  \end{macro}
%
%  \begin{macro}{\ClearHookRule}
%    A special setup rule that removes an existing relation.
%    Basically {@@_rule_gclear:nnn} plus fixing the property list for debugging.
%    \fmi{Need an L3 interface, or maybe it should get dropped}
%    \begin{macrocode}
\NewDocumentCommand \ClearHookRule { m m m }
{ \hook_gset_rule:nnnn {#1}{#2}{unrelated}{#3} }
%    \end{macrocode}
%  \end{macro}
%
%
% \begin{macro}{\IfHookExistTF,\IfHookEmptyTF}
%    \begin{macrocode}
\NewExpandableDocumentCommand \IfHookExistTF { m }
  { \hook_if_exist:nTF {#1} }
\NewExpandableDocumentCommand \IfHookEmptyTF { m }
  { \hook_if_empty:nTF {#1} }
%    \end{macrocode}
% \end{macro}




%  \begin{macro}{\AtBeginDocument}
%    
%    \begin{macrocode}
\renewcommand\AtBeginDocument{\AddToHook{begindocument}}
%    \end{macrocode}
%  \end{macro}

%  \begin{macro}{\AtEndDocument}
%    
%    \begin{macrocode}
\renewcommand\AtEndDocument {\AddToHook{enddocument}}
%\renewcommand\AtEndDocument {\AddToHook{env/document/end}} % alternative impl
%    \end{macrocode}
%    
%  \end{macro}



%  \subsection{Set up existing \LaTeXe{} hooks}
%
%    As we are in a package calling \cs{NewHook} would label any
%    already set up hook code under the package name, but we want it
%    under the name \hook{top-level} so we pretend that \cs{@currname}
%    is empty.
%    \begin{macrocode}
\begingroup
  \def\@currname{}
%    \end{macrocode}
%    
%    \begin{macrocode}
  \NewHook{begindocument}
  \NewHook{enddocument}
%    \end{macrocode}
%    We need to initialize the mechanism at \verb=\begin{document}=
%    but obviously before everything else, so we sneak\footnote{This
%    needs to move to \cs{document} directly.}
%    \cs{@@_initialize_all:} into the \LaTeXe{} name of the hook.
%
%    We can't use \cs{tl_gput_left:Nn} because that complains about
%    \cs{@begindocumenthook} not starting with \cs{g_} so we do this
%    through the backdoor.
%    \begin{macrocode}
%  \tex_global:D\tl_put_left:Nn \@begindocumenthook
%      {\@@_initialize_all:}
%    \end{macrocode}
%    There aren't many other hooks at the moment:
%    \begin{macrocode}
  \NewHook{rmfamily}
  \NewHook{sffamily}
  \NewHook{ttfamily}
  \NewHook{defaultfamily}
%    \end{macrocode}
%    Not checked what this one does and whether it should be there (or
%    is a real ``hook''.
%    \begin{macrocode}
  \NewHook{documentclass}
%    \end{macrocode}
%    
%    \begin{macrocode}
\endgroup
%    \end{macrocode}
%
%
%
% \section{Generic hooks for environments}
%
%
%    \begin{macrocode}
\let\begin\relax  % avoid redeclaration message
%    \end{macrocode}
%    
%    \begin{macrocode}
\DeclareRobustCommand\begin[1]{%
  \UseHook{env/#1/before}%
  \@ifundefined{#1}%
    {\def\reserved@a{\@latex@error{Environment~#1~undefined}\@eha}}%
    {\def\reserved@a{\def\@currenvir{#1}%
        \edef\@currenvline{\on@line}%
        \@execute@begin@hook{#1}%
        \csname #1\endcsname}}%
  \@ignorefalse
  \begingroup\@endpefalse\reserved@a}
%    \end{macrocode}
%
%    Before the \cs{document} code is executed we have to first undo
%    the \cs{endgroup} as there should be none for this environment to
%    avoid that changes on top-level unnecessarily go to \TeX's
%    savestack, and we have to initialize all hooks in the hook system.
%    So we need to test for this environment name. But once it has be
%    found all this testing is no longer needed and so we redefine
%    \cs{@execute@begin@hook} to simply use the hook
%    \begin{macrocode}
\def\@execute@begin@hook #1{%
  \expandafter\ifx\csname #1\endcsname\document
    \endgroup
    \gdef\@execute@begin@hook##1{\UseHook{env/##1/begin}}%
    \@@_initialize_all:
    \@execute@begin@hook{#1}%
%    \end{macrocode}
%    If this is an environment before \verb=\begin{document}= we just
%    run the hook.
%    \begin{macrocode}
  \else
    \UseHook{env/#1/begin}%
  \fi
}    
%    \end{macrocode}
%    
%    \begin{macrocode}
\@namedef{end~}#1{%
  \UseHook{env/#1/end}%
  \csname end#1\endcsname\@checkend{#1}%
  \expandafter\endgroup\if@endpe\@doendpe\fi
  \UseHook{env/#1/after}%
  \if@ignore\@ignorefalse\ignorespaces\fi}%
%    \end{macrocode}
%    Version that fixes tlb3722 but the change should perhaps be made in
%    \pkg{tabularx} instead.
%    \begin{macrocode}
\@namedef{end~}#1{%
\romannumeral
\IfHookEmptyTF{env/#1/end}%
  {\expandafter\z@}%
  {\z@\UseHook{env/#1/end}}%
\csname end#1\endcsname\@checkend{#1}%
\expandafter\endgroup\if@endpe\@doendpe\fi
\UseHook{env/#1/after}%
\if@ignore\@ignorefalse\ignorespaces\fi}%
%    \end{macrocode}
%    
%    
%
%    We provide 4 high-level hook interfaces directly, the others only when
%    etoolbox is loaded
%    \begin{macrocode}
\newcommand\AtBeginEnvironment[1]    {\AddToHook{env/#1/begin}}
\newcommand\AtEndEnvironment[1]      {\AddToHook{env/#1/end}}
\newcommand\BeforeBeginEnvironment[1]{\AddToHook{env/#1/before}}
\newcommand\AfterEndEnvironment[1]   {\AddToHook{env/#1/after}}
%    \end{macrocode}
%    
%    \begin{macrocode}
%    We prevent the original package from loading:
%    \fmi{When the package is updated this needs removing!}
%    \begin{macrocode}
\expandafter\let\csname ver@etoolbox.sty\endcsname\fmtversion
%    \end{macrocode}

%    
%    
%    
%    
%    
%    
%    
%    
%    
%
% \section{Generic hooks for file loads}
%
%
%
%
%
%    
%
% \section{Hooks in \cs{begin} document}
%
%    Can't have \texttt{@{}@} notation here as this is \LaTeXe{} code
%    \ldots{} and makes for puzzling errors if the double \texttt{@}
%    signs get substituted.
%    \begin{macrocode}
%<@@=>
\ExplSyntaxOff
%    \end{macrocode}
%
%    The \hook{begindocument} hook was already set up earlier, here is now
%    the additional one (which was originally from the \pkg{etoolbox}
%    package under the name \texttt{afterpreamble}.
%    \begin{macrocode}
\NewHook{begindocument/end}
%    \end{macrocode}
%
%

%  \begin{macro}{\document}
%    
%    \begin{macrocode}
\def\document{%
%    \end{macrocode}
%    We do cancel the grouping as part of the \cs{begin} handling
%    (this is now done inside \cs{begin} instead) so that the
%    \hook{env/\meta{env}/begin} hook is not hidden inside \cs{begingroup}
%    \texttt{...} \cs{endgroup}.
%    \begin{macrocode}
%  \endgroup
%    \end{macrocode}
%    
%    \begin{macrocode}
  \@kernel@after@env@document@begin
%    \end{macrocode}
%
% Added hook to load \textsf{l3backend} code:
%    \begin{macrocode}
  \@expl@sys@load@backend@@
  \ifx\@unusedoptionlist\@empty\else
    \@latex@warning@no@line{Unused global option(s):^^J%
            \@spaces[\@unusedoptionlist]}%
  \fi
  \@colht\textheight
  \@colroom\textheight \vsize\textheight
  \columnwidth\textwidth
  \@clubpenalty\clubpenalty
  \if@twocolumn
    \advance\columnwidth -\columnsep
    \divide\columnwidth\tw@ \hsize\columnwidth \@firstcolumntrue
  \fi
  \hsize\columnwidth \linewidth\hsize
  \begingroup\@floatplacement\@dblfloatplacement
    \makeatletter\let\@writefile\@gobbletwo
    \global \let \@multiplelabels \relax
    \@input{\jobname.aux}%
  \endgroup
  \if@filesw
    \immediate\openout\@mainaux\jobname.aux
    \immediate\write\@mainaux{\relax}%
  \fi
  \process@table
  \let\glb@currsize\@empty  % Force math initialization.
  \normalsize
  \everypar{}%
  \ifx\normalsfcodes\@empty
    \ifnum\sfcode`\.=\@m
      \let\normalsfcodes\frenchspacing
    \else
      \let\normalsfcodes\nonfrenchspacing
    \fi
  \fi
  \ifx\document@default@language\m@ne
    \chardef\document@default@language\language
  \fi
  \@noskipsecfalse
  \let \@refundefined \relax
%    \end{macrocode}
%    
%    \begin{macrocode}
%  \let\AtBeginDocument\@firstofone
%  \@begindocumenthook
  \UseOneTimeHook{begindocument}%
  \@kernel@after@begindocument
%    \end{macrocode}
%    
%    \begin{macrocode}
  \ifdim\topskip<1sp\global\topskip 1sp\relax\fi
  \global\@maxdepth\maxdepth
  \global\let\@begindocumenthook\@undefined
  \ifx\@listfiles\@undefined
    \global\let\@filelist\relax
    \global\let\@addtofilelist\@gobble
  \fi
  \gdef\do##1{\global\let ##1\@notprerr}%
  \@preamblecmds
  \global\let \@nodocument \relax
  \global\let\do\noexpand
%    \end{macrocode}
%    
%    \begin{macrocode}
  \UseOneTimeHook{begindocument/end}%
  \ignorespaces}
%    \end{macrocode}
%    
%    \begin{macrocode}
\let\@kernel@after@begindocument\@empty
%    \end{macrocode}
%  \end{macro}
%
%
%  \begin{macro}{\@kernel@after@env@document@begin,\@kernel@hook@begindocument}
%    
%    \begin{macrocode}
\edef \@kernel@after@env@document@begin{%
  \let\expandafter\noexpand\csname
       g__hook_env/document/begin_code_tl\endcsname
  \noexpand\@empty}
%    \end{macrocode}
%    \begin{macrocode}
\let\@kernel@hook@begindocument\@empty
%    \end{macrocode}
%    
%  \end{macro}
%
%
% \section{Hooks in \cs{enddocument}}
%
%
%    
%    The \hook{enddocument} hook was already set up earlier, here are now
%    the additional ones:
%    \begin{macrocode}
\NewHook{enddocument/afterlastpage}
\NewHook{enddocument/afteraux}
\NewHook{enddocument/info}
\NewHook{enddocument/end}
%    \end{macrocode}



%  \begin{macro}{\enddocument}
%    
%    
%    \begin{macrocode}
\def\enddocument{%
   \UseHook{enddocument}%
   \@kernel@after@enddocument
   \@checkend{document}%
   \clearpage
   \UseHook{enddocument/afterlastpage}%
   \@kernel@after@enddocument@afterlastpage
   \begingroup
     \if@filesw
       \immediate\closeout\@mainaux
       \let\@setckpt\@gobbletwo
       \let\@newl@bel\@testdef
       \@tempswafalse
       \makeatletter \@@input\jobname.aux
     \fi
     \UseHook{enddocument/afteraux}%
%    \end{macrocode}
%    Next hook is expect to contain only code for writing info
%    messages on the terminal.
%    \begin{macrocode}
     \UseHook{enddocument/info}%
   \endgroup
   \UseHook{enddocument/end}%
   \deadcycles\z@\@@end}
%    \end{macrocode}
%    The two kernel hooks above are used by the shipout code.   
%    \begin{macrocode}
\let\@kernel@after@enddocument\@empty
\let\@kernel@after@enddocument@afterlastpage\@empty
%    \end{macrocode}
%  \end{macro}
%
%
%
%  \begin{macro}{\@enddocument@kernel@warnings}
%    
%    \begin{macrocode}
\def\@enddocument@kernel@warnings{%
   \ifdim \font@submax >\fontsubfuzz\relax
     \@font@warning{Size substitutions with differences\MessageBreak
                up to \font@submax\space have occurred.\@gobbletwo}%
   \fi
   \@defaultsubs
   \@refundefined
   \if@filesw
     \ifx \@multiplelabels \relax
       \if@tempswa
         \@latex@warning@no@line{Label(s) may have changed.
             Rerun to get cross-references right}%
       \fi
     \else
       \@multiplelabels
     \fi
   \fi
}
%    \end{macrocode}
%    
%    \begin{macrocode}
\AddToHook{enddocument/info}[kernel/filelist]{\@dofilelist}
\AddToHook{enddocument/info}[kernel/warnings]{\@enddocument@kernel@warnings}
\DeclareHookRule{enddocument/info}{kernel/filelist}{before}{kernel/warnings}
%    \end{macrocode}
%  \end{macro}
%
%
% \subsection{Adjusting at \pkg{atveryend} interfaces}
%
%    With the new hook management all of \pkg{atveryend} is taken care
%    of.
%
%    We therefore prevent the package from loading:
%    \begin{macrocode}
\expandafter\let\csname ver@atveryend.sty\endcsname\fmtversion
%    \end{macrocode}
%
%
%    Here are new definitions for its interfaces now pointing to the
%    hooks in \cs{enddocument}
%    \begin{macrocode}
\newcommand\AfterLastShipout  {\AddToHook{enddocument/afterlastpage}}
\newcommand\AtVeryEndDocument {\AddToHook{enddocument/afteraux}}
%    \end{macrocode}
%    Next one is a bit of a fake, but the result should normally be as
%    expected. If not one needs to add a rule to sort the code chunks
%    in \hook{enddocument/info}.
%    \begin{macrocode}
\newcommand\AtEndAfterFileList{\AddToHook{enddocument/info}}
%    \end{macrocode}
%    
%    \begin{macrocode}
\newcommand\AtVeryVeryEnd     {\AddToHook{enddocument/end}}
%    \end{macrocode}

%  \begin{macro}{\BeforeClearDocument}
%    This one is the only one we don't implement or rather don't have
%    a dedicated hook in the code.
%    \begin{macrocode}
\ExplSyntaxOn
\newcommand\BeforeClearDocument[1]
  { \AtEndDocument{#1}
    \@DEPRECATED{BeforeClearDocument \tl_to_str:n{#1}}
  }
\cs_new:Npn\@DEPRECATED #1
   {\iow_term:x{======~DEPRECATED~USAGE~#1~==========}}
%    \end{macrocode}
%    
%    \begin{macrocode}
\ExplSyntaxOff
%    \end{macrocode}
%  \end{macro}
%
%    \begin{macrocode}
%</2ekernel>
%    \end{macrocode}

%
% \section{A package version of the code for testing}
%

%    \begin{macrocode}
%<*package>
%    \end{macrocode}



%    \begin{macrocode}
\RequirePackage{xparse}
\ProvidesExplPackage{lthooks}{\lthooksdate}{\lthooksversion}
                    {Hook management interface for LaTeX2e}
%    \end{macrocode}
%
%
% \subsection{Core hook management code (kernel part)}
%
%    This should run in older formats so we can't use
%    \cs{IfFormatAtLeastTF} right now.
%    \begin{macrocode}
\@ifl@t@r\fmtversion{2020/10/01}
                    {}
                    {% \iffalse meta-comment
%
% Copyright (C)  2020-2021
%       Frank Mittelbach, Phelype Oleinik & LaTeX Team
%
% This file is part of the LaTeX base system.
% -------------------------------------------
%
% It may be distributed and/or modified under the
% conditions of the LaTeX Project Public License, either version 1.3c
% of this license or (at your option) any later version.
% The latest version of this license is in
%    https://www.latex-project.org/lppl.txt
% and version 1.3c or later is part of all distributions of LaTeX
% version 2008 or later.
%
% This file has the LPPL maintenance status "maintained".
%
% The list of all files belonging to the LaTeX base distribution is
% given in the file `manifest.txt'. See also `legal.txt' for additional
% information.
%
% The list of derived (unpacked) files belonging to the distribution
% and covered by LPPL is defined by the unpacking scripts (with
% extension .ins) which are part of the distribution.
%
% \fi
%
% \iffalse
%
%%% From File: lthooks.dtx
%
%    \begin{macrocode}
\def\lthooksversion{v1.0k}
\def\lthooksdate{2021/04/17}
%    \end{macrocode}
%
%<*driver>
\documentclass{l3doc}

% bug fix fo l3doc.cls
\ExplSyntaxOn
\cs_set_protected:Npn \__codedoc_macro_typeset_one:nN #1#2
  {
    \vbox_set:Nn \l__codedoc_macro_box
      {
        \vbox_unpack_drop:N \l__codedoc_macro_box
        \hbox { \llap { \__codedoc_print_macroname:nN {#1} #2
            \MacroFont       % <----- without it the \ is in lmr10 if a link is made
            \      
        } }
      }
    \int_incr:N \l__codedoc_macro_int
  }
\ExplSyntaxOff

\providecommand\InternalDetectionOff{}
\providecommand\InternalDetectionOn{}

\EnableCrossrefs
\CodelineIndex
\begin{document}
  \DocInput{lthooks.dtx}
\end{document}
%</driver>
%
% \fi
%
%
% \providecommand\hook[1]{\texttt{#1}}
%
% \providecommand\fmi[1]{\marginpar{\footnotesize FMi: #1}}
% \providecommand\fmiinline[1]{\begin{quote}\itshape\footnotesize FMi: #1\end{quote}}
% \providecommand\pho[1]{\marginpar{\footnotesize PhO: #1}}
% \providecommand\phoinline[1]{\begin{quote}\itshape\footnotesize PhO: #1\end{quote}}
%
%    
%
% \title{The \texttt{lthooks} package\thanks{This package has version
%    \lthooksversion\ dated \lthooksdate, \copyright\ \LaTeX\
%    Project.}}
%
% \author{Frank Mittelbach\thanks{Code improvements for speed and other goodies by Phelype Oleinik}}
%
% \maketitle
%
%
% \tableofcontents
%
% \section{Introduction}
%
%    Hooks are points in the code of commands or environments where it
%    is possible to add processing code into existing commands. This
%    can be done by different packages that do not know about each
%    other and to allow for hopefully safe processing it is necessary
%    to sort different chunks of code added by different packages into
%    a suitable processing order.
%
%    This is done by the packages adding chunks of code (via
%    \cs{AddToHook}) and labeling their code with some label by
%    default using the package name as a label.
%
%    At \verb=\begin{document}= all code for a hook is then sorted
%    according to some rules (given by \cs{DeclareHookRule}) for fast
%    execution without processing overhead. If the hook code is
%    modified afterwards (or the rules are changed),
%    a new version for fast processing is generated.
%
%    Some hooks are used already in the preamble of the document. If
%    that happens then the hook is prepared for execution (and sorted)
%    already at that point.
%
%
% \section{Package writer interface}
%
%    The hook management system is offered as a set of CamelCase
%    commands for traditional \LaTeXe{} packages (and for use in the
%    document preamble if needed) as well as \texttt{expl3} commands
%    for modern packages, that use the L3 programming layer of
%    \LaTeX{}. Behind the scenes, a single set of data structures is
%    accessed so that packages from both worlds can coexist and access
%    hooks in other packages.
%
%
%
% \subsection{\LaTeXe\ interfaces}
%
% \subsubsection{Declaring hooks}
%
%    With a few exceptions, hooks have to be declared before they can
%    be used. The exceptions are the generic hooks for commands,
%    environments (i.e., executed at \cs{begin} and \cs{end}) and
%    hooks run when loading files, e.g. before and after a package is
%    loaded, etc. Their hook names depend on the command,
%    environment or the
%    file name and so declaring them beforehand is not practical.
%
%
% \begin{function}{\NewHook}
%   \begin{syntax}
%     \cs{NewHook} \Arg{hook}
%   \end{syntax}
%   Creates a new \meta{hook}.
%    If this is a hook provided as part of a package it is suggested
%    that the \meta{hook} name is always structured as follows:
%    \meta{package-name}\texttt{/}\meta{hook-name}. If necessary you
%    can further subdivide the name by adding more \texttt{/} parts.
%    If a hook name is already taken, an error is raised and the hook
%    is not created.
%
%    The \meta{hook} can be specified using the dot-syntax to denote
%    the current package name. See section~\ref{sec:default-label}.
% \end{function}
%
% \begin{function}{\NewReversedHook}
%   \begin{syntax}
%     \cs{NewReversedHook} \Arg{hook}
%   \end{syntax}
%     Like \cs{NewHook} declares a new \meta{hook}.
%     the difference is that the code chunks for this hook are in
%     reverse order by default (those added last are executed first).
%     Any rules for the hook are applied after the default ordering.
%     See sections~\ref{sec:order} and \ref{sec:reversed-order}
%    for further details.
%
%    The \meta{hook} can be specified using the dot-syntax to denote
%    the current package name. See section~\ref{sec:default-label}.
% \end{function}
%
%
% \begin{function}{\NewMirroredHookPair}
%   \begin{syntax}
%     \cs{NewMirroredHookPair} \Arg{hook-1} \Arg{hook-2}
%   \end{syntax}
%     A shorthand for
%    \cs{NewHook}\Arg{hook-1}\cs{NewReversedHook}\Arg{hook-2}.
%
%    The \meta{hooks} can be specified using the dot-syntax to denote
%    the current package name. See section~\ref{sec:default-label}.
% \end{function}
%
% \begin{function}{\DisableHook}
%   \begin{syntax}
%     \cs{DisableHook} \Arg{hook}
%   \end{syntax}
%    After this declaration the \meta{hook} is no longer usable: Any
%    attempt to add further code to it will result in an error and any
%    use, e.g., via \cs{UseHook}, will simply do nothing.
%
%    This is intended to be used with generic command hooks (see
%    \texttt{ltcmdhooks-doc}) as depending on the definition of the
%    command such generic hooks may be unusable. If that is known, a
%    package developer can disable such hooks up front.
% \end{function}
%
%
% \subsubsection{Using hooks in code}
%
%
% \begin{function}{\UseHook}
%   \begin{syntax}
%     \cs{UseHook} \Arg{hook}
%   \end{syntax}
%    Execute the hook code inside a command or environment.
%
%    Before \verb=\begin{document}= the fast execution code for a hook
%    is not set up, so in order to use a hook there it is explicitly
%    initialized first. As that involves assignments using a hook at
%    those times is not 100\% the same as using it after
%    \verb=\begin{document}=.
%
%    The \meta{hook} \emph{cannot} be specified using the dot-syntax.
%    A leading |.| is treated literally.
% \end{function}
%
% \begin{function}{\UseOneTimeHook}
%   \begin{syntax}
%     \cs{UseOneTimeHook} \Arg{hook}
%   \end{syntax}
%    Some hooks are only used (and can be only used) in one place, for
%    example, those in \verb=\begin{document}= or
%    \verb=\end{document}=. Once we have passed that point adding to
%    the hook through a defined \cs{\meta{addto-cmd}} command (e.g.,
%    \cs{AddToHook} or \cs{AtBeginDocument}, etc.\@) would have no
%    effect (as would the use of such a command inside the hook code
%    itself). It is therefore customary to redefine
%    \cs{\meta{addto-cmd}} to simply  process its argument, i.e.,
%    essentially make it behave like \cs{@firstofone}.
%
%    \cs{UseOneTimeHook} does that: it records that the hook has been
%    consumed and any further attempt to add to it will result in
%    executing the code to be added immediately.
%
%    \fmiinline{Maybe add an error version as well?}
%
%    The \meta{hook} \emph{cannot} be specified using the dot-syntax.
%    A leading |.| is treated literally.
% \end{function}
%
%
% \subsubsection{Updating code for hooks}
%
% \begin{function}{\AddToHook}
%   \begin{syntax}
%     \cs{AddToHook} \Arg{hook}\oarg{label}\Arg{code}
%   \end{syntax}
%    Adds \meta{code} to the \meta{hook} labeled by \meta{label}.
%    When the optional argument \meta{label} is not provided, the
%    \meta{default label} is used (see section~\ref{sec:default-label}).
%    If \cs{AddToHook} is used in a package/class, the
%    \meta{default label} is the package/class name, otherwise it is
%    \hook{top-level} (the \hook{top-level} label is treated
%    differently:  see section~\ref{sec:top-level}).
%
%    If there already exists code under the \meta{label} then the new
%    \meta{code} is appended to the existing one (even if this is a reversed hook).
%    If you want to replace existing code under the
%    \meta{label}, first apply \cs{RemoveFromHook}.
%
%    The hook doesn't have to exist for code to be added to
%    it. However, if it is not declared, then obviously the
%    added \meta{code} will never be executed.  This
%    allows for hooks to work regardless of package loading order and
%    enables packages to add to hooks from other packages without
%    worrying whether they are actually used in the current document.
%    See section~\ref{sec:querying}.
%
%    The \meta{hook} and \meta{label} can be specified using the
%    dot-syntax to denote the current package name.
%    See section~\ref{sec:default-label}.
% \end{function}
%
% \begin{function}{\RemoveFromHook}
%   \begin{syntax}
%     \cs{RemoveFromHook} \Arg{hook}\oarg{label}
%   \end{syntax}
%    Removes any code labeled by \meta{label} from the \meta{hook}.
%    When the optional argument \meta{label} is not provided, the
%    \meta{default label} is used (see section~\ref{sec:default-label}).
%
%    If the code for that \meta{label} wasn't yet added to the
%    \meta{hook}, an order is set so that when some code attempts to add
%    that label, the removal order takes action and the code is not
%    added.
%
%    If the optional argument is \texttt{*}, then all code chunks are
%    removed. This is rather dangerous as it drops code from other
%    packages one may not know about!
%
%    The \meta{hook} and \meta{label} can be specified using the
%    dot-syntax to denote the current package name.
%    See section~\ref{sec:default-label}.
% \end{function}
%
% \medskip
%
% In contrast to the \texttt{voids} relationship between two labels
% in a \cs{DeclareHookRule} this is a destructive operation as the
% labeled code is removed from the hook data structure, whereas the
% relationship setting can be undone by providing a different
% relationship later.
%
% A useful application for this declaration inside the document body
% is when one wants to temporarily add code to hooks and later remove
% it again, e.g.,
%\begin{verbatim}
%   \AddToHook{env/quote/before}{\small}
%   \begin{quote}
%     A quote set in a smaller typeface
%   \end{quote}
%   ...
%   \RemoveFromHook{env/quote/before}
%   ... now back to normal for further quotes
%\end{verbatim}
% Note that you can't cancel the setting with
%\begin{verbatim}
%   \AddToHook{env/quote/before}{}
%\end{verbatim}
% because that only \enquote{adds} a further empty chunk of code to
% the hook. Adding \cs{normalsize} would work but that means the hook
% then contained \cs{small}\cs{normalsize} which means to font size
% changes for no good reason.
%
% The above is only needed if one wants to typeset several quotes in a
% smaller typeface. If the hook is only needed once then
% \cs{AddToHookNext} is simpler, because it resets itself after one use.
%
%
% \begin{function}{\AddToHookNext}
%   \begin{syntax}
%     \cs{AddToHookNext} \Arg{hook}\Arg{code}
%   \end{syntax}
%    Adds \meta{code} to the next invocation of the \meta{hook}.
%    The code is executed after the normal hook code has finished and
%    it is executed only once, i.e. it is deleted after it was used.
%
%    Using the declaration is a global operation, i.e., the code is
%    not lost, even if the declaration is used inside a group and the
%    next invocation happens after the group. If the declaration is
%    used several times before the hook is executed then all code is
%    executed in the order in which it was declared.\footnotemark
%
%    It is possible to nest declarations using the same hook (or
%    different hooks), e.g.,
%   \begin{quote}
%     \cs{AddToHookNext}\Arg{hook}\verb={=\meta{code-1}^^A
%     \cs{AddToHookNext}\Arg{hook}\Arg{code-2}\verb=}=
%   \end{quote}
%    will execute \meta{code-1} next time the \meta{hook} is used and at
%    that point puts \meta{code-2} into  the \meta{hook} so that it gets
%    executed on following time the hook is run.
%
%    A hook doesn't have to exist for code to be added to it.  This
%    allows for hooks to work regardless of package loading
%    order.
%    See section~\ref{sec:querying}.
%
%    The \meta{hook} can be specified using the dot-syntax to denote
%    the current package name.  See section~\ref{sec:default-label}.
% \end{function}\footnotetext{There is
%    no mechanism to reorder such code chunks (or delete them).}
%
% \subsubsection{Hook names and default labels}
% \label{sec:default-label}
%
% It is best practice to use \cs{AddToHook} in packages or classes
% \emph{without specifying a \meta{label}} because then the package
% or class name is automatically used, which is helpful if rules are
% needed, and avoids mistyping the \meta{label}.
%
% Using an explicit \meta{label} is only necessary in very specific
% situations, e.g., if you want to add several chunks of code into a
% single hook and have them placed in different parts of the hook
% (by providing some rules).
%
% The other case is when you develop a larger package with several
% sub-packages. In that case you may want to use the same
% \meta{label} throughout the sub-packages in order to avoid
% that the labels change if you internally reorganize your code.
%
% Except for \cs{UseHook}, \cs{UseOneTimeHook} and \cs{IfHookEmptyTF}
% (and their \pkg{expl3} interfaces \cs{hook_use:n},
% \cs{hook_use_once:n} and \cs{hook_if_empty:nTF}, all \meta{hook}
% and \meta{label} arguments are processed in the same way: first,
% spaces are trimmed around the argument, then it is fully expanded
% until only character tokens remain.  If the full expansion of the
% \meta{hook} or \meta{label} contains a non-expandable non-character
% token, a low-level \TeX{} error is raised (namely, the \meta{hook} is
% expanded using \TeX's \cs{csname}\ldots\cs{endcsname}, as such,
% Unicode characters are allowed in \meta{hook} and \meta{label}
% arguments).  The arguments of \cs{UseHook}, \cs{UseOneTimeHook},
% \cs{IfHookEmptyTF}, and \cs{IfHookExistsTF} are
% processed much in the same way except that spaces are not trimmed
% around the argument, for better performance.
%
% It is not enforced, but highly recommended that the hooks defined by
% a package, and the \meta{labels} used to add code to other hooks
% contain the package name to easily identify the source of the code
% chunk and to prevent clashes.  This should be the standard practice,
% so this hook management code provides a shortcut to refer to the
% current package in the name of a \meta{hook} and in a \meta{label}.
% If the \meta{hook} name or the \meta{label} consist just of a single dot
% (|.|), or starts with a dot followed by a slash (|./|) then the dot
% denotes the \meta{default label} (usually the current package or class
% name---see~\cs{SetDefaultHookLabel}).
% A \enquote{|.|} or \enquote{|./|} anywhere else in a \meta{hook} or in
% \meta{label} is treated literally and is not replaced.
%
% For example,
% inside the package \texttt{mypackage.sty}, the default label is
% \texttt{mypackage}, so the instructions:
% \begin{verbatim}
%   \NewHook   {./hook}
%   \AddToHook {./hook}[.]{code}     % Same as \AddToHook{./hook}{code}
%   \AddToHook {./hook}[./sub]{code}
%   \DeclareHookRule{begindocument}{.}{before}{babel}
%   \AddToHook {file/after/foo.tex}{code}
% \end{verbatim}
%    are equivalent to:
% \begin{verbatim}
%   \NewHook   {mypackage/hook}
%   \AddToHook {mypackage/hook}[mypackage]{code}
%   \AddToHook {mypackage/hook}[mypackage/sub]{code}
%   \DeclareHookRule{begindocument}{mypackage}{before}{babel}
%   \AddToHook {file/after/foo.tex}{code}                  % unchanged
% \end{verbatim}
%
% The \meta{default label} is automatically set equal to the name of the
% current package or class at the time the package is loaded.  If the
% hook command is used outside of a package, or the current file wasn't
% loaded with \cs{usepackage} or \cs{documentclass}, then the
% \texttt{top-level} is used as the \meta{default label}.  This may have
% exceptions---see \cs{PushDefaultHookLabel}.
%
% This syntax is available in all \meta{label} arguments and most
% \meta{hook} arguments, both in the \LaTeXe{} interface, and the \LaTeX3
% interface described in section~\ref{sec:l3hook-interface}.
%
% Note, however, that the replacement of |.| by the \meta{default label}
% takes place when the hook command is executed, so actions that are
% somehow executed after the package ends will have the wrong
% \meta{default label} if the dot-syntax is used.  For that reason,
% this syntax is not available in \cs{UseHook} (and \cs{hook_use:n})
% because the hook is most of the time used outside of the package file
% in which it was defined. This syntax is also not available in the hook
% conditionals \cs{IfHookEmptyTF} (and \cs{hook_if_empty:nTF}), because these
% conditionals are used in some performance-critical parts of the hook
% management code, and because they are usually used to refer to other
% package's hooks, so the dot-syntax doesn't make much sense.
%
% In some cases, for example in large packages, one may want to separate
% it in logical parts, but still use the main package name as
% \meta{label}, then the \meta{default label} can be set using
% \cs{SetDefaultHookLabel} or
% \cs{PushDefaultHookLabel}..\cs{PopDefaultHookLabel}.
%
% \begin{function}{\PushDefaultHookLabel,\PopDefaultHookLabel}
%   \begin{syntax}
%     \cs{PushDefaultHookLabel} \Arg{default label}
%     \quad \meta{code}
%     \cs{PopDefaultHookLabel}
%   \end{syntax}
%   \cs{PushDefaultHookLabel} sets the current \meta{default label} to
%   be used in \meta{label} arguments, or when replacing a leading
%   ``|.|'' (see above).  \cs{PopDefaultHookLabel} reverts the
%   \meta{default label} to its previous value.
%
%   Inside a package or class, the \meta{default label} is equal to the
%   package or class name, unless explicitly changed.  Everywhere else,
%   the \meta{default label} is |top-level| (see
%   section~\ref{sec:top-level}) unless explicitly changed.
%
%   The effect of \cs{PushDefaultHookLabel} holds until the next
%   \cs{PopDefaultHookLabel}.  \cs{usepackage} (and \cs{RequirePackage}
%   and \cs{documentclass}) internally use
%   \begin{quote}
%     \cs{PushDefaultHookLabel}\Arg{package name} \\
%     \null \quad  \meta{package code} \\
%     \cs{PopDefaultHookLabel}
%   \end{quote}
%   to set the \meta{default label} for the package or class file.
%   Inside the \meta{package code} the \meta{default label} can also be
%   changed with \cs{SetDefaultHookLabel}.  \cs{input} and other
%   file input-related commands from the \LaTeX{} kernel do not use
%   \cs{PushDefaultHookLabel}, so code within files loaded by these
%   commands does \emph{not} get a dedicated \meta{label}! (that is, the
%   \meta{default label} is the current active one when the file was
%   loaded.)
%
%   Packages that provide their own package-like interfaces
%   (Ti\textit{k}Z's \cs{usetikzlibrary}, for example) can use
%   \cs{PushDefaultHookLabel} and \cs{PopDefaultHookLabel} to set
%   dedicated labels and emulate \cs{usepackage}-like hook behaviour
%   within those contexts.
%
%   The |top-level| label is treated differently, and is reserved to the
%   user document, so it is not allowed to change the
%   \meta{default label} to |top-level|.
% \end{function}
%
% \begin{function}{\SetDefaultHookLabel}
%   \begin{syntax}
%     \cs{SetDefaultHookLabel} \Arg{default label}
%   \end{syntax}
%   Similarly to \cs{PushDefaultHookLabel},
%   sets the current \meta{default label} to
%   be used in \meta{label} arguments, or when replacing a leading
%   ``|.|''. The effect holds until the label is changed again or until
%   the next \cs{PopDefaultHookLabel}.  The difference between
%   \cs{PushDefaultHookLabel} and \cs{SetDefaultHookLabel} is that the
%   latter does not save the current \meta{default label}.
%
%   This command is useful when a large package is composed of several
%   smaller packages, but all should have the same \meta{label}, so
%   \cs{SetDefaultHookLabel} can be used at the beginning of each
%   package file to set the correct label.
%
%   \cs{SetDefaultHookLabel} is not allowed in the main document, where
%   the \meta{default label} is |top-level| and there is no
%   \cs{PopDefaultHookLabel} to end its effect.
%   It is also not allowed to change the \meta{default label} to
%   |top-level|.
% \end{function}
%
% \subsubsection{The \texttt{top-level} label}
% \label{sec:top-level}
%
% The |top-level| label, assigned to code added from the main document,
% is different from other labels.  Code added to hooks (usually
% \cs{AtBeginDocument}) in the preamble is almost always to change
% something defined by a package, so it should go at the very end of the
% hook.
%
% Therefore, code added in the |top-level| is always executed at the end
% of the hook, regardless of where it was declared.  If the hook is
% reversed (see \cs{NewReversedHook}), the |top-level| chunk is executed
% at the very beginning instead.
%
% Rules regarding |top-level| have no effect:  if a user wants to have a
% specific set of rules for a code chunk, they should use a different
% label to said code chunk, and provide a rule for that label instead.
%
% The |top-level| label is exclusive for the user, so trying to add code
% with that label from a package results in an error.
%
% \subsubsection{Defining relations between hook code}
%
% The default assumption is that code added to hooks by different
% packages are independent and the order in which they are executed is
% irrelevant. While this is true in many cases it is obviously false
% in others.
%
% Before the hook management system was introduced
% packages had to take elaborate precaution to determine of some other
% package got loaded as well (before or after) and find some ways to
% alter its behavior accordingly. In addition is was often the user's
% responsibility to load packages in the right order so that code
% added to hooks got added in the right order and some cases even
% altering the loading order wouldn't resolve the conflicts.
%
% With the new hook management system it is now possible to define
% rules (i.e., relationships) between code chunks added by different
% packages and explicitly describe in which order they should be
% processed.
%
% \begin{function}{\DeclareHookRule}
%   \begin{syntax}
%     \cs{DeclareHookRule} \Arg{hook}\Arg{label1}\Arg{relation}\Arg{label2}
%   \end{syntax}
%    Defines a relation between \meta{label1} and \meta{label2} for a
%    given \meta{hook}. If \meta{hook} is \texttt{??} this defines a default
%    relation for all hooks that use the two labels, i.e., that have
%    chunks of code labeled with \meta{label1} and \meta{label2}.
%    Rules specific to a given hook take precedence over default
%    rules that use \texttt{??} as the \meta{hook}.
%
%    Currently, the supported relations are the following:
%    \begin{itemize}
%
%    \item[\texttt{before} or \texttt{\string<}]
%
%      Code for \meta{label1} comes before code for \meta{label2}.
%
%    \item[\texttt{after} or \texttt{\string>}]
%      Code for \meta{label1} comes after code for \meta{label2}.
%
%    \item[\texttt{incompatible-warning}]
%
%      Only code for either \meta{label1} or \meta{label2} can appear
%      for that hook (a way to say that two packages---or parts of
%      them---are incompatible). A warning is raised if both labels
%      appear in the same hook.
%
%    \item[\texttt{incompatible-error}]
%
%      Like \texttt{incompatible-error} but instead of a warning a
%      \LaTeX{} error is raised, and the code for both labels are
%      dropped from that hook until the conflict is resolved.
%
%    \item[\texttt{voids}]
%
%      Code for \meta{label1} overwrites code for \meta{label2}. More
%      precisely, code for \meta{label2} is dropped for that
%      hook. This can be used, for example if one package is a
%      superset in functionality of another one and therefore wants to
%      undo code in some hook and replace it with its own version.
%
%    \item[\texttt{unrelated}]
%
%       The order of code for \meta{label1} and \meta{label2} is
%      irrelevant. This rule is there to undo an incorrect rule
%      specified earlier.
%
%    \end{itemize}
%    There can only be a single relation between two labels for a
%    given hook,
%    i.e., a later \cs{DeclareHookrule} overwrites any previous
%    declaration.
%
%    The \meta{hook} and \meta{label} can be specified using the
%    dot-syntax to denote the current package name.
%    See section~\ref{sec:default-label}.
%
% \end{function}
%
%
% \begin{function}{\ClearHookRule}
%   \begin{syntax}
%     \cs{ClearHookRule}\Arg{hook}\Arg{label1}\Arg{label2}
%   \end{syntax}
%    Syntactic sugar for saying that \meta{label1} and \meta{label2}
%    are unrelated for the given \meta{hook}.
% \end{function}
%
%
%
% \begin{function}{\DeclareDefaultHookRule}
%   \begin{syntax}
%     \cs{DeclareDefaultHookRule}\Arg{label1}\Arg{relation}\Arg{label2}
%   \end{syntax}
%   This sets up a relation between \meta{label1} and \meta{label2}
%    for all hooks unless overwritten by a specific rule for a hook.
%    Useful for cases where one package has a specific relation to
%    some other package, e.g., is \texttt{incompatible} or always
%    needs a special ordering \texttt{before} or \texttt{after}.
%    (Technically it is just a shorthand for using \cs{DeclareHookRule}
%    with \texttt{??} as the hook name.)
%
%    Declaring default rules is only supported in the document
%    preamble.\footnotemark{}
%
%    The \meta{label} can be specified using the dot-syntax to denote
%    the current package name. See section~\ref{sec:default-label}.
% \end{function}\footnotetext{Trying to do so, e.g., via
%    \cs{DeclareHookRule} with \texttt{??}  has bad side-effects and
%    is not supported (though not explicitly caught for performance
%    reasons).}
%
%
%
% \subsubsection{Querying hooks}
% \label{sec:querying}
%
% Simpler data types, like token lists, have three possible states; they
% can:
% \begin{itemize}
%   \item exist and be empty;
%   \item exist and be non-empty; and
%   \item not exist (in which case emptiness doesn't apply);
% \end{itemize}
% Hooks are a bit more complicated: they have four possible states.
% A hook may exist or not, and either way it may or may not be empty.
% This means that even a hook that doesn't exist may be non-empty.
%
% This seemingly strange state may happen when, for example, package~$A$
% defines hook \hook{A/foo}, and package $B$ adds some code to that
% hook.  However, a document may load package $B$ before package $A$, or
% may not load package $A$ at all.  In both cases some code is added to
% hook \hook{A/foo} without that hook being defined yet, thus that
% hook is said to be non-empty, whereas it doesn't exist.  Therefore,
% querying the existence of a hook doesn't imply its emptiness, neither
% does the other way around.
%
% A hook is said to be empty when no code was added to it, either to
% its permanent code pool, or to its ``next'' token list.  The hook
% doesn't need to be declared to have code added to its code pool.
% A hook is said to exist when it was declared with \cs{NewHook} or
% some variant thereof.  Generic \hook{file} and \hook{env} hooks are
% automatically declared when code is added to them.
%
% \begin{function}[EXP]{\IfHookEmptyTF}
%   \begin{syntax}
%     \cs{IfHookEmptyTF} \Arg{hook} \Arg{true code} \Arg{false code}
%   \end{syntax}
%   Tests if the \meta{hook} is empty (\emph{i.e.}, no code was added to
%   it using either \cs{AddToHook} or \cs{AddToHookNext}), and
%   branches to either \meta{true code} or \meta{false code} depending
%   on the result.
%
%    The \meta{hook} \emph{cannot} be specified using the dot-syntax.
%    A leading |.| is treated literally.
% \end{function}
%
% \begin{function}[EXP]{\IfHookExistsTF}
%   \begin{syntax}
%     \cs{IfHookExistsTF} \Arg{hook} \Arg{true code} \Arg{false code}
%   \end{syntax}
%    This conditional was deprecated (and will be removed soon) as it
%    is not a reliable way to gain reliable information about a hook.
%
%\fmiinline{further docu update?}
%
%   \itshape Original description: Tests if the \meta{hook} exists (if
%    it was created with either
%   \cs{NewHook}, \cs{NewReversedHook}, or \cs{NewMirroredHookPair}), and
%   branches to either \meta{true code} or \meta{false code} depending
%   on the result.
%
%   The existence of a hook usually doesn't mean much from the viewpoint
%   of code that tries to add/remove code from that hook, since package
%   loading order may vary, thus the creation of hooks is asynchronous
%   to adding and removing code from it, so this test should be used
%   sparingly.
%
%   Generic hooks are declared at the time code is added to them, so the
%   result of \cs{hook_if_exist:n} will change once code is added to
%   said hook (unless the hook was previously declared).
%
%    The \meta{hook} \emph{cannot} be specified using the dot-syntax.
%    A leading |.| is treated literally.
% \end{function}
%
%
% \subsubsection{Displaying hook code}
%
%    If one has to adjust the code execution in a hook using a hook
%    rule it is helpful to get some information about the code
%    associated with a hook, its current order and the existing rules.
%
% \begin{function}{\ShowHook,\LogHook}
%   \begin{syntax}
%     \cs{ShowHook} \Arg{hook}
%   \end{syntax}
%   Displays information about the \meta{hook} such as
%   \begin{itemize}
%   \item
%      the code chunks (and their labels) added to it,
%   \item
%      any rules set up to order them,
%   \item
%      the computed order in which the chunks are executed,
%   \item
%      any code executed on the next invocation only.
%   \end{itemize}
% \end{function}
%
%   \cs{LogHook} prints the information to the |.log| file, and
%   \cs{ShowHook} prints them to the terminal/command window and starts
%   \TeX's prompt (only in \cs{errorstopmode}) to wait for user action.
%
%    The \meta{hook} can be specified using the dot-syntax to denote
%    the current package name. See section~\ref{sec:default-label}.
%
%^^A   % Code for the listing below:
%^^A   \NewHook{example-hook}
%^^A   \AddToHook{example-hook}{[code from 'top-level']}
%^^A   \AddToHook{example-hook}[foo]{[code from package 'foo']}
%^^A   \AddToHook{example-hook}[bar]{[from package 'bar']}
%^^A   \AddToHook{example-hook}[baz]{[package 'baz' is here]}
%^^A   \AddToHookNext{example-hook}{[one-time code]}
%^^A   \DeclareHookRule{example-hook}{baz}{before}{foo}
%^^A   \DeclareDefaultHookRule{bar}{after}{baz}
%^^A   \ShowHook{example-hook}
%
%   \def\theFancyVerbLine{\textcolor[gray]{0.5}{%^^A
%     \sffamily\tiny\arabic{FancyVerbLine}}}
%
%   \bigskip
%   Suppose a hook \texttt{example-hook} whose output of
%   \cs{ShowHook}|{example-hook}| is:
%   \begin{verbatim}[numbers=left]
%   -> The hook 'example-hook':
%   > Code chunks:
%   >     foo -> [code from package 'foo']
%   >     bar -> [from package 'bar']
%   >     baz -> [package 'baz' is here]
%   > Document-level (top-level) code (executed last):
%   >     -> [code from 'top-level']
%   > Extra code for next invocation:
%   >     -> [one-time code]
%   > Rules:
%   >     foo|baz with relation >
%   >     baz|bar with default relation <
%   > Execution order (after applying rules):
%   >     baz, foo, bar.
%   \end{verbatim}
%
%   In the listing above, lines~3 to~5 show the three code chunks added
%   to the hook and their respective labels in the format
%   \begin{quote}
%   \quad \meta{label}\verb| -> |\meta{code}
%   \end{quote}
%
%   Line~7 shows the code chunk added by the user in the main document
%   (labeled |top-level|) in the format
%   \begin{quote}
%   \quad\verb|Document-level (top-level) code (executed |%^^A
%              \meta{first\texttt{\string|}last}\verb|):|\\
%   \quad\verb|    -> |\meta{\texttt{top-level} code}
%   \end{quote}
%   This code will be either the first or last code executed by the hook
%   (|last| if the hook is normal, |first| if it is reversed).  This
%   chunk is not affected by rules and does not take part in sorting.
%
%   Line~9 shows the code chunk for the next execution of the hook in
%   the format
%   \begin{quote}
%   \quad \verb|-> |\meta{next-code}
%   \end{quote}
%   This code will be used and disappear at the next
%   \verb|\UseHook{example-hook}|, in contrast to the chunks mentioned
%   earlier, which can only be removed from that hook by doing
%   \verb|\RemoveFromHook{|\meta{label}|}[example-hook]|.
%
%   Lines~11 and~12 show the rules declared that affect this hook in the
%   format
%   \begin{quote}
%   \quad \meta{label-1}\verb+|+\meta{label-2}| with |%^^A
%         \meta{\texttt{default}?}| relation |\meta{relation}
%   \end{quote}
%   which means that the \meta{relation} applies to \meta{label-1} and
%   \meta{label-2}, in that order, as detailed in \cs{DeclareHookRule}.
%   If the relation is \texttt{default} it means that that rule applies
%   to \meta{label-1} and \meta{label-2} in \emph{all} hooks, (unless
%   overridden by a non-default relation).
%
%   Finally, line~14 lists the labels in the hook after sorting;
%   that is, in the order they will be executed when the hook is used.
%
%
% \subsubsection{Debugging hook code}
%
% \begin{function}{\DebugHooksOn,\DebugHooksOff}
%   \begin{syntax}
%     \cs{DebugHooksOn}
%   \end{syntax}
%    Turn the debugging of hook code on or off. This displays changes
%    made to the hook data structures. The output is rather coarse and
%      not really intended for normal use.
% \end{function}
%
%
% \subsection{L3 programming layer (\texttt{expl3}) interfaces}
% \label{sec:l3hook-interface}
%
%
% This is a quick summary of the \LaTeX3 programming interfaces for
% use with packages written in \texttt{expl3}. In contrast to the
% \LaTeXe{} interfaces they always use mandatory arguments only, e.g.,
% you always have to specify the \meta{label} for a code chunk.  We
% therefore suggest to use the declarations discussed in the previous
% section even in \texttt{expl3} packages, but the choice is yours.
%
%
% \begin{function}
%   {\hook_new:n,\hook_new_reversed:n,\hook_new_pair:nn}
%   \begin{syntax}
%     \cs{hook_new:n} \Arg{hook}
%     \cs{hook_new_reversed:n} \Arg{hook}
%     \cs{hook_new_pair:nn} \Arg{hook-1} \Arg{hook-2}
%   \end{syntax}
%   Creates a new \meta{hook} with normal or reverse ordering of code
%    chunks. \cs{hook_new_pair:nn} creates a pair of such hooks with
%    \Arg{hook-2} being a reversed hook.
%    If a hook name is already taken, an error is raised and the hook
%    is not created.
%
%    The \meta{hook} can be specified using the dot-syntax to denote
%    the current package name. See section~\ref{sec:default-label}.
% \end{function}
%
%
%
% \begin{function}{\hook_use:n}
%   \begin{syntax}
%     \cs{hook_use:n} \Arg{hook}
%   \end{syntax}
%    Executes the \Arg{hook} code followed (if set up) by the code for next
%    invocation only, then empties that next invocation code.
%
%    The \meta{hook} \emph{cannot} be specified using the dot-syntax.
%    A leading |.| is treated literally.
% \end{function}
%
% \begin{function}{\hook_use_once:n}
%   \begin{syntax}
%     \cs{hook_use_once:n} \Arg{hook}
%   \end{syntax}
%     Changes the \Arg{hook} status so that from now on any addition to
%     the hook code is executed immediately. Then execute any
%     \Arg{hook} code already set up.
%
%    The \meta{hook} \emph{cannot} be specified using the dot-syntax.
%    A leading |.| is treated literally.
% \end{function}
%
% \begin{function}{\hook_gput_code:nnn}
%   \begin{syntax}
%     \cs{hook_gput_code:nnn} \Arg{hook} \Arg{label} \Arg{code}
%   \end{syntax}
%    Adds a chunk of \meta{code} to the \meta{hook} labeled
%    \meta{label}. If the label already exists the \meta{code} is
%    appended to the already existing code.
%
%    If code is added to an external \meta{hook} (of the kernel or
%    another package) then the convention is to use the package name
%    as the \meta{label} not some internal module name or some other
%    arbitrary string.
%
%    The \meta{hook} and \meta{label} can be specified using the
%    dot-syntax to denote the current package name.
%    See section~\ref{sec:default-label}.
% \end{function}
%
% \begin{function}
%   {\hook_gput_next_code:nn}
%   \begin{syntax}
%     \cs{hook_gput_next_code:nn} \Arg{hook} \Arg{code}
%   \end{syntax}
%    Adds a chunk of \meta{code} for use only in the next invocation of the
%    \meta{hook}. Once used it is gone.
%
%    This is simpler than \cs{hook_gput_code:nnn}, the code is simply
%    appended to the hook in the order of declaration at the very end,
%    i.e., after all standard code for the hook got executed.
%
%    Thus if one needs to undo what the standard does one has to do
%    that as part of \meta{code}.
%
%    The \meta{hook} can be specified using the dot-syntax to denote
%    the current package name. See section~\ref{sec:default-label}.
% \end{function}
%
%
%
%  \begin{function}{\hook_gremove_code:nn}
%   \begin{syntax}
%     \cs{hook_gremove_code:nn} \Arg{hook} \Arg{label}
%   \end{syntax}
%    Removes any code for \meta{hook} labeled \meta{label}.
%
%    If the code for that \meta{label} wasn't yet added to the
%    \meta{hook}, an order is set so that when some code attempts to add
%    that label, the removal order takes action and the code is not
%    added.
%
%    If the second argument is \texttt{*}, then all code chunks are
%    removed. This is rather dangerous as it drops code from other
%    packages one may not know about, so think twice before using
%    that!
%
%    The \meta{hook} and \meta{label} can be specified using the
%    dot-syntax to denote the current package name.
%    See section~\ref{sec:default-label}.
% \end{function}
%
%
%  \begin{function}{\hook_gset_rule:nnnn}
%   \begin{syntax}
%     \cs{hook_gset_rule:nnnn} \Arg{hook} \Arg{label1} \Arg{relation} \Arg{label2}
%   \end{syntax}
%    Relate \meta{label1} with \meta{label2} when used in \meta{hook}.
%    See \cs{DeclareHookRule} for the allowed \meta{relation}s.
%    If \meta{hook} is \texttt{??} a default rule is specified.
%
%    The \meta{hook} and \meta{label} can be specified using the
%    dot-syntax to denote the current package name.
%    See section~\ref{sec:default-label}.
%    The dot-syntax is parsed in both \meta{label} arguments, but it
%    usually makes sense to be used in only one of them.
% \end{function}
%
% \begin{function}[pTF]{\hook_if_empty:n}
%   \begin{syntax}
%     \cs{hook_if_empty:nTF} \Arg{hook} \Arg{true code} \Arg{false code}
%   \end{syntax}
%   Tests if the \meta{hook} is empty (\emph{i.e.}, no code was added to
%   it using either \cs{AddToHook} or \cs{AddToHookNext}), and
%   branches to either \meta{true code} or \meta{false code} depending
%   on the result.
%
%    The \meta{hook} \emph{cannot} be specified using the dot-syntax.
%    A leading |.| is treated literally.
% \end{function}
%
% \begin{function}[pTF]{\@@_if_declared:n}
%   \begin{syntax}
%     \cs{@@_if_declared:nTF} \Arg{hook} \Arg{true code} \Arg{false code}
%   \end{syntax}
%   Tests if the \meta{hook} exists (if it was created with either
%   \cs{NewHook}, \cs{NewReversedHook}, or \cs{NewMirroredHookPair}), and
%   branches to either \meta{true code} or \meta{false code} depending
%   on the result.
%
%   The existence of a hook usually doesn't mean much from the viewpoint
%   of code that tries to add/remove code from that hook, since package
%   loading order may vary, thus the creation of hooks is asynchronous
%   to adding and removing code from it, so this test should be used
%   sparingly.
%
%   Generic hooks are declared at the time code is added to them, so the
%   result of \cs{@@_if_declared:n} will change once code is added to
%   said hook (unless the hook was previously declared).
%
%    The \meta{hook} \emph{cannot} be specified using the dot-syntax.
%    A leading |.| is treated literally.
% \end{function}
%
% \begin{function}{\hook_show:n,\hook_log:n}
%   \begin{syntax}
%     \cs{hook_show:n} \Arg{hook}
%   \end{syntax}
%   Displays information about the \meta{hook} such as
%   \begin{itemize}
%   \item
%      the code chunks (and their labels) added to it,
%   \item
%      any rules set up to order them,
%   \item
%      the computed order in which the chunks are executed,
%   \item
%      any code executed on the next invocation only.
%   \end{itemize}
%
%   \cs{hook_log:n} prints the information to the |.log| file, and
%   \cs{hook_show:n} prints them to the terminal/command window and starts
%   \TeX's prompt (only if \cs{errorstopmode}) to wait for user action.
%
%    The \meta{hook} can be specified using the dot-syntax to denote
%    the current package name. See section~\ref{sec:default-label}.
% \end{function}
%
% \begin{function}{\hook_debug_on:,\hook_debug_off:}
%   \begin{syntax}
%     \cs{hook_debug_on:}
%   \end{syntax}
%    Turns the debugging of hook code on or off. This displays changes
%    to the hook data.
% \end{function}
%
%
%
% \subsection{On the order of hook code execution} \label{sec:order}
%
%    Chunks of code for a \meta{hook} under different labels are supposed
%    to be independent if there are no special rules set up that
%    define a relation between the chunks. This means that you can't
%    make assumptions about the order of execution!
%
%    Suppose you have the following declarations:
%\begin{verbatim}
%    \NewHook{myhook}
%    \AddToHook{myhook}[packageA]{\typeout{A}}
%    \AddToHook{myhook}[packageB]{\typeout{B}}
%    \AddToHook{myhook}[packageC]{\typeout{C}}
%\end{verbatim}
%    then executing the hook with \cs{UseHook} will produce the
%    typeout \texttt{A} \texttt{B} \texttt{C} in that order.  In other
%    words, the execution order is computed to be \texttt{packageA},
%    \texttt{packageB}, \texttt{packageC} which you can verify with
%    \cs{ShowHook}\texttt{\{myhook\}}:
%\begin{verbatim}
%    -> The hook 'myhook':
%    > Code chunks:
%    >     packageA -> \typeout {A}
%    >     packageB -> \typeout {B}
%    >     packageC -> \typeout {C}
%    > Document-level (top-level) code (executed last):
%    >     ---
%    > Extra code for next invocation:
%    >     ---
%    > Rules:
%    >     ---
%    > Execution order:
%    >     packageA, packageB, packageC.
%\end{verbatim}
%    The reason is that the code chunks are internally saved in a property list
%    and the initial order of such a property list is the order in
%    which key-value pairs got added. However, that is only true if
%    nothing other than adding happens!
%
%    Suppose, or example, you want to replace the code chunk for
%    \texttt{packageA}, e.g.,
%\begin{verbatim}
%    \RemoveFromHook{myhook}[packageA]
%    \AddToHook{myhook}[packageA]{\typeout{A alt}}
%\end{verbatim}
%    then your order becomes  \texttt{packageB},
%    \texttt{packageC}, \texttt{packageA} because the label got removed
%    from the property list and then re-added (at its end).
%
%    While that may not be too surprising,  the execution order is
%    also sometimes altered if you add a redundant rule, e.g. if you specify
%\begin{verbatim}
%    \DeclareHookRule{myhook}{packageA}{before}{packageB}
%\end{verbatim}
%    instead of the previous lines we get
%\begin{verbatim}
%    -> The hook 'myhook':
%    > Code chunks:
%    >     packageA -> \typeout {A}
%    >     packageB -> \typeout {B}
%    >     packageC -> \typeout {C}
%    > Document-level (top-level) code (executed last):
%    >     ---
%    > Extra code for next invocation:
%    >     ---
%    > Rules:
%    >     packageB|packageA with relation >
%    > Execution order (after applying rules):
%    >     packageA, packageC, packageB.
%\end{verbatim}
%    As you can see the code chunks are still in the same order, but
%    in the execution order for the labels \texttt{packageB} and
%    \texttt{packageC} have
%    swapped places.
%    The reason is that, with the rule there are two orders that
%    satisfy it, and the algorithm for sorting happened to pick a
%    different one compared to the case without rules (where it
%    doesn't run at all as there is nothing to resolve).
%    Incidentally, if we had instead specified the redundant rule
%\begin{verbatim}
%    \DeclareHookRule{myhook}{packageB}{before}{packageC}
%\end{verbatim}
%    the execution order would not have changed.
%
%    In summary: it is not possible to rely on the order of execution
%    unless there are rules that partially or fully define the order
%    (in which you can rely on them being fulfilled).
%
%
% \subsection{The use of \enquote{reversed} hooks} \label{sec:reversed-order}
%
%    You may have wondered why you can declare a \enquote{reversed} hook
%    with \cs{NewReversedHook} and what that does exactly.
%
%    In short: the execution order of a reversed hook (without any
%    rules!) is exactly reversed to the order you would have gotten for
%    a hook declared with \cs{NewHook}.
%
%    This is helpful if you have a pair of hooks where you expect to see
%    code added that involves grouping, e.g., starting an environment
%    in the first and closing that environment in the second hook.
%    To give a somewhat contrived example\footnote{there are simpler
%    ways to achieve the same effect.}, suppose there is a package
%    adding the following:
%\begin{verbatim}
%    \AddToHook{env/quote/before}[package-1]{\begin{itshape}}
%    \AddToHook{env/quote/after} [package-1]{\end{itshape}}
%\end{verbatim}
%    As a result, all quotes will be in italics.
%    Now suppose further that another |package-too| makes the quotes
%    also in blue and therefore adds:
%\begin{verbatim}
%    \usepackage{color}
%    \AddToHook{env/quote/before}[package-too]{\begin{color}{blue}}
%    \AddToHook{env/quote/after} [package-too]{\end{color}}
%\end{verbatim}
%    Now if the \hook{env/quote/after} hook would be a normal hook we
%    would get the same execution order in  both hooks, namely:
%\begin{verbatim}
%    package-1, package-too
%\end{verbatim}
%    (or vice versa) and as a result, would get:
%\begin{verbatim}
%    \begin{itshape}\begin{color}{blue} ...
%    \end{itshape}\end{color}
%\end{verbatim}
%   and an error message that \verb=\begin{color}= ended by
%    \verb=\end{itshape}=.
%    With \hook{env/quote/after} declared as a reversed hook the
%    execution order is reversed and so all environments are closed in
%    the correct sequence and \cs{ShowHook} would give us the
%    following output:
%\begin{verbatim}
%    -> The hook 'env/quote/after':
%    > Code chunks:
%    >     package-1 -> \end {itshape}
%    >     package-too -> \end {color}
%    > Document-level (top-level) code (executed first):
%    >     ---
%    > Extra code for next invocation:
%    >     ---
%    > Rules:
%    >     ---
%    > Execution order (after reversal):
%    >     package-too, package-1.
%\end{verbatim}
%
%    The reversal of the execution order happens before applying any
%    rules, so if you alter the order you will probably have to alter
%    it in both hooks, not just in one, but that depends on the use case.
%
%
%
%
% \subsection{Difference between \enquote{normal} and
%    \enquote{one-time} hooks}
% \label{sec:onetime-hooks}
%
%    When executing a hook a developer has the choice of using
%    either \cs{UseHook} or \cs{UseOneTimeHook} (or their \pkg{expl3}
%    equivalents \cs{hook_use:n} and \cs{hook_use_once:n}).
%    This choice affects how \cs{AddToHook} is handled after the hook
%    has been executed for the first time.
%
%    With normal hooks adding code via \cs{AddToHook} means that the
%    code chunk is added to the hook data structure and then used each time
%    \cs{UseHook} is called.
%
%    With one-time hooks it this is handled slightly differently:
%    After \cs{UseOneTimeHook} has been called, any further attempts to
%    add code to the hook via \cs{AddToHook} will simply execute the
%    \meta{code} immediately.
%
%    This has some consequences one needs to be aware of:
%    \begin{itemize}
%    \item
%
%      If \meta{code} is added to a normal hook after the hook was
%      executed and it is never executed again for one or the other
%      reason, then this new \meta{code} will never be executed.
%
%    \item
%
%      In contrast if that happens with a one-time hook the \meta{code} is
%      executed immediately.
%
%    \end{itemize}
%    In particular this means that construct such as
%\begin{quote}
%    \cs{AddToHook}\verb={myhook}=\\
%    \phantom{\cs{AddToHook}}\verb={= \meta{code-1}
%                                     \cs{AddToHook}\verb={myhook}=\Arg{code-2}
%                                     \meta{code-3} \verb=}=
%\end{quote}
%    works for one-time hooks\footnote{This is sometimes used with
%    \cs{AtBeginDocument} which is why it is supported.} (all three
%    code chunks are executed one after another), but it makes little
%    sense with a normal hook, because with a normal hook the first time
%    \verb=\UseHook{myhook}= is executed it would 
%    \begin{itemize}
%    \item
%       execute \meta{code-1},
%    \item
%       then execute \verb=\AddToHook{myhook}{code-2}= which adds the
%    code chunk \meta{code-2} to the hook for use on the next invocation,
%    \item
%       and finally execute \meta{code-3}.
%    \end{itemize}
%    The second time \cs{UseHook} is called it would execute the
%    above and in addition \meta{code-2} as that was added as a code
%    chunk to the hook in the meantime. So each time the hook is used
%    another copy of \meta{code-2} is added and so that code chunk
%    is executed $\meta{\# of invocations} -1$ times.
%
%
%
%
%
% \subsection{Private \LaTeX{} kernel hooks}
%
%    There are a few places where it is absolutely essential for
%    \LaTeX{} to function correctly that code is executed in a precisely
%    defined order. Even that could have been implemented with the
%    hook management (by adding various rules to ensure the
%    appropriate ordering with respect to other code added by
%    packages). However, this makes every document unnecessary
%    slow, because there has to be sorting even through the result is
%    predetermined. Furthermore it forces package writers to
%    unnecessarily add such rules if they add further code to the hook
%    (or break \LaTeX{}).
%
%    For that reason such code is not using the hook management, but
%    instead private kernel commands directly before or after a public
%    hook with the following naming
%    convention: \cs{@kernel@before@\meta{hookname}} or
%    \cs{@kernel@after@\meta{hookname}}. For example, in
%    \cs{enddocument} you find
%\begin{verbatim}
%   \UseHook{enddocument}%
%   \@kernel@after@enddocument
%\end{verbatim}
%    which means first the user/package-accessible \hook{enddocument}
%    hook is executed and then the internal kernel hook. As their name
%    indicates these kernel commands should not be altered by third-party
%    packages, so please refrain from that in the interest of
%    stability and instead use the public hook next to it.\footnote{As
%    with everything in \TeX{} there is not enforcement of this rule,
%    and by looking at the code it is easy to find out how the kernel
%    adds to them. The main reason of this section is therefore to say
%    \enquote{please don't do that, this is unconfigurable code!}}
%
%
%
% \subsection{Legacy \LaTeXe{} interfaces}
%
% \newcommand\onetimetext{%
%   This is a one-time hook, so after it is executed, all further
%   attempts to add code to it will execute such code immediately
%   (see section~\ref{sec:onetime-hooks}).}
%
%  \LaTeXe{} offered a small number of hooks together with commands to
%    add to them. They are listed here and are retained for backwards
%    compatibility.
%
%  With the new hook management several additional hooks have been added
%    to \LaTeX\ and more will follow. See the next section for what
%    is already available.
%
%
% \begin{function}{\AtBeginDocument}
%   \begin{syntax}
%     \cs{AtBeginDocument} \oarg{label} \Arg{code}
%   \end{syntax}
%   If used without the optional argument \meta{label}, it works essentially
%    like before, i.e., it is adding \meta{code} to the hook
%    \hook{begindocument} 
%    (which is executed inside \verb=\begin{document}=).
%    However, all code added this way is labeled with the label
%    \hook{top-level} (see section~\ref{sec:top-level})
%    if done outside of a package or class or with the
%    package/class name if called inside such a file
%    (see section~\ref{sec:default-label}).
%
%    This way one can add further code to the hook using
%    \cs{AddToHook} or \cs{AtBeginDocument} using a different label
%    and explicitly order the code chunks as necessary, e.g., run some
%    code before or after another package's code.  When using the
%    optional argument the call is equivalent to running
%    \cs{AddToHook} \texttt{\{begindocument\}} \oarg{label}
%    \Arg{code}.
%
%    \cs{AtBeginDocument} is a wrapper around the \hook{begindocument}
%    hook (see section~\ref{sec:begindocument-hooks}), which is a
%    one-time hook.  As such, after the \hook{begindocument} hook is
%    executed at \verb=\begin{document}= any attempt to add \meta{code}
%    to this hook with \cs{AtBeginDocument} or with \cs{AddToHook} will
%    cause that \meta{code} to execute immediately instead.
%    See section~\ref{sec:onetime-hooks} for more on one-time hooks.
%
%    For important packages with known order requirement we may over
%    time add rules to the kernel (or to those packages) so that they
%    work regardless of the loading-order in the document.
% \end{function}
%
% \begin{function}{\AtEndDocument}
%   \begin{syntax}
%     \cs{AtEndDocument} \oarg{label} \Arg{code}
%   \end{syntax}
%   Like \cs{AtBeginDocument} but for the \hook{enddocument} hook.
% \end{function}
%
%    There is also \cs{AtBeginDvi}  which is discussed in conjunction
%    with the shipout hooks.
%
%    \bigskip
%
%    The few hooks that existed previously in \LaTeXe{} used internally
%    commands such as \cs{@begindocumenthook} and packages sometimes
%    augmented them directly rather than working through
%    \cs{AtBeginDocument}. For that reason there is currently support
%    for this, that is, if the system detects that such an internal
%    legacy hook command contains code it adds it to the new hook
%    system under the label \texttt{legacy} so that it doesn't get
%    lost.
%
%    However, over time the remaining cases of direct usage need
%    updating because in one of the future release of \LaTeX{} we will
%    turn this legacy support off, as it does unnecessary slow down
%    the processing.
%
%
% \subsection{\LaTeXe{} commands and environments augmented by
%    hooks}
%
%  \emph{intro to be written}
%
% \subsubsection{Generic hooks for all environments}
%
%    Every environment \meta{env} has now four associated hooks coming
%    with it:
%    \begin{description}
%    \item[\hook{env/\meta{env}/before}]
%
%       This hook is executed as part of \cs{begin} as the very first
%       action, in particular prior to starting the environment group.
%       Its scope is therefore not restricted by the environment.
%
%    \item[\hook{env/\meta{env}/begin}]
%
%       This hook is executed as part of \cs{begin} directly in front
%       of the code specific to the environment start (e.g., the
%       second argument of \cs{newenvironment}).  Its scope is the
%       environment body.
%
%    \item[\hook{env/\meta{env}/end}]
%
%       This hook is executed as part of \cs{end} directly in front of the
%       code specific to the end of the environment (e.g., the third
%       argument of \cs{newenvironment}).
%
%    \item[\hook{env/\meta{env}/after}]
%
%       This hook is executed as part of \cs{end} after the
%       code specific to the environment end and after the environment
%       group has ended.
%       Its scope is therefore not restricted by the environment.
%
%       The hook is implemented as a reversed hook so if two packages
%       add code to \hook{env/\meta{env}/before} and to
%       \hook{env/\meta{env}/after} they can add surrounding
%       environments and the order of closing them happens in the
%       right sequence.
%
%    \end{description}
%    Generic environment hooks are never one-time hooks even with
%    environments that are supposed to appear only once in a
%    document.\footnote{Thus if one adds code to such hooks after the
%    environment has been processed, it will only be executed if the
%    environment appears again and if that doesn't happen the code
%    will never get executed.}  In contrast to other hooks there is
%    also no need to declare them using \cs{NewHook}.
%
%    The hooks are only executed if \cs{begin}\Arg{env} and
%    \cs{end}\Arg{env} is used. If the environment code is executed
%    via low-level calls to \cs{\meta{env}} and \cs{end\meta{env}}
%    (e.g., to avoid the environment grouping) they are not
%    available. If you want them available in code using this method,
%    you would need to add them yourself, i.e., write something like
%\begin{verbatim}
%  \UseHook{env/quote/before}\quote
%      ...
%  \endquote\UseHook{env/quote/after}
%\end{verbatim}
%    to add the outer hooks, etc.
%
%
% \begin{function}{\BeforeBeginEnvironment}
%   \begin{syntax}
%     \cs{BeforeBeginEnvironment} \oarg{label} \Arg{code}
%   \end{syntax}
%   This declaration adds to the \hook{env/\meta{env}/before} hook
%    using the \meta{label}.  If \meta{label} is not given, the
%    \meta{default label} is used (see section~\ref{sec:default-label}).
% \end{function}
%
% \begin{function}{\AtBeginEnvironment}
%   \begin{syntax}
%     \cs{AtBeginEnvironment} \oarg{label} \Arg{code}
%   \end{syntax}
%   Like \cs{BeforeBeginEnvironment} but adds to the \hook{env/\meta{env}/begin} hook.
% \end{function}
%
% \begin{function}{\AtEndEnvironment}
%   \begin{syntax}
%     \cs{AtEndEnvironment} \oarg{label} \Arg{code}
%   \end{syntax}
%   Like \cs{BeforeBeginEnvironment} but adds to the \hook{env/\meta{env}/end} hook.
% \end{function}
%
% \begin{function}{\AfterEndEnvironment}
%   \begin{syntax}
%     \cs{AfterEndEnvironment} \oarg{label} \Arg{code}
%   \end{syntax}
%   Like \cs{BeforeBeginEnvironment} but adds to the \hook{env/\meta{env}/after} hook.
% \end{function}
%
%
% \subsubsection{Generic hooks for commands}
%
%    Similar to environments there are now (at least in theory) two
%    generic hooks available for any \LaTeX{} command. These are
%    \begin{description}
%    \item[\hook{cmd/\meta{name}/before}]
%
%       This hook is executed at the very start of the command
%       execution.
%
%    \item[\hook{cmd/\meta{name}/after}]
%       This hook is executed at the very end of the command body.  It is
%       implemented as a reversed hook.
%    \end{description}
%    In practice there are restrictions and especially the
%    \hook{after} hook works only with a subset of commands. Details
%    on restrictions documented in
%    \texttt{ltcmdhooks-doc.pdf} or with code in
%    \texttt{ltcmdhooks-code.pdf}.
%
%
%
%
% \subsubsection{Generic hooks provided by file loading operations}
%
%    There are several hooks added to \LaTeX{}'s
%    process of loading file via its high-level interfaces such as
%    \cs{input}, \cs{include}, \cs{usepackage}, etc. These are documented in
%    \texttt{ltfilehook-doc.pdf} or with code in
%    \texttt{ltfilehook-code.pdf}.
%
%
%
% \subsubsection{Hooks provided by \cs{begin}\texttt{\{document\}}}
% \label{sec:begindocument-hooks}
%
%    Until 2020 \cs{begin}\texttt{\{document\}} offered exactly one
%    hook that one could add to using
%    \cs{AtBeginDocument}. Experiences over the years have shown that
%    this single hook in one place was not enough and as part of
%    adding the general hook management system a number of additional
%    hooks have been added at this point. The places for these hooks have
%    been chosen to provide the same support as offered by external
%    packages, such as \pkg{etoolbox} and others that augmented
%    \cs{document} to gain better control.
%
%    Supported are now the following hooks (all of them one-time hooks):
%    \begin{description}
%
%
%    \item[\hook{begindocument/before}]
%
%      This hook is executed at the very start of \cs{document}, one can
%      think of it as a hook for code at the end of the preamble
%      section and this is how it is used by \pkg{etoolbox}'s
%      \cs{AtEndPreamble}.
%
%      \onetimetext
%
%    \item[\hook{begindocument}]
%
%      This hook is added to when using \cs{AtBeginDocument} and it is
%      executed after the \texttt{.aux} file as be read in and most
%      initialization are done, so they can be altered and inspected by
%      the hook code. It is followed by a small number of further
%      initializations that shouldn't be altered and are therefore
%      coming later.
%
%      The hook should not be used to add material for typesetting as
%      we are still in \LaTeX's initialization phase and not in the
%      document body. If such material needs to be added to the document
%      body use the next hook instead.
%
%      \onetimetext
%
%    \item[\hook{begindocument/end}]
%
%      This hook is executed at the end of the \cs{document} code in
%      other words at the beginning of the document body. The only
%      command that follows it is \cs{ignorespaces}.
%
%      \onetimetext
%
%    \end{description}
%    The generic hooks executed by \cs{begin} also exist, i.e.,
%    \hook{env/document/before} and \hook{env/document/begin}, but
%    with this special environment it is better use the dedicated
%    one-time hooks above.
%
%
%
%
% \subsubsection{Hooks provided by \cs{end}\texttt{\{document\}}}
%
%    \LaTeXe{} always provided \cs{AtEndDocument} to add code to the
%    execution of \verb=\end{document}= just in front of the code that
%    is normally executed there. While this was a big improvement over
%    the situation in \LaTeX\,2.09 it was not flexible enough for a
%    number of use cases and so packages, such as \pkg{etoolbox},
%    \pkg{atveryend} and others patched \cs{enddocument} to add
%    additional points where code could be hooked into.
%
%    Patching using packages is always problematical as leads to
%    conflicts (code availability, ordering of patches, incompatible
%    patches, etc.).  For this reason a number of additional hooks
%    have been added to the \cs{enddocument} code to allow packages
%    to add code in various places in a controlled way without the
%    need for overwriting or patching the core code.
%
%    Supported are now the following hooks (all of them one-time hooks):
%    \begin{description}
%
%    \item[\hook{enddocument}]
%
%      The hook associated with \cs{AtEndDocument}. It is immediately
%      called at the beginning of \cs{enddocument}.
%
%      When this hook is executed there may be still unprocessed
%      material (e.g., floats on the deferlist) and the hook may add
%      further material to be typeset. After it, \cs{clearpage} is
%      called to ensure that all such material gets typeset. If there
%      is nothing waiting the \cs{clearpage} has no effect.
%
%      \onetimetext
%
%    \item[\hook{enddocument/afterlastpage}]
%
%      As the name indicates this hook should not receive code that
%      generates material for further pages. It is the right place to
%      do some final housekeeping and possibly write out some
%      information to the \texttt{.aux} file (which is still open at
%      this point to receive data). It is also the correct place to
%      set up any testing code to be run when the \texttt{.aux} file
%      is re-read in the next step.
%
%
%      After this hook has been executed the \texttt{.aux} file is
%      closed for writing and then read back in to do some tests
%      (e.g., looking for missing references or duplicated labels, etc.).
%
%      \onetimetext
%
%    \item[\hook{enddocument/afteraux}]
%
%      At this point, the \texttt{.aux} file has been reprocessed and so
%      this is a possible place for final checks and display of
%      information to the user. However, for the latter you might
%      prefer the next hook, so that your information is displayed after the
%      (possibly longish) list of files if that got requested via \cs{listfiles}.
%
%      \onetimetext
%
%    \item[\hook{enddocument/info}]
%
%      This hook is meant to receive code that write final information
%      messages to the terminal. It follows immediately after the
%      previous hook (so both could have been combined, but then
%      packages adding further code would always need to also supply
%      an explicit rule to specify where it should go.
%
%      This hook already contains some code added by the kernel (under
%      the labels \texttt{kernel/filelist} and
%      \texttt{kernel/warnings}), namely the list of files when
%      \cs{listfiles} has been used and the warnings for duplicate
%      labels, missing references, font substitutions etc.
%
%      \onetimetext
%
%    \item[\hook{enddocument/end}]
%
%      Finally, this hook is executed just in front of the final call
%      to \cs{@{}@end}.
%
%      \onetimetext % is it even possible to add code after this one?
%
%    \end{description}
%
%
%    There is also the hook \hook{shipout/lastpage}. This hook is
%    executed as part of the last \cs{shipout} in the document to
%    allow package to add final \cs{special}'s to that page. Where
%    this hook is executed in relation to those from the above list
%    can vary from document to document. Furthermore to determine correctly
%    which of the \cs{shipout}s is the last one, \LaTeX{} needs to be run
%    several times, so initially it might get executed on the wrong
%    page. See section~\ref{sec:shipout} for where to find the details.
%
%
%    It is in also possible to use the generic \hook{env/document/end}
%    hook which is executed by \cs{end}, i.e., just in front of the
%    first hook above. Note however that the other generic \cs{end}
%    environment hook, i.e., \hook{env/document/after} will never get
%    executed, because by that time \LaTeX{} has finished the document
%    processing.
%
%
%
%
% \subsubsection{Hooks provided \cs{shipout} operations}
% \label{sec:shipout}
%
%    There are several hooks and mechanisms added to \LaTeX{}'s
%    process of generating pages. These are documented in
%    \texttt{ltshipout-doc.pdf} or with code in
%    \texttt{ltshipout-code.pdf}.
%
%
%
% \subsubsection{Hooks provided in NFSS commands}
%
%    In languages that need to support for more than one script in
%    parallel (and thus several sets of fonts), e.g., Latin and
%    Japanese fonts, NFSS font commands, such as \cs{sffamily}, need
%    to switch both the Latin family to ``Sans Serif'' and in addition
%    alter a second set of fonts.
%
%    To support this several NFSS have hooks in which such support can
%    be added.
%    \begin{description}
%
%    \item[\hook{rmfamily}]
%
%      After \cs{rmfamily} has done its initial checks and prepared a
%      any font series update this hook is executed and only
%      afterwards \cs{selectfont}.
%
%    \item[\hook{sffamily}]
%
%      Like the \hook{rmfamily} hook but for the \cs{sffamily} command.
%
%    \item[\hook{ttfamily}]
%
%      Like the \hook{rmfamily} hook but for the \cs{ttfamily} command.
%
%    \item[\hook{normalfont}]
%
%      The \cs{normalfont} command resets font encoding family series
%      and shape to their document defaults. It then executes this
%      hook and finally calls \cs{selectfont}.
%
%    \item[\hook{expand@font@defaults}]
%
%      The internal \cs{expand@font@defaults} command expands and
%      saves the current defaults for the meta families (rm/sf/tt) and
%      the meta series (bf/md). If the NFSS machinery has been
%      augmented, e.g., for Chinese or Japanese fonts, then further
%      defaults may need to be set at this point. This can be done in
%      this hook which is executed at the end of this macro.
%
%    \item[\hook{bfseries/defaults}, \hook{bfseries}]
%
%      If the \cs{bfdefault} was explicitly changed by the user its
%      new value is used to set the bf series defaults for the meta
%      families (rm/sf/tt) when \cs{bfseries} is called. In the
%      \hook{bfseries/defaults} hook further adjustments can be made
%      in this case.  This hook is only executed if such a change is
%      detected. In contrast the \hook{bfseries} hook is always
%      executed just before \cs{selectfont} is called to change to the
%      new series.
%
%
%    \item[\hook{mdseries/defaults}, \hook{mdseries}]
%
%       These two hooks are like the previous ones but used in
%      \cs{mdseries} command.
%
%    \end{description}
%
%
% \StopEventually{\setlength\IndexMin{200pt}  \PrintIndex  }
%
%
% \section{The Implementation}
%
%
% \subsection{Loading further extensions}
%
%    \begin{macrocode}
%<@@=hook>
%    \end{macrocode}
%
% \changes{v1.0i}{2021/03/18}
%         {Use \cs{NewModuleRelease}.}
%    \begin{macrocode}
%<*2ekernel|latexrelease>
\ExplSyntaxOn
%<latexrelease>\NewModuleRelease{2020/10/01}{lthooks}
%<latexrelease>                 {The~hook~management~system}
%    \end{macrocode}
%
%  \subsection{Debugging}
%
%  \begin{macro}{\g_@@_debug_bool}
%    Holds the current debugging state.
%    \begin{macrocode}
\bool_new:N \g_@@_debug_bool
%    \end{macrocode}
%  \end{macro}
%
%  \begin{macro}{\hook_debug_on:,\hook_debug_off:}
%  \begin{macro}{\@@_debug:n}
%  \begin{macro}{\@@_debug_gset:}
%    Turns debugging on and off by redefining \cs{@@_debug:n}.
%    \begin{macrocode}
\cs_new_eq:NN \@@_debug:n \use_none:n
\cs_new_protected:Npn \hook_debug_on:
  {
    \bool_gset_true:N \g_@@_debug_bool
    \@@_debug_gset:
  }
\cs_new_protected:Npn \hook_debug_off:
  {
    \bool_gset_false:N \g_@@_debug_bool
    \@@_debug_gset:
  }
\cs_new_protected:Npn \@@_debug_gset:
  {
    \cs_gset_protected:Npx \@@_debug:n ##1
      { \bool_if:NT \g_@@_debug_bool {##1} }
  }
%    \end{macrocode}
%  \end{macro}
%  \end{macro}
%  \end{macro}
%
%
%
%  \subsection{Borrowing from internals of other kernel modules}
%
%
% \begin{macro}[EXP]{\@@_str_compare:nn}
%   Private copy of \cs{__str_if_eq:nn}
% \InternalDetectionOff  
%    \begin{macrocode}
\cs_new_eq:NN \@@_str_compare:nn \__str_if_eq:nn
%    \end{macrocode}
% \InternalDetectionOn
% \end{macro}
%
%  \subsection{Declarations}
%
%  \begin{macro}{\l_@@_tmpa_bool}
%    Scratch boolean used throughout the package.
%    \begin{macrocode}
\bool_new:N \l_@@_tmpa_bool
%    \end{macrocode}
%  \end{macro}
%
%  \begin{macro}{\l_@@_return_tl,\l_@@_tmpa_tl,\l_@@_tmpb_tl}
%    Scratch variables used throughout the package.
%    \begin{macrocode}
\tl_new:N \l_@@_return_tl
\tl_new:N \l_@@_tmpa_tl
\tl_new:N \l_@@_tmpb_tl
%    \end{macrocode}
%  \end{macro}
%
%  \begin{macro}{\g_@@_all_seq}
%    In a few places we need a list of all hook names ever defined so
%    we keep track if them in this sequence.
%    \begin{macrocode}
\seq_new:N \g_@@_all_seq
%    \end{macrocode}
%  \end{macro}
%
% \begin{macro}{\g_@@_removal_list_prop}
%   A token list to hold delayed removals.
%    \begin{macrocode}
\tl_new:N \g_@@_removal_list_tl
%    \end{macrocode}
% \end{macro}
%
% \begin{macro}{\l_@@_cur_hook_tl}
%   Stores the name of the hook currently being sorted.
%    \begin{macrocode}
\tl_new:N \l_@@_cur_hook_tl
%    \end{macrocode}
% \end{macro}
%
% \begin{macro}{\l_@@_work_prop}
%   A property list holding a copy of the
%   \cs[no-index]{g_@@_\meta{hook}_code_prop} of the hook being sorted
%   to work on, so that changes don't act destructively on the hook data
%   structure.
%    \begin{macrocode}
\prop_new:N \l_@@_work_prop
%    \end{macrocode}
% \end{macro}
%
%  \begin{macro}{\g_@@_execute_immediately_prop}
%    List of hooks that from no on should not longer receive code.
%    \begin{macrocode}
\prop_new:N \g_@@_execute_immediately_prop
%    \end{macrocode}
%  \end{macro}
%
%  \begin{macro}{\g_@@_used_prop}
%    All hooks that receive code (for use in debugging display).
%    \begin{macrocode}
\prop_new:N \g_@@_used_prop
%    \end{macrocode}
%  \end{macro}
%
% \begin{macro}{\g_@@_hook_curr_name_tl,\g_@@_name_stack_seq}
%   Default label used for hook commands, and a stack to keep track of
%   packages within packages.
%    \begin{macrocode}
\tl_new:N \g_@@_hook_curr_name_tl
\seq_new:N \g_@@_name_stack_seq
%    \end{macrocode}
% \end{macro}
%
% \begin{macro}{\@@_tmp:w}
%   Temporary macro for generic usage.
%    \begin{macrocode}
\cs_new_eq:NN \@@_tmp:w ?
%    \end{macrocode}
% \end{macro}
%
% \begin{macro}{\tl_gremove_once:Nx,\tl_show:x,\tl_log:x}
%   Some variants of \pkg{expl3} functions. \fmi{should be moved to expl3}
%    \begin{macrocode}
\cs_generate_variant:Nn \tl_gremove_once:Nn { Nx }
\cs_generate_variant:Nn \tl_show:n { x }
\cs_generate_variant:Nn \tl_log:n { x }
%    \end{macrocode}
% \end{macro}
%
% \begin{macro}{\s_@@_mark}
%   Scan mark used for delimited arguments.
%    \begin{macrocode}
\scan_new:N \s_@@_mark
%    \end{macrocode}
% \end{macro}
%
% \begin{macro}{\@@_tl_set:Nn,\@@_tl_set:Nx,
%               \@@_tl_set:cn,\@@_tl_set:cx}
%   Private copies of a few \pkg{expl3} functions.  \pkg{l3debug} will
%   only add debugging to the public names, not to these copies, so we
%   don't have to use \cs{debug_suspend:} and \cs{debug_resume:}
%   everywhere.
%
%   Functions like \cs{@@_tl_set:Nn} have to be redefined, rather than
%   copied because in \pkg{expl3} they use
%   \cs[no-index]{__kernel_tl_(g)set:Nx}, which is also patched by
%   \pkg{l3debug}.
%   \changes{v1.0h}{2021/01/07}{Manually define some \pkg{l3tl} commands
%     to work around \pkg{expl3} changes}
%    \begin{macrocode}
\cs_new_protected:Npn \@@_tl_set:Nn #1#2
  { \cs_set_nopar:Npx #1 { \__kernel_exp_not:w {#2} } }
\cs_new_protected:Npn \@@_tl_set:Nx #1#2
  { \cs_set_nopar:Npx #1 {#2} }
\cs_generate_variant:Nn \@@_tl_set:Nn { c }
\cs_generate_variant:Nn \@@_tl_set:Nx { c }
%    \end{macrocode}
% \end{macro}
%
% \begin{macro}{\@@_tl_gset:Nn,\@@_tl_gset:No,\@@_tl_gset:Nx,
%               \@@_tl_gset:cn,\@@_tl_gset:co,\@@_tl_gset:cx}
%   Same as above.
%    \begin{macrocode}
\cs_new_protected:Npn \@@_tl_gset:Nn #1#2
  { \cs_gset_nopar:Npx #1 { \__kernel_exp_not:w {#2} } }
\cs_new_protected:Npn \@@_tl_gset:No #1#2
  { \cs_gset_nopar:Npx #1 { \__kernel_exp_not:w \exp_after:wN {#2} } }
\cs_new_protected:Npn \@@_tl_gset:Nx #1#2
  { \cs_gset_nopar:Npx #1 {#2} }
\cs_generate_variant:Nn \@@_tl_gset:Nn { c }
\cs_generate_variant:Nn \@@_tl_gset:No { c }
\cs_generate_variant:Nn \@@_tl_gset:Nx { c }
%    \end{macrocode}
% \end{macro}
%
% \begin{macro}{\@@_tl_gput_right:Nn,\@@_tl_gput_right:No,\@@_tl_gput_right:cn}
%   Same as above.
%    \begin{macrocode}
\cs_new_protected:Npn \@@_tl_gput_right:Nn #1#2
  { \@@_tl_gset:Nx #1 { \__kernel_exp_not:w \exp_after:wN { #1 #2 } } }
\cs_generate_variant:Nn \@@_tl_gput_right:Nn { No, cn }
%    \end{macrocode}
% \end{macro}
%
% \begin{macro}{\@@_tl_gput_left:Nn,\@@_tl_gput_left:No}
%   Same as above.
%    \begin{macrocode}
\cs_new_protected:Npn \@@_tl_gput_left:Nn #1#2
  {
    \@@_tl_gset:Nx #1
      { \__kernel_exp_not:w {#2} \__kernel_exp_not:w \exp_after:wN {#1} }
  }
\cs_generate_variant:Nn \@@_tl_gput_left:Nn { No }
%    \end{macrocode}
% \end{macro}
%
% \begin{macro}{\@@_tl_gset_eq:NN}
%   Same as above.
%    \begin{macrocode}
\cs_new_eq:NN \@@_tl_gset_eq:NN \tl_gset_eq:NN
%    \end{macrocode}
% \end{macro}
%
% \begin{macro}{\@@_tl_gclear:N,\@@_tl_gclear:c}
%   Same as above.
%    \begin{macrocode}
\cs_new_protected:Npn \@@_tl_gclear:N #1
  { \@@_tl_gset_eq:NN #1 \c_empty_tl }
\cs_generate_variant:Nn \@@_tl_gclear:N { c }
%    \end{macrocode}
% \end{macro}
%
%
%
% \subsection{Providing new hooks}
%
% \subsubsection{The data structures of a hook}
%
% \begin{macro}{\g_@@_..._code_prop,\@@~...,\@@_next~...}
%
%    Hooks have a \meta{name} and for each hook we have to provide a number of
%    data structures. These are
%    \begin{description}
%    \item[\cs{g_@@_\meta{name}_code_prop}] A property list holding the code
%    for the hook in separate chunks. The keys are by default the
%    package names that add code to the hook, but it is possible
%    for packages to define other keys. 
%
%    \item[{\cs[no-index]{g_@@_\meta{name}_rule_\meta{label1}\string|\meta{label2}_tl}}]
%    A token list holding the relation between \meta{label1} and
%    \meta{label2} in the \meta{name}.  The \meta{labels} are lexically
%    (reverse) sorted to ensure that two labels always point to the same
%    token list.  For global rules, the \meta{name} is |??|.
%
%    \item[\cs{@@~\meta{name}}] The code that is actually executed
%    when the hook is called in the document is stored in this token
%    list. It is constructed from the code chunks applying the
%    information.
%    This token list is named like that so that in case of an error
%    inside the hook, the reported token list in the error is shorter,
%    and to make it simpler to normalize hook names in
%    \cs{@@_make_name:n}.
%
%    \item[\cs{g_@@_\meta{name}_reversed_tl}] Some hooks are
%    \enquote{reversed}.  This token list stores a |-| for such hook
%    so that it can be identified.  The |-| character is used because
%    $\meta{reversed}1$ is $+1$ for normal hooks and $-1$ for reversed
%    ones.
%
%    \item[\cs{@@_toplevel~\meta{name}}] This token list stores the code
%    inserted in the hook from the user's document, in the |top-level|
%    label.  This label is special, and doesn't participate in sorting.
%    Instead, all code is appended to it and executed after (or before,
%    if the hook is reversed) the normal
%    hook code, but before the |next| code chunk.
%
%    \item[\cs{@@_next~\meta{name}}] Finally there is extra code
%    (normally empty) that is used on the next invocation of the hook
%    (and then deleted). This can be used to define some special
%    behavior for a single occasion from within the document.  This token
%    list follows the same naming scheme than the main \cs{@@~\meta{name}}
%    token list.  It is called \cs{@@_next~\meta{name}} rather than
%    \cs{@@~next_\meta{name}} because otherwise a hook whose name is
%    |next_|\meta{name} would clash with the next code-token list of the
%    hook called \meta{name}.
%
%    \end{description}
%  \end{macro}
%
%
% \subsubsection{On the existence of hooks}
% \label{sec:existence}
%
%    A hook may be in different states of existence. Here we give an
%    overview of internal commands to set up hooks and explain how the
%    different states are distinguished. The actual implementation
%    then follows in the next sections.
%
% \fmi{reorder presentation of commands: new first}
%
% \begin{description}[format=\cs]
%   \setlist[itemize]{leftmargin=5cm,format=\cs}
%   \item[@@_declare:n] initializes the basic data structure for the
%     hook so that it may contain code.  It creates the hook code pool
%     (\cs[no-index]{g_@@_\meta{hook}_code_prop}) and the |top-level|
%     and |next| token lists.  A hook is implicitly declared with
%     \cs{@@_declare:n} when any is added to it.  A hook declared only
%     with \cs{@@_declare:n} will not be usable with \cs{hook_use:n}.
%     This state is detected by \cs{@@_if_exist:n} by
%     \cs[no-index]{g_@@_\meta{hook}_code_prop} being defined.
%     \begin{itemize}
%       \item [@@_if_exist:n]      returns |true|.
%       \item [@@_if_declared:nTF] returns |false|.
%       \item [@@_if_created:nTF]  returns |false|.
%       \item [@@_if_disabled:nTF] returns |false|.
%     \end{itemize}
%   \item[@@_create:n] initializes all hook data structures, as done
%     by \cs{hook_new:n}, except that doing \cs{hook_new:n} on that
%     hook will not result in an error.  This is done implicitly by
%     \cs{hook_gput_code:n} when adding code to a generic hook.
%     This state is detected by \cs{@@_if_declared:n} by
%     \cs[no-index]{@@~\meta{hook}} being defined.
%     \begin{itemize}
%       \item [@@_if_exist:n]      returns |true|.
%       \item [@@_if_declared:nTF] returns |true|.
%       \item [@@_if_created:nTF]  returns |false|.
%       \item [@@_if_disabled:nTF] returns |false|.
%     \end{itemize}
%   \item[hook_new:n] same as \cs{@@_create:n} except it also doesn't
%     allow the hook to be declared again with \cs{hook_new:n}.
%     This state is detected by \cs{@@_if_created:n} by
%     \cs[no-index]{g_@@_\meta{hook}_created_tl} being defined.
%     \begin{itemize}
%       \item [@@_if_exist:n]      returns |true|.
%       \item [@@_if_declared:nTF] returns |true|.
%       \item [@@_if_created:nTF]  returns |true|.
%       \item [@@_if_disabled:nTF] returns |false|.
%     \end{itemize}
% \end{description}
% A hook may additionally be disabled:
% \begin{description}[format=\cs]
%   \setlist[itemize]{leftmargin=5cm,format=\cs}
%   \item[hook_disable:n] This forces the creation of the
%     \cs[no-index]{g_@@_\meta{hook}_created_tl} so that the hook
%     errors when used with \cs{hook_new:n}, then it undefines
%     \cs[no-index]{@@~\meta{hook}} so that it may not be executed.
%     Since by using the normal hook declaration commands from above
%     we can't get this combination, if a hook has a
%     \cs[no-index]{g_@@_\meta{hook}_created_tl} and not a
%     \cs[no-index]{@@~\meta{hook}} then it is disabled.
%     \cs{@@_if_disabled:n} uses that to check if the hook is
%     disabled.
%     \begin{itemize}
%       \item [@@_if_exist:n] may return |true| or |false|.
%       \item [@@_if_declared:nTF] returns |false|.
%       \item [@@_if_created:nTF]  returns |true|.
%       \item [@@_if_disabled:nTF] returns |true|.
%     \end{itemize}
% \end{description}
%
%
%
% \subsubsection{Setting hooks up}
%
%
%  \begin{macro}{\hook_new:n,\@@_new:n,\@@_create:n}
%    The \cs{hook_new:n} declaration declare a new hook and expects
%    the hook \meta{name} as its argument, e.g.,
%    \hook{begindocument}.
%    \begin{macrocode}
\cs_new_protected:Npn \hook_new:n #1
  { \@@_normalize_hook_args:Nn \@@_new:n {#1} }
%    \end{macrocode}
%
%    \begin{macrocode}
\cs_new_protected:Npn \@@_new:n #1
  {
%    \end{macrocode}
%   We check if the hook was already \emph{explicitly} declared with
%   \cs{hook_new:n}, and if it already exists we complain, otherwise set
%   the \enquote{created} flag for the hook so that it errors next time
%   \cs{hook_new:n} is used.
%    \begin{macrocode}
    \@@_if_created:nTF {#1}
      { \msg_error:nnn { hooks } { exists } {#1} }
      {
        \tl_new:c { g_@@_#1_created_tl }
        \@@_create:n {#1}
      }
  }
%    \end{macrocode}
%
%    \begin{macrocode}
\cs_new_protected:Npn \@@_create:n #1
  {
%    \end{macrocode}
%   Now we check if the hook's data structure can be safely created
%   without \pkg{expl3} raising errors, then
%   we add the hook name to the list of all hooks and
%   allocate the necessary data structures for the new hook,
%   otherwise just do nothing.
%    \begin{macrocode}
    \tl_if_exist:cF { @@~#1 }
      {
        \seq_gput_right:Nn \g_@@_all_seq {#1}
%    \end{macrocode}
%    This is only used by the actual code of the current hook, so
%    declare it normally:
%    \begin{macrocode}
        \tl_new:c { @@~#1 }
%    \end{macrocode}
%    Now ensure that the base data structure for the hook exists:
%    \begin{macrocode}
        \@@_declare:n {#1}
%    \end{macrocode}
%    The \cs{g_@@_\meta{hook}_labels_clist} holds the sorted list of
%    labels (once it got sorted). This is used only for debugging.
%    \begin{macrocode}
        \clist_new:c { g_@@_#1_labels_clist }
%    \end{macrocode}
%    Some hooks should reverse the default order of code chunks. To
%    signal this we have a token list which is empty for normal hooks
%    and contains a \verb=-= for reversed hooks.
%    \begin{macrocode}
        \tl_new:c { g_@@_#1_reversed_tl }
%    \end{macrocode}
%    The above is all in L3 convention, but we also provide an
%    interface to legacy \LaTeXe{} hooks of the form \cs{@...hook},
%    e.g., \cs{@begindocumenthook}.
%    there have been a few of them and they have been added to
%    using \cs{g@addto@macro}. If there exists such a macro matching
%    the name of the new hook, i.e.,
%    \verb+\@+\meta{hook-name}\texttt{hook} and it is not empty then
%    we add its contents as a code chunk under the label \texttt{legacy}.
%    \begin{quote}
%       \textbf{Warning: this support will vanish in future releases!}
%    \end{quote}
%
%    \begin{macrocode}
        \@@_include_legacy_code_chunk:n {#1}
      }
  }
%    \end{macrocode}
%  \end{macro}
%
%
%
% \begin{macro}{\@@_declare:n}
%   This function declares the basic data structures for a hook without
%   actually declaring the hook itself.  This is needed to allow adding
%   to undeclared hooks.  Here it is unnecessary to check whether all
%   variables exist, since all three are declared at the same time
%   (either all of them exist, or none).
%    \begin{macrocode}
\cs_new_protected:Npn \@@_declare:n #1
  {
    \@@_if_exist:nF {#1}
      {
        \prop_new:c { g_@@_#1_code_prop }
        \tl_new:c { @@_toplevel~#1 }
        \tl_new:c { @@_next~#1 }
      }
  }
%    \end{macrocode}
%  \end{macro}
%
%
%
%  \begin{macro}{\hook_new_reversed:n,\@@_new_reversed:n}
%
%    Declare a new hook. The default ordering of code chunks is
%    reversed, signaled by setting the token list to a minus sign.
%    \begin{macrocode}
\cs_new_protected:Npn \hook_new_reversed:n #1
  { \@@_normalize_hook_args:Nn \@@_new_reversed:n {#1} }
\cs_new_protected:Npn \@@_new_reversed:n #1
  {
    \@@_new:n {#1}
%    \end{macrocode}
%    If the hook already exists the above will generate an error
%    message, so the next line should be executed (but it is --- too
%    bad).
%    \begin{macrocode}
    \tl_gset:cn { g_@@_#1_reversed_tl } { - }
  }
%    \end{macrocode}
%  \end{macro}
%
%  \begin{macro}{\hook_new_pair:nn}
%    A shorthand for declaring a normal and a (matching) reversed hook in one go.
%    \begin{macrocode}
\cs_new_protected:Npn \hook_new_pair:nn #1#2
  { \hook_new:n {#1} \hook_new_reversed:n {#2} }
%    \end{macrocode}
%  \end{macro}
%
%
% \begin{macro}{\@@_include_legacy_code_chunk:n}
%    The \LaTeX{} legacy concept for hooks uses with hooks the
%    following naming scheme in the code: \cs{@...hook}.
%
%    If this macro is not empty we add it under the label
%    \texttt{legacy} to the current hook and then empty it globally.
%    This way packages or classes directly manipulating commands such
%    as \cs{@begindocumenthook} still get their hook data added.
%    \begin{quote}
%       \textbf{Warning: this support will vanish in future releases!}
%    \end{quote}
%    \begin{macrocode}
\cs_new_protected:Npn \@@_include_legacy_code_chunk:n #1
  {
%    \end{macrocode}
%    If the macro doesn't exist (which is the usual case) then nothing
%    needs to be done.
%    \begin{macrocode}
    \tl_if_exist:cT { @#1hook }
%    \end{macrocode}
%    Of course if the legacy hook exists but is empty, there is no need
%    to add anything under \texttt{legacy} the legacy label.
%    \begin{macrocode}
      {
        \tl_if_empty:cF { @#1hook }
          {
            \exp_args:Nnnv \@@_hook_gput_code_do:nnn {#1}
                                  { legacy } { @#1hook }
%    \end{macrocode}
%    Once added to the hook, we need to clear it otherwise it might
%    get added again  later if the hook data gets updated.
%    \begin{macrocode}
            \@@_tl_gclear:c { @#1hook }
          }
      }
  }
%    \end{macrocode}
% \end{macro}
%
%
%
% \subsubsection{Disabling hooks}
%
% \begin{macro}{\hook_disable:n}
% \begin{macro}[pTF]{\@@_if_disabled:n}
%   Disables a hook by creating its
%   \cs[no-index]{g_@@_\meta{hook}_created_tl} and undefining its
%   \cs[no-index]{@@~\meta{hook}} token list.  This does not clear any
%   code that may be already stored in the hook's structure, but doesn't
%   allow adding more code.  \cs{@@_if_disabled:nTF} uses that specific
%   combination to check if the hook is disabled.
%    \begin{macrocode}
\cs_new_protected:Npn \hook_disable:n #1
  {
    \tl_gclear_new:c { g_@@_#1_created_tl }
    \cs_undefine:c { @@~#1 }
  }
\prg_new_conditional:Npnn \@@_if_disabled:n #1 { p, T, F, TF }
  {
    \bool_lazy_and:nnTF
        { \tl_if_exist_p:c { g_@@_#1_created_tl } }
        { ! \tl_if_exist_p:c { @@~#1 } }
      { \prg_return_true: }
      { \prg_return_false: }
  }
%    \end{macrocode}
% \end{macro}
% \end{macro}
%
%
% \subsection{Parsing a label}
%
% \begin{macro}[EXP]{\@@_parse_label_default:n}
%   This macro checks if a label was given (not \cs{c_novalue_tl}), and
%   if so, tries to parse the label looking for a leading \verb|.| to
%   replace by \cs{@@_currname_or_default:}.
%    \begin{macrocode}
\cs_new:Npn \@@_parse_label_default:n #1
  {
    \tl_if_novalue:nTF {#1}
      { \@@_currname_or_default: }
      { \tl_trim_spaces_apply:nN {#1} \@@_parse_dot_label:n }
  }
%    \end{macrocode}
% \end{macro}
%
% \begin{macro}[EXP]{\@@_parse_dot_label:n}
% \begin{macro}[EXP]{
%     \@@_parse_dot_label:w,
%     \@@_parse_dot_label_cleanup:w,
%     \@@_parse_dot_label_aux:w
%   }
%   Start by checking if the label is empty, which raises an error, and
%   uses the fallback value.  If not,
%   split the label at a \verb|./|, if any, and check if no tokens are
%   before the \verb|./|, or if the only character is a \verb|.|.
%   If these requirements are fulfilled, the leading
%   \verb|.| is replaced with \cs{@@_currname_or_default:}.  Otherwise
%   the label is returned unchanged.
%    \begin{macrocode}
\cs_new:Npn \@@_parse_dot_label:n #1
  {
    \tl_if_empty:nTF {#1}
      {
        \msg_expandable_error:nn { hooks } { empty-label }
        \@@_currname_or_default:
      }
      {
        \str_if_eq:nnTF {#1} { . }
          { \@@_currname_or_default: }
          { \@@_parse_dot_label:w #1 ./ \s_@@_mark }
      }
  }
\cs_new:Npn \@@_parse_dot_label:w #1 ./ #2 \s_@@_mark
  {
    \tl_if_empty:nTF {#1}
      { \@@_parse_dot_label_aux:w #2 \s_@@_mark }
      {
        \tl_if_empty:nTF {#2}
          { \@@_make_name:n {#1} }
          { \@@_parse_dot_label_cleanup:w #1 ./ #2 \s_@@_mark }
      }
  }
\cs_new:Npn \@@_parse_dot_label_cleanup:w #1 ./ \s_@@_mark {#1}
\cs_new:Npn \@@_parse_dot_label_aux:w #1 ./ \s_@@_mark
  { \@@_currname_or_default: / \@@_make_name:n {#1} }
%    \end{macrocode}
% \end{macro}
% \end{macro}
%
% \begin{macro}[EXP]{\@@_currname_or_default:}
%   Uses \cs{g_@@_hook_curr_name_tl} if it is set, otherwise tries
%   \cs{@currname}.  If neither is set, raises an error and uses the
%   fallback value \verb|label-missing|.
%    \begin{macrocode}
\cs_new:Npn \@@_currname_or_default:
  {
    \tl_if_empty:NTF \g_@@_hook_curr_name_tl
      {
        \tl_if_empty:NTF \@currname
          {
            \msg_expandable_error:nnn { hooks } { should-not-happen }
              { Empty~default~label. }
            \@@_make_name:n { label-missing }
          }
          { \@currname }
      }
      { \g_@@_hook_curr_name_tl }
  }
%    \end{macrocode}
% \end{macro}
%
% \begin{macro}[EXP]{\@@_make_name:n,\@@_make_name:w}
%   Provides a standard sanitization of a hook's name.
%   It uses \cs{cs:w} to build a control sequence out of the hook name,
%   then uses \cs{cs_to_str:N} to get the string representation of that,
%   without the escape character.  \cs{cs:w}-based expansion is used
%   instead of |e|-based because Unicode characters don't behave well
%   inside \cs{expanded}.  The macro adds the \cs{@@~} prefix to the
%   hook name to reuse the hook's code token list to build the csname
%   and avoid leaving \enquote{public} control sequences defined
%   (as~\cs{relax}) in TeX's memory.
%    \begin{macrocode}
\cs_new:Npn \@@_make_name:n #1
  {
    \exp_after:wN \exp_after:wN \exp_after:wN \@@_make_name:w
    \exp_after:wN \token_to_str:N \cs:w @@~ #1 \cs_end:
  }
\exp_last_unbraced:NNNNo
\cs_new:Npn \@@_make_name:w #1 \tl_to_str:n { @@~ } { }
%    \end{macrocode}
% \end{macro}
%
% \begin{macro}{\@@_normalize_hook_args:Nn}
% \begin{macro}{\@@_normalize_hook_args:Nnn}
% \begin{macro}{\@@_normalize_hook_rule_args:Nnnnn}
% \begin{macro}{\@@_normalize_hook_args_aux:Nn}
%   Standard route for normalising hook and label arguments.  The main
%   macro does the entire operation within a group so that csnames made
%   by \cs{@@_make_name:n} are wiped off before continuing.  This means
%   that this function cannot be used for \cs{hook_use:n}!
%    \begin{macrocode}
\cs_new_protected:Npn \@@_normalize_hook_args_aux:Nn #1 #2
  {
    \group_begin:
    \use:e
      {
        \group_end:
        \exp_not:N #1 #2
      }
  }
\cs_new_protected:Npn \@@_normalize_hook_args:Nn #1 #2
  {
    \@@_normalize_hook_args_aux:Nn #1
      { { \@@_parse_label_default:n {#2} } }
  }
\cs_new_protected:Npn \@@_normalize_hook_args:Nnn #1 #2 #3
  {
    \@@_normalize_hook_args_aux:Nn #1
      {
        { \@@_parse_label_default:n {#2} }
        { \@@_parse_label_default:n {#3} }
      }
  }
\cs_new_protected:Npn \@@_normalize_hook_rule_args:Nnnnn #1 #2 #3 #4 #5
  {
    \@@_normalize_hook_args_aux:Nn #1
      {
        { \@@_parse_label_default:n {#2} }
        { \@@_parse_label_default:n {#3} }
        { \tl_trim_spaces:n {#4} }
        { \@@_parse_label_default:n {#5} }
      }
  }
%    \end{macrocode}
% \end{macro}
% \end{macro}
% \end{macro}
% \end{macro}
%
%
% \begin{macro}{\hook_gput_code:nnn}
% \begin{macro}{\@@_gput_code:nnn,\@@_gput_code:nxv,\@@_hook_gput_code_do:nnn}
%
%    With \cs{hook_gput_code:nnn}\Arg{hook}\Arg{label}\Arg{code} a
%    chunk of \meta{code} is added to an existing \meta{hook} labeled
%    with \meta{label}.
%    \begin{macrocode}
\cs_new_protected:Npn \hook_gput_code:nnn #1 #2
  { \@@_normalize_hook_args:Nnn \@@_gput_code:nnn {#1} {#2} }
%    \end{macrocode}
%
%    \begin{macrocode}
\cs_new_protected:Npn \@@_gput_code:nnn #1 #2 #3
  {
%    \end{macrocode}
%    First check if the hook was used as a one-time hook:
%    \begin{macrocode}
    \prop_if_in:NnTF \g_@@_execute_immediately_prop {#1}
      {#3}
      {
%    \end{macrocode}
%    Then check if the current \meta{hook}/\meta{label} pair was marked
%    for removal, in which case \cs{@@_unmark_removal:nn} is used to
%    remove that mark (once).  This may happen when a package removes
%    code from another package which was not yet loaded:  the removal
%    order is stored, and at this stage it is executed by not adding to
%    the hook.
%    \begin{macrocode}
        \@@_if_marked_removal:nnTF {#1} {#2}
          { \@@_unmark_removal:nn {#1} {#2} }
          {
%    \end{macrocode}
%    If no removal is queued, we are free to add.  Start by checking if
%    the hook exists.
%    \begin{macrocode}
            \@@_if_declared:nTF {#1}
%    \end{macrocode}
%    If so we simply add (or append) the new code to the property list
%    holding different chunks for the hook. At \verb=\begin{document}=
%    this is then sorted into a token list for fast execution.
%    \begin{macrocode}
              {
                \@@_hook_gput_code_do:nnn {#1} {#2} {#3}
%    \end{macrocode}
%    However, if there is an update within the document we need to alter
%    this execution code which is done by
%    \cs{@@_update_hook_code:n}. In the preamble this does nothing.
%    \begin{macrocode}
                \@@_update_hook_code:n {#1}
              }
%    \end{macrocode}
%
%    If the hook does not exist, however, before giving up try to
%    declare it as a generic hook, if its name matches one of the valid
%    patterns.
%    \begin{macrocode}
              {
                \@@_if_disabled:nTF {#1}
                  { \msg_error:nnn { hooks } { hook-disabled } {#1} }
                  { \@@_try_declaring_generic_hook:nnn {#1} {#2} {#3} }
              }
          }
      }
  }
%    \end{macrocode}
%
%    \begin{macrocode}
\cs_generate_variant:Nn \@@_gput_code:nnn { nxv }
%    \end{macrocode}
%
%    This macro will unconditionally add a chunk of code to the given hook.
%    \begin{macrocode}
\cs_new_protected:Npn \@@_hook_gput_code_do:nnn #1 #2 #3
  {
%    \end{macrocode}
%    However, first some debugging info if debugging is enabled:
%    \begin{macrocode}
    \@@_debug:n{\iow_term:x{****~ Add~ to~
                      \@@_if_declared:nF {#1} { undeclared~ }
                      hook~ #1~ (#2)
                      \on@line\space <-~ \tl_to_str:n{#3}} }
%    \end{macrocode}
%    Then try to get the code chunk labeled \verb=#2= from the hook.
%    If there's code already there, then append \verb=#3= to that,
%    otherwise just put \verb=#3=.  If the current label is |top-level|,
%    the code is added to a dedicated token list
%    \cs[no-index]{@@_toplevel~\meta{hook}} that goes at the end of the
%    hook (or at the beginning, for a reversed hook), just before
%   \cs[no-index]{@@_next~\meta{hook}}.
%    \begin{macrocode}
    \str_if_eq:nnTF {#2} { top-level }
      {
        \str_if_eq:eeTF { top-level } { \@@_currname_or_default: }
          {
%    \end{macrocode}
%    If the hook's basic structure does not exist, we need to declare it
%    with \cs{@@_declare:n}.
%    \begin{macrocode}
            \@@_declare:n {#1}
            \@@_tl_gput_right:cn { @@_toplevel~#1 } {#3}
          }
          { \msg_error:nnn { hooks } { misused-top-level } {#1} }
      }
      {
        \prop_get:cnNTF { g_@@_#1_code_prop } {#2} \l_@@_return_tl
          {
            \prop_gput:cno { g_@@_#1_code_prop } {#2}
              { \l_@@_return_tl #3 }
          }
          { \prop_gput:cnn { g_@@_#1_code_prop } {#2} {#3} }
      }
  }
%    \end{macrocode}
% \end{macro}
% \end{macro}
%
% \begin{macro}{\@@_gput_undeclared_hook:nnn}
%   Often it may happen that a package $A$ defines a hook \verb=foo=,
%   but package $B$, that adds code to that hook, is loaded before $A$.
%   In such case we need to add code to the hook before its declared.
%    \begin{macrocode}
\cs_new_protected:Npn \@@_gput_undeclared_hook:nnn #1 #2 #3
  {
    \@@_declare:n {#1}
    \@@_hook_gput_code_do:nnn {#1} {#2} {#3}
  }
%    \end{macrocode}
% \end{macro}
%
% \begin{macro}{\@@_try_declaring_generic_hook:nnn}
% \begin{macro}{\@@_try_declaring_generic_next_hook:nn}
%   These entry-level macros just pass the arguments along to the
%   common \cs{@@_try_declaring_generic_hook:nNNnn} with the right
%   functions to execute when some action is to be taken.
%
%   The wrapper \cs{@@_try_declaring_generic_hook:nnn} then defers
%   \cs{hook_gput_code:nnn} if the generic hook was declared, or to
%   \cs{@@_gput_undeclared_hook:nnn} otherwise (the hook was tested for
%   existence before, so at this point if it isn't generic, it doesn't
%   exist).
%
%   The wrapper \cs{@@_try_declaring_generic_next_hook:nn} for
%   next-execution hooks does the same: it defers the code to
%   \cs{hook_gput_next_code:nn} if the generic hook was declared, or
%   to \cs{@@_gput_next_do:nn} otherwise.
%    \begin{macrocode}
\cs_new_protected:Npn \@@_try_declaring_generic_hook:nnn #1
  {
    \@@_try_declaring_generic_hook:nNNnn {#1}
      \hook_gput_code:nnn \@@_gput_undeclared_hook:nnn
  }
\cs_new_protected:Npn \@@_try_declaring_generic_next_hook:nn #1
  {
    \@@_try_declaring_generic_hook:nNNnn {#1}
      \hook_gput_next_code:nn \@@_gput_next_do:nn
  }
%    \end{macrocode}
%
% \begin{macro}{
%     \@@_try_declaring_generic_hook:nNNnn,
%     \@@_try_declaring_generic_hook_split:nNNnn
%   }
% \begin{macro}[TF]{\@@_try_declaring_generic_hook:wn}
%   \cs{@@_try_declaring_generic_hook:nNNnn} now splits the hook name
%   at the first \texttt{/} (if any) and first checks if it is a
%   file-specific hook (they require some normalization) using
%   \cs{@@_if_file_hook:wTF}. If not then check it is one of a
%   predefined set for generic names. We also split off the second
%   component to see if we have to make a reversed hook.  In either case
%   the function returns \meta{true} for a generic hook and \meta{false}
%   in other cases.
%    \begin{macrocode}
\cs_new_protected:Npn \@@_try_declaring_generic_hook:nNNnn #1
  {
    \@@_if_file_hook:wTF #1 / / \s_@@_mark
      {
        \exp_args:Ne \@@_try_declaring_generic_hook_split:nNNnn
          { \exp_args:Ne \@@_file_hook_normalize:n {#1} }
      }
      { \@@_try_declaring_generic_hook_split:nNNnn {#1} }
  }
%    \end{macrocode}
%
%    \begin{macrocode}
\cs_new_protected:Npn \@@_try_declaring_generic_hook_split:nNNnn #1 #2 #3
  {
    \@@_try_declaring_generic_hook:wnTF #1 / / / \scan_stop: {#1}
      { #2 }
      { #3 } {#1}
  }
%    \end{macrocode}
%
%    \begin{macrocode}
\prg_new_protected_conditional:Npnn \@@_try_declaring_generic_hook:wn
    #1 / #2 / #3 / #4 \scan_stop: #5 { TF }
  {
    \tl_if_empty:nTF {#2}
      { \prg_return_false: }
      {
        \prop_if_in:NnTF \c_@@_generics_prop {#1}
          {
            \@@_if_declared:nF {#5}
              {
%    \end{macrocode}
%    If the hook doesn't exist yet we check if it is a \texttt{cmd}
%    hook and if so we attempt patching the command in addition to
%    declaring the hook.
%
%    For some commands this will not be possible, in which case
%    \cs{@@_patch_cmd_or_delay:Nnn} will generate an appropriate error
%    message.
%    \begin{macrocode}
                \str_if_eq:nnT {#1} { cmd }
                  { \@@_try_put_cmd_hook:n {#5} }
%    \end{macrocode}
%
%    Declare the hook always even if it can't really be used (error
%    message generated elsewhere).
%
%    Here we use \cs{@@_create:n}, so that a \cs{hook_new:n} is still
%    possible later.
%    \begin{macrocode}
                \@@_create:n {#5}
              }
            \prop_if_in:NnTF \c_@@_generics_reversed_ii_prop {#2}
              { \tl_gset:cn { g_@@_#5_reversed_tl } { - } }
              {
                \prop_if_in:NnT \c_@@_generics_reversed_iii_prop {#3}
                  { \tl_gset:cn { g_@@_#5_reversed_tl } { - } }
              }
            \prg_return_true:
          }
          { \prg_return_false: }
      }
  }
%    \end{macrocode}
% \end{macro}
% \end{macro}
% \end{macro}
% \end{macro}
%
% \begin{macro}[pTF]{\@@_if_file_hook:w}
%   \cs{@@_if_file_hook:wTF} checks if the argument is a valid
%   file-specific hook (not, for example, |file/before|, but
%   |file/before/foo.tex|).  If it is a file-specific hook, then it
%   executes the \meta{true} branch, otherwise \meta{false}.
%
%   A file-specific hook is \texttt{file/\meta{position}/\meta{name}}.
%   If any of these parts don't exist, it is a general file hook or not
%   a file hook at all, so the conditional evaluates to \meta{false}.
%   Otherwise, it checks that the first part is |file| and that the
%   \meta{position} is in the \cs{c_@@_generics_file_prop}.
%
%   A property list is used here to avoid having to worry with catcodes,
%   because \pkg{expl3}'s file name parsing turns all characters into
%   catcode-12 tokens, which might differ from hand-input letters.
%    \begin{macrocode}
\prg_new_conditional:Npnn \@@_if_file_hook:w
    #1 / #2 / #3 \s_@@_mark { TF }
  {
    \str_if_eq:nnTF {#1} { file }
      {
        \bool_lazy_or:nnTF
            { \tl_if_empty_p:n {#3} }
            { \str_if_eq_p:nn {#3} { / } }
          { \prg_return_false: }
          {
            \prop_if_in:NnTF \c_@@_generics_file_prop {#2}
              { \prg_return_true: }
              { \prg_return_false: }
          }
      }
      { \prg_return_false: }
  }
%    \end{macrocode}
% \end{macro}
%
% \begin{macro}[EXP]{\@@_file_hook_normalize:n}
% \begin{macro}[EXP]{\@@_strip_double_slash:n,\@@_strip_double_slash:w}
%   When a file-specific hook is found, before being declared it is
%   lightly normalized by \cs{@@_file_hook_normalize:n}.  The current
%   implementation just replaces two consecutive slashes (|//|) by a
%   single one, to cope with simple cases where the user did something
%   like \verb|\def\input@path{{./mypath/}}|, in which case a hook would
%   have to be \verb|\AddToHook{file/after/./mypath//file.tex}|.
%    \begin{macrocode}
\cs_new:Npn \@@_file_hook_normalize:n #1
  { \@@_strip_double_slash:n {#1} }
\cs_new:Npn \@@_strip_double_slash:n #1
  { \@@_strip_double_slash:w #1 // \s_@@_mark }
%    \end{macrocode}
%   This function is always called after testing if the argument is a
%   file hook with \cs{@@_if_file_hook:wTF}, so we can assume it has
%   three parts (it is either \verb|file/before/...| or
%   \verb|file/after/...|), so we use \verb|#1/#2/#3 //| instead of just
%   \verb|#1 //| to prevent losing a slash if the file name is empty.
%   \changes{v1.0h}{2021/01/07}{Assume hook name has at least three
%     nonempty parts (gh/464)}
%    \begin{macrocode}
\cs_new:Npn \@@_strip_double_slash:w #1/#2/#3 // #4 \s_@@_mark
  {
    \tl_if_empty:nTF {#4}
      { #1/#2/#3 }
      { \@@_strip_double_slash:w #1/#2/#3 / #4 \s_@@_mark }
  }
%    \end{macrocode}
% \end{macro}
% \end{macro}
%
%  \begin{macro}{\c_@@_generics_prop}
%    Property list holding the generic names. We don't provide any user
%    interface to this as this is meant to be static.
%    \begin{description}
%    \item[\texttt{cmd}]
%      The generic hooks used for commands.
%    \item[\texttt{env}]
%      The generic hooks used in \cs{begin} and \cs{end}.
%    \item[\texttt{file}, \texttt{package}, \texttt{class}, \texttt{include}]
%      The generic hooks used when loading a file
%    \end{description}
%    \begin{macrocode}
\prop_const_from_keyval:Nn \c_@@_generics_prop
     {cmd=,env=,file=,package=,class=,include=}
%    \end{macrocode}
%  \end{macro}
%
%  \begin{macro}{\c_@@_generics_reversed_ii_prop,
%                \c_@@_generics_reversed_iii_prop,
%                \c_@@_generics_file_prop}
%    Some of the generic hooks are supposed to use reverse ordering, these are
%    the following (only the second or third sub-component is checked):
%    \begin{macrocode}
\prop_const_from_keyval:Nn \c_@@_generics_reversed_ii_prop {after=,end=}
\prop_const_from_keyval:Nn \c_@@_generics_reversed_iii_prop {after=}
\prop_const_from_keyval:Nn \c_@@_generics_file_prop {before=,after=}
%    \end{macrocode}
%  \end{macro}
%
% \begin{macro}{\hook_gremove_code:nn}
% \begin{macro}{\@@_gremove_code:nn}
%    
%    With \cs{hook_gremove_code:nn}\Arg{hook}\Arg{label} any code
%    for \meta{hook} stored under \meta{label} is removed.
%    \begin{macrocode}
\cs_new_protected:Npn \hook_gremove_code:nn #1 #2
  { \@@_normalize_hook_args:Nnn \@@_gremove_code:nn {#1} {#2} }
\cs_new_protected:Npn \@@_gremove_code:nn #1 #2
  {
%    \end{macrocode}
%    First check that the hook code pool exists.  \cs{@@_if_declared:nTF}
%    isn't used here because it should be possible to remove code from a
%    hook before its defined (see section~\ref{sec:querying}).
%    \begin{macrocode}
    \@@_if_exist:nTF {#1}
      {
%    \end{macrocode}
%    Then remove the chunk and run \cs{@@_update_hook_code:n} so
%    that the execution token list reflects the change if we are after
%    \verb=\begin{document}=.
%
%    If all code is to be removed, clear the code pool
%    \cs[no-index]{g_@@_\meta{hook}_code_prop}, the top-level code
%    \cs[no-index]{@@_toplevel~\meta{hook}}, and the next-execution code
%    \cs[no-index]{@@_next~\meta{hook}}.
%    \begin{macrocode}
        \str_if_eq:nnTF {#2} {*}
          {
            \prop_gclear:c { g_@@_#1_code_prop }
            \@@_tl_gclear:c { @@_toplevel~#1 }
            \@@_tl_gclear:c { @@_next~#1 }
          }
          {
%    \end{macrocode}
%    If the label is |top-level| then clear the token list, as all code
%    there is under the same label.  Marked removal is not implemented
%    for |top-level| because it is hard to reliably know that no code
%    was added to \cs[no-index]{@@_toplevel~\meta{hook}} (granted that
%    an empty code could be interpreted as that, but then it differs in
%    behaviour from other labels, in which an empty chunk is still valid
%    for removal).  Besides, it doesn't make much (if any) sense for
%    packages to remove |top-level| code.  So here the chunk is just
%    cleared unconditionally.
%    \begin{macrocode}
            \str_if_eq:nnTF {#2} { top-level }
              { \@@_tl_gclear:c { @@_toplevel~#1 } }
              {
%    \end{macrocode}
%    Otherwise check if the label being removed exists in the code pool.
%    If it does, just call \cs{@@_gremove_code_do:nn} to do the removal,
%    otherwise mark it to be removed.
%    \begin{macrocode}
                \prop_get:cnNTF { g_@@_#1_code_prop } {#2} \l_@@_return_tl
                  { \@@_gremove_code_do:nn }
                  { \@@_mark_removal:nn }
                      {#1} {#2}
              }
          }
%    \end{macrocode}
%    Finally update the code, if the hook exists.
%    \begin{macrocode}
        \@@_if_declared:nT {#1}
          { \@@_update_hook_code:n {#1} }
      }
%    \end{macrocode}
%
%    If the code pool for this hook doesn't exist it means that nothing
%    tried to add to it before, so we just queue this removal order for
%    later.
%    \begin{macrocode}
      { \@@_mark_removal:nn {#1} {#2} }
  }
%    \end{macrocode}
%
% \begin{macro}{\@@_gremove_code_do:nn}
%   Remove code for a given label.
%    \begin{macrocode}
\cs_new_protected:Npn \@@_gremove_code_do:nn #1 #2
  { \prop_gremove:cn { g_@@_#1_code_prop } {#2} }
%    \end{macrocode}
% \end{macro}
% \end{macro}
% \end{macro}
%
% \begin{macro}{\@@_mark_removal:nn}
%   Marks \meta{label} (\verb=#2=) to be removed from \meta{hook}
%   (\verb=#1=).  The number of removals should be fairly small, and
%   \cs{tl_gremove_once:Nx} is fairly efficient even for longer token
%   lists, so we use a single global token list, rather than one for
%   each hook.
%
%   A hand-crafted token list is used here because property lists don't
%   hold repeated items, so multiple usages of \cs{@@_mark_removal:nn}
%   would be cancelled by a single \cs{@@_unmark_removal:nn}.
%    \begin{macrocode}
\cs_new_protected:Npn \@@_mark_removal:nn #1 #2
  {
    \tl_gput_right:Nx \g_@@_removal_list_tl
      { \@@_removal_tl:nn {#1} {#2} }
  }
%    \end{macrocode}
% \end{macro}
%
% \begin{macro}{\@@_unmark_removal:nn}
%   Unmarks \meta{label} (\verb=#2=) to be removed from \meta{hook}
%   (\verb=#1=).  \cs{tl_gremove_once:Nx} is used rather than
%   \cs{tl_gremove_all:Nx} so that two additions are needed to cancel
%   two marked removals, rather than only one.
%    \begin{macrocode}
\cs_new_protected:Npn \@@_unmark_removal:nn #1 #2
  {
    \tl_gremove_once:Nx \g_@@_removal_list_tl
      { \@@_removal_tl:nn {#1} {#2} }
  }
%    \end{macrocode}
% \end{macro}
%
% \begin{macro}[TF]{\@@_if_marked_removal:nn}
%   Checks if the \cs{g_@@_removal_list_tl} contains the current
%   \meta{label} (\verb=#2=) and \meta{hook} (\verb=#1=).
%    \begin{macrocode}
\prg_new_protected_conditional:Npnn \@@_if_marked_removal:nn #1 #2 { TF }
  {
    \exp_args:NNx \tl_if_in:NnTF \g_@@_removal_list_tl
      { \@@_removal_tl:nn {#1} {#2} }
      { \prg_return_true: } { \prg_return_false: }
  }
%    \end{macrocode}
% \end{macro}
%
% \begin{macro}[rEXP]{\@@_removal_tl:nn}
%   Builds a token list with \verb=#1= and \verb=#2= which can only be
%   matched by \verb=#1= and \verb=#2=.  The |&|$_4$ anchors a removal,
%   so that \verb=#1= can't be mistaken by \verb=#2= and vice versa, and
%   the two |$|$_3$ delimit the two arguments
%    \begin{macrocode}
\cs_new:Npn \@@_removal_tl:nn #1 #2
  { & \tl_to_str:n {#2} $ \tl_to_str:n {#1} $ }
%    \end{macrocode}
% \end{macro}
%
%
% \begin{macro}{
%     \g_@@_??_code_prop,
%     \@@~??,
%     \g_@@_??_reversed_tl,
%   }
%
%    Initially these variables simply used an empty ``label'' name (not
%    two question marks). This was a bit unfortunate, because then
%    \texttt{l3doc} complains about \verb=__= in the middle of a
%    command name when trying to typeset the documentation. However
%    using a ``normal'' name such as \texttt{default} has the
%    disadvantage of that being not really distinguishable from a real
%    hook name. I now have settled for \texttt{??} which needs some
%    gymnastics to get it into the csname, but since this is used a
%    lot things should be fast, so this is not done with \texttt{c}
%    expansion in the code later on.
%
%    \cs{@@~??} isn't used, but it has to be defined to trick
%    the code into thinking that \verb=??= is actually a hook.
%    \begin{macrocode}
\prop_new:c {g_@@_??_code_prop}
\prop_new:c {@@~??}
%    \end{macrocode}
%
%    Default rules are always given in normal ordering (never in
%    reversed ordering). If such a rule is applied to a reversed
%    hook it behaves as if the rule is reversed (e.g.,
%    \texttt{after} becomes \texttt{before})
%    because those rules are applied first and then the order is reversed.
%    \begin{macrocode}
\tl_new:c {g_@@_??_reversed_tl}
%    \end{macrocode}
%  \end{macro}
%
%  \subsection{Setting rules for hooks code}
%
%  \begin{macro}{\hook_gset_rule:nnnn}
%  \begin{macro}{\@@_gset_rule:nnnn}
%
%    \fmi{needs docu correction given new implementation}
%
%    With
%    \cs{hook_gset_rule:nnnn}\Arg{hook}\Arg{label1}\Arg{relation}\Arg{label2}
%    a relation is defined between the two code labels for the given
%    \meta{hook}.  The special hook \texttt{??} stands for \emph{any}
%    hook describing a default rule.
%    \begin{macrocode}
\cs_new_protected:Npn \hook_gset_rule:nnnn #1#2#3#4
  {
    \@@_normalize_hook_rule_args:Nnnnn \@@_gset_rule:nnnn
      {#1} {#2} {#3} {#4}
  }
%    \end{macrocode}
%    
%    \begin{macrocode}
\cs_new_protected:Npn \@@_gset_rule:nnnn #1#2#3#4
  {
%    \end{macrocode}
%    First we ensure the basic data structure of the hook exists:
%    \begin{macrocode}
    \@@_declare:n {#1}
%    \end{macrocode}
%    Then we clear any previous relationship between both labels.
%    \begin{macrocode}
    \@@_rule_gclear:nnn {#1} {#2} {#4}
%    \end{macrocode}
%    Then we call the function to handle the given rule. Throw an error if the
%    rule is invalid.
%    \begin{macrocode}
    \cs_if_exist_use:cTF { @@_rule_#3_gset:nnn }
      {
          {#1} {#2} {#4}
        \@@_update_hook_code:n {#1}
      }
      { \msg_error:nnnnnn { hooks } { unknown-rule }
                          {#1} {#2} {#3} {#4}        }
  }
%    \end{macrocode}
% \end{macro}
% \end{macro}
%
% \begin{macro}{\@@_rule_before_gset:nnn, \@@_rule_after_gset:nnn,
%               \@@_rule_<_gset:nnn, \@@_rule_>_gset:nnn}
%    Then we add the new rule.  We need to normalize the rules here to
%    allow for faster processing later.  Given a pair of labels
%    $l_A$ and $l_B$, the rule $l_A>l_B$ is the same as $l_B<l_A$
%    only presented differently.  But by normalizing the
%    forms of the rule to a single representation, say, $l_B<l_A$, reduces
%    the time spent looking for the rules later considerably.
%
%    Here we do that normalization by using \cs[no-index]{(pdf)strcmp} to
%    lexically sort labels $l_A$ and $l_B$ to a fixed order.  This order
%    is then enforced every time these two labels are used together.
%
%    Here we use \cs{@@_label_pair:nn}~\Arg{hook}~\Arg{l_A}~\Arg{l_B}
%    to build a string \texttt{$l_B$\string|$l_A$} with a fixed order, and
%    use \cs{@@_label_ordered:nnTF} to apply the correct rule to the pair
%    of labels, depending if it was sorted or not.
%    \begin{macrocode}
\cs_new_protected:Npn \@@_rule_before_gset:nnn #1#2#3
  {
    \@@_tl_gset:cx { g_@@_#1_rule_ \@@_label_pair:nn {#2} {#3} _tl }
      { \@@_label_ordered:nnTF {#2} {#3} { < } { > } }
  }
\cs_new_eq:cN { @@_rule_<_gset:nnn } \@@_rule_before_gset:nnn
%    \end{macrocode}
%
%    \begin{macrocode}
\cs_new_protected:Npn \@@_rule_after_gset:nnn #1#2#3
  {
    \@@_tl_gset:cx { g_@@_#1_rule_ \@@_label_pair:nn {#3} {#2} _tl }
      { \@@_label_ordered:nnTF {#3} {#2} { < } { > } }
  }
\cs_new_eq:cN { @@_rule_>_gset:nnn } \@@_rule_after_gset:nnn
%    \end{macrocode}
%  \end{macro}
%
% \begin{macro}{\@@_rule_voids_gset:nnn}
%   This rule removes (clears, actually) the code from label |#3| if
%   label |#2| is in the hook |#1|.
%    \begin{macrocode}
\cs_new_protected:Npn \@@_rule_voids_gset:nnn #1#2#3
  {
    \@@_tl_gset:cx { g_@@_#1_rule_ \@@_label_pair:nn {#2} {#3} _tl }
      { \@@_label_ordered:nnTF {#2} {#3} { -> } { <- } }
  }
%    \end{macrocode}
%  \end{macro}
%
% \begin{macro}{
%     \@@_rule_incompatible-error_gset:nnn,
%     \@@_rule_incompatible-warning_gset:nnn,
%   }
%   These relations make an error/warning if labels |#2| and |#3| appear
%   together in hook |#1|.
%    \begin{macrocode}
\cs_new_protected:cpn { @@_rule_incompatible-error_gset:nnn } #1#2#3
  { \@@_tl_gset:cn { g_@@_#1_rule_ \@@_label_pair:nn {#2} {#3} _tl }
                   { xE } }
\cs_new_protected:cpn { @@_rule_incompatible-warning_gset:nnn } #1#2#3
  { \@@_tl_gset:cn { g_@@_#1_rule_ \@@_label_pair:nn {#2} {#3} _tl }
                   { xW } }
%    \end{macrocode}
%  \end{macro}
%
% \begin{macro}{\@@_rule_unrelated_gset:nnn, \@@_rule_gclear:nnn}
%    Undo a setting. \cs{@@_rule_unrelated_gset:nnn} doesn't need to do anything,
%    since we use \cs{@@_rule_gclear:nnn} before setting any rule.
%    \begin{macrocode}
\cs_new_protected:Npn \@@_rule_unrelated_gset:nnn #1#2#3 { }
\cs_new_protected:Npn \@@_rule_gclear:nnn #1#2#3
  { \cs_undefine:c { g_@@_#1_rule_ \@@_label_pair:nn {#2} {#3} _tl } }
%    \end{macrocode}
%  \end{macro}
%
% \begin{macro}[EXP]{\@@_label_pair:nn}
%   Ensure that the lexically greater label comes first.
%    \begin{macrocode}
\cs_new:Npn \@@_label_pair:nn #1#2
  {
    \if_case:w \@@_str_compare:nn {#1} {#2} \exp_stop_f:
           #1 | #1 %  0
    \or:   #1 | #2 % +1
    \else: #2 | #1 % -1
    \fi:
  }
%    \end{macrocode}
%  \end{macro}
%
% \begin{macro}[pTF]{\@@_label_ordered:nn}
%   Check that labels |#1| and |#2| are in the correct order (as
%   returned by \cs{@@_label_pair:nn}) and if so return true, else
%   return false.
%    \begin{macrocode}
\prg_new_conditional:Npnn \@@_label_ordered:nn #1#2 { TF }
  {
    \if_int_compare:w \@@_str_compare:nn {#1} {#2} > 0 \exp_stop_f:
      \prg_return_true:
    \else
      \prg_return_false:
    \fi:
  }
%    \end{macrocode}
%  \end{macro}
%
% \begin{macro}[EXP]{\@@_if_label_case:nnnnn}
%   To avoid doing the string comparison twice in \cs{@@_initialize_single:NNn}
%   (once with \cs{str_if_eq:nn} and again with \cs{@@_label_ordered:nn}),
%   we use a three-way branching macro that will compare |#1| and |#2|
%   and expand to \cs{use_i:nnn} if they are equal, \cs{use_ii:nn} if
%   |#1| is lexically greater, and \cs{use_iii:nn} otherwise.
%    \begin{macrocode}
\cs_new:Npn \@@_if_label_case:nnnnn #1#2
   {
     \cs:w use_
       \if_case:w \@@_str_compare:nn {#1} {#2}
          i \or: ii \else: iii \fi: :nnn
     \cs_end:
   }
%    \end{macrocode}
%  \end{macro}
%
%  \begin{macro}{\@@_update_hook_code:n}
%    Before \verb=\begin{document}=  this does nothing, in the body it
%    reinitializes the hook code using the altered data.
%    \begin{macrocode}
\cs_new_eq:NN \@@_update_hook_code:n \use_none:n
%    \end{macrocode}
%  \end{macro}
%
%  \begin{macro}{\@@_initialize_all:}
%    Initialize all known hooks (at \verb=\begin{document}=), i.e.,
%    update the fast execution token lists to hold the necessary code
%    in the right  order.
%    \begin{macrocode}
\cs_new_protected:Npn \@@_initialize_all: {
%    \end{macrocode}
%    First we change \cs{@@_update_hook_code:n} which so far was a
%    no-op to now initialize one hook. This way any later updates to
%    the hook will run that code and also update the execution token
%    list.
%    \begin{macrocode}
  \cs_gset_eq:NN \@@_update_hook_code:n \@@_initialize_hook_code:n
%    \end{macrocode}
%    Now we loop over all hooks that have been defined and update each
%    of them.
%    \begin{macrocode}
  \@@_debug:n { \prop_gclear:N \g_@@_used_prop }
  \seq_map_inline:Nn \g_@@_all_seq
      {
        \@@_update_hook_code:n {##1}
      }
%    \end{macrocode}
%    If we are debugging we show results hook by hook for all hooks
%    that have data. 
%    \begin{macrocode}
  \@@_debug:n
     { \iow_term:x{^^JAll~ initialized~ (non-empty)~ hooks:}
       \prop_map_inline:Nn \g_@@_used_prop
           { \iow_term:x{^^J~ ##1~ ->~
               \exp_not:v {@@~##1}~ }
           }
     }
%    \end{macrocode}
%    After all hooks are initialized we change the ``use'' to just
%    call the hook code and not initialize it (as it was done in the
%    preamble.
%    \begin{macrocode}
  \cs_gset_eq:NN \hook_use:n \@@_use_initialized:n
  \cs_gset_eq:NN \@@_preamble_hook:n \use_none:n
}
%    \end{macrocode}
%  \end{macro}
%
%
%
%  \begin{macro}{\@@_initialize_hook_code:n}
%    Initializing or reinitializing the fast execution hook code. In
%    the preamble this is selectively done in case a hook gets used
%    and at \verb=\begin{document}= this is done for all hooks and
%    afterwards only if the hook code changes.
%    \begin{macrocode}
\cs_new_protected:Npn \@@_initialize_hook_code:n #1
  {
    \@@_debug:n{ \iow_term:x{^^JUpdate~ code~ for~ hook~
                                    '#1' \on@line :^^J} }
%    \end{macrocode}
%    This does the sorting and the updates.
%    First thing we do is to check if a legacy hook macro exists and
%    if so we add it to the hook under the label \texttt{legacy}. This
%    might make the hook non-empty so we have to do this before
%    the then following test.
%    \begin{macrocode}
    \@@_include_legacy_code_chunk:n {#1}
%    \end{macrocode}
%    If there aren't any code
%    chunks for the current hook, there is no point in even starting
%    the sorting routine so we make a quick test for that and in that
%    case just update \cs{@@~\meta{hook}} to hold the |top-level| and
%    |next| code chunks. If there are code chunks we call
%    \cs{@@_initialize_single:NNn} and pass to it ready made csnames
%    as they are needed several times inside. This way we save a bit
%    on processing time if we do that up front.
%    \begin{macrocode}
    \@@_if_declared:nT {#1}
      {
        \prop_if_empty:cTF {g_@@_#1_code_prop}
          {
            \@@_tl_gset:co { @@~#1 }
              {
                \cs:w @@_toplevel~#1 \exp_after:wN \cs_end:
                \cs:w @@_next~#1 \cs_end:
              }
          }
          {
%    \end{macrocode}
%    By default the algorithm sorts the code chunks and then saves the
%    result in a token list for fast execution by adding the code one
%    after another using \cs{tl_gput_right:NV}. When we sort code for
%    a reversed hook, all we have to do is to add the code chunks in
%    the opposite order into the token list. So all we have to do
%    in preparation is to change two definitions used later on.
%    \begin{macrocode}
            \@@_if_reversed:nTF {#1}
              { \cs_set_eq:NN \@@_tl_gput:Nn    \@@_tl_gput_left:Nn
                \cs_set_eq:NN \@@_clist_gput:NV \clist_gput_left:NV  }
              { \cs_set_eq:NN \@@_tl_gput:Nn    \@@_tl_gput_right:Nn
                \cs_set_eq:NN \@@_clist_gput:NV \clist_gput_right:NV }
%    \end{macrocode}
%
%    When sorting, some relations (namely \verb|voids|) need to
%    act destructively on the code property lists to remove code that
%    shouldn't appear in the sorted hook token list, so we temporarily
%    save the old code property list so that it can be restored later.
%    \begin{macrocode}
            \prop_set_eq:Nc \l_@@_work_prop { g_@@_#1_code_prop }
            \@@_initialize_single:ccn
              { @@~#1 } { g_@@_#1_labels_clist } {#1}
%    \end{macrocode}
%    For debug display we want to keep track of those hooks that
%    actually got code added to them, so we record that in plist. We
%    use a plist to ensure that we record each hook name only once,
%    i.e., we are only interested in storing the keys and the value is
%    arbitrary.
%    \begin{macrocode}
            \@@_debug:n{ \exp_args:NNx \prop_gput:Nnn
                                       \g_@@_used_prop {#1}{} }
          }
      }
  }
%    \end{macrocode}
%  \end{macro}
%
%
% \begin{macro}[EXP]{\@@_tl_csname:n,\@@_seq_csname:n}
%   It is faster to pass a single token and expand it when necessary
%   than to pass a bunch of character tokens around.
%   \fmi{note to myself: verify}
%    \begin{macrocode}
\cs_new:Npn \@@_tl_csname:n #1 { l_@@_label_#1_tl }
\cs_new:Npn \@@_seq_csname:n #1 { l_@@_label_#1_seq }
%    \end{macrocode}
% \end{macro}
%
%
%  \begin{macro}{\l_@@_labels_seq,\l_@@_labels_int,\l_@@_front_tl,
%      \l_@@_rear_tl,\l_@@_label_0_tl}
%
%    For the sorting I am basically implementing Knuth's algorithm for
%    topological sorting as given in TAOCP volume 1 pages 263--266.
%    For this algorithm we need a number of local variables:
%    \begin{itemize}
%    \item
%       List of labels used in the current hook to label code chunks:
%    \begin{macrocode}
\seq_new:N \l_@@_labels_seq
%    \end{macrocode}
%    \item
%      Number of labels used in the current hook. In Knuth's algorithm
%      this is called $N$:
%    \begin{macrocode}
\int_new:N \l_@@_labels_int
%    \end{macrocode}
%    \item
%      The sorted code list to be build is managed using two pointers
%      one to the front of the queue and one to the rear. We model this
%      using token list pointers. Knuth calls them $F$ and $R$:
%    \begin{macrocode}
\tl_new:N \l_@@_front_tl
\tl_new:N \l_@@_rear_tl
%    \end{macrocode}
%    \item
%      The data for the start of the queue is kept in this token list,
%      it corresponds to what Don calls \texttt{QLINK[0]} but since we
%      aren't manipulating individual words in memory it is slightly
%      differently done:
%    \begin{macrocode}
\tl_new:c { \@@_tl_csname:n { 0 } }
%    \end{macrocode}
%
%    \end{itemize}
%  \end{macro}
%
%
%  \begin{macro}{\@@_initialize_single:NNn,\@@_initialize_single:ccn}
%
%    \cs{@@_initialize_single:NNn} implements the sorting of the code
%    chunks for a hook and saves the result in the token list for fast
%    execution (\verb=#4=). The arguments are \meta{hook-code-plist},
%    \meta{hook-code-tl}, \meta{hook-top-level-code-tl},
%    \meta{hook-next-code-tl},
%    \meta{hook-ordered-labels-clist} and \meta{hook-name} (the latter
%    is only used for debugging---the \meta{hook-rule-plist} is accessed
%    using the \meta{hook-name}).
%
%    The additional complexity compared to Don's algorithm is that we
%    do not use simple positive integers but have arbitrary
%    alphanumeric labels. As usual Don's data structures are chosen in
%    a way that one can omit a lot of tests and I have mimicked that as
%    far as possible. The result is a restriction I do not test for at
%    the moment: a label can't be equal to the number 0!  \fmi{Needs
%    checking for, just in case}
%
%    ^^A #1 <- \@@~#1
%    ^^A #2 <- \g_@@_#1_labels_clist
%    ^^A #3 <- #1
%    \begin{macrocode}
\cs_new_protected:Npn \@@_initialize_single:NNn #1#2#3
  {
%    \end{macrocode}
%    Step T1: Initialize the data structure \ldots
%    \begin{macrocode}
    \seq_clear:N \l_@@_labels_seq
    \int_zero:N  \l_@@_labels_int
%    \end{macrocode}
%
%    Store the name of the hook:
%    \begin{macrocode}
    \tl_set:Nn \l_@@_cur_hook_tl {#3}
%    \end{macrocode}
%    
%    We loop over the property list holding the code and record all
%    labels listed there. Only rules for those labels are of interest
%    to us. While we are at it we count them (which gives us the $N$
%    in Knuth's algorithm.  The prefix |label_| is added to the variables
%    to ensure that labels named |front|, |rear|, |labels|, or |return|
%    don't interact with our code.
%    \begin{macrocode}
    \prop_map_inline:Nn \l_@@_work_prop
       {
         \int_incr:N \l_@@_labels_int
         \seq_put_right:Nn \l_@@_labels_seq {##1}
         \@@_tl_set:cn { \@@_tl_csname:n {##1} } { 0 }
         \seq_clear_new:c { \@@_seq_csname:n {##1} }
       }
%    \end{macrocode}
%    Steps T2 and T3: Sort the relevant rules into the data structure\ldots
%    
%    This loop constitutes a square matrix of the labels in
%    \cs{l_@@_work_prop} in the
%    vertical and the horizontal directions.  However since the rule
%    $l_A\meta{rel}l_B$ is the same as $l_B\meta{rel}^{-1}l_A$ we can cut
%    the loop short at the diagonal of the matrix (\emph{i.e.}, when
%    both labels are equal), saving a good amount of time.  The way the
%    rules were set up (see the implementation of \cs{@@_rule_before_gset:nnn}
%    above) ensures that we have no rule in the ignored side of the
%    matrix, and all rules are seen.  The rules are applied in
%    \cs{@@_apply_label_pair:nnn}, which takes the properly-ordered pair
%    of labels as argument.
%    \begin{macrocode}
    \prop_map_inline:Nn \l_@@_work_prop
      {
        \prop_map_inline:Nn \l_@@_work_prop
          {
            \@@_if_label_case:nnnnn {##1} {####1}
              { \prop_map_break: }
              { \@@_apply_label_pair:nnn {##1} {####1} }
              { \@@_apply_label_pair:nnn {####1} {##1} }
                  {#3}
          }
      }
%    \end{macrocode}
%    Take a breath and take a look at the data structures that have
%    been set up:
%    \begin{macrocode}
    \@@_debug:n { \@@_debug_label_data:N \l_@@_work_prop }
%    \end{macrocode}
%    
%
%    Step T4:
%    \begin{macrocode}
    \tl_set:Nn \l_@@_rear_tl { 0 }
    \tl_set:cn { \@@_tl_csname:n { 0 } } { 0 }
    \seq_map_inline:Nn \l_@@_labels_seq
      {
        \int_compare:nNnT { \cs:w \@@_tl_csname:n {##1} \cs_end: } = 0
            {
              \tl_set:cn { \@@_tl_csname:n { \l_@@_rear_tl } }{##1}
              \tl_set:Nn \l_@@_rear_tl {##1}
            }
      }
    \tl_set_eq:Nc \l_@@_front_tl { \@@_tl_csname:n { 0 } }
%    \end{macrocode}
%    
%    \begin{macrocode}
    \@@_tl_gclear:N #1
    \clist_gclear:N #2
%    \end{macrocode}
%
%    The whole loop combines steps T5--T7:
%    \begin{macrocode}
    \bool_while_do:nn { ! \str_if_eq_p:Vn \l_@@_front_tl { 0 } }
      {
%    \end{macrocode}
%    This part is step T5:
%    \begin{macrocode}
        \int_decr:N \l_@@_labels_int
        \prop_get:NVN \l_@@_work_prop \l_@@_front_tl \l_@@_return_tl
        \exp_args:NNV \@@_tl_gput:Nn #1 \l_@@_return_tl
%    \end{macrocode}
%    
%    \begin{macrocode}
        \@@_clist_gput:NV #2 \l_@@_front_tl
        \@@_debug:n{ \iow_term:x{Handled~ code~ for~ \l_@@_front_tl} }
%    \end{macrocode}
%
%    This is step T6 except that we don't use a pointer $P$ to move
%    through the successors, but instead use \verb=##1= of the mapping
%    function.
%    \begin{macrocode}
        \seq_map_inline:cn { \@@_seq_csname:n { \l_@@_front_tl } }
          {
            \tl_set:cx { \@@_tl_csname:n {##1} }
                       { \int_eval:n
                           { \cs:w \@@_tl_csname:n {##1} \cs_end: - 1 }
                       }
            \int_compare:nNnT
                { \cs:w \@@_tl_csname:n {##1} \cs_end: } = 0
                {
                  \tl_set:cn { \@@_tl_csname:n { \l_@@_rear_tl } } {##1}
                  \tl_set:Nn \l_@@_rear_tl            {##1}
                }
          }
%    \end{macrocode}
%    and step T7:
%    \begin{macrocode}
        \tl_set_eq:Nc \l_@@_front_tl
                      { \@@_tl_csname:n { \l_@@_front_tl } }
%    \end{macrocode}
%
%    This is step T8: If we haven't moved the code for all labels
%    (i.e., if \cs{l_@@_labels_int} is still greater than zero) we
%    have a loop and our partial order can't be flattened out.
%    \begin{macrocode}
      }
    \int_compare:nNnF \l_@@_labels_int = 0
      {
        \iow_term:x{====================}
        \iow_term:x{Error:~ label~ rules~ are~ incompatible:}
%    \end{macrocode}
%
%    This is not really the information one needs in the error case
%    but will do for now \ldots \fmi{fix}
%    \begin{macrocode}
        \@@_debug_label_data:N \l_@@_work_prop
        \iow_term:x{====================}
      }
%    \end{macrocode}
%    After we have added all hook code to \verb=#1= we finish it off
%    with adding extra code for the |top-level| (\verb=#2=) and for one
%    time execution (\verb=#3=).  These should normally be empty.  The
%    |top-level| code is added with \cs{@@_tl_gput:Nn} as that might
%    change for a reversed hook (then |top-level| is the very first code
%    chunk added).  The |next| code is always added last.
%    \begin{macrocode}
    \exp_args:NNo \@@_tl_gput:Nn #1 { \cs:w @@_toplevel~#3 \cs_end: }
    \@@_tl_gput_right:No #1 { \cs:w @@_next~#3 \cs_end: }
  }
%    \end{macrocode}
%
%    \begin{macrocode}
\cs_generate_variant:Nn \@@_initialize_single:NNn { cc }
%    \end{macrocode}
%  \end{macro}
%
%
%
%  \begin{macro}{\@@_tl_gput:Nn,\@@_clist_gput:NV}
%    These append either on the right (normal hook) or on the left
%    (reversed hook). This is setup up in
%    \cs{@@_initialize_hook_code:n}, elsewhere their behavior is undefined.
%    \begin{macrocode}
\cs_new:Npn \@@_tl_gput:Nn    { \ERROR }
\cs_new:Npn \@@_clist_gput:NV { \ERROR }
%    \end{macrocode}
%  \end{macro}
%
%
%
%  \begin{macro}{\@@_apply_label_pair:nnn,\@@_label_if_exist_apply:nnnF}
%
%    This is the payload of steps T2 and T3 executed in the loop described
%    above. This macro assumes |#1| and |#2| are ordered, which means that
%    any rule pertaining the pair |#1| and |#2| is
%    \cs{g_@@_\meta{hook}_rule_\#1\string|\#2_tl}, and not
%    \cs{g_@@_\meta{hook}_rule_\#2\string|\#1_tl}.  This also saves a great deal
%    of time since we only need to check the order of the labels once.
%
%    The arguments here are \meta{label1}, \meta{label2}, \meta{hook}, and
%    \meta{hook-code-plist}.  We are about to apply the next rule and
%    enter it into the data structure.  \cs{@@_apply_label_pair:nnn} will
%    just call \cs{@@_label_if_exist_apply:nnnF} for the \meta{hook}, and
%    if no rule is found, also try the \meta{hook} name \verb=??=
%    denoting a default hook rule.
%
%    \cs{@@_label_if_exist_apply:nnnF} will check if the rule exists for
%    the given hook, and if so call \cs{@@_apply_rule:nnn}.
%    \begin{macrocode}
\cs_new_protected:Npn \@@_apply_label_pair:nnn #1#2#3
  {
%    \end{macrocode}
%    Extra complication: as we use default rules and local hook specific
%    rules we first have to check if there is a local rule and if that
%    exist use it. Otherwise check if there is a default rule and use
%    that.
%    \begin{macrocode}
    \@@_label_if_exist_apply:nnnF {#1} {#2} {#3}
      {
%    \end{macrocode}
%    If there is no hook-specific rule we check for a default one and
%    use that if it exists.
%    \begin{macrocode}
        \@@_label_if_exist_apply:nnnF {#1} {#2} { ?? } { }
      }
  }
\cs_new_protected:Npn \@@_label_if_exist_apply:nnnF #1#2#3
  {
    \if_cs_exist:w g_@@_ #3 _rule_ #1 | #2 _tl \cs_end:
%    \end{macrocode}
%    What to do precisely depends on the type of rule we have
%    encountered. If it is a \texttt{before} rule it will be handled by the
%    algorithm but other types need to be managed differently. All
%    this is done in \cs{@@_apply_rule:nnnN}.
%    \begin{macrocode}
      \@@_apply_rule:nnn {#1} {#2} {#3}
      \exp_after:wN \use_none:n
    \else:
      \use:nn
    \fi:
  }
%    \end{macrocode}
%  \end{macro}
%
%
%
%
%  \begin{macro}{\@@_apply_rule:nnn}
%    This is the code executed in steps T2 and T3 while looping through
%    the matrix  This is part of step T3. We are about to apply the next
%    rule and enter it into the data structure. The arguments are
%    \meta{label1}, \meta{label2}, \meta{hook-name}, and \meta{hook-code-plist}.
%    \begin{macrocode}
\cs_new_protected:Npn \@@_apply_rule:nnn #1#2#3
  {
    \cs:w @@_apply_
      \cs:w g_@@_#3_reversed_tl \cs_end: rule_
        \cs:w g_@@_ #3 _rule_ #1 | #2 _tl \cs_end: :nnn \cs_end:
      {#1} {#2} {#3}
  }
%    \end{macrocode}
% \end{macro}
%
%  \begin{macro}{\@@_apply_rule_<:nnn,\@@_apply_rule_>:nnn}
%    The most common cases are \texttt{\string<} and \texttt{\string>} so we handle
%    that first.  They are relations $\prec$ and $\succ$ in TAOCP, and
%    they dictate sorting.
%    \begin{macrocode}
\cs_new_protected:cpn { @@_apply_rule_<:nnn } #1#2#3
  {
    \@@_debug:n { \@@_msg_pair_found:nnn {#1} {#2} {#3} }
    \tl_set:cx { \@@_tl_csname:n {#2} }
       { \int_eval:n{ \cs:w \@@_tl_csname:n {#2} \cs_end: + 1 } }
    \seq_put_right:cn{ \@@_seq_csname:n {#1} }{#2}
  }
\cs_new_protected:cpn { @@_apply_rule_>:nnn } #1#2#3
  {
    \@@_debug:n { \@@_msg_pair_found:nnn {#1} {#2} {#3} }
    \tl_set:cx { \@@_tl_csname:n {#1} }
       { \int_eval:n{ \cs:w \@@_tl_csname:n {#1} \cs_end: + 1 } }
    \seq_put_right:cn{ \@@_seq_csname:n {#2} }{#1}
  }
%    \end{macrocode}
% \end{macro}
%
% \begin{macro}{\@@_apply_rule_xE:nnn,\@@_apply_rule_xW:nnn}
%   These relations make two labels incompatible within a hook.
%   |xE| makes raises an error if the labels are found in the same
%   hook, and |xW| makes it a warning.
%    \begin{macrocode}
\cs_new_protected:cpn { @@_apply_rule_xE:nnn } #1#2#3
  {
    \@@_debug:n { \@@_msg_pair_found:nnn {#1} {#2} {#3} }
    \msg_error:nnnnnn { hooks } { labels-incompatible }
      {#1} {#2} {#3} { 1 }
    \use:c { @@_apply_rule_->:nnn } {#1} {#2} {#3}
    \use:c { @@_apply_rule_<-:nnn } {#1} {#2} {#3}
  }
\cs_new_protected:cpn { @@_apply_rule_xW:nnn } #1#2#3
  {
    \@@_debug:n { \@@_msg_pair_found:nnn {#1} {#2} {#3} }
    \msg_warning:nnnnnn { hooks } { labels-incompatible }
      {#1} {#2} {#3} { 0 }
  }
%    \end{macrocode}
% \end{macro}
%
%  \begin{macro}{\@@_apply_rule_->:nnn,\@@_apply_rule_<-:nnn}
%    If we see \texttt{\detokenize{->}} we have to drop code for label
%    \verb=#3= and carry on. We could do a little better and drop
%    everything for that label since it doesn't matter where we sort
%    in the empty code. However that would complicate the algorithm a
%    lot with little gain.\footnote{This also hase the advantage that
%    the result of the sorting doesn't change which might otherwise
%    (for unrelated chunks) if we aren't careful.} So we still
%    unnecessarily try to sort it in and depending on the rules that
%    might result in a loop that is otherwise resolved. If that turns
%    out to be a real issue, we can improve the code.
%
%    Here the code is removed from \cs{l_@@_cur_hook_tl} rather than
%    \verb=#3= because the latter may be \verb=??=, and the default
%    hook doesn't store any code.  Removing from \cs{l_@@_cur_hook_tl}
%    makes default rules \verb=->= and  \verb=<-= work properly.
%    \begin{macrocode}
\cs_new_protected:cpn { @@_apply_rule_->:nnn } #1#2#3
  {
    \@@_debug:n
       {
         \@@_msg_pair_found:nnn {#1} {#2} {#3}
         \iow_term:x{--->~ Drop~ '#2'~ code~ from~
           \iow_char:N \\ g_@@_ \l_@@_cur_hook_tl _code_prop ~
           because~ of~ '#1' }
       }
    \prop_put:Nnn \l_@@_work_prop {#2} { }
  }
\cs_new_protected:cpn { @@_apply_rule_<-:nnn } #1#2#3
  {
    \@@_debug:n
       {
         \@@_msg_pair_found:nnn {#1} {#2} {#3}
         \iow_term:x{--->~ Drop~ '#1'~ code~ from~
           \iow_char:N \\ g_@@_ \l_@@_cur_hook_tl _code_prop ~
           because~ of~ '#2' }
       }
    \prop_put:Nnn \l_@@_work_prop {#1} { }
  }
%    \end{macrocode}
%  \end{macro}
%
% \begin{macro}{
%     \@@_apply_-rule_<:nnn,
%     \@@_apply_-rule_>:nnn,
%     \@@_apply_-rule_<-:nnn,
%     \@@_apply_-rule_->:nnn,
%     \@@_apply_-rule_x:nnn,
%   }
%   Reversed rules.
%    \begin{macrocode}
\cs_new_eq:cc { @@_apply_-rule_<:nnn  } { @@_apply_rule_>:nnn }
\cs_new_eq:cc { @@_apply_-rule_>:nnn  } { @@_apply_rule_<:nnn }
\cs_new_eq:cc { @@_apply_-rule_<-:nnn } { @@_apply_rule_<-:nnn }
\cs_new_eq:cc { @@_apply_-rule_->:nnn } { @@_apply_rule_->:nnn }
\cs_new_eq:cc { @@_apply_-rule_xE:nnn  } { @@_apply_rule_xE:nnn }
\cs_new_eq:cc { @@_apply_-rule_xW:nnn  } { @@_apply_rule_xW:nnn }
%    \end{macrocode}
% \end{macro}
%
%
% \begin{macro}{\@@_msg_pair_found:nnn}
%   A macro to avoid moving this many tokens around.
%    \begin{macrocode}
\cs_new_protected:Npn \@@_msg_pair_found:nnn #1#2#3
  {
    \iow_term:x{~ \str_if_eq:nnTF {#3} {??} {default} {~normal} ~
        rule~ \@@_label_pair:nn {#1} {#2}:~
        \use:c { g_@@_#3_rule_ \@@_label_pair:nn {#1} {#2} _tl } ~
        found}
  }
%    \end{macrocode}
% \end{macro}
%
%
%  \begin{macro}{\@@_debug_label_data:N}
%    
%    \begin{macrocode}
\cs_new_protected:Npn \@@_debug_label_data:N #1 {
  \iow_term:x{Code~ labels~ for~ sorting:}
  \iow_term:x{~ \seq_use:Nnnn\l_@@_labels_seq {~and~}{,~}{~and~} }
  \iow_term:x{^^J Data~ structure~ for~ label~ rules:}
  \prop_map_inline:Nn #1
       {
         \iow_term:x{~ ##1~ =~ \tl_use:c{ \@@_tl_csname:n {##1} }~ ->~
           \seq_use:cnnn{ \@@_seq_csname:n {##1} }{~->~}{~->~}{~->~}
         }
       }
  \iow_term:x{}
}
%    \end{macrocode}
%  \end{macro}
%
%
%
% \begin{macro}{\hook_show:n,\hook_log:n}
% \begin{macro}{\@@_log_line:x,\@@_log_line_indent:x}
% \begin{macro}{\@@_log:nN}
%   This writes out information about the hook given in its argument
%   onto the \texttt{.log} file and the terminal, if \cs{show_hook:n} is
%   used.  Internally both share the same structure, except that at the
%   end, \cs{hook_show:n} triggers \TeX's prompt.
%    \begin{macrocode}
\cs_new_protected:Npn \hook_log:n #1
  {
    \cs_set_eq:NN \@@_log_cmd:x \iow_log:x
    \@@_normalize_hook_args:Nn \@@_log:nN {#1} \tl_log:x
  }
\cs_new_protected:Npn \hook_show:n #1
  {
    \cs_set_eq:NN \@@_log_cmd:x \iow_term:x
    \@@_normalize_hook_args:Nn \@@_log:nN {#1} \tl_show:x
  }
\cs_new_protected:Npn \@@_log_line:x #1
  { \@@_log_cmd:x { >~#1 } }
\cs_new_protected:Npn \@@_log_line_indent:x #1
  { \@@_log_cmd:x { >~\@spaces #1 } }
%    \end{macrocode}
%
%    \begin{macrocode}
\cs_new_protected:Npn \@@_log:nN #1 #2
  {
    \@@_preamble_hook:n {#1}
    \@@_log_cmd:x { ^^J ->~The~hook~'#1': }
%    \end{macrocode}
%
%    \begin{macrocode}
    \@@_if_declared:nF {#1}
      { \@@_log_line:x { The~hook~is~not~declared. } }
    \@@_if_disabled:nT {#1}
      { \@@_log_line:x { The~hook~is~disabled. } }
    \hook_if_empty:nTF {#1}
      { #2 { The~hook~is~empty } }
      {
        \@@_log_line:x { Code~chunks: }
        \prop_if_empty:cTF { g_@@_#1_code_prop }
          { \@@_log_line_indent:x { --- } }
          {
            \prop_map_inline:cn { g_@@_#1_code_prop }
              { \@@_log_line_indent:x { ##1~->~\tl_to_str:n {##2} } }
          }
%    \end{macrocode}
%
%    If there is code in the |top-level| token list, print it:
%    \begin{macrocode}
        \@@_log_line:x
          {
            Document-level~(top-level)~code
            \@@_if_declared:nT {#1}
              { ~(executed~\@@_if_reversed:nTF {#1} {first} {last} ) } :
          }
        \@@_log_line_indent:x
          {
            \tl_if_empty:cTF { @@_toplevel~#1 }
              { --- }
              { -> ~ \exp_args:Nv \tl_to_str:n { @@_toplevel~#1 } }
          }
%    \end{macrocode}
%
%    \begin{macrocode}
        \@@_log_line:x { Extra~code~for~next~invocation: }
        \@@_log_line_indent:x
          {
            \tl_if_empty:cTF { @@_next~#1 }
              { --- }
%    \end{macrocode}
%
%    If the token list is not empty we want to display it but without
%    the first tokens (the code to clear itself) so we call a helper
%    command to  get rid of them.
%    \begin{macrocode}
              { ->~ \exp_args:Nv \@@_log_next_code:n { @@_next~#1 } }
          }
%    \end{macrocode}
%
%   Loop through the rules in a hook and for every rule found, print it.
%   If no rule is there, print |---|.  The boolean \cs{l_@@_tmpa_bool}
%   here indicates if the hook has no rules.
%    \begin{macrocode}
        \@@_log_line:x { Rules: }
        \bool_set_true:N \l_@@_tmpa_bool
        \@@_list_rules:nn {#1}
          {
            \bool_set_false:N \l_@@_tmpa_bool
            \@@_log_line_indent:x
              {
                ##2~ with~
                \str_if_eq:nnT {##3} {??} { default~ }
                relation~ ##1
              }
          }
        \bool_if:NT \l_@@_tmpa_bool
          { \@@_log_line_indent:x { --- } }
%    \end{macrocode}
%
%   When the hook is declared (that is, the sorting algorithm is applied
%   to that hook) and not empty
%    \begin{macrocode}
        \bool_lazy_and:nnTF
            { \@@_if_declared_p:n {#1} }
            { ! \hook_if_empty_p:n {#1} }
          {
            \@@_log_line:x
              {
                Execution~order
                \bool_if:NTF \l_@@_tmpa_bool
                  { \@@_if_reversed:nT {#1} { ~(after~reversal) } }
                  { ~(after~
                    \@@_if_reversed:nT {#1} { reversal~and~ }
                    applying~rules)
                  } :
              }
            #2 % \tl_show:n
              {
                \@spaces
                \clist_if_empty:cTF { g_@@_#1_labels_clist }
                  { --- }
                  { \clist_use:cn {g_@@_#1_labels_clist} { ,~ } }
              }
          }
          {
            #2
              {
                Hook~ \@@_if_declared:nTF {#1}
                  {code~pool~empty} {not~declared}
              }
          }
      }
  }
%    \end{macrocode}
%
% \begin{macro}{\@@_log_next_code:n}
%    To display the code for next invocation only (i.e., from
%    \cs{AddToHookNext} we have to remove the first two tokens at the
%    front which are \cs{tl_gclear:N} and the token list to clear.
%    \begin{macrocode}
\cs_new:Npn \@@_log_next_code:n #1
  { \exp_args:No \tl_to_str:n { \use_none:nn #1 } }
%    \end{macrocode}
% \end{macro}
%
% \end{macro}
% \end{macro}
% \end{macro}
%
% \begin{macro}{\@@_list_rules:nn}
% \begin{macro}{\@@_list_one_rule:nnn,\@@_list_if_rule_exists:nnnF}
%   This macro takes a \meta{hook} and an \meta{inline function} and
%   loops through each pair of \meta{labels} in the \meta{hook}, and if
%   there is a relation between this pair of \meta{labels}, the
%   \meta{inline function} is executed with |#1|${}={}$\meta{relation},
%   |#2|${}={}$\meta{label_1}\verb=|=\meta{label_2},
%   and |#3|${}={}$\meta{hook} (the latter may be the argument |#1| to
%   \cs{@@_list_rules:nn}, or |??| if it is a default rule).
%    \begin{macrocode}
\cs_new_protected:Npn \@@_list_rules:nn #1 #2
  {
    \cs_set_protected:Npn \@@_tmp:w ##1 ##2 ##3 {#2}
    \prop_map_inline:cn { g_@@_#1_code_prop }
      {
        \prop_map_inline:cn { g_@@_#1_code_prop }
          {
            \@@_if_label_case:nnnnn {##1} {####1}
              { \prop_map_break: }
              { \@@_list_one_rule:nnn {##1} {####1} }
              { \@@_list_one_rule:nnn {####1} {##1} }
                  {#1}
          }
      }
  }
%    \end{macrocode}
%
%   These two are quite similar to \cs{@@_apply_label_pair:nnn} and
%   \cs{@@_label_if_exist_apply:nnnF}, respectively, but rather than
%   applying the rule, they pass it to the \meta{inline function}.
%    \begin{macrocode}
\cs_new_protected:Npn \@@_list_one_rule:nnn #1#2#3
  {
    \@@_list_if_rule_exists:nnnF {#1} {#2} {#3}
      { \@@_list_if_rule_exists:nnnF {#1} {#2} { ?? } { } }
  }
\cs_new_protected:Npn \@@_list_if_rule_exists:nnnF #1#2#3
  {
    \if_cs_exist:w g_@@_ #3 _rule_ #1 | #2 _tl \cs_end:
      \exp_args:Nv \@@_tmp:w
        { g_@@_ #3 _rule_ #1 | #2 _tl } { #1 | #2 } {#3}
      \exp_after:wN \use_none:nn
    \fi:
    \use:n
  }
%    \end{macrocode}
% \end{macro}
% \end{macro}
%
% \begin{macro}{\@@_debug_print_rules:n}
%   A shorthand for debugging that prints similar to \cs{prop_show:N}.
%    \begin{macrocode}
\cs_new_protected:Npn \@@_debug_print_rules:n #1
  {
    \iow_term:n { The~hook~#1~contains~the~rules: }
    \cs_set_protected:Npn \@@_tmp:w ##1
      {
        \@@_list_rules:nn {#1}
          {
            \iow_term:x
              {
                > ##1 {####2} ##1 => ##1 {####1}
                \str_if_eq:nnT {####3} {??} { ~(default) }
              }
          }
      }
    \exp_args:No \@@_tmp:w { \use:nn { ~ } { ~ } }
  }
%    \end{macrocode}
% \end{macro}
%
%  \subsection{Specifying code for next invocation}
%
% \begin{macro}{\hook_gput_next_code:nn}
% \begin{macro}{%
%     \@@_gput_next_code:nn,
%     \@@_gput_next_do:nn,
%     \@@_gput_next_do:Nnn,
%     \@@_clear_next:n
%   }
%    \begin{macrocode}
\cs_new_protected:Npn \hook_gput_next_code:nn #1
  { \@@_normalize_hook_args:Nn \@@_gput_next_code:nn {#1} }
\cs_new_protected:Npn \@@_gput_next_code:nn #1 #2
  {
    \@@_if_disabled:nTF {#1}
      { \msg_error:nnn { hooks } { hook-disabled } {#1} }
      {
        \@@_declare:n {#1}
        \@@_if_declared:nTF {#1}
          { \@@_gput_next_do:nn {#1} {#2} }
          { \@@_try_declaring_generic_next_hook:nn {#1} {#2} }
      }
  }
%    \end{macrocode}
%
%    \begin{macrocode}
\cs_new_protected:Npn \@@_gput_next_do:nn #1
  {
    \exp_args:Nc \@@_gput_next_do:Nnn
      { @@_next~#1 } {#1}
  }
%    \end{macrocode}
%   First check if the ``next code'' token list is empty:  if so we need
%   to add a \cs{tl_gclear:c} to clear it, so the code lasts for one
%   usage only.  The token list is cleared early so that nested usages
%   don't get lost.  \cs{tl_gclear:c} is used instead of
%   \cs{tl_gclear:N} in case the hook is used in an expansion-only
%   context, so the token list doesn't expand before \cs{tl_gclear:N}:
%   that would make an infinite loop.  Also in case the main code token
%   list is empty, the hook code has to be updated to add the next
%   execution token list.
%    \begin{macrocode}
\cs_new_protected:Npn \@@_gput_next_do:Nnn #1 #2
  {
    \tl_if_empty:cT { @@~#2 }
      { \@@_update_hook_code:n {#2} }
    \tl_if_empty:NT #1
      { \@@_tl_gset:Nn #1 { \@@_clear_next:n {#2} } }
    \@@_tl_gput_right:Nn #1
  }
\cs_new_protected:Npn \@@_clear_next:n #1
  { \cs_gset_eq:cN { @@_next~#1 } \c_empty_tl }
%    \end{macrocode}
%  \end{macro}
%  \end{macro}
%
% \subsection{Using the hook}
%
% \begin{macro}{\hook_use:n}
% \begin{macro}[EXP]{\@@_use_initialized:n}
% \begin{macro}{\@@_use_undefined:w,\@@_use_end:}
% \begin{macro}{\@@_preamble_hook:n}
%   \cs{hook_use:n} as defined here is used in the preamble, where
%   hooks aren't initialized by default.  \cs{@@_use_initialized:n} is
%   also defined, which is the non-\tn{protected} version for use within
%   the document.  Their definition is identical, except for the
%   \cs{@@_preamble_hook:n} (which wouldn't hurt in the expandable
%   version, but it would be an unnecessary extra expansion).
%
%   \cs{@@_use_initialized:n} holds the expandable definition while in
%   the preamble. \cs{@@_preamble_hook:n} initializes the hook in the
%   preamble, and is redefined to \cs{use_none:n} at |\begin{document}|.
%
%   Both versions do the same internally:  check if the hook exist as
%   given, and if so use it as quickly as possible.  If it doesn't
%   exist, the a call to \cs{@@_use:wn} checks for file hooks.
%
%   At |\begin{document}|, all hooks are initialized, and any change in
%   them causes an update, so \cs{hook_use:n} can be made expandable.
%   This one is better not protected so that it can expand into nothing
%   if containing no code. Also important in case of generic hooks that
%   we do not generate a \cs{relax} as a side effect of checking for a
%   csname. In contrast to the \TeX{} low-level
%   \verb=\csname ...\endcsname= construct \cs{tl_if_exist:c} is
%   careful to avoid this.
%    \begin{macrocode}
\cs_new_protected:Npn \hook_use:n #1
  {
    \tl_if_exist:cTF { @@~#1 }
      {
        \@@_preamble_hook:n {#1}
        \cs:w @@~#1 \cs_end:
      }
      { \@@_use:wn #1 / \s_@@_mark {#1} }
  }
\cs_new:Npn \@@_use_initialized:n #1
  {
    \if_cs_exist:w @@~#1 \cs_end:
    \else:
      \@@_use_undefined:w
    \fi:
    \cs:w @@~#1 \@@_use_end:
  }
\cs_new:Npn \@@_use_undefined:w #1 #2 @@~#3 \@@_use_end:
  {
    #1 % fi
    \@@_use:wn #3 / \s_@@_mark {#3}
  }
\cs_new_protected:Npn \@@_preamble_hook:n #1
  { \@@_initialize_hook_code:n {#1} }
\cs_new_eq:NN \@@_use_end: \cs_end:
%    \end{macrocode}
% \end{macro}
% \end{macro}
% \end{macro}
% \end{macro}
%
% \begin{macro}[EXP]{\@@_use:wn}
% \begin{macro}{\@@_try_file_hook:n,\@@_if_exist_use:n}
%   \cs{@@_use:wn} does a quick check to test if the current hook is a
%   file hook: those need a special treatment.  If it is not, the hook
%   does not exist.  If it is, then \cs{@@_try_file_hook:n} is called,
%   and checks that the current hook is a file-specific hook using
%   \cs{@@_if_file_hook:wTF}.  If it's not, then it's a generic |file/|
%   hook and is used if it exist.
%
%   If it is a file-specific hook, it passes through the same
%   normalization as during declaration, and then it is used if defined.
%   \cs{@@_if_exist_use:n} checks if the hook exist, and calls
%   \cs{@@_preamble_hook:n} if so, then uses the hook.
%    \begin{macrocode}
\cs_new:Npn \@@_use:wn #1 / #2 \s_@@_mark #3
  {
    \str_if_eq:nnTF {#1} { file }
      { \@@_try_file_hook:n {#3} }
      { } % Hook doesn't exist
  }
\cs_new_protected:Npn \@@_try_file_hook:n #1
  {
    \@@_if_file_hook:wTF #1 / / \s_@@_mark
      {
        \exp_args:Ne \@@_if_exist_use:n
          { \exp_args:Ne \@@_file_hook_normalize:n {#1} }
      }
      { \@@_if_exist_use:n {#1} } % file/ generic hook (e.g. file/before)
  }
\cs_new_protected:Npn \@@_if_exist_use:n #1
  {
    \tl_if_exist:cT { @@~#1 }
      {
        \@@_preamble_hook:n {#1}
        \cs:w @@~#1 \cs_end:
      }
  }
%    \end{macrocode}
% \end{macro}
% \end{macro}
%
%  \begin{macro}{\hook_use_once:n}
%    For hooks that can and should be used only once we have a special
%    use command that remembers the hook name in
%    \cs{g_@@_execute_immediately_prop}. This has the effect that any
%    further code added to the hook is executed immediately rather
%    than stored in the hook.
%
%    The code needs some gymnastics to prevent space trimming from the
%    hook name, since \cs{hook_use:n} and \cs{hook_use_once:n} are
%    documented to not trim spaces.
%
%    \pho{Should this raise an error if the hook doesn't exist?}
%    \begin{macrocode}
\cs_new_protected:Npn \hook_use_once:n #1
  {
    \tl_if_exist:cT { @@~#1 }
      {
        \tl_set:Nn \l_@@_return_tl {#1}
        \@@_normalize_hook_args:Nn \@@_use_once_store:n
          { \l_@@_return_tl }
        \hook_use:n {#1}
      }
  }
\cs_new_protected:Npn \@@_use_once_store:n #1
  { \prop_gput:Nnn \g_@@_execute_immediately_prop {#1} { } }
%    \end{macrocode}
%  \end{macro}
%
% \subsection{Querying a hook}
%
% Simpler data types, like token lists, have three possible states; they
% can exist and be empty, exist and be non-empty, and they may not
% exist, in which case emptiness doesn't apply (though
% \cs{tl_if_empty:N} returns false in this case).
%
% Hooks are a bit more complicated: they have four possible states.
% A hook may exist or not, and either way it may or may not be empty
% (even a hook that doesn't exist may be non-empty).
%
% A hook is said to be empty when no code was added to it, either to
% its permanent code pool, or to its ``next'' token list.  The hook
% doesn't need to be declared to have code added to its code pool
% (it may happen that a package $A$ defines a hook \hook{foo}, but
% it's loaded after package $B$, which adds some code to that hook.
% In this case it is important that the code added by package $B$ is
% remembered until package $A$ is loaded).
%
% A hook is said to exist when it was declared with \cs{hook_new:n} or
% some variant thereof.
%
% \begin{macro}[pTF]{\hook_if_empty:n}
%   Test if a hook is empty (that is, no code was added to that hook).
%   A \meta{hook} being empty means that all three of its
%   \cs[no-index]{g_@@_\meta{hook}_code_prop}, its
%   \cs[no-index]{@@_toplevel~\meta{hook}} and its
%   \cs[no-index]{@@_next~\meta{hook}} are empty.
%    \begin{macrocode}
\prg_new_conditional:Npnn \hook_if_empty:n #1 { p , T , F , TF }
  {
    \@@_if_exist:nTF {#1}
      {
        \bool_lazy_and:nnTF
            { \prop_if_empty_p:c { g_@@_#1_code_prop } }
            {
              \bool_lazy_and_p:nn
                { \tl_if_empty_p:c { @@_toplevel~#1 } }
                { \tl_if_empty_p:c { @@_next~#1 } }
            }
          { \prg_return_true: }
          { \prg_return_false: }
      }
      { \prg_return_true: }
  }
%    \end{macrocode}
% \end{macro}
%
% \begin{macro}[pTF]{\@@_if_declared:n}
%   A canonical way to test if a hook exists.  A hook exists if the
%   token list that stores the sorted code for that hook,
%   \cs[no-index]{@@~\meta{hook}}, exists.  The property list
%   \cs[no-index]{g_@@_\meta{hook}_code_prop} cannot be used here
%   because often it is necessary to add code to a hook without knowing
%   if such hook was already declared, or even if it will ever be
%   (for example, in case the package that defines it isn't loaded).
%\fmi{docu update?}
%    \begin{macrocode}
\prg_new_conditional:Npnn \@@_if_declared:n #1 { p , T , F , TF }
  {
    \tl_if_exist:cTF { @@~#1 }
      { \prg_return_true: }
      { \prg_return_false: }
  }
%    \end{macrocode}
% \end{macro}
%
% \begin{macro}[pTF]{\@@_if_exist:n}
%   An internal check if the hook has already been declared with
%   \cs{@@_declare:n}.  This means that the hook was already used somehow
%   (a code chunk or rule was added to it), but it still wasn't declared
%   with \cs{hook_new:n}.
%\fmi{docu update?}
%    \begin{macrocode}
\prg_new_conditional:Npnn \@@_if_exist:n #1 { p , T , F , TF }
  {
    \prop_if_exist:cTF { g_@@_#1_code_prop }
      { \prg_return_true: }
      { \prg_return_false: }
  }
%    \end{macrocode}
% \end{macro}
%  
%  
%  
%  \begin{macro}[pTF]{\@@_if_created:n}
%
%    I used \cs{@@_if_created:n} because \cs{@@_if_exist:n} seems a
%    bit misleading (and it already exists, to check if a hook was
%    already declared with \cs{@@_declare:n}).
%\fmi{docu update?}
%    \begin{macrocode}
\prg_new_conditional:Npnn \@@_if_created:n #1 { p, T, F, TF }
  {
    \tl_if_exist:cTF { g_@@_#1_created_tl }
      { \prg_return_true: }
      { \prg_return_false: }
  }
%    \end{macrocode}
%  \end{macro}
%
% \begin{macro}[pTF]{\@@_if_reversed:n}
%   An internal conditional that checks if a hook is reversed.
%    \begin{macrocode}
\prg_new_conditional:Npnn \@@_if_reversed:n #1 { p , T , F , TF }
  {
    \if_int_compare:w \cs:w g_@@_#1_reversed_tl \cs_end: 1 < 0 \exp_stop_f:
      \prg_return_true:
    \else:
      \prg_return_false:
    \fi:
  }
%    \end{macrocode}
% \end{macro}
%
%
%  \subsection{Messages}
%
%    \begin{macrocode}
\msg_new:nnnn { hooks } { labels-incompatible }
  {
    Labels~`#1'~and~`#2'~are~incompatible
    \str_if_eq:nnF {#3} {??} { ~in~hook~`#3' } .~
    \int_compare:nNnTF {#4} = { 1 }
      { The~ code~ for~ both~ labels~ will~ be~ dropped. }
      { You~ may~ see~ errors~ later. }
  }
  { LaTeX~found~two~incompatible~labels~in~the~same~hook.~
    This~indicates~an~incompatibility~between~packages.  }
%    \end{macrocode}
%    
%    \begin{macrocode}
\msg_new:nnnn { hooks } { exists }
    { Hook~`#1'~ has~ already~ been~ declared. }
    { There~ already~ exists~ a~ hook~ declaration~ with~ this~
      name.\\
      Please~ use~ a~ different~ name~ for~ your~ hook.}
%    \end{macrocode}
%    
%
%    \begin{macrocode}
\msg_new:nnnn { hooks } { hook-disabled }
  { Cannot~add~code~to~disabled~hook~`#1'. }
  {
    The~hook~`#1`~you~tried~to~add~code~to~was~previously~disabled~
    with~\iow_char:N\\hook_disable:n~or~\iow_char:N\\DisableHook,~so~
    it~cannot~have~code~added~to~it.
  }
%    \end{macrocode}
%
%    \begin{macrocode}
\msg_new:nnn { hooks } { empty-label }
  {
    Empty~code~label~\msg_line_context:.~
    Using~`\@@_currname_or_default:'~instead.
  }
%    \end{macrocode}
%
%    \begin{macrocode}
\msg_new:nnn { hooks } { no-default-label }
  {
    Missing~(empty)~default~label~\msg_line_context:. \\
    This~command~was~ignored.
  }
%    \end{macrocode}
%
%    \begin{macrocode}
\msg_new:nnnn { hooks } { unknown-rule }
  { Unknown~ relationship~ `#3'~
    between~ labels~ `#2'~ and~ `#4'~
    \str_if_eq:nnF {#1} {??} { ~in~hook~`#1' }. ~
    Perhaps~ a~ missspelling?
  }
  {
    The~ relation~ used~ not~ known~ to~ the~ system.~ Allowed~ values~ are~
    `before'~ or~ `<',~
    `after'~ or~ `>',~
    `incompatible-warning',~
    `incompatible-error',~
    `voids'~ or~
    `unrelated'.
  }
%    \end{macrocode}
%    
%    \begin{macrocode}
\msg_new:nnnn { hooks } { misused-top-level }
  {
    Illegal~\iow_char:N \\AddToHook{#1}[top-level]{...}.\\
    'top-level'~is~reserved~for~the~user's~document.
  }
  {
    The~'top-level'~label~is~meant~for~user~code~only,~and~should~only~
    be~used~(sparingly)~in~the~main~document.~Use~the~default~label~
    '\@@_currname_or_default:'~for~this~\@cls@pkg,~or~another~
    suitable~label.
  }
%    \end{macrocode}
%
%    \begin{macrocode}
\msg_new:nnn { hooks } { set-top-level }
  {
    You~cannot~change~the~default~label~#1~`top-level'.~Illegal \\
    \use:nn { ~ } { ~ } \iow_char:N \\#2{#3} \\
    \msg_line_context:.
  }
%    \end{macrocode}
%
%    \begin{macrocode}
\msg_new:nnn { hooks } { ddhl-deprecated }
  {
    \iow_char:N \\DeclareDefaultHookLabel~is~deprecated.\\
    Use~\iow_char:N \\SetDefaultHookLabel~instead.\\ \\
    The~deprecated~name~will~be~removed~in~the~next~release.
  }
%    \end{macrocode}
%
%    \begin{macrocode}
\msg_new:nnn { hooks } { extra-pop-label }
  {
    Extra~\iow_char:N \\PopDefaultHookLabel. \\
    This~command~will~be~ignored.
  }
\msg_new:nnn { hooks } { missing-pop-label }
  {
    Missing~\iow_char:N \\PopDefaultHookLabel. \\
    The~label~`#1'~was~pushed~but~never~popped.~Something~is~wrong.
  }
%    \end{macrocode}
%
%    \begin{macrocode}
\msg_new:nnn { hooks } { should-not-happen }
  {
    ERROR!~This~should~not~happen.~#1 \\
    Please~report~at~https://github.com/latex3/latex2e.
  }
%    \end{macrocode}
%
%  \subsection{\LaTeXe{} package interface commands}
%
%
%
%  \begin{macro}{\NewHook,\NewReversedHook,\NewMirroredHookPair}
%    Declaring new hooks \ldots
%    \begin{macrocode}
\NewDocumentCommand \NewHook             { m }{ \hook_new:n {#1} }
\NewDocumentCommand \NewReversedHook     { m }{ \hook_new_reversed:n {#1} }
\NewDocumentCommand \NewMirroredHookPair { mm }{ \hook_new_pair:nn {#1}{#2} }
%    \end{macrocode}
%  \end{macro}
%
%
%
%  \begin{macro}{\DisableHook}
%    Disabling a (generic) hook.
%    \begin{macrocode}
\NewDocumentCommand \DisableHook { m }{ \hook_disable:n {#1} }
%    \end{macrocode}
%  \end{macro}
%  \begin{macro}{\AddToHook}
%    
%    \begin{macrocode}
\NewDocumentCommand \AddToHook { m o +m }
  { \hook_gput_code:nnn {#1} {#2} {#3} }
%    \end{macrocode}
%  \end{macro}
%
%  \begin{macro}{\AddToHookNext}
%    
%    \begin{macrocode}
\NewDocumentCommand \AddToHookNext { m +m }
  { \hook_gput_next_code:nn {#1} {#2} }
%    \end{macrocode}
%  \end{macro}
%
%
%  \begin{macro}{\RemoveFromHook}
%    
%    \begin{macrocode}
\NewDocumentCommand \RemoveFromHook { m o }
  { \hook_gremove_code:nn {#1} {#2} }
%    \end{macrocode}
%  \end{macro}
%
% \begin{macro}{\SetDefaultHookLabel}
% \begin{macro}{\PushDefaultHookLabel}
% \begin{macro}{\PopDefaultHookLabel}
% \begin{macro}{\DeclareDefaultHookLabel}
% \begin{macro}{\@@_curr_name_push:n,\@@_curr_name_push_aux:n}
% \begin{macro}{\@@_curr_name_pop:}
% \begin{macro}{\@@_end_document_label_check:}
%   The token list \cs{g_@@_hook_curr_name_tl} stores the name of the
%   current package/file to be used as label for hooks.
%   Providing a consistent interface is tricky, because packages can
%   be loaded within packages, and some packages may not use
%   \cs{SetDefaultHookLabel} to change the default label (in which
%   case \cs{@currname} is used).
%
%   To pull that one off, we keep a stack that contains the default
%   label for each level of input.  The bottom of the stack contains the
%   default label for the |top-level| (this stack should never go
%   empty). If we're building the format, set the default label to be
%   |top-level|:
%    \begin{macrocode}
\tl_gset:Nn \g_@@_hook_curr_name_tl { top-level }
%    \end{macrocode}
%
%   Then, in case we're in \pkg{latexrelease} we push something on
%   the stack to support roll forward.  But in some rare cases,
%   \pkg{latexrelease} may be loaded inside another package (notably
%   \pkg{platexrelease}), so we'll first push the |top-level| entry:
%   \changes{v1.0i}{2021/03/18}
%           {Only add \texttt{top-level} if not already there.}
%    \begin{macrocode}
%<latexrelease>\seq_if_empty:NT \g_@@_name_stack_seq
%<latexrelease>  { \seq_gput_right:Nn \g_@@_name_stack_seq { top-level } }
%    \end{macrocode}
%   then we dissect the \cs{@currnamestack}, adding \cs{@currname} to
%   the stack:
% \changes{v1.0f}{2020/11/24}{Support for roll forward (gh/434)}
%    \begin{macrocode}
%<latexrelease>\cs_set_protected:Npn \@@_tmp:w #1 #2 #3
%<latexrelease>  {
%<latexrelease>    \quark_if_recursion_tail_stop:n {#1}
%<latexrelease>    \seq_gput_right:Nn \g_@@_name_stack_seq {#1}
%<latexrelease>    \@@_tmp:w
%<latexrelease>  }
%<latexrelease>\exp_after:wN \@@_tmp:w \@currnamestack
%<latexrelease>  \q_recursion_tail \q_recursion_tail
%<latexrelease>  \q_recursion_tail \q_recursion_stop
%    \end{macrocode}
%   and finally set the default label to be the \cs{@currname}:
%   \changes{v1.0i}{2021/03/18}
%           {Remove the (empty) \enquote{top-level} from \cs{@currnamestack}.}
%    \begin{macrocode}
%<latexrelease>\tl_gset:Nx \g_@@_hook_curr_name_tl { \@currname }
%<latexrelease>\seq_gpop_right:NN \g_@@_name_stack_seq \l_@@_tmpa_tl
%    \end{macrocode}
%
%   Two commands keep track of the stack: when a file is input,
%   \cs{@@_curr_name_push:n} pushes the current default label to the
%   stack, and sets the new default label in one go:
%    \begin{macrocode}
\cs_new_protected:Npn \@@_curr_name_push:n #1
  { \exp_args:Nx \@@_curr_name_push_aux:n { \@@_make_name:n {#1} } }
\cs_new_protected:Npn \@@_curr_name_push_aux:n #1
  {
    \tl_if_blank:nTF {#1}
      { \msg_error:nn { hooks } { no-default-label } }
      {
        \str_if_eq:nnTF {#1} { top-level }
          {
            \msg_error:nnnnn { hooks } { set-top-level }
              { to } { PushDefaultHookLabel } {#1}
          }
          {
            \seq_gpush:NV \g_@@_name_stack_seq \g_@@_hook_curr_name_tl
            \tl_gset:Nn \g_@@_hook_curr_name_tl {#1}
          }
      }
  }
%    \end{macrocode}
%   and when an input is over, the topmost item of the stack is popped,
%   since the label will not be used again, and \cs{g_@@_hook_curr_name_tl}
%   is updated to the now topmost item of the stack:
%    \begin{macrocode}
\cs_new_protected:Npn \@@_curr_name_pop:
  {
    \seq_gpop:NNTF \g_@@_name_stack_seq \l_@@_return_tl
      { \tl_gset_eq:NN \g_@@_hook_curr_name_tl \l_@@_return_tl }
      { \msg_error:nn { hooks } { extra-pop-label } }
  }
%    \end{macrocode}
%
%   At the end of the document we want to check if there was no
%   \cs{@@_curr_name_push:} without a matching \cs{@@_curr_name_pop:}
%   (not a critical error, but it might indicate that something else is
%   not quite right):
%    \begin{macrocode}
\tl_gput_right:Nn \@kernel@after@enddocument@afterlastpage
  { \@@_end_document_label_check: }
\cs_new_protected:Npn \@@_end_document_label_check:
  {
    \seq_gpop:NNT \g_@@_name_stack_seq \l_@@_return_tl
      {
        \msg_error:nnx { hooks } { missing-pop-label }
          { \g_@@_hook_curr_name_tl }
        \tl_gset_eq:NN \g_@@_hook_curr_name_tl \l_@@_return_tl
        \@@_end_document_label_check:
      }
  }
%    \end{macrocode}
%
%   The token list \cs{g_@@_hook_curr_name_tl} is but a mirror of the top
%   of the stack.
%
%   Now define a wrapper that replaces the top of the stack with the
%   argument, and updates \cs{g_@@_hook_curr_name_tl} accordingly.
%    \begin{macrocode}
\NewDocumentCommand \SetDefaultHookLabel { m }
  {
    \seq_if_empty:NTF \g_@@_name_stack_seq
      {
        \msg_error:nnnnn { hooks } { set-top-level }
          { for } { SetDefaultHookLabel } {#1}
      }
      { \exp_args:Nx \@@_set_default_label:n { \@@_make_name:n {#1} } }
  }
\cs_new_protected:Npn \@@_set_default_label:n #1
  {
    \str_if_eq:nnTF {#1} { top-level }
      {
        \msg_error:nnnnn { hooks } { set-top-level }
          { to } { SetDefaultHookLabel } {#1}
      }
      { \tl_gset:Nn \g_@@_hook_curr_name_tl {#1} }
  }
\NewDocumentCommand \DeclareDefaultHookLabel { m }
  {
    \msg_error:nn { hooks } { ddhl-deprecated }
    \SetDefaultHookLabel {#1}
  }
%    \end{macrocode}
%
%   The label is only automatically updated with \cs{@onefilewithoptions}
%   (\cs{usepackage} and \cs{documentclass}), but some packages, like
%   Ti\emph{k}Z, define package-like interfaces, like
%   \cs{usetikzlibrary} that are wrappers around \cs{input}, so they
%   inherit the default label currently in force (usually |top-level|,
%   but it may change if loaded in another package).  To provide a
%   package-like behaviour also for hooks in these files, we provide
%   high-level access to the default label stack.
%    \begin{macrocode}
\NewDocumentCommand \PushDefaultHookLabel { m }
  { \@@_curr_name_push:n {#1} }
\NewDocumentCommand \PopDefaultHookLabel { }
  { \@@_curr_name_pop: }
%    \end{macrocode}
%
%   The current label stack holds the labels for all files but the
%   current one (more or less like \cs{@currnamestack}), and the current
%   label token list, \cs{g_@@_hook_curr_name_tl}, holds the label for
%   the current file.  However \cs{@pushfilename} happens before
%   \cs{@currname} is set, so we need to look ahead to get the
%   \cs{@currname} for the label.  \pkg{expl3} also requires the current
%   file in \cs{@pushfilename}, so here we abuse
%   \cs{@expl@push@filename@aux@@@@} to do \cs{@@_curr_name_push:n}.
%    \begin{macrocode}
\cs_gset_protected:Npn \@expl@push@filename@aux@@@@ #1#2#3
  {
    \@@_curr_name_push:n {#3}
    \str_gset:Nx \g_file_curr_name_str {#3}
    #1 #2 {#3}
  }
%    \end{macrocode}
% \end{macro}
% \end{macro}
% \end{macro}
% \end{macro}
% \end{macro}
% \end{macro}
% \end{macro}
%
%
%
%
%  \begin{macro}{\UseHook,\UseOneTimeHook}
%    Avoid the overhead of \pkg{xparse} and its protection that we
%    don't want here (since the hook should vanish without trace if empty)!
%    \begin{macrocode}
\cs_new:Npn \UseHook        { \hook_use:n }
\cs_new:Npn \UseOneTimeHook { \hook_use_once:n }
%    \end{macrocode}
%  \end{macro}
%
%
%
% \begin{macro}{\ShowHook,\LogHook}
%    \begin{macrocode}
\cs_new_protected:Npn \ShowHook { \hook_show:n }
\cs_new_protected:Npn \LogHook { \hook_log:n }
%    \end{macrocode}
% \end{macro}
%
%  \begin{macro}{\DebugHooksOn,\DebugHooksOff}
%    
%    \begin{macrocode}
\cs_new_protected:Npn \DebugHooksOn  { \hook_debug_on:  }
\cs_new_protected:Npn \DebugHooksOff { \hook_debug_off: }
%    \end{macrocode}
%  \end{macro}
%
%
%
%  \begin{macro}{\DeclareHookRule}
%    
%    \begin{macrocode}
\NewDocumentCommand \DeclareHookRule { m m m m }
                    { \hook_gset_rule:nnnn {#1}{#2}{#3}{#4} }
%    \end{macrocode}
%  \end{macro}
%
%  \begin{macro}{\DeclareDefaultHookRule}
%    This declaration is only supported before \verb=\begin{document}=.
%    \begin{macrocode}
\NewDocumentCommand \DeclareDefaultHookRule { m m m }
                    { \hook_gset_rule:nnnn {??}{#1}{#2}{#3} }
\@onlypreamble\DeclareDefaultHookRule
%    \end{macrocode}
%  \end{macro}
%
%  \begin{macro}{\ClearHookRule}
%    A special setup rule that removes an existing relation.
%    Basically {@@_rule_gclear:nnn} plus fixing the property list for debugging.
%    \fmi{Need an L3 interface, or maybe it should get dropped?}
%    \begin{macrocode}
\NewDocumentCommand \ClearHookRule { m m m }
{ \hook_gset_rule:nnnn {#1}{#2}{unrelated}{#3} }
%    \end{macrocode}
%  \end{macro}
%
%
% \begin{macro}[EXP]{\IfHookEmptyTF}
%   Here we avoid the overhead of \pkg{xparse}, since \cs{IfHookEmptyTF}
%   is used in \cs{end} (that is, every \LaTeX{} environment).  As a
%   further optimisation, use \cs{let} rather than \cs{def} to avoid one
%   expansion step.
%    \begin{macrocode}
\cs_new_eq:NN \IfHookEmptyTF \hook_if_empty:nTF
%    \end{macrocode}
% \end{macro}
%
% \begin{macro}[EXP]{\IfHookExistsTF}
%    Marked for removal!
% \phoinline{\cs{IfHookExistsTF} is used in \texttt{jlreq.cls},
% \texttt{pxatbegshi.sty}, \texttt{pxeverysel.sty},
% \texttt{pxeveryshi.sty}, so the public name may be an alias of the
% internal conditional for a while.  Regardless, those packages' use for
% \cs{IfHookExistsTF} is not really correct and can be changed.}
%    \begin{macrocode}
\cs_new_eq:NN \IfHookExistsTF \@@_if_declared:nTF
%    \end{macrocode}
% \end{macro}
%
%%%%%%%%%%%%%%%%%%%%%%%%%%%%%%%%%%%%%%%%%%%%%%%%%%%%%%%%%%5
%
% \subsection{Internal commands needed elsewhere}
%
% Here we set up a few horrible (but consistent) \LaTeXe{} names to
% allow for internal commands to be used outside this module. We
% have to unset the \texttt{@\/@} since we want double ``at'' sign
% in place of double underscores.
%
%    \begin{macrocode}
%<@@=>
%    \end{macrocode}
%
%  \begin{macro}{\@expl@@@initialize@all@@,
%                \@expl@@@hook@curr@name@pop@@}
%
% \InternalDetectionOff
%    \begin{macrocode}
\cs_new_eq:NN \@expl@@@initialize@all@@
              \__hook_initialize_all:
%    \end{macrocode}
%
%    \begin{macrocode}
\cs_new_eq:NN \@expl@@@hook@curr@name@pop@@
              \__hook_curr_name_pop:
%    \end{macrocode}
% \InternalDetectionOn
%  \end{macro}
%
%    Rolling back here doesn't undefine the interface commands as they
%    may be used in packages without rollback functionality. So we
%    just make them do nothing which may or may not work depending on
%    the code usage.
% \changes{v1.0d}{2020/10/04}{Definition \cs{AddToHookNext} was supposed
%                             to be for \cs{AddToHook} vice versa (gh/401)}
%    \begin{macrocode}
%
%<latexrelease>\IncludeInRelease{0000/00/00}%
%<latexrelease>                 {lthooks}{The~hook~management}%
%<latexrelease>
%<latexrelease>\def\NewHook#1{}
%<latexrelease>\def\NewReversedHook#1{}
%<latexrelease>\def\NewMirroredHookPair#1#2{}
%<latexrelease>
%<latexrelease>\long\def\AddToHookNext#1#2{}
%<latexrelease>
%<latexrelease>\def\AddToHook#1{\@gobble@AddToHook@args}
%<latexrelease>\providecommand\@gobble@AddToHook@args[2][]{}
%<latexrelease>
%<latexrelease>\def\RemoveFromHook#1{\@gobble@RemoveFromHook@arg}
%<latexrelease>\providecommand\@gobble@RemoveFromHook@arg[1][]{}
%<latexrelease>
%<latexrelease>\def \UseHook        #1{}
%<latexrelease>\def \UseOneTimeHook #1{}
%<latexrelease>\def \ShowHook #1{}
%<latexrelease>\let \DebugHooksOn \@empty
%<latexrelease>\let \DebugHooksOff\@empty
%<latexrelease>
%<latexrelease>\def \DeclareHookRule #1#2#3#4{}
%<latexrelease>\def \DeclareDefaultHookRule #1#2#3{}
%<latexrelease>\def \ClearHookRule #1#2#3{}
%    \end{macrocode}
%    If the hook management is not provided we make the test for existence
%    false and the test for empty true in the hope that this is most
%    of the time reasonable. If not a package would need to guard
%    against running in an old kernel.
%    \begin{macrocode}
%<latexrelease>\long\def \IfHookExistsTF #1#2#3{#3}
%<latexrelease>\long\def \IfHookEmptyTF #1#2#3{#2}
%<latexrelease>
%<latexrelease>\EndModuleRelease
\ExplSyntaxOff
%</2ekernel|latexrelease>
%    \end{macrocode}
%
%
% \Finale
%
%
%%%%%%%%%%%%%%%%%%%%%%%%%%%%%%%%%%%%%%%%%%%%%  
\endinput
%%%%%%%%%%%%%%%%%%%%%%%%%%%%%%%%%%%%%%%%%%%%%  

                     % \iffalse meta-comment
%%
%% Copyright (C) 2020-2024
%% Frank Mittelbach, The LaTeX Project
%%
%
% This file is part of the LaTeX base system.
% -------------------------------------------
%
% It may be distributed and/or modified under the
% conditions of the LaTeX Project Public License, either version 1.3c
% of this license or (at your option) any later version.
% The latest version of this license is in
%    https://www.latex-project.org/lppl.txt
% and version 1.3c or later is part of all distributions of LaTeX
% version 2008 or later.
%
% This file has the LPPL maintenance status "maintained".
%
% The list of all files belonging to the LaTeX base distribution is
% given in the file `manifest.txt'. See also `legal.txt' for additional
% information.
%
% The list of derived (unpacked) files belonging to the distribution
% and covered by LPPL is defined by the unpacking scripts (with
% extension .ins) which are part of the distribution.
%
% \fi
%
% \iffalse
%
%%% From File: ltshipout.dtx
%
%<*driver>
% \fi
\ProvidesFile{ltshipout.dtx}
             [2024/02/11 v1.0n LaTeX Kernel (Shipout)]
% \iffalse
%
\documentclass{l3doc}
\GetFileInfo{ltshipout.dtx}

\providecommand\InternalDetectionOff{}
\providecommand\InternalDetectionOn{}

\EnableCrossrefs
\CodelineIndex
\begin{document}
  \DocInput{ltshipout.dtx}
\end{document}
%</driver>
%
% \fi
%
%
% \long\def\fmi#1{\begin{quote}\itshape Todo: #1\end{quote}}
%
% \providecommand\hook[1]{\texttt{#1}}
% \providecommand\pkg[1]{\texttt{#1}}
%
%
% \title{The \texttt{ltshipout} documentation\thanks{This file has version
%    \fileversion\ dated \filedate, \copyright\ \LaTeX\
%    Project.}}
%
% \author{Frank Mittelbach, \LaTeX{} Project Team}
%
% \maketitle
%
%
% \tableofcontents
%
% \section{Introduction}
%
%    The code provides an interface to the \cs{shipout} primitive of
%    \TeX{} which is called when a finished pages is finally
%    \enquote{shipped out} to the target output file, e.g., the
%    \texttt{.dvi} or \texttt{.pdf} file.
%    A good portion of the code is based on ideas by Heiko Oberdiek
%    implemented in his packages \pkg{atbegshi} and \pkg{atenddvi}
%    even though the interfaces are somewhat
%    different.\footnote{Heiko's interfaces are emulated by the kernel
%    code, if a document requests his packages, so older documents
%    will continue to work.}
%
%  \subsection{Overloading the \cs{shipout} primitive}
%
%
% \begin{function}{\shipout}
%    With this implementation \TeX's shipout primitive is no longer
%    available for direct use. Instead \cs{shipout} is running some
%    (complicated) code that picks up the box to be shipped out
%    regardless of how that is done, i.e., as a constructed \cs{vbox}
%    or \cs{hbox} or as a box register.
%
%    It then stores it in a named box register.  This box can then be
%    manipulated through a set of hooks after which it is shipped out
%    for real.
%
%    Each shipout that actually happens (i.e., where the material is
%    not discarded for one or the other reason) is recorded and the
%    total number is available in a readonly variable and in a
%    \LaTeX{} counter.
% \end{function}
%
%
% \begin{function}{\RawShipout}
%    This command implements a simplified shipout that bypasses the
%    foreground and background 
%    hooks, e.g., only \hook{shipout/firstpage} and
%    \hook{shipout/lastpage} are executed and the total shipout
%    counters are incremented.
%
%    The command doesn't use \cs{ShipoutBox} but its own private box
%    register so that it can be used inside of shipout hooks to do
%    some additional shipouts while already in the output routine with
%    the current page being stored in \cs{ShipoutBox}. It does have
%    access to \cs{ShipoutBox} if it is used in \hook{shipout/before}
%    (or \hook{shipout/after}) and can use its content.
%
%    It is safe to use it in \hook{shipout/before} or
%    \hook{shipout/after} but not necessarily in the other
%    \hook{shipout/...} hooks as they are intended for special
%    processing.
% \end{function}
%
%  \begin{variable}{\ShipoutBox,\l_shipout_box}
%    This box register is called \cs{ShipoutBox} (alternatively
%    available via the L3 name \cs{l_shipout_box}).
%
%    This box is a ``local'' box and assignments to it should be done
%    only locally. Global assignments (as done by some packages with
%    older code where this is box is known as 255) may work but they are
%    conceptually wrong and may result in errors under certain
%    circumstances.
%
%    During the execution of \hook{shipout/before} this box contains
%    the accumulated material for the page, but not yet any material
%    added by other shipout hooks.
%    During execution of \hook{shipout/after}, i.e., after the shipout
%    has happened, the box also contains any background or foreground
%    material.
%
%    Material from the hooks \hook{shipout/firstpage} or
%    \hook{shipout/lastpage} is not included (but only used during the
%    actual shipout) to facilitate reuse of the box data (e.g.,
%    \hook{shipout/firstpage} material should never be added to a
%    later page of the output).
%  \end{variable}
%
%
%  \begin{variable}{\l_shipout_box_ht_dim,
%                   \l_shipout_box_dp_dim,\l_shipout_box_wd_dim,
%                   \l_shipout_box_ht_plus_dp_dim}
%    The shipout box dimensions are available in the L3 registers
%    \cs{l_shipout_box_ht_dim}, etc.\ (there are no \LaTeXe{}
%    names).\footnotemark{} These variables can be used
%    inside the hook code for \hook{shipout/before},
%    \hook{shipout/foreground} and \hook{shipout/background} if needed.
%  \end{variable}
%  \footnotetext{Might need changing, but HO's version as strings
%    is not really helpful I think).}
%
%
%
%
% \subsection{Provided hooks}
%
%  \begin{variable}{shipout/before,shipout/after,
%                   shipout/foreground,shipout/background,
%                   shipout/firstpage,
%                   shipout/lastpage}
%    The code for \cs{shipout} offers a number of hooks into which packages (or the
%    user) can add code to support different use cases.
%    These are:
%    \begin{description}
%    \item[\hook{shipout/before}]
%
%       This hook is executed after the finished page has been stored in
%       \cs{ShipoutBox} / \cs{l_shipout_box}).
%       It can be used to alter that box content or to discard it
%       completely (see \cs{DiscardShipoutBox} below).
%
%       You can use \cs{RawShipout} inside this hook for special use
%       cases. It can make use of \cs{ShipoutBox} (which doesn't yet
%       include the background and foreground material).
%
%       \textbf{Note:} It is not possible (or say advisable) to try
%       and use this hook to typeset material with the intention to
%       return it to main vertical list, it will go wrong and give
%       unexpected results in many cases---for starters it will appear
%       after the current page not before or it will vanish or the
%       vertical spacing will be wrong!
%    \end{description}
%  \end{variable}
%
%    \begin{description}
%    \item[\hook{shipout/background}]
%
%       This hook adds a picture environment into the background of
%       the page with the \texttt{(0,0)} coordinate in the top-left
%       corner using a \cs{unitlength} of \texttt{1pt}.
%
%       It should therefore only receive \cs{put} commands or other
%       commands suitable in a \texttt{picture} environment and the
%       vertical coordinate values would normally be
%       negative.
%
%       Technically this is implemented by adding a zero-sized
%       \cs{hbox} as the very first item into the \cs{ShipoutBox}
%       containing that \texttt{picture} environment. Thus the rest of
%       the box content will overprint what ever is typeset by that hook.
%
%
%    \item[\hook{shipout/foreground}]
%
%       This hook adds a picture environment into the foreground of
%       the page with the \texttt{(0,0)} coordinate in the top-left
%       corner using a \cs{unitlength} of \texttt{1pt}.
%
%       Technically this is implemented by adding a zero-sized
%       \cs{hbox} as the very last item into the \cs{ShipoutBox} and
%       raising it up so that it still has its \texttt{(0,0)} point in
%       the top-left corner.
%       But being placed after the main box content it will be typeset
%       later and thus overprints it (i.e., is in the foreground).
%
%    \item[\hook{shipout}]
%       This hook is executed after foreground and/or background
%       material has been added, i.e., just in front of the actual
%       shipout operation. Its purpose is to allow manipulation of the
%       finalized box (stored in \cs{ShipoutBox}) with the extra
%       material also in place (which is not yet the case in
%       \hook{shipout/before}).
%
%       It cannot be used to cancel the shipout operation via
%       \cs{DiscardShipoutBox} (that has
%       to happen in \hook{shipout/before}, if desired!
%
%
%    \item[\hook{shipout/firstpage}]
%
%       The material from this hook is executed only once at the very
%       beginning of the first output page that is shipped out (i.e.,
%       not discarded at the last minute). It should only contain
%       \cs{special} or similar commands needed to direct post processors
%       handling the \texttt{.dvi} or \texttt{.pdf} output.\footnotemark
%  \footnotetext{In
%         \LaTeXe{} that was already existing, but implemented using a box
%         register with the name \cs{@begindvibox}.}
%
%       ^^A \fmi{not sure it has to be that restrictive.}
%
%       This hook is added to the very first page regardless of how it
%       is shipped out (i.e., with \cs{shipout} or \cs{RawShipout}).
%
%    \item[\hook{shipout/lastpage}]
%
%       The corresponding hook to add \cs{special}s at the very end of
%       the output file. It is only executed on the very last page of
%       the output file ---
%       or rather on the page that \LaTeX{} believes is the last one.
%       Again it is executed regardless of the shipout method.
%
%       It may not be possible for \LaTeX{} to correctly determine
%       which page is the last one without several reruns. If this
%       happens and the hook is non-empty then \LaTeX{} will add an
%       extra page to place the material and also request a rerun to
%       get the correct placement sorted out.
%
%    \item[\hook{shipout/after}]
%
%       This hook is executed after a shipout has happened. If the
%       shipout box is discarded this hook is not looked at.
%
%       You can use \cs{RawShipout} inside this hook for special use
%       cases and the main \cs{ShipoutBox} is still available at this
%       point (but in contrast to \hook{shipout/before} it now
%       includes the background and foreground material).
%
%       \textbf{Note:} Just like \hook{shipout/before} this hook is
%       not meant to be used for adding typeset material back
%       to the main vertical list---it might vanish or the
%       vertical spacing will be wrong!
%
%
%    \end{description}
%
%    As mentioned above the hook \hook{shipout/before} is executed
%    first and can manipulate the prepared shipout box stored in
%    \cs{ShipoutBox} or set things up for use in \cs{write} during the
%    actual shipout. It is even run if there was a
%    \cs{DiscardShipoutBox} request in the document.
%
%    The other hooks (except \hook{shipout} and \hook{shipout/after}) are added inside
%    hboxes to the box being shipped out in the following order:
%    \begin{center}
%    \begin{tabular}{ll}
%       \hook{shipout/firstpage}   & only on the first page \\
%       \hook{shipout/background}  &                        \\
%       \meta{boxed content of \cs{ShipoutBox}} &             \\
%       \hook{shipout/foreground}  &                       \\
%       \hook{shipout/lastpage}    & only on the last page \\
%    \end{tabular}
%    \end{center}
%    If any of the hooks has no code then the corresponding box is
%    added at that point.
%
%    Once the (page) box has got the above extra content it can
%    again be manipulated using the \hook{shipout} hook and then
%    is shipped out for real.
%
%    Once the (page) box has been shipped out the \hook{shipout/after}
%    hook is called (while you are still inside the output routine). It
%    is not called if the shipout box was discarded.
%
%    In a document that doesn't produce pages, e.g., only makes
%    \cs{typeout}s, none of the hooks are ever executed (as there is no
%    \cs{shipout}) not even the \hook{shipout/lastpage} hook.
%
%    If \cs{RawShipout} is used instead of \cs{shipout} then only the
%    hooks \hook{shipout/firstpage} and \hook{shipout/lastpage} are
%    executed (on the first or last page), all others are bypassed.
%
%
%
% \subsection{Legacy \LaTeX{} commands}
%
% \begin{function}{\AtBeginDvi,\AtEndDvi}
%    \begin{syntax}
%      \cs{AtBeginDvi} \Arg{code}
%    \end{syntax}
%    \cs{AtBeginDvi} is the existing \LaTeXe{} interface to fill the
%    \hook{shipout/firstpage} hook. This is not really a good name
%    as it is not just supporting \texttt{.dvi} but also \texttt{.pdf}
%    output or \texttt{.xdv}.
%
%    \cs{AtEndDvi} is the counterpart that was not available in the
%    kernel but only through the package \pkg{atenddvi}. It fills the
%    \hook{shipout/lastpage} hook.
%
%    Neither interface can set a code label but uses the current default label.
%
% \end{function}
%
%    As these two wrappers have been available for a long time we
%    continue offering them (but not enhancing them, e.g., by providing
%    support for code labels).
%
%    For new code we strongly suggest using the high-level hook
%    management commands directly instead of ``randomly-named''
%    wrappers.  This will lead to code that is easier to understand
%    and to maintain and it also allows you to set code labels if
%    needed.
%
%    For this reason we do not provide any other ``new'' wrapper
%    commands for the above hooks in the kernel, but only keep
%    the existing ones for backward compatibility.
%
%
% \subsection{Special commands for use inside the hooks}
%
% \begin{function}{\DiscardShipoutBox,\shipout_discard:}
%   \begin{syntax}
%     \cs{AddToHookNext} \texttt{\{shipout/before\} \{...\cs{DiscardShipoutBox}...\}}
%   \end{syntax}
%    The \cs{DiscardShipoutBox} declaration (L3 name
%    \cs{shipout_discard:})
%    requests that on the next
%    shipout the page box is thrown away instead of being shipped to
%    the \texttt{.dvi} or \texttt{.pdf} file.
%
%    Typical applications wouldn't do this unconditionally, but have
%    some processing logic that decides to use or not to use the page.
%
%    Note that if this declaration is used directly in the document it
%    may depend on the placement to which page it applies, given that
%    \LaTeX{} output routine is called in an asynchronous manner!
%    Thus normally one would use this only as part of the
%    \hook{shipout/before} code.
%
%    \fmi{Once we have a new mark mechanism available we can improve
%    on that and make sure that the declaration applies to the page
%    that contains it --- not done (yet)}
%
%    \cs{DiscardShipoutBox} cannot be used in any of the \hook{shipout/...}
%    hooks other than \hook{shipout/before}.
%  \end{function}
%
%    In the \pkg{atbegshi} package there are a number of additional
%    commands for use inside the \hook{shipout/before} hook. They
%    should normally not be needed any more as one can instead simply
%    add code to the hooks \hook{shipout/before}, \hook{shipout},
%    \hook{shipout/background} or
%    \hook{shipout/foreground}.\footnote{If that assumption turns out to
%    be wrong it would be trivial to change them to public functions
%    (right now they are private).} If \pkg{atbegshi} gets loaded then
%    those commands become available as public functions with their original
%    names as given below.
%
%
%
% \subsection{Provided Lua\TeX\ callbacks}
% 
%  \begin{variable}{pre_shipout_filter}
%    Under Lua\TeX{} the \texttt{pre\_shipout\_filter} Lua callback is
%    provided which gets called directly after the
%    \hook{shipout} hook, immediately before the shipout
%    primitive gets invoked.
%    The signature is
%    \begin{verbatim}
%     function(<node> head)
%       return true
%     end
%    \end{verbatim}
%    The \texttt{head} is the list node corresponding to the box to be shipped out.
%    The return value should always be \texttt{true}.
%  \end{variable}
%
% \subsection{Information counters}
%
%
%  \begin{variable}{\ReadonlyShipoutCounter,\g_shipout_readonly_int}
%   \begin{syntax}
%     \cs{ifnum}\cs{ReadonlyShipoutCounter}\texttt{=...}
%     \cs{int_use:N} \cs{g_shipout_readonly_int} \texttt{\% expl3 usage}
%   \end{syntax}
%    This integer holds the number of pages shipped out up to now
%    (including the one to be shipped out when inside the output
%    routine). More precisely, it is incremented only after it is
%    clear that a page will be shipped out, i.e., after the
%    \hook{shipout/before} hook (because that might discard the page)!
%    In contrast \hook{shipout/after} sees the incremented value.
%
%    Just like with the \texttt{page} counter its value is
%    only accurate within the output routine. In the body of the
%    document it may be off by one as the output routine is called
%    asynchronously!
%
%    Also important: it \emph{must not} be set, only read. There are
%    no provisions to prevent that restriction, but if you manipulate
%    it, chaos will be the result. To emphasize this fact it is not
%    provided as a \LaTeX{} counter but as a \TeX{} counter (i.e., a
%    command), so \cs{Alph}\verb={=\cs{ReadonlyShipoutCounter}\verb=}=
%    etc, would not work.
%  \end{variable}
%
%  \begin{variable}{totalpages,\g_shipout_totalpages_int}
%   \begin{syntax}
%     \cs{arabic}\texttt{\{totalpages\}}
%     \cs{int_use:N} \cs{g_shipout_totalpage_int} \texttt{\% expl3 usage}
%   \end{syntax}
%    In contrast to \cs{ReadonlyShipoutCounter}, the
%    \texttt{totalpages} counter is a \LaTeX{} counter and incremented
%    for each shipout attempt including those pages that are discarded
%    for one or the other reason. Again \hook{shipout/before} sees
%    the counter before it is incremented. In contrast
%    \hook{shipout/after} sees the incremented value.
%
%    Furthermore, while it is incremented for each page, its value is
%    never used by \LaTeX. It can therefore be freely reset or changed by user
%    code, for example, to additionally count a number of pages that
%    are not build by \LaTeX\ but are added in a later part of the
%    process, e.g., cover pages or picture pages made externally.
%
%    Important: as this is a page-related counter its value is only
%    reliable inside the output routine!
%  \end{variable}
%
%  \begin{variable}{\PreviousTotalPages}
%   \begin{syntax}
%     \cs{thetotalpages}/\cs{PreviousTotalPages}
%   \end{syntax}
%    Command that expands to the number of total pages from the
%    previous run.  If there was no previous run or if used in the
%    preamble it expands to
%    \texttt{0}. Note that this is a command and not a counter, so in order
%    to display the number in, say, Roman numerals you have to assign
%    its value to a counter and then use \cs{Roman} on that counter.
%  \end{variable}
%
%
% \subsection{Debugging shipout code}
%
% \begin{function}{\DebugShipoutsOn,\DebugShipoutsOff,
%         \shipout_debug_on:,\shipout_debug_off:}
%   \begin{syntax}
%     \cs{DebugShipoutsOn}
%   \end{syntax}
%    Turn the debugging of shipout code on or off. This displays
%    changes made to the shipout data structures.  \fmi{This needs
%    some rationalizing and may not stay this way.}
% \end{function}
%
%
%
%
% \section{Emulating commands from other packages}
%
%    The packages in this section are no longer necessary, but as they
%    are used by other packages, they are emulated when they are
%    explicitly loaded with \cs{usepackage} or \cs{RequirePackage}.
%
%    Please note that the emulation only happens if the package is
%    explicitly requested, i.e., the commands documented below are not
%    automatically available in the \LaTeX{} kernel!  If you write a
%    new package we suggest to use the appropriate kernel hooks
%    directly instead of loading the emulation.
%
%
% \subsection{Emulating \pkg{atbegshi}}
%
%
% \begin{function}{\AtBeginShipoutUpperLeft,\AtBeginShipoutUpperLeftForeground}
%   \begin{syntax}
%     \cs{AddToHook} \texttt{\{shipout/before\} \{...\cs{AtBeginShipoutUpperLeft}}\Arg{code}\texttt{...\}}
%   \end{syntax}
%    This adds a \texttt{picture} environment into the background of the shipout
%    box expecting \meta{code} to contain \texttt{picture}
%    commands. The same effect can be obtained by simply using kernel
%    features as follows:
%    \begin{quote}
%      \cs{AddToHook}\texttt{\{shipout/background\}}\Arg{code}
%    \end{quote}
%    There is one technical difference: if
%    \cs{AtBeginShipoutUpperLeft} is used several times each
%    invocation is put into its own box inside the shipout box whereas
%    all \meta{code} going into \hook{shipout/background} ends up
%    all in the same box in the order it is added or sorted based on
%    the rules for the hook chunks.
%
%    \cs{AtBeginShipoutUpperLeftForeground} is similar with the
%    difference that the \texttt{picture} environment is placed in the
%    foreground. To model it with the kernel functions use the hook
%    \hook{shipout/foreground} instead.
% \end{function}
%
%
% \begin{function}{\AtBeginShipoutAddToBox,\AtBeginShipoutAddToBoxForeground}
%   \begin{syntax}
%     \cs{AddToHook} \texttt{\{shipout/before\} \{...\cs{AtBeginShipoutAddToBox}}\Arg{code}\texttt{...\}}
%   \end{syntax}
%    These work like \cs{AtBeginShipoutUpperLeft} and
%    \cs{AtBeginShipoutUpperLeftForeground} with the difference that
%    \meta{code} is directly placed into an \cs{hbox} inside the
%    shipout box and not surrounded by a \texttt{picture} environment.
%
%    To emulate them using \hook{shipout/background} or
%    \hook{shipout/foreground} you may have to wrap \meta{code} into
%    a \cs{put} statement but if the code is not doing any typesetting
%    just adding it to the hook should be sufficient.
% \end{function}
%
%
%
% \begin{function}{\AtBeginShipoutBox}
%    This is the name of the shipout box as \pkg{atbegshi} knows it.
% \end{function}
%
%
% \begin{function}{\AtBeginShipoutOriginalShipout}
%    This is the name of the \cs{shipout} primitive as \pkg{atbegshi}
%    knows it. This bypasses all the mechanisms set up by the \LaTeX{}
%    kernel and there are various scenarios in which it can therefore
%    fail. It should only be used to run existing legacy
%    \pkg{atbegshi} code but not in newly developed applications.
%
%    The kernel alternative is \cs{RawShipout} which is integrated
%    with the \LaTeX{} mechanisms and updates, for example, the
%    \cs{ReadonlyShipoutCounter} counter. Please use \cs{RawShipout}
%    for new code if you want to bypass the before, foreground and
%    background hooks.
% \end{function}
%
% \begin{function}{\AtBeginShipoutInit}
%   By default \pkg{atbegshi} delayed its action until
%    \verb=\begin{document}=.  This command was forcing it in an earlier
%    place. With the new concept it does nothing.
% \end{function}
%
% \begin{function}{\AtBeginShipout,\AtBeginShipoutNext}
%   \begin{syntax}
% \cs{AtBeginShipout}\Arg{code} $\equiv$ \cs{AddToHook}\texttt{\{shipout/before\}}\Arg{code}
% \cs{AtBeginShipoutNext}\Arg{code} $\equiv$ \cs{AddToHookNext}\texttt{\{shipout/before\}}\Arg{code}
%   \end{syntax}
%   This is equivalent to filling the \hook{shipout/before} hook
%    by  either using \cs{AddToHook} or \cs{AddToHookNext}, respectively.
% \end{function}
%
% \begin{function}{\AtBeginShipoutFirst,\AtBeginShipoutDiscard}
%   The \pkg{atbegshi} names for \cs{AtBeginDvi} and \cs{DiscardShipoutBox}.
% \end{function}
%
%
%
%
% \subsection{Emulating \pkg{everyshi}}
%
%    The \pkg{everyshi} package is providing commands to run arbitrary
%    code just before the shipout starts.
%    One point of difference: in the new shipout hooks the page is
%    available as \cs{ShipoutBox} for inspection of change, one should
%    not manipulate box 255 directly inside \hook{shipout/before}, so
%    old code doing this would change to use \cs{ShipoutBox} instead
%    of \texttt{255} or \cs{@cclv}.
%
% \begin{function}{\EveryShipout}
%   \begin{syntax}
% \cs{EveryShipout}\Arg{code} $\equiv$ \cs{AddToHook}\texttt{\{shipout/before\}}\Arg{code}
%   \end{syntax}
% \end{function}
%
% \begin{function}{\AtNextShipout}
%   \begin{syntax}
% \cs{AtNextShipout}\Arg{code} $\equiv$ \cs{AddToHookNext}\texttt{\{shipout/before\}}\Arg{code}
%   \end{syntax}
% \end{function}
%
%    However, most use cases for \pkg{everyshi} are attempts to put
%    some picture or text into the background or foreground of the page
%    and that can be done today simply by using the
%    \hook{shipout/background} and 
%    \hook{shipout/foreground} hooks without any need to coding.
%
%
% \subsection{Emulating \pkg{atenddvi}}
%
%    The \pkg{atenddvi} package implemented only a single command:
%    \cs{AtEndDvi} and that is now available out of the box so the
%    emulation makes the package a no-op.
%
%
%
% \subsection{Emulating \pkg{everypage}}
%
%    This package patched the original \cs{@begindvi} hook and replaced 
%    it with its own version.
%    Its functionality is now covered by the hooks offered by the
%    kernel so that there is no need for such patching any longer.
%
% \begin{function}{\AddEverypageHook}
%   \begin{syntax}
% \cs{AddEverypageHook}\Arg{code} $\equiv$
% \qquad\cs{AddToHook}\texttt{\{shipout/background\}\{\cs{put}(1in,-1in)\Arg{code}\}}
%   \end{syntax}
%    \cs{AddEverypageHook} is adding something into the
%    background of every page at a position of 1in to the right and
%    1in down from the top left corner of the page.
%     By using the kernel hook directly you can put your material
%    directly to the right place, i.e., use other coordinates in the
%    \cs{put} statement above.
% \end{function}
%
% \begin{function}{\AddThispageHook}
%   \begin{syntax}
% \cs{AddThispageHook}\Arg{code} $\equiv$
% \qquad\cs{AddToHookNext}\texttt{\{shipout/background\}\{\cs{put}(1in,-1in)\Arg{code}\}}
%   \end{syntax}
%    The \cs{AddThispageHook} wrapper is similar but uses
%    \cs{AddToHookNext}.
% \end{function}
%
%
% \MaybeStop{\setlength\IndexMin{200pt}  \PrintIndex  }
%
%
% \section{The Implementation}
%    \begin{macrocode}
%<@@=shipout>
%    \end{macrocode}
%    
%    At the moment the whole module rolls back in one go, but if we
%    make any modifications in later releases this will then need
%    splitting.
%    \begin{macrocode}
%<*2ekernel|latexrelease>
%<latexrelease>\IncludeInRelease{2020/10/01}%
%<latexrelease>                 {\shipout}{Hook management (shipout)}%
%    \end{macrocode}
%
%
%    \begin{macrocode}
\ExplSyntaxOn
%    \end{macrocode}
%
%
%  \subsection{Debugging}
%
%  \begin{macro}{\g_@@_debug_bool}
%    Holds the current debugging state.
%    \begin{macrocode}
\bool_new:N \g_@@_debug_bool
%    \end{macrocode}
%  \end{macro}
%
%  \begin{macro}{\shipout_debug_on:,\shipout_debug_off:}
%  \begin{macro}{\@@_debug:n}
%  \begin{macro}{\@@_debug_gset:}
%    Turns debugging on and off by redefining \cs{@@_debug:n}.
%    \begin{macrocode}
\cs_new_eq:NN \@@_debug:n  \use_none:n
\cs_new_protected:Npn \shipout_debug_on:
  {
    \bool_gset_true:N \g_@@_debug_bool
    \@@_debug_gset:
  }
\cs_new_protected:Npn \shipout_debug_off:
  {
    \bool_gset_false:N \g_@@_debug_bool
    \@@_debug_gset:
  }
\cs_new_protected:Npn \@@_debug_gset:
  {
    \cs_gset_protected:Npx \@@_debug:n ##1
      { \bool_if:NT \g_@@_debug_bool {##1} }
  }
%    \end{macrocode}
%  \end{macro}
%  \end{macro}
%  \end{macro}
%
%
%
%
%
%  \begin{macro}{\ShipoutBox,\l_shipout_box}
%    The box filled with the page to be shipped out (both L3 and
%       \LaTeXe{} name).
%    \begin{macrocode}
\box_new:N  \l_shipout_box
%    \end{macrocode}
%    
%    \begin{macrocode}
\cs_set_eq:NN \ShipoutBox \l_shipout_box
%    \end{macrocode}
%  \end{macro}
%
%
%  \begin{macro}{\l_@@_raw_box}
%    The \cs{RawShipout} gets its own box but it is internal as there
%    is no hook manipulation for it.
%    \begin{macrocode}
\box_new:N  \l_@@_raw_box
%    \end{macrocode}
%  \end{macro}
%
%
%  \begin{macro}{\@@_finalize_box:}
%    For Lua\TeX{} invoke the \texttt{pre\_shipout\_filter} callback.
%  \changes{v1.0i}{2021/01/22}{Add \texttt{pre\_shipout\_filter} Lua callback}
%    \begin{macrocode}
\sys_if_engine_luatex:TF
  {
    \newprotectedluacmd \@@_finalize_box:
    \exp_args:Nx \everyjob {
      \exp_not:V \everyjob
      \exp_not:N \lua_now:n {
        luatexbase.create_callback('pre_shipout_filter', 'list')
        local~call, getbox, setbox = luatexbase.call_callback, tex.getbox, tex.setbox~
        lua.get_functions_table()[\the \allocationnumber] = function()
          local~head = getbox(\the \l_shipout_box)
          local~result = call('pre_shipout_filter', head)
          if~not (result == head) then~
            setbox(\the \l_shipout_box, result~or~nil)
          end~
        end
      }
    }
  } {
    \cs_set_eq:NN \@@_finalize_box: \scan_stop:
  }
%    \end{macrocode}
%  \end{macro}
%
%
%  \begin{macro}{\@@_execute:}
%    This is going to the be the code run by \cs{shipout}. The code
%    follows closely the  ideas from \pkg{atbegshi}, so not
%    documenting that here for now.
%    \begin{macrocode}
\cs_set_protected:Npn \@@_execute: {
  \tl_set:Nx \l_@@_group_level_tl
     { \int_value:w \tex_currentgrouplevel:D }
  \tex_afterassignment:D \@@_execute_test_level:
  \tex_setbox:D \l_shipout_box
}
%    \end{macrocode}
%  \end{macro}
%
%
%  \begin{macro}{\shipout}
%    Overloading the \cs{shipout} primitive:
%    \begin{macrocode}
\cs_gset_eq:NN \shipout \@@_execute:
%    \end{macrocode}
%  \end{macro}
%
%
%  \begin{macro}{\l_@@_group_level_tl}
%    Helper token list to record the group level at which
%    \cs{@@_execute:} is encountered.
%    \begin{macrocode}
\tl_new:N \l_@@_group_level_tl
%    \end{macrocode}
%  \end{macro}
%
%
%  \begin{macro}{\@@_execute_test_level:}
%    If the group level has changed then we are still constructing
%    \cs{l_shipout_box} and to continue we need to wait until the
%    current group has finished, hence the \cs{tex_aftergroup:D}.
%    \begin{macrocode}
\cs_new:Npn \@@_execute_test_level: {
  \int_compare:nNnT
     \l_@@_group_level_tl < \tex_currentgrouplevel:D 
     \tex_aftergroup:D \@@_execute_cont:
}
%    \end{macrocode}
%  \end{macro}
%
%
%  \begin{macro}{\@@_execute_cont:}
%    This does the actual shipout running several hooks as part of it.
%    The code for them is passed as argument \verb=#2= to \verb=#4=
%    to \cs{@@_execute_main_cont:Nnnn}; the first argument is the box
%    to be shipped out.
%    \begin{macrocode}
\cs_new:Npn \@@_execute_cont: {
  \@@_execute_main_cont:Nnnn
     \l_shipout_box
     { \hook_use:n {shipout/before} }
     { \hook_if_empty:nF {shipout/foreground}
         { \@@_add_foreground_picture:n
             { \hook_use:n {shipout/foreground} } }
%    \end{macrocode}
%    If the user hook for the background (\hook{shipout/background}) has
%    no code, there might still code in the kernel hook so we need to
%    test for this too.
%    We only test for the \cs{@kernel@before@shipout@background}
%    though. If the  \cs{@kernel@after@shipout@background} needs
%    executing even if the user hook is empty then we can add another
%    test (or the kernel could put something into the before hook).
%                   
% \changes{v1.0d}{2020/11/23}{Check for both kernel and user hook (gh/431)}
% \changes{v1.0f}{2021/01/08}{Added another kernel hook for more
%    flexibility (cf.\ \texttt{https://github.com/pgf-tikz/pgf/issues/960}}
%    \begin{macrocode}
       \bool_lazy_and:nnF
         { \hook_if_empty_p:n {shipout/background} }
         { \tl_if_empty_p:N \@kernel@before@shipout@background }
         { \@@_add_background_picture:n
           { \@kernel@before@shipout@background
             \hook_use:n {shipout/background}
             \@kernel@after@shipout@background }
         }
     }
     { \hook_use:n {shipout/after} }
}
%    \end{macrocode}
%  \end{macro}
%
%
%  \begin{macro}{\@@_execute_main_cont:Nnnn}
%    When we have reached this point the shipout box has been
%    processed and is available in \cs{l_shipout_box} and ready for
%    real ship out (unless it gets discarded during the process).
%
%    The three arguments hold hook code that is executed just before the
%    actual shipout (\verb=#1=), within the shipout adding
%    background and foreground material (\verb=#2=) and after the
%    shipout has happened (\verb=#3=).
%    These are passed as arguments because the same code without those
%    hooks is also used when doing a ``raw'' shipout implemented by
%    \cs{RawShipout}.
%    The only hook that is always executed is that for the very last
%    page, i.e., \hook{shipout/lastpage}.
%
%    First we quickly check if it is void (can't happen in the
%    standard \LaTeX{} output routine but \cs{shipout} might be called
%    from a package that has some special processing logic). If it is
%    void we aren't shipping anything out and processing ends.\footnote{In that
%    case we don't reset the deadcycles, that would be up to the OR
%    processing logic to do.}
%    \begin{macrocode}
\cs_new:Npn \@@_execute_main_cont:Nnnn #1#2#3#4 {
  \box_if_empty:NTF #1
    { \@latex@warning@no@line{ Ignoring~ void~ shipout~ box } }
    {
%    \end{macrocode}
%    Otherwise we assume that we will ship something and prepare for
%    final adjustments (in particular setting the state of
%    \cs{protect} while we are running the hook code).
%    We also save the current \cs{protect} state to restore it later.
%    \begin{macrocode}
%      \bool_gset_false:N \g_@@_discard_bool  % setting this would disable
                                              % \DiscardShipoutBox on doc-level
      \cs_set_eq:NN \@@_saved_protect: \protect
      \set@typeset@protect
%    \end{macrocode}
%    We also store the current shipout box dimension in registers, so that
%    they can be used in the hook code.\footnote{This is not really
%    necessary as the code could access them via \cs{box_ht:N}, etc.,
%    but it is perhaps convenient.}
%    \begin{macrocode}
      \@@_get_box_size:N #1
%    \end{macrocode}
%    Then we execute the \hook{shipout/before} hook (or nothing in
%    case of \cs{RawShipout}).
%    \begin{macrocode}
      #2
%    \end{macrocode}
%    In \cs{g_shipout_totalpages_int} we count all shipout attempts so
%    we increment that counter already here (the other one is
%    incremented later when we know for sure that we do a
%    \cs{shipout}.
%
%     We increment it after running the above hook so that the values
%    for \cs{g_shipout_totalpages_int} and \cs{g_shipout_readonly_int}
%    are in sync while the
%    hook is executed (in the case that totalpages isn't manually
%    altered or through discarding pages that is).
%    \begin{macrocode}
      \int_gincr:N \g_shipout_totalpages_int
%    \end{macrocode}
%    The above hook might contain code that requests the page to be discarded so
%    we now test for it.
%    \begin{macrocode}
      \bool_if:NTF \g_@@_discard_bool
        { \@latex@info@no@line{Completed~ page~ discarded}
          \bool_gset_false:N \g_@@_discard_bool
%    \end{macrocode}
%    As we are discarding the page box and not shipping anything out,
%    we need to do some house cleaning and reset \TeX's deadcycles so
%    that it doesn't complain about too many calls to the OR without
%    any shipout.
%    \begin{macrocode}
          \tex_deadcycles:D \c_zero_int
%    \end{macrocode}
%    \fmi{In \pkg{atbegshi} the box was dropped but is that actually
%    needed? Or the resetting of \cs{protect} to its kernel value?}
%    \begin{macrocode}
%          \group_begin:
%            \box_set_eq_drop:NN #1 #1
%          \group_end:
%          \cs_set_eq:NN \protect \exp_not:N
        }
%    \end{macrocode}
%    Even if there was no explicit request to discard the box it is
%    possible that the code for the hook \hook{shipout/before} has
%    voided the box (by mistake or deliberately). We therefore test
%    once more but this time make it a warning, because the best
%    practice way is to use the request mechanism.
%    \begin{macrocode}
        { \box_if_empty:NTF #1
            { \@latex@warning@no@line { Ignoring~ void~ shipout~ box.
                 \MessageBreak The~ shipout~ box~ was~ voided~ by~ hook~ code }
            }
%    \end{macrocode}
%    Finally, if the box is still non-empty we are nearly ready to
%    ship it out.
%    First we increment the total page counter so that we can later
%    test if we have reached the final page according to our available
%    information.\footnote{Doing that earlier would be wrong because we
%    might end up with the last page counted but discard and then we
%    have no place to add the final objects into the output file.}
%    \begin{macrocode}
            {
              \int_gincr:N \g_shipout_readonly_int
              \@@_debug:n {
                \typeout{Absolute~ page~ =~ \int_use:N \g_shipout_readonly_int
                         \space (target:~ \@abspage@last)}
              }
%    \end{macrocode}
%    Then we store the box sizes again (as they may have
%    changed) and then look at the hooks \hook{shipout/foreground}
%    and \hook{shipout/background}. If either or both are non-empty
%    we add a \texttt{picture} environment to the box (in the
%    foreground and/or in the background) and execute the hook code
%    inside that environment.
%    
%    \begin{macrocode}
              \@@_get_box_size:N #1
%    \end{macrocode}
%    The first hook we run is the \hook{shipout/firstpage} hook. This
%    is only done once, then the \cs{@@_run_firstpage_hook:}
%    command redefines itself to do nothing. If the hook contains
%    \cs{special}s for integration at the top of the page they will be
%    temporarily stored in a safe place and added later with
%    \cs{@@_add_firstpage_specials:}.
%    \begin{macrocode}
              \@@_run_firstpage_hook:
%    \end{macrocode}
%    Run the hooks for background and foreground or, if this
%    is called by \cs{RawShipout}, copy the box \cs{l_@@_raw_box} to
%    \cs{l_shipout_box} so that firstpage and lastpage material gets
%    added if necessary (that is always done to \cs{l_shipout_box}.
%    \begin{macrocode}
              #3
%    \end{macrocode}
%    We then run \cs{@@_add_firstpage_specials:} that adds
%    the content of the hook \hook{shipout/firstpage} to the
%    start of the first page (if non-empty). It is then redefined to
%    do nothing on later pages.
%    \begin{macrocode}
              \@@_add_firstpage_specials:
%    \end{macrocode}
%    Then we check if we have to add the \hook{shipout/lastpage} hook
%    or the corresponding kernel hook
%    because we have reached the last page. This test will be false for
%    all but one (and hopefully the correct) page.
% \changes{v1.0d}{2020/11/23}{Check for both kernel and user hook (gh/431)}
%    \begin{macrocode}
              \int_compare:nNnT \@abspage@last = \g_shipout_readonly_int
                { \bool_lazy_and:nnF
                    { \hook_if_empty_p:n {shipout/lastpage} }
                    { \tl_if_empty_p:N \@kernel@after@shipout@lastpage }
                    { \@@_debug:n { \typeout{Executing~ lastpage~ hook~
                          on~ page~ \int_use:N \g_shipout_readonly_int } }
                      \_@@_add_foreground_box:n
                          { \UseHook{shipout/lastpage}
                            \@kernel@after@shipout@lastpage }
%    \end{macrocode}
%    We record that we have handled the \hook{shipout/lastpage} hook
%    but only if we really did.
% \changes{v1.0m}{2022/05/08}{Handle case where shipout/lastpage is
%    run too early (gh/813)}
%    \begin{macrocode}
                      \bool_gset_true:N \g_@@_lastpage_handled_bool
                    }
                }
%    \end{macrocode}
%    
%  \changes{v1.0n}{2022/11/08}{Add shipout hook (gh/920)}
%    \begin{macrocode}
              \hook_use:n {shipout}
              \@@_finalize_box:
%    \end{macrocode}
%    Finally we run the actual \TeX{} primitive for shipout. As that will
%    expand delayed \cs{write} statements inside the page in which
%    protected commands should not expand we first change \cs{protect}
%    to the appropriate definition for that case.
%    \begin{macrocode}
              \cs_set_eq:NN \protect \exp_not:N
              \tex_shipout:D \box_use:N \l_shipout_box
%    \end{macrocode}
%    The \cs{l_shipout_box} may contain the firstpage material if this
%    was the very first shipout. That makes it unsuitable for reuse in
%    another shipout, so as a safety measure the next command resets
%    \cs{l_shipout_box} to its earlier state if that is necessary. On
%    later pages this is then a no-op.
%    \begin{macrocode}
              \@@_drop_firstpage_specials:
%    \end{macrocode}
%    The \hook{shipout/after} hook (if in \verb=#4=) needs to run with
%    \cs{protect}ed commands again being executed, because that hook
%    will ``typeset'' material added at the top of the next page.
%    \begin{macrocode}
              \set@typeset@protect
              #4
            }
        }
%    \end{macrocode}
%    Restore the value of \cs{protect} in case \cs{shipout} is called
%    outside of the output routine (where it is automatically restored
%    because of the implicit group).
%    \begin{macrocode}
      \cs_set_eq:NN \protect \@@_saved_protect:
    }
}
%    \end{macrocode}
%  \end{macro}
%
%
%
%
%  \begin{macro}{\@@_execute_raw:,
%                \@@_execute_test_level_raw:}
%
%    This implements the ``raw'' shipout which bypasses the before,
%    foreground, background and after hooks. It follows the same pattern than
%    \cs{_@@_execute_raw:} except that it finally calls
%    \cs{_@@_execute_main_cont:Nnnn} with three empty arguments.
%    instead of the hook code.
%    \begin{macrocode}
\cs_set_protected:Npn \@@_execute_raw: {
  \tl_set:Nx \l_@@_group_level_tl
     { \int_value:w \tex_currentgrouplevel:D }
  \tex_afterassignment:D \@@_execute_test_level_raw:
  \tex_setbox:D \l_@@_raw_box
}
%    \end{macrocode}
%    
%    \begin{macrocode}
\cs_new:Npn \@@_execute_test_level_raw: {
  \int_compare:nNnT
     \l_@@_group_level_tl < \tex_currentgrouplevel:D
     \tex_aftergroup:D \@@_execute_nohooks_cont:
}
%    \end{macrocode}
%    Well, not totally empty arguments, we add some debugging if we
%    are actually doing a shipout.
%    \begin{macrocode}
\cs_new:Npn \@@_execute_nohooks_cont: {
  \@@_execute_main_cont:Nnnn \l_@@_raw_box
     {} { \@@_debug:n{ \typeout{Doing~ raw~ shipout~ ...} }
          \box_set_eq:NN \l_shipout_box \l_@@_raw_box } {} 
} 
%    \end{macrocode}
%  \end{macro}

%  \begin{macro}{\RawShipout}
%    The interface name for raw shipout.
% \changes{v1.0g}{2021/01/10}{Macro added}
%    \begin{macrocode}
\cs_gset_eq:NN \RawShipout \@@_execute_raw:
%    \end{macrocode}
%  \end{macro}
%
%
%
%
%  \begin{macro}{\@@_saved_protect:}
%    Remember the current \cs{protect} state.
%    \begin{macrocode}
\cs_new_eq:NN  \@@_saved_protect: \protect
%    \end{macrocode}
%  \end{macro}
%
%  \begin{macro}{shipout/before,shipout,shipout/after,
%                   shipout/foreground,shipout/background,
%                   shipout/firstpage,shipout/lastpage}
%    Declaring all hooks for the shipout code.
%  \changes{v1.0n}{2022/11/08}{Add shipout hook (gh/920)}
%    \begin{macrocode}
\hook_new:n{shipout/before}
\hook_new:n{shipout}
\hook_new:n{shipout/after}
\hook_new:n{shipout/foreground}
\hook_new:n{shipout/background}
\hook_new:n{shipout/firstpage}
\hook_new:n{shipout/lastpage}
%    \end{macrocode}
%  \end{macro}
%
%
%
%  \begin{macro}{\@kernel@after@shipout@lastpage,
%                \@kernel@before@shipout@background,
%                \@kernel@after@shipout@background}
%    And here are the internal kernel hooks going before or after the
%    public ones where needed.
% \changes{v1.0g}{2021/01/10}{Internal hook
%                             \cs{@kernel@after@shipout@background} added}
%    \begin{macrocode}
\let\@kernel@after@shipout@lastpage\@empty
\let\@kernel@before@shipout@background\@empty
\let\@kernel@after@shipout@background\@empty
%    \end{macrocode}
%  \end{macro}
%
%
%  \begin{macro}{\@@_run_firstpage_hook:}
%
% \changes{v1.0h}{2021/01/19}{Handling of firstpage hook altered}
%
%    There are three commands to handle the \hook{shipout/firstpage}
%    hook:
%    \cs{@@_run_firstpage_hook:}, \cs{@@_add_firstpage_specials:} and
%    \cs{@@_drop_firstpage_specials:}. 
%
%    That hook is supposed to contain \cs{special}s and similar
%    material to be placed at the very beginning of the output page
%    and so it needs careful placing to avoid that anything else gets
%    in front of it. And this means we have to wait with this until
%    other hooks such as \hook{shipout/background} have added their
%    bits. It is also important that such \cs{special}s show up only
%    on the very first page, so if this page gets saved before
%    \cs{shipout} for later reuse, we have to make sure that they
%    aren't in the saved version.
%
%    In addition the hook may also contain code to be executed ``first'', e.g.,
%    visible from code in \hook{shipout/background} and this conflicts
%    with adding the \cs{special}s late.
%
%    Therefore the processing is split into different parts:
%    \cs{@@_run_firstpage_hook:} is done early and checks if there is
%    any material in the hook. 
%    \begin{macrocode}
\cs_new:Npn \@@_run_firstpage_hook: {
  \hook_if_empty:nTF {shipout/firstpage}
%    \end{macrocode}
%    If not then we define the other two commands to do nothing.
%    \begin{macrocode}
       {
         \cs_gset_eq:NN \@@_add_firstpage_specials:  \prg_do_nothing:
         \cs_gset_eq:NN \@@_drop_firstpage_specials: \prg_do_nothing:
       }
%    \end{macrocode}
%    If there is material we execute inside a box, which means any
%    \cs{special} will end up in that box and any other code is
%    executed and can have side effects (as long as they are global).
%    \begin{macrocode}
       {
         \hbox_set:Nn \l_@@_firstpage_box { \UseHook{shipout/firstpage} }
       }
%    \end{macrocode}
%    Once we are here we change the definition to do nothing next time
%    and we also change the command used to implement \cs{AtBeginDvi}
%    to become a warning and not add further material to a hook that
%    is never used again.
%    \begin{macrocode}
  \cs_gset_eq:NN \@@_run_firstpage_hook: \prg_do_nothing:
  \cs_gset:Npn \@@_add_firstpage_material:Nn ##1 ##2 {
    \@latex@warning{ First~ page~ is~ already~ shipped~ out,~ ignoring
                     \MessageBreak \string##1 }
  }
}
%    \end{macrocode}
% \end{macro}
%
%
%
%  \begin{macro}{\@@_add_firstpage_specials:,\@@_drop_firstpage_specials:}
%    The \cs{@@_add_firstpage_specials:} then adds the \cs{special}s
%    stored in \cs{l_@@_firstpage_box} to the page to be shipped out
%    when the time is ready. Note that if there was no material in the
%    \hook{shipout/firstpage} hook then this command gets redefined to
%    do nothing. But for most documents there is something, e.g., some
%    PostScript header, or some meta data declaration, etc.\ so by 
%    default we assume there is something to do.
%    \begin{macrocode}
\cs_new:Npn \@@_add_firstpage_specials: {
%    \end{macrocode}
%    First we make a copy of the \cs{l_shipout_box} that we can
%    restore it later on.
%    \begin{macrocode}
  \box_set_eq:NN \l_@@_raw_box \l_shipout_box
%    \end{macrocode}
%    Adding something to the beginning means adding it to the
%    background as that layer is done first in the output. 
%    \begin{macrocode}
  \@@_add_background_box:n { \hbox_unpack_drop:N \l_@@_firstpage_box }
%    \end{macrocode}
%    After the actual shipout \cs{@@_drop_firstpage_specials:} is
%    run to
%    restore the earlier content of \cs{l_shipout_box} and then
%    redefines itself again to do nothing.
%
%    As a final act we change the definition to do nothing next time.
%    \begin{macrocode}
  \cs_gset_eq:NN \@@_add_firstpage_specials: \prg_do_nothing:
}
%    \end{macrocode}
%
%    The \cs{@@_drop_firstpage_specials:} is run after the shipout has
%    occurred but before the \hook{shipout/afterpage} hook is executed.
%    That is the point where we have to restore the \cs{ShipoutBox} to
%    its state without the \hook{shipout/firstpage} material.
%    \begin{macrocode}
\cs_new:Npn \@@_drop_firstpage_specials: {
    \box_set_eq:NN \l_shipout_box \l_@@_raw_box
%    \end{macrocode}
%    If there was no such material then \cs{@@_run_firstpage_hook:}
%    will have changed the definition to a no-op already. Otherwise
%    this is what we do here.
%    \begin{macrocode}
    \cs_gset_eq:NN \@@_drop_firstpage_specials:  \prg_do_nothing:
  }
%    \end{macrocode}
%  \end{macro}
%

%  \begin{macro}{\l_@@_firstpage_box}
%    The box to hold any firstpage \cs{special}s.
%    \begin{macrocode}
\box_new:N \l_@@_firstpage_box
%    \end{macrocode}
%  \end{macro}

%
%  \begin{macro}{\g_@@_lastpage_handled_bool}
%    A boolean to signal if we have already handled the
%    \hook{shipout/lastpage} hook.
%    \begin{macrocode}
\bool_new:N \g_@@_lastpage_handled_bool
%    \end{macrocode}
%  \end{macro}
%
%
%
%  \begin{macro}{\@@_add_firstpage_material:Nn}
%    This command adds material to the
%    \hook{shipout/firstpage} hook. It is used in
%    \cs{AtBeginDvi}, etc. The first argument is the
%    command through which is it called. Initially this is ignored but
%    once we are passed the first page it can be used to generate a
%    warning message mentioning the right user command.
%    \begin{macrocode}
\cs_new:Npn \@@_add_firstpage_material:Nn #1#2 {
   \AddToHook{shipout/firstpage}{#2}
}
%    \end{macrocode}
%  \end{macro}
%
%
%
%
%
%
%
%  \begin{macro}{\@@_get_box_size:N}
%    Store the box dimensions in dimen registers.
%    \fmi{This could/should perhaps be generalized to set height depth and
%    width given an arbitrary box.}
%    \begin{macrocode}
\cs_new:Npn \@@_get_box_size:N #1 {
  \dim_set:Nn \l_shipout_box_ht_dim { \box_ht:N #1 }
  \dim_set:Nn \l_shipout_box_dp_dim { \box_dp:N #1 }
  \dim_set:Nn \l_shipout_box_wd_dim { \box_wd:N #1 }
  \dim_set:Nn \l_shipout_box_ht_plus_dp_dim
      { \l_shipout_box_ht_dim + \l_shipout_box_dp_dim }
}
%    \end{macrocode}
%  \end{macro}
%
%  \begin{macro}{\l_shipout_box_ht_dim,
%                   \l_shipout_box_dp_dim,\l_shipout_box_wd_dim,
%                   \l_shipout_box_ht_plus_dp_dim}
%    And here are the variables set by \cs{@@_get_box_size:N}.
%    \begin{macrocode}
\dim_new:N \l_shipout_box_ht_dim
\dim_new:N \l_shipout_box_dp_dim
\dim_new:N \l_shipout_box_wd_dim
\dim_new:N \l_shipout_box_ht_plus_dp_dim
%    \end{macrocode}
%  \end{macro}
%
%
%
%
%
%  \begin{macro}{\g_@@_discard_bool}
%    Indicate whether or not the current page box should be discarded
%    \begin{macrocode}
\bool_new:N \g_@@_discard_bool
%    \end{macrocode}
%  \end{macro}
%
%
%
%  \begin{macro}{\l_@@_tmp_box,\l_@@_saved_badness_tl}
%    We need a box for the background and foreground material and a
%    token register to remember badness settings as we disable  them
%    during the buildup below.
%    \begin{macrocode}
\box_new:N \l_@@_tmp_box
\tl_new:N  \l_@@_saved_badness_tl
%    \end{macrocode}
%  \end{macro}
%
%
%  \begin{macro}{\@@_add_background_box:n}
%    In standard \LaTeX{} the shipout box is always a \cs{vbox} but
%    here we are allow for other usage as well, in case some package
%    has its own output routine.
%    \begin{macrocode}
\cs_new:Npn \@@_add_background_box:n #1
{ \@@_get_box_size:N \l_shipout_box
%    \end{macrocode}
%    But we start testing for a vertical box as that should be the
%    normal case.
%    \begin{macrocode}
  \box_if_vertical:NTF \l_shipout_box
      {
%    \end{macrocode}
%    Save current values of \cs{vfuzz} and \cs{vbadness} then change
%    them to allow box manipulations without warnings.
%    \begin{macrocode}
        \tl_set:Nx \l_@@_saved_badness_tl
           { \vfuzz=\the\vfuzz\relax
             \vbadness=\the\vbadness\relax }
        \vfuzz=\c_max_dim
        \vbadness=\c_max_int
%    \end{macrocode}
%    Then we reconstruct \cs{l_shipout_box} \ldots
%    \begin{macrocode}
        \vbox_set_to_ht:Nnn \l_shipout_box \l_shipout_box_ht_plus_dp_dim 
             {
%    \end{macrocode}
%    \ldots{} the material in \verb=#1= is placed into a horizontal
%    box with zero dimensions.
%    \begin{macrocode}
               \hbox_set:Nn \l_@@_tmp_box
                    { \l_@@_saved_badness_tl #1 }
               \box_set_wd:Nn \l_@@_tmp_box \c_zero_dim
               \box_set_ht:Nn \l_@@_tmp_box \c_zero_dim
               \box_set_dp:Nn \l_@@_tmp_box \c_zero_dim
%    \end{macrocode}
%    The we typeset that box followed by whatever was in
%    \cs{l_shipout_box} before (unpacked).
%    \begin{macrocode}
               \skip_zero:N \baselineskip
               \skip_zero:N \lineskip
               \skip_zero:N \lineskiplimit
               \box_use:N \l_@@_tmp_box
               \vbox_unpack:N \l_shipout_box
%    \end{macrocode}
%    The \cs{kern} ensures that the box has no depth which is
%    afterwards explicitly corrected.
%    \begin{macrocode}
               \kern \c_zero_dim
             }
        \box_set_ht:Nn \l_shipout_box \l_shipout_box_ht_dim
        \box_set_dp:Nn \l_shipout_box \l_shipout_box_dp_dim 
%    \end{macrocode}
%    \fmi{The whole boxing maneuver looks a bit like overkill to me, but for
%    the moment I leave.}               
%    \begin{macrocode}
        \l_@@_saved_badness_tl
      }
      {
%    \end{macrocode}
%    A horizontal box is handled in a similar way. The last case would
%    be a void box in which case we do nothing hence the missing
%    \texttt{F} branch. 
%    \begin{macrocode}
        \box_if_horizontal:NT \l_shipout_box
            {
              \tl_set:Nx \l_@@_saved_badness_tl
                 { \hfuzz=\the\hfuzz\relax
                   \hbadness=\the\hbadness\relax }
              \hfuzz=\c_max_dim
              \hbadness=\c_max_int
              \hbox_set_to_wd:Nnn \l_shipout_box \l_shipout_box_wd_dim
                   {
                     \hbox_set:Nn \l_@@_tmp_box
                          { \l_@@_saved_badness_tl #1 }
                     \box_set_wd:Nn \l_@@_tmp_box \c_zero_dim
                     \box_set_ht:Nn \l_@@_tmp_box \c_zero_dim
                     \box_set_dp:Nn \l_@@_tmp_box \c_zero_dim
                     \box_move_up:nn
                         \l_shipout_box_ht_dim 
                         { \box_use:N \l_@@_tmp_box }
                     \hbox_unpack:N \l_shipout_box
                   }
              \l_@@_saved_badness_tl
            }
      }
}
%    \end{macrocode}
%  \end{macro}
%
%
%
%
%  \begin{macro}{\@@_add_foreground_box:n}
%    Foreground boxes are done in the same way, only the order and
%    placement of boxes has to be done differently.
%    \begin{macrocode}
\cs_new:Npn \@@_add_foreground_box:n #1
{
  \box_if_vertical:NTF \l_shipout_box
    {
      \tl_set:Nx \l_@@_saved_badness_tl
         { \vfuzz=\the\vfuzz\relax
           \vbadness=\the\vbadness\relax }
      \vfuzz=\c_max_dim
      \vbadness=\c_max_int
      \vbox_set_to_ht:Nnn \l_shipout_box \l_shipout_box_ht_plus_dp_dim
           {
             \hbox_set:Nn \l_@@_tmp_box
                  { \l_@@_saved_badness_tl #1 }
             \box_set_wd:Nn \l_@@_tmp_box \c_zero_dim
             \box_set_ht:Nn \l_@@_tmp_box \c_zero_dim
             \box_set_dp:Nn \l_@@_tmp_box \c_zero_dim
             \skip_zero:N \baselineskip
             \skip_zero:N \lineskip
             \skip_zero:N \lineskiplimit
             \vbox_unpack:N \l_shipout_box
             \kern -\l_shipout_box_ht_plus_dp_dim
             \box_use:N \l_@@_tmp_box
             \kern  \l_shipout_box_ht_plus_dp_dim
           }
      \l_@@_saved_badness_tl
      \box_set_ht:Nn \l_shipout_box \l_shipout_box_ht_dim
      \box_set_dp:Nn \l_shipout_box \l_shipout_box_dp_dim 
    }
    {
      \box_if_horizontal:NT \l_shipout_box
        {
          \tl_set:Nx \l_@@_saved_badness_tl
            { \hfuzz=\the\hfuzz\relax
              \hbadness=\the\hbadness\relax }
          \hfuzz=\c_max_dim
          \hbadness=\c_max_int
          \hbox_set_to_wd:Nnn \l_shipout_box \l_shipout_box_wd_dim
               {
                 \hbox_unpack:N \l_shipout_box
                 \kern -\box_wd:N \l_shipout_box
                 \hbox_set:Nn \l_@@_tmp_box
                     { \l_@@_saved_badness_tl #1 }
                 \box_set_wd:Nn \l_@@_tmp_box \c_zero_dim
                 \box_set_ht:Nn \l_@@_tmp_box \c_zero_dim
                 \box_set_dp:Nn \l_@@_tmp_box \c_zero_dim
                 \box_move_up:nn { \box_ht:N \l_shipout_box }
                               { \box_use:N \l_@@_tmp_box }
                 \kern \box_wd:N \l_shipout_box
               }%
               \l_@@_saved_badness_tl
        }
    }
}
%    \end{macrocode}
%  \end{macro}
%
%
%
%
%  \begin{macro}{\@@_init_page_origins:,\c_@@_horigin_tl,\c_@@_vorigin_tl}
%    Two constants holding the offset of the top-left with respect to
%    the media box.
%
%    Setting the constants this way is courtesy of Bruno.
%
%    We delay setting the constants to the last possible place as
%    there might be updates in the preamble or even in the
%    \hook{begindocument} hook that affects their setup.
%    \begin{macrocode}
\cs_new:Npn \@@_init_page_origins: {
  \tl_const:Nx \c_@@_horigin_tl
     {
       \cs_if_exist_use:NTF \pdfvariable { horigin }
          { \cs_if_exist_use:NF \pdfhorigin { 1in } }
     }
  \tl_const:Nx \c_@@_vorigin_tl
     {
       \cs_if_exist_use:NTF \pdfvariable { vorigin }
          { \cs_if_exist_use:NF \pdfvorigin { 1in } }
     }
%    \end{macrocode}
%    After the constants have been set there is no need to execute
%    this command again, in fact it would raise an error, so we
%    redefine it to do nothing.
%    \begin{macrocode}
  \cs_gset_eq:NN \@@_init_page_origins: \prg_do_nothing:
}     
%    \end{macrocode}
%  \end{macro}
%
%
%  \begin{macro}{\@@_picture_overlay:n}
%    Put the argument into a \texttt{picture} environment that doesn't take up
%    any size and uses \texttt{1pt} for \cs{unitlength}.
%    \fmi{Could perhaps be generalized as it might be useful elsewhere. For
%    now it is not.}
%    \begin{macrocode}
\cs_new:Npn \@@_picture_overlay:n #1 {
%    \end{macrocode}
%    The very first time this is executed we have to initialize (and
%    freeze) the origins.
%    \begin{macrocode}
    \@@_init_page_origins:
%    \end{macrocode}
%    
%    \begin{macrocode}
    \kern -\c_@@_horigin_tl \scan_stop:
    \vbox_to_zero:n {
      \kern -\c_@@_vorigin_tl \scan_stop:
      \unitlength 1pt \scan_stop:
%    \end{macrocode}
%    This mimics a simple zero-sized picture environment. The \cs{hss}
%    is need in case there is horizontal material (without using
%    \cs{put} with a positive width.
% \changes{v1.0b}{2020/09/09}
%         {Prevent overfull box warnings (gh/387)}
%    \begin{macrocode}
      \hbox_set_to_wd:Nnn \l_@@_tmp_box \c_zero_dim
                          { \ignorespaces #1 \hss }
      \box_set_ht:Nn \l_@@_tmp_box \c_zero_dim
      \box_set_dp:Nn \l_@@_tmp_box \c_zero_dim
      \box_use:N \l_@@_tmp_box
      \tex_vss:D
    }
}
%    \end{macrocode}
%  \end{macro}
%
%
%  \begin{macro}{\@@_add_background_picture:n}
%    Put a \texttt{picture} env in  the background of the shipout box
%    with its reference point in the top-left corner.
%
%    \begin{macrocode}
\cs_new:Npn \@@_add_background_picture:n #1 {
   \@@_add_background_box:n { \@@_picture_overlay:n {#1} }
}
%    \end{macrocode}
%  \end{macro}
%
%
%
%  \begin{macro}{\@@_add_foreground_picture:n}
%    
%    Put a \texttt{picture} env in  the foreground of the shipout box
%    with its reference point in the top-left corner.
%    \begin{macrocode}
\cs_new:Npn \@@_add_foreground_picture:n #1 {
   \@@_add_foreground_box:n { \@@_picture_overlay:n {#1} }
}
%    \end{macrocode}
%  \end{macro}
%
%
%  \begin{macro}{\shipout_discard:}
%    Request that the next shipout box should be discarded. At the
%    moment this is just setting a boolean, but we may want to augment
%    this behavior that the position of the call is taken into account
%    (in case \LaTeX{} looks ahead and is not using the position for
%    on the next page).
%    \begin{macrocode}
\cs_new_protected:Npn \shipout_discard: {
  \bool_gset_true:N \g_@@_discard_bool
}
%    \end{macrocode}
%  \end{macro}
%
%
%
%
% \subsection{Handling the end of job hook}
%
%    At the moment this is partly solved by using the existing hooks.
%    But rather than putting the code into these hooks it should be
%    moved to the right place directly as we shouldn't prefill hooks
%    with material unless it needs to interact with other code. 
%
%
%    
%  \begin{macro}{\g_shipout_readonly_int,\ReadonlyShipoutCounter}
%    We count every shipout activity that makes a page (but not those
%    that are discarded) in order to know how many pages got produced.
%    \begin{macrocode}
\int_new:N \g_shipout_readonly_int
%    \end{macrocode}
%    For \LaTeXe{} it is available as a command (i.e., a \TeX{}
%    counter only.
%    \begin{macrocode}
\cs_new_eq:NN \ReadonlyShipoutCounter  \g_shipout_readonly_int
%    \end{macrocode}
%  \end{macro}
%
%  \begin{macro}{\g_shipout_totalpages_int,\c@totalpages}
%    We count every shipout attempt (even those that are discarded) in
%    this counter. It is not used in the code but may get used in user
%    code.
%    \begin{macrocode}
\int_new:N \g_shipout_totalpages_int
%    \end{macrocode}
%    For \LaTeXe{} this is offered as a \LaTeX{} counter so can be
%    easily typeset inside the output routine to display things like
%    \enquote{\cs{thepage}\texttt{/}\cs{thetotalpages}}, etc.
%    \begin{macrocode}
\cs_new_eq:NN \c@totalpages \g_shipout_totalpages_int
\cs_new:Npn \thetotalpages { \arabic{totalpages} }
%    \end{macrocode}
%  \end{macro}
%
%
%
%
%  \begin{macro}{\@abspage@last}
%    In \cs{@abspage@last} record the number of pages from the last
%    run. This is written to the \texttt{.aux} and this way made
%    available to the next run. In case there is no \texttt{.aux} file
%    or the statement is missing from it we initialize it with the
%    largest possible number in \TeX{}. We use this as the default
%    because then we are inserting the \hook{shipout/lastpage} on
%    the last page (or after the last page) but not on page 1 for a
%    multipage document.
%    \begin{macrocode}
\xdef\@abspage@last{\number\maxdimen}
%    \end{macrocode}
%  \end{macro}
%
%
% \begin{macro}{\enddocument}
%
%    Instead of using the hooks \hook{enddocument} and
%    \hook{enddocument/afterlastpage} we add this code to private
%    kernel hooks to be 100\% sure when it is executed and to avoid
%    cluttering the hooks with data that is always there.
%
%    Inside \cs{enddocument} there is a \cs{clearpage}. Just before
%    that we execute this code here. There is a good chance that we
%    are on the last page. Therefore, if we don't know the value from
%    the last run, we assume that the current page is the right
%    one. So we set \cs{@abspage@last} and as a result the next
%    shipout will run the \hook{shipout/lastpage} code. Of course,
%    if there are floats that still need a placement this guess will
%    be wrong but then rerunning the document will give us the correct
%    value next time around.
%
% \begin{macro}{\@kernel@after@enddocument}
%    \begin{macrocode}
\g@addto@macro \@kernel@after@enddocument {
  \int_compare:nNnT \@abspage@last = \maxdimen
    {
%    \end{macrocode}
%    We use \LaTeXe{} coding as \cs{@abspage@last} is not an L3 name.
%    \begin{macrocode}
      \xdef\@abspage@last{ \int_eval:n {\g_shipout_readonly_int + 1} }
    }
}
%    \end{macrocode}
% \end{macro}
%
% \begin{macro}{\@kernel@after@enddocument@afterlastpage}
%    Once the \cs{clearpage} has done its work inside \cs{enddocument}
%    we know for sure how many pages this document has, so we record
%    that in the \texttt{.aux} file for the next run.
%
%    \begin{macrocode}
\g@addto@macro \@kernel@after@enddocument@afterlastpage {
%    \end{macrocode}
%    There is one special case: If no output is produced then there is
%    no point in a) recording the number as 0 will never match the
%    page number of a real page and b) adding an extra page to ran the
%    \hook{shipout/lastpage} is pointless as well (as it would
%    remain forever). So we test for this and run the code only if
%    there have been pages.
%    \begin{macrocode}
  \int_compare:nNnF \g_shipout_readonly_int = 0
    {
%    \end{macrocode}
%     This ends up in the \texttt{.aux} so we use \LaTeXe{} names here.
%     \fmi{This needs an interface for \cs{nofiles} in expl3, doesn't at the moment!}
%    \begin{macrocode}
     \if@filesw
        \iow_now:Nx \@auxout {
          \gdef\string\@abspage@last {\int_use:N \g_shipout_readonly_int}}
     \fi
%    \end{macrocode}
%    But we may have guessed wrongly earlier and have run it too early
%    or we still have to run the
%    \hook{shipout/lastpage} even though there is no page to place
%    it into. If that is the case we make a trivial extra page and put
%    it there. This temporary page will then vanish again on the next
%    run but helps to keep pdf viewers happy.
%    In either case we should put out an appropriate ``rerun'' warning.
% \changes{v1.0m}{2022/05/08}{Handle case where shipout/lastpage is
%    run too early (gh/813)}
%    \begin{macrocode}
      \bool_if:NTF \g_@@_lastpage_handled_bool
         {
%    \end{macrocode}
%    If the hook was already executed, we have to test if that total
%    shipouts match the shipouts from last run (because that
%    corresponds to the page it was executed). If not we output a warning.
%    \begin{macrocode}
           \int_compare:nNnF \@abspage@last = \g_shipout_readonly_int
            {
              \@latex@warning@no@line{Hook~ 'shipout/lastpage'~ executed~
                on~ wrong~ page~ (\@abspage@last\space not~
                \int_use:N\g_shipout_readonly_int).\MessageBreak
                Rerun~ to~ correct~ this}%
            } 
         }
         {
%    \end{macrocode}
%    If the hook was not run, we need to add an extra page and place
%    it there.
%    However, making this extra page in case the hook is actually
%    empty would be forcing a rerun without any reason, so we check
%    that condition and also check if
%    \cs{@kernel@after@shipout@lastpage} contains any code. If both
%    are empty we omit the page generation.
%    \begin{macrocode}
          \bool_lazy_and:nnF
            { \hook_if_empty_p:n {shipout/lastpage} }
            { \tl_if_empty_p:N \@kernel@after@shipout@lastpage }
            {
              \tex_shipout:D\vbox to\textheight
                {
                  \hbox:n { \UseHook{shipout/lastpage}
                            \@kernel@after@shipout@lastpage }  
%    \end{macrocode}
%    This extra page could be totally empty except for the hook
%    content, but to help the user understanding why it is there we
%    put some text into it.
%    \begin{macrocode}
                  \@@_excuse_extra_page:
                  \null
                }
%    \end{macrocode}
%    At this point we also signal to \LaTeX{}'s endgame that a rerun is
%    necessary so that an appropriate message can be shown on the
%    terminal. We do this by simply defining a command used as a flag and
%    tested in \cs{enddocument}.
%    \begin{macrocode}
              \cs_gset_eq:NN \@extra@page@added \relax
            }
         }
     }
}
%    \end{macrocode}
% \end{macro}
% \end{macro}
%
%
%
%  \begin{macro}{\@@_excuse_extra_page:}
%    Say mea culpa \ldots
%    \begin{macrocode}
\cs_new:Npn \@@_excuse_extra_page: {
  \vfil
  \begin{center}
    \bfseries Temporary~ page! 
  \end{center}
    \LaTeX{}~ was~ unable~ to~ guess~ the~ total~ number~ of~ pages~
    correctly.~ ~ As~ there~ was~ some~ unprocessed~ data~ that~
    should~ have~ been~ added~ to~ the~ final~ page~ this~ extra~
    page~ has~ been~ added~ to~ receive~ it.
    \par
    If~ you~ rerun~ the~ document~ (without~ altering~ it)~ this~
    surplus~ page~ will~ go~ away,~ because~ \LaTeX{}~ now~ knows~
    how~ many~ pages~ to~ expect~ for~ this~ document.
  \vfil
}
%    \end{macrocode}
%  \end{macro}
%
%
%  \begin{macro}{\PreviousTotalPages,\@kernel@before@begindocument}
%    In the preamble before the \texttt{aux} file was read
%    \cs{PreviousTotalPages} is always zero.
%    \begin{macrocode}
\def\PreviousTotalPages{0}
%    \end{macrocode}
%    In the \texttt{aux} file there should be an update for
%    \cs{@abspage@last} recording the number of pages from the
%    previous run. If not that macro holds the value of
%    \cs{maxdimen}. So we test for it and update
%    \cs{PreviousTotalPages} if there was a real value. This should
%    happen just before the \hook{begindocument} hook is executed so
%    that the value can be used inside that hook.
%    \begin{macrocode}
\g@addto@macro\@kernel@before@begindocument
  {\ifnum\@abspage@last<\maxdimen
     \xdef\PreviousTotalPages{\@abspage@last}\fi}
%    \end{macrocode}
%  \end{macro}
%
%
% \section{Legacy \LaTeXe{} interfaces}
%
%
%
%  \begin{macro}{\DiscardShipoutBox}
%    Request that the next shipout box is to be discarded.
%    \begin{macrocode}
\cs_new_eq:NN \DiscardShipoutBox \shipout_discard:
%    \end{macrocode}
%  \end{macro}
%
%
%  \begin{macro}{\AtBeginDvi}
%    If we roll forward from an earlier kernel \cs{AtBeginDvi} is
%    defined so we better not use \cs{cs_new_protected:Npn} here.
% \changes{v1.0d}{2020/11/24}{Support for roll forward (gh/434)}
%    \begin{macrocode}
\cs_set_protected:Npn \AtBeginDvi
                      {\@@_add_firstpage_material:Nn \AtBeginDvi}
%    \end{macrocode}
%  \end{macro}
%
%
%  \begin{macro}{\DebugShipoutsOn,\DebugShipoutsOff}
%    
%    \begin{macrocode}
\cs_new_eq:NN \DebugShipoutsOn  \shipout_debug_on:
\cs_new_eq:NN \DebugShipoutsOff \shipout_debug_off:
%    \end{macrocode}
%  \end{macro}
%
%
% \section{Internal commands needed elsewhere}
%
%    These internal commands use double and triple \texttt{@} signs so
%    we need to stop getting them translated to the module name.
%    \begin{macrocode}
%<@@=>
%    \end{macrocode}
%
%  \begin{macro}{\@expl@@@shipout@add@firstpage@material@@Nn,
%                \@expl@@@shipout@add@background@box@@n,
%                \@expl@@@shipout@add@foreground@box@@n,
%                \@expl@@@shipout@add@background@picture@@n,
%                \@expl@@@shipout@add@foreground@picture@@n}
%    Some internals needed elsewhere.
%
%  \InternalDetectionOff
%    \begin{macrocode}
\cs_set_eq:NN \@expl@@@shipout@add@firstpage@material@@Nn
              \__shipout_add_firstpage_material:Nn
%    \end{macrocode}
%
%    \begin{macrocode}
\cs_set_eq:NN \@expl@@@shipout@add@background@box@@n
              \__shipout_add_background_box:n
%    \end{macrocode}
%
%    \begin{macrocode}
\cs_set_eq:NN \@expl@@@shipout@add@foreground@box@@n
              \__shipout_add_foreground_box:n
%    \end{macrocode}
%
%    \begin{macrocode}
\cs_set_eq:NN \@expl@@@shipout@add@background@picture@@n
              \__shipout_add_background_picture:n
%    \end{macrocode}
%
%    \begin{macrocode}
\cs_set_eq:NN \@expl@@@shipout@add@foreground@picture@@n
              \__shipout_add_foreground_picture:n
%    \end{macrocode}
%  \InternalDetectionOn
%  \end{macro}
%
%    \begin{macrocode}
\ExplSyntaxOff
%</2ekernel|latexrelease>
%<latexrelease>\EndIncludeInRelease
%    \end{macrocode}
%
%
%    Rolling back here doesn't undefine the interface commands as they
%    may be used in packages without rollback functionality. So we
%    just make them do nothing which may or may not work depending on
%    the code usage.
%    
%    \begin{macrocode}
%<latexrelease>\IncludeInRelease{0000/00/00}%
%<latexrelease>                 {\shipout}{Hook management (shipout)}%
%<latexrelease>
%    \end{macrocode}
%    If we roll forward then \cs{tex\_shipout:D} may not be defined in
%    which case \cs{shipout} does have it original definition and so
%    we must not \cs{let} it to something else which is \cs{relax}! 
% \changes{v1.0d}{2020/11/24}{Support for roll forward (gh/434)}
%    \begin{macrocode}
%<latexrelease>\ifcsname tex_shipout:D\endcsname
%<latexrelease>\expandafter\let\expandafter\shipout
%<latexrelease>                \csname tex_shipout:D\endcsname
%<latexrelease>\fi
%<latexrelease>
%<latexrelease>\let \RawShipout\@undefined
%<latexrelease>\let \ShipoutBox\@undefined
%<latexrelease>\let \ReadonlyShipoutCounter \@undefined
%<latexrelease>\let \c@totalpages \@undefined
%<latexrelease>\let \thetotalpages \@undefined
%<latexrelease>
%<latexrelease>\let \DiscardShipoutBox \@undefined
%<latexrelease>\let \DebugShipoutsOn \@undefined
%<latexrelease>\let \DebugShipoutsOff \@undefined
%<latexrelease>
%<latexrelease>\DeclareRobustCommand \AtBeginDvi [1]{%
%<latexrelease>  \global \setbox \@begindvibox
%<latexrelease>    \vbox{\unvbox \@begindvibox #1}%
%<latexrelease>}
%<latexrelease>
%<latexrelease>\let \AtBeginShipout \@undefined
%<latexrelease>\let \AtBeginShipoutNext \@undefined
%<latexrelease>
%<latexrelease>\let \AtBeginShipoutFirst \@undefined
%<latexrelease>
%<latexrelease>\let \ShipoutBoxHeight \@undefined
%<latexrelease>\let \ShipoutBoxDepth \@undefined
%<latexrelease>\let \ShipoutBoxWidth \@undefined
%<latexrelease>
%    \end{macrocode}
%    We do not undo a substitution when rolling back. As the file
%    support gets undone the underlying data is no longer used (and
%    sufficiently obscure that it should not interfere with existing
%    commands) and properly removing it would mean we need to make the
%    \cs{undeclare@...} and its support macros available in all earlier
%    kernel releases which is pointless (and actually worse).
%    
%    \begin{macrocode}
%<latexrelease>
%<latexrelease>\let  \AtEndDvi \@undefined
%    \end{macrocode}
%    We do not reenable a disabled package load when rolling back. As the file
%    support gets undone the underlying data is no longer checked (and
%    sufficiently obscure that it should not interfere with existing
%    commands) and properly removing it would mean we need to make the
%    \cs{reenable@package@load} command available in all earlier
%    kernel releases which is pointless (and actually worse).
%    \begin{macrocode}
%\reenable@package@load{atenddvi}
%    \end{macrocode}
%
%    \begin{macrocode}
%<latexrelease>
%<latexrelease>\EndIncludeInRelease
%<*2ekernel>
%    \end{macrocode}
%
%
%
%
% \section{Package emulation for compatibility}
%
%
% \subsection{Package \pkg{atenddvi} emulation}
%
%
%  \begin{macro}{\AtEndDvi}
%    This package has only one public command, so simulating it is easy
%    and actually sensible to provide as part of the kernel.
% \changes{v1.0l}{2022/01/06}{Correctly simulate \cs{AtEndDvi} without
%    extending the syntax}
%    \begin{macrocode}
%</2ekernel>
%<*2ekernel|latexrelease>
%<latexrelease>\IncludeInRelease{2020/10/01}%
%<latexrelease>                 {\AtEndDvi}{atenddvi emulation}%
\ExplSyntaxOn
\cs_new_protected:Npn \AtEndDvi #1 {\AddToHook{shipout/lastpage}{#1}}
\ExplSyntaxOff
%    \end{macrocode}
%    As the package is integrate we prevent loading (no need to roll that back):
%    \begin{macrocode}
\disable@package@load{atenddvi}
   {\PackageWarning{atenddvi}
     {Functionality of this package is already\MessageBreak
      provided by LaTeX.\MessageBreak\MessageBreak
      It is there no longer necessary to load it\MessageBreak
      and you can safely remove it.\MessageBreak
      Found on}}
%    \end{macrocode}
%
%    \begin{macrocode}
%</2ekernel|latexrelease>
%    \end{macrocode}
%    
%    \begin{macrocode}
%<latexrelease>\EndIncludeInRelease
%<latexrelease>\IncludeInRelease{0000/00/00}%
%<latexrelease>                 {\AtEndDvi}{atenddvi emulation}%
%<latexrelease>\let \AtEndDvi \@undefined
%<latexrelease>\EndIncludeInRelease
%<*2ekernel>
%    \end{macrocode}
%  \end{macro}
%
%    
%
%    \begin{macrocode}
%</2ekernel>
%    \end{macrocode}
%
%
% \subsection{Package \pkg{atbegshi} emulation}
%
%
%
%
%
%    \begin{macrocode}
%<*atbegshi-ltx>
\ProvidesPackage{atbegshi-ltx}
   [2021/01/10 v1.0c
     Emulation of the original atbegshi^^Jpackage with kernel methods]
%    \end{macrocode}
%
%
%  \begin{macro}{\AtBeginShipoutBox}
%    \begin{macrocode}
\let \AtBeginShipoutBox \ShipoutBox
%    \end{macrocode}
%  \end{macro}
%
%
%
%  \begin{macro}{\AtBeginShipoutInit}
%    Compatibility only, we aren't delaying \ldots
%    \begin{macrocode}
\let \AtBeginShipoutInit \@empty 
%    \end{macrocode}
%  \end{macro}
%
%
%  \begin{macro}{\AtBeginShipout,\AtBeginShipoutNext}
%    Filling hooks
% \changes{v1.0l}{2022/01/06}{Correctly simulate \cs{AtBeginShipout}
%    and \cs{AtBeginShipoutNext} without extending the syntax}
%    \begin{macrocode}
\protected\long\def\AtBeginShipout     #1{\AddToHook{shipout/before}{#1}}
\protected\long\def\AtBeginShipoutNext #1{\AddToHookNext{shipout/before}{#1}}
%    \end{macrocode}
%  \end{macro}
%
%
%  \begin{macro}{\AtBeginShipoutFirst}
%    Slightly more complex as we need to know the name of the command under which the
%    \hook{shipout/firstpage} hook is filled.
%    \begin{macrocode}
\protected \def \AtBeginShipoutFirst
   {\@expl@@@shipout@add@firstpage@material@@Nn \AtBeginShipoutFirst}
%    \end{macrocode}
%  \end{macro}
%
%
%  \begin{macro}{\AtBeginShipoutDiscard}
%    Just a different name.
%    \begin{macrocode}
\let \AtBeginShipoutDiscard \DiscardShipoutBox
%    \end{macrocode}
%  \end{macro}
%
%
%  \begin{macro}{\AtBeginShipoutAddToBox,\AtBeginShipoutAddToBoxForeground,
%                \AtBeginShipoutUpperLeft,\AtBeginShipoutUpperLeftForeground}
%    We don't expose them.
%    \begin{macrocode}
\let \AtBeginShipoutAddToBox
              \@expl@@@shipout@add@background@box@@n
\let \AtBeginShipoutAddToBoxForeground
              \@expl@@@shipout@add@foreground@box@@n 
%    \end{macrocode}
%    
%    \begin{macrocode}
\let \AtBeginShipoutUpperLeft
              \@expl@@@shipout@add@background@picture@@n
\let \AtBeginShipoutUpperLeftForeground
              \@expl@@@shipout@add@foreground@picture@@n 
%    \end{macrocode}
%  \end{macro}
%
%  \begin{macro}{\AtBeginShipoutOriginalShipout}
%    This offers the raw \cs{shipout} primitive of the engine. A page
%    shipped out with this is not counted by
%    \cs{ReadonlyShipoutCounter} counter and thus the mechanism to
%    place \cs{special}s at the very end of the output might fail,
%    etc. It should therefore not be used in new applications but is
%    only provided to allow  running legacy code. For new code use the
%    commands provided by the kernel instead.
%    \begin{macrocode}
\ExplSyntaxOn
\cs_new_eq:NN \AtBeginShipoutOriginalShipout \tex_shipout:D
%    \end{macrocode}
%  \end{macro}
%
%
%  \begin{macro}{\ShipoutBoxHeight,\ShipoutBoxWidth,\ShipoutBoxDepth}
%    This is somewhat different from the original in \pkg{atbegshi}
%    where \cs{ShipoutBoxHeight} etc.\ only holds the
%    \verb=\the\ht<box>= value. This may has some implications in some
%    use cases and if that is a problem then it might need changing.
%    \begin{macrocode}
\cs_new:Npn \ShipoutBoxHeight { \dim_use:N \l_shipout_box_ht_dim }
\cs_new:Npn \ShipoutBoxDepth  { \dim_use:N \l_shipout_box_dp_dim }
\cs_new:Npn \ShipoutBoxWidth  { \dim_use:N \l_shipout_box_wd_dim }
\ExplSyntaxOff
%    \end{macrocode}
%  \end{macro}
%
%    
%    \begin{macrocode}
%</atbegshi-ltx>
%    \end{macrocode}
%
%    If the package is requested we substitute the one above:
%    \begin{macrocode}
%<*2ekernel>
\declare@file@substitution{atbegshi.sty}{atbegshi-ltx.sty}
%</2ekernel>
%    \end{macrocode}
%
%
%
%
%
%
% \subsection{Package \pkg{everyshi} emulation}
%
%    This is now directly handled in that package so emulation is not
%    necessary any more.
%
%
%    Rather important :-)
%    \begin{macrocode}
%<@@=>
%    \end{macrocode}
%
%    \Finale
%
%
%
%%%%%%%%%%%%%%%%%%%%%%%%%%%%%%%%%%%%%%%%%%%%%  
\endinput
%%%%%%%%%%%%%%%%%%%%%%%%%%%%%%%%%%%%%%%%%%%%%  

                     % \iffalse meta-comment
%%
%% File: ltfilehook.dtx (C) Copyright 2020 Frank Mittelbach,
%%                                         Phelype Oleinik & LaTeX Team
%
% This file is part of the LaTeX base system.
% -------------------------------------------
%
% It may be distributed and/or modified under the
% conditions of the LaTeX Project Public License, either version 1.3c
% of this license or (at your option) any later version.
% The latest version of this license is in
%    https://www.latex-project.org/lppl.txt
% and version 1.3c or later is part of all distributions of LaTeX
% version 2008 or later.
%
% This file has the LPPL maintenance status "maintained".
%
% The list of all files belonging to the LaTeX base distribution is
% given in the file `manifest.txt'. See also `legal.txt' for additional
% information.
%
% The list of derived (unpacked) files belonging to the distribution
% and covered by LPPL is defined by the unpacking scripts (with
% extension .ins) which are part of the distribution.
%
% \fi
%
% \iffalse
%
%%% From File: ltfilehook.dtx
%
%    \begin{macrocode}
\providecommand\ltfilehookversion{v1.0d}
\providecommand\ltfilehookdate{2020/11/24}
%    \end{macrocode}
%
%<*driver>

\documentclass{l3doc}

% bug fix fo l3doc.cls
\ExplSyntaxOn
\cs_set_protected:Npn \__codedoc_macro_typeset_one:nN #1#2
  {
    \vbox_set:Nn \l__codedoc_macro_box
      {
        \vbox_unpack_drop:N \l__codedoc_macro_box
        \hbox { \llap { \__codedoc_print_macroname:nN {#1} #2
            \MacroFont       % <----- without it the \ is in lmr10 if a link is made
            \      
        } }
      }
    \int_incr:N \l__codedoc_macro_int
  }
\ExplSyntaxOff

\usepackage{structuredlog}  % for demonstration

\EnableCrossrefs
\CodelineIndex
\begin{document}
  \DocInput{ltfilehook.dtx}
\end{document}
%</driver>
%
% \fi
%
%
% \long\def\fmi#1{\begin{quote}\itshape Todo: #1\end{quote}}
%
% \let\hook\texttt
%
% \title{The \texttt{ltfilehook} package\thanks{This package has version
%    \ltfilehookversion\ dated \ltfilehookdate, \copyright\ \LaTeX\
%    Project.}}
%
% \author{Frank Mittelbach}
%
% \maketitle
%
%
%
% \tableofcontents
%
% \section{Introduction}
%
%
%
% \subsection{Provided hooks}
%
%    The code offers a number of hooks into which packages (or the
%    user) can add code to support different use cases.
%    Many hooks are offered as pairs (i.e., the second hook is
%    reversed. Also important to know is that these pairs are
%    properly nested with respect to other pairs of hooks.
%
%    There are hooks that are executed for all files of a certain type
%    (if they contain code), e.g., for all \enquote{include files} or
%    all \enquote{packages},
%    and there are also hooks that are specific to a single file,
%    e.g., do something after the package \texttt{foo.sty} has been
%    loaded.
%
%
% \subsection{General hooks for file reading}
% \label{sec:general-file-hooks}
%
%    There are four hooks that are called for each file that is read
%    using document-level commands such as \cs{input}, \cs{include},
%    \cs{usepackage}, etc.  They are not called for files read using
%    internal low-level methods, such as \cs{@input} or \cs{openin}.
%
%  \begin{variable}{file/before,file/before/...,
%                   file/after/...,file/after,
%                   }
%    These are:
%    \begin{description}
%    \item[\texttt{file/before}, \texttt{file/before/\meta{file-name}}]
%
%       These hooks are executed in that order just before the file is
%       loaded for reading. The code of the first hook is used
%       with every file, while the second is executed only for the
%       file with matching \meta{file-name} allowing you to specify
%       code that only applies to one file.
%
%    \item[\texttt{file/after/\meta{file-name}}, \texttt{file/after}]
%
%       These hooks are after the file with name \meta{file-name} has
%       been fully consumed. The order is swapped (the specific one
%       comes first) so that the \texttt{before} and \texttt{after}
%       hooks nest properly, which is important if any of them involve
%       grouping (e.g., contain environments, for example).
%       Furthermore both hooks are reversed hooks to support correct
%       nesting of different packages adding code to both
%       \texttt{/before} and \texttt{/after} hooks.
%
%    \end{description}
%  \end{variable}
%
%
%     So the overall sequence of hook processing for any file read
%     through the user interface commands of \LaTeX{} is:
%
%  \begin{tabbing}
%    mm\=mm\=mm\=mm\=\kill
%    \>\cs{UseHook}\marg{\hook{file/before}} \\
%    \>\cs{UseHook}\marg{\hook{file/before/\meta{file name}}} \\
%    \>\> \meta{file contents} \\
%    \>\cs{UseHook}\marg{\hook{file/after/\meta{file name}}} \\
%    \>\cs{UseHook}\marg{\hook{file/after}}
%  \end{tabbing}
%
%    The file hooks only refer to the file by its name and extension,
%    so the \meta{file name} should be the file name as it is on the
%    filesystem with extension (if any) and without paths.  Different
%    from \cs{input} and similar commands, the \texttt{.tex}
%    extension is not assumed in hook \meta{file name}, so \texttt{.tex}
%    files must be specified
%    with their extension to be recognized.
%    Files within subfolders should also be addressed by their name and
%    extension only.
%
%    Extensionless files also work, and should then be given without
%    extension.  Note however that \TeX{} prioritizes \texttt{.tex}
%    files, so if two files \texttt{foo} and \texttt{foo.tex} exist in
%    the search path, only the latter will be seen.
%
%    When a file is input, the \meta{file name} is available in
%    \cs{CurrentFile}, which is then used when accessing the
%    \hook{file/before/\meta{file name}} and
%    \hook{file/after/\meta{file name}}.
%
%  \begin{variable}{\CurrentFile}
%    The name of the file about to be read (or just finished) is
%    available to the hooks through \cs{CurrentFile} (there is no
%    \texttt{expl3} name for it for now).  The file is always provided
%    with its extension, i.e., how it appears on your hard drive, but
%    without any specified path to it. For example,
%    \verb=\input{sample}= and \verb=\input{app/sample.tex}= would
%    both have \cs{CurrentFile} being \texttt{sample.tex}.
%  \end{variable}
%
%  \begin{variable}{\CurrentFilePath}
%    The path to the current file (complement to \cs{CurrentFile}) is
%    available in \cs{CurrentFilePath} if needed.
%    The paths returned in \cs{CurrentFilePath} are only user paths,
%    given through \cs{input@path} (or \pkg{expl3}'s equivalent
%    \cs{l_file_search_path_seq}) or by directly typing in the path
%    in the \cs{input} command or equivalent.  Files located by
%    \texttt{kpsewhich} get the path added internally by the \TeX{}
%    implementation, so at the macro level it looks as if the file were
%    in the current folder, so the path in \cs{CurrentFilePath} is empty
%    in these cases (package and class files, mostly).
%  \end{variable}
%
%  \begin{variable}{\CurrentFileUsed,\CurrentFilePathUsed}
%    In normal circumstances these are identical to \cs{CurrentFile} and
%    \cs{CurrentFilePath}.  They will differ when a file substitution
%    has occurred for \cs{CurrentFile}.  In that case,
%    \cs{CurrentFileUsed} and \cs{CurrentFilePathUsed} will hold the
%    actual file name and path loaded by \LaTeX, while \cs{CurrentFile}
%    and \cs{CurrentFilePath} will hold the names that were
%    \emph{asked for}.  Unless doing very specific work on the file
%    being read, \cs{CurrentFile} and \cs{CurrentFilePath} should be
%    enough.
%  \end{variable}
%
% \subsection{Hooks for package and class files}
%
%    Commands to load package and class files (e.g., \cs{usepackage},
%    \cs{RequirePackage}, \cs{LoadPackageWithOptions}, etc.) offer the
%    hooks from section~\ref{sec:general-file-hooks} when they are
%    used to load a  package or class file, e.g.,
%    \texttt{file/after/array.sty} would be called after the
%    \pkg{array} package got loaded. But as packages and classes form as special group
%    of files, there are some additional hooks available that only
%    apply when a package or class is loaded.
%
%
%  \begin{variable}{
%                   package/before,package/after,
%                   package/before/...,package/after/...,
%                   class/before,class/after,
%                   class/before/...,class/after/...,
%                   }
%    These are:
%    \begin{description}
%    \item[\texttt{package/before}, \texttt{package/after}]
%
%      These hooks are called for each package being loaded.
%
%    \item[\texttt{package/before/\meta{name}},
%    \texttt{package/after/\meta{name}}]
%
%      These hooks are additionally called if the package name is
%      \meta{name} (without extension).
%
%    \item[\texttt{class/before}, \texttt{class/after}]
%
%      These hooks are called for each class being loaded.
%
%    \item[\texttt{class/before/\meta{name}}, \texttt{class/after/\meta{name}}]
%
%      These hooks are additionally called if the class name is
%      \meta{name} (without extension).
%
%    \end{description}
%  \end{variable}
%     All \hook{/after} hooks are implemented as reversed hooks.
%
%     \noindent The overall sequence of execution for \cs{usepackage}
%     and friends is therefore:
%  \begin{tabbing}
%    mm\=mm\=mm\=mm\=\kill
%    \>\cs{UseHook}\marg{\hook{package/before}} \\
%    \>\cs{UseHook}\marg{\hook{package/before/\meta{package name}}} \\[5pt]
%    \>\>\cs{UseHook}\marg{\hook{file/before}} \\
%    \>\>\cs{UseHook}\marg{\hook{file/before/\meta{package name}.sty}} \\
%    \>\>\> \meta{package contents} \\
%    \>\>\cs{UseHook}\marg{\hook{file/after/\meta{package name}.sty}} \\
%    \>\>\cs{UseHook}\marg{\hook{file/after}} \\[5pt]
%    \>\>\emph{code from \cs{AtEndOfPackage} if
%                used inside the package} \\[5pt]
%    \>\cs{UseHook}\marg{\hook{package/after/\meta{package name}}} \\
%    \>\cs{UseHook}\marg{\hook{package/after}} 
%  \end{tabbing}
%    and similar for class file loading, except that \texttt{package/}
%    is replaced by \texttt{class/} and \cs{AtEndOfPackage} by
%    \cs{AtEndOfClass}.
%
%    If a package or class is not loaded (or it was loaded before the
%    hooks were set) none of the hooks are executed!
%
% \subsection{Hooks for \cs{include} files}
%
%    To manage \cs{include} files, \LaTeX{} issues a \cs{clearpage}
%    before and after loading such a file. Depending on the use case
%    one may want to execute code before or after these
%    \cs{clearpage}s especially for the one that is issued at the end.
%
%    Executing code before the final \cs{clearpage}, means that the
%    code is processed while the last page of the included material is
%    still under construction.  Executing code after it means that all
%    floats from inside the include file are placed (which
%    might have added further pages) and the final page has finished.
%
%    Because of these different scenarios we offer hooks in three
%    places.\footnote{If you want to execute code before the first
%     \cs{clearpage} there is no need to use a hook---you can write it
%     directly in front of the \cs{include}.}
%    None of the hooks are executed when an \cs{include} file is
%    bypassed because of an \cs{includeonly} declaration. They are,
%    however, all executed if \LaTeX{} makes an attempt to load the
%    \cs{include} file (even if it doesn't exist and all that happens
%    is \enquote{\texttt{No file \meta{filename}.tex}}).
%
%
%  \begin{variable}{include/before,include/before/...,
%                   include/end,include/end/...,
%                   include/after,include/after/...,
%                  }
%    These are:
%    \begin{description}
%
%    \item[\texttt{include/before}, \texttt{include/before/\meta{name}}]
%
%      These hooks are executed (in that order) after the initial
%      \cs{clearpage} and after \texttt{.aux} file is changed to use
%      \texttt{\meta{name}.aux}, but before the
%      \texttt{\meta{name}.tex} file is loaded. In other words they are executed
%      at the very beginning of the first page of the \cs{include}
%      file.
%
%
%    \item[\texttt{include/end/\meta{name}}, \texttt{include/end}]
%
%      These hooks are executed (in that order) after \LaTeX{} has
%      stopped reading from the \cs{include} file, but before it has
%      issued a \cs{clearpage} to output any deferred floats.
%
%
%    \item[\texttt{include/after/\meta{name}}, \texttt{include/after}]
%
%      These hooks are executed (in that order) after \LaTeX{} has
%      issued the \cs{clearpage} but before is has switched back
%      writing to the main \texttt{.aux} file. Thus technically we are
%      still inside the \cs{include} and if the hooks generate any
%      further typeset material including anything that writes to the
%      \texttt{.aux} file, then it would be considered part of the
%      included material and bypassed if it is not loaded because of
%      some \cs{includeonly} statement.\footnotemark
%
%    \end{description}
%  \end{variable}\footnotetext{For that reason
%      another \cs{clearpage} is executed after these hooks which
%      normally does nothing, but starts a new page if further material
%      got added this way.}
%
% \subsection{High-level interfaces for \LaTeX{}}
%
%    We do not provide any high-level \LaTeX{} commands (like
%    \pkg{filehook} or \pkg{scrlfile} do) but think that for package
%    writers the commands from for hook management are sufficient.
%
%
%
% \subsection{Internal interfaces for \LaTeX{}}
%
% \begin{function}{\declare@file@substitution,\undeclare@file@substitution}
%   \begin{syntax}
%     \cs{declare@file@substitution}   \Arg{file} \Arg{replacement-file}
%     \cs{undeclare@file@substitution} \Arg{file}
%   \end{syntax}
%    If \meta{file} is requested for loading replace it with
%    \meta{replacement-file}. \cs{CurrentFile} remains pointing to
%    \meta{file} but \cs{CurrentFileUsed} will show the file actually
%    loaded.
%
%    The main use case for this declaration is to provide a corrected
%    version of a package that can't be changed (due to its license)
%    but no longer functions because of \LaTeX{} kernel changes, for
%    example, or to provide a version that makes use of new kernel
%    functionality while the original package remains available for
%    use with older releases.
%
%    The \cs{undeclare@file@substitution} declaration undoes a
%    substitution made earlier.
%
%    \begin{quote}
%      \em
%      Please do not misuse this functionality and replace a file with
%      another unless if really needed and only if the new version is
%      implementating the same functionality as the original one!
%    \end{quote}
%  \end{function}
%
% \begin{function}{\disable@package@load,\reenable@package@load}
%   \begin{syntax}
%     \cs{disable@package@load}  \Arg{package}  \Arg{alternate-code}
%     \cs{reenable@package@load} \Arg{package}
%   \end{syntax}
%    If \meta{package} is requested do not load it but instead run
%    \meta{alternate-code} which could issue a warning, error or any
%    other code.
%
%    The main use case is for classes that want to restrict the set of
%    supported packages or contain code that make the use of some
%    packages impossible. So rather than waiting until the document
%    breaks they can set up informative messages why certain packages
%    are not available.
%
%    The function is only implemented for packages not for arbitrary
%    files.
%  \end{function}
%
%
% \subsection{A sample package for structuring the log output}
%
%    As an application we provide the package \pkg{structuredlog} that
%    adds lines to the \texttt{.log} when a file is opened and closed
%    for reading keeping track of nesting level es well.
%    For example, for the current document it adds the lines
%\begin{verbatim}
%    = (LEVEL 1 START) t1lmr.fd
%    = (LEVEL 1 STOP) t1lmr.fd
%    = (LEVEL 1 START) supp-pdf.mkii
%    = (LEVEL 1 STOP) supp-pdf.mkii
%    = (LEVEL 1 START) nameref.sty
%    == (LEVEL 2 START) refcount.sty
%    == (LEVEL 2 STOP) refcount.sty
%    == (LEVEL 2 START) gettitlestring.sty
%    == (LEVEL 2 STOP) gettitlestring.sty
%    = (LEVEL 1 STOP) nameref.sty
%    = (LEVEL 1 START) ltfilehook-doc.out
%    = (LEVEL 1 STOP) ltfilehook-doc.out
%    = (LEVEL 1 START) ltfilehook-doc.out
%    = (LEVEL 1 STOP) ltfilehook-doc.out
%    = (LEVEL 1 START) ltfilehook-doc.hd
%    = (LEVEL 1 STOP) ltfilehook-doc.hd
%    = (LEVEL 1 START) ltfilehook.dtx
%    == (LEVEL 2 START) ot1lmr.fd
%    == (LEVEL 2 STOP) ot1lmr.fd
%    == (LEVEL 2 START) omllmm.fd
%    == (LEVEL 2 STOP) omllmm.fd
%    == (LEVEL 2 START) omslmsy.fd
%    == (LEVEL 2 STOP) omslmsy.fd
%    == (LEVEL 2 START) omxlmex.fd
%    == (LEVEL 2 STOP) omxlmex.fd
%    == (LEVEL 2 START) umsa.fd
%    == (LEVEL 2 STOP) umsa.fd
%    == (LEVEL 2 START) umsb.fd
%    == (LEVEL 2 STOP) umsb.fd
%    == (LEVEL 2 START) ts1lmr.fd
%    == (LEVEL 2 STOP) ts1lmr.fd
%    == (LEVEL 2 START) t1lmss.fd
%    == (LEVEL 2 STOP) t1lmss.fd
%    = (LEVEL 1 STOP) ltfilehook.dtx
%\end{verbatim}
%    Thus if you inspect an issue in the \texttt{.log} it is easy to
%    figure out in which file it occurred, simply by searching back for
%    \texttt{LEVEL} and if it is a \texttt{STOP} then remove 1 from
%    the level value and search further for \texttt{LEVEL} with that value
%    which should then be the \texttt{START}  level of the file you are in.
%
% \StopEventually{\setlength\IndexMin{200pt}  \PrintIndex  }
%
%
% \section{The Implementation}
%    \begin{macrocode}
%<*2ekernel>
%    \end{macrocode}
%
%    \begin{macrocode}
%<@@=filehook>
%    \end{macrocode}
%
%
% \subsection{Document and package-level commands}
%
%
% \begin{macro}{\CurrentFile,\CurrentFilePath}
% \begin{macro}{\CurrentFileUsed,\CurrentFilePathUsed}
%   User-level macros that hold the current file name and file path.
%   These are used internally as well because the code takes care to
%   protect against a possible redefinition of these macros in the
%   loaded file (it's necessary anyway to make hooks work with nested
%   \cs{input}).  The versions |\...Used| hold the \emph{actual} file
%   name and path that is loaded by \LaTeX, whereas the other two hold
%   the name as requested.  They will differ in case there's a file
%   substitution.
%    \begin{macrocode}
%</2ekernel>
%<*2ekernel|latexrelease>
%<latexrelease>\IncludeInRelease{2020/10/01}%
%<latexrelease>                 {\CurrentFile}{Hook management file}%
\ExplSyntaxOn
\tl_new:N \CurrentFile
\tl_new:N \CurrentFilePath
\tl_new:N \CurrentFileUsed
\tl_new:N \CurrentFilePathUsed
\ExplSyntaxOff
%</2ekernel|latexrelease>
%<latexrelease>\EndIncludeInRelease
%    \end{macrocode}
%
%    \begin{macrocode}
%<latexrelease>\IncludeInRelease{0000/00/00}%
%<latexrelease>                 {\CurrentFile}{Hook management file}%
%<latexrelease>
%<latexrelease>\let \CurrentFile         \@undefined
%<latexrelease>\let \CurrentFilePath     \@undefined
%<latexrelease>\let \CurrentFileUsed     \@undefined
%<latexrelease>\let \CurrentFilePathUsed \@undefined
%<latexrelease>
%<latexrelease>\EndIncludeInRelease
%<*2ekernel>
%    \end{macrocode}
% \end{macro}
% \end{macro}
%
%
%
% \subsection{\pkg{expl3} helpers}
%
%    \begin{macrocode}
%</2ekernel>
%<*2ekernel|latexrelease>
%<latexrelease>\IncludeInRelease{2020/10/01}%
%<latexrelease>          {\@@_file_parse_full_name:nN}{File helpers}%
\ExplSyntaxOn
%    \end{macrocode}
%
% \begin{macro}{
%     \@@_file_parse_full_name:nN,
%     \@@_full_name:nn,
%     \@@_set_curr_file_assign:nnnNN
%   }
%   A utility macro to trigger \pkg{expl3}'s file-parsing and lookup,
%   and return a normalized representation of the file name.  If the
%   queried file doesn't exist, no normalisation takes place.
%   The output of \cs{@@_file_parse_full_name:nN} is passed on to the
%   |#2|---a 3-argument macro that takes the \meta{path}, \meta{base},
%   and \meta{ext} parts of the file name.
%    
%    \begin{macrocode}
\cs_new:Npn \@@_file_parse_full_name:nN #1
  {
    \exp_args:Nf \file_parse_full_name_apply:nN
      {
        \exp_args:Nf \@@_full_name:nn
          { \file_full_name:n {#1} } {#1}
      }
  }
\cs_new:Npn \@@_full_name:nn #1 #2
  {
    \tl_if_empty:nTF {#1}
      { \tl_trim_spaces:n {#2} }
      { \tl_trim_spaces:n {#1} }
  }
%    \end{macrocode}
% \end{macro}
%
% \begin{macro}{
%     \@@_if_no_extension:nTF,
%     \@@_drop_extension:N
%   }
%   Some actions depend on whether the file extension was explicitly
%   given, and sometimes the extension has to be removed.  The macros
%   below use \cs{@@_file_parse_full_name:nN} to split up the file name
%   and either check if \meta{ext} (|#3|) is empty, or discard it.
%    \begin{macrocode}
\cs_new:Npn \@@_if_no_extension:nTF #1
  {
    \exp_args:Ne \tl_if_empty:nTF
      { \file_parse_full_name_apply:nN {#1} \use_iii:nnn }
  }
\cs_new_protected:Npn \@@_drop_extension:N #1
  {
    \tl_gset:Nx #1
      {
        \exp_args:NV \@@_file_parse_full_name:nN #1
          \@@_drop_extension_aux:nnn
      }
  }
\cs_new:Npn \@@_drop_extension_aux:nnn #1 #2 #3
   { \tl_if_empty:nF {#1} { #1 / } #2 }
%    \end{macrocode}
% \end{macro}
%
% \begin{macro}{\g_@@_input_file_seq,\l_@@_internal_tl}
% \begin{macro}{\@@_file_push:,\@@_file_pop:}
% \begin{macro}{\@@_file_pop_assign:nnnn}
%   Yet another stack, to keep track of \cs{CurrentFile} and
%   \cs{CurrentFilePath} with nested \cs{input}s.  At the beginning of
%   \cs{InputIfFileExists}, the current value of \cs{CurrentFilePath}
%   and \cs{CurrentFile} is pushed to \cs{g_@@_input_file_seq}, and
%   at the end, it is popped and the value reassigned.  Some other
%   places don't use \cs{InputIfFileExists} directly (\cs{include}) or
%   need \cs{CurrentFile} earlier (\cs{@onefilewithoptions}), so these
%   are manually used elsewhere as well.
%    \begin{macrocode}
\tl_new:N \l_@@_internal_tl
\seq_new:N \g_@@_input_file_seq
\cs_new_protected:Npn \@@_file_push:
  {
    \seq_gpush:Nx \g_@@_input_file_seq
      {
        { \CurrentFilePathUsed } { \CurrentFileUsed }
        { \CurrentFilePath     } { \CurrentFile     }
      }
  }
\cs_new_protected:Npn \@@_file_pop:
  {
    \seq_gpop:NNTF \g_@@_input_file_seq \l_@@_internal_tl
      { \exp_after:wN \@@_file_pop_assign:nnnn \l_@@_internal_tl }
      {
        \msg_error:nnn { hooks } { should-not-happen }
          { Tried~to~pop~from~an~empty~file~name~stack. }
      }
  }
\cs_new_protected:Npn \@@_file_pop_assign:nnnn #1 #2 #3 #4
  {
    \tl_set:Nn \CurrentFilePathUsed {#1}
    \tl_set:Nn \CurrentFileUsed {#2}
    \tl_set:Nn \CurrentFilePath {#3}
    \tl_set:Nn \CurrentFile {#4}
  }
\ExplSyntaxOff
%    \end{macrocode}
% \end{macro}
% \end{macro}
% \end{macro}
%    
%    \begin{macrocode}
%</2ekernel|latexrelease>
%<latexrelease>\EndIncludeInRelease
%    \end{macrocode}
%
%    When rolling forward the following expl3 functions may not be defined.
%    If we roll back the code does nothing.
% \changes{v1.0d}{2020/11/24}{Support for roll forward (gh/434)}
%    \begin{macrocode}
%<latexrelease>\IncludeInRelease{2020/10/01}%
%<latexrelease>                 {\file_parse_full_name_apply:nN}{Roll forward help}%
%<latexrelease>
%<latexrelease>\ExplSyntaxOn
%<latexrelease>\cs_if_exist:NF\file_parse_full_name_apply:nN
%<latexrelease>{
%<latexrelease>\cs_new:Npn \file_parse_full_name_apply:nN #1
%<latexrelease>  {
%<latexrelease>    \exp_args:Ne \__file_parse_full_name_auxi:nN
%<latexrelease>      { \__kernel_file_name_sanitize:n {#1} }
%<latexrelease>  }
%<latexrelease>\cs_new:Npn \__file_parse_full_name_auxi:nN #1
%<latexrelease>  {
%<latexrelease>    \__file_parse_full_name_area:nw { } #1
%<latexrelease>      / \s__file_stop
%<latexrelease>  }
%<latexrelease>\cs_new:Npn \__file_parse_full_name_area:nw #1 #2 / #3 \s__file_stop
%<latexrelease>  {
%<latexrelease>    \tl_if_empty:nTF {#3}
%<latexrelease>      { \__file_parse_full_name_base:nw { } #2 . \s__file_stop {#1} }
%<latexrelease>      { \__file_parse_full_name_area:nw { #1 / #2 } #3 \s__file_stop }
%<latexrelease>  }
%<latexrelease>\cs_new:Npn \__file_parse_full_name_base:nw #1 #2 . #3 \s__file_stop
%<latexrelease>  {
%<latexrelease>    \tl_if_empty:nTF {#3}
%<latexrelease>      {
%<latexrelease>        \tl_if_empty:nTF {#1}
%<latexrelease>          {
%<latexrelease>            \tl_if_empty:nTF {#2}
%<latexrelease>              { \__file_parse_full_name_tidy:nnnN { } { } }
%<latexrelease>              { \__file_parse_full_name_tidy:nnnN { .#2 } { } }
%<latexrelease>          }
%<latexrelease>          { \__file_parse_full_name_tidy:nnnN {#1} { .#2 } }
%<latexrelease>      }
%<latexrelease>      { \__file_parse_full_name_base:nw { #1 . #2 } #3 \s__file_stop }
%<latexrelease>  }
%<latexrelease>\cs_new:Npn \__file_parse_full_name_tidy:nnnN #1 #2 #3 #4
%<latexrelease>  {
%<latexrelease>    \exp_args:Nee #4
%<latexrelease>      {
%<latexrelease>        \str_if_eq:nnF {#3} { / } { \use_none:n }
%<latexrelease>        #3 \prg_do_nothing:
%<latexrelease>      }
%<latexrelease>      { \use_none:n #1 \prg_do_nothing: }
%<latexrelease>      {#2}
%<latexrelease>  }
%<latexrelease>}  
%<latexrelease>\ExplSyntaxOff
%<latexrelease>
%<latexrelease>\EndIncludeInRelease
%<*2ekernel>
%    \end{macrocode}
%
%    \begin{macrocode}
%<@@=>
%    \end{macrocode}
%
% \subsection{Declaring the file-related hooks}
%
%  All hooks starting with \texttt{file/} \texttt{include/},
%  \texttt{class/} or \texttt{package/} are generic and will be
%  allocated if code is added to them. Thus there is no need to
%  explicitly declare any hook in the code below.
%
%  Furthermore, those named \texttt{.../after} or \texttt{.../end} are
%  automatically declared as reversed hooks if filled with code, so this
%  is also automatically taken care of.
%
%
%
%
% \subsection{Patching \LaTeX{}'s \cs{InputIfFileExists} command}
%
%   Most of what we have to do is adding \cs{UseHook} into several
%  \LaTeXe{} core commands, because of some circular dependencies in the
%  kernel we  do this only now and not in \texttt{ltfiles}.
%
% \begin{macro}{\InputIfFileExists}
% \begin{macro}{\@input@file@exists@with@hooks}
% \begin{macro}{\unqu@tefilef@und}
%    \cs{InputIfFileExists} loads any file if it is available so we
%    have to add the hooks \texttt{file/before} and
%    \texttt{file/after} in the right places. If the file doesn't
%    exist no hooks should be executed.
%    \begin{macrocode}
%</2ekernel>
%<latexrelease>\IncludeInRelease{2020/10/01}%
%<latexrelease>          {\InputIfFileExists}{Hook management (files)}%
%<*2ekernel|latexrelease>
%    \end{macrocode}
%    
%    \begin{macrocode}
\let\InputIfFileExists\@undefined
\DeclareRobustCommand \InputIfFileExists[2]{%
  \IfFileExists{#1}%
    {%
     \@expl@@@filehook@file@push@@
     \@filehook@set@CurrentFile
%    \end{macrocode}
%    If the file exists then \cs{CurrentFile} holds its name. But we
%    can't rely on that still being true after the file has been
%    processed. Thus for using the name in the file hooks we need to
%    preserve the name and then restore it for the
%    \texttt{file/after/...} hook.
%
%    The hook always refers to the file requested by the user.  The hook
%    is \emph{always} loaded for \cs{CurrentFile} which usually is the
%    same as \cs{CurrentFileUsed}.  In the case of a file replacement,
%    the \cs{CurrentFileUsed} holds the actual file loaded.  In any case
%    the file names are normalized so that the hooks work on the real
%    file name, rather than what the user typed in.
%
%    \pkg{expl3}'s \cs{file_full_name:n} normalizes the file
%    name (to factor out differences in the |.tex| extension), and
%    then does a file lookup to take into account a possible path from
%    \cs{l_file_search_path_seq} and \cs{input@path}.  However only
%    the file name and extension are returned so that file hooks can
%    refer to the file by their name only.  The path to the file is
%    returned in \cs{CurrentFilePath}.
%    \begin{macrocode}
      \edef\reserved@a{%
        \@expl@@@filehook@file@pop@assign@@nnnn
          {\CurrentFilePathUsed}%
          {\CurrentFileUsed}%
          {\CurrentFilePath}%
          {\CurrentFile}%
%    \end{macrocode}
%
%    We pre-expand \cs{@filef@und} so that in case another file is
%    loaded in the true branch of \cs{InputIfFileExists}, these don't
%    change their value meanwhile. This isn't a worry with
%    \cs[no-index]{CurrentFile...} because they are kept in a stack.
%
% \changes{v1.0d}{2020/11/20}
%   {Move loading to \cs{@input@file@exists@with@hooks} and expand
%    \cs{@filef@und} to avoid getting the wrong file name in the case of
%    a substitution.}
%    \begin{macrocode}
        \noexpand\@input@file@exists@with@hooks{\@filef@und}}%
      \expandafter\@swaptwoargs\expandafter
        {\reserved@a}%
        {#2}%
      \@expl@@@filehook@file@pop@@
    }%
}
%    \end{macrocode}
%
% Before adding to the file list we need to make all (letter) characters
% catcode~11, because several packages use constructions like
% \begin{verbatim}
% \filename@parse{<filename>}
% \ifx\filename@ext\@clsextension
%   ...
% \fi
% \end{verbatim}
% and that doesn't work if \cs{filename@ext} is \cs{detokenize}d.
% Making \cs{@clsextension} a string doesn't help much because some
% packages define their own \cs[no-index]{<prefix>@someextension} with
% normal catcodes.  This is not entirely correct because packages loaded
% (somehow) with catcode~12 alphabetic tokens (say, as the result of
% a \cs{string} or \cs{detokenize} command, or from a \TeX{} string like
% \cs{jobname}) will have these character tokens incorrectly turned into
% letter tokens.  This however is rare, so we'll go for the all-letters
% approach (grepping the packages in \TeX{} Live didn't bring up any
% obvious candidate for breaking with this catcode change).
%    \begin{macrocode}
\def\@input@file@exists@with@hooks#1{%
  \edef\reserved@a{\unqu@tefilef@und#1\@nil}%
  \@addtofilelist{\string@makeletter\reserved@a}%
  \UseHook{file/before}%
%    \end{macrocode}
%    The current file name is available in \cs{CurrentFile} so we use
%    that in the specific hook.
%    \begin{macrocode}
  \UseHook{file/before/\CurrentFile}%
  \@@input #1% <- trailing space comes from \@filef@und
%    \end{macrocode}
%    And it is restored here so we can use it once more.
%    \begin{macrocode}
  \UseHook{file/after/\CurrentFile}%
  \UseHook{file/after}}
\def\unqu@tefilef@und"#1" \@nil{#1}
%<latexrelease>\EndIncludeInRelease
%</2ekernel|latexrelease>
%    \end{macrocode}
%    
% \changes{v0.9b}
%         {1993/12/04}{Macro added}
% \changes{v0.9p}
%         {1994/01/18}{New Definition}
% \changes{v0.3b}{1994/03/13}
%         {Use new cmd \cs{@addtofilelist}}
% Now define |\InputIfFileExists| to input |#1| if it seems to exist.
% Immediately prior to the input, |#2| is executed.
% If the file |#1| does not exist, execute `|#3|'.
% \changes{v1.0t}{1995/05/25}
%         {(CAR) added \cs{long}}
% \changes{v1.1o}{2019/02/07}{Expand \cs{@filef@und} before executing
%   second argument (github/109)}
% \changes{v1.2b}{2019/08/27}{Make command robust}
%    \begin{macrocode}
%<latexrelease>\IncludeInRelease{2019/10/01}%
%<latexrelease>          {\InputIfFileExists}{Hook management (files)}%
%<latexrelease>
%<latexrelease>\DeclareRobustCommand \InputIfFileExists[2]{%
%<latexrelease>  \IfFileExists{#1}%
%<latexrelease>    {%
%<latexrelease>  \expandafter\@swaptwoargs\expandafter
%<latexrelease>      {\@filef@und}{#2\@addtofilelist{#1}\@@input}}}
%<latexrelease>\let\@input@file@exists@with@hooks\@undefined
%<latexrelease>\let\unqu@tefilef@und\@undefined
%<latexrelease>\EndIncludeInRelease
%    \end{macrocode}
%
%    \begin{macrocode}
%<latexrelease>\IncludeInRelease{0000/00/00}%
%<latexrelease>          {\InputIfFileExists}{Hook management (files)}%
%<latexrelease>\long\def \InputIfFileExists#1#2{%
%<latexrelease>  \IfFileExists{#1}%
%<latexrelease>    {#2\@addtofilelist{#1}\@@input \@filef@und}}
%<latexrelease>\let\@input@file@exists@with@hooks\@undefined
%<latexrelease>\let\unqu@tefilef@und\@undefined
%<latexrelease>\EndIncludeInRelease
%<*2ekernel>
%    \end{macrocode}
%  \end{macro}
%  \end{macro}
%  \end{macro}
%
%
%
%
%
%
% \subsection{Declaring a file substitution}
%
%    \begin{macrocode}
%<@@=filehook>
%    \end{macrocode}
%
%    \begin{macrocode}
%</2ekernel>
%<*2ekernel|latexrelease>
%<latexrelease>\IncludeInRelease{2020/10/01}%
%<latexrelease>          {\@@_subst_add:nn}{Declaring file substitution}%
\ExplSyntaxOn
%    \end{macrocode}
%
%
% \begin{macro}{\@@_subst_add:nn,\@@_subst_remove:n,
%               \@@_subst_file_normalize:Nn,\@@_subst_empty_name_chk:NN}
%   \cs{@@_subst_add:nn} declares a file substitution by
%   doing a (global) definition of the form
%   |\def|\cs{@file-subst@\meta{file}}|{|\meta{replacement}|}|.
%   The file names are properly sanitised, and normalized with the same
%   treatment done for the file hooks.  That is, a file replacement is
%   declared by using the file name (and extension, if any) only, and
%   the file path should not be given.  If a file name is empty it is
%   replaced by |.tex| (the empty csname is used to check that).
%    \begin{macrocode}
\cs_new_protected:Npn \@@_subst_add:nn #1 #2
  {
    \group_begin:
      \cs_set:cpx { } { \exp_not:o { \cs:w\cs_end: } }
      \int_set:Nn \tex_escapechar:D { -1 }
      \cs_gset:cpx
        {
          @file-subst@
          \@@_subst_file_normalize:Nn \use_ii_iii:nnn {#1}
        }
        { \@@_subst_file_normalize:Nn \@@_file_name_compose:nnn {#2} }
    \group_end:
  }
\cs_new_protected:Npn \@@_subst_remove:n #1
  {
    \group_begin:
      \cs_set:cpx { } { \exp_not:o { \cs:w\cs_end: } }
      \int_set:Nn \tex_escapechar:D { -1 }
      \cs_undefine:c
        {
          @file-subst@
          \@@_subst_file_normalize:Nn \use_ii_iii:nnn {#1}
        }
    \group_end:
  }
\cs_new:Npn \@@_subst_file_normalize:Nn #1 #2
  {
    \exp_after:wN \@@_subst_empty_name_chk:NN
      \cs:w \exp_after:wN \cs_end:
        \cs:w \@@_file_parse_full_name:nN {#2} #1 \cs_end:
  }
\cs_new:Npn \@@_subst_empty_name_chk:NN #1 #2
  { \if_meaning:w #1 #2 .tex \else: \token_to_str:N #2 \fi: }
%    \end{macrocode}
% \end{macro}
%    
% \begin{macro}[int]{\use_ii_iii:nnn}
%    A variant of \cs[no-index]{use_...} to discard the first of three
%    arguments.
%    \fmi{this should move to \pkg{expl3}}
%    \begin{macrocode}
\cs_gset:Npn \use_ii_iii:nnn #1 #2 #3 {#2 #3}
%    \end{macrocode}
% \end{macro}
%
%
%    \begin{macrocode}
\ExplSyntaxOff
%</2ekernel|latexrelease>
%<latexrelease>\EndIncludeInRelease
%<*2ekernel>
%    \end{macrocode}
%
%
%
% \begin{macro}{\declare@file@substitution}
% \begin{macro}{\undeclare@file@substitution}
%    For two internals we provide \LaTeXe{} names so that we can use
%    them elsewhere in the kernel (and so that they can be used in
%    packages if really needed, e.g., \pkg{scrlfile}).
%    \begin{macrocode}
%</2ekernel>
%<*2ekernel|latexrelease>
%<latexrelease>\IncludeInRelease{2020/10/01}%
%<latexrelease>          {\declare@file@substitution}{File substitution}%
\ExplSyntaxOn
\cs_new_eq:NN \declare@file@substitution   \@@_subst_add:nn
\cs_new_eq:NN \undeclare@file@substitution \@@_subst_remove:n
\ExplSyntaxOff
%</2ekernel|latexrelease>
%<latexrelease>\EndIncludeInRelease
%    \end{macrocode}
%
%    \begin{macrocode}
%<latexrelease>\IncludeInRelease{0000/00/00}%
%<latexrelease>          {\declare@file@substitution}{File substitution}%
%<latexrelease>
%<latexrelease>\let \declare@file@substitution   \@undefined
%<latexrelease>\let \undeclare@file@substitution \@undefined
%<latexrelease>
%<latexrelease>\EndIncludeInRelease
%<*2ekernel>
%    \end{macrocode}
% \end{macro}
% \end{macro}
%
%
%
%
%    \begin{macrocode}
%<@@=>
\ExplSyntaxOff
%    \end{macrocode}
%
% \subsection{Selecting a file (\cs{set@curr@file})}
%
% \begin{macro}{\set@curr@file,\@curr@file,\@curr@file@reqd}
%   Now we hook into \cs{set@curr@file} to resolve a possible file
%   substitution, and add \cs{@expl@@@filehook@set@curr@file@@nNN}
%   at the end, after \cs{@curr@file} is set.
%
%   A file name is built using
%   \cs{expandafter}\cs{string}\cs{csname}\meta{filename}\cs{endcsname}
%   to avoid expanding utf8 active characters.  The \cs{csname} expands
%   the normalisation machinery and the routine to resolve a file
%   substitution, returning a control sequence with the same name as the
%   file.
%
%   It happens that when \meta{filename} is empty, the generated control
%   sequence is \cs{csname\cs{endcsname}}, and doing \cs{string} on
%   that results in the file |csnameendcsname.tex|.  To guard against
%   that we \cs{ifx}-compare the generated control sequence with the
%   empty csname.  To do so, \cs{csname\cs{endcsname}} has to be
%   defined, otherwise it would be equal to \cs{relax} and we would have
%   false positives.  Here we define \cs{csname\cs{endcsname}} to
%   expand to itself to avoid it matching the definition of some other
%   control sequence.
%    \begin{macrocode}
%</2ekernel>
%<*2ekernel|latexrelease>
%<latexrelease>\IncludeInRelease{2020/10/01}%
%<latexrelease>          {\set@curr@file}{Setting current file name}%
\def\set@curr@file#1{%
  \begingroup
    \escapechar\m@ne
    \expandafter\def\csname\expandafter\endcsname
      \expandafter{\csname\endcsname}%
%    \end{macrocode}
%   Two file names are set here: \cs{@curr@file@reqd} which is the file
%   requested by the user, and \cs{@curr@file} which should be the same,
%   except when we have a file substitution, in which case it holds the
%   actual loaded file.  \cs{@curr@file} is resolved first, to check if
%   a substitution happens.  If it doesn't,
%   \cs{@expl@@@filehook@if@file@replaced@@TF} short-cuts and just copies
%   \cs{@curr@file}, otherwise the full normalisation procedure is
%   executed.
%
%   At this stage the file name is parsed and normalized, but if the
%   input doesn't have an extension, the default |.tex| is \emph{not}
%   added to \cs{@curr@file} because for applications other than
%   \cs{input} (graphics, for example) the default extension may not
%   be |.tex|.  First check if the input has an extension, then if the
%   input had no extension, call \cs{@expl@@@filehook@drop@extension@@N}.  In case
%   of a file substitution, \cs{@curr@file} will have an extension.
%    \begin{macrocode}
    \@expl@@@filehook@if@no@extension@@nTF{#1}%
      {\@tempswatrue}{\@tempswafalse}%
    \@kernel@make@file@csname\@curr@file
      \@expl@@@filehook@resolve@file@subst@@w {#1}%
    \@expl@@@filehook@if@file@replaced@@TF
      {\@kernel@make@file@csname\@curr@file@reqd
         \@expl@@@filehook@normalize@file@name@@w{#1}%
       \if@tempswa \@expl@@@filehook@drop@extension@@N\@curr@file@reqd \fi}%
      {\if@tempswa \@expl@@@filehook@drop@extension@@N\@curr@file \fi
       \global\let\@curr@file@reqd\@curr@file}%
    \@expl@@@filehook@clear@replacement@flag@@
  \endgroup}
%</2ekernel|latexrelease>
%<latexrelease>\EndIncludeInRelease
%    \end{macrocode}
%    
%    \begin{macrocode}
%<latexrelease>\IncludeInRelease{2019/10/01}%
%<latexrelease>          {\set@curr@file}{Setting current file name}%
%<latexrelease>\def\set@curr@file#1{%
%<latexrelease>  \begingroup
%<latexrelease>    \escapechar\m@ne
%<latexrelease>    \xdef\@curr@file{%
%<latexrelease>      \expandafter\expandafter\expandafter\unquote@name
%<latexrelease>      \expandafter\expandafter\expandafter{%
%<latexrelease>      \expandafter\string
%<latexrelease>        \csname\@firstofone#1\@empty\endcsname}}%
%<latexrelease>  \endgroup
%<latexrelease>}
%<latexrelease>\EndIncludeInRelease
%    \end{macrocode}
%    
%    \begin{macrocode}
%<latexrelease>\IncludeInRelease{0000/00/00}%
%<latexrelease>          {\set@curr@file}{Setting current file name}%
%<latexrelease>\let\set@curr@file\@undefined
%<latexrelease>\EndIncludeInRelease
%<*2ekernel>
%    \end{macrocode}
% \end{macro}
%
%
%
%
% \begin{macro}{\@filehook@set@CurrentFile}
% \begin{macro}{\@kernel@make@file@csname,\@set@curr@file@aux}
%
%    \fmi{This should get internalized using \texttt{@expl@} names}
%    \begin{macrocode}
%</2ekernel>
%<*2ekernel|latexrelease>
%<latexrelease>\IncludeInRelease{2020/10/01}%
%<latexrelease>          {\@kernel@make@file@csname}{Make file csname}%
%    \end{macrocode}
%
%    \begin{macrocode}
\def\@kernel@make@file@csname#1#2#3{%
  \xdef#1{\expandafter\@set@curr@file@aux
    \csname\expandafter#2\@firstofone#3\@nil\endcsname}}
%    \end{macrocode}
%   This auxiliary compares \cs{\meta{filename}} with
%   \cs{csname\cs{endcsname}} to check if the empty |.tex| file was
%   requested.
%    \begin{macrocode}
\def\@set@curr@file@aux#1{%
  \expandafter\ifx\csname\endcsname#1%
    .tex\else\string#1\fi}
%    \end{macrocode}
%    
%   Then we call \cs{@expl@@@filehook@set@curr@file@@nNN} once for
%   \cs{@curr@file} to set \cs[no-index]{CurrentFile(Path)Used} and once for
%   \cs{@curr@file@reqd} to set \cs[no-index]{CurrentFile(Path)}.
%   Here too the slower route is only used if a substitution happened,
%   but here \cs{@expl@@@filehook@if@file@replaced@@TF} can't be used because
%   the flag is reset at the \cs{endgroup} above, so we check if
%   \cs{@curr@file} and \cs{@curr@file@reqd} differ.  This macro is
%   issued separate from \cs{set@curr@file} because it changes
%   \cs{CurrentFile}, and side-effects would quickly get out of control.
%    \begin{macrocode}
\def\@filehook@set@CurrentFile{%
  \@expl@@@filehook@set@curr@file@@nNN{\@curr@file}%
    \CurrentFileUsed\CurrentFilePathUsed
  \ifx\@curr@file@reqd\@curr@file
    \let\CurrentFile\CurrentFileUsed
    \let\CurrentFilePath\CurrentFilePathUsed
  \else
    \@expl@@@filehook@set@curr@file@@nNN{\@curr@file@reqd}%
      \CurrentFile\CurrentFilePath
  \fi}
%</2ekernel|latexrelease>
%<latexrelease>\EndIncludeInRelease
%<*2ekernel>
%    \end{macrocode}
% \end{macro}
% \end{macro}
%
%
%
% \begin{macro}{\@@_set_curr_file:nNN,
%               \@@_set_curr_file_assign:nnnNN}
%   When inputting a file, \cs{set@curr@file} does a file lookup
%   (in \cs{input@path} and \cs{l_file_search_path_seq}) and returns the
%   actual file name (\meta{base} plus \meta{ext}) in
%   \cs{CurrentFileUsed}, and in case there's a file substitution, the
%   requested file in \cs{CurrentFile} (otherwise both are the same).
%   Only the base and extension are returned,
%   regardless of the input (both \texttt{path/to/file.tex} and
%   \texttt{file.tex} end up as \texttt{file.tex} in \cs{CurrentFile}).
%   The path is returned in \cs{CurrentFilePath}, in case it's needed.
%    \begin{macrocode}
%</2ekernel>
%<*2ekernel|latexrelease>
%<latexrelease>\IncludeInRelease{2020/10/01}%
%<latexrelease>          {@@_set_curr_file:nNN}{Set curr file}%
\ExplSyntaxOn
%<@@=filehook>
\cs_new_protected:Npn \@@_set_curr_file:nNN #1
  {
    \exp_args:Nf \@@_file_parse_full_name:nN {#1}
      \@@_set_curr_file_assign:nnnNN
  }
\cs_new_protected:Npn \@@_set_curr_file_assign:nnnNN #1 #2 #3 #4 #5
  {
    \str_set:Nn #5 {#1}
    \str_set:Nn #4 {#2#3}
  }
\ExplSyntaxOff
%</2ekernel|latexrelease>
%<latexrelease>\EndIncludeInRelease
%<*2ekernel>
%    \end{macrocode}
% \end{macro}
%
%
%
%
% \subsection{Replacing a file and detecting loops}
%
% \begin{macro}{\@@_resolve_file_subst:w}
% \begin{macro}{\@@_normalize_file_name:w}
% \begin{macro}{\@@_file_name_compose:nnn}
%   Start by sanitising the file with \cs{@@_file_parse_full_name:nN}
%   then do \cs{@@_file_subst_begin:nnn}\Arg{path}\Arg{name}\Arg{ext}.
%    \begin{macrocode}
%</2ekernel>
%<*2ekernel|latexrelease>
%<latexrelease>\IncludeInRelease{2020/10/01}%
%<latexrelease>          {\@@_resolve_file_subst:w}{Replace files detect loops}%
\ExplSyntaxOn
\cs_new:Npn \@@_resolve_file_subst:w #1 \@nil
  { \@@_file_parse_full_name:nN {#1} \@@_file_subst_begin:nnn }
\cs_new:Npn \@@_normalize_file_name:w #1 \@nil
  { \@@_file_parse_full_name:nN {#1} \@@_file_name_compose:nnn }
\cs_new:Npn \@@_file_name_compose:nnn #1 #2 #3
  { \tl_if_empty:nF {#1} { #1 / } #2#3 }
%    \end{macrocode}
% \end{macro}
% \end{macro}
%
% \begin{macro}{flag @@_file_replaced}
% \begin{macro}{\@@_if_file_replaced:TF}
% \begin{macro}{\@@_clear_replacement_flag:}
%   Since the file replacement is done expandably in a \cs{csname}, use
%   a flag to remember if a substitution happened.  We use this in
%   \cs{set@curr@file} to short-circuit some of it in case no
%   substitution happened (by far the most common case, so it's worth
%   optimising).  The flag raised during the file substitution algorithm
%   must be explicitly cleared after the \cs{@@_if_file_replaced:TF}
%   conditional is no longer needed, otherwise further uses of
%   \cs{@@_if_file_replaced:TF} will wrongly return true.
%    \begin{macrocode}
\flag_new:n { @@_file_replaced }
\cs_new:Npn \@@_if_file_replaced:TF #1 #2
  { \flag_if_raised:nTF { @@_file_replaced } {#1} {#2} }
\cs_new_protected:Npn \@@_clear_replacement_flag:
  { \flag_clear:n { @@_file_replaced } }
%    \end{macrocode}
% \end{macro}
% \end{macro}
% \end{macro}
%
% \begin{macro}{\@@_file_subst_begin:nnn}
%   First off, start by checking if the current file ($\meta{name} +
%   \meta{ext}$) has a declared substitution.  If not, then just put
%   that as the name (including a possible \meta{path} in this case):
%   this is the default case with no substitutions, so it's the first to
%   be checked.  The auxiliary \cs{@@_file_subst_tortoise_hare:nn} sees
%   that there's no replacement for |#2#3| and does nothing else.
%    \begin{macrocode}
\cs_new:Npn \@@_file_subst_begin:nnn #1 #2 #3
  {
    \@@_file_subst_tortoise_hare:nn { #2#3 } { #2#3 }
      { \@@_file_name_compose:nnn {#1} {#2} {#3} }
  }
\ExplSyntaxOff
%</2ekernel|latexrelease>
%<latexrelease>\EndIncludeInRelease
%<*2ekernel>
%    \end{macrocode}
% \end{macro}
%
%
%
%
% \subsubsection{The Tortoise and Hare algorithm}
%
% \begin{macro}{\@@_file_subst_tortoise_hare:nn}
% \begin{macro}{\@@_file_subst_loop:NN,\@@_file_subst_loop:cc}
%   If there is a substitution (\meta{true} in the first
%   \cs{cs_if_exist:cTF} below), then first check if there is no
%   substitution down the line:  this should be the second most common
%   case, of one file replaced by another.  In that case just leave the
%   substitution there and the job is done.  If any substitution
%   happens, then the \cs{flag @@_file_replaced} is raised
%   (conditionally, because checking if a flag is raised is much faster
%   than raising it over and over again).
%
%   If, however there are more substitutions, then we need to check for
%   a possible loop in the substitutions, which would otherwise put
%   \TeX{} in an infinite loop if just an exhaustive expansion was used.
%
%   To detect a loop, the \emph{Tortoise and Hare} algorithm is used.
%   The name of the algorithm is an analogy to Aesop's fable, in which
%   the Hare outruns a Tortoise.  The two pointers here are the csnames
%   which contains each file replacement, both of which start at the
%   position zero, which is the file requested.  In the inner part of
%   the macro below, \cs{@@_file_subst_loop:cc} is called with
%   \cs[no-index]{@file-subst@\meta{file}} and
%   \cs[no-index]{@file-subst@\cs[no-index]{@file-subst@\meta{file}}};
%   that is, the substitution of \meta{file} and the substution of that
%   substution:  the Tortoise walks one step while the Hare walks two.
%
%   Within \cs{@@_file_subst_loop:NN} the two substitutions are
%   compared, and if they lead to the same file it means that there is
%   a loop in the substitutions.  If there's no loop,
%   \cs{@@_file_subst_tortoise_hare:nn} is called again with the
%   Tortoise at position~1 and the hare at~2.  Again, the substitutions
%   are checked ahead of the Hare pointer to check that it won't run too
%   far;  in case there is no loop in the declarations, eventually one
%   of the \cs{cs_if_exist:cTF} below will go \meta{false} and the
%   algorithm will end;  otherwise it will run until the Hare reaches
%   the same spot as the tortoise and a loop is detected.
%    \begin{macrocode}
%</2ekernel>
%<*2ekernel|latexrelease>
%<latexrelease>\IncludeInRelease{2020/10/01}%
%<latexrelease>         {\@@_file_subst_tortoise_hare:nn}{Tortoise and Hare}%
\ExplSyntaxOn
\cs_new:Npn \@@_file_subst_tortoise_hare:nn #1 #2 #3
  {
    \cs_if_exist:cTF { @file-subst@ #2 }
      {
        \flag_if_raised:nF { @@_file_replaced }
          { \flag_raise:n { @@_file_replaced } }
        \cs_if_exist:cTF { @file-subst@ \use:c { @file-subst@ #2 } }
          {
            \@@_file_subst_loop:cc
              { @file-subst@ #1 }
              { @file-subst@ \use:c { @file-subst@ #2 } }
          }
          { \use:c { @file-subst@ #2 } }
      }
      { #3 }
  }
%    \end{macrocode}
%   This is just an auxiliary to check if a loop was found, and continue
%   the algorithm otherwise.  If a loop is found, the |.tex| file is
%   used as fallback and \cs{@@_file_subst_cycle_error:cN} is called to
%   report the error.
%    \begin{macrocode}
\cs_new:Npn \@@_file_subst_loop:NN #1 #2
  {
    \token_if_eq_meaning:NNTF #1 #2
      {
        .tex
        \@@_file_subst_cycle_error:cN { @file-subst@ #1 } #1
      }
      { \@@_file_subst_tortoise_hare:nn {#1} {#2} {#2} }
  }
\cs_generate_variant:Nn \@@_file_subst_loop:NN { cc }
%    \end{macrocode}
% \end{macro}
% \end{macro}
%
% \begin{macro}{
%     \@@_file_subst_cycle_error:NN,
%     \@@_file_subst_cycle_error:cN,
%   }
%   Showing this type of error expandably is tricky, as we have a very
%   limited amount of characters to show and a potentially large list.
%   As a work around, several errors are printed, each showing one step
%   of the loop, until all the error messages combined show the loop.
%    \begin{macrocode}
\cs_new:Npn \@@_file_subst_cycle_error:NN #1 #2
  {
    \__kernel_msg_expandable_error:nnff { kernel } { file-cycle }
      {#1} { \use:c { @file-subst@ #1 } }
    \token_if_eq_meaning:NNF #1 #2
      { \@@_file_subst_cycle_error:cN { @file-subst@ #1 } #2 }
  }
\cs_generate_variant:Nn \@@_file_subst_cycle_error:NN { c }
%    \end{macrocode}
%
%   And the error message:
%    \begin{macrocode}
\__kernel_msg_new:nnn { kernel } { file-cycle }
  { File~loop!~#1~replaced~by~#2... }
%    \end{macrocode}
% \end{macro}
% \end{macro}
%
%    \begin{macrocode}
\ExplSyntaxOff
%</2ekernel|latexrelease>
%<latexrelease>\EndIncludeInRelease
%<*2ekernel>
%    \end{macrocode}
%    
%
%    \begin{macrocode}
%<@@=>
%    \end{macrocode}
%
%
% \subsection{Preventing a package from loading}
%
%    We support the use case of preventing a package from loading but not
%    any other type of files (e.g., classes).
%
% \begin{macro}{\disable@package@load}
% \begin{macro}{\reenable@package@load}
% \begin{macro}{\@disable@packageload@do}
%   \cs{disable@package@load} defines
%   \cs[no-index]{@pkg-disable@\meta{package}} to expand to some code |#2|
%   instead of loading the package.
%    \begin{macrocode}
%</2ekernel>
%<*2ekernel|latexrelease>
%<latexrelease>\IncludeInRelease{2020/10/01}%
%<latexrelease>          {\disable@package@load}{Disable packages}%
\def\disable@package@load#1#2{%
  \global\@namedef{@pkg-disable@#1.\@pkgextension}{#2}}
%    \end{macrocode}
%    
%    \begin{macrocode}
\def\@disable@packageload@do#1#2{%
  \@ifundefined{@pkg-disable@#1}{#2}%
     {\@nameuse{@pkg-disable@#1}}}
%    \end{macrocode}
%
%   \cs{reenable@package@load} undefines
%   \cs[no-index]{@pkg-disable@\meta{package}} to reallow loading a package.
%    \begin{macrocode}
\def\reenable@package@load#1{%
  \global\expandafter\let
  \csname @pkg-disable@#1.\@pkgextension \endcsname \@undefined}
%    \end{macrocode}
%
%    
%    \begin{macrocode}
%</2ekernel|latexrelease>
%<latexrelease>\EndIncludeInRelease
%<latexrelease>\IncludeInRelease{0000/00/00}%
%<latexrelease>          {\disable@package@load}{Disable packages}%
%<latexrelease>
%<latexrelease>\let\disable@package@load   \@undefined
%<latexrelease>\let\@disable@packageload@do\@undefined
%<latexrelease>\let\reenable@package@load  \@undefined
%<latexrelease>\EndIncludeInRelease
%<*2ekernel>
%    \end{macrocode}
% \end{macro}
% \end{macro}
% \end{macro}
%
%
%
%
%
% \subsection{High-level interfaces for \LaTeX{}}
%
%    None so far and the general feeling for now is that the hooks are
%    enough. Packages like \pkg{filehook}, etc., may use them to set
%    up their interfaces (samples are given below) but for the now the
%    kernel will not provide any.
%
%
%
% \subsection{Internal commands needed elsewhere}
%
%    Here we set up a few horrible (but consistent) \LaTeXe{} names to
%    allow for internal commands to be used outside this module (and
%    in parts that still use \LaTeXe{} syntax. We have to unset the
%    \texttt{@\/@} since we want double ``at'' sign in place of double
%    underscores.
%
%    \begin{macrocode}
%<@@=>
%    \end{macrocode}
% 
%    \begin{macrocode}
%</2ekernel>
%<*2ekernel|latexrelease>
%<latexrelease>\IncludeInRelease{2020/10/01}%
%<latexrelease>    {\@expl@@@filehook@if@no@extension@@nTF}{2e tmp interfaces}%
\ExplSyntaxOn
%    \end{macrocode}
%    
%    \begin{macrocode}
\cs_new_eq:NN \@expl@@@filehook@if@no@extension@@nTF
              \__filehook_if_no_extension:nTF
%    \end{macrocode}
%
%    \begin{macrocode}
\cs_new_eq:NN \@expl@@@filehook@set@curr@file@@nNN
              \__filehook_set_curr_file:nNN
%    \end{macrocode}
%    
%    \begin{macrocode}
\cs_new_eq:NN \@expl@@@filehook@resolve@file@subst@@w
              \__filehook_resolve_file_subst:w
%    \end{macrocode}
%
%    \begin{macrocode}
\cs_new_eq:NN \@expl@@@filehook@normalize@file@name@@w
              \__filehook_normalize_file_name:w
%    \end{macrocode}
%    
%    \begin{macrocode}
\cs_new_eq:NN \@expl@@@filehook@if@file@replaced@@TF
              \__filehook_if_file_replaced:TF
%    \end{macrocode}
%
%    \begin{macrocode}
\cs_new_eq:NN \@expl@@@filehook@clear@replacement@flag@@
              \__filehook_clear_replacement_flag:
%    \end{macrocode}
%
%    \begin{macrocode}
\cs_new_eq:NN \@expl@@@filehook@drop@extension@@N
              \__filehook_drop_extension:N
%    \end{macrocode}
%    
%    \begin{macrocode}
\cs_new_eq:NN \@expl@@@filehook@file@push@@
               \__filehook_file_push:
%    \end{macrocode}
%    
%    \begin{macrocode}
\cs_new_eq:NN \@expl@@@filehook@file@pop@@
              \__filehook_file_pop:
%    \end{macrocode}
%    
%    \begin{macrocode}
\cs_new_eq:NN \@expl@@@filehook@file@pop@assign@@nnnn
              \__filehook_file_pop_assign:nnnn
%    \end{macrocode}
%    
%
%    \begin{macrocode}
\ExplSyntaxOff
%    \end{macrocode}
%
% This one specifically has to be undefined because it is left over in
% the input stream from \cs{InputIfFileExists} and executed when
% \pkg{latexrelease} is loaded.
%    \begin{macrocode}
%</2ekernel|latexrelease>
%<latexrelease>\EndIncludeInRelease
%<latexrelease>
%<latexrelease>\IncludeInRelease{0000/00/00}%
%<latexrelease>    {\@expl@@@filehook@if@no@extension@@nTF}{2e tmp interfaces}%
%<latexrelease>\let\@expl@@@filehook@file@pop@@\@empty
%<latexrelease>\EndIncludeInRelease
%<*2ekernel>
%    \end{macrocode}
%
%    This ends the kernel code in this file.
%    \begin{macrocode}
%</2ekernel>
%    \end{macrocode}
%
%
%
% \section{A sample package for structuring the log output}
%
%    \begin{macrocode}
%<*structuredlog>
%<@@=filehook>
%    \end{macrocode}
%
%    \begin{macrocode}
\ProvidesExplPackage
    {structuredlog}{\ltfilehookdate}{\ltfilehookversion}
    {Structuring the TeX transcript file}
%    \end{macrocode}
%
% \begin{macro}{\g_@@_nesting_level_int}
%   Stores the current package nesting level.
%    \begin{macrocode}
\int_new:N \g_@@_nesting_level_int
%    \end{macrocode}
%   Initialise the counter with the number of files in the
%   \cs{@currnamestack} (the number of items divided by $3$) minus one,
%   because this package is skipped when printing to the log.
%    \begin{macrocode}
\int_gset:Nn \g_@@_nesting_level_int
  { ( \tl_count:N \@currnamestack ) / 3 - 1 }
%    \end{macrocode}
% \end{macro}
%
% \begin{macro}{\@@_log_file_record:n}
%   This macro is responsible for increasing and decresing the file
%   nesting level, as well as printing to the log.  The argument is
%   either |STOPTART| or |STOP| and the action it takes on the nesting
%   integer depends on that.
%    \begin{macrocode}
\cs_new_protected:Npn \@@_log_file_record:n #1
  {
    \str_if_eq:nnT {#1} {START} { \int_gincr:N \g_@@_nesting_level_int }
    \iow_term:x
      {
        \prg_replicate:nn { \g_@@_nesting_level_int } { = } ~
        ( LEVEL ~ \int_use:N \g_@@_nesting_level_int \c_space_tl #1 ) ~
        \CurrentFileUsed
%    \end{macrocode}
%   If there was a file replacement, show that as well:
%    \begin{macrocode}
        \str_if_eq:NNF \CurrentFileUsed \CurrentFile
          { ~ ( \CurrentFile \c_space_tl requested ) }
        \iow_newline:
      }
    \str_if_eq:nnT {#1} {STOP} { \int_gdecr:N \g_@@_nesting_level_int }
  }
%    \end{macrocode}
%
%   Now just hook the macro above in the generic |file/before|\ldots
%    \begin{macrocode}
\AddToHook{file/before}{ \@@_log_file_record:n { START } }
%    \end{macrocode}
%   \ldots and |file/after| hooks.
%   We don't want to install the \hook{file/after} hook immediately,
%   because that would mean it is the first time executed when the
%   package finishes. We therefore put the declaration inside
%   \cs{AddToHookNext} so that it gets only installed when we have
%   left this package.
%    \begin{macrocode}
\AddToHookNext{file/after}
  { \AddToHook{file/after}{ \@@_log_file_record:n { STOP } } }
%    \end{macrocode}
% \end{macro}
%
%    \begin{macrocode}
%<@@=>
%</structuredlog>
%    \end{macrocode}
%
%
%
%
%
%
% \section{Package emulations}
%
%
% \subsection{Package \pkg{atveryend} emulation}
%
%    With the new hook management and the hooks in \cs{enddocument}
%    all of \pkg{atveryend} is taken care of.
%    We can make an emulation only here after the substitution
%    functionality is available:
%    \begin{macrocode}
%<*2ekernel>
\declare@file@substitution{atveryend.sty}{atveryend-ltx.sty}
%</2ekernel>
%    \end{macrocode}
%
%    Here is the package file we point to:
%    \begin{macrocode}
%<*atveryend-ltx>
\ProvidesPackage{atveryend-ltx}
   [2020/08/19 v1.0a
     Emulation of the original atvery package^^Jwith kernel methods]
%    \end{macrocode}
%
%
%    Here are new definitions for its interfaces now pointing to the
%    hooks in \cs{enddocument}
%    \begin{macrocode}
\newcommand\AfterLastShipout  {\AddToHook{enddocument/afterlastpage}}
\newcommand\AtVeryEndDocument {\AddToHook{enddocument/afteraux}}
%    \end{macrocode}
%    Next one is a bit of a fake, but the result should normally be as
%    expected. If not, one needs to add a rule to sort the code chunks
%    in \hook{enddocument/info}.
%    \begin{macrocode}
\newcommand\AtEndAfterFileList{\AddToHook{enddocument/info}}
%    \end{macrocode}
%    
%    \begin{macrocode}
\newcommand\AtVeryVeryEnd     {\AddToHook{enddocument/end}}
%    \end{macrocode}
%
%  \begin{macro}{\BeforeClearDocument}
%    This one is the only one we don't implement or rather don't have
%    a dedicated hook in the code. 
%    \begin{macrocode}
\ExplSyntaxOn
\newcommand\BeforeClearDocument[1]
  { \AtEndDocument{#1}
    \atveryend@DEPRECATED{BeforeClearDocument \tl_to_str:n{#1}}
  }
%    \end{macrocode}
%    
%    \begin{macrocode}
\cs_new:Npn\atveryend@DEPRECATED #1
   {\iow_term:x{======~DEPRECATED~USAGE~#1~==========}}
\ExplSyntaxOff
%    \end{macrocode}
%  \end{macro}
%
%    
%    \begin{macrocode}
%</atveryend-ltx>
%    \end{macrocode}
%
%
%
%    \Finale
%
%
%%%%%%%%%%%%%%%%%%%%%%%%%%%%%%%%%%%%%%%%%%%%%  
\endinput
%%%%%%%%%%%%%%%%%%%%%%%%%%%%%%%%%%%%%%%%%%%%%  
%


                    }
%    \end{macrocode}
%
%  \subsection{Package options}
%
%    For now we offer a simple debug option which turns on a lot of
%    strange \cs{typeout} messages, nothing fancy.
%    \begin{macrocode}
\ExplSyntaxOn
%    \end{macrocode}
%    
%    \begin{macrocode}
\hook_debug_off:
\DeclareOption { debug } { \hook_debug_on:
                           \shipout_debug_on: }
%    \end{macrocode}
%
%
%    For now we offer a simple debug option which turns on a lot of
%    strange \cs{typeout} messages, nothing fancy.
%    \begin{macrocode}
\shipout_debug_off:
\DeclareOption { debug-shipout } { \shipout_debug_on: }
%    \end{macrocode}
%    
%    \begin{macrocode}
\ProcessOptions
%    \end{macrocode}
%
%
%  \subsection{Temporarily patching package until changed}
%
%
%    \pkg{etoolbox} support until that package is patched:
%    \begin{macrocode}
\RequirePackage{etoolbox-ltx}
%    \end{macrocode}
%
%
%    \pkg{filehook} support until that package is patched:
%    \begin{macrocode}
\RequirePackage{filehook-ltx}
%    \end{macrocode}
%
%
%
%    \begin{macrocode}
%</package>
%    \end{macrocode}
%
%
%
% \Finale
%


