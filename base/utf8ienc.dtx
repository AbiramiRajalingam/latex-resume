% \iffalse meta-comment
%
% Copyright (C) 1993-2021
% The LaTeX Project and any individual authors listed elsewhere
% in this file.
%
% This file is part of the LaTeX base system.
% -------------------------------------------
%
% It may be distributed and/or modified under the
% conditions of the LaTeX Project Public License, either version 1.3c
% of this license or (at your option) any later version.
% The latest version of this license is in
%    https://www.latex-project.org/lppl.txt
% and version 1.3c or later is part of all distributions of LaTeX
% version 2008 or later.
%
% This file has the LPPL maintenance status "maintained".
%
% The list of all files belonging to the LaTeX base distribution is
% given in the file `manifest.txt'. See also `legal.txt' for additional
% information.
%
% The list of derived (unpacked) files belonging to the distribution
% and covered by LPPL is defined by the unpacking scripts (with
% extension .ins) which are part of the distribution.
%
% \fi
%
% \iffalse
%<*driver>
\documentclass{ltxdoc}
\usepackage[latin1,utf8]{inputenc}% force utf8 to be re-loaded, to get version info
\GetFileInfo{utf8.def}
\title{Providing some UTF-8 support via \texttt{inputenc}}
\date{\fileversion\space\filedate{} printed \today}
 \author{%
  David Carlisle \and Frank Mittelbach \and
  Chris Rowley\thanks{Borrowing heavily from tables by Sebastian Rahtz; some table
      and code cleanup by Javier Bezos}}
\begin{document}
 \MaintainedByLaTeXTeam{latex}
 \maketitle
 \tableofcontents
 \DocInput{utf8ienc.dtx}
\end{document}
%</driver>
% \fi
%
% \newpage
%
% \section{Introduction}
%
%
% \subsection{Background and general stuff}
%
% For many reasons what this package provides is a long way from any
% type of `Unicode compliance'.
%
% In stark contrast to 8-bit character sets, with 16 or more bits it can
% easily be very inefficient to support the full range.\footnote{In
% fact, \LaTeX's current 8-bit support does not go so far as to make
% all 8-bit characters into valid input.}  Moreover, useful support of
% character input by a typesetting system overwhelmingly means finding
% an acceptable visual representation of a sequence of characters and
% this, for \LaTeX{}, means having available a suitably encoded 8-bit
% font.
%
% Unfortunately it is not possible to predict exactly what valid UTF-8
% octet sequences will appear in a particular file so it is best to
% make all the unsupported but valid sequences produce a reasonably
% clear and noticeable error message.
%
% There are two directions from which to approach the question of what
% to load.  One is to specify the ranges of Unicode characters that will
% result in some sensible typesetting; this requires the provider to
% ensure that suitable fonts are loaded and that these input characters
% generate the correct typesetting via the encodings of those fonts.  The
% other is to inspect the font encodings to be used and use these to
% define which input Unicode characters should be supported.
%
% For Western European languages, at least, going in either direction
% leads to many straightforward decisions and a few that are more
% subjective.  In both cases some of the specifications are \TeX{}
% specific whilst most are independent of the particular typesetting
% software in use.
%
% As we have argued elsewhere, \LaTeX{} needs to refer to characters via
% `seven-bit-text' names and, so far, these have been chosen by
% reference to historical sources such as Plain \TeX{} or Adobe encoding
% descriptions.  It is unclear whether this ad hoc naming structure should
% simply be extended or whether it would be useful to
% supplement it with standardised internal Unicode character names such as
% one or more of the following:\footnote{Burkhard und Holger Mittelbach
%   spielen mit mir!  Sie haben etwas hier geschrieben.}
%
% \begin{verbatim}
%   \ltxutwochar <4 hex digits>
%
%   \ltxuchar {<hex digits>}
%     B H U R R R
%
%   \ltxueightchartwo   <2 utf8 octets as 8-bit char tokens>
%   \ltxueightcharthree <3 utf8 octets ...>
%   \ltxueightcharfour  <4 utf8 octets ...>
% \end{verbatim}
%
%
% \subsection{More specific stuff}
%
% In addition to setting up the mechanism for reading UTF-8 characters
% and specifying the \LaTeX-level support available, this package
% contains support for some default historically expected \TeX-related
% characters and some example `Unicode definition files' for standard
% font encodings.
%
%
% \subsection{Notes}
%
% This package does not support Unicode combining characters as \TeX{}
% is not really equipped to make this possible.
%
%  No attempt is made to be useful beyond Latin, and maybe Cyrillic,
%  for European languages (as of now).
%
%
% \subsection{Basic operation of the code}
%
% The \texttt{inputenc} package makes the upper 8-bit characters active and
%    assigns to all of them an error message. It then waits for the
%    input encoding files to change this set-up.  Similarly, whenever
%    |\inputencoding| is encountered in a document, first the upper
%    8-bit characters are set back to produce an error and then the
%    definitions for the new input encoding are loaded, changing some of the
%    previous settings.
%
%    The 8-bit input encodings currently supported by \texttt{inputenc}
%    all use declarations such as |\DeclareInputText| and the like to map an
%    8-bit number to some \LaTeX{} internal form, e.g.~to |\"a|.
%
%    The situation when supporting UTF-8 as the input encoding is
%    different, however. Here we only have to set up the actions of
%    those 8-bit numbers that can be the first octet in a UTF-8
%    representation of a Unicode character.  But we cannot simply set
%    this to some internal \LaTeX{} form since the Unicode character
%    consists of more than one octet; instead we have to define this
%    starting octet to parse the right number of further octets that
%    together form the UTF-8 representation of some Unicode character.
%
%    Therefore when switching to \texttt{utf8} within the
%    \texttt{inputenc} framework the characters with numbers (hex)
%    from \texttt{"C2} to \texttt{"DF} are defined to parse for a
%    second octet following, the characters from \texttt{"E0} to
%    \texttt{"EF} are defined to parse for two more octets and finally
%    the characters from \texttt{"F0} to \texttt{"F3} are defined to
%    parse for three additional octets.  These additional octets are
%    always in the range \texttt{"80} to \texttt{"B9}.
%
%    Thus, when such a character is encountered in the document (so
%    long as expansion is not prohibited) a defined number of
%    additional octets (8-bit characters) are read and from them a
%    unique control sequence name is immediately constructed.
%
%    This control sequence is either defined (good) or undefined
%    (likely); in the latter case the user gets an error message
%    saying that this UTF-8 sequence (or, better, Unicode character)
%    is not supported.
%
%    If the control sequence is set up to do something useful then it will
%    expand to a \LaTeX{} internal form: e.g.~for the utf8 sequence of
%    two octets \texttt{"C3 "A4} we get |\"a| as the
%    internal form which then, depending on the font encoding,
%    eventually resolves to the single glyph `latin-a-umlaut' or to
%    the composite glyph `latin-a with an umlaut accent'.
%
%    These mappings from (UTF-8 encoded) Unicode characters to \LaTeX{}
%    internal forms are made indirectly.  The code below provides a
%    declaration |\DeclareUnicodeCharacter| which maps Unicode numbers
%    (as hexadecimal) to \LaTeX{} internal forms.
%
%    This mapping needs to be set up only once so it is done at
%    |\begin{document}| by looking at the list of font encodings that
%    are loaded by the document and providing mappings related to
%    those font encodings whenever these are available. Thus at most
%    only those Unicode characters that can be represented by the glyphs
%    available in these encodings will be defined.
%
%    Technically this is done by loading one file per encoding,
%    if available, that is supposed to provide the necessary mapping
%    information.
%
%
% \StopEventually{}
%
%
%
%
% \section{Coding}
%
% \subsection{Housekeeping}
%
%    The usual introductory bits and pieces:
%
%    \begin{macrocode}
%<utf8>\ProvidesFile{utf8.def}
%<test>\ProvidesFile{utf8-test.tex}
%<+lcy> \ProvidesFile{lcyenc.dfu}
%<+ly1> \ProvidesFile{ly1enc.dfu}
%<+oms> \ProvidesFile{omsenc.dfu}
%<+ot1> \ProvidesFile{ot1enc.dfu}
%<+ot2> \ProvidesFile{ot2enc.dfu}
%<+t1>  \ProvidesFile{t1enc.dfu}
%<+t2a> \ProvidesFile{t2aenc.dfu}
%<+t2b> \ProvidesFile{t2benc.dfu}
%<+t2c> \ProvidesFile{t2cenc.dfu}
%<+ts1> \ProvidesFile{ts1enc.dfu}
%<+x2>  \ProvidesFile{x2enc.dfu}
%<+all> \ProvidesFile{utf8enc.dfu}
%<-utf8-2018>   [2020/11/26 v1.2k UTF-8 support for inputenc]
%    \end{macrocode}
%
%    \begin{macrocode}
%<*utf8>
%    \end{macrocode}
%    This is a temporary fix for the e-p\TeX{} / e-up\TeX{} engines that do not yet
%    have a |\ifincsname| primitive. Once this is available the extra file will
%    be dropped.
% \changes{v1.2h}{2019/07/09}{Temp rollback fix for e-pTeX}
%    \begin{macrocode}
\ifx\ifincsname\@undefined  % old e-pTeX or e-upTeX engines
  \input utf8-2018.def
  \expandafter\@firstofone
\else
  \expandafter\@gobble
\fi
  \endinput
%    \end{macrocode}
%
%    \begin{macrocode}
\makeatletter
%    \end{macrocode}
%    We restore the |\catcode| of space (which is set to ignore in
%    \texttt{inputenc}) while reading \texttt{.def} files. Otherwise
%    we would need to explicitly use |\space| all over the place in
%    error and log messages.
% \changes{v1.1d}{2004/05/08}{Explicitly set catcode of space}
%    \begin{macrocode}
\catcode`\ \saved@space@catcode
%    \end{macrocode}
%
%
%
% \subsection{Parsing UTF-8 input}
%
%    A UTF-8 char (that is not actually a 7-bit char, i.e.~a single
%    octet) is parsed as follows: each starting octet is an active
%    \TeX{} character token; each of these is defined below to be a
%    macro with one to three arguments nominally (depending on the
%    starting octet). It calls one of |\UTFviii@two@octets|,
%    |\UTFviii@three@octets|, or |\UTFviii@four@octets| which then
%    actually picks up the remaining octets as the argument(s).
%
%    \begin{itemize}
%  \item When typesetting we pick up the necessary number of additional
%    octets, check if they form a command that \LaTeX{} knows about
%    (via \cs{csname} \texttt{u8:}\cs{string}
%     \verb=#1=\cs{string} \verb=#2...=\cs{endcsname}) and if so use that
%    for typesetting.  \cs{string} is needed as the octets may (all?) be
%    active and we want the literal values in the name.
%
%  \item If the UTF-8 character is going to be part of a label, then it is
%    essentially becoming part of some csname and with the
%    test \cs{ifincsname} we can find this out. If so, we render the whole
%    sequence off octets harmless by using \cs{string} too when the
%    starting octet executes (\cs{UTF@...@octets@string}).
%
%  \item Another possible case is that \cs{protect} has \emph{not} the meaning
%    of \cs{typeset@protect}. In that case we may do a \cs{write} or we may do
%    a \cs{protected@edef} or \ldots{}  In all such cases we want to keep the
%    sequence of octets unchanged, but we can't use \cs{string} this time, since at
%    least in the case of \cs{protect@edef} the result may later be
%    typeset after all (in fact that is quite likely) and so at that
%    point the starting octet needs to be an active character again
%    (the others could be stringified). So for this case we use \cs{noexpand}
%    ((\cs{UTF@...s@octets@noexpand}).
%  \end{itemize}
%
% \begin{macro}{\UTFviii@two@octets}
% \changes{v1.2a}{2018/03/24}{Macros made `\cs{long} for improved error messages}%
%    Putting that all together the code for a start octet of a two
%    byte sequence would then look like this:
%    \begin{macrocode}
\long\def\UTFviii@two@octets{%
  \ifincsname
    \expandafter \UTF@two@octets@string
  \else
    \ifx \protect\@typeset@protect \else
      \expandafter\expandafter\expandafter \UTF@two@octets@noexpand
    \fi
  \fi
  \UTFviii@two@octets@combine
}
%    \end{macrocode}

%    \cs{ifcsname} is tested first because that can be true even if we
%    are otherwise doing typesetting. If this is the case we use
%    \cs{string} on the whole octet
%    sequence. \cs{UTF@two@octets@string} not only does this but also
%    gets rid of the command \cs{UTFviii@two@octets@combine} in the input
%    stream by picking it up as a first argument and dropping it.
%
%     If this is not the case and we are doing typesetting (i.e.,
%    \cs{protect} is \cs{typeset@protect}), then we execute
%    \cs{UTFviii@two@octets@combine} which picks up all octets and typesets
%    the character (or generates an error if it doesn't know how to
%    typeset it).
%
%    However, if we are not doing typesetting, then we execute the
%    command \cs{UTFviii@two@octets@noexpand} which works like
%    \cs{UTF@two@octets@string} but uses \cs{noexpand} instead of
%    \cs{string}. This way the sequence is temporary rendered harmless,
%    e.g., would display as is or stays put inside a
%    \cs{protected@edef}. But if the result is later reused the
%    starting octet is still active and so will be able to construct
%    the UTF-8 character again.
% \end{macro}
%
%
% \begin{macro}{\UTFviii@three@octets}
% \changes{v1.2a}{2018/03/24}{Macros made `\cs{long} for improved error messages}%
% \begin{macro}{\UTFviii@four@octets}
% \changes{v1.2a}{2018/03/24}{Macros made `\cs{long} for improved
%                             error messages}
%    The definitions for the other starting octets
%    are the same except that they pick up more octets after them.
%    \begin{macrocode}
\long\def\UTFviii@three@octets{%
  \ifincsname
    \expandafter \UTF@three@octets@string
  \else
    \ifx \protect\@typeset@protect \else
      \expandafter\expandafter\expandafter \UTF@three@octets@noexpand
    \fi
  \fi
  \UTFviii@three@octets@combine
}
%    \end{macrocode}
%    \begin{macrocode}
\long\def\UTFviii@four@octets{%
  \ifincsname
    \expandafter \UTF@four@octets@string
  \else
    \ifx \protect\@typeset@protect \else
      \expandafter\expandafter\expandafter \UTF@four@octets@noexpand
    \fi
  \fi
  \UTFviii@four@octets@combine
}
%    \end{macrocode}
% \end{macro}
% \end{macro}
%
% \begin{macro}{\UTFviii@two@octets@noexpand}
% \begin{macro}{\UTFviii@three@octets@noexpand}
% \begin{macro}{\UTFviii@four@octets@noexpand}
%    These temporarily prevent the active chars from expanding.
%    \begin{macrocode}
\long\def\UTF@two@octets@noexpand#1#2{\noexpand#2\noexpand}
\long\def\UTF@three@octets@noexpand#1#2#3{\noexpand#2\noexpand#3\noexpand}
\long\def\UTF@four@octets@noexpand#1#2#3#4{\noexpand#2\noexpand#3\noexpand#4\noexpand}
%    \end{macrocode}
% \end{macro}
% \end{macro}
% \end{macro}

% \begin{macro}{\UTFviii@two@octets@string}
% \begin{macro}{\UTFviii@three@octets@string}
% \begin{macro}{\UTFviii@four@octets@string}
%    And the same with \cs{string} for use in \cs{csname} constructions.
%    \begin{macrocode}
\long\def\UTF@two@octets@string#1#2{\string#2\string}
\long\def\UTF@three@octets@string#1#2#3{\string#2\string#3\string}
\long\def\UTF@four@octets@string#1#2#3#4{\string#2\string#3\string#4\string}
%    \end{macrocode}
% \end{macro}
% \end{macro}
% \end{macro}


% \begin{macro}{\UTFviii@two@octets@combine}
% \begin{macro}{\UTFviii@three@octets@combine}
% \begin{macro}{\UTFviii@four@octets@combine}
%    From the arguments a control sequence with a name of the form
%    \verb=u8:#1#2...= is constructed where the |#i| ($i>1$) are the
%    arguments and |#1| is the starting octet (as a \TeX{} active character
%    token).  Since some or even all of these characters are active
%    we need to use |\string| when building
%    the \cs{csname}.
%
%    The \cs{csname} thus constructed can of course be undefined but to
%    avoid producing an unhelpful low-level undefined command error we
%    pass it to |\UTFviii@defined| which is responsible for producing
%    a more sensible error message (not yet done!!).  If, however, it is
%    defined we simply execute the thing (which should then expand to
%    an encoding specific internal \LaTeX{} form).
%    \begin{macrocode}
\long\def\UTFviii@two@octets@combine#1#2{\expandafter
    \UTFviii@defined\csname u8:\string#1\string#2\endcsname}
%    \end{macrocode}
%
%    \begin{macrocode}
\long\def\UTFviii@three@octets@combine#1#2#3{\expandafter
    \UTFviii@defined\csname u8:\string#1\string#2\string#3\endcsname}
%    \end{macrocode}
%
%    \begin{macrocode}
\long\def\UTFviii@four@octets@combine#1#2#3#4{\expandafter
    \UTFviii@defined\csname u8:\string#1\string#2\string#3\string#4\endcsname}
%    \end{macrocode}
% \end{macro}
% \end{macro}
% \end{macro}
%
%
% \begin{macro}{\UTFviii@defined}
%    This tests whether its argument is different from |\relax|: it
%    either calls for a sensible error message (not done), or it gets
%    the |\fi| out of the way (in case the command has arguments) and
%    executes it.
%    \begin{macrocode}
\def\UTFviii@defined#1{%
  \ifx#1\relax
%    \end{macrocode}
% Test if the sequence is invalid UTF-8 or valid UTF-8 but without
% a \LaTeX\ definition.
%    \begin{macrocode}
     \if\relax\expandafter\UTFviii@checkseq\string#1\relax\relax
%    \end{macrocode}
%    The endline character has a special definition within the
%    inputenc package (it is gobbling spaces). For this reason we
%    can't produce multiline strings without some precaution.
% \changes{v1.1b}{2004/02/09}{No newlines allowed in error messages}
% \changes{v1.1g}{2005/09/27}{Real spaces do not show up so use \cs{space}}
% \changes{v1.1o}{2015/08/28}{Show Unicode number of character in hex}
% \changes{v1.2a}{2018/03/24}{Error message improved for non-UTF-8 sequences}%
%    \begin{macrocode}
      \UTFviii@undefined@err{#1}%
%    \end{macrocode}
%
%    \begin{macrocode}
     \else
      \PackageError{inputenc}{Invalid UTF-8 byte sequence}%
                             \UTFviii@invalid@help
     \fi
%    \end{macrocode}
%
%    \begin{macrocode}
  \else\expandafter
    #1%
  \fi
}
%    \end{macrocode}
% \end{macro}
%
% \begin{macro}{\UTFviii@invalid@err}
% \begin{macro}{\UTFviii@invalid@help}
% \changes{v1.2a}{2018/03/24}{Macro added}%
% \changes{v1.2f}{2018/10/05}{Show invalid byte in hex}%
%    \begin{macrocode}
\def\UTFviii@invalid@err#1{%
 \PackageError{inputenc}{Invalid UTF-8 byte "\UTFviii@hexnumber{`#1}}%
                        \UTFviii@invalid@help}
%    \end{macrocode}
%
%    \begin{macrocode}
\def\UTFviii@invalid@help{%
   The document does not appear to be in UTF-8 encoding.\MessageBreak
   Try adding \noexpand\UseRawInputEncoding as the first line of the file\MessageBreak
   or specify an encoding such as \noexpand\usepackage[latin1]{inputenc}\MessageBreak
   in the document preamble.\MessageBreak
   Alternatively, save the file in UTF-8 using your editor or another tool}
%    \end{macrocode}
% \end{macro}
% \end{macro}
%
% \begin{macro}{\UTFviii@undefined@err}
% \changes{v1.2a}{2018/03/24}{Macro added}%
%    \begin{macrocode}
\def\UTFviii@undefined@err#1{%
  \PackageError{inputenc}{Unicode character \expandafter
                          \UTFviii@splitcsname\string#1\relax
                          \MessageBreak
                          not set up for use with LaTeX}%
                         {You may provide a definition with\MessageBreak
                          \noexpand\DeclareUnicodeCharacter}%
   }
%    \end{macrocode}
% \end{macro}
%
%
% \begin{macro}{\UTFviii@checkseq}
% \begin{macro}{\UTFviii@check@continue}
% \changes{v1.2a}{2018/03/24}{Macro added}%
% Check that the csname consists of a valid UTF-8 sequence.
%    \begin{macrocode}
\def\UTFviii@checkseq#1:#2#3{%
 \ifnum`#2<"80 %
   \ifx\relax#3\else1\fi
 \else
   \ifnum`#2<"C0 %
     1 %
   \else
     \expandafter\expandafter\expandafter\UTFviii@check@continue
     \expandafter\expandafter\expandafter#3%
   \fi
  \fi}
%    \end{macrocode}
%
%    \begin{macrocode}
\def\UTFviii@check@continue#1{%
  \ifx\relax#1%
  \else
  \ifnum`#1<"80 1\else\ifnum`#1>"BF 1\fi\fi
  \expandafter\UTFviii@check@continue
  \fi
}
%    \end{macrocode}
% \end{macro}
% \end{macro}
%
%
% \begin{macro}{\UTFviii@loop}
%    This bit of code derived from \texttt{xmltex}  defines the active
%    character corresponding to starting octets to call |\UTFviii@two@octets|
%    etc as appropriate. 
%    The starting octet itself is passed directly as the first
%    argument, the others are picked up later en route.
%
%    The |\UTFviii@loop| loops through the numbers starting at
%    |\count@| and ending at |\@tempcnta|${} - 1$, each time executing
%    the code in |\UTFviii@tmp|.
%
%    All this is done in a group so that temporary catcode changes
%    etc.~vanish after everything is set up.
%
%    \begin{macrocode}
\begingroup
\catcode`\~13
\catcode`\"12
%    \end{macrocode}
%
%    \begin{macrocode}
\def\UTFviii@loop{%
  \uccode`\~\count@
  \uppercase\expandafter{\UTFviii@tmp}%
  \advance\count@\@ne
  \ifnum\count@<\@tempcnta
  \expandafter\UTFviii@loop
  \fi}
%    \end{macrocode}
%
% Handle the single byte control characters.
% \changes{v1.2a}{2018/03/24}{Loop over C0 controls added}%
% C0 controls are valid UTF-8 but defined to give the ``Character not defined error''
% They may be defined with |\DeclareUnicodeCharacter|.
%    \begin{macrocode}
    \def\UTFviii@tmp{\xdef~{\noexpand\UTFviii@undefined@err{:\string~}}}
% 0 ^^@ null
    \count@"1
    \@tempcnta9
% 9 ^^I tab
% 10 ^^J nl
\UTFviii@loop
    \count@11
    \@tempcnta12
\UTFviii@loop
% 12 ^^L
% 13 ^^M
    \count@14
    \@tempcnta32
\UTFviii@loop
%    \end{macrocode}
%
%
% Bytes with leading bits |10| are not valid UTF-8 starting bytes
%    \begin{macrocode}
    \count@"80
    \@tempcnta"C2
    \def\UTFviii@tmp{\xdef~{\noexpand\UTFviii@invalid@err\string~}}
\UTFviii@loop
%    \end{macrocode}
%
%    Setting up 2-byte UTF-8: The starting bytes is passed as an
%    active character so that it can be reprocessed later!
%    \begin{macrocode}
    \count@"C2
    \@tempcnta"E0
    \def\UTFviii@tmp{\xdef~{\noexpand\UTFviii@two@octets\noexpand~}}
\UTFviii@loop
%    \end{macrocode}
%
%    Setting up 3-byte UTF-8:
%    \begin{macrocode}
    \count@"E0
    \@tempcnta"F0
    \def\UTFviii@tmp{\xdef~{\noexpand\UTFviii@three@octets\noexpand~}}
\UTFviii@loop
%    \end{macrocode}
%
%    Setting up 4-byte UTF-8:
% \changes{v1.2e}{2018/09/28}{Fix "F4 lead byte}%
%    \begin{macrocode}
    \count@"F0
    \@tempcnta"F5
    \def\UTFviii@tmp{\xdef~{\noexpand\UTFviii@four@octets\noexpand~}}
\UTFviii@loop
%    \end{macrocode}
%
% Bytes above F4 are not valid UTF-8 starting bytes as they would encode numbers beyond
% the Unicode range
%    \begin{macrocode}
    \count@"F5
    \@tempcnta"100
    \def\UTFviii@tmp{\xdef~{\noexpand\UTFviii@invalid@err\string~}}
\UTFviii@loop
%    \end{macrocode}
%
%    \begin{macrocode}
\endgroup
%    \end{macrocode}
%
% \end{macro}
%
%    For this case we must disable the warning generated by
%    \texttt{inputenc} if it doesn't see any new |\DeclareInputText|
%    commands.
%    \begin{macrocode}
\@inpenc@test
%    \end{macrocode}
%
%
%    If this file (\texttt{utf8.def}) is not being read while setting
%    up \texttt{inputenc}, i.e.~in the preamble, but when
%    |\inputencoding| is called somewhere within the document, we do not
%    need to input the specific Unicode mappings again. We therefore
%    stop reading the file at this point.
%    \begin{macrocode}
\ifx\@begindocumenthook\@undefined
  \makeatother
%    \end{macrocode}
%    The |\fi| must be on the same line as |\endinput| or else it will
%    never be seen!
%    \begin{macrocode}
  \endinput \fi
%    \end{macrocode}
%
%
% \subsection{Mapping Unicode codes to \LaTeX{} internal forms}
%
%
% \begin{macro}{\DeclareUnicodeCharacter}
%    The |\DeclareUnicodeCharacter| declaration defines a mapping from
%    a Unicode character code point to a \LaTeX{} internal form. The first
%    argument is the Unicode number as hexadecimal digits and the second is
%    the actual \LaTeX{} internal form.
%
%    We start by making sure that some characters have the right
%    |\catcode| when they are used in the definitions below.
%    \begin{macrocode}
\begingroup
\catcode`\"=12
\catcode`\<=12
\catcode`\.=12
\catcode`\,=12
\catcode`\;=12
\catcode`\!=12
\catcode`\~=13
%    \end{macrocode}
%
%    \begin{macrocode}
\gdef\DeclareUnicodeCharacter#1#2{%
  \count@"#1\relax
  \wlog{ \space\space defining Unicode char U+#1 (decimal \the\count@)}%
  \begingroup
%    \end{macrocode}
%    Next we do the parsing of the number stored in |\count@| and assign the
%    result to |\UTFviii@tmp|. Actually all this could be done in-line,
%    the macro |\parse@XML@charref| is only there to extend this code
%    to parsing Unicode numbers in other contexts one day (perhaps).
%    \begin{macrocode}
    \parse@XML@charref
%    \end{macrocode}
%
%    Here is an example of what is happening, for the pair \texttt{"C2 "A3}
%    (which is the utf8 representation for the character \textsterling{}).
%    After |\parse@XML@charref| we have, stored in |\UTFviii@tmp|, a
%    single command with two character tokens as arguments:
%    \begin{quote}
%      [$t_{C2}$ and $t_{A3}$ are the characters corresponding to these
%      two octets]\\
%      |\UTFviii@two@octets| $t_{\rm C2}t_{\rm A3}$
%    \end{quote}
%    what we actually need to produce is a definition of the form
%    \begin{quote}
%    |\def\u8:|$t_{\rm C2}$$t_{\rm A3}$ |{|\textit{\LaTeX{} internal form}|}|\,.
%    \end{quote}
%    So here we temporarily redefine the prefix commands
%    |\UTFviii@two@octets|, etc.~to
%    generate the csname that we wish to define>  the |\string|s are
%    added in case these tokens are still active.
%    \begin{macrocode}
    \def\UTFviii@two@octets##1##2{\csname u8:##1\string##2\endcsname}%
    \def\UTFviii@three@octets##1##2##3{\csname u8:##1%
                                     \string##2\string##3\endcsname}%
    \def\UTFviii@four@octets##1##2##3##4{\csname u8:##1%
                           \string##2\string##3\string##4\endcsname}%
%    \end{macrocode}
%    Now we simply:-) need to use the right number of |\expandafter|s to
%    finally construct the definition: expanding |\UTFviii@tmp| once to get
%    its contents, a second time to replace the prefix command by its
%    |\csname| expansion, and a third time to turn the expansion into
%    a csname after which the |\gdef| finally gets applied.
%    We add an irrelevant |\IeC| and braces around the definition, in
%    order to avoid any space after the command being gobbled up
%    when the text is written out to an auxiliary file (see
%    \texttt{inputenc} for further details
%    \begin{macrocode}
    \expandafter\expandafter\expandafter
    \expandafter\expandafter\expandafter
    \expandafter
     \gdef\UTFviii@tmp{\IeC{#2}}%
   \endgroup
}
%    \end{macrocode}
% \end{macro}
%
%
% \begin{macro}{\parse@XML@charref}
%    This macro parses a Unicode number (decimal) and returns its
%    UTF-8 representation as a sequence of non-active \TeX{} character
%    tokens. In the
%    original code it had two arguments delimited by \texttt{;} here,
%    however, we supply the Unicode number implicitly.
%    \begin{macrocode}
\gdef\parse@XML@charref{%
%    \end{macrocode}
%    We need to keep a few things local, mainly the |\uccode|'s that
%    are set up below. However, the group originally used here is
%    actually unnecessary since we call this macro only within another
%    group; but it will be important to restore the group if this
%    macro gets used for other purposes.
%    \begin{macrocode}
%  \begingroup
%    \end{macrocode}
%    The original code from \texttt{xmltex} supported the convention that a
%    Unicode slot number could be given either as a decimal or as a
%    hexadecimal (by starting with \texttt{x}).  We do not do this so
%    this code is also removed.  This could be reactivated if one
%    wants to support document commands that accept Unicode numbers
%    (but then the first case needs to be changed from an error
%    message back to something more useful again).
%    \begin{macrocode}
%  \uppercase{\count@\if x\noexpand#1"\else#1\fi#2}\relax
%    \end{macrocode}
%    As |\count@| already contains the right value we make
%    |\parse@XML@charref| work without arguments.
% \changes{v1.1g}{2005/09/27}{Real spaces do not show up so use \cs{space}}
% \changes{v1.2a}{2018/03/24}{Allow control characters if active}
% In the case single byte UTF-8 sequences, only allow definition if
% the character os already active.  The definition of |\UTFviii@tmp|
% looks slightly strange but is designed for the sequence of |\expandafter|
% in |\DeclareUnicodeCharacter|.
%
%    \begin{macrocode}
  \ifnum\count@<"A0\relax
    \ifnum\catcode\count@=13
      \uccode`\~=\count@\uppercase{\def\UTFviii@tmp{\@empty\@empty~}}%
    \else
      \PackageError{inputenc}%
                   {Cannot define non-active Unicode char value < 00A0}%
                   \@eha
      \def\UTFviii@tmp{\UTFviii@tmp}%
    \fi
%    \end{macrocode}
%    The code below is derived from \texttt{xmltex} and generates the UTF-8 byte sequence
%    for the number in |\count@|.
%
%    The reverse operation (just used in error messages)
%    has now been added as \cs{decode@UTFviii}.
%    \begin{macrocode}
  \else\ifnum\count@<"800\relax
     \parse@UTFviii@a,%
     \parse@UTFviii@b C\UTFviii@two@octets.,%
  \else\ifnum\count@<"10000\relax
     \parse@UTFviii@a;%
     \parse@UTFviii@a,%
     \parse@UTFviii@b E\UTFviii@three@octets.{,;}%
   \else
%    \end{macrocode}
%
% Test added here for out of range values, the 4-octet definitions are still set up
% so that |\DeclareUnicodeCharacter| does something sensible if the user scrolls
% past this error.
%    \begin{macrocode}
     \ifnum\count@>"10FFFF\relax
           \PackageError{inputenc}%
                {\UTFviii@hexnumber\count@\space too large for Unicode}%
                {Values between 0 and 10FFFF are permitted}%
      \fi
%    \end{macrocode}
%
%    \begin{macrocode}
     \parse@UTFviii@a;%
     \parse@UTFviii@a,%
     \parse@UTFviii@a!%
     \parse@UTFviii@b F\UTFviii@four@octets.{!,;}%
    \fi
    \fi
  \fi
%  \endgroup
}
%    \end{macrocode}
% \end{macro}
%
% \begin{macro}{\parse@UTFviii@a}
%    \ldots so somebody else can document this part :-)
% \changes{v1.1b}{2004/02/09}{Space in the wrong place \cs{count @64}}
%    \begin{macrocode}
\gdef\parse@UTFviii@a#1{%
     \@tempcnta\count@
     \divide\count@ 64
     \@tempcntb\count@
     \multiply\count@ 64
     \advance\@tempcnta-\count@
     \advance\@tempcnta 128
     \uccode`#1\@tempcnta
     \count@\@tempcntb}
%    \end{macrocode}
% \end{macro}
%
% \begin{macro}{\parse@UTFviii@b}
%    \ldots same here
%    \begin{macrocode}
\gdef\parse@UTFviii@b#1#2#3#4{%
     \advance\count@ "#10\relax
     \uccode`#3\count@
     \uppercase{\gdef\UTFviii@tmp{#2#3#4}}}
%    \end{macrocode}
% \end{macro}
%
% \begin{macro}{\decode@UTFviii}
% \changes{v1.1o}{2015/08/28}{Macro added}
%    In the reverse direction, take a sequence of octets(bytes)
%    representing a character in UTF-8 and construct the Unicode number.
%    The sequence is terminated by |\relax|.
%
%    In this version, if the sequence is not valid UTF-8 you probably
%    get a low level arithmetic error from |\numexpr| or stray characters
%    at the end. Getting a better error message would be somewhat expensive.
%    As the main use is for reporting characters in messages, this is done
%    just using expansion, so |\numexpr| is used, A stub returning 0 is defined
%    if |\numexpr| is not available.
%    \begin{macrocode}
\ifx\numexpr\@undefined
%    \end{macrocode}
%
%    \begin{macrocode}
\gdef\decode@UTFviii#1{0}
%    \end{macrocode}
%
%    \begin{macrocode}
\else
%    \end{macrocode}
%
% If the input is malformed UTF-8 there may not be enough closing ) so
% add 5 so there are always some remaining then cleanup and remove
% any remaining ones at the end. This avoids |\numexpr| parse errors
% while outputting a package error.
%    \begin{macrocode}
\gdef\decode@UTFviii#1\relax{%
  \expandafter\UTFviii@cleanup
    \the\numexpr\dec@de@UTFviii#1\relax)))))\@empty}
%    \end{macrocode}
%
%    \begin{macrocode}
\gdef\UTFviii@cleanup#1)#2\@empty{#1}
%    \end{macrocode}
%
%    \begin{macrocode}
\gdef\dec@de@UTFviii#1{%
\ifx\relax#1%
\else
  \ifnum`#1>"EF
    ((((`#1-"F0)%
  \else
    \ifnum`#1>"DF
      (((`#1-"E0)%
    \else
      \ifnum`#1>"BF
        ((`#1-"C0)%
      \else
        \ifnum`#1>"7F
        )*64+(`#1-"80)%
        \else
        +`#1 %
        \fi
      \fi
    \fi
  \fi
  \expandafter\dec@de@UTFviii
\fi}
%    \end{macrocode}
%
%    \begin{macrocode}
\fi
%    \end{macrocode}
% \end{macro}
%
% \begin{macro}{\UTFviii@hexnumber}
% \changes{v1.1o}{2015/08/28}{Macro added}
%    Convert a number to a sequence of uppercase hex digits.
% If |\numexpr| is not available, it returns its argument unchanged.
%    \begin{macrocode}
\ifx\numexpr\@undefined
%    \end{macrocode}
%    \begin{macrocode}
\global\let\UTFviii@hexnumber\@firstofone
\global\UTFviii@hexdigit\hexnumber@
%    \end{macrocode}
%    \begin{macrocode}
\else
%    \end{macrocode}
%    \begin{macrocode}
\gdef\UTFviii@hexnumber#1{%
\ifnum#1>15 %
\expandafter\UTFviii@hexnumber\expandafter{\the\numexpr(#1-8)/16\relax}%
\fi
\UTFviii@hexdigit{\numexpr#1\ifnum#1>0-((#1-8)/16)*16\fi\relax}%
}
%    \end{macrocode}
%
% Almost but not quite |\hexnumber@|.
%    \begin{macrocode}
\gdef\UTFviii@hexdigit#1{\ifcase\numexpr#1\relax
  0\or1\or2\or3\or4\or5\or6\or7\or8\or9\or
  A\or B\or C\or D\or E\or F\fi}
%    \end{macrocode}
%
%    \begin{macrocode}
\fi
%    \end{macrocode}
% \end{macro}
%
% \begin{macro}{\UTFviii@splitcsname}
% \changes{v1.1o}{2015/08/28}{Macro added}
% \begin{macro}{\UTFviii@hexcodepoint}
% \changes{v1.2e}{2018/09/28}{Macro added: Format codepoint properly}
%    Split a csname representing a unicode character and return
%    the character and the unicode number in hex.
%    \begin{macrocode}
\gdef\UTFviii@hexcodepoint#1{U+%
 \ifnum#1<16 0\fi
 \ifnum#1<256 0\fi
 \ifnum#1<4096 0\fi
 \UTFviii@hexnumber{#1}%
}%
\gdef\UTFviii@splitcsname#1:#2\relax{%
%    \end{macrocode}
% \changes{v1.2b}{2018/03/26}{add percent as \cs{endlinechar} not -1 in the format}%
% Need to pre-expand the argument to ensure cleanup in case of mal-formed UTF-8.
%    \begin{macrocode}
#2 (\expandafter\UTFviii@hexcodepoint\expandafter{%
                     \the\numexpr\decode@UTFviii#2\relax})%
}
%    \end{macrocode}
% \end{macro}
% \end{macro}
%
%    \begin{macrocode}
\endgroup
%    \end{macrocode}
%
%    \begin{macrocode}
\@onlypreamble\DeclareUnicodeCharacter
%    \end{macrocode}
%    These are preamble only as long as we don't support Unicode
%    charrefs in documents.
%    \begin{macrocode}
\@onlypreamble\parse@XML@charref
\@onlypreamble\parse@UTFviii@a
\@onlypreamble\parse@UTFviii@b
%    \end{macrocode}
%
%
% \subsection{Loading Unicode mappings at begin document}
%
% The original plan was to set up the UTF-8 support at
% |\begin{document}|; but then any text characters used in the preamble
% (as people do even though advised against it) would fail in one way or
% the other.
% So the implementation was changed and the Unicode definition files
% for already defined encodings are loaded here.
%
%    We loop through all defined font encodings
%    (stored in |\cdp@list|) and for each load a file
%    \textit{name}\texttt{enc.dfu} if it exist. That file is then
%    supposed to contain |\DeclareUnicodeCharacter| declarations.
%    \begin{macrocode}
\begingroup
  \def\cdp@elt#1#2#3#4{%
    \wlog{Now handling font encoding #1 ...}%
    \lowercase{%
        \InputIfFileExists{#1enc.dfu}}%
           {\wlog{... processing UTF-8 mapping file for font %
                     encoding #1}%
%    \end{macrocode}
% \changes{v1.1m}{2008/04/05}{Ensure we don't lose spaces in the log}
%    The previous line is written to the log with the newline char being
%    ignored (thus not producing a space). Therefore either everything has to
%    be on a single input line or some special care must be taken.  From this
%    point on we ignore spaces again, i.e., while we are reading the
%    \texttt{.dfu} file. The |\endgroup| below will restore it again.
% \changes{v1.1d}{2004/05/08}{Explicitly set catcode of space}
% \changes{v1.1g}{2005/09/27}{We lost the ``false'' case}
%    \begin{macrocode}
            \catcode`\ 9\relax}%
          {\wlog{... no UTF-8 mapping file for font encoding #1}}%
  }
  \cdp@list
\endgroup
%    \end{macrocode}
% However, we don't know if there are font encodings still to be
% loaded (either with \texttt{fontenc} or directly with |\input| by
% some package). Font encoding files are loaded only if the
% corresponding encoding has not been loaded yet, and they always
% begin with |\DeclareFontEncoding|. We now redefine the internal
% kernel version of the latter to load the Unicode file if available.
%
%    \begin{macrocode}
\def\DeclareFontEncoding@#1#2#3{%
  \expandafter
  \ifx\csname T@#1\endcsname\relax
    \def\cdp@elt{\noexpand\cdp@elt}%
    \xdef\cdp@list{\cdp@list\cdp@elt{#1}%
                    {\default@family}{\default@series}%
                    {\default@shape}}%
    \expandafter\let\csname#1-cmd\endcsname\@changed@cmd
    \begingroup
      \wlog{Now handling font encoding #1 ...}%
      \lowercase{%
        \InputIfFileExists{#1enc.dfu}}%
           {\wlog{... processing UTF-8 mapping file for font %
                      encoding #1}}%
           {\wlog{... no UTF-8 mapping file for font encoding #1}}%
    \endgroup
  \else
     \@font@info{Redeclaring font encoding #1}%
  \fi
  \global\@namedef{T@#1}{#2}%
  \global\@namedef{M@#1}{\default@M#3}%
  \xdef\LastDeclaredEncoding{#1}%
  }
%</utf8>
%    \end{macrocode}
%
%
%
% \section{Mapping characters ---\newline based on font (glyph) encodings}
%
% This section is a first attempt to provide Unicode definitions for
% characters whose standard glyphs are currently provided by the
% standard \LaTeX{} font-encodings |T1|, |OT1|, etc. They are by
% no means completed and need checking.
%
% For example, one should check the already existing input encodings
%    for glyphs that may in fact be available and required,
%    e.g.~\texttt{latin4} has a number of glyphs with the |\=|
%    accent. Since the |T1| encoding does not provide such glyphs,
%    these characters are not listed below (yet).
%
% The list below was generated by looking at the current \LaTeX{} font
%    encoding files, e.g., \texttt{t1enc.def} and using the work by
%    Sebastian Rahtz (in \texttt{ucharacters.sty}) with a few
%    modifications. In combinations such as |\^\i| the preferred form
%    is that and not |\^i|.
%
% This list has been built from several sources, obviously including
% the Unicode Standard itself. These sources include Passive \TeX{} by
% Sebastian Rahtz, the \texttt{unicode}
% package by Dominique P. G. Unruh (mainly for Latin encodings) and
% \texttt{text4ht} by Eitan Gurari (for Cyrillic ones).
%
% Note that it strictly follows the Mittelbach principles for
% input character encodings: thus it offers no support for using utf8
% representations of math symbols such as $\times$ or $\div$ (in math mode).
%
%
% \subsection{About the table itself}
%
% In addition to generating individual files, the table below is, at present,
% a one-one (we think) partial relationship between the (ill-defined) set
% of LICRs and the Unicode slots "0080 to "FFFF.  At present these entries
% are used only to define a collection of partial mappings from Unicode
% slots to LICRs; each of these mappings becomes full if we add an exception
% value (`not defined') to the set of LICRs.
%
% It is probably not essential for the relationship in the full table to be
% one-one; this raises questions such as: the exact role of LICRs;
% the formal relationships on the set of LICRs; the (non-mathematical)
% relationship between
% LICRs and Unicode (which has its own somewhat fuzzy equivalences);
% and ultimately what a character is and what a character representation
% and/or name is.
%
% It is unclear the extent to which entries in this table should
% resemble the closely related ones in the 8-bit \texttt{inputenc} files.
% The Unicode standard claims that the first 256 slots `are' ASCII and
% Latin-1.
%
% Of course, \TeX{} itself typically does not treat even many perfectly
% `normal text' 7-bit slots as text characters, so it is unclear
% whether \LaTeX{} should even attempt to deal in any consistent way with
% those Unicode slots that are not definitive text characters.
%
%
% \subsection{The mapping table}
%
% \begingroup
% \hfuzz=12pt ^^A just because
%
% Note that the first argument must be a hex-digit number greater
% than \texttt{00BF} and at most \texttt{10FFFF}.
%
% There are few notes about inconsistencies etc at the end of the table.
%
% \changes{v1.1o}{2015/08/28}{Add U+00A0 and U+00AD}
% \changes{v1.1q}{2015/12/02}{Add remaining latin uses of accents in T1}
% \changes{v1.1r}{2015/12/03}{Add some more ogonek cases}
% \changes{v1.1s}{2016/01/11}{Add some more caron and acute}
% \changes{v1.1t}{2017/01/28}{Add caron cominations for GgYy}
%    \begin{macrocode}
%<all,t1,ot1,ly1>\DeclareUnicodeCharacter{00A0}{\nobreakspace}
%<all,t1,ot1,ly1>\DeclareUnicodeCharacter{00A1}{\textexclamdown}
%<all,ts1,ly1>\DeclareUnicodeCharacter{00A2}{\textcent}
%<all,ts1,t1,ot1,ly1>\DeclareUnicodeCharacter{00A3}{\textsterling}
%<all,x2,ts1,t2c,t2b,t2a,ly1,lcy>\DeclareUnicodeCharacter{00A4}{\textcurrency}
%<all,ts1,ly1>\DeclareUnicodeCharacter{00A5}{\textyen}
%<all,ts1,ly1>\DeclareUnicodeCharacter{00A6}{\textbrokenbar}
%<all,x2,ts1,t2c,t2b,t2a,oms,ly1>\DeclareUnicodeCharacter{00A7}{\textsection}
%<all,ts1>\DeclareUnicodeCharacter{00A8}{\textasciidieresis}
%<all,ts1,utf8>\DeclareUnicodeCharacter{00A9}{\textcopyright}
%<all,ts1,ly1,utf8>\DeclareUnicodeCharacter{00AA}{\textordfeminine}
%<*all,x2,t2c,t2b,t2a,t1,ot2,ly1,lcy>
\DeclareUnicodeCharacter{00AB}{\guillemotleft}
%</all,x2,t2c,t2b,t2a,t1,ot2,ly1,lcy>
%<all,ts1>\DeclareUnicodeCharacter{00AC}{\textlnot}
%<all,t1,ot1,ly1>\DeclareUnicodeCharacter{00AD}{\-}
%<all,ts1,ly1,utf8>\DeclareUnicodeCharacter{00AE}{\textregistered}
%<all,ts1>\DeclareUnicodeCharacter{00AF}{\textasciimacron}
%<all,ts1,ly1>\DeclareUnicodeCharacter{00B0}{\textdegree}
%<all,ts1>\DeclareUnicodeCharacter{00B1}{\textpm}
%<all,ts1>\DeclareUnicodeCharacter{00B2}{\texttwosuperior}
%<all,ts1>\DeclareUnicodeCharacter{00B3}{\textthreesuperior}
%<all,ts1>\DeclareUnicodeCharacter{00B4}{\textasciiacute}
%<all,ts1,ly1>\DeclareUnicodeCharacter{00B5}{\textmu} % micro sign
%<all,ts1,oms,ly1>\DeclareUnicodeCharacter{00B6}{\textparagraph}
%<all,oms,ts1,ly1>\DeclareUnicodeCharacter{00B7}{\textperiodcentered}
%<all,ot1>\DeclareUnicodeCharacter{00B8}{\c\ }
%<all,ts1>\DeclareUnicodeCharacter{00B9}{\textonesuperior}
%<all,ts1,ly1,utf8>\DeclareUnicodeCharacter{00BA}{\textordmasculine}
%<*all,x2,t2c,t2b,t2a,t1,ot2,ly1,lcy>
\DeclareUnicodeCharacter{00BB}{\guillemotright}
%</all,x2,t2c,t2b,t2a,t1,ot2,ly1,lcy>
%<all,ts1,ly1>\DeclareUnicodeCharacter{00BC}{\textonequarter}
%<all,ts1,ly1>\DeclareUnicodeCharacter{00BD}{\textonehalf}
%<all,ts1,ly1>\DeclareUnicodeCharacter{00BE}{\textthreequarters}
%<all,t1,ot1,ly1>\DeclareUnicodeCharacter{00BF}{\textquestiondown}
%<all,t1,ly1>\DeclareUnicodeCharacter{00C0}{\@tabacckludge`A}
%<all,t1,ly1>\DeclareUnicodeCharacter{00C1}{\@tabacckludge'A}
%<all,t1,ly1>\DeclareUnicodeCharacter{00C2}{\^A}
%<all,t1,ly1>\DeclareUnicodeCharacter{00C3}{\~A}
%<all,t1,ly1>\DeclareUnicodeCharacter{00C4}{\"A}
%<all,t1,ot1,ly1>\DeclareUnicodeCharacter{00C5}{\r A}
%<all,t1,ot1,ly1,lcy>\DeclareUnicodeCharacter{00C6}{\AE}
%<all,t1,ly1>\DeclareUnicodeCharacter{00C7}{\c C}
%<all,t1,ly1>\DeclareUnicodeCharacter{00C8}{\@tabacckludge`E}
%<all,t1,ly1>\DeclareUnicodeCharacter{00C9}{\@tabacckludge'E}
%<all,t1,ly1>\DeclareUnicodeCharacter{00CA}{\^E}
%<all,t1,ly1>\DeclareUnicodeCharacter{00CB}{\"E}
%<all,t1,ly1>\DeclareUnicodeCharacter{00CC}{\@tabacckludge`I}
%<all,t1,ly1>\DeclareUnicodeCharacter{00CD}{\@tabacckludge'I}
%<all,t1,ly1>\DeclareUnicodeCharacter{00CE}{\^I}
%<all,t1,ly1>\DeclareUnicodeCharacter{00CF}{\"I}
%<all,t1,ly1>\DeclareUnicodeCharacter{00D0}{\DH}
%<all,t1,ly1>\DeclareUnicodeCharacter{00D1}{\~N}
%<all,t1,ly1>\DeclareUnicodeCharacter{00D2}{\@tabacckludge`O}
%<all,t1,ly1>\DeclareUnicodeCharacter{00D3}{\@tabacckludge'O}
%<all,t1,ly1>\DeclareUnicodeCharacter{00D4}{\^O}
%<all,t1,ly1>\DeclareUnicodeCharacter{00D5}{\~O}
%<all,t1,ly1>\DeclareUnicodeCharacter{00D6}{\"O}
%<all,ts1>\DeclareUnicodeCharacter{00D7}{\texttimes}
%<all,t1,ot1,ly1,lcy>\DeclareUnicodeCharacter{00D8}{\O}
%<all,t1,ly1>\DeclareUnicodeCharacter{00D9}{\@tabacckludge`U}
%<all,t1,ly1>\DeclareUnicodeCharacter{00DA}{\@tabacckludge'U}
%<all,t1,ly1>\DeclareUnicodeCharacter{00DB}{\^U}
%<all,t1,ly1>\DeclareUnicodeCharacter{00DC}{\"U}
%<all,t1,ly1>\DeclareUnicodeCharacter{00DD}{\@tabacckludge'Y}
%<all,t1,ly1>\DeclareUnicodeCharacter{00DE}{\TH}
%<all,t1,ot1,ly1,lcy>\DeclareUnicodeCharacter{00DF}{\ss}
%<all,t1,ly1>\DeclareUnicodeCharacter{00E0}{\@tabacckludge`a}
%<all,t1,ly1>\DeclareUnicodeCharacter{00E1}{\@tabacckludge'a}
%<all,t1,ly1>\DeclareUnicodeCharacter{00E2}{\^a}
%<all,t1,ly1>\DeclareUnicodeCharacter{00E3}{\~a}
%<all,t1,ly1>\DeclareUnicodeCharacter{00E4}{\"a}
%<all,t1,ly1>\DeclareUnicodeCharacter{00E5}{\r a}
%<all,t1,ot1,ly1,lcy>\DeclareUnicodeCharacter{00E6}{\ae}
%<all,t1,ly1>\DeclareUnicodeCharacter{00E7}{\c c}
%<all,t1,ly1>\DeclareUnicodeCharacter{00E8}{\@tabacckludge`e}
%<all,t1,ly1>\DeclareUnicodeCharacter{00E9}{\@tabacckludge'e}
%<all,t1,ly1>\DeclareUnicodeCharacter{00EA}{\^e}
%<all,t1,ly1>\DeclareUnicodeCharacter{00EB}{\"e}
%<all,t1,ot1,ly1>\DeclareUnicodeCharacter{00EC}{\@tabacckludge`\i}
%<all,t1,ot1,ly1>\DeclareUnicodeCharacter{00ED}{\@tabacckludge'\i}
%<all,t1,ot1,ly1>\DeclareUnicodeCharacter{00EE}{\^\i}
%<all,t1,ot1,ly1>\DeclareUnicodeCharacter{00EF}{\"\i}
%<all,t1,ly1>\DeclareUnicodeCharacter{00F0}{\dh}
%<all,t1,ly1>\DeclareUnicodeCharacter{00F1}{\~n}
%<all,t1,ly1>\DeclareUnicodeCharacter{00F2}{\@tabacckludge`o}
%<all,t1,ly1>\DeclareUnicodeCharacter{00F3}{\@tabacckludge'o}
%<all,t1,ly1>\DeclareUnicodeCharacter{00F4}{\^o}
%<all,t1,ly1>\DeclareUnicodeCharacter{00F5}{\~o}
%<all,t1,ly1>\DeclareUnicodeCharacter{00F6}{\"o}
%<all,ts1>\DeclareUnicodeCharacter{00F7}{\textdiv}
%<all,t1,ot1,ly1,lcy>\DeclareUnicodeCharacter{00F8}{\o}
%<all,t1,ly1>\DeclareUnicodeCharacter{00F9}{\@tabacckludge`u}
%<all,t1,ly1>\DeclareUnicodeCharacter{00FA}{\@tabacckludge'u}
%<all,t1,ly1>\DeclareUnicodeCharacter{00FB}{\^u}
%<all,t1,ly1>\DeclareUnicodeCharacter{00FC}{\"u}
%<all,t1,ly1>\DeclareUnicodeCharacter{00FD}{\@tabacckludge'y}
%<all,t1,ly1>\DeclareUnicodeCharacter{00FE}{\th}
%<all,t1,ly1>\DeclareUnicodeCharacter{00FF}{\"y}
%<all,t1>\DeclareUnicodeCharacter{0100}{\@tabacckludge=A}
%<all,t1>\DeclareUnicodeCharacter{0101}{\@tabacckludge=a}
%<all,t1>\DeclareUnicodeCharacter{0102}{\u A}
%<all,t1>\DeclareUnicodeCharacter{0103}{\u a}
%<all,t1>\DeclareUnicodeCharacter{0104}{\k A}
%<all,t1>\DeclareUnicodeCharacter{0105}{\k a}
%<all,t1>\DeclareUnicodeCharacter{0106}{\@tabacckludge'C}
%<all,t1>\DeclareUnicodeCharacter{0107}{\@tabacckludge'c}
%<all,t1>\DeclareUnicodeCharacter{0108}{\^C}
%<all,t1>\DeclareUnicodeCharacter{0109}{\^c}
%<all,t1>\DeclareUnicodeCharacter{010A}{\.C}
%<all,t1>\DeclareUnicodeCharacter{010B}{\.c}
%<all,t1>\DeclareUnicodeCharacter{010C}{\v C}
%<all,t1>\DeclareUnicodeCharacter{010D}{\v c}
%<all,t1>\DeclareUnicodeCharacter{010E}{\v D}
%<all,t1>\DeclareUnicodeCharacter{010F}{\v d}
%<all,t1>\DeclareUnicodeCharacter{0110}{\DJ}
%<all,t1>\DeclareUnicodeCharacter{0111}{\dj}
%<all,t1>\DeclareUnicodeCharacter{0112}{\@tabacckludge=E}
%<all,t1>\DeclareUnicodeCharacter{0113}{\@tabacckludge=e}
%<all,t1>\DeclareUnicodeCharacter{0114}{\u E}
%<all,t1>\DeclareUnicodeCharacter{0115}{\u e}
%<all,t1>\DeclareUnicodeCharacter{0116}{\.E}
%<all,t1>\DeclareUnicodeCharacter{0117}{\.e}
%<all,t1>\DeclareUnicodeCharacter{0118}{\k E}
%<all,t1>\DeclareUnicodeCharacter{0119}{\k e}
%<all,t1>\DeclareUnicodeCharacter{011A}{\v E}
%<all,t1>\DeclareUnicodeCharacter{011B}{\v e}
%<all,t1>\DeclareUnicodeCharacter{011C}{\^G}
%<all,t1>\DeclareUnicodeCharacter{011D}{\^g}
%<all,t1>\DeclareUnicodeCharacter{011E}{\u G}
%<all,t1>\DeclareUnicodeCharacter{011F}{\u g}
%<all,t1>\DeclareUnicodeCharacter{0120}{\.G}
%<all,t1>\DeclareUnicodeCharacter{0121}{\.g}
%<all,t1>\DeclareUnicodeCharacter{0122}{\c G}
%<all,t1>\DeclareUnicodeCharacter{0123}{\c g}
%<all,t1>\DeclareUnicodeCharacter{0124}{\^H}
%<all,t1>\DeclareUnicodeCharacter{0125}{\^h}
%<all,t1>\DeclareUnicodeCharacter{0128}{\~I}
%<all,t1>\DeclareUnicodeCharacter{0129}{\~\i}
%<all,t1>\DeclareUnicodeCharacter{012A}{\@tabacckludge=I}
%<all,t1>\DeclareUnicodeCharacter{012B}{\@tabacckludge=\i}
%<all,t1>\DeclareUnicodeCharacter{012C}{\u I}
%<all,t1>\DeclareUnicodeCharacter{012D}{\u\i}
%<all,t1>\DeclareUnicodeCharacter{012E}{\k I}
%    \end{macrocode}
% \changes{v1.2g}{2019/02/06}{Corrected 012F (github/122)}
%    \begin{macrocode}
%<all,t1>\DeclareUnicodeCharacter{012F}{\k i}
%<all,t1>\DeclareUnicodeCharacter{0130}{\.I}
%<all,t2c,t2b,t2a,t1,ot2,ot1,ly1,lcy>\DeclareUnicodeCharacter{0131}{\i}
%<all,t1>\DeclareUnicodeCharacter{0132}{\IJ}
%<all,t1>\DeclareUnicodeCharacter{0133}{\ij}
%<all,t1>\DeclareUnicodeCharacter{0134}{\^J}
%<all,t1>\DeclareUnicodeCharacter{0135}{\^\j}
%<all,t1>\DeclareUnicodeCharacter{0136}{\c K}
%<all,t1>\DeclareUnicodeCharacter{0137}{\c k}
%<all,t1>\DeclareUnicodeCharacter{0139}{\@tabacckludge'L}
%<all,t1>\DeclareUnicodeCharacter{013A}{\@tabacckludge'l}
%<all,t1>\DeclareUnicodeCharacter{013B}{\c L}
%<all,t1>\DeclareUnicodeCharacter{013C}{\c l}
%<all,t1>\DeclareUnicodeCharacter{013D}{\v L}
%<all,t1>\DeclareUnicodeCharacter{013E}{\v l}
%<all,t1,ot1,ly1>\DeclareUnicodeCharacter{0141}{\L}
%<all,t1,ot1,ly1>\DeclareUnicodeCharacter{0142}{\l}
%<all,t1>\DeclareUnicodeCharacter{0143}{\@tabacckludge'N}
%<all,t1>\DeclareUnicodeCharacter{0144}{\@tabacckludge'n}
%<all,t1>\DeclareUnicodeCharacter{0145}{\c N}
%<all,t1>\DeclareUnicodeCharacter{0146}{\c n}
%<all,t1>\DeclareUnicodeCharacter{0147}{\v N}
%<all,t1>\DeclareUnicodeCharacter{0148}{\v n}
%<all,t1>\DeclareUnicodeCharacter{014A}{\NG}
%<all,t1>\DeclareUnicodeCharacter{014B}{\ng}
%<all,t1>\DeclareUnicodeCharacter{014C}{\@tabacckludge=O}
%<all,t1>\DeclareUnicodeCharacter{014D}{\@tabacckludge=o}
%<all,t1>\DeclareUnicodeCharacter{014E}{\u O}
%<all,t1>\DeclareUnicodeCharacter{014F}{\u o}
%<all,t1>\DeclareUnicodeCharacter{0150}{\H O}
%<all,t1>\DeclareUnicodeCharacter{0151}{\H o}
%<all,t1,ot1,ly1,lcy>\DeclareUnicodeCharacter{0152}{\OE}
%<all,t1,ot1,ly1,lcy>\DeclareUnicodeCharacter{0153}{\oe}
%<all,t1>\DeclareUnicodeCharacter{0154}{\@tabacckludge'R}
%<all,t1>\DeclareUnicodeCharacter{0155}{\@tabacckludge'r}
%<all,t1>\DeclareUnicodeCharacter{0156}{\c R}
%<all,t1>\DeclareUnicodeCharacter{0157}{\c r}
%<all,t1>\DeclareUnicodeCharacter{0158}{\v R}
%<all,t1>\DeclareUnicodeCharacter{0159}{\v r}
%<all,t1>\DeclareUnicodeCharacter{015A}{\@tabacckludge'S}
%<all,t1>\DeclareUnicodeCharacter{015B}{\@tabacckludge's}
%<all,t1>\DeclareUnicodeCharacter{015C}{\^S}
%<all,t1>\DeclareUnicodeCharacter{015D}{\^s}
%<all,t1>\DeclareUnicodeCharacter{015E}{\c S}
%<all,t1>\DeclareUnicodeCharacter{015F}{\c s}
%<all,t1,ly1>\DeclareUnicodeCharacter{0160}{\v S}
%<all,t1,ly1>\DeclareUnicodeCharacter{0161}{\v s}
%<all,t1>\DeclareUnicodeCharacter{0162}{\c T}
%<all,t1>\DeclareUnicodeCharacter{0163}{\c t}
%<all,t1>\DeclareUnicodeCharacter{0164}{\v T}
%<all,t1>\DeclareUnicodeCharacter{0165}{\v t}
%<all,t1>\DeclareUnicodeCharacter{0168}{\~U}
%<all,t1>\DeclareUnicodeCharacter{0169}{\~u}
%<all,t1>\DeclareUnicodeCharacter{016A}{\@tabacckludge=U}
%<all,t1>\DeclareUnicodeCharacter{016B}{\@tabacckludge=u}
%<all,t1>\DeclareUnicodeCharacter{016C}{\u U}
%<all,t1>\DeclareUnicodeCharacter{016D}{\u u}
%<all,t1>\DeclareUnicodeCharacter{016E}{\r U}
%<all,t1>\DeclareUnicodeCharacter{016F}{\r u}
%<all,t1>\DeclareUnicodeCharacter{0170}{\H U}
%<all,t1>\DeclareUnicodeCharacter{0171}{\H u}
%<all,t1>\DeclareUnicodeCharacter{0172}{\k U}
%<all,t1>\DeclareUnicodeCharacter{0173}{\k u}
%    \end{macrocode}
%
% \changes{v1.1p}{2015/09/07}{Welsh circumflex combinations}
%    \begin{macrocode}
%<all,t1,ot1,ly1>\DeclareUnicodeCharacter{0174}{\^W}
%<all,t1,ot1,ly1>\DeclareUnicodeCharacter{0175}{\^w}
%<all,t1,ot1,ly1>\DeclareUnicodeCharacter{0176}{\^Y}
%<all,t1,ot1,ly1>\DeclareUnicodeCharacter{0177}{\^y}
%<all,t1,ly1>\DeclareUnicodeCharacter{0178}{\"Y}
%<all,t1>\DeclareUnicodeCharacter{0179}{\@tabacckludge'Z}
%<all,t1>\DeclareUnicodeCharacter{017A}{\@tabacckludge'z}
%<all,t1>\DeclareUnicodeCharacter{017B}{\.Z}
%<all,t1>\DeclareUnicodeCharacter{017C}{\.z}
%<all,t1,ly1>\DeclareUnicodeCharacter{017D}{\v Z}
%<all,t1,ly1>\DeclareUnicodeCharacter{017E}{\v z}
%<all,ts1,ly1>\DeclareUnicodeCharacter{0192}{\textflorin}
%    \end{macrocode}
% \changes{v1.1s}{2016/01/11}{add 01CD-01F4}
%    \begin{macrocode}
%<all,t1>\DeclareUnicodeCharacter{01CD}{\v A}
%<all,t1>\DeclareUnicodeCharacter{01CE}{\v a}
%<all,t1>\DeclareUnicodeCharacter{01CF}{\v I}
%<all,t1>\DeclareUnicodeCharacter{01D0}{\v \i}
%<all,t1>\DeclareUnicodeCharacter{01D1}{\v O}
%<all,t1>\DeclareUnicodeCharacter{01D2}{\v o}
%<all,t1>\DeclareUnicodeCharacter{01D3}{\v U}
%<all,t1>\DeclareUnicodeCharacter{01D4}{\v u}
%<all,t1>\DeclareUnicodeCharacter{01E2}{\@tabacckludge=\AE}
%<all,t1>\DeclareUnicodeCharacter{01E3}{\@tabacckludge=\ae}
%<all,t1>\DeclareUnicodeCharacter{01E6}{\v G}
%<all,t1>\DeclareUnicodeCharacter{01E7}{\v g}
%<all,t1>\DeclareUnicodeCharacter{01E8}{\v K}
%<all,t1>\DeclareUnicodeCharacter{01E9}{\v k}
%<all,t1>\DeclareUnicodeCharacter{01EA}{\k O}
%<all,t1>\DeclareUnicodeCharacter{01EB}{\k o}
%<all,t1>\DeclareUnicodeCharacter{01F0}{\v\j}
%<all,t1>\DeclareUnicodeCharacter{01F4}{\@tabacckludge'G}
%<all,t1>\DeclareUnicodeCharacter{01F5}{\@tabacckludge'g}
%    \end{macrocode}
% \changes{v1.1o}{2015/08/28}{comma accent latex/4414}
%    \begin{macrocode}
%<all,t1,ot1,ly1>\DeclareUnicodeCharacter{0218}{\textcommabelow S}
%<all,t1,ot1,ly1>\DeclareUnicodeCharacter{0219}{\textcommabelow s}
%<all,t1,ot1,ly1>\DeclareUnicodeCharacter{021A}{\textcommabelow T}
%<all,t1,ot1,ly1>\DeclareUnicodeCharacter{021B}{\textcommabelow t}
%    \end{macrocode}
%
%    \begin{macrocode}
%<all,t1>\DeclareUnicodeCharacter{0232}{\@tabacckludge=Y}
%<all,t1>\DeclareUnicodeCharacter{0233}{\@tabacckludge=y}
%<all,t2c,t2b,t2a,t1,ot2,ot1,ly1,lcy>\DeclareUnicodeCharacter{0237}{\j}
%<all,ly1,utf8>\DeclareUnicodeCharacter{02C6}{\textasciicircum}
%<all,ts1>\DeclareUnicodeCharacter{02C7}{\textasciicaron}
%<all,ly1,utf8>\DeclareUnicodeCharacter{02DC}{\textasciitilde}
%<all,ts1>\DeclareUnicodeCharacter{02D8}{\textasciibreve}
%<all,t1>\DeclareUnicodeCharacter{02D9}{\.{}}
%<all,t1>\DeclareUnicodeCharacter{02DB}{\k{}}
%<all,ts1>\DeclareUnicodeCharacter{02DD}{\textacutedbl}
%    \end{macrocode}
%    The Cyrillic code points have been recently checked (2007) and extended
%    and corrected by Matthias Noe (\verb=a9931078@unet.univie.ac.at=) --- thanks.
% \changes{v1.1j}{2007/11/09}{Added a few new unicode decls in cyrillic (pr/3988)}
% \changes{v1.1k}{2007/11/11}{Added and further unicode decls in cyrillic}
% \changes{v1.1n}{2015/06/27}{correct accent https://tex.stackexchange.com/q/252521}
%    \begin{macrocode}
%<*all,x2,t2c,t2b,t2a,ot2,lcy>
\DeclareUnicodeCharacter{0400}{\@tabacckludge`\CYRE}
%</all,x2,t2c,t2b,t2a,ot2,lcy>
%<all,x2,t2c,t2b,t2a,ot2,lcy>\DeclareUnicodeCharacter{0401}{\CYRYO}
%<all,x2,t2a,ot2>\DeclareUnicodeCharacter{0402}{\CYRDJE}
%<*all,x2,t2c,t2b,t2a,ot2,lcy>
\DeclareUnicodeCharacter{0403}{\@tabacckludge'\CYRG}
%</all,x2,t2c,t2b,t2a,ot2,lcy>
%<all,x2,t2a,ot2,lcy>\DeclareUnicodeCharacter{0404}{\CYRIE}
%<all,x2,t2c,t2b,t2a,ot2>\DeclareUnicodeCharacter{0405}{\CYRDZE}
%<all,x2,t2c,t2b,t2a,ot2,lcy>\DeclareUnicodeCharacter{0406}{\CYRII}
%<all,x2,t2a,lcy>\DeclareUnicodeCharacter{0407}{\CYRYI}
%<all,x2,t2c,t2b,t2a,ot2>\DeclareUnicodeCharacter{0408}{\CYRJE}
%<all,x2,t2b,t2a,ot2>\DeclareUnicodeCharacter{0409}{\CYRLJE}
%<all,x2,t2b,t2a,ot2>\DeclareUnicodeCharacter{040A}{\CYRNJE}
%<all,x2,t2a,ot2>\DeclareUnicodeCharacter{040B}{\CYRTSHE}
%<*all,x2,t2c,t2b,t2a,ot2,lcy>
\DeclareUnicodeCharacter{040C}{\@tabacckludge'\CYRK}
\DeclareUnicodeCharacter{040D}{\@tabacckludge`\CYRI}
%</all,x2,t2c,t2b,t2a,ot2,lcy>
%<all,x2,t2b,t2a,lcy>\DeclareUnicodeCharacter{040E}{\CYRUSHRT}
%<all,x2,t2c,t2a,ot2>\DeclareUnicodeCharacter{040F}{\CYRDZHE}
%<*all,x2,t2c,t2b,t2a,ot2,lcy>
\DeclareUnicodeCharacter{0410}{\CYRA}
\DeclareUnicodeCharacter{0411}{\CYRB}
\DeclareUnicodeCharacter{0412}{\CYRV}
\DeclareUnicodeCharacter{0413}{\CYRG}
\DeclareUnicodeCharacter{0414}{\CYRD}
\DeclareUnicodeCharacter{0415}{\CYRE}
\DeclareUnicodeCharacter{0416}{\CYRZH}
\DeclareUnicodeCharacter{0417}{\CYRZ}
\DeclareUnicodeCharacter{0418}{\CYRI}
\DeclareUnicodeCharacter{0419}{\CYRISHRT}
\DeclareUnicodeCharacter{041A}{\CYRK}
\DeclareUnicodeCharacter{041B}{\CYRL}
\DeclareUnicodeCharacter{041C}{\CYRM}
\DeclareUnicodeCharacter{041D}{\CYRN}
\DeclareUnicodeCharacter{041E}{\CYRO}
\DeclareUnicodeCharacter{041F}{\CYRP}
\DeclareUnicodeCharacter{0420}{\CYRR}
\DeclareUnicodeCharacter{0421}{\CYRS}
\DeclareUnicodeCharacter{0422}{\CYRT}
\DeclareUnicodeCharacter{0423}{\CYRU}
\DeclareUnicodeCharacter{0424}{\CYRF}
\DeclareUnicodeCharacter{0425}{\CYRH}
\DeclareUnicodeCharacter{0426}{\CYRC}
\DeclareUnicodeCharacter{0427}{\CYRCH}
\DeclareUnicodeCharacter{0428}{\CYRSH}
\DeclareUnicodeCharacter{0429}{\CYRSHCH}
\DeclareUnicodeCharacter{042A}{\CYRHRDSN}
\DeclareUnicodeCharacter{042B}{\CYRERY}
\DeclareUnicodeCharacter{042C}{\CYRSFTSN}
\DeclareUnicodeCharacter{042D}{\CYREREV}
\DeclareUnicodeCharacter{042E}{\CYRYU}
\DeclareUnicodeCharacter{042F}{\CYRYA}
\DeclareUnicodeCharacter{0430}{\cyra}
\DeclareUnicodeCharacter{0431}{\cyrb}
\DeclareUnicodeCharacter{0432}{\cyrv}
\DeclareUnicodeCharacter{0433}{\cyrg}
\DeclareUnicodeCharacter{0434}{\cyrd}
\DeclareUnicodeCharacter{0435}{\cyre}
\DeclareUnicodeCharacter{0436}{\cyrzh}
\DeclareUnicodeCharacter{0437}{\cyrz}
\DeclareUnicodeCharacter{0438}{\cyri}
\DeclareUnicodeCharacter{0439}{\cyrishrt}
\DeclareUnicodeCharacter{043A}{\cyrk}
\DeclareUnicodeCharacter{043B}{\cyrl}
\DeclareUnicodeCharacter{043C}{\cyrm}
\DeclareUnicodeCharacter{043D}{\cyrn}
\DeclareUnicodeCharacter{043E}{\cyro}
\DeclareUnicodeCharacter{043F}{\cyrp}
\DeclareUnicodeCharacter{0440}{\cyrr}
\DeclareUnicodeCharacter{0441}{\cyrs}
\DeclareUnicodeCharacter{0442}{\cyrt}
\DeclareUnicodeCharacter{0443}{\cyru}
\DeclareUnicodeCharacter{0444}{\cyrf}
\DeclareUnicodeCharacter{0445}{\cyrh}
\DeclareUnicodeCharacter{0446}{\cyrc}
\DeclareUnicodeCharacter{0447}{\cyrch}
\DeclareUnicodeCharacter{0448}{\cyrsh}
\DeclareUnicodeCharacter{0449}{\cyrshch}
\DeclareUnicodeCharacter{044A}{\cyrhrdsn}
\DeclareUnicodeCharacter{044B}{\cyrery}
\DeclareUnicodeCharacter{044C}{\cyrsftsn}
\DeclareUnicodeCharacter{044D}{\cyrerev}
\DeclareUnicodeCharacter{044E}{\cyryu}
\DeclareUnicodeCharacter{044F}{\cyrya}
\DeclareUnicodeCharacter{0450}{\@tabacckludge`\cyre}
\DeclareUnicodeCharacter{0451}{\cyryo}
%</all,x2,t2c,t2b,t2a,ot2,lcy>
%<all,x2,t2a,ot2>\DeclareUnicodeCharacter{0452}{\cyrdje}
%<*all,x2,t2c,t2b,t2a,ot2,lcy>
\DeclareUnicodeCharacter{0453}{\@tabacckludge'\cyrg}
%</all,x2,t2c,t2b,t2a,ot2,lcy>
%<all,x2,t2a,ot2,lcy>\DeclareUnicodeCharacter{0454}{\cyrie}
%<all,x2,t2c,t2b,t2a,ot2>\DeclareUnicodeCharacter{0455}{\cyrdze}
%<all,x2,t2c,t2b,t2a,ot2,lcy>\DeclareUnicodeCharacter{0456}{\cyrii}
%<all,x2,t2a,lcy>\DeclareUnicodeCharacter{0457}{\cyryi}
%<all,x2,t2c,t2b,t2a,ot2>\DeclareUnicodeCharacter{0458}{\cyrje}
%<all,x2,t2b,t2a,ot2>\DeclareUnicodeCharacter{0459}{\cyrlje}
%<all,x2,t2b,t2a,ot2>\DeclareUnicodeCharacter{045A}{\cyrnje}
%<all,x2,t2a,ot2>\DeclareUnicodeCharacter{045B}{\cyrtshe}
%<*all,x2,t2c,t2b,t2a,ot2,lcy>
\DeclareUnicodeCharacter{045C}{\@tabacckludge'\cyrk}
\DeclareUnicodeCharacter{045D}{\@tabacckludge`\cyri}
%</all,x2,t2c,t2b,t2a,ot2,lcy>
%<all,x2,t2b,t2a,lcy>\DeclareUnicodeCharacter{045E}{\cyrushrt}
%<all,x2,t2c,t2a,ot2>\DeclareUnicodeCharacter{045F}{\cyrdzhe}
%<all,x2,ot2>\DeclareUnicodeCharacter{0462}{\CYRYAT}
%<all,x2,ot2>\DeclareUnicodeCharacter{0463}{\cyryat}
%<all,x2>\DeclareUnicodeCharacter{046A}{\CYRBYUS}
%<all,x2>\DeclareUnicodeCharacter{046B}{\cyrbyus}
%    \end{macrocode}
%    The next two declarations are questionable, the encoding definition
%    should probably contain |\CYROTLD| and |\cyrotld|. Or alternatively, if
%    the characters in the X2 encodings are really meant to represent the
%    historical characters in Ux0472 and Ux0473 (they look like them) then
%    they would need to change instead.
%
%    However, their looks are probably a font designers decision and the next
%    two mappings are wrong or rather the names in OT2 should change for
%    consistency.
%
%    On the other hand the names |\CYROTLD| are somewhat questionable as the
%    Unicode standard only describes ``Cyrillic barred O'' while |TLD| refers
%    to a tilde (which is more less what the ``Cyrillic FITA looks according
%    to the Unicode book).
%    \begin{macrocode}
%<all,ot2>\DeclareUnicodeCharacter{0472}{\CYRFITA}
%<all,ot2>\DeclareUnicodeCharacter{0473}{\cyrfita}
%    \end{macrocode}
%
%    \begin{macrocode}
%<all,x2,ot2>\DeclareUnicodeCharacter{0474}{\CYRIZH}
%<all,x2,ot2>\DeclareUnicodeCharacter{0475}{\cyrizh}
%    \end{macrocode}
%    While the double grave accent seems to exist in X2, T2A, T2B and T2C
%    encoding, the letter izhitsa exists only in X2 and OT2. Therefore,
%    izhitsa with double grave seems to be possible only using X2.
%    \begin{macrocode}
%<all,x2>\DeclareUnicodeCharacter{0476}{\C\CYRIZH}
%<all,x2>\DeclareUnicodeCharacter{0477}{\C\cyrizh}
%    \end{macrocode}
%
%    \begin{macrocode}
%<all,t2c>\DeclareUnicodeCharacter{048C}{\CYRSEMISFTSN}
%<all,t2c>\DeclareUnicodeCharacter{048D}{\cyrsemisftsn}
%<all,t2c>\DeclareUnicodeCharacter{048E}{\CYRRTICK}
%<all,t2c>\DeclareUnicodeCharacter{048F}{\cyrrtick}
%<all,x2,t2a,lcy>\DeclareUnicodeCharacter{0490}{\CYRGUP}
%<all,x2,t2a,lcy>\DeclareUnicodeCharacter{0491}{\cyrgup}
%<all,x2,t2b,t2a>\DeclareUnicodeCharacter{0492}{\CYRGHCRS}
%<all,x2,t2b,t2a>\DeclareUnicodeCharacter{0493}{\cyrghcrs}
%<all,x2,t2c,t2b>\DeclareUnicodeCharacter{0494}{\CYRGHK}
%<all,x2,t2c,t2b>\DeclareUnicodeCharacter{0495}{\cyrghk}
%<all,x2,t2b,t2a>\DeclareUnicodeCharacter{0496}{\CYRZHDSC}
%<all,x2,t2b,t2a>\DeclareUnicodeCharacter{0497}{\cyrzhdsc}
%<all,x2,t2a>\DeclareUnicodeCharacter{0498}{\CYRZDSC}
%<all,x2,t2a>\DeclareUnicodeCharacter{0499}{\cyrzdsc}
%<all,x2,t2c,t2b,t2a>\DeclareUnicodeCharacter{049A}{\CYRKDSC}
%<all,x2,t2c,t2b,t2a>\DeclareUnicodeCharacter{049B}{\cyrkdsc}
%<all,x2,t2a>\DeclareUnicodeCharacter{049C}{\CYRKVCRS}
%<all,x2,t2a>\DeclareUnicodeCharacter{049D}{\cyrkvcrs}
%<all,x2,t2c>\DeclareUnicodeCharacter{049E}{\CYRKHCRS}
%<all,x2,t2c>\DeclareUnicodeCharacter{049F}{\cyrkhcrs}
%<all,x2,t2a>\DeclareUnicodeCharacter{04A0}{\CYRKBEAK}
%<all,x2,t2a>\DeclareUnicodeCharacter{04A1}{\cyrkbeak}
%<all,x2,t2c,t2b,t2a>\DeclareUnicodeCharacter{04A2}{\CYRNDSC}
%<all,x2,t2c,t2b,t2a>\DeclareUnicodeCharacter{04A3}{\cyrndsc}
%<all,x2,t2b,t2a>\DeclareUnicodeCharacter{04A4}{\CYRNG}
%<all,x2,t2b,t2a>\DeclareUnicodeCharacter{04A5}{\cyrng}
%<all,x2,t2c>\DeclareUnicodeCharacter{04A6}{\CYRPHK}
%<all,x2,t2c>\DeclareUnicodeCharacter{04A7}{\cyrphk}
%<all,x2,t2c>\DeclareUnicodeCharacter{04A8}{\CYRABHHA}
%<all,x2,t2c>\DeclareUnicodeCharacter{04A9}{\cyrabhha}
%<all,x2,t2a>\DeclareUnicodeCharacter{04AA}{\CYRSDSC}
%<all,x2,t2a>\DeclareUnicodeCharacter{04AB}{\cyrsdsc}
%<all,x2,t2c>\DeclareUnicodeCharacter{04AC}{\CYRTDSC}
%<all,x2,t2c>\DeclareUnicodeCharacter{04AD}{\cyrtdsc}
%<all,x2,t2b,t2a>\DeclareUnicodeCharacter{04AE}{\CYRY}
%<all,x2,t2b,t2a>\DeclareUnicodeCharacter{04AF}{\cyry}
%<all,x2,t2a>\DeclareUnicodeCharacter{04B0}{\CYRYHCRS}
%<all,x2,t2a>\DeclareUnicodeCharacter{04B1}{\cyryhcrs}
%<all,x2,t2c,t2b,t2a>\DeclareUnicodeCharacter{04B2}{\CYRHDSC}
%<all,x2,t2c,t2b,t2a>\DeclareUnicodeCharacter{04B3}{\cyrhdsc}
%<all,x2,t2c>\DeclareUnicodeCharacter{04B4}{\CYRTETSE}
%<all,x2,t2c>\DeclareUnicodeCharacter{04B5}{\cyrtetse}
%<all,x2,t2c,t2b,t2a>\DeclareUnicodeCharacter{04B6}{\CYRCHRDSC}
%<all,x2,t2c,t2b,t2a>\DeclareUnicodeCharacter{04B7}{\cyrchrdsc}
%<all,x2,t2a>\DeclareUnicodeCharacter{04B8}{\CYRCHVCRS}
%<all,x2,t2a>\DeclareUnicodeCharacter{04B9}{\cyrchvcrs}
%<all,x2,t2c,t2b,t2a>\DeclareUnicodeCharacter{04BA}{\CYRSHHA}
%<all,x2,t2c,t2b,t2a>\DeclareUnicodeCharacter{04BB}{\cyrshha}
%<all,x2,t2c>\DeclareUnicodeCharacter{04BC}{\CYRABHCH}
%<all,x2,t2c>\DeclareUnicodeCharacter{04BD}{\cyrabhch}
%<all,x2,t2c>\DeclareUnicodeCharacter{04BE}{\CYRABHCHDSC}
%<all,x2,t2c>\DeclareUnicodeCharacter{04BF}{\cyrabhchdsc}
%    \end{macrocode}
%    The character |\CYRpalochka| is not defined by OT2 and LCY. However it is
%    looking identical to |\CYRII| and the Unicode standard explicitly refers
%    to that (and to Latin I). So perhaps those encodings could get an alias?
%    On the other hand, why are there two distinct slots in the T2 encodings
%    even though they are so pressed for space? Perhaps they don't always look
%    alike.
%    \begin{macrocode}
%<all,x2,t2c,t2b,t2a>\DeclareUnicodeCharacter{04C0}{\CYRpalochka}
%    \end{macrocode}
%
%    \begin{macrocode}
%<all,x2,t2c,t2b,t2a,ot2,lcy>\DeclareUnicodeCharacter{04C1}{\U\CYRZH}
%<all,x2,t2c,t2b,t2a,ot2,lcy>\DeclareUnicodeCharacter{04C2}{\U\cyrzh}
%<all,x2,t2b>\DeclareUnicodeCharacter{04C3}{\CYRKHK}
%<all,x2,t2b>\DeclareUnicodeCharacter{04C4}{\cyrkhk}
%    \end{macrocode}
%    According to the Unicode standard Ux04C5 should be an L with ``tail'' not
%    with descender (which also exists as Ux04A2) but it looks as if the char
%    names do not make this distinction). Should they?
%    \begin{macrocode}
%<all,x2,t2c,t2b>\DeclareUnicodeCharacter{04C5}{\CYRLDSC}
%<all,x2,t2c,t2b>\DeclareUnicodeCharacter{04C6}{\cyrldsc}
%    \end{macrocode}
%
%    \begin{macrocode}
%<all,x2,t2c,t2b>\DeclareUnicodeCharacter{04C7}{\CYRNHK}
%<all,x2,t2c,t2b>\DeclareUnicodeCharacter{04C8}{\cyrnhk}
%<all,x2,t2b>\DeclareUnicodeCharacter{04CB}{\CYRCHLDSC}
%<all,x2,t2b>\DeclareUnicodeCharacter{04CC}{\cyrchldsc}
%    \end{macrocode}
%    According to the Unicode standard Ux04CD should be an M with ``tail'' not
%    with descender. However this time there is no M with descender in the
%    Unicode standard.
%    \begin{macrocode}
%<all,x2,t2c>\DeclareUnicodeCharacter{04CD}{\CYRMDSC}
%<all,x2,t2c>\DeclareUnicodeCharacter{04CE}{\cyrmdsc}
%    \end{macrocode}
%
%    \begin{macrocode}
%<all,x2,t2c,t2b,t2a,ot2,lcy>\DeclareUnicodeCharacter{04D0}{\U\CYRA}
%<all,x2,t2c,t2b,t2a,ot2,lcy>\DeclareUnicodeCharacter{04D1}{\U\cyra}
%<all,x2,t2c,t2b,t2a,ot2,lcy>\DeclareUnicodeCharacter{04D2}{\"\CYRA}
%<all,x2,t2c,t2b,t2a,ot2,lcy>\DeclareUnicodeCharacter{04D3}{\"\cyra}
%<all,x2,t2a>\DeclareUnicodeCharacter{04D4}{\CYRAE}
%<all,x2,t2a>\DeclareUnicodeCharacter{04D5}{\cyrae}
%<all,x2,t2c,t2b,t2a,ot2,lcy>\DeclareUnicodeCharacter{04D6}{\U\CYRE}
%<all,x2,t2c,t2b,t2a,ot2,lcy>\DeclareUnicodeCharacter{04D7}{\U\cyre}
%<all,x2,t2c,t2b,t2a>\DeclareUnicodeCharacter{04D8}{\CYRSCHWA}
%<all,x2,t2c,t2b,t2a>\DeclareUnicodeCharacter{04D9}{\cyrschwa}
%<all,x2,t2c,t2b,t2a>\DeclareUnicodeCharacter{04DA}{\"\CYRSCHWA}
%<all,x2,t2c,t2b,t2a>\DeclareUnicodeCharacter{04DB}{\"\cyrschwa}
%<all,x2,t2c,t2b,t2a,ot2,lcy>\DeclareUnicodeCharacter{04DC}{\"\CYRZH}
%<all,x2,t2c,t2b,t2a,ot2,lcy>\DeclareUnicodeCharacter{04DD}{\"\cyrzh}
%<all,x2,t2c,t2b,t2a,ot2,lcy>\DeclareUnicodeCharacter{04DE}{\"\CYRZ}
%<all,x2,t2c,t2b,t2a,ot2,lcy>\DeclareUnicodeCharacter{04DF}{\"\cyrz}
%<all,x2,t2c,t2b>\DeclareUnicodeCharacter{04E0}{\CYRABHDZE}
%<all,x2,t2c,t2b>\DeclareUnicodeCharacter{04E1}{\cyrabhdze}
%<all,x2,t2c,t2b,t2a,ot2,lcy>\DeclareUnicodeCharacter{04E2}{\@tabacckludge=\CYRI}
%<all,x2,t2c,t2b,t2a,ot2,lcy>\DeclareUnicodeCharacter{04E3}{\@tabacckludge=\cyri}
%<all,x2,t2c,t2b,t2a,ot2,lcy>\DeclareUnicodeCharacter{04E4}{\"\CYRI}
%<all,x2,t2c,t2b,t2a,ot2,lcy>\DeclareUnicodeCharacter{04E5}{\"\cyri}
%<all,x2,t2c,t2b,t2a,ot2,lcy>\DeclareUnicodeCharacter{04E6}{\"\CYRO}
%<all,x2,t2c,t2b,t2a,ot2,lcy>\DeclareUnicodeCharacter{04E7}{\"\cyro}
%<all,x2,t2c,t2b,t2a>\DeclareUnicodeCharacter{04E8}{\CYROTLD}
%<all,x2,t2c,t2b,t2a>\DeclareUnicodeCharacter{04E9}{\cyrotld}
%<all,x2,t2c,t2b,t2a,ot2,lcy>\DeclareUnicodeCharacter{04EC}{\"\CYREREV}
%<all,x2,t2c,t2b,t2a,ot2,lcy>\DeclareUnicodeCharacter{04ED}{\"\cyrerev}
%<all,x2,t2c,t2b,t2a,ot2,lcy>\DeclareUnicodeCharacter{04EE}{\@tabacckludge=\CYRU}
%<all,x2,t2c,t2b,t2a,ot2,lcy>\DeclareUnicodeCharacter{04EF}{\@tabacckludge=\cyru}
%<all,x2,t2c,t2b,t2a,ot2,lcy>\DeclareUnicodeCharacter{04F0}{\"\CYRU}
%<all,x2,t2c,t2b,t2a,ot2,lcy>\DeclareUnicodeCharacter{04F1}{\"\cyru}
%<all,x2,t2c,t2b,t2a,ot2,lcy>\DeclareUnicodeCharacter{04F2}{\H\CYRU}
%<all,x2,t2c,t2b,t2a,ot2,lcy>\DeclareUnicodeCharacter{04F3}{\H\cyru}
%<all,x2,t2c,t2b,t2a,ot2,lcy>\DeclareUnicodeCharacter{04F4}{\"\CYRCH}
%<all,x2,t2c,t2b,t2a,ot2,lcy>\DeclareUnicodeCharacter{04F5}{\"\cyrch}
%<all,x2,t2b>\DeclareUnicodeCharacter{04F6}{\CYRGDSC}
%<all,x2,t2b>\DeclareUnicodeCharacter{04F7}{\cyrgdsc}
%<all,x2,t2c,t2b,t2a,ot2,lcy>\DeclareUnicodeCharacter{04F8}{\"\CYRERY}
%<all,x2,t2c,t2b,t2a,ot2,lcy>\DeclareUnicodeCharacter{04F9}{\"\cyrery}
%<all,t2b>\DeclareUnicodeCharacter{04FA}{\CYRGDSCHCRS}
%<all,t2b>\DeclareUnicodeCharacter{04FB}{\cyrgdschcrs}
%<all,x2,t2b>\DeclareUnicodeCharacter{04FC}{\CYRHHK}
%<all,x2,t2b>\DeclareUnicodeCharacter{04FD}{\cyrhhk}
%<all,t2b>\DeclareUnicodeCharacter{04FE}{\CYRHHCRS}
%<all,t2b>\DeclareUnicodeCharacter{04FF}{\cyrhhcrs}
%<all,ts1>\DeclareUnicodeCharacter{0E3F}{\textbaht}
%<all,t1>\DeclareUnicodeCharacter{1E02}{\.B}
%<all,t1>\DeclareUnicodeCharacter{1E03}{\.b}
%<all,t1>\DeclareUnicodeCharacter{1E9E}{\SS}
%<all,x2,t2c,t2b,t2a,t1,utf8>\DeclareUnicodeCharacter{200C}{\textcompwordmark}
%    \end{macrocode}
% \changes{v1.1s}{2016/02/28}{Add more hyphens and dashes}
%    \begin{macrocode}
%<all,t1>\DeclareUnicodeCharacter{2010}{-}
%<all,t1>\DeclareUnicodeCharacter{2011}{\mbox{-}}
%    \end{macrocode}
% U+2012 should be the width of a digit, endash is OK in many fonts including cm.
%    \begin{macrocode}
%<all,t1>\DeclareUnicodeCharacter{2012}{\textendash}
%<*all,x2,t2c,t2b,t2a,t1,ot2,ot1,ly1,lcy>
\DeclareUnicodeCharacter{2013}{\textendash}
\DeclareUnicodeCharacter{2014}{\textemdash}
%    \end{macrocode}
% U+2015 is Horizontal bar
%    \begin{macrocode}
%<all,t1>\DeclareUnicodeCharacter{2015}{\textemdash}
%</all,x2,t2c,t2b,t2a,t1,ot2,ot1,ly1,lcy>
%<all,ts1>\DeclareUnicodeCharacter{2016}{\textbardbl}
%<*all,x2,t2c,t2b,t2a,t1,ot2,ot1,lcy>
\DeclareUnicodeCharacter{2018}{\textquoteleft}
\DeclareUnicodeCharacter{2019}{\textquoteright}
%</all,x2,t2c,t2b,t2a,t1,ot2,ot1,lcy>
%<all,t1>\DeclareUnicodeCharacter{201A}{\quotesinglbase}
%<*all,x2,t2c,t2b,t2a,t1,ot2,ot1,ly1,lcy>
\DeclareUnicodeCharacter{201C}{\textquotedblleft}
\DeclareUnicodeCharacter{201D}{\textquotedblright}
%</all,x2,t2c,t2b,t2a,t1,ot2,ot1,ly1,lcy>
%<all,x2,t2c,t2b,t2a,t1,lcy>\DeclareUnicodeCharacter{201E}{\quotedblbase}
%<all,ts1,oms,ly1>\DeclareUnicodeCharacter{2020}{\textdagger}
%<all,ts1,oms,ly1>\DeclareUnicodeCharacter{2021}{\textdaggerdbl}
%<all,ts1,oms,ly1>\DeclareUnicodeCharacter{2022}{\textbullet}
%<all,ly1,utf8>\DeclareUnicodeCharacter{2026}{\textellipsis}
%<*all,x2,ts1,t2c,t2b,t2a,t1,ly1>
\DeclareUnicodeCharacter{2030}{\textperthousand}
%</all,x2,ts1,t2c,t2b,t2a,t1,ly1>
%<*all,x2,ts1,t2c,t2b,t2a,t1>
\DeclareUnicodeCharacter{2031}{\textpertenthousand}
%</all,x2,ts1,t2c,t2b,t2a,t1>
%<all,t1,ly1>\DeclareUnicodeCharacter{2039}{\guilsinglleft}
%<all,t1,ly1>\DeclareUnicodeCharacter{203A}{\guilsinglright}
%<all,ts1>\DeclareUnicodeCharacter{203B}{\textreferencemark}
%<all,ts1>\DeclareUnicodeCharacter{203D}{\textinterrobang}
%<all,ts1>\DeclareUnicodeCharacter{2044}{\textfractionsolidus}
%<all,ts1>\DeclareUnicodeCharacter{204E}{\textasteriskcentered}
%<all,ts1>\DeclareUnicodeCharacter{2052}{\textdiscount}
%<all,ts1>\DeclareUnicodeCharacter{20A1}{\textcolonmonetary}
%<all,ts1>\DeclareUnicodeCharacter{20A4}{\textlira}
%<all,ts1>\DeclareUnicodeCharacter{20A6}{\textnaira}
%<all,ts1>\DeclareUnicodeCharacter{20A9}{\textwon}
%<all,ts1>\DeclareUnicodeCharacter{20AB}{\textdong}
%<all,ts1>\DeclareUnicodeCharacter{20AC}{\texteuro}
%<all,ts1>\DeclareUnicodeCharacter{20B1}{\textpeso}
%<all,ts1>\DeclareUnicodeCharacter{2103}{\textcelsius}
%<all,x2,ts1,t2c,t2b,t2a,ot2,lcy>\DeclareUnicodeCharacter{2116}{\textnumero}
%<all,ts1>\DeclareUnicodeCharacter{2117}{\textcircledP}
%<all,ts1>\DeclareUnicodeCharacter{211E}{\textrecipe}
%<all,ts1>\DeclareUnicodeCharacter{2120}{\textservicemark}
%<all,ts1,ly1,utf8>\DeclareUnicodeCharacter{2122}{\texttrademark}
%<all,ts1>\DeclareUnicodeCharacter{2126}{\textohm}
%<all,ts1>\DeclareUnicodeCharacter{2127}{\textmho}
%<all,ts1>\DeclareUnicodeCharacter{212E}{\textestimated}
%<all,ts1>\DeclareUnicodeCharacter{2190}{\textleftarrow}
%<all,ts1>\DeclareUnicodeCharacter{2191}{\textuparrow}
%<all,ts1>\DeclareUnicodeCharacter{2192}{\textrightarrow}
%<all,ts1>\DeclareUnicodeCharacter{2193}{\textdownarrow}
%<all,x2,ts1,t2c,t2b,t2a>\DeclareUnicodeCharacter{2329}{\textlangle}
%<all,x2,ts1,t2c,t2b,t2a>\DeclareUnicodeCharacter{232A}{\textrangle}
%<all,ts1>\DeclareUnicodeCharacter{2422}{\textblank}
%<all,x2,t2c,t2b,t2a,t1,utf8>\DeclareUnicodeCharacter{2423}{\textvisiblespace}
%<all,ts1>\DeclareUnicodeCharacter{25E6}{\textopenbullet}
%<all,ts1>\DeclareUnicodeCharacter{25EF}{\textbigcircle}
%<all,ts1>\DeclareUnicodeCharacter{266A}{\textmusicalnote}
%    \end{macrocode}
% \changes{v1.2h}{2019/02/07}{Added 27E8 and 27E9 (github/110)}
%    \begin{macrocode}
%<all,x2,ts1,t2c,t2b,t2a>\DeclareUnicodeCharacter{27E8}{\textlangle}
%<all,x2,ts1,t2c,t2b,t2a>\DeclareUnicodeCharacter{27E9}{\textrangle}
%<all,t1>\DeclareUnicodeCharacter{1E20}{\@tabacckludge=G}
%<all,t1>\DeclareUnicodeCharacter{1E21}{\@tabacckludge=g}
%    \end{macrocode}
%
%    When doing cut-and-paste from other documents f-ligatures might show
%    up as Unicode characters. We translate them back to individual
%    characters so that they get accepted. If supported by the font
%    (which is normally the case) they are then reconstructed as
%    ligatures so they come out as desired. Otherwise they will come
%    out as individual characters which is fine too.
% \changes{v1.2j}{2019/07/11}{Add typical f-ligatures and s-ligatures and st (github/154)}
% \changes{v1.kj}{2019/11/14}{Fixed setting for FB05 = long-s t lig (github/154)}
%    \begin{macrocode}
%<all,t1,ot1,ly1,t2a,t2b,t2c>\DeclareUnicodeCharacter{FB00}{ff}   % ff
%<all,t1,ot1,ly1,t2a,t2b,t2c>\DeclareUnicodeCharacter{FB01}{fi}   % fi
%<all,t1,ot1,ly1,t2a,t2b,t2c>\DeclareUnicodeCharacter{FB02}{fl}   % fl
%<all,t1,ot1,ly1,t2a,t2b,t2c>\DeclareUnicodeCharacter{FB03}{ffi}  % ffi
%<all,t1,ot1,ly1,t2a,t2b,t2c>\DeclareUnicodeCharacter{FB04}{ffl}  % ffl
%<all,t1,ot1,ly1,t2a,t2b,t2c>\DeclareUnicodeCharacter{FB05}{st}   % ſt  -- this is the long s (not f)
%<all,t1,ot1,ly1,t2a,t2b,t2c>\DeclareUnicodeCharacter{FB06}{st}   % st
%    \end{macrocode}
%
%
% \changes{v1.2c}{2018/04/05}{Add BOM}
%    \begin{macrocode}
%<all,ts1,utf8>\DeclareUnicodeCharacter{FEFF}{\ifhmode\nobreak\fi}
%    \end{macrocode}
%
%
% \endgroup
% \subsection{Notes}
%
% \changes{v1.1e}{2004/05/22}{Added notes on inconsistency with `8-bit files'.}
% The following inputs are inconsistent with the 8-bit inputenc files
% since they will always only produce the `text character'.  This is an
% area where inputenc is notoriously confused.
% \begin{verbatim}
% %<all,ts1,t1,ot1,ly1>\DeclareUnicodeCharacter{00A3}{\textsterling}
% %<*all,x2,ts1,t2c,t2b,t2a,oms,ly1>
% \DeclareUnicodeCharacter{00A7}{\textsection}
% %</all,x2,ts1,t2c,t2b,t2a,oms,ly1>
% %<all,ts1,utf8>\DeclareUnicodeCharacter{00A9}{\textcopyright}
% %<all,ts1>\DeclareUnicodeCharacter{00B1}{\textpm}
% %<all,ts1,oms,ly1>\DeclareUnicodeCharacter{00B6}{\textparagraph}
% %<all,ts1,oms,ly1>\DeclareUnicodeCharacter{2020}{\textdagger}
% %<all,ts1,oms,ly1>\DeclareUnicodeCharacter{2021}{\textdaggerdbl}
% %<all,ly1,utf8>\DeclareUnicodeCharacter{2026}{\textellipsis}
% \end{verbatim}
%
%    The following definitions are in an encoding file but have no
%    direct equivalent in Unicode, or they simply do not make sense in that
%    context (or we have not yet found anything or \ldots :-).  For
%    example, the non-combining accent characters are certainly
%    available somewhere but these are not equivalent to a \TeX{}
%    accent command.
%\begin{verbatim}
%\DeclareTextSymbol{\j}{OT1}{17}
%\DeclareTextSymbol{\SS}{T1}{223}
%\DeclareTextSymbol{\textcompwordmark}{T1}{23}
%
%\DeclareTextAccent{\"}{OT1}{127}
%\DeclareTextAccent{\'}{OT1}{19}
%\DeclareTextAccent{\.}{OT1}{95}
%\DeclareTextAccent{\=}{OT1}{22}
%\DeclareTextAccent{\H}{OT1}{125}
%\DeclareTextAccent{\^}{OT1}{94}
%\DeclareTextAccent{\`}{OT1}{18}
%\DeclareTextAccent{\r}{OT1}{23}
%\DeclareTextAccent{\u}{OT1}{21}
%\DeclareTextAccent{\v}{OT1}{20}
%\DeclareTextAccent{\~}{OT1}{126}
%\DeclareTextCommand{\b}{OT1}[1]
%\DeclareTextCommand{\c}{OT1}[1]
%\DeclareTextCommand{\d}{OT1}[1]
%\DeclareTextCommand{\k}{T1}[1]
%\end{verbatim}
%
%
%
% \subsection{Mappings for OT1 glyphs}
%
%    This is even more incomplete as again it covers only the single
%    glyphs from |OT1| plus some that have been explicitly defined for
%    this encoding. Everything that is provided in |T1|, and that
%    could be provided as composite glyphs via |OT1|, could and
%    probably should be set up as well.  Which leaves the many things
%    that are not provided in |T1| but can be provided in |OT1| (and
%    in |T1|) by composite glyphs.
%
% Stuff not mapped (note that |\j| ($\jmath$) is not equivalent to any
% Unicode character):
%\begin{verbatim}
%\DeclareTextSymbol{\j}{OT1}{17}
%\DeclareTextAccent{\"}{OT1}{127}
%\DeclareTextAccent{\'}{OT1}{19}
%\DeclareTextAccent{\.}{OT1}{95}
%\DeclareTextAccent{\=}{OT1}{22}
%\DeclareTextAccent{\^}{OT1}{94}
%\DeclareTextAccent{\`}{OT1}{18}
%\DeclareTextAccent{\~}{OT1}{126}
%\DeclareTextAccent{\H}{OT1}{125}
%\DeclareTextAccent{\u}{OT1}{21}
%\DeclareTextAccent{\v}{OT1}{20}
%\DeclareTextAccent{\r}{OT1}{23}
%\DeclareTextCommand{\b}{OT1}[1]
%\DeclareTextCommand{\c}{OT1}[1]
%\DeclareTextCommand{\d}{OT1}[1]
%\end{verbatim}
%
%
%
% \subsection{Mappings for OMS glyphs}
%
% Characters like |\textbackslash| are not mapped as they are
%    (primarily) only in the lower 127 and the code here only sets up
%    mappings for UTF-8 characters that are at least 2 octets long.
%\begin{verbatim}
%\DeclareTextSymbol{\textbackslash}{OMS}{110}        % "6E
%\DeclareTextSymbol{\textbar}{OMS}{106}              % "6A
%\DeclareTextSymbol{\textbraceleft}{OMS}{102}        % "66
%\DeclareTextSymbol{\textbraceright}{OMS}{103}       % "67
%\end{verbatim}
%
% But the following (and some others) might actually lurk in Unicode
%    somewhere\ldots
%\begin{verbatim}
%\DeclareTextSymbol{\textasteriskcentered}{OMS}{3}   % "03
%\DeclareTextCommand{\textcircled}{OMS}
%\end{verbatim}
%
%
%
%
% \subsection{Mappings for TS1 glyphs}
%
% Exercise for somebody else.
%
%
% \subsection{Mappings for \texttt{latex.ltx} glyphs}
%
% There is also a collection of characters already set up in the kernel,
% one way or the other. Since these do not clearly relate to any
%    particular font encoding they are mapped when the
% \texttt{utf8} support is first set up.
%
% Also there are a number of |\providecommand|s in the various input
% encoding files which may or may not go into this part.
% \changes{v1.1b}{2004/02/09}{Added commands already defined in the kernel}
%    \begin{macrocode}
%<*utf8>
% This space is intentionally empty ...
%</utf8>
%    \end{macrocode}
%
% \subsection{Old \texttt{utf8.def} file as a temp fix for p\TeX{} and friends}
%
%
% \changes{v1.2h}{2019/07/09}{Temp rollback fix for e-pTeX}
%    \begin{macrocode}
%<*utf8-2018>
\ProvidesFile{utf8.def}
   [2018/10/05 v1.2f UTF-8 support for inputenc]
\makeatletter
\catcode`\ \saved@space@catcode
\long\def\UTFviii@two@octets#1#2{\expandafter
    \UTFviii@defined\csname u8:#1\string#2\endcsname}
\long\def\UTFviii@three@octets#1#2#3{\expandafter
    \UTFviii@defined\csname u8:#1\string#2\string#3\endcsname}
\long\def\UTFviii@four@octets#1#2#3#4{\expandafter
    \UTFviii@defined\csname u8:#1\string#2\string#3\string#4\endcsname}
\def\UTFviii@defined#1{%
  \ifx#1\relax
     \if\relax\expandafter\UTFviii@checkseq\string#1\relax\relax
      \UTFviii@undefined@err{#1}%
     \else
      \PackageError{inputenc}{Invalid UTF-8 byte sequence}%
                             \UTFviii@invalid@help
     \fi
  \else\expandafter
    #1%
  \fi
}
\def\UTFviii@invalid@err#1{%
 \PackageError{inputenc}{Invalid UTF-8 byte "\UTFviii@hexnumber{`#1}}%
                        \UTFviii@invalid@help}
\def\UTFviii@invalid@help{%
   The document does not appear to be in UTF-8 encoding.\MessageBreak
   Try adding \noexpand\UseRawInputEncoding as the first line of the file\MessageBreak
   or specify an encoding such as \noexpand\usepackage[latin1]{inputenc}\MessageBreak
   in the document preamble.\MessageBreak
   Alternatively, save the file in UTF-8 using your editor or another tool}
\def\UTFviii@undefined@err#1{%
  \PackageError{inputenc}{Unicode character \expandafter
                          \UTFviii@splitcsname\string#1\relax
                          \MessageBreak
                          not set up for use with LaTeX}%
                         {You may provide a definition with\MessageBreak
                          \noexpand\DeclareUnicodeCharacter}%
   }
\def\UTFviii@checkseq#1:#2#3{%
 \ifnum`#2<"80 %
   \ifx\relax#3\else1\fi
 \else
   \ifnum`#2<"C0 %
     1 %
   \else
     \expandafter\expandafter\expandafter\UTFviii@check@continue
     \expandafter\expandafter\expandafter#3%
   \fi
  \fi}
\def\UTFviii@check@continue#1{%
  \ifx\relax#1%
  \else
  \ifnum`#1<"80 1\else\ifnum`#1>"BF 1\fi\fi
  \expandafter\UTFviii@check@continue
  \fi
}
\begingroup
\catcode`\~13
\catcode`\"12
\def\UTFviii@loop{%
  \uccode`\~\count@
  \uppercase\expandafter{\UTFviii@tmp}%
  \advance\count@\@ne
  \ifnum\count@<\@tempcnta
  \expandafter\UTFviii@loop
  \fi}
    \def\UTFviii@tmp{\xdef~{\noexpand\UTFviii@undefined@err{:\string~}}}
    \count@"1
    \@tempcnta9
\UTFviii@loop
    \count@11
    \@tempcnta12
\UTFviii@loop
    \count@14
    \@tempcnta32
\UTFviii@loop
    \count@"80
    \@tempcnta"C2
    \def\UTFviii@tmp{\xdef~{\noexpand\UTFviii@invalid@err\string~}}
\UTFviii@loop
    \count@"C2
    \@tempcnta"E0
    \def\UTFviii@tmp{\xdef~{\noexpand\UTFviii@two@octets\string~}}
\UTFviii@loop
    \count@"E0
    \@tempcnta"F0
    \def\UTFviii@tmp{\xdef~{\noexpand\UTFviii@three@octets\string~}}
\UTFviii@loop
    \count@"F0
    \@tempcnta"F5
    \def\UTFviii@tmp{\xdef~{\noexpand\UTFviii@four@octets\string~}}
\UTFviii@loop
    \count@"F5
    \@tempcnta"100
    \def\UTFviii@tmp{\xdef~{\noexpand\UTFviii@invalid@err\string~}}
\UTFviii@loop
\endgroup
\@inpenc@test
\ifx\@begindocumenthook\@undefined
  \makeatother
  \endinput \fi
\begingroup
\catcode`\"=12
\catcode`\<=12
\catcode`\.=12
\catcode`\,=12
\catcode`\;=12
\catcode`\!=12
\catcode`\~=13
\gdef\DeclareUnicodeCharacter#1#2{%
  \count@"#1\relax
  \wlog{ \space\space defining Unicode char U+#1 (decimal \the\count@)}%
  \begingroup
    \parse@XML@charref
    \def\UTFviii@two@octets##1##2{\csname u8:##1\string##2\endcsname}%
    \def\UTFviii@three@octets##1##2##3{\csname u8:##1%
                                     \string##2\string##3\endcsname}%
    \def\UTFviii@four@octets##1##2##3##4{\csname u8:##1%
                           \string##2\string##3\string##4\endcsname}%
    \expandafter\expandafter\expandafter
    \expandafter\expandafter\expandafter
    \expandafter
     \gdef\UTFviii@tmp{\IeC{#2}}%
   \endgroup
}
\gdef\parse@XML@charref{%
  \ifnum\count@<"A0\relax
    \ifnum\catcode\count@=13
      \uccode`\~=\count@\uppercase{\def\UTFviii@tmp{\@empty\@empty~}}%
    \else
      \PackageError{inputenc}%
                   {Cannot define non-active Unicode char value < 00A0}%
                   \@eha
      \def\UTFviii@tmp{\UTFviii@tmp}%
    \fi
  \else\ifnum\count@<"800\relax
     \parse@UTFviii@a,%
     \parse@UTFviii@b C\UTFviii@two@octets.,%
  \else\ifnum\count@<"10000\relax
     \parse@UTFviii@a;%
     \parse@UTFviii@a,%
     \parse@UTFviii@b E\UTFviii@three@octets.{,;}%
   \else
     \ifnum\count@>"10FFFF\relax
           \PackageError{inputenc}%
                {\UTFviii@hexnumber\count@\space too large for Unicode}%
                {Values between 0 and 10FFFF are permitted}%
      \fi
     \parse@UTFviii@a;%
     \parse@UTFviii@a,%
     \parse@UTFviii@a!%
     \parse@UTFviii@b F\UTFviii@four@octets.{!,;}%
    \fi
    \fi
  \fi
}
\gdef\parse@UTFviii@a#1{%
     \@tempcnta\count@
     \divide\count@ 64
     \@tempcntb\count@
     \multiply\count@ 64
     \advance\@tempcnta-\count@
     \advance\@tempcnta 128
     \uccode`#1\@tempcnta
     \count@\@tempcntb}
\gdef\parse@UTFviii@b#1#2#3#4{%
     \advance\count@ "#10\relax
     \uccode`#3\count@
     \uppercase{\gdef\UTFviii@tmp{#2#3#4}}}
\ifx\numexpr\@undefined
\gdef\decode@UTFviii#1{0}
\else
\gdef\decode@UTFviii#1\relax{%
  \expandafter\UTFviii@cleanup
    \the\numexpr\dec@de@UTFviii#1\relax)))))\@empty}
\gdef\UTFviii@cleanup#1)#2\@empty{#1}
\gdef\dec@de@UTFviii#1{%
\ifx\relax#1%
\else
  \ifnum`#1>"EF
    ((((`#1-"F0)%
  \else
    \ifnum`#1>"DF
      (((`#1-"E0)%
    \else
      \ifnum`#1>"BF
        ((`#1-"C0)%
      \else
        \ifnum`#1>"7F
        )*64+(`#1-"80)%
        \else
        +`#1 %
        \fi
      \fi
    \fi
  \fi
  \expandafter\dec@de@UTFviii
\fi}
\fi
\ifx\numexpr\@undefined
\global\let\UTFviii@hexnumber\@firstofone
\global\UTFviii@hexdigit\hexnumber@
\else
\gdef\UTFviii@hexnumber#1{%
\ifnum#1>15 %
\expandafter\UTFviii@hexnumber\expandafter{\the\numexpr(#1-8)/16\relax}%
\fi
\UTFviii@hexdigit{\numexpr#1\ifnum#1>0-((#1-8)/16)*16\fi\relax}%
}
\gdef\UTFviii@hexdigit#1{\ifcase\numexpr#1\relax
  0\or1\or2\or3\or4\or5\or6\or7\or8\or9\or
  A\or B\or C\or D\or E\or F\fi}
\fi
\gdef\UTFviii@hexcodepoint#1{U+%
 \ifnum#1<16 0\fi
 \ifnum#1<256 0\fi
 \ifnum#1<4096 0\fi
 \UTFviii@hexnumber{#1}%
}%
\gdef\UTFviii@splitcsname#1:#2\relax{%
#2 (\expandafter\UTFviii@hexcodepoint\expandafter{%
                     \the\numexpr\decode@UTFviii#2\relax})%
}
\endgroup
\@onlypreamble\DeclareUnicodeCharacter
\@onlypreamble\parse@XML@charref
\@onlypreamble\parse@UTFviii@a
\@onlypreamble\parse@UTFviii@b
\begingroup
  \def\cdp@elt#1#2#3#4{%
    \wlog{Now handling font encoding #1 ...}%
    \lowercase{%
        \InputIfFileExists{#1enc.dfu}}%
           {\wlog{... processing UTF-8 mapping file for font %
                     encoding #1}%
            \catcode`\ 9\relax}%
          {\wlog{... no UTF-8 mapping file for font encoding #1}}%
  }
  \cdp@list
\endgroup
\def\DeclareFontEncoding@#1#2#3{%
  \expandafter
  \ifx\csname T@#1\endcsname\relax
    \def\cdp@elt{\noexpand\cdp@elt}%
    \xdef\cdp@list{\cdp@list\cdp@elt{#1}%
                    {\default@family}{\default@series}%
                    {\default@shape}}%
    \expandafter\let\csname#1-cmd\endcsname\@changed@cmd
    \begingroup
      \wlog{Now handling font encoding #1 ...}%
      \lowercase{%
        \InputIfFileExists{#1enc.dfu}}%
           {\wlog{... processing UTF-8 mapping file for font %
                      encoding #1}}%
           {\wlog{... no UTF-8 mapping file for font encoding #1}}%
    \endgroup
  \else
     \@font@info{Redeclaring font encoding #1}%
  \fi
  \global\@namedef{T@#1}{#2}%
  \global\@namedef{M@#1}{\default@M#3}%
  \xdef\LastDeclaredEncoding{#1}%
  }
\DeclareUnicodeCharacter{00A9}{\textcopyright}
\DeclareUnicodeCharacter{00AA}{\textordfeminine}
\DeclareUnicodeCharacter{00AE}{\textregistered}
\DeclareUnicodeCharacter{00BA}{\textordmasculine}
\DeclareUnicodeCharacter{02C6}{\textasciicircum}
\DeclareUnicodeCharacter{02DC}{\textasciitilde}
\DeclareUnicodeCharacter{200C}{\textcompwordmark}
\DeclareUnicodeCharacter{2026}{\textellipsis}
\DeclareUnicodeCharacter{2122}{\texttrademark}
\DeclareUnicodeCharacter{2423}{\textvisiblespace}
\DeclareUnicodeCharacter{FEFF}{\ifhmode\nobreak\fi}
\endinput
</utf8-2018>
%    \end{macrocode}
%
%
% \section{A test document}
%
%    Here is a very small test document which may or may not survive
%    if the current document is transferred from one place to the
%    other.
%    \begin{macrocode}
%<*test>
\documentclass{article}

\usepackage[latin1,utf8]{inputenc}
\usepackage[T1]{fontenc}
\usepackage{trace}

\scrollmode  % to run past the error below

\begin{document}

 German umlauts in UTF-8: ^^c3^^a4^^c3^^b6^^c3^^bc  %%% äöü

\inputencoding{latin1}  % switch to latin1

 German umlauts in UTF-8 but read by latin1 (and will produce one
 error since \verb=\textcurrency= is not provided):
 ^^c3^^a4^^c3^^b6^^c3^^bc

\inputencoding{utf8}    % switch back to utf8

 German umlauts in UTF-8: ^^c3^^a4^^c3^^b6^^c3^^bc


Some codes that should produce errors as nothing is set up
for them: ^^c3F ^^e1^^a4^^b6

And some that are not legal utf8 sequences: ^^c3X ^^e1XY

\showoutput
\tracingstats=2
\stop
%</test>
%    \end{macrocode}
%
% \Finale
%
\endinput
