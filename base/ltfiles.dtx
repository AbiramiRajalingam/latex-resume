% \iffalse meta-comment
%
% Copyright (C) 1993-2020
% The LaTeX3 Project and any individual authors listed elsewhere
% in this file.
%
% This file is part of the LaTeX base system.
% -------------------------------------------
%
% It may be distributed and/or modified under the
% conditions of the LaTeX Project Public License, either version 1.3c
% of this license or (at your option) any later version.
% The latest version of this license is in
%    https://www.latex-project.org/lppl.txt
% and version 1.3c or later is part of all distributions of LaTeX
% version 2008 or later.
%
% This file has the LPPL maintenance status "maintained".
%
% The list of all files belonging to the LaTeX base distribution is
% given in the file `manifest.txt'. See also `legal.txt' for additional
% information.
%
% The list of derived (unpacked) files belonging to the distribution
% and covered by LPPL is defined by the unpacking scripts (with
% extension .ins) which are part of the distribution.
%
% \fi
%
% \iffalse
%%% From File: ltfiles.dtx
%<*driver>
% \fi
\ProvidesFile{ltfiles.dtx}
             [2020-06-05 v1.2h LaTeX Kernel (File Handling)]
% \iffalse
\documentclass{ltxdoc}
\GetFileInfo{ltfiles.dtx}
\title{\filename}
\date{\filedate}
 \author{%
  Johannes Braams\and
  David Carlisle\and
  Alan Jeffrey\and
  Leslie Lamport\and
  Frank Mittelbach\and
  Chris Rowley\and
  Rainer Sch\"opf}

\begin{document}
 \MaintainedByLaTeXTeam{latex}
 \maketitle
 \DocInput{\filename}
\end{document}
%</driver>
% \fi
%
%
% \changes{v1.0h}{1994/05/21}{Use new error commands}
% \changes{v1.0n}{1994/11/17}
%         {\cs{@tempa} to \cs{reserved@a}}
% \changes{v1.0u}{1995/07/13}{Updates to docu}
% \changes{v1.1m}{2015/02/21}
%         {Removed autoload support}
%
% \section{File Handling}
%
% The following user commands are defined in this part:
%
%  \DescribeMacro{\document} (ie |\begin{document}|)\\
%      Reads in the .AUX files and |\catcode|'s |@| to 12.
%
%  \DescribeMacro{\nofiles}\\
%       Suppresses all file output by setting |\@filesw| false.
%
%  \DescribeMacro{\includeonly}\marg{NAME1, ... ,NAMEn}\\
%       Causes only parts NAME1, ... ,NAMEn to be read by
%         their |\include| commands.  Works by setting \@partsw true
%         and setting |\@partlist| to NAME1, ... ,NAMEn.
%
%  \DescribeMacro{\include}\marg{NAME}\\
%        Does an |\input| NAME unless |\@partsw| is true and
%         NAME is not in |\@partlist|.  If |\@filesw| is true, then
%         it directs .AUX output to NAME.AUX, including a
%         checkpoint at the end.
%
% \DescribeMacro{\input}\marg{NAME}\\
%        The same as TeX's |\input|, except it allows optional
%         braces around the file name. In \LaTeXe, it also avoids
%         the primitive `missing file' error, if the file can not be
%         found.
%
% \DescribeMacro{\IfFileExists}\marg{NAME}\marg{then}\marg{else}\\
%   If the file exists on the system, execute \emph{then} otherwise
%   execute \emph{else}.
%
% \DescribeMacro{\InputIfFileExists}\marg{NAME}\marg{then}\marg{else}\\
%   If the file exists on the system, execute \emph{then} and input
%   \emph{NAME}  otherwise execute \emph{else}.
%
% \StopEventually{}
%
% \changes{v1.0a}{1994/03/07}{Initial version, split from latex.dtx}
% \changes{v1.0a}{1994/03/07}{Long lines wrapped to 72 columns}
%
% \begin{oldcomments}
%
%    \begin{macrocode}
%<*2ekernel>
\message{files,}
%    \end{macrocode}
%
%  VARIABLES, SWITCHES AND INTERNAL COMMANDS:
%    \@mainaux    : Output file number for main .AUX file.
%    \@partaux    : Output file number for current part's .AUX file.
%    \@auxout     : Either \@mainout or \@partout, depending on
%                   which .AUX file output goes to.
%    \@input{foo} : If file foo exists, then \input's it,
%                   otherwise types a warning message.
%    @filesw       : Switch -- set false if no .AUX, .TOC, .IDX etc
%                     files are to be written
%    @partsw      : Set true by a \includeonly command.
%    \@partlist   : Set to the argument of the \includeonly command.
%
%    \cp@FOO      : The checkpoint for \include'd file FOO.TEX, written
%                   by \@writeckpt at the end of file FOO.AUX
%
%
% \includeonly{FILELIST} ==
%  BEGIN
%   \@partsw   := T
%   \@partlist := FILELIST
%  END
%
% \include{FILE} ==
%  BEGIN
%   \clearpage
%   if \@filesw = T
%     then  \immediate\write\@mainaux{\string\@input{FILE.AUX}}
%   fi
%   if  \@partsw = T
%     then \@tempswa := F
%          \reserved@b == FILE
%          for \reserved@a := \@partlist
%              do if eval(\reserved@a) = eval(\reserved@b)
%                   then \@tempswa := T          fi
%              od
%   fi
%
%   if \@tempswa = T
%      then \@auxout := \@partaux
%           if \@filesw = T
%             then  \immediate\openout\@partaux{FILE.AUX}
%                   \immediate\write\@partaux{\relax}
%           fi
%           \@input{FILE.TEX}
%           \clearpage
%           \@writeckpt{FILE}
%           if @filesw then \closeout \@partaux fi
%           \@auxout := \@mainaux
%      else \cp@FILE
%   fi
%  END
%
% \@writeckpt{FILE} ==
%  BEGIN
%    if \@filesw = T
%        \immediate\write on file \@partaux:
%                  \@setckpt{FILE}{                  %% }
%        for \reserved@a := \cl@@ckpt
%           do  \immediate\write on file \@partaux:
%                   \global\string\setcounter
%                       {eval(\reserved@a)}{eval(\c@eval(\reserved@a))}
%           od                                     %% {
%        \immediate\write on file \@partaux:  }
%    fi
%  END
%
% \@setckpt{FILE}{LIST} ==
%  BEGIN
%    G \cp@FILE := LIST
%  END
%
%  INITIALIZATION
%    \@tempswa := T
%
% \end{oldcomments}
%
%
% \task{???}{Do we use @unused or mainaux?}
%  \begin{macro}{\@inputcheck}
%  \begin{macro}{\@unused}
%    Allocate read stream for testing and output stream.
% \changes{v1.0l}{1994/11/07}
%      {move here from ltdefns, remove duplicate \cs{@mainaux}}
%    \begin{macrocode}
\newread\@inputcheck
\newwrite\@unused
%    \end{macrocode}
%  \end{macro}
%  \end{macro}
%
% \begin{macro}{\@mainaux}
% \begin{macro}{\@partaux}
%    \begin{macrocode}
\newwrite\@mainaux
\newwrite\@partaux
%    \end{macrocode}
%  \end{macro}
%  \end{macro}
%
% \begin{macro}{\if@filesw}
% \begin{macro}{\if@partsw}
%    \begin{macrocode}
\newif\if@filesw \@fileswtrue
\newif\if@partsw \@partswfalse
%    \end{macrocode}
%  \end{macro}
%  \end{macro}
%
% \begin{macro}{\@clubpenalty}
%    This stores the current normal (non-infinite) value of
%    \cs{clubpenalty}; it should therefore be reset whenever the
%    normal value is changed (as in the bibliography in the standard
%    styles).
% \changes{v1.1h}{1996/10/05}{Added setting its value}
%    \begin{macrocode}
\newcount\@clubpenalty
\@clubpenalty \clubpenalty
%    \end{macrocode}
%  \end{macro}
%
%  \begin{macro}{\document}
% \changes{v0.9e}{1993/12/09}{Hook added}
%    \begin{macrocode}
%</2ekernel>
%<latexrelease>\IncludeInRelease{2020/06/05}%
%<latexrelease>  {\document}{Added hook to load l3backend code}%
%<*2ekernel|latexrelease>
%    \end{macrocode}
%    Cancel the |\begingroup| from |\begin|.
%    \begin{macrocode}
\def\document{\endgroup
%    \end{macrocode}
% \changes{v1.2h}{2020/06/05}{Added hook to load \textsf{l3backend} code}
%    \begin{macrocode}
  \@expl@sys@load@backend@@
%    \end{macrocode}
%    If some options on |\documentclass| haven't been used by any
%    package we will now give a warning since this is most certainly a
%    misspelling.
%    \begin{macrocode}
  \ifx\@unusedoptionlist\@empty\else
    \@latex@warning@no@line{Unused global option(s):^^J%
            \@spaces[\@unusedoptionlist]}%
  \fi
  \@colht\textheight
  \@colroom\textheight \vsize\textheight
  \columnwidth\textwidth
  \@clubpenalty\clubpenalty
  \if@twocolumn
    \advance\columnwidth -\columnsep
    \divide\columnwidth\tw@ \hsize\columnwidth \@firstcolumntrue
  \fi
  \hsize\columnwidth \linewidth\hsize
  \begingroup\@floatplacement\@dblfloatplacement
    \makeatletter\let\@writefile\@gobbletwo
%    \end{macrocode}
% \changes{v1.1a}{1995/10/24}
%            {Removed multiplelabels switch}
%    \begin{macrocode}
    \global \let \@multiplelabels \relax
    \@input{\jobname.aux}%
  \endgroup
  \if@filesw
    \immediate\openout\@mainaux\jobname.aux
    \immediate\write\@mainaux{\relax}%
  \fi
%    \end{macrocode}
%
% Dateline 1991/03/26: FMi added |\process@table| to support NFSS;
% This will also work with old lfonts if no other style defines
% |\process@table|.  The following line forces the initialization of
% the math fonts.
%    \begin{macrocode}
  \process@table
  \let\glb@currsize\@empty  % Force math initialization.
%    \end{macrocode}
% \changes{v0.9t}{1994/01/31}
%     {set \cs{@normalsize} or \cs{normalsize} if necessary}
% \changes{v1.0d}{1994/03/28}
%     {(DPC) remove \cs{@normalsize check}}
% \changes{v1.0d}{1994/03/28}
%     {(DPC) Use \cs{normalsize} not \cs{@normalsize}}
% \changes{v1.0g}{1994/05/13}{Added execution of \cs{every@size}}
% \changes{v1.0m}{1994/11/07}{Renamed \cs{every@size} to
%                \cs{every@math@size}.}
% \changes{v1.0q}{1995/04/25}
%            {Removed execution of \cs{every@size} latex/1407}
%    \begin{macrocode}
  \normalsize
  \everypar{}%
%    \end{macrocode}
%
% So that punctuation in headings is not disturbed by verbatim
% or other local changes to the space factor codes, save the document
% default here. This will be locally reset by the output routine.
% For special cases a class may want to define |\normalsfcodes|
% directly, in case that definition will be used.
% (This is an old bug, problem existed in \LaTeX2.0x and plain \TeX.)
% \changes{v1.1k}{1997/04/14}
%            {Set the document space factor defaults. latex/2404}
%    \begin{macrocode}
  \ifx\normalsfcodes\@empty
    \ifnum\sfcode`\.=\@m
      \let\normalsfcodes\frenchspacing
    \else
      \let\normalsfcodes\nonfrenchspacing
    \fi
  \fi
%    \end{macrocode}
% For similar reasons also save the default language, this will be reset
% locally in the output routine. In particular it allows hyphenation
% in the page head even if the page break happens in verbatim.
% If this has already been set by a package, set to the value of |\language|
% at this spoint.
% \changes{v1.1n}{2017/03/10}{Save language default}
%    \begin{macrocode}
  \ifx\document@default@language\m@ne
    \chardef\document@default@language\language
  \fi
%    \end{macrocode}
%
% Way back in 1991 (08/26) FMi \& RmS set the |\@noskipsec| switch
% to true in the preamble and to false here.
% This was done to trap lists and related text in the preamble but it
% does not catch everything; hence Change 1.1g was introduced.
%    \begin{macrocode}
  \@noskipsecfalse
%    \end{macrocode}
% \changes{v1.1a}{1995/10/24}
%            {Removed refundefined switch}
%    \begin{macrocode}
  \let \@refundefined \relax
%    \end{macrocode}
%    Just before disabling the preamble commands we execute the begin
%    document hook which contains any code contributed by
%    |\AtBeginDocument|. Also disable the gathering of the file list,
%    if no |\listfiles| has been issued. |\AtBeginDocument| is redefined
%    at this point so that and such commands that get into the hook do
%    not chase their tail\ldots
% \changes{v1.1e}{1996/04/24}
%            {(DPC) Reset \cs{AtBeginDocument} eg for latex/1297}
%    \begin{macrocode}
  \let\AtBeginDocument\@firstofone
  \@begindocumenthook
%    \end{macrocode}
%    Most of the following assignments will be done globally in case
%    the user adds something like |\begin{multicols}| to the document
%    hook, i.e. starts are group in |\begin{document}|.
% \changes{v1.0r}{1995/04/27}
%         {Added \cs{global} to support groups in hook}
% \changes{v1.0c}{1994/03/16}
%         {(DPC) directly add file list settings}
% \changes{v1.0v}{1995/08/16}{set \cs{topskip} globally}
% \changes{v1.0v}{1995/08/16}{set \cs{@maxdepth}}
%
% \changes{v1.0s}{1995/05/25}
%         {Added check for \cs{topskip} zero}
% \changes{v1.0t}{1995/05/25}
%         {Corrected typo}
%    Since a value of exactly 0pt for \cs{topskip} causes
%    \cs{twocolumn[]} to misbehave, we add this check, hoping
%    that it will not cause any problems elsewhere.
%    \begin{macrocode}
  \ifdim\topskip<1sp\global\topskip 1sp\relax\fi
  \global\@maxdepth\maxdepth
  \global\let\@begindocumenthook\@undefined
  \ifx\@listfiles\@undefined
    \global\let\@filelist\relax
    \global\let\@addtofilelist\@gobble
  \fi
%    \end{macrocode}
%    At the very end we disable all preamble commands. This has to
%    happen after the begin document hooks was executed so that this
%    hook can still use such commands.
% \changes{v0.9o}{1994/01/15}
%           {move \cs{@preamblecmds} after document hook}
% \changes{v1.0v}{1995/08/16}{set \cs{do} globally}
%    \begin{macrocode}
  \gdef\do##1{\global\let ##1\@notprerr}%
  \@preamblecmds
%    \end{macrocode}
%    The next line saves tokens and also allows |\@nodocument| to be
%    used directly to trap preamble errors.
% \changes{v1.1g}{1996/09/29}{Added disabling of \cs{@nodocument}}
%    \begin{macrocode}
  \global\let \@nodocument \relax
%    \end{macrocode}
%    The next line is a pure safety measure in case a do list is ever
%    expanded at the wrong place. In addition it will save a few
%    tokens to get rid of the above definition.
%    \begin{macrocode}
  \global\let\do\noexpand
%    \end{macrocode}
% \changes{v1.1c}{1995/12/05}{\cs{ignorespaces} added for latex/1933}
%    Use of |\AtBeginDocument| hook might mean that we are already in
%    horizontal mode, so ignore the space after |\begin{document}|.
%    \begin{macrocode}
  \ignorespaces}
%</2ekernel|latexrelease>
%<latexrelease>\EndIncludeInRelease
%<latexrelease>
%<latexrelease>\IncludeInRelease{2017/04/15}%
%<latexrelease>  {\document}{Save language for hyphenation}%
%<latexrelease>\def\document{\endgroup
%<latexrelease>  \ifx\@unusedoptionlist\@empty\else
%<latexrelease>    \@latex@warning@no@line{Unused global option(s):^^J%
%<latexrelease>            \@spaces[\@unusedoptionlist]}%
%<latexrelease>  \fi
%<latexrelease>  \@colht\textheight
%<latexrelease>  \@colroom\textheight \vsize\textheight
%<latexrelease>  \columnwidth\textwidth
%<latexrelease>  \@clubpenalty\clubpenalty
%<latexrelease>  \if@twocolumn
%<latexrelease>    \advance\columnwidth -\columnsep
%<latexrelease>    \divide\columnwidth\tw@ \hsize\columnwidth \@firstcolumntrue
%<latexrelease>  \fi
%<latexrelease>  \hsize\columnwidth \linewidth\hsize
%<latexrelease>  \begingroup\@floatplacement\@dblfloatplacement
%<latexrelease>    \makeatletter\let\@writefile\@gobbletwo
%<latexrelease>    \global \let \@multiplelabels \relax
%<latexrelease>    \@input{\jobname.aux}%
%<latexrelease>  \endgroup
%<latexrelease>  \if@filesw
%<latexrelease>    \immediate\openout\@mainaux\jobname.aux
%<latexrelease>    \immediate\write\@mainaux{\relax}%
%<latexrelease>  \fi
%<latexrelease>  \process@table
%<latexrelease>  \let\glb@currsize\@empty  % Force math initialization.
%<latexrelease>  \normalsize
%<latexrelease>  \everypar{}%
%<latexrelease>  \ifx\normalsfcodes\@empty
%<latexrelease>    \ifnum\sfcode`\.=\@m
%<latexrelease>      \let\normalsfcodes\frenchspacing
%<latexrelease>    \else
%<latexrelease>      \let\normalsfcodes\nonfrenchspacing
%<latexrelease>    \fi
%<latexrelease>  \fi
%<latexrelease>  \ifx\document@default@language\m@ne
%<latexrelease>    \chardef\document@default@language\language
%<latexrelease>  \fi
%<latexrelease>  \@noskipsecfalse
%<latexrelease>  \let \@refundefined \relax
%<latexrelease>  \let\AtBeginDocument\@firstofone
%<latexrelease>  \@begindocumenthook
%<latexrelease>  \ifdim\topskip<1sp\global\topskip 1sp\relax\fi
%<latexrelease>  \global\@maxdepth\maxdepth
%<latexrelease>  \global\let\@begindocumenthook\@undefined
%<latexrelease>  \ifx\@listfiles\@undefined
%<latexrelease>    \global\let\@filelist\relax
%<latexrelease>    \global\let\@addtofilelist\@gobble
%<latexrelease>  \fi
%<latexrelease>  \gdef\do##1{\global\let ##1\@notprerr}%
%<latexrelease>  \@preamblecmds
%<latexrelease>  \global\let \@nodocument \relax
%<latexrelease>  \global\let\do\noexpand
%<latexrelease>  \ignorespaces}
%<latexrelease>\EndIncludeInRelease
%<latexrelease>
%<latexrelease>\IncludeInRelease{0000/00/00}%
%<latexrelease>  {\document}{Save language for hyphenation}
%<latexrelease>\def\document{\endgroup
%<latexrelease>  \ifx\@unusedoptionlist\@empty\else
%<latexrelease>    \@latex@warning@no@line{Unused global option(s):^^J%
%<latexrelease>            \@spaces[\@unusedoptionlist]}%
%<latexrelease>  \fi
%<latexrelease>  \@colht\textheight
%<latexrelease>  \@colroom\textheight \vsize\textheight
%<latexrelease>  \columnwidth\textwidth
%<latexrelease>  \@clubpenalty\clubpenalty
%<latexrelease>  \if@twocolumn
%<latexrelease>    \advance\columnwidth -\columnsep
%<latexrelease>    \divide\columnwidth\tw@ \hsize\columnwidth
%<latexrelease>    \@firstcolumntrue
%<latexrelease>  \fi
%<latexrelease>  \hsize\columnwidth \linewidth\hsize
%<latexrelease>  \begingroup\@floatplacement\@dblfloatplacement
%<latexrelease>    \makeatletter\let\@writefile\@gobbletwo
%<latexrelease>    \global \let \@multiplelabels \relax
%<latexrelease>    \@input{\jobname.aux}%
%<latexrelease>  \endgroup
%<latexrelease>  \if@filesw
%<latexrelease>    \immediate\openout\@mainaux\jobname.aux
%<latexrelease>    \immediate\write\@mainaux{\relax}%
%<latexrelease>  \fi
%<latexrelease>  \process@table
%<latexrelease>  \let\glb@currsize\@empty
%<latexrelease>  \normalsize
%<latexrelease>  \everypar{}%
%<latexrelease>  \ifx\normalsfcodes\@empty
%<latexrelease>    \ifnum\sfcode`\.=\@m
%<latexrelease>      \let\normalsfcodes\frenchspacing
%<latexrelease>    \else
%<latexrelease>      \let\normalsfcodes\nonfrenchspacing
%<latexrelease>    \fi
%<latexrelease>  \fi
%<latexrelease>  \@noskipsecfalse
%<latexrelease>  \let \@refundefined \relax
%<latexrelease>  \let\AtBeginDocument\@firstofone
%<latexrelease>  \@begindocumenthook
%<latexrelease>  \ifdim\topskip<1sp\global\topskip 1sp\relax\fi
%<latexrelease>  \global\@maxdepth\maxdepth
%<latexrelease>  \global\let\@begindocumenthook\@undefined
%<latexrelease>  \ifx\@listfiles\@undefined
%<latexrelease>    \global\let\@filelist\relax
%<latexrelease>    \global\let\@addtofilelist\@gobble
%<latexrelease>  \fi
%<latexrelease>  \gdef\do##1{\global\let ##1\@notprerr}%
%<latexrelease>  \@preamblecmds
%<latexrelease>  \global\let \@nodocument \relax
%<latexrelease>  \global\let\do\noexpand
%<latexrelease>  \ignorespaces}
%<latexrelease>\EndIncludeInRelease
%<*2ekernel>
%    \end{macrocode}
%
%    \begin{macrocode}
\@onlypreamble\document
%    \end{macrocode}
%  \end{macro}
%
% \begin{macro}{\normalsfcodes}
% \changes{v1.1k}{1997/04/14}
%            {Macro added (from patch file) latex/2404}
% The setting of |\@empty| is just a flag. This command may be defined
% in a class or package file. If it is still |\@empty| at
% |\begin{document}| it will be defined to be |\frenchspacing| or
% |\nonfrenchspacing|, depending on which of those appears to be in
% effect at that point.
%    \begin{macrocode}
\let\normalsfcodes\@empty
%    \end{macrocode}
%  \end{macro}
%
% \begin{macro}{\nofiles}
% \changes{v1.0k}{1994/11/04}
%   {Added setting of \cs{protected@write},
%    \cs{makeindex} and \cs{makeglossary} to \cs{nofiles}.  ASAJ.}
% \changes{v1.0o}{1994/11/30}
%         {There is no \cs{@gobblethree}\ldots}
% \changes{v1.0t}{1995/05/25}
%         {(CAR) added \cs{long}}
% \changes{v1.1f}{1996/05/17}
%         {added \cs{write} to \cs{protected@write} for latex/2146}
% Set |\@fileswfalse| which suppresses the places where \LaTeX\ makes
% |\immediate| writes. The |\makeindex| and |\makeglossary| are
% disabled. |\protected@write| is redefined not to write to the file
% specified, but rather to write a blank line to the log file. This
% ensures that a \meta{whatsit} node is still created, and so spacing
% is not affected by the |\nofiles| command; to ensure this more
% generally, the |\if@nobreak| test is needed.
% \changes{v1.1i}{1996/11/05}
%                {Standard \cs{if@nobreak} test added}
%    \begin{macrocode}
\def\nofiles{%
  \@fileswfalse
  \typeout{No auxiliary output files.^^J}%
  \long\def\protected@write##1##2##3%
    {\write\m@ne{}\if@nobreak\ifvmode\nobreak\fi\fi}%
  \let\makeindex\relax
  \let\makeglossary\relax}
\@onlypreamble\nofiles
%    \end{macrocode}
%  \end{macro}
%
%
% \begin{macro}{\protected@write}
%    This takes three arguments: an output stream, some initialization
%    code, and some text to write.  It then writes this, with
%    appropriate handling of |\protect| and |\thepage|.
% \changes{v1.0k}{1994/11/04}{Macro added  ASAJ.}
% \changes{v1.0t}{1995/05/25}
%         {(CAR) added \cs{long}}
%    \begin{macrocode}
\long\def \protected@write#1#2#3{%
      \begingroup
       \let\thepage\relax
       #2%
       \let\protect\@unexpandable@protect
       \edef\reserved@a{\write#1{#3}}%
       \reserved@a
      \endgroup
      \if@nobreak\ifvmode\nobreak\fi\fi
}
%    \end{macrocode}
% \end{macro}
%
%    \begin{macrocode}
\let\@auxout=\@mainaux
%    \end{macrocode}
%
%
% \begin{macro}{\includeonly}
% \changes{v1.0p}{1995/04/22}{Allow blanks in argument}
% \changes{v1.2a}{2019/07/01}{Support UTF-8}
% \changes{v1.2g}{2020/05/02}{Improved support for spaces in filenames
%    (gh/217)}
%    \begin{macrocode}
%</2ekernel>
%<*2ekernel|latexrelease>
%<latexrelease>\IncludeInRelease{2020/10/01}%
%<latexrelease>                 {\includeonly}{Spaces in file names}%
\def\includeonly#1{%
  \@partswtrue
%    \end{macrocode}
%    Because the argument to |\includeonly| is a comma-separated list
%    of filenames where there may be comma's precedeing some of the
%    filenames or trailing them. Therefore we need to take the list
%    apart, remove the unwanted spaces while leaving the spaces
%    \emph{in} the filenames intact.
%    \begin{macrocode}
  \let\@partlist\@empty
  \@for\reserved@a:=#1 \do
    {
      \expandafter\set@curr@file@trim@spaces\expandafter{\reserved@a}%
      \ifx\@partlist\@empty
        \edef\@partlist{\@curr@file}%
      \else
        \edef\@partlist{\@partlist,\@curr@file}%
      \fi
    }%
  }
\@onlypreamble\includeonly
%    \end{macrocode}
% \end{macro}
%
% \begin{macro}{\include}
% \changes{v0.9p}{1994/01/18}
%         {Use \cs{@input@} so include files are listed.}
% In the definition of |\include|, |\def\reserved@b| changed to
% |\edef\reserved@b| to be consistent with the |\edef| in
% |\includeonly|.
% (Suggested by Rainer Sch\"opf \& Frank Mittelbach.
% Change made 20 Jul 88.)
%
% Changed definition of |\include| to allow space at end of file name
% --- otherwise, typing |\include{foo }| would cause \LaTeX\ to
% overwrite |foo.tex|.  Change made 24 May 89, suggested by Rainer
% Sch\"opf  and Frank Mittelbach
%
% Made |\include| check for being used inside an |\include|'d file, as
% this will not work and cause surprising results.
% \changes{v1.2g}{2020/05/02}{Get rid of leading and trailing spaces
%    from the filename (gh/217)}
% \changes{v1.2g}{2020-05-02}{Pass the filename to \cs{@include} by
%    value instead of by reference (gh/217)} 
%    \begin{macrocode}
\def\include#1{\relax
  \ifnum\@auxout=\@partaux
    \@latex@error{\string\include\space cannot be nested}\@eha
  \else
    \set@curr@file@trim@spaces{#1}%
%    \end{macrocode}
%    For historical reasons \cs{@include} expects an argument
%    delimited by a space. This is kept (though uncessary now) to avoid
%    errors in other packages that use \cs{@include} directly.
%    \begin{macrocode}
    \expandafter\@include\expandafter{\@curr@file} % deliberate space
  \fi}
%    \end{macrocode}
% \end{macro}
%
%
%
%  \begin{macro}{\set@curr@file@trim@spaces}
%
%    For |\include| and |\includeonly| we need a variant that removes
%    any leading and trailing spaces from the filename, while leaving
%    any spaces \emph{inside} the filename intact. In order simplify
%    the implementaion we borrow some code from the \texttt{expl3}
%    language, \textbf{\emph{beware}: this may change unannounced}.
%    \begin{macrocode}
\def\set@curr@file@trim@spaces#1{%
  \@expl@tl@trim@spaces@apply@@nN {#1} \set@curr@file }
%    \end{macrocode}
%  \end{macro}
%
%
%
%    \begin{macrocode}
%</2ekernel|latexrelease>
%<latexrelease>\EndIncludeInRelease
%<latexrelease>\IncludeInRelease{2019/10/01}%
%<latexrelease>                 {\includeonly}{Spaces in file names}%
%<latexrelease>
%<latexrelease>\def\includeonly#1{%
%<latexrelease>  \@partswtrue
%<latexrelease>  \set@curr@file{\zap@space#1 \@empty}%
%<latexrelease>  \let\@partlist\@curr@file
%<latexrelease>  }
%<latexrelease>
%<latexrelease>\def\include#1{\relax
%<latexrelease>  \ifnum\@auxout=\@partaux
%<latexrelease>    \@latex@error{\string\include\space cannot be nested}\@eha
%<latexrelease>  \else
%<latexrelease>    \set@curr@file{#1 }%
%<latexrelease>    \expandafter\@include\@curr@file
%<latexrelease>  \fi}
%<latexrelease>
%<latexrelease>\let\set@curr@file@trim@spaces\@undefined
%<latexrelease>\EndIncludeInRelease
%    \end{macrocode}
%    
%    \begin{macrocode}
%<latexrelease>\IncludeInRelease{0000/00/00}%
%<latexrelease>                 {\includeonly}{Spaces in file names}%
%<latexrelease>\def\includeonly#1{%
%<latexrelease>  \@partswtrue
%<latexrelease>  \edef\@partlist{\zap@space#1 \@empty}}
%<latexrelease>
%<latexrelease>\def\include#1{\relax
%<latexrelease>  \ifnum\@auxout=\@partaux
%<latexrelease>    \@latex@error{\string\include\space cannot be nested}\@eha
%<latexrelease>  \else \@include#1 \fi}
%<latexrelease>
%<latexrelease>\EndIncludeInRelease
%<*2ekernel>
%    \end{macrocode}

%
% \begin{macro}{\@include}
% \changes{v1.2g}{2020/05/02}{Support spaces in filenames by enclosing
%    the names of \texttt{.aux}-files in quotes (gh/217)}
%    \begin{macrocode}
%</2ekernel>
%<*2ekernel|latexrelease>
%<latexrelease>\IncludeInRelease{2020/10/01}%
%<latexrelease>                 {\@include}{Spaces in file names}%
%    \end{macrocode}
%    
%    \begin{macrocode}
\def\@include#1 {%
  \clearpage
  \if@filesw
    \immediate\write\@mainaux{\string\@input{"#1.aux"}}%
  \fi
  \@tempswatrue
  \if@partsw
    \@tempswafalse
    \edef\reserved@b{#1}%
    \@for\reserved@a:=\@partlist\do
      {\ifx\reserved@a\reserved@b\@tempswatrue\fi}%
  \fi
  \if@tempswa
    \let\@auxout\@partaux
    \if@filesw
      \immediate\openout\@partaux "#1.aux"
      \immediate\write\@partaux{\relax}%
    \fi
    \@input@{#1.tex}%
    \clearpage
    \@writeckpt{#1}%
    \if@filesw
      \immediate\closeout\@partaux
    \fi
  \else
%    \end{macrocode}
% If the file is not included, reset |\deadcycles|, so that a long
% list of non-included files does not generate an `Output loop'
% error.
% \changes{v1.1j}{1997/01/08}{reset \cs{deadcycles} latex/2365}
%    \begin{macrocode}
    \deadcycles\z@
    \@nameuse{cp@#1}%
  \fi
  \let\@auxout\@mainaux}
%<latexrelease>\EndIncludeInRelease
%<latexrelease>
%<latexrelease>\IncludeInRelease{0000/00/00}%
%<latexrelease>                 {\@include}{Spaces in file names}%
%<latexrelease>\def\@include#1 {%
%<latexrelease>  \clearpage
%<latexrelease>  \if@filesw
%<latexrelease>    \immediate\write\@mainaux{\string\@input{#1.aux}}%
%<latexrelease>  \fi
%<latexrelease>  \@tempswatrue
%<latexrelease>  \if@partsw
%<latexrelease>    \@tempswafalse
%<latexrelease>    \edef\reserved@b{#1}%
%<latexrelease>    \@for\reserved@a:=\@partlist\do
%<latexrelease>      {\ifx\reserved@a\reserved@b\@tempswatrue\fi}%
%<latexrelease>  \fi
%<latexrelease>  \if@tempswa
%<latexrelease>    \let\@auxout\@partaux
%<latexrelease>    \if@filesw
%<latexrelease>      \immediate\openout\@partaux #1.aux
%<latexrelease>      \immediate\write\@partaux{\relax}%
%<latexrelease>    \fi
%<latexrelease>    \@input@{#1.tex}%
%<latexrelease>    \clearpage
%<latexrelease>    \@writeckpt{#1}%
%<latexrelease>    \if@filesw
%<latexrelease>      \immediate\closeout\@partaux
%<latexrelease>    \fi
%<latexrelease>  \else
%<latexrelease>    \deadcycles\z@
%<latexrelease>    \@nameuse{cp@#1}%
%<latexrelease>  \fi
%<latexrelease>  \let\@auxout\@mainaux}
%<latexrelease>
%<latexrelease>\EndIncludeInRelease
%<*2ekernel>
%    \end{macrocode}
% \end{macro}
%
% \begin{macro}{\@writeckpt}
%    \begin{macrocode}
\def\@writeckpt#1{%
  \if@filesw
    \immediate\write\@partaux{\string\@setckpt{#1}\@charlb}%
    {\let\@elt\@wckptelt \cl@@ckpt}%
    \immediate\write\@partaux{\@charrb}%
  \fi}
%    \end{macrocode}
% \end{macro}
%
% \begin{macro}{\@wckptelt}
%    \begin{macrocode}
\def\@wckptelt#1{%
  \immediate\write\@partaux{%
    \string\setcounter{#1}{\the\@nameuse{c@#1}}}}
%    \end{macrocode}
% \end{macro}
%
% \begin{macro}{\@setckpt}
% RmS 93/08/31: introduced |\@setckpt|
%    \begin{macrocode}
\def\@setckpt#1{\global\@namedef{cp@#1}}
%    \end{macrocode}
% \end{macro}
%
% \begin{macro}{\@charlb}
% \begin{macro}{\@charrb}
% The following defines |\@charlb| and |\@charrb| to be |{| and |}|,
% respectively with |\catcode| 11.
%    \begin{macrocode}
{\catcode`[=1 \catcode`]=2
\catcode`{=11 \catcode`}=11
\gdef\@charlb[{]
\gdef\@charrb[}]
]% }brace matching
%    \end{macrocode}
% \end{macro}
% \end{macro}
%
%
% \subsection{Safe Input Macros}
%
%
%

%  \begin{macro}{\@curr@file}
%  \begin{macro}{\set@curr@file}
%
%   File name handling is done by generating a csname from the
%   provided file name (which means that UTF-8 octets gets turned into
%   strings as this is what happens if they appear in a csname due to
%   the code in \texttt{utf8.def}). By setting \cs{escapchar} to
%   \texttt{-1} we ensure that we don't get a backslash in front. As a
%   result we end up with all characters as catcode 12 (plus
%   spaces). We then sometimes add quotes around the construct
%   (removing any existing inner quotes. Sometimes we only remove the
%   quotes if they have been supplied by the user. There is clearly
%   some room for improvement.
%
%   A side effect of the new code is that we will see quotes around
%   file name displays where there haven't been any before.
%
%   For compatibilty with existing code using |{abc}.tex| or |{one.two}.png|
%   an initial brace group is discarded before expansion and |\string| is applied
%   The content of the brace group is discarded. This means that a leading space
%   will be lost unless protected (by |{ }| or |" "| or |\space|) but filenames
%   with a space are hopefully rare.
%
%  \changes{v1.2a}{2019/07/01}{Support UTF-8}
%  \changes{v1.2c}{2019/10/11}{Remove one brace group}
%  \changes{v1.2d}{2019/10/26}{remove quotes}
%  \changes{v1.2e}{2019/11/09}{expand and \cs{string} before removing quotes}
%    \begin{macrocode}
%</2ekernel>
%<*2ekernel|latexrelease>
%<latexrelease>\IncludeInRelease{2019/10/01}%
%<latexrelease>                 {\set@curr@file}{Quote file names}%
\def\set@curr@file#1{%
  \begingroup
    \escapechar\m@ne
    \xdef\@curr@file{%
      \expandafter\expandafter\expandafter\unquote@name
      \expandafter\expandafter\expandafter{%
      \expandafter\string
        \csname\@firstofone#1\@empty\endcsname}}%
  \endgroup
}
%    \end{macrocode}
%  \end{macro}
%  \end{macro}
%
%
%  \begin{macro}{\quote@name}
%  \begin{macro}{\quote@@name}
%  \begin{macro}{\unquote@name}
%     Quoting spaces
%\begin{verbatim}
% a b c     -> "a b c"
% "a b c"   -> "a b c"
% a" "b" "c -> "a b c"
%           -> ""
%\end{verbatim}
%    \begin{macrocode}
\def\quote@name#1{"\quote@@name#1\@gobble""}
\def\quote@@name#1"{#1\quote@@name}
%    \end{macrocode}
%    and removing quotes \ldots
%    \begin{macrocode}
\def\unquote@name#1{\quote@@name#1\@gobble"}
%    \end{macrocode}
%  \end{macro}
%  \end{macro}
%  \end{macro}
%
%
%
%  \begin{macro}{\IfFileExists}
%
% \changes{v1.2a}{2019/07/01}{Support UTF-8}
% \changes{v1.2b}{2019/08/27}{Make command robust}
% \changes{v1.2d}{2019/10/26}{dont quote name}
%    \begin{macrocode}
\DeclareRobustCommand\IfFileExists[1]{%
  \set@curr@file{#1}%
  \expandafter\IfFileExists@\expandafter{\@curr@file}}
%    \end{macrocode}
%  \end{macro}
%
% \begin{macro}{\IfFileExists@}
% \changes{v0.9b}{1993/12/04}{Macro added}
% \changes{v0.9p}{1994/01/18}{New Definition}
% \changes{v1.0t}{1995/05/25}{(CAR) added \cs{long}}
% \changes{v1.2d}{2019/10/26}{quote on openin}%
% Argument |#1| is |\@curr@file| so catcode 12 string with no quotes.
%    \begin{macrocode}
\long\def \IfFileExists@#1#2#3{%
  \openin\@inputcheck"#1" %
  \ifeof\@inputcheck
    \ifx\input@path\@undefined
      \def\reserved@a{#3}%
    \else
      \def\reserved@a{\@iffileonpath{#1}{#2}{#3}}%
    \fi
  \else
    \closein\@inputcheck
    \edef\@filef@und{"#1" }%
    \def\reserved@a{#2}%
  \fi
  \reserved@a}
%    \end{macrocode}
%  \end{macro}
%
% \begin{macro}{\@iffileonpath}
% If the file is not found by |\openin|, and |\input@path| is defined,
% look in all the directories specified in |\input@path|.
% \changes{v0.9p}{1994/01/18}{Macro added}
% \changes{v1.0f}{1994/05/02}
%     {\cs{@break@loop} renamed to \cs{@break@tfor}}
% \changes{v1.0t}{1995/05/25}
%         {(CAR) added \cs{long}}
% \changes{v1.1d}{1996/01/10}
%     {Change argument handling to not require doubled hash. latex/2024}
% \changes{v1.2d}{2019/10/26}{quote on openin}%
% \changes{v1.2f}{2019/11/11}{make \cs{@filef@und} match quoting used on \cs{openin}}%
%    \begin{macrocode}
\long\def\@iffileonpath#1{%
  \let\reserved@a\@secondoftwo
  \expandafter\@tfor\expandafter\reserved@b\expandafter
             :\expandafter=\input@path\do{%
    \openin\@inputcheck\expandafter\quote@name\expandafter{\reserved@b#1} %
    \ifeof\@inputcheck\else
      \edef\@filef@und{\expandafter\quote@name\expandafter{\reserved@b#1} }%
      \let\reserved@a\@firstoftwo%
      \closein\@inputcheck
      \@break@tfor
    \fi}%
  \reserved@a}
%    \end{macrocode}
%  \end{macro}
%
%    \begin{macrocode}
%</2ekernel|latexrelease>
%<latexrelease>\EndIncludeInRelease
%<latexrelease>\IncludeInRelease{0000/00/00}%
%<latexrelease>                 {\set@curr@file}{Quote file names}%
%<latexrelease>
%<latexrelease>\let\quote@name\@undefined
%<latexrelease>\let\quote@@name\@undefined
%<latexrelease>\let\unquote@name\@undefined
%<latexrelease>\let\set@curr@file\@undefined
%<latexrelease>
%<latexrelease>\let\IfFileExists@\@undefined
%<latexrelease>
%<latexrelease>\long\def \IfFileExists#1#2#3{%
%<latexrelease>  \openin\@inputcheck#1 %
%<latexrelease>  \ifeof\@inputcheck
%<latexrelease>    \ifx\input@path\@undefined
%<latexrelease>      \def\reserved@a{#3}%
%<latexrelease>    \else
%<latexrelease>      \def\reserved@a{\@iffileonpath{#1}{#2}{#3}}%
%<latexrelease>    \fi
%<latexrelease>  \else
%<latexrelease>    \closein\@inputcheck
%<latexrelease>    \edef\@filef@und{#1 }%
%<latexrelease>    \def\reserved@a{#2}%
%<latexrelease>  \fi
%<latexrelease>  \reserved@a}
%<latexrelease>
%<latexrelease>\long\def\@iffileonpath#1{%
%<latexrelease>  \let\reserved@a\@secondoftwo
%<latexrelease>  \expandafter\@tfor\expandafter\reserved@b\expandafter
%<latexrelease>             :\expandafter=\input@path\do{%
%<latexrelease>    \openin\@inputcheck\reserved@b#1 %
%<latexrelease>    \ifeof\@inputcheck\else
%<latexrelease>      \edef\@filef@und{\reserved@b#1 }%
%<latexrelease>      \let\reserved@a\@firstoftwo%
%<latexrelease>      \closein\@inputcheck
%<latexrelease>      \@break@tfor
%<latexrelease>    \fi}%
%<latexrelease>  \reserved@a}
%<latexrelease>
%<latexrelease>\EndIncludeInRelease
%<*2ekernel>
%    \end{macrocode}
%
%
%
%
% \begin{macro}{\InputIfFileExists}
% \changes{v0.9b}
%         {1993/12/04}{Macro added}
% \changes{v0.9p}
%         {1994/01/18}{New Definition}
% \changes{v0.3b}{1994/03/13}
%         {Use new cmd \cs{@addtofilelist}}
% Now define |\InputIfFileExists| to input |#1| if it seems to exist.
% Immediately prior to the input, |#2| is executed.
% If the file |#1| does not exist, execute `|#3|'.
% \changes{v1.0t}{1995/05/25}
%         {(CAR) added \cs{long}}
% \changes{v1.1o}{2019/02/07}{Expand \cs{@filef@und} before executing
%   second argument (github/109)}
% \changes{v1.2b}{2019/08/27}{Make command robust}
%    \begin{macrocode}
%</2ekernel>
%<latexrelease>\IncludeInRelease{2019/10/01}%
%<latexrelease>  {\InputIfFileExists}{Don't lose the file name}%
%<*2ekernel|latexrelease>
%    \begin{macrocode}
\DeclareRobustCommand \InputIfFileExists[2]{%
  \IfFileExists{#1}%
    {%
  \expandafter\@swaptwoargs\expandafter
      {\@filef@und}{#2\@addtofilelist{#1}\@@input}}}
%    \end{macrocode}
%  \end{macro}
%
%
%  \begin{macro}{\@swaptwoargs}
%
% \changes{v1.1o}{2019/02/07}{Helper macro added}
%    Swap two arguments and return them unbraced (like
%   \cs{@firstoftwo} etc).
%    \begin{macrocode}
\long\def\@swaptwoargs#1#2{#2#1}
%    \end{macrocode}
%  \end{macro}
%
%
%    \begin{macrocode}
%</2ekernel|latexrelease>
%<latexrelease>\EndIncludeInRelease
%<latexrelease>\IncludeInRelease{0000/00/00}%
%<latexrelease>  {\InputIfFileExists}{Don't lose the file name}%
%<latexrelease>\long\def \InputIfFileExists#1#2{%
%<latexrelease>  \IfFileExists{#1}%
%<latexrelease>    {#2\@addtofilelist{#1}\@@input \@filef@und}}
%<latexrelease>
%<latexrelease>\let\@swaptwoargs\@undefined
%<latexrelease>\EndIncludeInRelease
%<*2ekernel>
%    \end{macrocode}
%
%
%
%  \begin{macro}{\input}
%    Input a file: if the argument is given in braces use safe input
%    macros, otherwise use \TeX's primitive |\input| command (which is
%    called |\@@input| in \LaTeX).
% \changes{v0.9b}{1993/12/04}{Macro reimplemented}
%    \begin{macrocode}
\def\input{\@ifnextchar\bgroup\@iinput\@@input}
%    \end{macrocode}
%  \end{macro}
%
% \begin{macro}{\@iinput}
%    Define |\@iinput| (i.e., |\input|) in terms of
%    |\InputIfIfileExists|.
% \changes{v0.9b}{1993/12/04}{Macro reimplemented}
%    \begin{macrocode}
%</2ekernel>
%<*2ekernel|latexrelease>
%<latexrelease>\IncludeInRelease{2019/10/01}%
%<latexrelease>                 {\@iinput}{Quote file names}%
\def\@iinput#1{%
  \InputIfFileExists{#1}{}%
  {\filename@parse\@curr@file
   \edef\reserved@a{\noexpand\@missingfileerror
     {\filename@area\filename@base}%
     {\ifx\filename@ext\relax tex\else\filename@ext\fi}}%
   \reserved@a}}
%</2ekernel|latexrelease>
%<latexrelease>\EndIncludeInRelease
%<latexrelease>\IncludeInRelease{0000/00/00}%
%<latexrelease>                 {\@iinput}{Quote file names}%
%<latexrelease>\def\@iinput#1{%
%<latexrelease>  \InputIfFileExists{#1}{}%
%<latexrelease>  {\filename@parse{#1}%
%<latexrelease>   \edef\reserved@a{\noexpand\@missingfileerror
%<latexrelease>     {\filename@area\filename@base}%
%<latexrelease>     {\ifx\filename@ext\relax tex\else\filename@ext\fi}}%
%<latexrelease>   \reserved@a}}
%<latexrelease>\EndIncludeInRelease
%<*2ekernel>
%    \end{macrocode}
%  \end{macro}




%
% \begin{macro}{\@input}
% \changes{v0.9b}{1993/12/04}{Macro reimplemented}
% \changes{v0.9p}{1994/01/18}%
%         {do not use a different definition for \cs{input@path}}
%    Define |\@input| in terms of |\IfIfileExists|.
%    So this is a `safe input' command, but the files input are not
%    listed by |\listfiles|.
%
%    We don't want |.aux|, |.toc| files etc be listed by |\listfiles|.
%    However, something like |.bbl| probably should be listed and thus
%    should be implemented not by |\@input|.
%    \begin{macrocode}
\def\@input#1{%
  \IfFileExists{#1}{\@@input\@filef@und}{\typeout{No file #1.}}}
%    \end{macrocode}
%  \end{macro}
%
% \begin{macro}{\@input@}
% \changes{v0.9p}{1994/01/18}{Macro added}
% \task{???}{share code with `@input ?}
% Version of |\@input| that does add the file to |\@filelist|.
%    \begin{macrocode}
\def\@input@#1{\InputIfFileExists{#1}{}{\typeout{No file #1.}}}
%    \end{macrocode}
%  \end{macro}
%
% \begin{macro}{\@missingfileerror}
% This `error' command avoids \TeX's primitive missing file loop.
%
% Missing file error. Prompt for a new filename, offering a default
% extension.
% \changes{LaTeX2e}{1993/11/21}
%         {Stop infinite looping on \cs{@er@ext}}
% \changes{LaTeX2e}{1993/11/28}
%         {Use filename parser from dircheck}
% \changes{LaTeX2e}{1994/03/15}
%         {Quit on x or X just like a real error}
% \changes{LaTeX2e}{1994/05/26}
%         {Modify message format}
% \changes{v1.0j}{1994/11/03}
%         {Move here from ltclass}
% \changes{v1.0w}{1995/10/06}
%         {Autoload error}
%    \begin{macrocode}
\gdef\@missingfileerror#1#2{%
     \typeout{^^J! LaTeX Error: File `#1.#2' not found.^^J^^J%
      Type X to quit or <RETURN> to proceed,^^J%
      or enter new name. (Default extension: #2)^^J}%
     \message{Enter file name: }%
      {\endlinechar\m@ne
       \global\read\m@ne to\@gtempa}%
    \ifx\@gtempa\@empty
    \else
      \def\reserved@a{x}\ifx\reserved@a\@gtempa\batchmode\@@end\fi
      \def\reserved@a{X}\ifx\reserved@a\@gtempa\batchmode\@@end\fi
      \filename@parse\@gtempa
      \edef\filename@ext{%
        \ifx\filename@ext\relax#2\else\filename@ext\fi}%
     \edef\reserved@a{%
       \noexpand\InputIfFileExists
         {\filename@area\filename@base.\filename@ext}%
         {}%
         {\noexpand\@missingfileerror
            {\filename@area\filename@base}{\filename@ext}}}%
      \reserved@a
    \fi}
%    \end{macrocode}
% \end{macro}
%
% \begin{macro}{\@obsoletefile}
%    For compatibility with \LaTeX~2.09 document styles, we distribute
%    files called |article.sty|, |book.sty|, |report.sty|,
%    |slides.sty| and |letter.sty|.  These use the command
%    |\@obsoletefile|, which produces a warning message.
%    \changes{v0.9m}{1993/12/20}{Added this command, removed
%                               @oldfilewarning}
% \changes{v1.0f}{1994/05/02}{Make \cs{@onlypreamble}}
%    \begin{macrocode}
\def\@obsoletefile#1#2{%
   \@latex@warning@no@line{inputting `#1' instead of obsolete `#2'}}
\@onlypreamble\@obsoletefile
%    \end{macrocode}
%
% \subsection{Listing files}
%
% \begin{macro}{\@filelist}
% A list of files input so far. The initial value of |\@gobble| eats
% the comma before the first file name.
%    \begin{macrocode}
\let\@filelist\@gobble
%    \end{macrocode}
% \end{macro}
%
% \begin{macro}{\@addtofilelist}
% Add to the  list of files input so far.
% \changes{LaTeX2e}{1994/03/13}
%         {Macro added}
% \changes{LaTeX2e}{1995/10/01}
%         {Macro added}
% This `real' definition is only used for `cfg' files during initex.
% An initial definition of |\@gobble| has already been set.
%    \begin{macrocode}
%\def\@addtofilelist#1{\xdef\@filelist{\@filelist,#1}}
%    \end{macrocode}
% \end{macro}
%
% \begin{macro}{\listfiles}
% A preamble command to cause |\end{document}| to list files input
% from the main file.
% \changes{LaTeX2e}{1993/11/22}
%         {Removed checking for \cs{@unknownversion}}
% \changes{LaTeX2e}{1994/01/17}
%         {New Version, adds `.tex' if needed, and lines up columns}
% \changes{LaTeX2e}{1994/05/13}
%         {Stop \cs{listfiles} being run twice}
% \changes{v1.0i}{1994/10/18}
%         {code moved here from ltclass}
%    \begin{macrocode}
\def\listfiles{%
  \let\listfiles\relax
  \def\@listfiles##1##2##3##4##5##6##7##8##9\@@{%
     \def\reserved@d{\\}%
     \@tfor\reserved@c:=##1##2##3##4##5##6##7##8\do{%
       \ifx\reserved@c\reserved@d
         \edef\filename@area{ \filename@area}%
       \fi}}%
%    \end{macrocode}
%
% \changes{v1.0o}{1994/11/30}
%         {Use \cs{@dofilelist}}
%    \begin{macrocode}
  \def\@dofilelist{%
     \typeout{^^J *File List*}%
     \@for\@currname:=\@filelist\do{%
       \filename@parse\@currname
       \edef\reserved@a{%
          \filename@base.%
          \ifx\filename@ext\relax tex\else\filename@ext\fi}%
       \expandafter\let\expandafter\reserved@b
                              \csname ver@\reserved@a\endcsname
       \expandafter\expandafter\expandafter\@listfiles\expandafter
             \filename@area\filename@base\\\\\\\\\\\\\\\\\\\@@
       \typeout{%
         \filename@area\reserved@a
         \ifx\reserved@b\relax\else\@spaces\reserved@b\fi}}%
     \typeout{ ***********^^J}}}
%    \end{macrocode}
%
% \changes{LaTeX2e}{1994/03/13}
%         {Reset \cs{@addtofilelist} at begin document}
% \changes{LaTeX2e}{1994/03/16}
%         {Move this code directly into \cs{document}}
% The |\@filelist| will be de-activated if |\listfiles| does not
% appear in the preamble. |\begin{document}| contains code equivalent
% to the following:
%\begin{verbatim}
% \AtBeginDocument{%
%   \ifx\@listfiles\@undefined
%     \let\@filelist\relax
%     \let\@addtofilelist\@gobble
%   \fi}
%\end{verbatim}
%    \begin{macrocode}
\@onlypreamble\listfiles
%    \end{macrocode}
% \end{macro}
%
%  \begin{macro}{\@dofilelist}
% \changes{v1.0o}{1994/11/30}
%         {Macro added}
%    \begin{macrocode}
\let\@dofilelist\relax
%    \end{macrocode}
%  \end{macro}
%
%    \begin{macrocode}
%</2ekernel>
%    \end{macrocode}
% \end{macro}
%
% \Finale
%
