% \iffalse meta-comment
%
% Copyright (C) 1993-2020
% The LaTeX3 Project and any individual authors listed elsewhere
% in this file.
%
% This file is part of the LaTeX base system.
% -------------------------------------------
%
% It may be distributed and/or modified under the
% conditions of the LaTeX Project Public License, either version 1.3c
% of this license or (at your option) any later version.
% The latest version of this license is in
%    https://www.latex-project.org/lppl.txt
% and version 1.3c or later is part of all distributions of LaTeX
% version 2008 or later.
%
% This file has the LPPL maintenance status "maintained".
%
% The list of all files belonging to the LaTeX base distribution is
% given in the file `manifest.txt'. See also `legal.txt' for additional
% information.
%
% The list of derived (unpacked) files belonging to the distribution
% and covered by LPPL is defined by the unpacking scripts (with
% extension .ins) which are part of the distribution.
%
% \fi
%
% \iffalse
%%% From File: ltlength.dtx
%
%<*driver>
% \fi
\ProvidesFile{ltlength.dtx}
             [2019/08/27 v1.1d LaTeX Kernel (Lengths)]
% \iffalse
\documentclass{ltxdoc}
\GetFileInfo{ltlength.dtx}
\title{\filename}
\date{\filedate}
 \author{%
  Johannes Braams\and
  David Carlisle\and
  Alan Jeffrey\and
  Leslie Lamport\and
  Frank Mittelbach\and
  Chris Rowley\and
  Rainer Sch\"opf}

\begin{document}
 \MaintainedByLaTeXTeam{latex}
 \maketitle
 \DocInput{\filename}
\end{document}
%</driver>
% \fi
%
%
% \changes{v1.0c}{1994/03/29}
%     {Create file ltcntlen from parts of ltmiscen and ltherest.}
% \changes{v1.1a}{1994/05/19}{Extract file ltlength from ltcntlen.}
% \changes{v1.1b}{1995/08/11}{Doc typos fixed for latex/753}
% \changes{v1.1d}{2019/08/27}{Make various command robust}
%
% \section{Lengths}
%
% \DescribeMacro{\newlength}
% Declare |#1| to be a new length command.
%
% \DescribeMacro{\setlength}
% Set the length command, |#1|, to the value |#2|.
%
% \DescribeMacro{\addtolength}
% Increase the value of  the length command, |#1|, by the value |#2|.
%
% \DescribeMacro{\settowidth}
% Set the length, |#1| to the width of a box containing |#2|.
%
% \DescribeMacro{\settoheight}
% Set the length, |#1| to the height of a box containing |#2|.
%
% \DescribeMacro{\settodepth}
% Set the length, |#1| to the depth of a box containing |#2|.
%
% \StopEventually{}
%
%    \begin{macrocode}
%<*2ekernel>
\message{lengths,}
%    \end{macrocode}
%
% \begin{macro}{\newlength}
%    \begin{macrocode}
\def\newlength#1{\@ifdefinable#1{\newskip#1}}
%    \end{macrocode}
% \end{macro}
%
% \begin{macro}{\setlength}
% \changes{v1.1c}{2015/01/08}{add = to ensure first length argument is terminated.
%             (latexrelease)}
%    \begin{macrocode}
%</2ekernel>
%<latexrelease>\IncludeInRelease{2015/01/01}%
%<latexrelease>                 {\setlength}{Using \setlength with \dimen0}%
%<*2ekernel|latexrelease>
%    \end{macrocode}
%    \begin{macrocode}
\def\setlength#1#2{#1 #2\relax}
%</2ekernel|latexrelease>
%<latexrelease>\EndIncludeInRelease
%<latexrelease>\IncludeInRelease{0000/00/00}%
%<latexrelease>                 {\setlength}{Using \setlength with \dimen0}%
%<latexrelease>\def\setlength#1#2{#1#2\relax}
%<latexrelease>\EndIncludeInRelease
%<*2ekernel>
%    \end{macrocode}
% \end{macro}
%
% \begin{macro}{\addtolength}
% |\relax| added 24 Mar 86
%    \begin{macrocode}
\def\addtolength#1#2{\advance#1 #2\relax}
%    \end{macrocode}
% \end{macro}
%
%
% \begin{macro}{\settoheight}
% \changes{LaTeX2e}{1993/11/22}{Macro added}
% \begin{macro}{\settodepth}
% \changes{LaTeX2e}{1993/11/22}{Macro added}
% \begin{macro}{\settowidth}
% \begin{macro}{\@settodim}
% \changes{LaTeX2e}{1993/11/22}{Macro added}
% \changes{v1.0a}{1994/03/07}{(DPC) Extra group for colour}
%    The obvious analogs of |\settowidth|.
%    \begin{macrocode}
\def\@settodim#1#2#3{\setbox\@tempboxa\hbox{{#3}}#2#1\@tempboxa
%    \end{macrocode}
%    Clear the memory afterwards (which might be a lot).
%    \begin{macrocode}
       \setbox\@tempboxa\box\voidb@x}
\DeclareRobustCommand\settoheight{\@settodim\ht}
\DeclareRobustCommand\settodepth {\@settodim\dp}
\DeclareRobustCommand\settowidth {\@settodim\wd}
%    \end{macrocode}
% \end{macro}
% \end{macro}
% \end{macro}
% \end{macro}
%
%
% \begin{macro}{\@settopoint}
%    This macro takes the contents of the skip register that is
%    supplied as its argument and removes the fractional part to make
%    it a whole number of points. This can be used in class files to
%    avoid values like |345.4666666pt| when calulating a dimension.
% \changes{LaTeX2e}{1993/11/22}{Macro added}
%    \begin{macrocode}
\def\@settopoint#1{\divide#1\p@\multiply#1\p@}
%</2ekernel>
%    \end{macrocode}
% \end{macro}
%
% \Finale
%
