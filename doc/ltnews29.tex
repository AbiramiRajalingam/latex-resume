% \iffalse meta-comment
%
% Copyright 2018
% The LaTeX3 Project and any individual authors listed elsewhere
% in this file.
%
% This file is part of the LaTeX base system.
% -------------------------------------------
%
% It may be distributed and/or modified under the
% conditions of the LaTeX Project Public License, either version 1.3c
% of this license or (at your option) any later version.
% The latest version of this license is in
%    http://www.latex-project.org/lppl.txt
% and version 1.3c or later is part of all distributions of LaTeX
% version 2005/12/01 or later.
%
% This file has the LPPL maintenance status "maintained".
%
% The list of all files belonging to the LaTeX base distribution is
% given in the file `manifest.txt'. See also `legal.txt' for additional
% information.
%
% The list of derived (unpacked) files belonging to the distribution
% and covered by LPPL is defined by the unpacking scripts (with
% extension .ins) which are part of the distribution.
%
% \fi
% Filename: ltnews29.tex
%
% This is issue 29 of LaTeX News.

\documentclass{ltnews}
\usepackage[T1]{fontenc}

\usepackage{lmodern,url,hologo,csquotes}

\providecommand\acro[1]{\textsc{#1}}
\providecommand\meta[1]{$\langle$\textit{#1}$\rangle$}


\providecommand\XeTeX{\hologo{XeTeX}}
\providecommand\LuaTeX{\hologo{LuaTeX}}
\providecommand\pdfTeX{\hologo{pdfTeX}}

\newcommand\githubissue[2][]{\ifhmode\unskip\fi
     \quad\penalty500\strut\nobreak\hfill
     \mbox{\small\itshape(github issue#1 #2)}\par}

\newcommand\sxissue[1]{\ifhmode\unskip\fi
     \quad\penalty500\strut\nobreak\hfill
     \mbox{\small\itshape\url(#1)}\par}

\let\cls\pkg
\newcommand\env[1]{\texttt{#1}}


\publicationmonth{December}
\publicationyear{2018}

\publicationissue{29}

\begin{document}

\maketitle
\tableofcontents

\setlength\rightskip{0pt plus 3em}

\bigskip

\section{Introduction}

The December 2018 release of \LaTeX{} is essentially a maintenance
release where we fixed a few (rather obscure) older and newer bugs in
the software. Perhaps the most interesting aspect is going to be a
better support for key value option lists, but its full potential will
only appear once the existing key value handlers starting using the
new functionality.

\section[Bug reports for core \LaTeXe{} and packages]
        {Bug reports for core \LaTeXe{} and packages maintained by the Project Team}



In spring 2018 we established a new issue tracking system (GitHub
issues) for the \LaTeX{} core and the packages maintained by the
\LaTeX{} Project team with an updated procedure for how to report a bug or
problem.

Initial experience is good and people reporting problems follow the
guidelines and produce helpful working examples that show the
problem---thanks for that.

The detailed requirements and the workflow for reporting a bug in the
core \LaTeX{} software is documented at
\begin{quote}
\url{https://www.latex-project.org/bugs/}
\end{quote}
and with further details also discussed in~\cite{Mittelbach:TB39-1}.


\section{Changes to the \LaTeX{} kernel}

\subsection{Preparation for improving key value support on option lists}

\emph{explain depending on implementation}
%
\githubissue{85}


\subsection{UTF-8:\ updates to the default input encoding}

In the April 2018 release of \LaTeX{} we changed the default encoding
from 7-bit \acro{ascii} to to UTF-8 if used with classic \TeX\ or
\hologo{pdfTeX}, see \textit{\LaTeX{} News~28}~\cite{ltnews28} for
details.

Now, after half a year of experience with the new default we made a
small number of adjustments to further improve the user experience.
These include
\begin{itemize}
\item
  Some improvements when displaying error messages about UTF-8
  characters that have not been set up for use with \LaTeX{} or are
  invalid for some other reason;
%
\githubissue[s]{60, 62 and 63}
%
\item
  Added a number of missing declarations for characters that are in
  fact available with the default fonts, e.g., \cs{j} \enquote{\j}
  (0237), \cs{SS} \enquote{\SS} (1E9E), \verb=\k{}= \enquote{\k{ }}
  (02DB) and \verb=\.{}= \enquote{\.{ }} (02D9)
  \item
    Add correct names for \cs{guillemetleft} \enquote{\guillemotleft}
    and \cs{guillemetright} \enquote{\guillemotright} in addition to
    the old (but wrong) Adobe names in all encoding files. Adobe
    called them by mistake Guillemot which is a bird.
%    
  \githubissue{65}
\end{itemize}



\subsection{Fixed \cs{verb*} and friends in \XeTeX{} and \LuaTeX{}}


The original \cs{verb*} and \texttt{verbatim*} in \LaTeX{} was coded
under the assumption that the position of the space character (i.e.,
ASCII 32) in a Typewriter Font contains a visible space
``\verb*= =''. This is correct in \pdfTeX{} for the predominant
%
font encodings \texttt{OT1} and \texttt{T1} but unfortunately no
longer true in Unicode engines using the \texttt{TU} encoding. Here
the slot normally contains a real space so that \cs{verb*} produced
the same results as \cs{verb}.

The \cs{verb*} code now always uses the newly introduced command
\cs{verbvisiblespace} to produce the visible space character and this
command will get an appropriate definition in the different
engines. With \pdfTeX{} it will simply call \cs{asciispace} which is a
posh name for ``select the character 32 in the current font'' but with
the Unicode engines the default definition is
\begin{verbatim}
 \DeclareRobustCommand\verbvisiblespace
    {\leavevmode
     {\usefont{OT1}{cmtt}{m}{n}\asciispace}}
\end{verbatim}
i.e., the visible space from Computer Modern Typewriter regardless of
the current typewriter font is chosen.  Internally it is ensured that
the character has exactly the same width as the other characters in
the current monospaced font, so that code displays line up properly.

It is possible to redefine this command to select your own character,
for example
\begin{verbatim}
 \DeclareRobustCommand\verbvisiblespace
    {\textvisiblespace}
\end{verbatim}
would select the the ``official'' visible space character of the
current font. This may look like the natural default but it wasn't
chosen as default because many fonts just don't have that unicode
character, or if they have one with a strange shape.
%
\githubissue[s]{69 and 70}

        
\subsection{Error message corrected}
Trying to redefine an undefined command could in a few cases generate
an error message with a missing space, e.g.,
\verb=\renewcommand\1{...}= gave
\begin{verbatim}
   LaTeX Error: \1undefined.
\end{verbatim}
This is now fixed.
%
\githubissue{41}


\subsection{Avoid page breaks caused by invisible commands}

Commands like \cs{label} or \cs{index} could generate a page break
possibility in places where a page break was otherwise prohibited,
e.g., if use between two consecutive headings. This has now been
corrected. If for some reason you really want a break and you relied
on this faulty behavior, you can always add it using \cs{pagebreak}
with or with optional argument.
%
\githubissue{81}



\subsection{Prevent spurious spaces when reading table of contents data}

When table of contents data is read in from a \texttt{.toc} file the
new line character at the end of each line is converted by \TeX{} to a
space. In normal processing this is harmless (as \TeX{} is doing this
processing in vertical mode and each line in the file represents a
single line (paragraph) in the table of contents. If, however, this is
done in horizontal mode, which is sometimes the case, then these
spaces appear in the output. If you then omit certain input lines
(e.g., because you do not display TOC data below a certain level),
then these spaces accumulate in the typeset output and you get
surprising gaps inside the text.

The new code now adds a \texttt{\%} sign at the end of problematic
lines so that \TeX{} will not generate such spaces that may survive
and disturb the printed result. As some third party packages have
augmented or changed the core \LaTeX{} functionality in that area (for
example, by adding additional arguments to the commands in TOC files)
the code uses a conservative approach and adds the \texttt{\%} signs
are only added if certain conditions are meet. Thus, some packages
might require updates if they want to benefit from the correction in
particular if they unconditionally overwrite \LaTeX{}'s
\cs{addcontentsline} definition.
%
\githubissue{73}


\subsection{Prevent protrusion in table of contents lines}

In \TeX{}'s internal processing model paragraph data is one of the
major data structures. As a result many things are internally modeled
as paragraphs even if they are conceptually not ``text paragraphs'' in
the traditional sense. In a few cases this has some surprising effects
not necessarily for the better. One example are standard TOC entries,
where you have a heading data followed by some dot leaders and a page
number at the right, e.g.
\begin{quote}
  \contentsline {subsection}{Error message corrected}{2}{section*.7}
\end{quote}
The space reserved for the page number is of a fixed width, so that
the dots always end in the same place. Well, did end in the same place
until the event of protrusion support in the \TeX{} engines. Now, with
the \pkg{microtype} package being loaded it became possible that the
page number protruded slightly into the margin (even though inside a
box) and as a result the page number box shifted. With enough bad luck
this then got you another dot in the line sticking out like a sore
thumb as exhibited in the question on StackExchange that triggered the
correction.

\LaTeX{} now takes care that there will be no protrusion happening on
such lines even if is generally enabled for the whole document.
%
\sxissue{https://tex.stackexchange.com/q/172785}



\subsection{Start LR-mode for \cs{thinspace} and friends}

In \LaTeX{} commands that are only intended for paragraph (L-R) mode
are carful to start paragraph mode if necessary so that they can be
used at a start of a paragraph without surprising consequences. For a
few horizontal spacing commands such as \cs{thinspace}
(a.k.a.\ ``\cs{,}'') and other support commands such as \cs{smash} or
\cs{phantom} this was overlooked so that they ended up adding vertical
space if used at the beginning of a paragraph or in case of \cs{smash}
on a paragraph of their own. This has now been corrected.  A
corresponding update has been made for \pkg{amsmath} as there these
commands are also defined.
%
\githubissue[s]{49 and 50}
        

        
\subsection{Guarding \cs{pfill} in \pkg{doc}}

For presenting index entries to code fragments and the like the
\pkg{doc} package has a \cs{pfill} command that generated a line of
dots leading from a command name to the page or code line numbers
inside the index. If necessarily it would automatically split over two
lines. That worked well enough for a quarter century, but we
discovered recently that it is broken inside of the \cls{ltugboat}
class where it sometimes produced ugly spaced out continuation lines.

The reason turned out to be a redefinition of the \LaTeX{} command
\cs{nobreakspace} (\verb=~=) inside that class which removed any
preceding space (and unfortunately also the dots on the continuation
line). While one can argue that this is a questionable redefinition,
it is so long in that class that changing it would certainly break
older documents. So instead we now guard against that removal.
%
\githubissue[s]{25 and 75}

        


        
\section{Changes to packages in the tools category}

\subsection{Sometimes the \pkg{trace} package turned off too much}

The \pkg{trace} package is a useful little tool for tracing macro
execution as it hides certain lengthly and typically uninteresting
expansions resulting from font loading and similar
activities. However, it had the problem that it did also reset other
tracing settings like \cs{showoutput} in such situations so that you
couldn't use \cs{showoutput} in the preamble and get a symbolic output
of all your pages in the document. This has now been corrected.



\subsection{Update to \pkg{xr}}

The \pkg{xr} package has been merged with the
\pkg{xcite} package so now handles \cs{cite} as well as \cs{ref} to
items defined in an external document.  In addition, the code that
reads the \texttt{.aux} file has been made more robust and now correctly
ignores conditionals (added by \pkg{hyperref} and other packages)
rather than generating low level parsing errors.
%
\sxissue{https://tex.stackexchange.com/a/452321}


\subsection{Column data for \env{multicols*} sometimes vanished}

In certain situations involving \env{multicols*} together with more
explicit \cs{columnbreak} requests than columns on current page, data
could vanish due to losing an internal penalty marking the end of the
environment. This has been corrected by explicitly reinserting that
penalty if necessary.
%
\githubissue{53}



\subsection{Prevent color leak in \pkg{array}}

In some cases using color inside a \env{tabular} cell could leak out
into the surrounding text.
%
\githubissue{72}



\section{Changes to packages in the amsmath category}

The changes in the kernel made for \cs{thinspace}, \cs{smash},
etc.\ had to be reflected in the \pkg{amsmath} package code so that
loading that package wouldn't revert them.
%
\githubissue[s]{49 and 50}





\section{Website updates}

\subsection{Publications area reorganized and extended}

To help readers to find relevant information in more convenient and
easy way the website area on publications by the \LaTeX{} Project Team
was reorganized and extended (many more abstracts added). We now
provide the articles, talks and supplementary data structured both by
year as well as by major topics~\cite{site-pub}. Feel free to take a
look.

\subsection{Japanese translations of the user's guide}

Yukitoshi FUJIMURA kindly translated two documents that are
distributed with standard \LaTeX{} to the Japanese language. These are
are
\begin{itemize}
\item
    \LaTeXe{} for authors
\item
    User’s Guide for the \pkg{amsmath}~\cite{amsldoc}
\end{itemize}
They can be found on the website documentation page~\cite{site-doc}.
You will now also find there a compiled version of the full \LaTeXe{}
source code (with index etc.\@) and a number of other goodies.


      

\begin{thebibliography}{9}
  
\bibitem{Mittelbach:TB39-1} Frank Mittelbach:
  \emph{New rules for reporting bugs in the \LaTeX{} core software}.  
  In: TUGboat, 39\#1, 2018.
  \url{https://latex-project.org/publications/}

\bibitem{site-doc} 
  \emph{\LaTeX{} documentation on the \LaTeX{} Project Website}.\\  
  \url{https://latex-project.org/documentation/}

\bibitem{site-pub} 
  \emph{\LaTeX{} Project publications on the \LaTeX{} Project Website}.\\
  \url{https://latex-project.org/publications/}

\bibitem{amsldoc} American Mathematical Society and The \LaTeX3 Project:
  \emph{User's Guide for the \texttt{amsmath} package} (Version 2.1).  
  April 2018.
  Available from
  \url{https://www.ctan.org}
  and distributed as part of every \LaTeX{} distribution.

\end{thebibliography}

\end{document}




\bibitem{Mittelbach:TB38-2-213} Frank Mittelbach:
  \emph{\LaTeX{} table columns with fixed widths}.  
  In: TUGboat, 38\#2, 2017.
  \url{https://www.latex-project.org/publications/}

