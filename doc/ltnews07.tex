% \iffalse meta-comment
%
% Copyright 1993-2019
% The LaTeX3 Project and any individual authors listed elsewhere
% in this file. 
% 
% This file is part of the LaTeX base system.
% -------------------------------------------
% 
% It may be distributed and/or modified under the
% conditions of the LaTeX Project Public License, either version 1.3c
% of this license or (at your option) any later version.
% The latest version of this license is in
%    http://www.latex-project.org/lppl.txt
% and version 1.3c or later is part of all distributions of LaTeX 
% version 2008 or later.
% 
% This file has the LPPL maintenance status "maintained".
% 
% The list of all files belonging to the LaTeX base distribution is
% given in the file `manifest.txt'. See also `legal.txt' for additional
% information.
% 
% The list of derived (unpacked) files belonging to the distribution 
% and covered by LPPL is defined by the unpacking scripts (with 
% extension .ins) which are part of the distribution.
% 
% \fi
% Filename: ltnews07.tex

% This is issue 7 of LaTeX News.

\documentclass
%    [lw35fonts]
   {ltnews}

% \usepackage[T1]{fontenc}

\publicationmonth{June}
\publicationyear{1997}
\publicationissue{7}

\begin{document}

\maketitle

\section{T1 encoded Computer Modern fonts}
As in the last release the base \LaTeX\ distribution contains
three different sets of `fd' files for T1 encoded fonts.

In this release the default installation uses \texttt{ec.ins}
and so installs files suitable for the current `EC fonts'
distribution. If you have still not updated to the EC fonts and
are using the earlier test versions, known as DC then you should
unpack \texttt{newdc.ins} (for DC release 1.2 or later) or
\texttt{olddc.ins} (for the original releases of the DC fonts).
This should be done after unpacking \texttt{unpack.ins} but
before making the format by running ini\TeX{} on \texttt{latex.ltx}.
There are further details in \texttt{install.txt}.


\section{T1 encoded Concrete fonts}
The Metafont sources for T1 encoded `Concrete' fonts have been
removed from the \textsf{mfnss} distribution as they were based
on the now obsolete DC fonts release 1.1. Similarly the
\textsf{cmextra.ins} install file in the \textsf{base} distribution no
longer generates fd files for the `Concrete' fonts.
To use these fonts in either T1 or OT1 encoding it is
recommended that you obtain Walter Schmidt's \textsf{ccfonts} package
and fonts from CTAN \texttt{macros/latex/contrib/supported/ccfonts}.


\section{Further input encodings}
Two more \textsf{inputenc} packages have been added: for latin5,
thanks to H. Turgut Uyar; and for latin3, thanks to J\"org Knappen.


\section{Normalising spacing after punctuation}
The command \verb|\normalsfcodes| was introduced at the last patch
release. This is normally given the correct definition automatically
and so need not be explicitly set. It is used to correct a problem,
reported by Donald Arseneau, that punctuation in page headers has
always (in all known \TeX\ formats) been potentially incorrect if the
page break happens while a local setting of the space codes (for
instance by the command \verb|\frenchspacing|) is in effect. A common
example of this happening in \LaTeX\ is in the \textsf{verbatim}
environment.


\section{Accessing Bold Math Symbols}
The \textsf{tools} distribution contains a new package, \textsf{bm},
which defines a command \verb|\bm| that allows individual bold symbols
to be accessed within a math expression (in contrast to
\verb|\boldmath| which makes whole math expressions default to bold
fonts). It is more general than the existing \textsf{amsbsy} package;
however, to ease the translation of documents between these two
packages, \textsf{bm} makes \verb|\boldsymbol| an alias for
\verb|\bm|.

This package was previously made available from the `contrib' area of
the CTAN archives, and as part of Y\&Y's \LaTeX\ support for the
MathTime fonts.


\section{Policy on standard classes}

Many of the problem reports we receive concerning the standard classes
are not concerned with bugs but are suggesting, more or less politely,
that the design decisions embodied in them are `not optimal' and
asking us to modify them.

There are several reasons why we have decided not to make such changes
to these files.
\begin{itemize}
\item
  However misguided, the current behaviour is clearly what was
  intended when these classes were designed.
\item
  It is not good practice to change such aspects of `standard classes'
  because many people will be relying on them.
\end{itemize}

We have therefore decided not to even consider making such
modifications, nor to spend time justifying that decision.  This does
not mean that we do not agree that there are many deficiencies in the
design of these classes, but we have many tasks with higher priority
than continually explaining why the standard classes for \LaTeX{}
cannot be changed.

We would, of course, welcome the production of better classes, or of
packages that can be used to enhance these classes.


\section{New addresses for TUG}
For information about joining the \TeX{} Users Group, and about lots
of other \LaTeX-related matters, please contact
them at their new address:
\begin{quote}\small
   \TeX{} Users Group, P.O. Box 1239,\\
   Three Rivers, CA~93271-1239, USA\\
   Fax:~+1~209~561~4584\\
   E-mail: \texttt{tug@mail.tug.org}\\
   URL: \texttt{http://www.tug.org/}
\end{quote}

\end{document}
