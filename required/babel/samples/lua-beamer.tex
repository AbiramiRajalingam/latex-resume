\PassOptionsToPackage{unicode}{hyperref}
\documentclass{beamer}

\usepackage[nil,bidi=basic-r]{babel}

\babelprovide[import=ar,main]{arabic}
\babelfont{rm}{FreeSerif}
\babelfont{sf}{FreeSerif}

\mode<presentation>{\usetheme{Luebeck}}

\title{ميكانيكا}

\author{From Wikipedia}
\date{\today}

\begin{document}

\begin{frame}
\titlepage
\end{frame}

\begin{frame}
\frametitle{\contentsname}
\tableofcontents
\end{frame}

\section{تطبيقية}

\begin{frame}
\frametitle{أقسام}
\begin{itemize}
\item فيزياء تطبيقية
\item فيزياء تجريبية
\item فيزياء نظرية
\end{itemize}
\end{frame}

\section{طاقة, حركة}

\begin{frame}
\frametitle{طاقة, حركة}
\begin{itemize}
\item ديناميكا حرارية
\item ميكانيكا
\begin{itemize}
\item كلاسيكية
\begin{itemize}
\item لاغرانج
\item هاملتوني
\end{itemize}
\item المتصل
\item سماوية
\end{itemize}
\end{itemize}
\end{frame}

\section{Blocks}

\begin{frame}
\frametitle{Blocks}

\begin{block}{Lorem}
  \foreignlanguage*{nil}{On 21 April 1820, during a lecture, Ørsted
  noticed a compass needle deflected from magnetic north when an
  electric current from a battery was switched on and off.}
\end{block}

\begin{block}{أورستد}
  لاحظ هانز أورستد في 21 أبريل 1820 وهو يُعد أحد التجارب أن إبرة
  البوصلة تنحرف عن اتجاهها نحو الشمال عندما كان يغلق ويفتح التيار في
  دائرة كهربائية يُعدها.
\end{block}
\end{frame}

\end{document}
