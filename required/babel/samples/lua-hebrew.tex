\documentclass{article}

\usepackage[unicode]{hyperref}

\ifx\directlua\undefined
%   With XeTeX, which often works with isolated words, but note
%   "multi-word" L text is messed up:
  \usepackage[bidi=bidi]{babel}
  \rightfootnoterule
\else
  \usepackage[bidi=basic,layout=tabular]{babel}
\fi

\babelprovide[main,import]{hebrew}
\babelprovide{rl}

\babelfont{rm}{FreeSerif}
% \babelfont{rm}{FrankRuehl}

\title{מיון עולם הטבע\\\large From Wikipedia}

\begin{document}

\maketitle

\tableofcontents

\section{המיון של קארולוס ליניאוס}

חלוקת עולם הטבע לשלוש ממלכות, בספרו של ארנסט הקל משנת
תורת המיון המדעית הראשונה הוצעה בשנת 1735 על ידי המדען השוודי קארולוס
ליניאוס (קארל פון לינה) בספרו ”Systema Naturae“ (בלטינית: "מערכת
הטבע"), והיא הביאה להתפתחותה של הטקסונומיה, ענף בתורת הביולוגיה שתפקידו
למיין ולארגן את כל היצורים במערך היררכי אחיד. ליניאוס מיין את עולם הטבע
לפי ההיררכיה הבאה:

\begin{center}
  \begin{tabular}{lll}
    \hline
    עברית	 & לטינית	  & אנגלית   \\
    \hline
    ממלכה	& Regnum	  & Kingdom \\
    מערכה	& Phylum   &         \\
    מחלקה	& Classis	 & Class   \\
    סדרה	& Ordo	 & Order   \\
    משפחה	& Familia	 & Family  \\
    סוג	& Genus    &         \\
    מין	& Species  &         \\
    \hline
  \end{tabular}
\end{center}

כל ממלכה מתחלקת למספר מערכות, בכל מערכה מספר מחלקות, וכן הלאה.

ניתן לזכור את ההיררכיה בעזרת המנמוניקה הבאה: ”Kings Play Chess On Fancy
Glass Stools“.

כאשר נזקקו הטקסונומים לטקסונים נוספים, הם השתמשו בטקסונים אלה בצרוף
הקידומות "על-" ו"תת-". כך למשל נוצרו "על-מחלקה" ו"תת-סדרה".

ליניאוס התייחס רק לשתי ממלכות: צמחים ובעלי חיים. ממלכת הצמחים כללה גם
אצות וחיידקים ואילו ממלכת בעלי החיים כללה גם חד תאיים כפרוטוזואה. עם
הזמן התגלו יצורים שלא ניתן היה לשייך אותם לאף אחת משתי הממלכות הללו,
ונוצר הצורך להוסיף ממלכות נוספות.

בתחילת שנות ה-80 של המאה ה-20 היה מקובל למיין את היצורים בעולם הטבע
לחמש הממלכות הבאות:

\begin{itemize}
  \item מונרה (Monera) שכללה חיידקים וכחוליות.
  \item פרוטוקטיסטה (Protoctista) שכללה חד תאיים כפרוטוזואה ואצות.
  \item פטריות (Fungi).
  \item צמחים (Plantae) שכללה גם כמה קבוצות של אצות.
  \item בעלי חיים (Animalia).
\end{itemize}

\section{מיון מודרני}

שיטת המיון של ליניאוס מבוססת על דמיון בתכונות היצורים ובמראם. כיום, עם
התפתחות הביולוגיה, הפכה שיטה זו למיושנת. שיטות מיון חדשות שהוצעו,
מבוססות בעיקר על דמיון גנטי. כתוצאה מכך, קבוצות רבות של יצורים שנחשבו
כיחידות סיסטמטיות, אינן עוד כאלה.

שיטות המיון המודרניות נוטות לחלק את היצורים לשלוש על-ממלכות:

\begin{itemize}
  \item חיידקים אמיתיים (Bacteria)
  \item חיידקים קדומים (Archaea)
  \item אוקריוטיים (Eucaryota)
\end{itemize}

חיידקים אמיתיים וחיידקים קדומים הם יצורים חסרי גרעין (פרוקריוטיים)
ואילו אוקריוטיים הם יצורים בעלי גרעין. האוקריוטיים עצמם ממוינים לקבוצות
של חד-תאיים כפרוטוזואה ואצות ולשלוש ממלכות: בעלי חיים (Metazoa), פטריות
וצומח (Viridiplantae). עם זאת, עד עתה לא הצליחה אף שיטה לתפוס את מקומו
של מיון ליניאוס הנפוץ והביולוגים עדיין משתמשים בה רבות.

המורכבות של מיון עולם הטבע מומחשת בשיטת מיון המערכה בעלי פרחים, הקבוצה
הגדולה ואחת החשובות ביותר של הצמחים, שעברה שינויים רבים במשך הזמן, ככל
שהתרחב הידע האנושי, והתגלו יחסים חדשים בין המשפחות השונות. המיון
המסורתי, שמתבסס על מכלול התכונות החיצוניות והמבנה של כל פרט, מיוצג על
ידי שיטת קרונקויסט שהוצעה ב־1981 על ידי ארתור קרונקויסט. עם זאת כיום
ברור שהשיטה הזאת איננה מייצגת נאמנה את מהלך ההתפתחות האבולוציונית.
ההסכמה הכללית לגבי מיון מערכת בעלי הפרחים החלה להתגבש רק עתה והיא
מתבססת על תכונות משותפות המתבטאות ברמה הגנטית. עבודה מרכזית בתחום זה
נעשתה על ידי קבוצת APG, שפרסמה ב־1998 מיון חדש ומקיף. עם המשך המחקר
הגנטי וגילוי ידע חדש, פורסם עדכון למיון זה בשנת 2003.

\section{קישורים חיצוניים}

\begin{itemize}
  \item ITIS – אתר של ממשלות ארצות הברית, קנדה ומקסיקו למיון עולם
  הטבע.\footnote{המיונים בוויקיפדיה נעשים על פי אתר זה.}
  \item Tree of Life Web Project – אתר פרויקט עץ החיים
\end{itemize}

% -----------

\part{פונקציה}

\section{הגדרה פורמלית}

מסמנים $f(x)=y$ אם ורק אם $(x,y) \in f$. במקרה כזה האיבר y קרוי
ה\textbf{תמונה} של $x$, ו-$x$ קרוי \textbf{מקור} של $y$. התנאי הראשון
מבטיח שלכל x ב-$X$ יש תמונה. התנאי השני מבטיח שתמונה זו היא יחידה. יחס
שהוא גם חד ערכי וגם מלא נקרא פונקציה.

אם מוותרים על התנאי הראשון (לא לכל איבר יש בהכרח תמונה) אז מתקבלת
פונקציה חלקית, ואם מוותרים על התנאי השני (ייתכנו איברים עם יותר
מתמונה אחת) מתקבלת פונקציה מרובה. אם מוותרים על שני התנאים יחדיו
מתקבל יחס במובנו הכללי.

שתי פונקציות $f,g$, עם אותו תחום וטווח, מוגדרות כשוות רק כאשר
$f(x)=g(x)$ לכל $x\in X$.

לכל $Z \subseteq X$ (תת-קבוצה כלשהי של $X$) הקבוצה $f(Z)$ היא תת-קבוצה
של Y המוגדרת: $f(Z) = \{f(z) \mid z\in Z \}$. כלומר זוהי התת-קבוצה של Y
הכוללת את כל האיברים שהם תמונות של איברי $Z$. אומרים על $f(Z)$ שהיא
התמונה של Z. בפרט, הקבוצה $f(X)$ הכוללת את כל האיברים ב-$Y$ שהם תמונה
של איבר כלשהו ב-$X$, נקראת ה\textbf{תמונה} של הפונקציה $f$.

לכל $Z \subseteq Y$ הקבוצה $f^{-1}(Z)$ היא תת-קבוצה של $X$ המוגדרת: $\{x
\in X \mid f(x)\in Z\}$. כלומר זוהי התת-קבוצה של $X$ הכוללת את כל האיברים
שהתמונה שלהם היא איבר ב-$Z$. אומרים על $f^{-1}(Z)$ שהיא המקור של $Z$.

אם $f: X \to Y$ היא פונקציה, ו-$Z \subseteq X$, אז הפונקציה $f|_Z : Z
\to Y$ המוגדרת $f|_Z(z) = f(z)$, נקראת \textbf{הצמצום של $f$ ל-$Z$}. זוהי
הפונקציה שזהה לפונקציה $f$, רק שתחומה הוא $Z$.

\section{Psalms}

\selectlanguage{rl}

Compare with \texttt{http://tanach.us/Tanach.xml?Ps1:1-1:6}.
Cantillation marks (the second paragraph) are wrong with both
\textsf{xetex} and \textsf{luatex}. The font is FreeSerif. Depending on
the font and the engine, the rendering may be better or worse.

\bigskip

\selectlanguage{hebrew}

אַשְׁרֵי־הָאִישׁ אֲשֶׁר לֹא הָלַךְ בַּעֲצַת רְשָׁעִים וּבְדֶרֶךְ
חַטָּאִים לֹא עָמָד וּבְמוֹשַׁב לֵצִים לֹא יָשָׁב׃ כִּי אִם בְּתוֹרַת
יְהוָה חֶפְצוֹ וּבְתוֹרָתוֹ יֶהְגֶּה יוֹמָם וָלָיְלָה׃ וְהָיָה כְּעֵץ
שָׁתוּל עַל־פַּלְגֵי מָיִם אֲשֶׁר פִּרְיוֹ יִתֵּן בְּעִתּוֹ וְעָלֵהוּ
לֹא־יִבּוֹל וְכֹל אֲשֶׁר־יַעֲשֶׂה יַצְלִיחַ׃ לֹא־כֵן הָרְשָׁעִים כִּי
אִם־כַּמֹּץ אֲשֶׁר־תִּדְּפֶנּוּ רוּחַ׃ עַל־כֵּן לֹא־יָקֻמוּ רְשָׁעִים
בַּמִּשְׁפָּט וְחַטָּאִים בַּעֲדַת צַדִּיקִים׃ כִּי־יוֹדֵעַ יְהוָה
דֶּרֶךְ צַדִּיקִים וְדֶרֶךְ רְשָׁעִים תֹּאבֵד׃

אַ֥שְֽׁרֵי־הָאִ֗ישׁ‪ אֲשֶׁ֤ר ׀ לֹ֥א הָלַךְ֮ בַּעֲצַ֪ת רְשָׁ֫עִ֥ים
וּבְדֶ֣רֶךְ חַ֭טָּאִים לֹ֥א עָמָ֑ד וּבְמוֹשַׁ֥ב לֵ֝צִ֗ים לֹ֣א יָשָֽׁב׃
כִּ֤י אִ֥ם בְּתוֹרַ֥ת יְהוָ֗ה חֶ֫פְצ֥וֹ וּֽבְתוֹרָת֥וֹ יֶהְגֶּ֗ה
יוֹמָ֥ם וָלָֽיְלָה׃ וְֽהָיָ֗ה כְּעֵץ֮ שָׁת֪וּל עַֽל־פַּלְגֵ֫י מָ֥יִם
אֲשֶׁ֤ר פִּרְי֨וֹ ׀ יִתֵּ֬ן בְּעִתּ֗וֹ וְעָלֵ֥הוּ לֹֽא־יִבּ֑וֹל וְכֹ֖ל
אֲשֶׁר־יַעֲשֶׂ֣ה יַצְלִֽיחַ׃ לֹא־כֵ֥ן הָרְשָׁעִ֑ים כִּ֥י אִם־כַּ֝מֹּ֗ץ
אֲ‍ֽשֶׁר־תִּדְּפֶ֥נּוּ רֽוּחַ׃ עַל־כֵּ֤ן ׀ לֹא־יָקֻ֣מוּ רְ֭שָׁעִים
בַּמִּשְׁפָּ֑ט וְ֝חַטָּאִ֗ים בַּעֲדַ֥ת צַדִּיקִֽים׃ כִּֽי־יוֹדֵ֣עַ
יְ֭הוָה דֶּ֣רֶךְ צַדִּיקִ֑ים וְדֶ֖רֶךְ רְשָׁעִ֣ים תֹּאבֵֽד׃

\end{document}
