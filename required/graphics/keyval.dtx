%
% \iffalse
%% keyval.dtx Copyright (C) 1993 1994 1995 1997 1998 1999 David Carlisle
%%            Copyright (C) 2000-2021 David Carlisle, LaTeX Project
%% The LaTeX Project and any individual authors listed elsewhere
%% in this file.
%%
%% This file is part of the Standard LaTeX `Graphics Bundle'.
%% It may be distributed under the terms of the LaTeX Project Public
%% License, as described in lppl.txt in the base LaTeX distribution.
%% Either version 1.3c or, at your option, any later version.
%%
%
%<*dtx>
          \ProvidesFile{keyval.dtx}
%</dtx>
%<*!plain>
%<package&!plain>\NeedsTeXFormat{LaTeX2e}
%<package&!plain>\ProvidesPackage{keyval}
%<driver>        \ProvidesFile{keyval.drv}
% \fi
%                \ProvidesFile{keyval.dtx}
                 [2014/10/28 v1.15 key=value parser (DPC)]
%
% \iffalse
%</!plain>
%<*driver>
\documentclass{ltxdoc}
\usepackage{keyval}
\begin{document}
 \DocInput{keyval.dtx}
\end{document}
%</driver>
% \fi
%
% \GetFileInfo{keyval.dtx}
% \title{The \textsf{keyval} package\thanks{This file
%        has version number \fileversion, last
%        revised \filedate.}}
% \author{David Carlisle}
% \date{\filedate}
% \MaintainedByLaTeXTeam{graphics}
% \maketitle
%
% \begin{abstract}
%  A \LaTeX\ package implementing a system allowing the setting
%  of parameters (or `named arguments' with a
%  \meta{key}${}={}$\meta{value} syntax.
%
%  Eg: |\foo[height=3in, shadow = true ]{bar}|
% \end{abstract}
%
%
% \changes{v1.03}
%         {1993/10/13}{Initial version}
% \changes{v1.04}{1993/11/15}
%         {Upgrade after comments from Timothy van Zandt}
% \changes{v1.05}{1993/11/17}
%         {Further small improvements}
% \changes{v1.06}{1994/02/01}
%         {Update to LaTeX2e}
% \changes{v1.07}{1994/03/15}
%         {New style ltxdoc}
% \changes{v1.08}{1994/09/12}
%         {Improve docstrip handling}
% \changes{v1.09}{1995/09/21}
%         {Move a comma for graphics/1698}
% \changes{v1.14}{2014/04/25}
%         {Make most definitions \cs{long} to allow \cs{par} in values: graphics/3446}
% \changes{v1.14}{2014/04/25}
%         {Only strip a single brace pair from values: graphics/3446}
% \changes{v1.15}{2014/05/08}
%         {revert this change: Only strip a single brace pair from values: graphics/3446}
%
%
% This package implements a system of defining and using sets of
% parameters, which are set using the syntax \meta{key}=\meta{value}.
%
% For each keyword in such a set, there exists a function which is
% called whenever the parameter appears in a parameter list. For
% instance if the set |dpc| is to have the keyword |scale| then I
% would define.\\
% | \define@key{dpc}{scale}{scale ({\tt\string#1})\\}|\\
% The first argument of |\define@key| is the  set of keywords being
% used, the second is the keyword, and the third is the function to
% call. This function will be given as |#1| the \meta{value} specified
% by the user.
%
% Normally it is an error to omit the `=\meta{value}' however if an
% optional \meta{value} is supplied when the keyword is defined, then
% just the keyword need be supplied.\\
% |\define@key{dpc}{clip}[true]{...}|\\
% For `|clip|' you can go `|clip = true|' or `|clip = false|' or
% just `|clip|', which is the same as `|clip = true|'
%
% To use these keywords, just call `|\setkeys|' with a comma
% separated list of settings, each of the form
% \meta{key}=\meta{value}, or just \meta{key}. Any white space around
% the `|=|' and `|,|' is ignored.
%
% As the \meta{key} is passed as a macro argument, if it consists
% entirely of a |{ }| group, the outer  braces are stripped off. Thus
% |,key=foo,| and |,key={foo},| are equivalent. This fact enables one to
% `hide' any commas or equals signs that must appear in the value. i.e.\
% in |foo={1,2,3},bar=4|, |foo| gets the value |1,2,3|, the comma after
% |1| does not terminate the keyval pair, as it is `hidden' by the
% braces.
%
% Empty entries, with nothing between the commas, are silently ignored.
% This means that it is not an error to have a comma after the last
% term, or before the first.
%
% \section{Example}
%
% We may extend the examples above to give  a `fake' graphics
% inclusion macro, with a syntax similar to that used in the psfig
% macros.
%
% \makeatletter
% \def\dpcgraphics{\@ifnextchar[\@dpcgraphics{\@dpcgraphics[]}}
%
% |\dpcgraphics| has one optional argument which is passed through
% |\setkeys|, and one mandatory argument, the filename. It actually
% just typesets its arguments, for demonstration.
%
% \def\@dpcgraphics[#1]#2{{\setkeys{dpc}{#1}INPUT: #2}}%
%
% \define@key{dpc}{scale}{scale ({\tt\string#1\relax})\\}
% \define@key{dpc}{height}{height ({\tt#1})\\}
% \define@key{dpc}{width}{width ({\tt#1})\\}
% \define@key{dpc}{bb}{bounding box ({\tt#1})\\}
% \define@key{dpc}{clip}[true]{clip ({\tt\string#1\relax})\\}
% \makeatother
%
% The declared keys are: |scale|, |height|, |width|, |bb|,
% and |clip|. Except for the last, they must all be given a value if
% used.
%
% Note how in the following, any white space around |=| or |,| is
% ignored, as are the `empty' arguments caused by extra commas. Note
% also that each macro receives \emph{exactly} the tokens that you
% specify as arguments, no premature expansion is done.
%
% \begin{verbatim}
% \def\dpcgraphics{\@ifnextchar[\@dpcgraphics{\@dpcgraphics[]}}
% \def\@dpcgraphics[#1]#2{{\setkeys{dpc}{#1}INPUT: #2}}
%
% \define@key{dpc}{scale}{scale ({\tt\string#1\relax})\\}
% \define@key{dpc}{clip}[true]{clip ({\tt\string#1\relax})\\}
% \end{verbatim}
%
% \begin{minipage}{.4\textwidth}
% \begin{verbatim}
% \def\scalemacro{9}
% \dpcgraphics
% [ height =4in, ,
%   width = 3in,
%   scale = \scalemacro,
%   bb = 20 20 300 400 ,
%   clip,
%  ]{aaa}
% \end{verbatim}
% \end{minipage}
% \hfill
% \begin{minipage}{.4\textwidth}
% \def\scalemacro{9}
%\dpcgraphics
% [ height =4in, ,
%   width = 3in,
%   scale = \scalemacro ,
%   bb = 20 20 300 400 ,
%   clip ,
%  ]{aaa}
% \end{minipage}
%
%
% \section{The Internal Interface}
% A declaration of the form:\\
% |\define@key{family}{key}{...}|\\
% Defines a macro |\KV@prefix@key| with one argument. When used in a
% keyval list, the macro receives the value as its argument.
%
% A declaration of the form:\\
% |\define@key{family}{key}[default]{...}|\\
% Defines a macro |\KV@family@key| as above, however it also defines the
% macro |\KV@family@key@default| as a macro with no arguments, and
% definition\\
%  |\KV@family@key{default}|.
%
% Thus if macros are defined using |\define@key|, the use of a key with
% no value \ldots|,foo,|\ldots\ is always equivalent to the use of the
% key with some value, \ldots|,foo=default,|\ldots. However a package
% writer may wish that the `default' behaviour for some key is not
% directly equivalent to using that key with a value. (In particular, as
% pointed out to me by Timothy Van Zandt, you may wish to omit error
% checking on the default value as you know it is correct.) In these
% cases one simply needs to define the two macros
% |\KV@|\meta{family}|@key| and |\KV@|\meta{family}|@key@default|
% directly using |\def| (or |\newcommand|). I do not supply a user
% interface for this type of definition, but it is supported in the
% sense that I will try to ensure that any future upgrades of this
% package do not break styles making use of these `low level'
% definitions.
%
% \StopEventually{}
%
% \section{The Macros}
%
% From version~1.05, all `internal' macros associated to keys have names
% of the form:\\
% |\KV@|\meta{family}|@|\meta{key} or
% |\KV@|\meta{family}|@|\meta{key}|@|\meta{default}
%
%    \begin{macrocode}
%<*package>
%    \end{macrocode}
%
% \begin{macro}{\setkeys}
% The top level macro. |#2| should be a comma separated values of the
% form  \meta{key} |=| \meta{value} or just simply \meta{key}.
% The macro associated with this key in the `family' |#1| is called with
% argument \meta{value}. The second form is only allowed if the key was
% declared with a default value.
%    \begin{macrocode}
\long\def\setkeys#1#2{%
%    \end{macrocode}
%  Save the `family' for later. Then begin acting on the comma
%  separated list.
% \changes{v1.11}{1998/06/05}
%         {Make \cs{@tempc} safe (in case it is an \cs{if}.}
%    \begin{macrocode}
  \def\KV@prefix{KV@#1@}%
  \let\@tempc\relax
  \KV@do#2,\relax,}
%    \end{macrocode}
% \end{macro}
%
% \begin{macro}{\KV@do}
% Iterate down the list of comma separated argument pairs.
% \changes{v1.14}{2014/04/25}
%         {Add \cs{@empty} not \cs{empty}}
%    \begin{macrocode}
\long\def\KV@do#1,{%
 \ifx\relax#1\@empty\else
  \KV@split#1==\relax
  \expandafter\KV@do\fi}
%    \end{macrocode}
% \end{macro}
%
% \begin{macro}{\KV@split}
% Split up the keyword and value, and call the appropriate command.
% This macro was slightly reorganised for version 1.04, after some
% suggestions from Timothy Van Zandt.
%    \begin{macrocode}
\long\def\KV@split#1=#2=#3\relax{%
  \KV@@sp@def\@tempa{#1}%
  \ifx\@tempa\@empty\else
    \expandafter\let\expandafter\@tempc
      \csname\KV@prefix\@tempa\endcsname
    \ifx\@tempc\relax
%<plain>      \KV@err
%<!plain>      \KV@errx
       {\@tempa\space undefined}%
    \else
      \ifx\@empty#3\@empty
        \KV@default
      \else
        \KV@@sp@def\@tempb{#2}%
        \expandafter\@tempc\expandafter{\@tempb}\relax
      \fi
    \fi
  \fi}
%    \end{macrocode}
% \end{macro}
%
% \begin{macro}{\KV@default}
% Run the default code, or raise an error.
%    \begin{macrocode}
\def\KV@default{%
  \expandafter\let\expandafter\@tempb
    \csname\KV@prefix\@tempa @default\endcsname
  \ifx\@tempb\relax
    \KV@err{No value specified for \@tempa}%
  \else
    \@tempb\relax
  \fi}
%    \end{macrocode}
% \end{macro}
%
% \begin{macro}{\KV@err}
% \changes{v1.10}{1997/11/10}
%         {Use \cs{PackageError}}
% \changes{v1.12}{1998/11/18}
%         {Option system added}
% Error messages.
%    \begin{macrocode}
%<plain>\def\KV@err#1{\errmessage{key-val: #1}}
%<*!plain>
\DeclareOption{unknownkeysallowed}{%
  \def\KV@errx#1{\PackageInfo{keyval}{#1}}}
\DeclareOption{unknownkeyserror}{%
  \def\KV@errx#1{\PackageError{keyval}{#1}\@ehc}}
\ExecuteOptions{unknownkeyserror}
\let\KV@err\KV@errx
\ProcessOptions
%</!plain>
%    \end{macrocode}
% \end{macro}
%
% \begin{macro}{\KV@@sp@def}
% \changes{v1.10}{1997/11/10}
%         {Reorganise to not require doubled hash tokens, and to not
%          lose initial brace groups.}
% \begin{macro}{\KV@@sp@b}
% \begin{macro}{\KV@@sp@c}
% \begin{macro}{\KV@@sp@d}
% |\KV@@sp@def|\meta{cmd}\meta{token list} is like |\def|, except that
% a space token at the beginning or end of \meta{token list} is
% removed before making the assignment. \meta{token list} may not
% contain the token |\@nil|, unless it is within a brace group.
% The names of these commands were changed at version~1.05 to ensure
% that they do not clash with `internal' macros in a key family `sp'.
%
% Since v1.10, |#| may appear in the second argument without it
% needing to be doubled as |##|. Also earlier versions would
% drop any initial brace group, so |{abc}d| would incorrectly
% be treated as |abcd|. The current version only removes brace
% groups that surround the entire value, so |{abcd}| \emph{is} treated
% correctly as |abcd|. Prior to v1.14, two levels of bracing are removed, if
% you require the entire argument to be a single brace group, you had
% use |{{{abcd}}}|, from v1.14 exactly one brace group is removed, so to make
% the entire value be a brace group you need |{{abc}}|.
%
%    \begin{macrocode}
\def\@tempa#1{%
%    \end{macrocode}
%
%    \begin{macrocode}
\long\def\KV@@sp@def##1##2{%
  \futurelet\KV@tempa\KV@@sp@d##2\@nil\@nil#1\@nil\relax##1}%
%    \end{macrocode}
%
% Early release removed initial space by having an `extra' argument
% in |\KV@@sp@b| but that removed too many braces, so now make
% |\KV@@sp@b| explicitly remove a single space token. That unfortunately
% means we need the new |\KV@@sp@d| command to add a space token if
% one was not there before.
%    \begin{macrocode}
\def\KV@@sp@d{%
  \ifx\KV@tempa\@sptoken
    \expandafter\KV@@sp@b
  \else
    \expandafter\KV@@sp@b\expandafter#1%
  \fi}%
%    \end{macrocode}
%
% \changes{v1.14}{2014/04/25}
%         {Add \cs{@empty} to avoid dropping a brace pair: graphics/3446}
% \changes{v1.15}{2014/05/08}
%         {revert last change}
%    \begin{macrocode}
\long\def\KV@@sp@b#1##1 \@nil{\KV@@sp@c##1}%
%    \end{macrocode}
%
%    \begin{macrocode}
%<plain>\def\@sptoken{#1}%
%    \end{macrocode}
%
% Make the above definitions, inserting the space token where needed.
%    \begin{macrocode}
  }
\@tempa{ }
%    \end{macrocode}
%
% \changes{v1.14}{2014/04/25}
%         {Add \cs{expandafter} to remove the extra \cs{@empty} token: graphics/3446}
% \changes{v1.15}{2014/05/08}
%         {revert last change}
%    \begin{macrocode}
\long\def\KV@@sp@c#1\@nil#2\relax#3{\KV@toks@{#1}\edef#3{\the\KV@toks@}}
%    \end{macrocode}
% \end{macro}
% \end{macro}
% \end{macro}
% \end{macro}
%
% \begin{macro}{\KV@toks@}
% Macro register used above to prevent |#| doubling.
% Avoid uding one of the normal scratch registers, as this code
% is not in a local group.
%    \begin{macrocode}
\newtoks\KV@toks@
%    \end{macrocode}
% \end{macro}
%
% \begin{macro}{\define@key}
% Define the command associated to the key |#2| in the family |#1|.
% First looks for a default argument (the default value for the
% key)
%    \begin{macrocode}
\def\define@key#1#2{%
  \@ifnextchar[{\KV@def{#1}{#2}}{\long\@namedef{KV@#1@#2}####1}}
%    \end{macrocode}
% \end{macro}
%
% \begin{macro}{\KV@def}
% Make the definitions of the command, and the default value.
%    \begin{macrocode}
\def\KV@def#1#2[#3]{%
  \long\@namedef{KV@#1@#2@default\expandafter}\expandafter
    {\csname KV@#1@#2\endcsname{#3}}%
  \long\@namedef{KV@#1@#2}##1}
%    \end{macrocode}
% \end{macro}
%
%    \begin{macrocode}
%</package>
%    \end{macrocode}
% \Finale
%
\endinput
