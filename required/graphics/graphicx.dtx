%
% \iffalse
%
%% graphicx.dtx Copyright (C) 1994      David Carlisle Sebastian Rahtz
%%              Copyright (C) 1995-2020 David Carlisle, LaTeX3 Project
%%
%% This file is part of the Standard LaTeX `Graphics Bundle'.
%% It may be distributed under the terms of the LaTeX Project Public
%% License, as described in lppl.txt in the base LaTeX distribution.
%% Either version 1.3c or, at your option, any later version.
%%
%<*dtx>
          \ProvidesFile{graphicx.dtx}
%</dtx>
%<package>\NeedsTeXFormat{LaTeX2e}[1995/12/01]
%<package>\ProvidesPackage{graphicx}
%<driver> \ProvidesFile{graphicx.drv}
% \fi
%         \ProvidesFile{graphicx.dtx}
          [2020/09/09 v1.2b  Enhanced LaTeX Graphics (DPC,SPQR)]
%
% \iffalse
%<*driver>
\documentclass{ltxdoc}
\newenvironment{key}[2]{\expandafter\macro\expandafter{%
   \csname KV@#1@#2\endcsname}}{\endmacro}
%
\begin{document}
 \DocInput{graphicx.dtx}
\end{document}
%</driver>
% \fi
%
% \GetFileInfo{graphicx.dtx}
%
% \title{The \textsf{graphicx} package\thanks{This file
%        has version number \fileversion, last
%        revised \filedate.}}
% \author{D. P. Carlisle\and S. P. Q. Rahtz}
% \date{\filedate}
% \MaintainedByLaTeXTeam{graphics}
% \maketitle
%
%
% \changes{v0.3}{1994/03/01}
%     {First DPC version (after prototype by SPQR).}
% \changes{v0.4b}{1994/05/30}
%     {Rename from egraphics. 8+3 name for CDROM}
% \changes{v1.0}{1996/05/29}
%     {Version 1 at last}
%
% \def\star{{\ttfamily*}}
% \makeatletter
% \def\Describe@Macro#1{\endgroup
%              \setbox0=\lastbox\llap{\PrintDescribeMacro{#1}}}%
% \makeatother
% \marginparsep0pt
%
% \section{Introduction}
%
% This package provides an alternative interface to the \LaTeXe\
% graphics functions. The command names provided are the same as in the
% standard package, and they use the same internal functions. However
% the meaning of the optional arguments is different. Note \emph{only}
% the optional arguments have changed: any document which only uses the
% graphics commands with the mandatory arguments and/or the star-forms
% will work identically (with essentially identical implementation) with
% the two packages.
%
% \section{Key=Value Interface}
% When the decision to produce \LaTeXe\ was made, certain `guiding
% principles' were made (and published in the original announcement).
% One of these was that all new features would `conform to the
% conventions of version 2.09'. Specifically this meant that new
% commands would obey the same basic syntax rules for arguments as the
% existing commands.
%
% Standard \LaTeX\ optional arguments are \emph{positional}. If a
% command were to take three optional arguments, then there would be no
% way of specifying only the third, one would have to give all three,
% even if  the first two were repeats of the default values. Basically
% this means that `standard' optional arguments are not suitable if
% there is more than one option. Various existing packages (for
% \LaTeX~2.09) have recognised this, and used `named arguments' in
% various forms. Perhaps the two most noticeable are |psfig| and
% |pstricks|. With `named arguments' (sometimes called `attributes')
% each option is not tied to a particular position, but rather given a
% name (or key) and any options that must be set are set by explicitly
% associating this name with the desired value.
%
% The members of the \LaTeX3 project do appreciate the importance of
% this kind of syntax, but felt that rather than extending the syntax of
% \LaTeX\ in an uncoordinated way, it would be better to keep with
% `standard arguments' in \LaTeXe, which is intended as a `consolidation
% of existing \LaTeX\ variants'. The long term planning for an eventual
% \LaTeX3 release will then be able to consider the whole \LaTeX\
% user interface, and a suitable syntax for named arguments. It is
% important that such an interface design is not hampered by having to
% retain compatibility with earlier attempts at a named argument
% syntax. For this reason this |graphicx| package, which uses the named
% argument mechanism from the |keyval| package should be considered `non
% standard' although it is supported by the same mechanism, and same
% authors as the `standard' |graphics| package.
%
% \section{The User Interface}
%
% \DescribeMacro
%     \includegraphics\star\oarg{key-val list}\marg{file}\\
% \DescribeMacro
%     \includegraphics\star\oarg{llx,lly}\oarg{urx,ury}\marg{file}\\
% Include a graphics file.
%
% The star form is just for
% compatibility with the standard interface, and essentially just adds
% |clip| to the keys specified. Similarly the second, two-optional
% argument form is for increased compatibility with the standard
% package. The two optional argument form is not needed in the |keyval|
% interface.
%
% Various `keys' or named arguments are supported.
% \begin{description}
% \item[bb] Set the bounding box. The argument should be four
% dimensions, separated by spaces.
% \item[bbllx,bblly,bburx,bbury] Set the bounding box. Mainly for
% compatibility with older packages. |bbllx=a,bblly=b,bburx=c,bbury=d|
% is equivalent to |bb = a b c d|.
% \item[natwidth,natheight] Again an alternative to |bb|.
% |natheight=h,natwidth=w| is equivalent to |bb = 0 0 h w|.
% \item[viewport] Modify the bounding box specified in the file.
% The four values specify a bounding box \emph{relative} to the
% |llx|,|lly| coordinate of the original box.
% \item[trim] Modify the bounding box specified in the file.
% The four values specify the amounts to remove from
% the left, bottom, right and top of the original box.
% \item[hiresbb] Boolean valued key. Defaults to |true|.
% Causes \TeX\ to look for |%%HiResBoundingBox| comments rather than
% the standard |%%BoundingBox|. May be set to |false| to override
% a default setting of true specified by the |hiresbb| package option.
% \item[angle] Rotation angle.
% \item[origin] Rotation origin (see |\rotatebox|, below).
% \item[width] Required width, a dimension (default units |bp|). The
% graphic will be scaled to make the width the specified dimension.
% \item[height] Required height. a dimension (default units |bp|).
% \item[totalheight] Required totalheight (i.e., height $+$ depth). a
% dimension (default units |bp|). Most useful after a rotation (when the
% height might be zero).
% \item[keepaspectratio] Boolean valued key (like |clip|). If it is set
%  to true, modify the meaning of the |width| and |height| (and
% |totalheight|) keys such that if both are specified then rather than
% distort the figure the figure is scaled such that neither dimension
% \emph{exceeds} the stated dimensions.
% \item[scale] Scale factor.
% \item[clip] Either `true' or `false' (or no value, which is equivalent
% to `true'). Clip the graphic to the bounding box (or viewport if one
% is specified).
% \item[draft] a boolean valued key, like `clip'. locally switches to
% draft mode, ie.\ do not include the graphic, but leave the
% correct space, and print the filename.
% \item[type] Specify the file type. (Normally determined from the file
% extension.)
% \item[ext] Specify the file extension.
%        \emph{Only} for use with |type|.
% \item[read] Specify the `read file' which is used for determining the
% size of the graphic. \emph{Only} for use with |type|.
% \item[command] Specify the file command.
%         \emph{Only} for use with |type|.
% \item[quiet] Turns off writing information about graphics to the |.log|.
% \item[page] The page of a multi-page PDF graphic to be used.
% \item[interpolate] Enables interpolation of bitmap images by viewers.
% \item[pagebox] Specifies which PDF box should be used for the natural image size, one of
% mediabox, cropbox, bleedbox, trimbox, artbox. The default is driver-specific.
% \end{description}
%
% The arguments are interpreted left to right. |clip|, |draft|, |bb|,,
% and |bbllx| etc.\ have the same effect wherever they appear. but the
% scaling and rotation keys interact.
%
% Any scaling that is specified \emph{before} rotation, is handled by
% the internal graphics inclusion function. Rotation, or any later
% scaling is handled by implicitly calling |\rotatebox| or |\scalebox|.
% So |[height=1in,angle=90]| scales the graphic to 1in, then rotates it,
% so it is one inch wide. |[angle=90,height=1in]|  first rotates, then
% scales the result so that it is 1in high. A driver that can scale
% included graphics, but not arbitrary text will not be able to support
% the second form, as it will require a call to |\scalebox|, but the
% first form should work as there the scaling is handled by
% |\Ginclude@graphics|.
%
% \DescribeMacro
%      \rotatebox\oarg{key-val list}\marg{angle}\marg{text}\\
% Rotate \emph{text}.
%
% The keys supported by |\rotatebox| are:
% \begin{description}
% \item[origin] Specify the centre of rotation. |origin=|\meta{label},
% where the labels are up to two of |lrctbB| (|B| denotes the
% baseline, as for PSTricks).
% \item[x,y] An alternative to
% |origin|. |x=|\meta{dimen}|,y=|\meta{dimen} The $x,y$ coordinate of
% the centre of rotation. As usual |\height| etc may be used.
% \item[units] Specify the units used in the main argument. eg
% |units=-360| would mean that the argument referred to degrees
% \emph{clockwise} instead of the default anti-clockwise rotation.
% \end{description}
%
% As an example |\rotatebox[origin=c]{180}{text}| will rotate ``text''
% around its centre, thus creating a final box of the same dimensions as
% the original box. This is to be contrasted to the default behaviour,
% which rotates around the reference point on the baseline, thus
% producing a box that is mainly \emph{below} the baseline.
%
% \StopEventually{}
%
%
% \section{Implementation}
%
%    \begin{macrocode}
%<*package>
%    \end{macrocode}
%
% One  new option is  handled by keyval. It suppresses the error
% normally generated if an unknow keyval key is used.
% (This helps porting between drivers that use extended interfaces.)
% \changes{v1.0e}{1999/01/10}{new unknownkeysallowed option}
%    \begin{macrocode}
\DeclareOption{unknownkeysallowed}
  {\PassOptionsToPackage\CurrentOption{keyval}}
%    \end{macrocode}
%
% All other options are handled by the \textsf{graphics} package.
%    \begin{macrocode}
\DeclareOption*{\PassOptionsToPackage\CurrentOption{graphics}}
%    \end{macrocode}
%
%    \begin{macrocode}
\ProcessOptions
%    \end{macrocode}
%
% This package requires these two building blocks.
%    \begin{macrocode}
\RequirePackage{keyval,graphics}
%    \end{macrocode}
%
%
% \subsection{Graphics Inclusion}
%
%
% First we declare the `bounding box' keys. These all use
% |\Gin@defaultbp| so that the \meta{value} can be given as a length in
% the usual \TeX\ units such as |cm| or as an integer, taken as |bp|.
%
% \begin{key}{Gin}{bb}
%    \begin{macrocode}
\define@key{Gin}{bb}
           {\Gin@bboxtrue\Gread@parse@bb#1 \\}
%    \end{macrocode}
% \end{key}
%
% \begin{key}{Gin}{bbllx}
% \begin{key}{Gin}{bblly}
% \begin{key}{Gin}{bburx}
% \begin{key}{Gin}{bbury}
% \changes{v0.3b}{1994/03/11}{use bbllx not llx}
% \changes{v0.3b}{1994/03/11}{add natheight and natwidth}
% \changes{v0.6a}{1995/04/11}{New \cs{Gin@defaultbb} code}
% \changes{v0.6b}{1995/06/26}{typos fixed for graphics/1685}
%    \begin{macrocode}
\define@key{Gin}{bbllx}
           {\Gin@bboxtrue\Gin@defaultbp\Gin@llx{#1}}
\define@key{Gin}{bblly}
           {\Gin@bboxtrue\Gin@defaultbp\Gin@lly{#1}}
\define@key{Gin}{bburx}
           {\Gin@bboxtrue\Gin@defaultbp\Gin@urx{#1}}
\define@key{Gin}{bbury}
           {\Gin@bboxtrue\Gin@defaultbp\Gin@ury{#1}}
%    \end{macrocode}
% \end{key}
% \end{key}
% \end{key}
% \end{key}
%
% \begin{key}{Gin}{hiresbb}
% \changes{v1.0b}{1996/10/29}
%      {hiresbb key added}
% If set to true (the default)  \TeX\ will look for bounding box
% comments of the form |%%HiResBoundingBox| (which typically have
% real values) instead of the standard |%%BoundingBox| (which should
% have integer values).
% It may be set to false to override a package option of hiresbb.
%    \begin{macrocode}
\define@key{Gin}{hiresbb}[true]{%
  \edef\Gread@BBox{%
    \@percentchar\@percentchar
    \csname if#1\endcsname HiRes\fi
    BoundingBox}}
%    \end{macrocode}
% \end{key}
%
% \begin{key}{Gin}{natheight}
% \begin{key}{Gin}{natheight}
% \changes{v0.6d}{1995/09/22}
%      {typos fixed in natheight and natwidth keys}
%    \begin{macrocode}
\let\KV@Gin@natwidth\KV@Gin@bburx
\let\KV@Gin@natheight\KV@Gin@bbury
%    \end{macrocode}
% \end{key}
% \end{key}
%
% \begin{key}{Gin}{viewport}
% \begin{key}{Gin}{trim}
% \changes{v0.6a}{1995/04/11}{viewport and trim keys added}
% \changes{v0.6c}{1995/09/07}{arithmetic in viewport key fixed}
% A `viewport' is a user-specified area of the graphic to be included.
% It should not be confused with the `Bounding Box' of a PS file.
% In fact, the origin for a viewport specification is the (llx,lly)
% lower left coordinate of the bounding box. If a viewport is
% specified, and clipping is turned on, clipping is based on the
% viewport, not on the boundingbox.
%
% Both `viewport' and `trim' were suggested (and originally, but
% differently, implemented) by Arthur Ogawa.
%    \begin{macrocode}
\define@key{Gin}{viewport}
           {\let\Gin@viewport@code\Gin@viewport\Gread@parse@vp#1 \\}
\define@key{Gin}{trim}
           {\let\Gin@viewport@code\Gin@trim\Gread@parse@vp#1 \\}
%    \end{macrocode}
% \end{key}
% \end{key}
%
% \begin{macro}{\Gread@parse@vp}
% Grabs four bounding box values like |\Gread@parse@bp| but saves them
% in alternative macros that are used in the viewport and trim cases to
% modify the bounding box read from the file.
%    \begin{macrocode}
\def\Gread@parse@vp#1 #2 #3 #4 #5\\{%
  \Gin@defaultbp\Gin@vllx{#1}%
  \Gin@defaultbp\Gin@vlly{#2}%
  \Gin@defaultbp\Gin@vurx{#3}%
  \Gin@defaultbp\Gin@vury{#4}}%
%    \end{macrocode}
% \end{macro}
%
%
% \begin{key}{Gin}{angle}
% Specify a rotation. This is just handled by wrapping the
% |\includegraphics| command in a call to the internal version of
% |\rotatebox|. Normally this is the `standard' version but if an
% |origin| key is used in |\includegraphics| then the \emph{keyval}
% version of origin is used, and the |origin| key is passed on.
%    \begin{macrocode}
\define@key{Gin}{angle}
           {\Gin@esetsize
            \@tempswatrue
            \edef\@tempa{\toks@{\noexpand\Gin@erotate{#1}{\the\toks@}}}%
            \@tempa}
%    \end{macrocode}
% \end{key}
%
% \begin{key}{Gin}{origin}
% Pass the origin key value on to |\rotatebox|. |\Gin@erotate| is
% initialised to |\Grot@box@std| later in the file, after the latter has
% been defined.
% \changes{v0.6e}{1995/09/28}
%     {key added}
%    \begin{macrocode}
\define@key{Gin}{origin}[c]{%
  \def\Gin@erotate{\Grot@box@kv[origin=#1]}}
%    \end{macrocode}
% \end{key}
%
% \begin{key}{Gin}{width}
% \begin{key}{Gin}{height}
% Save the required height and width. The actual scaling is done later.
%    \begin{macrocode}
\define@key{Gin}{width}{\def\Gin@ewidth{#1}}
\define@key{Gin}{height}{\def\Gin@eheight{#1}}
%    \end{macrocode}
% \end{key}
% \end{key}
%
% \begin{key}{Gin}{totalheight}
% \changes{v0.6a}{1995/04/11}{New totalheight key}
% The same as |height| key, but locally changes |\Gin@eresize| to
% |\totalheight| from its default value of |\height|.
%    \begin{macrocode}
\define@key{Gin}{totalheight}{%
  \def\Gin@eresize{\totalheight}\def\Gin@eheight{#1}}
%    \end{macrocode}
% \end{key}
%
% \begin{key}{Gin}{keepaspectratio}
% Boolean valued key (like |clip|). If it is set to true,
% modify the meaning of the |width| and |height| (and |totalheight|)
% keys such that if both are specified then rather than distort the
% figure the figure is scaled such that neither dimension \emph{exceeds}
% the stated dimensions.
% \changes{v0.6e}{1995/09/28}
%     {key added}
% \changes{v0.6h}{1995/12/06}
%     {Name changed to keepaspectratio}
%    \begin{macrocode}
\define@key{Gin}{keepaspectratio}[true]{%
  \lowercase{\Gin@boolkey{#1}}{iso}}
%    \end{macrocode}
% \end{key}
%
% \begin{key}{Gin}{scale}
% \changes{v0.6h}{1995/12/06}
%     {Use \cs{Gin@exclamation}}
% If the scaling is being handled externally, wrap |\includegraphics| in
% the internal form of |\scalebox|, otherwise locally define
% |\Gin@req@sizes| to calculate the required sizes based on
% scale factor.
%    \begin{macrocode}
\define@key{Gin}{scale}{%
  \if@tempswa
    \edef\@tempa{\toks@{\noexpand\Gscale@box{#1}[#1]{\the\toks@}}}%
    \@tempa
  \else
    \def\Gin@req@sizes{%
      \def\Gin@scalex{#1}\let\Gin@scaley\Gin@exclamation
      \Gin@req@height\Gin@scalex\Gin@nat@height
      \Gin@req@width\Gin@scalex\Gin@nat@width}%
  \fi
  \@tempswatrue}
%    \end{macrocode}
% \end{key}
%
% \begin{key}{Gin}{draft}
% Locally set the draft switch to true. This is used by the code in
% \textsf{graphics} package to suppress the file inclusion.
%    \begin{macrocode}
\define@key{Gin}{draft}[true]{%
  \lowercase{\Gin@boolkey{#1}}{draft}}
%    \end{macrocode}
% \end{key}
%
% \begin{key}{Gin}{clip}
% Locally set the clip switch to true. This is used by the code in
% \textsf{graphics} package to suppress the printing of anything outside
% the bounding box specified.
%    \begin{macrocode}
\define@key{Gin}{clip}[true]{%
  \lowercase{\Gin@boolkey{#1}}{clip}}
%    \end{macrocode}
% \end{key}
%
% \begin{key}{Gin}{type}
% If you use `type' you must use no extension in the main argument
% and you must use `ext'. You can also use `read' and `command'.
% \changes{v0.5a}{1994/11/29}
%     {remove \cs{Gin@sep}}
%    \begin{macrocode}
\define@key{Gin}{type}{%
  \def\Ginclude@graphics##1{%
    \begingroup
    \def\Gin@base{##1}%
    \edef\@tempa{{#1}{\Gin@eread}{\Gin@ecom{##1\Gin@eext}}}%
    \expandafter\Gin@setfile\@tempa
    \endgroup}}
%    \end{macrocode}
% \end{key}
%
% \begin{key}{Gin}{ext}
% \changes{v1.0d}{1997/06/09}{initialise ext to empty}
% Specify an extension, for use with the `type' key.
%    \begin{macrocode}
\define@key{Gin}{ext}{\def\Gin@eext{#1}}
\let\Gin@eext\@empty
%    \end{macrocode}
% \end{key}
%
% \begin{key}{Gin}{read}
% Specify a read file, for use with the `type' key.
% You may want to globally set this to |*| using |\setkeys|.
% |*| means read the graphic file for size info, as in
% |\DeclareGraphicsRule|.
% \changes{v0.5a}{1994/11/29}
%     {Add default * possibility.}
%    \begin{macrocode}
\define@key{Gin}{read}{%
 \def\Gin@eread{#1}%
 \def\@tempa{*}\ifx\@tempa\Gin@eread\def\Gin@eread{\Gin@eext}\fi}
\let\Gin@eread\@empty
%    \end{macrocode}
% \end{key}
%
% \begin{key}{Gin}{command}
% Specify a command, for use with the `type' key.
%    \begin{macrocode}
\define@key{Gin}{command}{\def\Gin@ecom##1{#1}}
\let\Gin@ecom\@firstofone
%    \end{macrocode}
% \end{key}
%
% \begin{key}{Gin}{decodearray}
% \changes{v1.1b}{2017/06/25}{New decodearray key}
%   For manipulating bitmap images.
%    \begin{macrocode}
\define@key{Gin}{decodearray}{%
  \def\Gin@decode{#1}%
}
%    \end{macrocode}
% \end{key}
%
% \begin{key}{Gin}{quiet}
% \changes{v1.1a}{2017/06/01}{New quiet key}
%   Skip writing to the log.
%    \begin{macrocode}
\define@key{Gin}{quiet}{%
  \let\Gin@log\@gobble
}
%    \end{macrocode}
% \end{key}
%
% \begin{key}{Gin}{page}
% \changes{v1.1a}{2017/06/01}{New page key}
%   Page of a multi-page (PDF) graphic.
%    \begin{macrocode}
\define@key{Gin}{page}{%
  \def\Gin@page{#1}%
  \ifx\Gin@page\@empty
  \else
    \edef\Gin@page{\number\Gin@page}%
  \fi
}
%    \end{macrocode}
% \end{key}
%
% \begin{key}{Gin}{interpolate}
% \changes{v1.1a}{2017/06/01}{New interpolate key}
%  Enable/disable interpolation of bitmap images by the viewer.
%    \begin{macrocode}
\define@key{Gin}{interpolate}[true]{%
  \lowercase{\Gin@boolkey{#1}}{interpolate}}
%    \end{macrocode}
% \end{key}
%
% \begin{key}{Gin}{pagebox}
% \changes{v1.1a}{2017/06/01}{New pagebox key}
%  Specify which PDF box to use for the natural image size in PDF inclusions.
%    \begin{macrocode}
\define@key{Gin}{pagebox}{%
        \expandafter\let\expandafter\Gin@pagebox
            \csname Gin@pagebox@#1\endcsname
        \ifx\Gin@pagebox\relax
          \let\Gin@pagebox\Gin@pagebox@cropbox
          \@warning{%
            Unknown value `#1' for `pagebox'.\MessageBreak
            Supported values:\MessageBreak
            mediabox, cropbox, bleedbox, trimbox, artbox%
          }%
        \fi
      }
      \def\Gin@pagebox@mediabox{mediabox}%
      \def\Gin@pagebox@cropbox{cropbox}%
      \def\Gin@pagebox@bleedbox{bleedbox}%
      \def\Gin@pagebox@trimbox{trimbox}%
      \def\Gin@pagebox@artbox{artbox}%
%    \end{macrocode}
% \end{key}
%
% \begin{macro}{\Gin@boolkey}
% Helper function for defining boolean valued functions. The order of
% arguments allows |\lowercase| to only act on the user-supplied
% argument.
% \changes{v0.4a}{1994/04/14}{Make `empty'!=true so clip!= works}
%    \begin{macrocode}
\def\Gin@boolkey#1#2{%
  \csname Gin@#2\ifx\relax#1\relax true\else#1\fi\endcsname}
%    \end{macrocode}
% \end{macro}
%
% \begin{macro}{\Gin@esetsize}
% \changes{v0.6h}{1995/12/06}
%     {Use \cs{Gin@exclamation}}
% Arrange for the final size to be set, either by wrapping the include
% graphics call in |\scalebox|, or by redefining |\Gin@req@sizes|
% appropriately.
%    \begin{macrocode}
\def\Gin@eresize{\height}
\def\Gin@esetsize{%
  \let\@tempa\Gin@exclamation
  \if@tempswa
%    \end{macrocode}
%  External. Wrap the |\includegraphics| command in a call to the
%  internal form of |\scalebox| to handle the rotation.
%    \begin{macrocode}
    \edef\@tempa{\toks@{\noexpand
             \Gscale@@box\noexpand\Gin@eresize
              {\Gin@ewidth}{\Gin@eheight}{\the\toks@}}}%
    \@tempa
  \else
%    \end{macrocode}
% Internal. Handle scaling with the |\includegraphics| command directly
% rather than calling |\scalebox|.
%    \begin{macrocode}
    \ifx\Gin@ewidth\@tempa
      \ifx\Gin@eheight\@tempa
%    \end{macrocode}
% No resizing.
%    \begin{macrocode}
      \else
%    \end{macrocode}
% Just height specified.
% \changes{v0.6e}{1995/09/28}
%     {Use \cs{setlength} to support calc package}
% \changes{v0.6h}{1995/12/06}
%     {Use \cs{Gin@exclamation}}
%    \begin{macrocode}
         \let\Gin@@eheight\Gin@eheight
         \def\Gin@req@sizes{%
           \Gscale@div\Gin@scaley\Gin@@eheight\Gin@nat@height
           \let\Gin@scalex\Gin@exclamation
           \setlength\Gin@req@height\Gin@@eheight
           \Gin@req@width\Gin@scaley\Gin@nat@width}%
      \fi
    \else
      \ifx\Gin@eheight\@tempa
%    \end{macrocode}
% Just width specified.
% \changes{v0.6e}{1995/09/28}
%     {Use \cs{setlength} to support calc package}
% \changes{v0.6h}{1995/12/06}
%     {Use \cs{Gin@exclamation}}
%    \begin{macrocode}
         \let\Gin@@ewidth\Gin@ewidth
         \def\Gin@req@sizes{%
           \Gscale@div\Gin@scalex\Gin@@ewidth\Gin@nat@width
           \let\Gin@scaley\Gin@exclamation
           \setlength\Gin@req@width\Gin@@ewidth
           \Gin@req@height\Gin@scalex\Gin@nat@height}%
      \else
%    \end{macrocode}
% Both height and width specified.
%    \begin{macrocode}
         \let\Gin@@ewidth\Gin@ewidth
         \let\Gin@@eheight\Gin@eheight
%    \end{macrocode}
% \changes{v0.6g}{1995/11/10}
%     {Use \cs{ifGin@iso} to support isoscale key}
% At this point can locally redefine |\Gin@nosize|. Instead
% of generating an error, just set the `natural' size to be the
% `requested size'. Previous versions of this package did not
% allow the use of |height| and |width| unless the natural size was
% known as otherwise \LaTeX\ can not calculate the scale factor.
% However many drivers (especially for bitmap formats) can work this out
% themselves, so as long as both |height| and |width| are given, so
% \LaTeX\ knows the size to leave, accept this. This assumes the code in
% the driver file will use the `required height' information, not the
% scale factors, which will be set to 1!.
% \changes{v1.0a}{1996/08/05}
%     {Missing percent added for graphics/2244}
%    \begin{macrocode}
         \def\Gin@nosize##1{%
           \KV@Gin@natwidth\Gin@@ewidth
           \KV@Gin@natheight\Gin@@eheight}%
%    \end{macrocode}
%
%    \begin{macrocode}
         \def\Gin@req@sizes{%
           \Gscale@div\Gin@scalex\Gin@@ewidth\Gin@nat@width
           \Gscale@div\Gin@scaley\Gin@@eheight\Gin@nat@height
%    \end{macrocode}
% \changes{v0.6f}{1995/09/29}
%     {Use \cs{ifGin@iso} to support isoscale key}
% Donald Arseneau requested this feature. If both |height|
% and |width| are chosen, choose the smaller scale factor rather than
% distort the graphic. This mode is turned on with the
% \textsf{keepaspectratio} key.
%    \begin{macrocode}
           \ifGin@iso
             \ifdim\Gin@scaley\p@>\Gin@scalex\p@
               \let\Gin@scaley\Gin@scalex
             \else
               \let\Gin@scalex\Gin@scaley
             \fi
           \fi
           \Gin@req@width\Gin@scalex\Gin@nat@width
           \Gin@req@height\Gin@scaley\Gin@nat@height}%
       \fi
     \fi
  \fi
  \let\Gin@ewidth\Gin@exclamation
  \let\Gin@eheight\Gin@ewidth}
%    \end{macrocode}
% \end{macro}
%
% \begin{macro}{\Gin@req@height}
% \begin{macro}{\Gin@req@width}
% The required final size.
%    \begin{macrocode}
\newdimen\Gin@req@height
\newdimen\Gin@req@width
%    \end{macrocode}
% \end{macro}
% \end{macro}
%
% \begin{macro}{\Gin@outer@scalex}
% \begin{macro}{\Gin@outer@scaley}
% Scale factors to pass to |\scalebox|.
%    \begin{macrocode}
\let\Gin@outer@scalex\relax
\let\Gin@outer@scaley\relax
%    \end{macrocode}
% \end{macro}
% \end{macro}
%
% \begin{macro}{\Gin@angle}
% Rotation angle.
%    \begin{macrocode}
\let\Gin@angle\relax
%    \end{macrocode}
% \end{macro}
%
% \begin{macro}{\Gin@ewidth}
% \changes{v0.6h}{1995/12/06}
%     {Use \cs{Gin@exclamation}}
% \begin{macro}{\Gin@eheight}
% Final size, initialised for no scaling.
%    \begin{macrocode}
\let\Gin@ewidth\Gin@exclamation
\let\Gin@eheight\Gin@ewidth
%    \end{macrocode}
% \end{macro}
% \end{macro}
%
% \begin{macro}{\Gin@scalex}
% \begin{macro}{\Gin@scaley}
% \changes{v0.6h}{1995/12/06}
%     {Use \cs{Gin@exclamation}}
% Scale factors. Initialised for no scaling.
%    \begin{macrocode}
\def\Gin@scalex{1}
\let\Gin@scaley\Gin@exclamation
%    \end{macrocode}
% \end{macro}
% \end{macro}
%
% \begin{macro}{\Gin@i}
% Use the same top level |\includegraphics| command as the standard
% interface. This will set the clipping switch, and then call |\Gin@i|.
% \changes{v0.3c}{1994/03/15}{Dont locally initialise width/height}
% \changes{v0.3c}{1994/03/15}{Always call \cmd{\Gin@ii}}
% \changes{v0.4c}{1994/09/12}{Use `standard' version if two opt.\ args.}
%    \begin{macrocode}
\def\Gin@i{%
 \def\Gin@req@sizes{%
   \Gin@req@height\Gin@nat@height
   \Gin@req@width\Gin@nat@width}%
 \@ifnextchar[\Gin@ii{\Gin@ii[]}}
%    \end{macrocode}
% \end{macro}
%
%
% \begin{macro}{\Gin@ii}
% Look for a second optional argument.
% If one optional argument is present, call |\setkeys| to process it,
%    \begin{macrocode}
\def\Gin@ii[#1]#2{%
    \def\@tempa{[}\def\@tempb{#2}%
    \ifx\@tempa\@tempb
      \def\@tempa{\Gin@iii[#1][}%
      \expandafter\@tempa
    \else
     \begingroup
       \@tempswafalse
       \toks@{\Ginclude@graphics{#2}}%
       \setkeys{Gin}{#1}%
       \Gin@esetsize
       \the\toks@
     \endgroup
     \fi}
%    \end{macrocode}
% \end{macro}
%
% \changes{v0.4b}{1994/05/06}
%     {Remove incorrect initialisation of \cs{Gin@scalex}}
%
% \section{Rotation}
%
% \begin{macro}{\rotatebox}
% Look for an optional argument.
% \changes{v0.3b}{1994/03/11}{Remove star form}
% \changes{v1.0g}{2014/04/25}{\cs{leavevmode} added graphics/1521}
% \changes{v1.2a}{2019/11/30}{Add \cs{protected} gh/208}
%    \begin{macrocode}
\protected\def\rotatebox{%
  \leavevmode
  \@ifnextchar[\Grot@box@kv\Grot@box@std}
%    \end{macrocode}
% \end{macro}
%
% \begin{macro}{\Grot@box@std}
% If no KV argument, just repeat the standard definition.
% \changes{v1.0g}{2014/04/25}{Made long  graphics/4296}
%    \begin{macrocode}
\long\def\Grot@box@std#1#2{%
  \Grot@setangle{#1}%
  \setbox\z@\hbox{{#2}}%
  \Grot@x\z@
  \Grot@y\z@
  \Grot@box}
%    \end{macrocode}
% \end{macro}
%
% \begin{macro}{\Grot@box@kv}
% \changes{v1.0g}{2014/04/25}{Made long  graphics/4296}
%    \begin{macrocode}
\long\def\Grot@box@kv[#1]#2#3{%
  \@begin@tempboxa\hbox{#3}%
    \Grot@x\width \divide\Grot@x\tw@
    \Grot@y\height \advance\Grot@y-\depth \divide\Grot@y\tw@
    \setkeys{Grot}{#1}%
    \setbox\z@\box\@tempboxa
    \Grot@setangle{#2}%
    \Grot@box
  \@end@tempboxa}
%    \end{macrocode}
% \end{macro}
%
%
% There are two ways of specifying the centre of rotation.
%
% \begin{key}{Grot}{origin}
% |origin=|\meta{label}, where the labels are up to two of |lrctbB|
% (|B| denotes the baseline, as for PSTricks).
%    \begin{macrocode}
\define@key{Grot}{origin}[c]{%
 \@tfor\@tempa:=#1\do{%
    \if l\@tempa \Grot@x\z@\else
    \if r\@tempa \Grot@x\width\else
    \if t\@tempa \Grot@y\height\else
    \if b\@tempa \Grot@y-\depth\else
    \if B\@tempa \Grot@y\z@\fi\fi\fi\fi\fi}}
%    \end{macrocode}
% \end{key}
%
% \begin{key}{Grot}{x}
% \begin{key}{Grot}{y}
% |x=|\meta{dimen}|,y=|\meta{dimen} The $x,y$ coordinate of the centre
% of rotation. As usual |\height| etc may be used.
%    \begin{macrocode}
\define@key{Grot}{x}{\setlength\Grot@x{#1}}
\define@key{Grot}{y}{\setlength\Grot@y{#1}}
%    \end{macrocode}
% \end{key}
% \end{key}
%
% \begin{key}{Grot}{units}
% `units' specifies the number or units in one anti-clockwise circle.
%  So the default is $360$. $-360$ gives clockwise rotation, $6.283185$
%  gives radians etc.
%    \begin{macrocode}
\define@key{Grot}{units}{%
  \def\Grot@setangle##1{%
  \dimen@##1\p@
  \dimen@ii#1\p@
  \divide\dimen@ii360\relax
  \divide\dimen@\dimen@ii
  \edef\Grot@angle{\number\dimen@}}}
%    \end{macrocode}
% \end{key}
%
% \begin{macro}{\Gin@erotate}
% Initialise the rotation command to use in |\includegraphics|.
% \changes{v0.6e}{1995/09/28}{macro added}
%    \begin{macrocode}
\let\Gin@erotate\Grot@box@std
%    \end{macrocode}
% \end{macro}
%
%    \begin{macrocode}
%</package>
%    \end{macrocode}
%
% \Finale
%
