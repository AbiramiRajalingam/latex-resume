%
% \iffalse
%
%% graphics.dtx Copyright (C) 1994      David Carlisle Sebastian Rahtz
%%              Copyright (C) 1995-2020 David Carlisle, LaTeX3 Project
%%
%% This file is part of the Standard LaTeX `Graphics Bundle'.
%% It may be distributed under the terms of the LaTeX Project Public
%% License, as described in lppl.txt in the base LaTeX distribution.
%% Either version 1.3c or, at your option, any later version.
%%
%% This file has the LPPL maintenance status "maintained".
%%
%<*dtx>
          \ProvidesFile{graphics.dtx}
%</dtx>
%<+package>\NeedsTeXFormat{LaTeX2e}[1995/12/01]
%<+package>\providecommand\DeclareRelease[3]{}
%<+package>\providecommand\DeclareCurrentRelease[2]{}
%<+package>
%<+package>\DeclareRelease{}{2017-06-25}{graphics-2017-06-25.sty}
%<+package>\DeclareCurrentRelease{}{2019-10-01}
%<+package>
%<+package>\ProvidesPackage{graphics}
%<driver> \ProvidesFile{graphics.drv}
% \fi
%         \ProvidesFile{graphics.dtx}
          [2020/08/30 v1.4c  Standard LaTeX Graphics (DPC,SPQR)]
%
% \iffalse
%<*driver>
\documentclass{ltxdoc}
\newenvironment{option}[1]{\expandafter\macro\expandafter{%
   \csname ds@#1\endcsname}}{\endmacro}
\begin{document}
 \DocInput{graphics.dtx}
\end{document}
%</driver>
% \fi
%
% \GetFileInfo{graphics.dtx}
%
% \title{The \textsf{graphics} package\thanks{This file
%        has version number \fileversion, last
%        revised \filedate.}}
% \author{D. P. Carlisle\and S. P. Q. Rahtz}
% \date{\filedate}
% \MaintainedByLaTeXTeam{graphics}
% \maketitle
%
%
% \changes{v0.3a}{1994/02/24}
%     {First DPC version (after prototype by SPQR).}
% \changes{v0.4e}{1994/05/30}
%     {Rename egraphics to graphicx}
% \changes{v1.0}{1996/05/29}
%     {Version 1 at last}
%
%
% \def\star{{\ttfamily*}}
% \makeatletter
% \def\Describe@Macro#1{\endgroup
%              \setbox0=\lastbox\llap{\PrintDescribeMacro{#1}}}%
% \makeatother
% \marginparsep0pt
%
% \section{Introduction}
%
% This package implements various `graphics' functions. The main
% features are a) inclusion of `graphics' files. b) Rotation of sections
% of the page, c) Scaling of sections of the page.
%
% The design is split into three `levels'.
% \begin{itemize}
% \item The user interface. This is the collection of commands designed
% to appear in a document text. Actually two separate user interface
% have been implemented. The `standard' interface, described here, and a
% more powerful, and more `user-friendly' interface provided by the
% |graphicx| package.
% \item The core functions. These functions, which are also implemented
% in this file do all the `main work'. The `user-interface functions
% just collect together the information from any optional-arguments or
% star-forms, and then call one of these functions.
% \item The driver files. It is not possible to achieve the
% functionality of this package just using \TeX. The |dvi| driver used
% must be given additional instructions. (Using the |\special| command
% of \TeX.) Unfortunately, the capabilities of various drivers differ,
% and the syntax required to pass instructions to the drivers is also
% not standardised. So the `core functions' never access |\special|
% directly, but rather call a series of commands that must be defined in
% a special file customised for each driver. The accompanying file,
% |drivers.dtx| has suitable files for a range of popular drivers.
% \end{itemize}
%
% \section{Package Options}
% Most of the options, such as |dvips|, |textures| etc., specify the
% driver that is to be used to print the document. You may wish to set
% up a configuration file so that this option always takes effect, even
% if not specified in the document. To do this, produce a file
% |graphics.cfg| containing the line:\\
% |\ExecuteOptions{dvips}|\\
% (or whichever other driver you wish.)
%
% Apart from the driver options there are a few other options to control
% the behaviour of the package.
% \begin{description}
% \item[draft]
% Do not include graphics files, but instead print a box of the size
% the graphic would take up, and the file name. This greatly speeds up
% previewing on most systems.
% \item[final]
% Turns off the |draft| option.
% \item[debugshow]
% Show a lot of tracing information on the terminal. If you are not me
% you probably do not want to use this option.
% \item[hiderotate]
% Do not show rotated text. Sometimes useful if your previewer can not
% rotate text.
% \item[hidescale]
% Do not show scaled text.
% \item[hiresbb]
% Look for Bounding Box lines of the form |%%HiResBoundingBox| instead
% of the standard |%%BoundingBox|. These are used by some applications
% to get round the restriction that BoundingBox comments should only
% have integer values.
% \item[setpagesize, nosetpagesize]
% The |setpagesize| option requests that the driver option sets the page size.
% (Whichever option is used, the page size is not set by this package if |\mag|
% has been changed from its default value.)
% \item[demo] Instead of including a graphics file, make
%   |\includegraphics| insert a black rectangle of size 150\,pt by
%   100\,pt unless either dimension was already specified by another
%   option.
% \end{description}
%
% \section{Standard Interface}
%
% \subsection{Graphics Inclusion}
%
% \DescribeMacro
%     \includegraphics\star\oarg{llx,lly}\oarg{urx,ury}\marg{file}\\
% Include a graphics file.
%
% If \star\ is present, then the graphic is `clipped' to the size
% specified. If \star\ is omitted, then any part  of the graphic that is
% outside the specified `bounding box' will  over-print the surrounding
% text.
%
% If the optional arguments are omitted, then the size of the graphic
% will be determined by reading an external file as described below.
% If \oarg{urx,ury} is present, then it should specify the coordinates
% of the top right corner of the image, as a pair of \TeX\ dimensions.
% If the units are omitted they default to |bp|. So |[1in,1in]| and
% |[72,72]| are equivalent. If only one optional argument appears, the
% lower left corner of the image is assumed to be at |[0,0]|. Otherwise
% \oarg{llx,lly} may be used to specify the coordinates of this point.
%
% \DescribeMacro
%    \graphicspath\marg{dir-list}\\
% This optional declaration may be used to specify a list of directories
% in which to search for graphics files. The format is as for the
% \LaTeXe\ primitive |\input@path|, a list of directories, each in a
% |{}| group (even if there is only one in the list). For example:
%  |\graphicspath{{eps/}{tiff/}}|
% would cause the system to look in the subdirectories |eps| and |tiff|
% of the current directory. The default setting of this path is
% |\input@path| that is: graphics files will be found wherever \TeX\
% files are found.
%
% \DescribeMacro
%   \DeclareGraphicsExtensions\marg{ext-list}\\
% This specifies the behaviour of the system when the filename argument to
% |\includegraphics| does not have an extension specified.
% Here \marg{ext-list} should be a comma-separated list of file
% extensions, each with a leading period (\texttt{.}).
% A file name is produced by appending \emph{sep} and one extension.
% If a file is found, the system acts as if that extension had been
% specified. If not, the next extension in \emph{ext-list} is tried.
% \changes{v1.0m}{2001/06/07}{Extended documentation (CAR) graphics/3228}
%
% Each use of |\DeclareGraphicsExtensions| overwrites all previous
% definitions. It is not possible to add an extension to an
% existing list.
%
% Early versions of this package defined a default argument for this
% command. This has been removed.
%
% \DescribeMacro
% \DeclareGraphicsRule
%                \marg{ext}\marg{type}\marg{read-file}\marg{command}\\
% Any number of these declarations can be made. They determine how the
% system behaves when a file with extension \emph{ext} is specified.
% (The extension may be specified explicitly or, if the argument to
% |\includegraphics| does not have an extension, it may be a default
% extension from the \emph{ext-list} specified with
% |\DeclareGraphicsExtensions|.)
%
% \emph{ext} is the \emph{extension} of the file. Any file with this
% extension will be processed by this graphics rule. Normally a file
% with an extension for which no rule has been declared will generate an
% error, however you may use |*| as the extension to define a
% \emph{default rule}. For instance the |dvips| driver file declares all
% files to be of type |eps| unless a more specific rule is declared.
%
% Since Version v0.6, extensions should be specified including the |.|
% that is, |.eps| not |eps|.
%
% \emph{type} is the `type' of file involved. All files of the same type
% will be input with the same internal command (which must be defined in
% a `driver file'). For example files with extensions |ps|, |eps|,
% |ps.gz| may all be classed as type |eps|.
%
% \emph{read-file} determines the extension of the file that should be
% read to determine size information. It may be the same as \emph{ext}
% but it may be different, for example |.ps.gz| files are not readable
% easily by \TeX, so you may want to put the bounding box information in
% a separate file with extension |.ps.bb|. If \emph{read-file} is empty,
% |{}|, then the system will not try to locate an external file for size
% info, and the size must be specified in the arguments of
% |\includegraphics|. As a special case |*| may be used to denote the
% same extension as the graphic file. This is mainly of use in
% conjunction with using |*| as the extension, as in that case the
% particular graphic extension is not known. For example
%\begin{verbatim}
% \DeclareGraphicsRule{*}{eps}{*}{}
%\end{verbatim}
% This would declare a default rule, such that all unknown extensions
% would be treated as EPS files, and the graphic file would be read for
% a BoundingBox comment.
%
% If the driver file specifies a procedure for
% reading size files for \emph{type}, that will be used, otherwise the
% procedure for reading |eps| files will be used. Thus the size of
% bitmap files may be specified in a file with a PostScript style
% |%%BoundingBox| line, if no other specific format is available.
%
% \emph{command} is usually empty, but if non empty it is used in place
% of the filename in the |\special|. Within this argument, |#1| may be
% used to denote the filename. Thus using the dvips driver, one may
% use\\
% |\DeclareGraphicsRule{.ps.gz}{eps}{.ps.bb}{`zcat #1}|\\
% the final argument causes dvips to use the |zcat| command to unzip the
% file before inserting it into the PostScript output.
%
% \subsection{Rotation}
%
% \DescribeMacro
%   \rotatebox\marg{angle}\marg{text}\\
% Rotate \emph{text} \emph{angle} degrees anti-clockwise. Normally
% the rotation is about the left-hand end of the baseline of
% \emph{text}.
%
% \subsection{Scaling}
%
% \DescribeMacro
%   \scalebox\marg{h-scale}\oarg{v-scale}\marg{text}\\
% Scale \emph{text} by the specified amounts. If \emph{v-scale} is
% omitted, the vertical scale factor is the same as the horizontal one.
%
% \DescribeMacro
%   \resizebox\star\marg{h-length}\marg{v-length}\marg{text}\\
% Scale \emph{text} so that the width is \emph{h-length}.
% If |!| is used as either length argument, the other argument is used
% to determine a scale factor that is used in both directions.
% Normally \emph{v-length} refers to the height of the box, but in the
% star form, it refers to the `height +  depth'.
% As normal for \LaTeXe\ box length arguments, |\height|,
% |\width|, |\totalheight| and |\depth| may be used to refer to the
% original size of the box.
%
% \section{The Key=Value Interface}
% As mentioned in the introduction, apart from the above `standard
% interface', there is an alternative syntax to the |\includegraphics|
% and |\rotatebox| commands that some people may prefer. It is provided
% by the accompanying |graphicx| package.
%
% \StopEventually{}
%
% \section{The Graphics Kernel Functions}
%
% \subsection{Graphics Inclusion}
%
% \DescribeMacro
%   {\Ginclude@graphics}\marg{file}\\
% Insert the contents of the file \emph{file} at the current point.
% |\Ginclude@graphics| may use the four macros |\Gin@llx|, |\Gin@lly|,
% |\Gin@urx|, |\Gin@ury| to determine the `bounding box' of the graphic.
% The result will be a \TeX\ box of width $\mathit{urx}-\mathit{llx}$
% and height $\mathit{ury}-\mathit{lly}$. If |\Gin@clip| is \meta{true}
% then part of the graphic that is outside this box should not be
% displayed. (Not all drivers can support this `clipping'.) Normally all
% these parameters are set by the `user interface level'.
%
%
% \DescribeMacro
%   {\Gread@eps}\marg{file}\\
% For each \emph{type} of graphics file supported, the driver file must
% define |\Ginclude@|\emph{type} and, optionally |\Gread@|\emph{type}.
% The read command is responsible for obtaining size information from
% the file specified in the |\DeclareGraphicsRule| command. However the
% kernel defines a function, |\Gread@eps|,  which can read PostScript
% files to find the
% |%%BoundingBox| comment. This function will be used for any type for
% which a specific function has not been declared. |\Gread@eps| accepts
% a generalised version of the bounding box comment. \TeX\ units may be
% used (but there must be no space before the unit). If the unit is
% omitted |bp| is assumed. So\\
% |%%BoundingBox 0 0 2in 3in|\\
% Would be accepted by this function, to produce a 2in wide, by 3in high
% graphic.
%
% \subsection{Rotation}
%
% \DescribeMacro
%   {\Grot@box}\\
% Rotate the contents of |\box0| through |\Grot@angle| degrees
% (anti-clockwise). The user-interface is responsible for setting the
% macro |\Grot@angle|, and putting the appropriate text in |\Grot@box|.
%
% \subsection{Scaling}
%
% \DescribeMacro
%   {\Gscale@box}\marg{xscale}\oarg{yscale}\marg{text}\\
% (The second argument is not optional.)
% Scale \emph{text} by the appropriate scale factors.
%
% \DescribeMacro
%  {\Gscale@box@dd}\marg{dima}\marg{dimb}\marg{text}\\
% Scale \emph{text} in both directions by a factor
% \emph{dima}/\emph{dimb}.
%
% \DescribeMacro
% {\Gscale@box@dddd}
%        \marg{dima}\marg{dimb}\marg{dimc}\marg{dimd}\marg{text}\\
% Scale \emph{text} in horizontally by a factor \emph{dima}/\emph{dimb},
% and vertically by a factor of  \emph{dimc}/\emph{dimd}.
%
% \DescribeMacro
% {\Gscale@div}\marg{cmd}\marg{dima}\marg{dimb}\\
% Define the macro \emph{cmd} to be the ratio of the lengths
% \emph{dima}/\emph{dimb}.
%
%
% \section{Interface to the Driver Files}
%
% \subsection{Graphics Inclusion}
%
% Each driver file must declare that its driver can include graphics of
% certain \emph{types}. It does this by declaring for each type a
% command of the form:\\
% |\Ginclude@|\emph{type}\\
% The Graphics kernel function will call this driver-defined function
% with the filename as argument, and certain additional information will
% be provided as follows.:
%
% \noindent\begin{tabular}{p{.4\textwidth}p{.5\textwidth}}
% |\Gin@llx|, |\Gin@lly|,\newline
% |\Gin@urx|, |\Gin@ury| &Macros storing the `bounding box'\\
% |\Gin@nat@width|\newline |\Gin@nat@height| &
%               Registers storing the natural size.\\
% |\Gin@req@width|\newline |\Gin@req@height| &
%               Registers storing the required size, after scaling.\\
% |\Gin@scalex|, |\Gin@scaley| & macros with the scale factors. A value
% of |!| means: Scale by the same amount as the other direction.\\
% |\ifGin@clip| & |\newif| token, true if the graphic should be
%                 `clipped' to  the bounding box.
% \end{tabular}
%
% Optionally the driver may define a command of the form:\\
% |\Gread@|\emph{type}\\
% This is responsible for reading an external file to find the bounding
% box information. If such a command is not declared, but a read-file is
% specified the command |\Gread@eps|, which is defined in the Graphics
% Kernel will be used.
%
% \subsection{Literal Postscript}
% Drivers that are producing PostScript output may want to define
% the following macros. They each take one argument which should be
% passed to an appropriate special. They are not used directly by this
% package but allow other packages to use the standard configuration
% file and package options to customise to various drivers:\\
% |\Gin@PS@raw|, Literal PostScript special.\\
% |\Gin@PS@restored|, Literal PostScript special, the driver will
%                    surround this with a save-restore pair.\\
% |\Gin@PS@literal@header|, Postscript to be inserted in the header
%                           section of the PostScript file.\\
% |\Gin@PS@file@header|, external file to be inserted in the header
%                           section of the PostScript file.
%
%
% \subsection{Rotation}
%
% |\Grot@start|, |\Grot@end| These macros must be defined to insert the
% appropriate |\special| to rotate the text between them by
% |\Grot@angle| degrees. The kernel function will make sure that the
% correct \TeX\ spacing is produced, these functions only need insert
% the |\special|.
%
% \subsection{Scaling}
%
% |\Gscale@start|, |\Gscale@end|, as for rotation, but here scale the
% text by |\Gscale@x| and |\Gscale@y|.
%
%
% \section{Implementation}
%
%    \begin{macrocode}
%<*package>
%    \end{macrocode}
%
% \subsection{Initialisation}
%
% \begin{macro}{\Gin@codes}
% First we save the catcodes of some characters, and set them to
% fixed values whilst this file is being read.
% \changes{v0.5a}{1994/07/20}
%     {Save and restore catcodes}
%    \begin{macrocode}
\edef\Gin@codes{%
 \catcode`\noexpand\^^A\the\catcode`\^^A\relax
 \catcode`\noexpand\"\the\catcode`\"\relax
 \catcode`\noexpand\*\the\catcode`\*\relax
 \catcode`\noexpand\!\the\catcode`\!\relax
 \catcode`\noexpand\:\the\catcode`\:\relax}
%    \end{macrocode}
%
%    \begin{macrocode}
\catcode`\^^A=\catcode`\%
\@makeother\"%
\catcode`\*=11
\@makeother\!%
\@makeother\:%
%    \end{macrocode}
% \end{macro}
%
% We will need to have an implementation of the trigonometric
% functions for the rotation feature. May as well load it now.
%    \begin{macrocode}
\RequirePackage{trig}
%    \end{macrocode}
%
%  \begin{macro}{\Grot@start}
%  \begin{macro}{\Grot@end}
% Initialise the rotation primitives.
%    \begin{macrocode}
\providecommand\Grot@start{\@latex@error{Rotation not supported}\@ehc
            \global\let\Grot@start\relax}
\providecommand\Grot@end{}
%    \end{macrocode}
%  \end{macro}
%  \end{macro}
%
%  \begin{macro}{\Gscale@start}
%  \begin{macro}{\Gscale@end}
% Initialise the scaling primitives.
%    \begin{macrocode}
\providecommand\Gscale@start{\@latex@error{Scaling not supported}\@ehc
            \global\let\Gscale@start\relax}
\providecommand\Gscale@end{}
%    \end{macrocode}
%  \end{macro}
%  \end{macro}
%
% \begin{macro}{\Gread@BBox}
% \changes{v1.0b}{1996/10/29}
%     {Moved to initial section, for hiresbb option}
% |%%BoundingBox| as a macro for testing with |\ifx|.
% This may be redefined by the hiresbb option.
%    \begin{macrocode}
\edef\Gread@BBox{\@percentchar\@percentchar BoundingBox}
%    \end{macrocode}
% \end{macro}
%
% \subsection{Options}
%
% \begin{option}{draft}
% \begin{option}{final}
% \changes{v0.3i}{1994/03/23}
%     {notdraft option added}
% \changes{v0.4a}{1994/04/13}
%     {Rename option to nodraft}
% \changes{v0.4d}{1994/05/06}
%     {Rename option to final}
%    \begin{macrocode}
\DeclareOption{draft}{\Gin@drafttrue}
\DeclareOption{final}{\Gin@draftfalse}
%    \end{macrocode}
% \end{option}
% \end{option}
%
%  \begin{macro}{\ifGin@draft}
% True in draft mode.
%    \begin{macrocode}
\newif\ifGin@draft
%    \end{macrocode}
%  \end{macro}
%
% \begin{option}{hiresbb}
% \changes{v1.0b}{1996/10/29}
%     {hiresbb option added}
% If given this option the package will look for bounding box comments
% of the form |%%HiResBoundingBox| (which typically have real values)
% instead of the standard |%%BoundingBox| (which should have integer
% values).
%    \begin{macrocode}
\DeclareOption{hiresbb}{%
  \edef\Gread@BBox{\@percentchar\@percentchar HiResBoundingBox}}
%    \end{macrocode}
% \end{option}
%
% \begin{option}{demo}
%   \changes{v1.0o}{2006/02/20}{demo option added}
%   If given this option the package will disregard the actual
%   graphics file and insert a black box unless width or height are
%   already specified.
%    \begin{macrocode}
\DeclareOption{demo}{%
  \AtBeginDocument{%
    \def\Ginclude@graphics#1{%
      \rule{\@ifundefined{Gin@@ewidth}{150pt}{\Gin@@ewidth}}%
      {\@ifundefined{Gin@@eheight}{100pt}{\Gin@@eheight}}}}}
%    \end{macrocode}
% \end{option}
%
% \begin{option}{setpagesize}
% \begin{option}{nosetpagesize}
% The |setpagesize| option requests that the driver option sets the page size.
% (Whichever option is used, the page size is not set by this package if |\mag|
% has been changed from its default value.)
% \changes{v1.0s}{2016/06/02}
%     {(no)setpagesize options added}
%    \begin{macrocode}
\newif\ifGin@setpagesize\Gin@setpagesizetrue
\DeclareOption{setpagesize}{\Gin@setpagesizetrue}
\DeclareOption{nosetpagesize}{\Gin@setpagesizefalse}
%    \end{macrocode}
% \end{option}
% \end{option}
%
%  \begin{macro}{\Gin@driver}
% Driver in use.
%    \begin{macrocode}
\providecommand\Gin@driver{}
%    \end{macrocode}
%  \end{macro}
%
% \begin{option}{dvips}
% \changes{v0.3g}{1994/03/15}
%     {Use dvips def file not development version}
% \begin{option}{xdvi}
% Tomas Rockicki's  PostScript driver (unix, MSDOS, VMS\ldots).
% The |X11| previewer |xdvi| supports basically the same set of
% |\specials|.
%    \begin{macrocode}
\DeclareOption{dvips}{\def\Gin@driver{dvips.def}}
\DeclareOption{xdvi}{\ExecuteOptions{dvips}}
%    \end{macrocode}
% \end{option}
% \end{option}
%
% \begin{option}{dvipdf}
% \changes{v1.0c}{1996/10/31}
%     {dvipdf added}
% Sergey Lesenko's dvipdf driver.
%    \begin{macrocode}
\DeclareOption{dvipdf}{\def\Gin@driver{dvipdf.def}}
%    \end{macrocode}
% \end{option}
%
% \begin{option}{dvipdfm}
% \changes{v1.0k}{1999/01/13}
%     {dvipdfm added}
% \changes{v1.2a}{2017/06/01}
%     {make dvipdfm an alias for dvipdfmx}
% Mark Wick's dvipdfm driver (now merged with xdvipdfmx).
%    \begin{macrocode}
\DeclareOption{dvipdfm}{\def\Gin@driver{dvipdfmx.def}}
%    \end{macrocode}
% \end{option}
%
% \begin{option}{dvipdfmx}
% \changes{v1.0m}{2005/11/14}
%     {dvipdfmx added}
% The driver for the dvipdfmx project (also supports xdvipdfmx).
%    \begin{macrocode}
\DeclareOption{dvipdfmx}{\def\Gin@driver{dvipdfmx.def}}
%    \end{macrocode}
% \end{option}
%
% \begin{option}{xetex}
% \changes{v1.0m}{2005/11/14}
%     {xetex added}
%  Jonathan Kew's \TeX\ variant.
%    \begin{macrocode}
\DeclareOption{xetex}{\def\Gin@driver{xetex.def}}
%    \end{macrocode}
% \end{option}
%
% \begin{option}{pdftex}
% \changes{v1.0d}{1997/06/07}
%     {pdftex added}
%  Han The Thanh's \TeX\ variant.
%    \begin{macrocode}
\DeclareOption{pdftex}{\def\Gin@driver{pdftex.def}}
%    \end{macrocode}
% \end{option}
%
% \begin{option}{luatex}
% \changes{v1.0q}{2016/01/03}
%     {luatex added}
%  Lua\TeX\ \TeX\ variant.
%    \begin{macrocode}
\DeclareOption{luatex}{\def\Gin@driver{luatex.def}}
%    \end{macrocode}
% \end{option}
%
% \begin{option}{luatex}
% \changes{v1.0r}{2016/05/09}
%     {dvisvgm added}
%  dvisvgm driver.
%    \begin{macrocode}
\DeclareOption{dvisvgm}{\def\Gin@driver{dvisvgm.def}}
%    \end{macrocode}
% \end{option}
%
% \begin{option}{dvipsone}
% \begin{option}{dviwindo}
% \changes{v0.3c}{1994/03/04}
%     {dviwindo support.}
% \changes{v0.7c}{1995/09/07}
%     {Merge dviwindo option with dvipsone}
% The drivers for the Y\&Y \TeX\ system.
%    \begin{macrocode}
\DeclareOption{dvipsone}{\def\Gin@driver{dvipsone.def}}
\DeclareOption{dviwindo}{\ExecuteOptions{dvipsone}}
%    \end{macrocode}
% \end{option}
% \end{option}
%
% \begin{option}{emtex}
% \begin{option}{dviwin}
% Two freely available sets of drivers for MSDOS, OS/2 and Windows.
%    \begin{macrocode}
\DeclareOption{emtex}{\def\Gin@driver{emtex.def}}
\DeclareOption{dviwin}{\def\Gin@driver{dviwin.def}}
%    \end{macrocode}
% \end{option}
% \end{option}
%
% \begin{option}{oztex}
% \changes{v1.0f}{1997/09/09}
%     {Merge dvips and oztex options}
% Oz\TeX\ (Macintosh).
% Since release 3 of Oz\TeX, merge with dvips back end.
%    \begin{macrocode}
\DeclareOption{oztex}{\ExecuteOptions{dvips}}
%    \end{macrocode}
% \end{option}
%
% \begin{option}{textures}
% Textures (Macintosh).
%    \begin{macrocode}
\DeclareOption{textures}{\def\Gin@driver{textures.def}}
%    \end{macrocode}
% \end{option}
%
% \begin{option}{pctexps}
% \begin{option}{pctexwin}
% \begin{option}{pctexhp}
% \begin{option}{pctex32}
% \changes{v1.0b}{1996/10/29}
%     {pctex32 option added}
% PC\TeX\ (MSDOS/Windows) .
%    \begin{macrocode}
\DeclareOption{pctexps}{\def\Gin@driver{pctexps.def}}
\DeclareOption{pctexwin}{\def\Gin@driver{pctexwin.def}}
\DeclareOption{pctexhp}{\def\Gin@driver{pctexhp.def}}
\DeclareOption{pctex32}{\def\Gin@driver{pctex32.def}}
%    \end{macrocode}
% \end{option}
% \end{option}
% \end{option}
% \end{option}
%
% \begin{option}{truetex}
% \changes{v1.0b}{1996/10/29}
%     {truetex and tcidef options added}
% \begin{option}{tcidvi}
% Kinch TrueTeX, and its version with extended special support as
% shipped by Scientific Word.
%    \begin{macrocode}
\DeclareOption{truetex}{\def\Gin@driver{truetex.def}}
\DeclareOption{tcidvi}{\def\Gin@driver{tcidvi.def}}
%    \end{macrocode}
% \end{option}
% \end{option}
%
% \begin{option}{vtex}
% \changes{v1.0h}{1998/05/27}
%     {vtex option added}
% V\TeX\ driver.
%    \begin{macrocode}
\DeclareOption{vtex}{\def\Gin@driver{vtex.def}}
%    \end{macrocode}
% \end{option}
%
% \begin{option}{dvi2ps}
% \begin{option}{dvialw}
% \begin{option}{dvilaser}
% \begin{option}{dvitops}
% \begin{option}{psprint}
% \begin{option}{pubps}
% \begin{option}{ln}
% \changes{v0.6b}{1994/12/12}
%     {ln support (untested)}
% \changes{v1.0b}{1996/10/29}
%     {options for historic drivers depreciated}
% If \emph{anyone} is using any of these driver options
% would they let me know. All these are essentially untried
% and untested as far as I know.
%    \begin{macrocode}
%\DeclareOption{dvi2ps}{\def\Gin@driver{dvi2ps.def}}
%\DeclareOption{dvialw}{\def\Gin@driver{dvialw.def}}
%\DeclareOption{dvilaser}{\def\Gin@driver{dvilaser.def}}
%\DeclareOption{dvitops}{\def\Gin@driver{dvitops.def}}
%\DeclareOption{psprint}{\def\Gin@driver{psprint.def}}
%\DeclareOption{pubps}{\def\Gin@driver{pubps.def}}
%\DeclareOption{ln}{\def\Gin@driver{ln.def}}
%    \end{macrocode}
% \end{option}
% \end{option}
% \end{option}
% \end{option}
% \end{option}
% \end{option}
% \end{option}
%
% \begin{option}{debugshow}
% You probably don't want to use this\ldots
%    \begin{macrocode}
\DeclareOption{debugshow}{\catcode`\^^A=9 \let\GDebug\typeout}
%    \end{macrocode}
% \end{option}
%
% A local configuration file may define more options.
% It should also make one driver option the default, by calling
% |\ExecuteOptions| with the appropriate option.
% \changes{v0.4f}{1994/07/27}
%      {Add missing 3rd argument to \cs{InputIfFileExists}}
%    \begin{macrocode}
\InputIfFileExists{graphics.cfg}{}{}
%    \end{macrocode}
%
% \begin{option}{hiderotate}
% \changes{v0.4a}{1994/04/13}
%     {Rename option to hiderotate}
%    \begin{macrocode}
\DeclareOption{hiderotate}{%
  \def\Grot@start{\begingroup\setbox\z@\hbox\bgroup}
  \def\Grot@end{\egroup\endgroup}}
%    \end{macrocode}
% \end{option}
%
% \begin{option}{hidescale}
% \changes{v0.4a}{1994/04/13}
%     {Rename option to hidescale}
%    \begin{macrocode}
\DeclareOption{hidescale}{%
  \def\Gscale@start{\begingroup\setbox\z@\hbox\bgroup}
  \def\Gscale@end{\egroup\endgroup}}
%    \end{macrocode}
% \end{option}
%
% After the options are processed, load the appropriate driver file.
% If a site wants a default driver (eg |textures|) it just needs to put
% |\ExecuteOptions{textures}| in a |graphics.cfg| file.
%    \begin{macrocode}
\ProcessOptions
%    \end{macrocode}
%
% Check that a driver has been specified (either as an option, or as a
% default option in the configuration file). Then load the `def' file
% for that option, if it has not already been loaded by some other
% package (for instance the \textsf{color} package).
% \changes{v0.5c}{1994/10/03}
%     {Error if no driver specified}
%    \begin{macrocode}
\if!\Gin@driver!
  \PackageError{graphics}
    {No driver specified}
    {You should make a default driver option in a file \MessageBreak
     graphics.cfg\MessageBreak
     eg: \protect\ExecuteOptions{textures}%
     }
\else
  \PackageInfo{graphics}{Driver file: \Gin@driver}
  \@ifundefined{ver@\Gin@driver}{\input{\Gin@driver}}{}
\fi
%    \end{macrocode}
%
%
% \subsection{Graphics Inclusion}
%
% This Graphics package uses a lot of dimension registers. \TeX\ only
% has a limited number of registers, so rather than allocate new ones,
% re-use some existing \LaTeX\ registers. This is safe as long as all
% uses of the registers are \emph{local}, and that you can be sure
% that you \emph{never} need to have access to both uses within the
% same scope.
%
% \begin{macro}{\Gin@llx}
% \begin{macro}{\Gin@lly}
% \begin{macro}{\Gin@urx}
% \begin{macro}{\Gin@ury}
% In fact these four lengths are now stored as macros not as dimen
% registers, mainly so that integer |bp| lengths may be passed exactly.
% \changes{v0.7a}{1995/04/11}
%     {Bounding box coords no longer in registers}
%    \begin{macrocode}
\def\Gin@llx{0}
\let\Gin@lly\Gin@llx
\let\Gin@urx\Gin@llx
\let\Gin@ury\Gin@llx
%    \end{macrocode}
% \end{macro}
% \end{macro}
% \end{macro}
% \end{macro}
%
% \begin{macro}{\Gin@nat@width}
% \begin{macro}{\Gin@nat@height}
% The `natural' size of the graphic, before any scaling.
%    \begin{macrocode}
\let\Gin@nat@width\leftmarginv
\let\Gin@nat@height\leftmarginvi
%    \end{macrocode}
% \end{macro}
% \end{macro}
%
%  \begin{macro}{\ifGin@clip}
% This switch is \meta{true} if any graphics outside the specified
% bounding box (really viewport) should not be printed.
%    \begin{macrocode}
\newif\ifGin@clip
%    \end{macrocode}
% \end{macro}
%
% \begin{macro}{\DeclareGraphicsExtensions}
% \changes{v0.6a}{1994/11/29}
%     {Remove optional argument, ignore spaces in main argument.}
% Declare a comma separated list of default extensions to be used
% if the file is specified with no extension.
%    \begin{macrocode}
\newcommand\DeclareGraphicsExtensions[1]{%
  \edef\Gin@extensions{\zap@space#1 \@empty}}
%    \end{macrocode}
% \end{macro}
%
%  \begin{macro}{\Gin@extensions}
% Initialise the list of possible extensions.
% \changes{v0.4b}{1994/04/20}
%     {Use \cmd{\providecommand} in case a previous def file has
%     already defined it}
%    \begin{macrocode}
\providecommand\Gin@extensions{}
%    \end{macrocode}
%  \end{macro}
%
% \begin{macro}{\includegraphics}
% \changes{v1.1a}{2017/03/17}
% {\cs{leavevmode} added before the group in \cs{Gin@iii} so that
% settings in \cs{everypar} are cleared}
% \changes{v1.4a}{2019/11/30}{Add \cs{protected} gh/208}
% Top level command for the standard interface, just look for a |*|.
%    \begin{macrocode}
\protected\def\includegraphics{%
%    \end{macrocode}
% Clear |\everypar| before starting a group.
%    \begin{macrocode}
  \leavevmode
%    \end{macrocode}
%
%    \begin{macrocode}
  \@ifstar
    {\Gin@cliptrue\Gin@i}%
    {\Gin@clipfalse\Gin@i}}
%    \end{macrocode}
% \end{macro}
%
% \begin{macro}{\Gin@i}
%  If an optional argument is present, call |\Gin@ii| to process it,
%    otherwise call |\Ginclude@graphics|.
% \changes{v0.3e}{1994/03/09}
%     {Now specify ll before ur coordinate}
%    \begin{macrocode}
\def\Gin@i{%
  \@ifnextchar[%]
    \Gin@ii
    {\Gin@bboxfalse\Ginclude@graphics}}
%    \end{macrocode}
% \end{macro}
%
% \begin{macro}{\Gin@ii}
% Look for a second optional argument.
%    \begin{macrocode}
\def\Gin@ii[#1]{%
  \@ifnextchar[%]
    {\Gin@iii[#1]}
    {\Gin@iii[0,0][#1]}}
%    \end{macrocode}
% \end{macro}
%
% \begin{macro}{\Gin@iii}
% Set the coordinates of the {\bfseries l}lower {\bfseries l}eft corner,
% and the coordinates of the {\bfseries u}pper {\bfseries r}ight
% corner. The coordinates may be any \TeX\ dimension, defaulting to |bp|.
% \changes{v0.6c}{1994/12/15}
%     {(Denis Roegel) Extra group to keep bb settings local}
%    \begin{macrocode}
\def\Gin@iii[#1,#2][#3,#4]#5{%
  \begingroup
  \Gin@bboxtrue
  \Gin@defaultbp\Gin@llx{#1}%
  \Gin@defaultbp\Gin@lly{#2}%
  \Gin@defaultbp\Gin@urx{#3}%
  \Gin@defaultbp\Gin@ury{#4}%
  \Ginclude@graphics{#5}%
  \endgroup}
%    \end{macrocode}
% \end{macro}
%
%  \begin{macro}{\Gin@defaultbp}
% \changes{v0.7a}{1995/04/11}{Macro added}
%  \begin{macro}{\Gin@def@bp}
% \changes{v0.7a}{1995/04/11}{Macro added}
% This macro grabs a length, |#2|,  which may or may not have a unit,
% and if a unit is supplied, converts to `bp' and stores the value in
% |#1|. If a unit is not supplied `bp' is assumed, and |#2| is directly
% stored in |#1|. Note that supplying `bp' is not quite the same as
% supplying no units, as in the former case a conversion via `pt' and
% back to `bp' takes place which can introduce rounding error. The error
% is invisibly small but files conforming to Adobe  DSC should have
% \emph{integer} Bounding Box Coordinates, and conceivably some drivers
% might demand integer values.
% (Although most seem to accept real values (if they accept bounding box
% coordinates at all) in the |\special|. This is the reason why the
% mechanism uses |\def| and not \TeX\ lengths, as in earlier releases of
% the package.
%    \begin{macrocode}
\def\Gin@defaultbp#1#2{%
  \afterassignment\Gin@def@bp\dimen@#2bp\relax{#1}{#2}}
%    \end{macrocode}
%
%    \begin{macrocode}
\def\Gin@def@bp#1\relax#2#3{%
   \if!#1!%
     \def#2{#3}%
    \else
      \dimen@.99626\dimen@
      \edef#2{\strip@pt\dimen@}%
    \fi}
%    \end{macrocode}
%  \end{macro}
%  \end{macro}
%
% \begin{macro}{\DeclareGraphicsRule}
% Declare what actions should be taken for a particular file
% extension.
%
%  |#1| extension, |#2| type, |#3| read-file, |#4| command,
% \changes{v0.5d}{1994/10/24}
%     {(Arthur Ogawa) Check arg3 is *, not arg2.}
%    \begin{macrocode}
\def\DeclareGraphicsRule#1#2#3#4{%
  \edef\@tempa{\string *}\def\@tempb{#3}%
  \expandafter\edef\csname Gin@rule@#1\endcsname##1%
       {{#2}%
        {\ifx\@tempa\@tempb\noexpand\Gin@ext\else#3\fi}%
        {\ifx\indent#4\indent##1\else#4\fi}}}
%    \end{macrocode}
% \end{macro}
%
% An example rule base.
%\begin{verbatim}
%                    ext    type  read  command
% \DeclareGraphicsRule{.ps}   {eps} {.ps}    {}
% \DeclareGraphicsRule{.eps}  {eps} {.eps}   {}
% \DeclareGraphicsRule{.ps.gz}{eps} {.ps.bb} {`zcat #1}
% \DeclareGraphicsRule{.pcx}  {bmp} {}      {}
%\end{verbatim}
%
% \begin{macro}{\graphicspath}
% User level command to set the input path for graphics files.
% A list of directories, each in a |{}| group.
%    \begin{macrocode}
\def\graphicspath#1{\def\Ginput@path{#1}}
%    \end{macrocode}
% \end{macro}
%
% \begin{macro}{\Ginput@path}
% The default graphic path is |\input@path|.
%    \begin{macrocode}
\ifx\Ginput@path\@undefined
  \let\Ginput@path\input@path
\fi
%    \end{macrocode}
% \end{macro}
%
% \begin{macro}{\Gin@getbase}
% \changes{v0.6a}{1994/11/29} {remove \cs{Gin@sep}}
% \changes{v1.3a}{2019/07/01} {Support UTF-8 and spaces}
% \changes{v1.3b}{2019/07/20} {Delay adding quotes to filename}
% Given a possible extension, |#1|, check whether the file exists. If
% it does set |\Gin@base| and |\Gin@ext| to the filename stripped of
% the extension, and the extension, respectively.
%    \begin{macrocode}
\def\Gin@getbase#1{%
  \edef\Gin@tempa{%
    \def\noexpand\@tempa####1#1\space{%
      \def\noexpand\Gin@base{####1}}}%
  \IfFileExists{\filename@area\filename@base#1}%
    {\Gin@tempa
     \edef\uq@filef@und{\expandafter\unquote@name
                        \expandafter{\@filef@und}}%
     \expandafter\@tempa\uq@filef@und
     \edef\Gin@ext{#1}}{}}%
%    \end{macrocode}
% \end{macro}
%
% \begin{macro}{\Gin@ext}
% Initialise the macro to hold the extension.
%    \begin{macrocode}
\let\Gin@ext\relax
%    \end{macrocode}
% \end{macro}
%
% \begin{macro}{\Gin@sepdefault}
% \changes{v0.6a}{1994/11/29}
%     {remove \cs{Gin@sep}}
% \begin{macro}{\Gin@gzext}
% \changes{v1.3d}{2019/11/01}
%     {macro added}
% This must match the token used by |\filename@parse| to delimit the
% extension.
%    \begin{macrocode}
\def\Gin@sepdefault{.}
%    \end{macrocode}
%
%    \begin{macrocode}
\edef\Gin@gzext{\detokenize{gz}}
%    \edef{macrocode}
% \end{macro}
%
% \begin{macro}{\set@curr@file}
% \begin{macro}{\quote@name}
% \changes{v1.3c}{2019/10/08}
%     {Definitions added for old formats}
%    \begin{macrocode}
\def\Gin@temp#1{%
  \begingroup
    \escapechar\m@ne
    \xdef\@curr@file{\expandafter\string\csname #1\endcsname}%
  \endgroup
}
\ifx\set@curr@file\@Gin@temp
  \let\set@curr@file\@undefined
\fi
%    \end{macrocode}
%
%    \begin{macrocode}
\ifx\set@curr@file\@undefined
\def\set@curr@file#1{%
  \begingroup
    \escapechar\m@ne
    \xdef\@curr@file{\expandafter\string\csname\@firstofone#1\@empty\endcsname}%
  \endgroup
}
\def\quote@name#1{"\quote@@name#1\@gobble""}
\def\quote@@name#1"{#1\quote@@name}
\def\unquote@name#1{\quote@@name#1\@gobble"}
\fi
%    \end{macrocode}
% \end{macro}
% \end{macro}
%
% \begin{macro}{\Ginclude@graphics}
%  The main internal function implementing graphics file inclusion.
%  |#1| is the file name.
% \changes{v0.4c}{1994/04/28}
%     {Improve the wording of error messages}
% \changes{v0.5a}{1994/07/20}
%     {Add default (*) rule possibility}
% \changes{v0.6a}{1994/11/29}
%     {remove \cs{Gin@sepdefault}}
% \changes{v1.3a}{2019/07/01} {Support UTF-8 and spaces}
%    The quoting business for graphic files needs further sorting
%    out. This should be handled differently, right now we quote and
%    unquote all over the place as we still use the old code base.
%
%    This also makes the file name displays weird!
%
%    Guard |\detokenize| use for plain classic tex.
%    \begin{macrocode}
\def\Ginclude@graphics#1{%
  \ifx\detokenize\@undefined\else
    \edef\Gin@extensions{\detokenize\expandafter{\Gin@extensions}}%
  \fi
  \begingroup
  \let\input@path\Ginput@path
%    \end{macrocode}
%    A lot of quote juggling going on here (room for improvements).
%
% \changes{v1.3d}{2019/11/01}
%     {\cs{curr@name} now unquoted}
%    \begin{macrocode}
  \set@curr@file{#1}%
  \expandafter\filename@parse\expandafter{\@curr@file}%
%    \end{macrocode}
% If extension is |.gz| tack on to previous extension, eg |.eps.gz| if available.
%    \begin{macrocode}
  \ifx\filename@ext\Gin@gzext
    \expandafter\filename@parse\expandafter{\filename@base}%
    \ifx\filename@ext\relax
      \let\filename@ext\Gin@gzext
    \else
      \edef\Gin@ext{\Gin@ext\Gin@sepdefault\Gin@gzext}%
    \fi
  \fi
  \ifx\filename@ext\relax
    \@for\Gin@temp:=\Gin@extensions\do{%
      \ifx\Gin@ext\relax
        \Gin@getbase\Gin@temp
      \fi}%
  \else
    \Gin@getbase{\Gin@sepdefault\filename@ext}%
%    \end{macrocode}
% At this point try adding an extension, either if the given file name
% has none, or if the extension matches no existing graphics inclusion
% rule, so that \verb|a.b| may find \verb|a.b.png|, if only the latter
% or if both files exist. If no file is found then revert to the
% extension as given to get better error reporting.
%
% \changes{v1.3d}{2019/11/01}
%     {Try adding an extension even if the filename had a dot.}
% \changes{v1.4c}{2020/08/30}
%     {Try adding an extension even if the filename had a dot AND
%      if the filename without the extension exists (but doesn't have
%      a known extension).}
%    \begin{macrocode}
    \ifnum0%
        \ifx\Gin@ext\relax 1%
        \else \@ifundefined{Gin@rule@\Gin@ext}{1}{0}%
        \fi >0
      \let\Gin@ext\relax
      \let\Gin@savedbase\filename@base
      \let\Gin@savedext\filename@ext
      \edef\filename@base{\filename@base\Gin@sepdefault\filename@ext}%
      \let\filename@ext\relax
      \@for\Gin@temp:=\Gin@extensions\do{%
          \ifx\Gin@ext\relax
            \Gin@getbase\Gin@temp
          \fi}%
%    \end{macrocode}
% Restore if no file found using the known extensions.
%    \begin{macrocode}
      \ifx\Gin@ext\relax
        \let\filename@base\Gin@savedbase
        \let\filename@ext\Gin@savedext
      \fi
    \fi
%    \end{macrocode}
%
%    \begin{macrocode}
%    \end{macrocode}
% \changes{v0.4d}{1994/05/06}
%     {Make file not found a warning not an error}
% \changes{v0.6a}{1994/11/29}
%     {remove \cs{Gin@sep}}
% If the user supplied an explicit extension, just give a warning if the
% file does not exist. (It may be created later.)
%    \begin{macrocode}
    \ifx\Gin@ext\relax
       \@warning{File `#1' not found}%
       \def\Gin@base{\filename@area\filename@base}%
%    \end{macrocode}
% \changes{v0.6c}{1994/12/15}
%     {(Piet van Oostrum) include `.' in \cs{Gin@ext}}%
%    \begin{macrocode}
       \edef\Gin@ext{\Gin@sepdefault\filename@ext}%
    \fi
  \fi
%    \end{macrocode}
% If no extension is supplied, it is an error if the file does not
% exist, as there is no way for the system to know which extension to
% supply.
%    \begin{macrocode}
    \ifx\Gin@ext\relax
         \@latex@error{File `#1' not found}%
         {I could not locate the file with any of these extensions:^^J%
          \Gin@extensions^^J\@ehc}%
    \else
       \@ifundefined{Gin@rule@\Gin@ext}%
%    \end{macrocode}
% \changes{v0.5a}{1994/07/20}
%     {Add default (*) rule possibility}
% Handle default rule.
%    \begin{macrocode}
         {\ifx\Gin@rule@*\@undefined
            \@latex@error{Unknown graphics extension: \Gin@ext}\@ehc
          \else
            \expandafter\Gin@setfile\Gin@rule@*{\Gin@base\Gin@ext}%
           \fi}%
         {\expandafter\expandafter\expandafter\Gin@setfile
             \csname Gin@rule@\Gin@ext\endcsname{\Gin@base\Gin@ext}}%
    \fi
  \endgroup}
%    \end{macrocode}
% \end{macro}
%
% \begin{macro}{\ifGread@}
% True if a file should be read to obtain the natural size.
%    \begin{macrocode}
\newif\ifGread@\Gread@true
%    \end{macrocode}
% \end{macro}
%
% \begin{macro}{\Gin@setfile}
% Set a file to the size specified in arguments, or in a `read file'.
% \changes{v0.5d}{1994/10/24}
%     {(Arthur Ogawa) Add missing \cs{expandafter}}
%    \begin{macrocode}
\def\Gin@setfile#1#2#3{%
  \ifx\\#2\\\Gread@false\fi
  \ifGin@bbox\else
    \ifGread@
      \csname Gread@%
         \expandafter\ifx\csname Gread@#1\endcsname\relax
           eps%
         \else
           #1%
         \fi
      \endcsname{\Gin@base#2}%
    \else
%    \end{macrocode}
% \changes{v0.5a}{1994/07/20}
%     {New error trap for missing size arguments}
% \changes{v0.7f}{1995/11/10}
%     {Use \cs{Gin@nosize}}
% By now the natural size should be known either from arguments or
% from the file. If not generate an error. (The \textsf{graphicx}
% interface relaxes this condition slightly.)
%    \begin{macrocode}
      \Gin@nosize{#3}%
    \fi
  \fi
%    \end{macrocode}
% \changes{v0.7a}{1995/04/11}{\cs{Gin@viewport@code} added.}
% The following call will modify the `natural size' if the user has
% supplied a viewport or trim specification. (Not available in the
% standard interface.)
%    \begin{macrocode}
  \Gin@viewport@code
%    \end{macrocode}
% Save the natural size, and then call |\Gin@req@sizes| which (in the
% key-val interface) will calculate the required size from the natural
% size, and any scaling info.
%    \begin{macrocode}
  \Gin@nat@height\Gin@ury bp%
  \advance\Gin@nat@height-\Gin@lly bp%
  \Gin@nat@width\Gin@urx bp%
  \advance\Gin@nat@width-\Gin@llx bp%
  \Gin@req@sizes
%    \end{macrocode}
%
% Call |\Ginclude@|\emph{type} to include the figure unless
% this is not defined, or draft mode is being used.
% \changes{v0.7h}{1996/02/20}
%      {missing \texttt{@} added in \texttt{Gread@} csnames}
%    \begin{macrocode}
  \expandafter\ifx\csname Ginclude@#1\endcsname\relax
    \Gin@drafttrue
    \expandafter\ifx\csname Gread@#1\endcsname\relax
      \@latex@error{Can not include graphics of type: #1}\@ehc
      \global\expandafter\let\csname Gread@#1\endcsname\@empty
    \fi
  \fi
  \leavevmode
  \ifGin@draft
      \hb@xt@\Gin@req@width{%
        \vrule\hss
        \vbox to \Gin@req@height{%
           \hrule \@width \Gin@req@width
           \vss
           \edef\@tempa{#3}%
           \rlap{ \ttfamily\expandafter\strip@prefix\meaning\@tempa}%
           \vss
           \hrule}%
        \hss\vrule}%
  \else
%    \end{macrocode}
% \changes{v0.3i}{1994/3/23}
%     {Add file list info}
% Support |\listfiles| and then set the final box to the required size.
%    \begin{macrocode}
    \@addtofilelist{#3}%
    \ProvidesFile{#3}[Graphic file (type #1)]%
    \setbox\z@\hbox{\csname Ginclude@#1\endcsname{#3}}%
    \dp\z@\z@
    \ht\z@\Gin@req@height
    \wd\z@\Gin@req@width
  \box\z@
  \fi}
%    \end{macrocode}
% \end{macro}
%
% \begin{macro}{\Gin@decode}
% \changes{v1.2c}{2017/06/25}
%     {New macro}
% In the standard interface this is a no-op, but needs to be defined
% to allow the caching code to be set up.
%    \begin{macrocode}
\let\Gin@decode\@empty
%    \end{macrocode}
% \end{macro}
%
%  \begin{macro}{\Gin@exclamation}
% \changes{v0.7g}{1995/12/06}{Macro added}
% Catcode 12 |!|, in case of French, or other language styles.
%    \begin{macrocode}
\def\Gin@exclamation{!}
%    \end{macrocode}
%  \end{macro}
%
% \begin{macro}{\Gin@page}
% \changes{v1.2a}{2017/06/01}
%     {New macro}
% In the standard interface this is a no-op, but needs to be defined
% to allow the caching code to be set up.
%    \begin{macrocode}
\let\Gin@page\@empty
%    \end{macrocode}
% \end{macro}
%
% \begin{macro}{\Gin@pagebox}
% \changes{v1.2a}{2017/06/01}
%     {New macro}
% In the standard interface always points to the |cropbox|.
%    \begin{macrocode}
\def\Gin@pagebox{cropbox}
%    \end{macrocode}
% \end{macro}
%
% \begin{macro}{\ifGin@interpolate}
% \changes{v1.2a}{2017/06/01}
%     {New macro}
%  In the standard setting a no-op.
%    \begin{macrocode}
\newif\ifGin@interpolate
%    \end{macrocode}
% \end{macro}
%
% \begin{macro}{\Gin@log}
%   \changes{v1.2a}{2017/06/01} {New macro} In the standard interface
%   this prints to the log but can be changed via keys in \textsf{graphicx}.
%    \begin{macrocode}
\let\Gin@log\wlog
%    \end{macrocode}
% \end{macro}
%
%
%  \begin{macro}{\Gin@req@sizes}
%  \begin{macro}{\Gin@scalex}
%  \begin{macro}{\Gin@scaley}
%  \begin{macro}{\Gin@req@height}
%  \begin{macro}{\Gin@req@width}
% In the standard interface there is no scaling, so the required size
% is the same as the natural size. In other interfaces |\Gin@req@sizes|
% will be responsible for setting these parameters. Here we can set them
% globally.
% \changes{v0.3g}{1994/03/15}
%     {Initialise y-scale to !!}
% \changes{v1.0g}{1998/05/14}
%     {Fix !! usage in changes entries. graphics/2724}
%    \begin{macrocode}
\let\Gin@req@sizes\relax
\def\Gin@scalex{1}%
\let\Gin@scaley\Gin@exclamation
\let\Gin@req@height\Gin@nat@height
\let\Gin@req@width\Gin@nat@width
%    \end{macrocode}
%  \end{macro}
%  \end{macro}
%  \end{macro}
%  \end{macro}
%  \end{macro}
%
%  \begin{macro}{\Gin@viewport@code}
% In the standard interface there is no possibility of specifying
% a viewport, so this is a no-op.
% \changes{v0.7a}{1995/04/11}{Macro added}
%    \begin{macrocode}
\let\Gin@viewport@code\relax
%    \end{macrocode}
%  \end{macro}
%
%  \begin{macro}{\Gin@nosize}
% \changes{v1.0p}{2014/10/14}{Fixed typo in error message}
% \changes{v0.7f}{1995/11/10}
%     {Macro added}
% This command is called in the case that the graphics type
% specifies no `read file' and the user supplied no size arguments.
% In the standard interface can only generate an error.
%    \begin{macrocode}
\def\Gin@nosize#1{%
  \@latex@error
      {Cannot determine size of graphic in #1 (no size specified)}%
      \@ehc}
%    \end{macrocode}
%  \end{macro}
%
% \subsection{Reading the BoundingBox in EPS files}
%
% \begin{macro}{\ifGin@bbox}
% \changes{v0.7a}{1995/04/11}{Name changed from \cs{ifGin@viewport}}
% This switch should be set \meta{true} once a size has been found,
% either in an argument, or in an external file.
%    \begin{macrocode}
\newif\ifGin@bbox
%    \end{macrocode}
%  \end{macro}
%
% \begin{macro}{\Gread@generic}
% \changes{v1.2a}{2017/06/01}
%     {New macro}
% \begin{macro}{\Gread@generic@aux}
% \begin{macro}{\Gread@eps}
% Read an EPS file (|#1|) and search for a line
% starting with |%%BoundingBox| and returns the result
% by setting four dimension registers
% |\Gin@llx|, |\Gin@lly|, |\Gin@urx| and |\Gin@ury|.
% \changes{v0.5e}{1994/11/02}
%     {Fix the catcodes of \cs{endlinechar} and ctrl-D}
% \changes{v0.7a}{1995/04/11}
%     {Fix more catcodes, for binary headers of eps files}
% \changes{v1.2a}{2017/06/01}
%     {Split to allow code reuse by \cs{Gread@extractbb} in
%     \texttt{dvipdfmx.def} driver}
% \begin{macro}{\Gread@eps@aux}
%    \begin{macrocode}
\def\Gread@generic#1#2{%
  \edef\Gread@attr@hash{%
    \ifx\Gin@pagebox\@empty\else
      :\Gin@pagebox
    \fi
    \ifx\Gin@page\@empty\else
      :P\Gin@page
    \fi
  }%
  \@ifundefined{#1 bbox\Gread@attr@hash}%
    {\Gread@generic@aux{#1}{#2}}
    {%
      \expandafter\global\expandafter\let\expandafter\@gtempa
        \csname #1 bbox\Gread@attr@hash\endcsname
    }%
  \expandafter\Gread@parse@bb\@gtempa\\%
}
\def\Gread@generic@aux#1#2{%
  \begingroup
%    \end{macrocode}
% Make it reasonably safe to have binary headers in the EPS file
% before the bounding box line.
%    \begin{macrocode}
  \@tempcnta\z@
  \loop\ifnum\@tempcnta<\@xxxii
     \catcode\@tempcnta14 %
     \advance\@tempcnta\@ne
  \repeat
  \catcode`\^^?14 %
  \let\do\@makeother
  \dospecials
%    \end{macrocode}
% Make sure tab and space are accepted as white space.
% \changes{v1.0e}{1997/09/02}
%     {Allow TAB in DSC comments graphics/2587}
% \changes{v1.0i}{1999/01/07}
%     {Fix catcode of hyphen. graphics/2846}
% \changes{v1.4b}{2020/08/09}
%     {Normalise \cs{endlinechar} gh/286}
%    \begin{macrocode}
  \catcode`\ 10 %
  \catcode`\^^I10 %
  \endlinechar13 %
  \catcode\endlinechar5 %
  \@makeother\:%
  \@makeother\-%
%    \end{macrocode}
% The first thing we need to do is to open the
% information file, if possible.
% \changes{v0.4d}{1994/05/06}
%     {Improve the error message if the info file is not there.}
% \changes{v1.2a}{2017/06/01}
%     {Allow for spaces in name of a file}
%    Due to the space handling code file names are now already quoted
%    so we should not add any quotes around \verb=#1= any more.
% \changes{v1.3b}{2019/07/20} {add quotes here again}
%    \begin{macrocode}
  \immediate\openin\@inputcheck\quote@name{#1} %
  #2{#1}%
%    \end{macrocode}
% \changes{v0.3i}{1994/03/23}
%     {Wording of error message improved}
% \changes{v1.0}{1996/05/29}
%     {Use \cs{@gtempa} not \cs{g@tempa} /2090}
% \changes{v1.2b}{2017/06/13}
%     {Stored bounding box data for reuse}
%    \begin{macrocode}
  \ifGin@bbox
    \expandafter\xdef\csname #1 bbox\Gread@attr@hash\endcsname{\@gtempa}%
  \else
    \@latex@error
      {Cannot determine size of graphic in #1 (no BoundingBox)}%
      \@ehc
    \gdef\@gtempa{0 0 72 72 }%
  \fi
  \endgroup
}
\def\Gread@eps#1{%
  \Gread@generic{#1}\Gread@eps@aux
}
\def\Gread@eps@aux#1{%
  \ifeof\@inputcheck
    \@latex@error{File `#1' not found}\@ehc
  \else
%    \end{macrocode}
% Now we'll scan lines until we find one that starts with
% |%%BoundingBox:|
% We need to reset the catcodes to read the file, and so this
% is done in a group.
%    \begin{macrocode}
     \Gread@true
     \let\@tempb\Gread@false
     \loop
       \read\@inputcheck to\@tempa
       \ifeof\@inputcheck
         \Gread@false
       \else
         \expandafter\Gread@find@bb\@tempa:.\\%
       \fi
     \ifGread@
     \repeat
    \immediate\closein\@inputcheck
  \fi
}
%    \end{macrocode}
% \end{macro}
% \end{macro}
% \end{macro}
% \end{macro}
%
% \begin{macro}{\Gread@find@bb}
% If a line in the EPS file starts with a |%%BoundingBox:|, we
% will examine it more closely. Note using the `extra' argument |#2#3|
% causes any space after the |:| to be gobbled.
%    \begin{macrocode}
\long\def\Gread@find@bb#1:#2#3\\{%
  \def\@tempa{#1}%
  \ifx\@tempa\Gread@BBox
    \Gread@test@atend#2#3()\\%
  \fi}
%    \end{macrocode}
% \end{macro}
% \begin{macro}{\Gread@test@atend}
% Determine if the stuff following the |%%BoundingBox| is `(atend)',
% which will involve further reading of the file. This is accomplished
% by making |\@tempb| into a no-op, so that finding a |%%BoundingBox|
% does not stop the loop.
%    \begin{macrocode}
\def\Gread@test@atend#1(#2)#3\\{%
  \def\@tempa{#2}%
  \ifx\@tempa\Gread@atend
    \Gread@true
    \let\@tempb\relax
  \else
    \gdef\@gtempa{#1}%
    \@tempb
    \Gin@bboxtrue
  \fi}
%    \end{macrocode}
% \end{macro}
%
% \begin{macro}{\Gread@parse@bb}
% We have |%%BoundingBox| and what follows is not `(atend)' so we
% will parse the rest of the line as a BB with four elements.
% PostScript files should never have units specified in the
% BoundingBox comment, but we allow arbitrary \TeX\ units in external
% files, or in other interfaces.
%    \begin{macrocode}
\def\Gread@parse@bb#1 #2 #3 #4 #5\\{%
  \Gin@defaultbp\Gin@llx{#1}%
  \Gin@defaultbp\Gin@lly{#2}%
  \Gin@defaultbp\Gin@urx{#3}%
  \Gin@defaultbp\Gin@ury{#4}}%
%    \end{macrocode}
% \end{macro}
%
% \begin{macro}{\Gread@atend}
% |atend| as a macro for testing with |\ifx|.
%    \begin{macrocode}
\def\Gread@atend{atend}
%    \end{macrocode}
% \end{macro}
%
% Viewport and trim, originally in |graphicx|.
%
% \begin{macro}{\Gin@viewport}
% \changes{v1.0c}{1996/10/31}{Original bb saved}
% If a viewport is specified, reset the bounding box coordinates
% by adding the original origin, |\Gin@llx|, |\Gin@lly| to the new
% values specified as the viewport. The original Bounding box
% coordinates are saved in |\Gin@ollx|\ldots\ some drivers
% might need this information (currently just |tcidvi|).
%    \begin{macrocode}
\def\Gin@viewport{%
  \let\Gin@ollx\Gin@llx
  \let\Gin@olly\Gin@lly
  \let\Gin@ourx\Gin@urx
  \let\Gin@oury\Gin@ury
  \dimen@\Gin@llx\p@\advance\dimen@ \Gin@vurx\p@
                      \edef\Gin@urx{\strip@pt\dimen@}%
  \dimen@\Gin@lly\p@\advance\dimen@ \Gin@vury\p@
                      \edef\Gin@ury{\strip@pt\dimen@}%
  \dimen@\Gin@llx\p@\advance\dimen@ \Gin@vllx\p@
                      \edef\Gin@llx{\strip@pt\dimen@}%
  \dimen@\Gin@lly\p@\advance\dimen@ \Gin@vlly\p@
                      \edef\Gin@lly{\strip@pt\dimen@}}
%    \end{macrocode}
% \end{macro}
%
% \begin{macro}{\Gin@trim}
% \changes{v1.0c}{1996/10/31}{Original bb saved}
% If a trim is specified, reset the bounding box coordinates
% by trimming the four specified values off each side of the
% graphic.
%    \begin{macrocode}
\def\Gin@trim{%
  \let\Gin@ollx\Gin@llx
  \let\Gin@olly\Gin@lly
  \let\Gin@ourx\Gin@urx
  \let\Gin@oury\Gin@ury
  \dimen@\Gin@llx\p@\advance\dimen@ \Gin@vllx\p@
                      \edef\Gin@llx{\strip@pt\dimen@}%
  \dimen@\Gin@lly\p@\advance\dimen@ \Gin@vlly\p@
                      \edef\Gin@lly{\strip@pt\dimen@}%
  \dimen@\Gin@urx\p@\advance\dimen@ -\Gin@vurx\p@
                      \edef\Gin@urx{\strip@pt\dimen@}%
  \dimen@\Gin@ury\p@\advance\dimen@ -\Gin@vury\p@
                      \edef\Gin@ury{\strip@pt\dimen@}}
%    \end{macrocode}
% \end{macro}
%
% \begin{macro}{\Gin@vllx}
% \begin{macro}{\Gin@vlly}
% \begin{macro}{\Gin@vurx}
% \begin{macro}{\Gin@vury}
% Four macros to hold the modifiers for the bounding box for viewport
% and trim specifications.
%    \begin{macrocode}
\let\Gin@vllx\Gin@llx\let\Gin@vlly\Gin@llx
\let\Gin@vurx\Gin@llx\let\Gin@vury\Gin@llx
%    \end{macrocode}
% \end{macro}
% \end{macro}
% \end{macro}
% \end{macro}
%
% \subsection{Rotation}
%
% As above, we will re-use some existing local registers.
%
% \begin{macro}{\Grot@height}
% \begin{macro}{\Grot@left}
% \begin{macro}{\Grot@right}
% \begin{macro}{\Grot@depth}
% Final Rotated box dimensions
%    \begin{macrocode}
\let\Grot@height\@ovxx
\let\Grot@left\@ovyy
\let\Grot@right\@ovdx
\let\Grot@depth\@ovdy
%    \end{macrocode}
% \end{macro}
% \end{macro}
% \end{macro}
% \end{macro}
%
% \begin{macro}{\Grot@h}
% \begin{macro}{\Grot@l}
% \begin{macro}{\Grot@r}
% \begin{macro}{\Grot@d}
% Original box dimensions
%    \begin{macrocode}
\let\Grot@l\@ovro
\let\Grot@r\@ovri
\let\Grot@h\@xdim
\let\Grot@d\@ydim
%    \end{macrocode}
% \end{macro}
% \end{macro}
% \end{macro}
% \end{macro}
%
% \begin{macro}{\Grot@x}
% \begin{macro}{\Grot@y}
% Coordinates of centre of rotation.
%    \begin{macrocode}
\let\Grot@x\@linelen
\let\Grot@y\@dashdim
%    \end{macrocode}
% \end{macro}
% \end{macro}
%
% \begin{macro}{\rotatebox}
% The angle is specified by |#1|. The box to be rotated is |#2|.
% In the standard interface the centre of rotation is $(0,0)$.
% Then finally call |\Grot@box| to rotate the box.
% \changes{v0.3f}{1994/03/11}{Remove star form}
% \changes{v0.3h}{1994/03/17}{Fix Typo}
% \changes{v0.7a}{1995/04/11}{\cs{leavevmode} added graphics/1521}
% \changes{v1.0n}{2001/07/07}
%     {Made long (CAR) graphics/2908 and 3345}
% \changes{v1.4a}{2019/11/30}{Add \cs{protected} gh/208}
%    \begin{macrocode}
\protected\long\def\rotatebox#1#2{%
  \leavevmode
  \Grot@setangle{#1}%
  \setbox\z@\hbox{{#2}}%
  \Grot@x\z@
  \Grot@y\z@
  \Grot@box}
%    \end{macrocode}
% \end{macro}
%
%
% \begin{macro}{\Grot@setangle}
% Set the internal macro used by |\Grot@box|. In the standard
% interface this is trivial, but other interfaces may have more
% interesting definitions. For example:
%\begin{verbatim}
% \def\Grot@setangle#1{%
%   \dimen@#1\p@
%   \dimen@-57.2968\dimen@
%   \edef\Grot@angle{\strip@pt\dimen@}}
%\end{verbatim}
% This would cause the argument of |\rotatebox| to be interpreted as
% an angle specified in \emph{radians}, \emph{clockwise}.
%    \begin{macrocode}
\def\Grot@setangle#1{\edef\Grot@angle{#1}}
%    \end{macrocode}
% \end{macro}
%
% \subsection{Deriving a `bounding box' for rotated object}
% We want to know the size of a `bounding box' enclosing the rotated
% box.
% We define two formulae (as \TeX\ macros) to work out the $x$ and $y$
% coordinates of vertices of the rotated box
% in relation to its original coordinates (i.e., its width, height
% and depth). The box we visualize with vertices $B$, $C$,
% $D$ and $E$ is illustrated below. The vertex
% $S$ is the reference point on the baseline. $O$ is the centre of
% rotation, which in the standard interface is always $S$.
%
% \begin{center}
% \setlength{\unitlength}{3pt}%
%
% \begin{picture}(34,36)(12,44)
% \thicklines
% \put(20,52){\dashbox{1}(20,21){}}
% \put(20,80){\line(0,-1){36}}
% \put(12,58){\line(1, 0){34}}
% \put(41,59){A}
% \put(40,74){B}
% \put(21,74){C}
% \put(21,49){D}
% \put(40,49){E}
% \put(21,59){S}
% \put(33,65){O}
% \put(33,65){\circle*{1}}
% \end{picture}
% \end{center}
%
% The formulae are, for a point $P$ and angle $\alpha$:
%\[
% \begin{array}{l}
% P'_x = P_x - O_x \\
% P'_y = P_y - O_y \\
% P''_x =  ( P'_x \times \cos(\alpha)) - ( P'_y \times \sin(\alpha) ) \\
% P''_y =  ( P'_x \times \sin(\alpha)) + ( P'_y \times \cos(\alpha) ) \\
% P'''_x = P''_x + O_x + L_x \\
% P'''_y = P''_y + O_y
% \end{array}
% \]
% The `extra' horizontal translation $L_x$ at the end is calculated so
% that the leftmost point of the resulting box has $x$-coordinate $0$.
% This is desirable as \TeX\ boxes must have the reference point at
% the left edge of the box.
%
% \begin{macro}{\Grot@Px}
% Work out new $x$ coordinate of point after rotation. The parameters
% |#2| and |#3| are the original $x$ and $y$ coordinates of the point.
% The new $x$ coordinate is stored in |#1|.
%    \begin{macrocode}
\def\Grot@Px#1#2#3{%
        #1\Grot@cos#2%
        \advance#1-\Grot@sin#3}
%    \end{macrocode}
% \end{macro}
% \begin{macro}{\Grot@Py}
% Work out new $y$ coordinate of point after rotation. The parameters
% |#2| and |#3| are the original $x$ and $y$ coordinates of the point.
% The new $y$ coordinate is stored in |#1|.
%    \begin{macrocode}
\def\Grot@Py#1#2#3{%
        #1\Grot@sin#2%
        \advance#1\Grot@cos#3}
%    \end{macrocode}
% \end{macro}
%
% \begin{macro}{\Grot@box}
% This is the tricky bit. We can rotate the box, but then need
% to work out how much space to leave for it on the page.
%
% We simplify matters by working out first which quadrant we are in, and
% then picking just the right values.
%
%    \begin{macrocode}
\def\Grot@box{%
  \begingroup
%    \end{macrocode}
% We are going to need to know the sine and cosine
% of  the angle; simplest to calculate these now.
%    \begin{macrocode}
  \CalculateSin\Grot@angle
  \CalculateCos\Grot@angle
  \edef\Grot@sin{\UseSin\Grot@angle}%
  \edef\Grot@cos{\UseCos\Grot@angle}%
^^A   \GDebug{Rotate: angle \Grot@angle, sine is \Grot@sin,
^^A             cosine is \Grot@cos}%
%    \end{macrocode}
% Save the four extents of the original box.
%    \begin{macrocode}
  \Grot@r\wd\z@  \advance\Grot@r-\Grot@x
  \Grot@l\z@     \advance\Grot@l-\Grot@x
  \Grot@h\ht\z@  \advance\Grot@h-\Grot@y
  \Grot@d-\dp\z@ \advance\Grot@d-\Grot@y
%    \end{macrocode}
% Now a straightforward test to see which quadrant we are
% operating in;
%    \begin{macrocode}
  \ifdim\Grot@sin\p@>\z@
    \ifdim\Grot@cos\p@>\z@
%    \end{macrocode}
% First quadrant:
% Height=$By$, Right=$Ex$, Left=$Cx$, Depth=$Dy$
%    \begin{macrocode}
      \Grot@Py\Grot@height \Grot@r\Grot@h%B
      \Grot@Px\Grot@right  \Grot@r\Grot@d%E
      \Grot@Px\Grot@left   \Grot@l\Grot@h%C
      \Grot@Py\Grot@depth  \Grot@l\Grot@d%D
    \else
%    \end{macrocode}
% Second quadrant:
% Height=$Ey$, Right=$Dx$, Left=$Bx$, Depth=$Cy$
%    \begin{macrocode}
      \Grot@Py\Grot@height \Grot@r\Grot@d%E
      \Grot@Px\Grot@right  \Grot@l\Grot@d%D
      \Grot@Px\Grot@left   \Grot@r\Grot@h%B
      \Grot@Py\Grot@depth  \Grot@l\Grot@h%C
    \fi
  \else
    \ifdim\Grot@cos\p@<\z@
%    \end{macrocode}
% Third quadrant:
% Height=$Dy$, Right=$Cx$, Left=$Ex$, Depth=$By$
%    \begin{macrocode}
      \Grot@Py\Grot@height \Grot@l\Grot@d%D
      \Grot@Px\Grot@right  \Grot@l\Grot@h%C
      \Grot@Px\Grot@left   \Grot@r\Grot@d%E
      \Grot@Py\Grot@depth  \Grot@r\Grot@h%B
    \else
%    \end{macrocode}
% Fourth quadrant:
% Height=$Cy$, Right=$Bx$, Left=$Dx$, Depth=$Ey$
%    \begin{macrocode}
      \Grot@Py\Grot@height \Grot@l\Grot@h%C
      \Grot@Px\Grot@right  \Grot@r\Grot@h%B
      \Grot@Px\Grot@left   \Grot@l\Grot@d%D
      \Grot@Py\Grot@depth  \Grot@r\Grot@d%E
    \fi
  \fi
%    \end{macrocode}
% Now we should translate back by $(O_x,O_y)$, but \TeX\ can not really
% deal with boxes that do not have the reference point at the left edge.
% (Everything with a $-$ve $x$-coordinate would over-print earlier
% text). So we modify the horizontal translation so that the
% reference point as understood by \TeX\ \emph{is} at the left edge.
% This means that the `centre of rotation' is not fixed by |\rotatebox|,
% but typically moves horizontally. We also need to find the image of
% the original reference point, $S$, as that is where the rotation
% specials must be inserted.
%
%    \begin{macrocode}
  \advance\Grot@height\Grot@y
  \advance\Grot@depth\Grot@y
  \Grot@Px\dimen@  \Grot@x\Grot@y
  \Grot@Py\dimen@ii \Grot@x\Grot@y
  \dimen@-\dimen@     \advance\dimen@-\Grot@left
  \dimen@ii-\dimen@ii \advance\dimen@ii\Grot@y
%    \end{macrocode}
%
%    \begin{macrocode}
^^A   \GDebug{Rotate: (l,r,h,d)^^J%
^^A Original \the\Grot@l,\the\Grot@r,\the\Grot@h,\the\Grot@d,^^J%
^^A New..... \the\Grot@left,\the\Grot@right,%
^^A          \the\Grot@height,\the\Grot@depth}%
%    \end{macrocode}
%
%    \begin{macrocode}
  \setbox\z@\hbox{%
    \kern\dimen@
    \raise\dimen@ii\hbox{\Grot@start\box\z@\Grot@end}}%
  \ht\z@\Grot@height
  \dp\z@-\Grot@depth
  \advance\Grot@right-\Grot@left\wd\z@\Grot@right
  \leavevmode\box\z@
  \endgroup}
%    \end{macrocode}
% \end{macro}
%
%
% \subsection{Stretching and Scaling}
%
%
%  \begin{macro}{\scalebox}
% The top level |\scalebox|. If the vertical scale factor is omitted it
% defaults to the horizontal scale factor, |#1|.
% \changes{v0.3d}{1994/03/06}{Better support for negative arguments.}
% \changes{v1.4a}{2019/11/30}{Add \cs{protected} gh/208}
%    \begin{macrocode}
\protected\def\scalebox#1{%
  \@ifnextchar[{\Gscale@box{#1}}{\Gscale@box{#1}[#1]}}
%    \end{macrocode}
%  \end{macro}
%
%  \begin{macro}{\Gscale@box}
% Internal version of |\scalebox|.
% \changes{v0.7a}{1995/04/11}
%     {\cs{leavevmode} moved earlier. graphics/1521}
% \changes{v1.0j}{1999/01/07}
%     {made long. graphics/2908}
% \changes{v1.1b}{2017/04/14}
%     {Adjust box resizing for math mode gh issue 6}
%    \begin{macrocode}
\long\def\Gscale@box#1[#2]#3{%
  \leavevmode
  \def\Gscale@x{#1}\def\Gscale@y{#2}%
  \setbox\z@\hbox{{#3}}%
  \setbox\tw@\hbox{\Gscale@start\rlap{\copy\z@}\Gscale@end}%
  \ifdim#2\p@<\z@
    \ht\tw@-#2\dp\z@
    \dp\tw@-#2\ht\z@
  \else
    \ht\tw@#2\ht\z@
    \dp\tw@#2\dp\z@
  \fi
  \ifdim#1\p@<\z@
    \hb@xt@-#1\wd\z@{\kern-#1\wd\z@\box\tw@\hss}%
  \else
    \hb@xt@#1\wd\z@{\box\tw@\kern#1\wd\z@\hss}%
  \fi}
%    \end{macrocode}
%  \end{macro}
%
%
%  \begin{macro}{\reflectbox}
% Just an abbreviation for the appropriate scale to get reflection.
% \changes{v0.3e}{1994/03/09}{Macro added}
% \changes{v1.4a}{2019/11/30}{Add \cs{protected} gh/208}
%    \begin{macrocode}
\protected\def\reflectbox{\Gscale@box-1[1]}
%    \end{macrocode}
%  \end{macro}
%
%
%  \begin{macro}{\resizebox}
% \changes{v0.3b}{1994/03/01}{Recode \cmd\resizebox.}
% \changes{v0.7b}{1995/04/27}
%         {Add \cs{leavevmode} for graphics/1512}
% \changes{v1.4a}{2019/11/30}{Add \cs{protected} gh/208}
% Look for a |*|, which specifies that a final vertical size refers to
% `height + depth' not just `height'.
%    \begin{macrocode}
\protected\def\resizebox{%
  \leavevmode
  \@ifstar{\Gscale@@box\totalheight}{\Gscale@@box\height}}
%    \end{macrocode}
%  \end{macro}
%
%  \begin{macro}{\Gscale@@box}
% Look for the |!| in the arguments.
% \changes{v0.5a}{1994/07/20}
%     {Support French active !!}
% \changes{v0.5d}{1994/10/24}
%     {Correct the support for !!}
% \changes{v0.7g}
%     {1995/12/06}{Use \cs{Gin@exclamation}}
%    \begin{macrocode}
\def\Gscale@@box#1#2#3{%
  \let\@tempa\Gin@exclamation
  \expandafter\def\expandafter\@tempb\expandafter{\string#2}%
  \expandafter\def\expandafter\@tempc\expandafter{\string#3}%
  \ifx\@tempb\@tempa
    \ifx\@tempc\@tempa
      \toks@{\mbox}%
    \else
      \toks@{\Gscale@box@dd{#3}#1}%
    \fi
  \else
    \ifx\@tempc\@tempa
      \toks@{\Gscale@box@dd{#2}\width}%
    \else
      \toks@{\Gscale@box@dddd{#2}\width{#3}#1}%
    \fi
  \fi
  \the\toks@}
%    \end{macrocode}
%  \end{macro}
%
%  \begin{macro}{\Gscale@box@dd}
% Scale the text |#3| in both directions by a factor $|#1|/|#2|$.
% \changes{v0.3i}{1994/03/23}
%     {Missing percent added}
% \changes{v1.0i}{1999/01/07}
%     {made long. graphics/2908}
%    \begin{macrocode}
\long\def\Gscale@box@dd#1#2#3{%
  \@begin@tempboxa\hbox{#3}%
    \setlength\@tempdima{#1}%
    \setlength\@tempdimb{#2}%
    \Gscale@div\@tempa\@tempdima\@tempdimb
    \Gscale@box\@tempa[\@tempa]{\box\@tempboxa}%
  \@end@tempboxa}
%    \end{macrocode}
%  \end{macro}
%
%  \begin{macro}{\Gscale@box@dddd}
% Scale the text |#5| horizontally by a factor $|#1|/|#2|$ and
% vertically by a factor $|#3|/|#4|$.
% \changes{v0.3i}{1994/03/23}
%     {Missing percent added}
% \changes{v0.7e}{1995/09/29}
%     {Add \cs{ifGin@iso} code added}
% \changes{v1.0i}{1999/01/07}
%     {made long. graphics/2908}
%    \begin{macrocode}
\long\def\Gscale@box@dddd#1#2#3#4#5{%
  \@begin@tempboxa\hbox{#5}%
    \setlength\@tempdima{#1}%
    \setlength\@tempdimb{#2}%
    \Gscale@div\@tempa\@tempdima\@tempdimb
    \setlength\@tempdima{#3}%
    \setlength\@tempdimb{#4}%
    \Gscale@div\@tempb\@tempdima\@tempdimb
    \ifGin@iso
      \ifdim\@tempa\p@>\@tempb\p@
        \let\@tempa\@tempb
      \else
        \let\@tempb\@tempa
      \fi
    \fi
    \Gscale@box\@tempa[\@tempb]{\box\@tempboxa}%
  \@end@tempboxa}
%    \end{macrocode}
%  \end{macro}
%
% \begin{macro}{\ifGin@iso}
% If this flag is true, then specifying two lengths to |\resizebox|
% scales the box by the same factor in either direction, such that
% neither length \emph{exceeds} the stated amount. No user interface
% to this flag in the standard package, but it is used by the
% |keepaspectratio| key to |\includegraphics| in the \textsf{graphicx}
% package.
% \changes{v0.7e}{1995/09/29}
%     {Macro added}
%    \begin{macrocode}
\newif\ifGin@iso
%    \end{macrocode}
%  \end{macro}
%
%  \begin{macro}{\Gscale@div}
% The macro |#1| is set to the ratio of the lengths |#2| and |#3|.
% \changes{v0.7a}{1995/04/11}
%     {Add trap for division by 0}
% \changes{v0.7d}{1995/09/27}
%     {Use \cs{setlength} to support calc package}
% \changes{v1.0a}{1996/09/16}
%     {Stop infinite loop if 2nd arg zero. graphics/2259}
%    \begin{macrocode}
\def\Gscale@div#1#2#3{%
  \setlength\dimen@{#3}%
  \ifdim\dimen@=\z@
    \PackageError{graphics}{Division by 0}\@eha
    \dimen@#2%
  \fi
  \edef\@tempd{\the\dimen@}%
  \setlength\dimen@{#2}%
  \count@65536\relax
  \ifdim\dimen@<\z@
    \dimen@-\dimen@
    \count@-\count@
  \fi
  \ifdim\dimen@>\z@
    \loop
%    \end{macrocode}
%
% \changes{v1.0u}{2016/10/09}
%     {avoid overflow for small lengths eg 5sp divided by 2sp}
%    \begin{macrocode}
      \ifdim\ifnum\count@<\tw@\maxdimen\else\dimen@\fi<8192\p@
        \dimen@\tw@\dimen@
        \divide\count@\tw@
    \repeat
    \dimen@ii\@tempd\relax
    \divide\dimen@ii\count@
    \divide\dimen@\dimen@ii
  \fi
  \edef#1{\strip@pt\dimen@}}
%    \end{macrocode}
%  \end{macro}
%
% Restore Catcodes
%    \begin{macrocode}
\Gin@codes
\let\Gin@codes\relax
%    \end{macrocode}
%
%    \begin{macrocode}
%</package>
%    \end{macrocode}
%
% \Finale
%
