% \iffalse meta-comment
%
%% File: latex2e-first-aid-for-external-files.dtx (C) Copyright 2020-2023
%%
%% The LaTeX Project and any individual authors listed elsewhere
%% in this file.
%
% It may be distributed and/or modified under the
% conditions of the LaTeX Project Public License, either version 1.3c
% of this license or (at your option) any later version.
% The latest version of this license is in
%    https://www.latex-project.org/lppl.txt
% and version 1.3c or later is part of all distributions of LaTeX
% version 2008 or later.
%
% This file has the LPPL maintenance status "maintained".
%
%
%
%<*driver>
\documentclass{ltxdoc}
\EnableCrossrefs
\CodelineIndex
\begin{document}
  \DocInput{latex2e-first-aid-for-external-files.dtx}
\end{document}
%</driver>
%
% \fi
%
%
% \providecommand\pkg[1]{\texttt{#1}}
%
% \title{First aid for external files and packages that need updating}
% \author{Frank Mittelbach, \LaTeX{} Project}
%
% \maketitle
%
%
% \begin{abstract}
%    This file contains some first aid for packages or classes that
%    require updates because of internal changes to \LaTeX{} but that
%    aren't yet reflected in the package/class code.
% \end{abstract}
%
% \tableofcontents
%
% \section{Introduction}
%
%    Over the years package writers have hooked into
%    various parts of internal \LaTeX{} commands (largely because
%    proper interfaces were missing in important places) and if
%    we are now gradually adding such interfaces these internal
%    commands do change and as a result patching into them stops
%    working.
%
%    As part of making such internal changes the \LaTeX{} Project team
%    attempts to check for such usage in packages, alert the package
%    maintainers and ensures that the packages get updated alongside
%    the core \LaTeX{} system. However it is not always possible to
%    get packages that will fail with a new kernel updated in time and
%    if that is the case we try to provide a temporary fix in this
%    file for them.
%    Once the package gets updated the fix will then be removed again.
%
%    For that reason, it is put into a separate bundle so that we can
%    update it easily without requiring the CTAN maintainers to
%    install a new full \LaTeX{} system just because we take out (or add)
%    a fix for a package here.
%
%    In the best case scenario the file documented here should be
%    empty. In practice it will probably always contain one or the
%    other fix while we are waiting for the package to get updated.
%
%    \begin{quote} \textbf{Important notice:} The fixes provided here
%    are not meant to be a permanent solution, but are only provided
%    to support the transition period. They are (usually) neither
%    complete nor necessarily the best solution! Furthermore, as they
%    are done from the ``outside'', they usually add some burden and
%    slow down \LaTeX{} processing, even if the package/class is not
%    used in the document.
%
%    We will therefore remove such code as soon as possible
%    again. In practice this means that if some package never gets
%    updated/corrected, then it will eventually fail to work, because
%    after one or at most two \LaTeX{} releases we will take out the
%    transition code to ensure that this ``first aid patching''
%    doesn't get out of bounds.
%    \end{quote}
%
%  \subsection{Minor kernel fixes}
%
%    If we encounter issues with the kernel code that should get fixed
%    before the next main release we normally generate a patch release
%    for \LaTeX{}.  However, depending on the complexity of the fix we
%    might first add the fix here and generate a full patch release
%    only when a number of such issues have accumulated. This way we
%    lessen the impact on CTAN maintainers because for each tach
%    release we have to make and distribute also a matching
%    development release.
%
% \MaybeStop{\setlength\IndexMin{200pt}  \PrintIndex  }
%
%
% \section{The Implementation}
%
%    This file is meant to be loaded during format generation which is
%    why we give it the extension \texttt{.ltx}.
%    \begin{macrocode}
%<*kernel>
%    \end{macrocode}
%
%    \begin{macrocode}
\def\LaTeXFirstAidDate{2023/09/20}
\def\LaTeXFirstAidVersion{v1.1b}
%    \end{macrocode}
%
%    \begin{macrocode}
\ProvidesFile{latex2e-first-aid-for-external-files.ltx}
             [\LaTeXFirstAidDate\space \LaTeXFirstAidVersion\space
               LaTeX kernel fixes to external files and packages]
%    \end{macrocode}
%
%             
%  \begin{macro}{\FirstAidNeededT}
%    This is a very simple help to ensure that we only apply first aid
%    to an unmodified package or class. It only works in the case the
%    file has already been loaded and the csname \cs{ver@\#1.\#2} got
%    defined (holding the current date, version, and short description
%    info). We then compare its content to a frozen string and make
%    the modification \verb=#3= only if both agree. If they differ we
%    assume that the package/class in question got updated by its
%    maintainer.
%    \begin{macrocode}
\ExplSyntaxOn             
\cs_new:Npn\FirstAidNeededT#1#2#3{
  \exp_args:Ncx\str_if_eq:onF{ver@#1.#2}{#3}
      { \typeout{==>~ First~ Aid~ for~ #1.#2~ no~ longer~ applied!^^J
          \@spaces Expected:^^J
          \@spaces\@spaces #3^^J
          \@spaces but~ found:^^J
          \@spaces\@spaces \use:c{ver@#1.#2}^^J
          \@spaces so~ I'm~ assuming~ it~ got~ fixed.
      } }
  \exp_args:Ncx\str_if_eq:onT{ver@#1.#2}{#3}
}
\ExplSyntaxOff
%    \end{macrocode}
%  \end{macro}
%    
%    \begin{macrocode}
%</kernel>
%    \end{macrocode}
%
%
% \subsection{The \pkg{filehook} package first aid}
%     
%    The \pkg{filehook} package implements hooks into file loading
%    commands. These days this is already provided by the kernel
%    albeit not with the same user interface. Until that package gets
%    updated (to use the kernel interfaces) we provide a
%    substitution. This does not offer all hooks from \pkg{filehook}
%    but all that have been used in packages available in \TeX{} Live.
%
%    Note that this doesn't fix \pkg{currfile} because that package
%    uses \pkg{filehook} but relies on the internals of the old
%    implementation.
%
%    The package has now got an update so we aren't activating the
%    first aid. However, at the moment it basically bypasses the new
%    hook mechanism and puts the old hooks in thereby disabling, for
%    example, the possibility to re-order code through rules.
%
%    We therefore keep \texttt{filehook-ltx.sty} around as a guideline
%    for further updates.
%
%    Replacing \pkg{filehook} with a leaner version would then work
%    like this:
%    \begin{macrocode}
%<*kernel>
%\declare@file@substitution{filehook.sty}{filehook-ltx.sty}
%</kernel>
%    \end{macrocode}
%
%    What follows is a simplified (partial) implementation of the  \pkg{filehook}
%    interfaces.
%    Not implemented are:
%\begin{verbatim}
%  \AtBeginOfFiles      \AtEndOfFiles
%  \AtBeginOfInputs     \AtEndOfInputs
%  \AtBeginOfInputFile  \AtEndOfInputFile
%\end{verbatim}
%
%    \begin{macrocode}
%<*filehook-ltx>
%    \end{macrocode}
%
%    \begin{macrocode}
\newcommand\AtBeginOfEveryFile [1]
  {\AddToHook{file/before}{#1}}
\newcommand\AtEndOfEveryFile [1]
  {\AddToHook{file/after}{#1}}
%    \end{macrocode}
%    
%    \begin{macrocode}
\newcommand\AtBeginOfIncludes [1]
  {\AddToHook{include/before}{#1}}
\newcommand\AtEndOfIncludes [1]
  {\AddToHook{include/end}{#1}}
\newcommand\AfterIncludes [1]
  {\AddToHook{include/after}{#1}}
%    \end{macrocode}
%    
%    \begin{macrocode}
\newcommand\AtBeginOfPackages [1]
  {\AddToHook{package/before}{#1}}
\newcommand\AtEndOfPackages [1]
  {\AddToHook{package/after}{#1}}
%    \end{macrocode}
%    
%    \begin{macrocode}
\newcommand\AtBeginOfClasses [1]
  {\AddToHook{class/before}{#1}}
\newcommand\AtEndOfClasses [1]
  {\AddToHook{class/after}{#1}}
%    \end{macrocode}
%
%    \begin{macrocode}
\newcommand\AtBeginOfFile [2]
  {\AddToHook{file/#1/before}{#2}}
\newcommand\AtEndOfFile [2]
  {\AddToHook{file/#1/after}{#2}}
%    \end{macrocode}
%
%    Some commands offered a starred form  
%    \begin{macrocode}
\DeclareDocumentCommand \AtBeginOfPackageFile {smm}
   {\IfBooleanTF{#1}%
     {\@ifpackageloaded{#2}%
         {#3}%
         {\AddToHook{package/#2/before}{#3}}}%
     {\AddToHook{package/#2/before}{#3}}%
   }
%    \end{macrocode}
%    
%    \begin{macrocode}
\DeclareDocumentCommand \AtEndOfPackageFile {smm}
   {\IfBooleanTF{#1}%
     {\@ifpackageloaded{#2}%
         {#3}%
         {\AddToHook{package/#2/after}{#3}}}%
     {\AddToHook{package/#2/after}{#3}}%
   }
%    \end{macrocode}
%
%    Are the * forms here of any use? I know they are use 3--4 times
%    on CTAN but I wonder if those are real or mistaken usages.  
%    \begin{macrocode}
\DeclareDocumentCommand \AtBeginOfClassFile {smm}
   {\IfBooleanTF{#1}%
     {\@ifclassloaded{#2}%
         {#3}%
         {\AddToHook{class/#2/before}{#3}}}%
     {\AddToHook{class/#2/before}{#3}}%
   }
\DeclareDocumentCommand \AtEndOfClassFile {smm}
   {\IfBooleanTF{#1}%
     {\@ifclassloaded{#2}%
         {#3}%
         {\AddToHook{class/#2/after}{#3}}}%
     {\AddToHook{class/#2/after}{#3}}%
   }
%    \end{macrocode}
%    
%    \begin{macrocode}
\newcommand\AtBeginOfIncludeFile [2]
  {\AddToHook{include/#1/before}{#2}}
\newcommand\AtEndOfIncludeFile [2]
  {\AddToHook{include/#1/end}{#2}}
\newcommand\AfterIncludeFile [2]
  {\AddToHook{include/#1/after}{#2}}
%    \end{macrocode}
%
%    \begin{macrocode}
%</filehook-ltx>
%    \end{macrocode}
%
%    \begin{macrocode}
%<*kernel>
%    \end{macrocode}
%
% \subsection{The \pkg{dinbrief} class first aid}
%
%    Again a case of a no longer correct \cs{endgroup} in document.
%    Here the fix is simply though.
%    \begin{macrocode}
\AddToHook{file/dinbrief.cls/after}[firstaid]{%
  \FirstAidNeededT{dinbrief}{cls}{2000/03/02 LaTeX2e class}%
                  {\AddToHook{env/document/begin}{\begingroup}}%
}
%    \end{macrocode}
%
%
% \subsection{The \pkg{unicode-math} package first aid}
%
%    If \pkg{unicode-math} is used together with \pkg{doc} there is a
%    problem because it changes the mathcodes without adjusting the
%    use in \pkg{doc} that assume standard settings. Could be fixed on
%    either side, but as \pkg{unicode-math} is derivating from the
%    standard, the right place is probably a fix in this package. For
%    now we do it here. See github/820.
%    \begin{macrocode}
\AddToHook{package/unicode-math/after}{%
  \AddToHook{cmd/mod@math@codes/after}{\mathcode`\|=28796 }}
%    \end{macrocode}
%
%
%
%
% \subsection{The \pkg{pgfpages} and \pkg{pgfmorepages} first aid}
%
%    \pkg{pgfpages} alters the \cs{shipout} primitive to support
%    multiple page up scenarios. If used together with \pkg{atbegshi}
%    that worked because the alterations done by  \pkg{atbegshi} came
%    later and so used the new definition provide by
%    \pkg{pgfpages}. Now that the code from  \pkg{atbegshi} is already
%    in the kernel this further redefinition doesn't happen with the
%    result that the change to \cs{shipout} comes to late and
%    breaks the kernel processes.
%
%    \begin{macrocode}
\ExplSyntaxOn
\AddToHook{file/pgfpages.sty/after}[firstaid]{%
%    \end{macrocode}
%    Undo overwriting \cs{shipout}:
%    \begin{macrocode}
  \cs_gset_eq:NN \shipout \pgfpages@originalshipout 
%    \end{macrocode}
%    Instead overwrite the L3 programming layer name of the
%    primitive. This is really an absolute no-go, but for now the
%    simplest solution to keep the original code running.
%
%    It will be replaced when the ``configuration points'' interface
%    for \LaTeX{} becomes available. At that point the package will be
%    able to set up a different strategy for doing shipouts and
%    without the need to overrite a primitive (which it did in the
%    past and which we do below) and then this code here can be taken
%    out again.
%    \begin{macrocode}
  \cs_set_eq:NN \pgfpages@originalshipout \tex_shipout:D
  \cs_set_eq:NN \tex_shipout:D \pgfpages@interceptshipout
}
\ExplSyntaxOff
%    \end{macrocode}
%
% Same issue with \pkg{pgfmorepages} but slightly different implementation (sigh).
%
%    \begin{macrocode}
\ExplSyntaxOn
\AddToHook{file/pgfmorepages.sty/after}[firstaid]{
  \cs_set_nopar:Npn \pgfhookintoshipout {
    \cs_set_eq:NN \pgfpages@originalshipout \tex_shipout:D
    \cs_set_eq:NN \tex_shipout:D \pgfpages@interceptshipout
  }
}
\ExplSyntaxOff
%    \end{macrocode}
%
% \subsection{The \pkg{babel} package}
%
% Turn off the \pkg{babel} hack.
%    \begin{macrocode}
\AddToHook{file/babel.sty/before}[firstaid]{\def\BabelCaseHack{}}
%    \end{macrocode}
%
%
% \subsection{The \pkg{songs} package first aid}
%
%    The songs package uses \cs{obeylines} and redefines \cs{par} for special effect.
%    this no longer works in \LaTeX{} 2022-06-01 (gh issue 367). The following
%    fixes at least one failure.
%    \begin{macrocode}
\AddToHook{file/songs.sty/after}[firstaid]{%
  \FirstAidNeededT{songs}{sty}{2018/09/12 v3.1 Songs package}%
                  {%
     \renewcommand\SB@obeylines{%
         \let\obeyedline\SB@par%
         \obeylines%
         \let\@par\SB@@par%
       }}%
}
%    \end{macrocode}
%
% \subsection{The \pkg{crop} package first aid}
%
%    The crop packages fails currently due to two \LaTeX{} changes:
%    It doesn't know that \cs{stockheight} and \cs{stockwidth} 
%    are now defined, and doesn't take into account that \cs{rlap}
%    is robust (https://github.com/rrthomas/crop/issues/2).
%    The first is addressed by setting the dimension if they are zero or
%    negative. For the second we locally change the meaning of \cs{protect}
%    
%    \begin{macrocode}
\AddToHook{file/crop.sty/after}[firstaid]{%
  \FirstAidNeededT{crop}{sty}{2017/11/19 1.10 crop marks (mf)}%
                  {%
  \ifdim\stockwidth  > 0pt \else \stockwidth\paperwidth \fi  
  \ifdim\stockheight > 0pt \else \stockheight\paperheight \fi 
  \renewcommand*\CROP@genreflect[1]{%
    \leavevmode
    \dimen0\CROP@horigin
    \kern2\dimen0
    \begingroup 
    \set@typeset@protect %change protect
    \reflectbox{%
        \hb@xt@\paperwidth{%
            \vbox to\paperheight{%
                #1%
                \vss
            }%
            \hss
        }%
    }%
   \endgroup
   }
  }%
}
%    \end{macrocode}
%    \begin{macrocode}
%</kernel>
%    \end{macrocode}
%
%
%    \begin{macrocode}
%<*kernel>
%    \end{macrocode}
%
%
%
%
% \subsection{The \pkg{bigfoot} first aid}
%
%    The \pkg{bigfoot} packages makes the assumption that two
%    \cs{newinsert} allocations have a recognisable order in their
%    numbers, the second one has a lower number. This was correct in
%    the classic \TeX{} implementation but with the extended
%    allocation possibilities of all modern engines is no longer the
%    case and there is a point where the allocations take a ``jump''
%    breaking the odering assumption. These days we are fairly close
%    to that point and depending on how many packages are loaded
%    before \pkg{bigfoot} the package breaks.
%
%    This firstaid therefore jumps over the problematical point by
%    pushing the count allocation to a safe value if necessary.
%    \begin{macrocode}
\AddToHook{file/bigfoot.sty/after}{%
   \ifnum\count10<\insc@unt
     \global\count10=\insc@unt
   \fi
%    \end{macrocode}
%    We also correct a bug that \pkg{bigfoot} tries to shift mark
%    registers, but in \LaTeX{} (at least since 2015) the allocation
%    number is not 266, so it does that to a random number of mark
%    registers (which sometimes blows up depending on the value in 266).
%    \begin{macrocode}
    \def\FN@allmarks#1{\@elt{#1}%
      \ifnum#1<\count256 %<--- problem: 266 isn't the counter for marks
        \expandafter\FN@allmarks\expandafter{\number\numexpr#1+\@ne}%
        \fi}%
}
%    \end{macrocode}
%
%
%
%
% \subsection{\pkg{ulem} first aid}
%
%    In 2020 we fixed various kernel commands to accept \pkg{calc}
%    syntax. The \pkg{ulem} package redefines some internals and that
%    now conflicts with the new definitions as they involve an extra
%    group. So we alter the definition of \cs{@hspace} if \pkg{ulem}
%    was loaded. This is not perfect, obviously, so it will go out the
%    moment \pkg{ulem} gets adjusted.
%
%    \begin{macrocode}
\AddToHook{file/ulem.sty/after}[firstaid]{%
   \def\@hspace#1{\begingroup\setlength\skip@{#1}%
                  \edef\x{\endgroup\hskip\the\skip@\relax}\x}%
   }
%    \end{macrocode}
%
%
% \subsection{\pkg{varwidth} first aid}
%
%    The \pkg{varwidth} package does a lot of low-level paragraph
%    manipulation assuming traditional \TeX{} paragraphs. However, with
%    the paragraph hooks we end up with one extra glue 0pt item on the
%    vertical list and if that isn't removed then the package doesn't
%    find its penalties.
%
%    So this needs to be removed as well by adding an additional
%    \cs{unskip}.
%     
%    \begin{macrocode}
\AddToHook{file/varwidth.sty/after}[firstaid]{%
  \FirstAidNeededT{varwidth}{sty}%
       {2009/03/30 ver 0.92; \space Variable-width minipages}%
     {%
\def\@vwid@sift{%
  \skip@\lastskip\unskip
  \ifdim\lastskip=\z@\unskip\fi   % <---- the first aid here (not just unskip)
  \dimen@\lastkern\unkern
  \count@\lastpenalty\unpenalty
  \setbox\z@\lastbox
  \ifvoid\z@ \advance\sift@deathcycles\@ne \else \sift@deathcycles\z@ \fi
  \ifnum\sift@deathcycles>33 
    \let\@vwid@sift\relax
    \PackageWarning{varwidth}{Failed to reprocess entire contents}%
  \fi
  \ifnum\count@=\@vwid@preeqp \@vwid@eqmodefalse\fi
  \ifnum\count@=\@vwid@posteqp \@vwid@eqmodetrue\fi
  \ifnum\count@=\@vwid@toppen % finished
    \let\@vwid@sift\relax
  \else\ifnum\count@=\@vwid@offsets
    \@vwid@setoffsets
  \else
    \ifnum\count@=\@vwid@postw
    \else
      \@vwid@resetb % reset box \z@ or measure it
    \fi
    \@vwid@append
  \fi\fi
  \@vwid@sift}%
     }%
   }
%    \end{macrocode}
%
% \subsection{The \pkg{german} class first aid}
%
%    Handling of \cs{protected} UTF-8
%    \begin{macrocode}
\AddToHook{file/german.sty/after}[firstaid]{%
  \FirstAidNeededT{german}{sty}{1998/07/08 v2.5e Support for writing german texts (br)}
                  {%
\let\grmn@active@dq@\@active@dq
\def\@active@dq{\protect\grmn@active@dq@}%
\germanTeX
}%
}
%    \end{macrocode}
%
%    \begin{macrocode}
\AddToHook{file/ngerman.sty/after}[firstaid]{%
  \FirstAidNeededT{ngerman}{sty}{1998/07/08 v2.5e Support for writing german texts (br)}
                  {%
\let\grmn@active@dq@\@active@dq
\def\@active@dq{\protect\grmn@active@dq@}%
\ngermanTeX
}%
}
%    \end{macrocode}
%
%    \begin{macrocode}
%</kernel>
%    \end{macrocode}
% \subsection{The \pkg{underscore} first aid}
% 
% The \pkg{underscore} package makes the underscore active. This means that the
% underscore can not be used in label and references unless the package option \texttt{strings} is used
% (which patches a selection of problematic commands like \cs{label} and \cs{ref}) 
% or \pkg{babel} is used which redefines\footnote{unless the recommended option \texttt{safe=none} is used} 
% a selection of problematic commands like \cs{@testdef} or \cs{@newl@bel}.
% 
% With the new property commands the work-around do not work. We therefore make the 
% underscore protected and use \cs{ifincsname} to allow its use in csnames.
% 
%    \begin{macrocode}
%<*underscore-ltx>
\ProvidesPackage{underscore-ltx}[2023/09/20 LaTeX firstaid to make underscore protected ]
\begingroup
 \catcode`\_=\active
 \protected\gdef _{%   
   \ifincsname % 
      \string_%
   \else
      \ifx\protect\@typeset@protect
         \ifmmode \sb \else \BreakableUnderscore \fi
      \else
         \ifx\protect\@unexpandable@protect \noexpand_%
         \else \protect_%
      \fi\fi
    \fi
    }
  \global\let\ActiveUnderscore=_
\endgroup
%</underscore-ltx>
%    \end{macrocode}
%    \begin{macrocode}
%<*kernel>
\AddToHook{file/underscore.sty/after}[firstaid]{%
  \FirstAidNeededT{underscore}{sty}{2006/09/13}{\RequirePackage{underscore-ltx}}}
%</kernel>
%    \end{macrocode}
%
% \Finale
%
