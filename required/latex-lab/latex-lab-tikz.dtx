% \iffalse meta-comment
%
%% File: latex-lab-tikz.dtx (C) Copyright 2023 LaTeX Project
%
% It may be distributed and/or modified under the conditions of the
% LaTeX Project Public License (LPPL), either version 1.3c of this
% license or (at your option) any later version.  The latest version
% of this license is in the file
%
%    https://www.latex-project.org/lppl.txt
%
%
% The development version of the bundle can be found below
%
%    https://github.com/latex3/latex2e/required/latex-lab
%
% for those people who are interested or want to report an issue.
%
\def\ltlabtikzdate{2024-10-09}
\def\ltlabtikzversion{0.80a}
%<*driver>
\documentclass{l3doc}
\EnableCrossrefs
\CodelineIndex
\begin{document}
  \DocInput{latex-lab-tikz.dtx}
\end{document}
%</driver>
%
% \fi
%
% \providecommand\tikzname{Ti\emph {k}Z}
% \title{The \textsf{latex-lab-tikz} package\\
% Support for the tagging of \tikzname\ pictures }
% \author{\LaTeX{} Project\thanks{Initial implementation done by Ulrike Fischer}}
% \date{v\ltlabtikzversion\ \ltlabtikzdate}
%
% \maketitle
%
% \newcommand{\xt}[1]{\textsl{\textsf{#1}}}
% \newcommand{\TODO}[1]{\textbf{[TODO:} #1\textbf{]}}
% \newcommand{\docclass}{document class \marginpar{\raggedright document class
% customizations}}
% \providecommand\hook[1]{\texttt{#1}}
% 
% \NewDocElement[printtype=\textit{socket},idxtype=socket,idxgroup=Sockets]{Socket}{socketdecl}
% \NewDocElement[printtype=\textit{hook},idxtype=hook,idxgroup=Hooks]{Hook}{hookdecl}
% \NewDocElement[printtype=\textit{plug},idxtype=plug,idxgroup=Plugs]{Plug}{plugdecl}
%

%
% \begin{abstract}
% \end{abstract}
%
% \section{Introduction}
%
% Tagging of \tikzname\ pictures is non trivial. 
% 
% At first such pictures have various purposes:
% \begin{itemize}
% \item They can be purely ornamental and decorative, e.g. some page border. This should normally 
% be tagged as artifact. 
% \item They can show a illustrative figure, similar to png graphics included with
% \cs{includegraphics}. This should normally be tagged as a Figure structure with alternative text
% and some of the content should perhaps be an artifact again (but not all content,
% as some PDF viewer ignore empty structures).
% \item They can be meant as normal text. For example the todonotes package uses a
% \tikzname\ picture to surround the text in a node with some colored background. 
% In this case the text should tagged e.g. as an Aside. 
% \item They can represent a symbol. Then we want to tag as Span or Figure structure element
% with an /ActualText or
% perhaps even simply in the stream with a SPAN-BDC with an /ActualText.
% \item and naturally there can be all sort of mixtures of these elements.
% \end{itemize}
% 
% At second \tikzname\ pictures uses lots of boxes and moves them around 
% and that makes is not easy to get the tagging right -- 
% at least with pdflatex where one has to 
% insert the literals at the right time.
% 
% At third in some cases, e.g., when the \tikzname\ picture is tagged as Figure, 
% one normally should calculate the BBox. 
% This is currently done with some low-level hacking into the pgf code.
% 
% The following is a first try to tag at least some of the \tikzname\ pictures.
% It is incomplete and should be used with care. Resulting structures and contents should
% be checked!
% 
% The main idea of the implementation is to use socket that allow to change the
% purpose of the \tikzname\ picture. This must be done before the actual environment as
% \tikzname\ processes the keys too late to allow to do this in the optional argument.
% 
% \subsection{Tagging recipes}
% 
% As \tikzname\ pictures have so varied purposes there are a number of 
% \enquote{tagging recipes}. Currently the recipe must be set before the 
% \tikzname\ picture with \verb+\tagpdfsetup{graphic/tagging=+\meta{recipe name}\verb+}+. 
% The following recipes exist:
% 
% \begin{description}
% \item[figure] This is the default receipe. 
% It surrounds the picture with a \texttt{Figure} tag and adds a BBox.
% Inside the figure tagging is suspended. 
% Such a figure should have an alternative text which describes the content. This alternative
% text can be set with the \texttt{alt} key:
% \begin{verbatim}
% \begin{tikzpicture}[alt=A duck]
% \duck
% \end{tikzpicture}
% \end{verbatim}
% This recipe is meant for meaningful pictures. 
% 
% \item[text] This surrounds the graphical parts with an artifact MC and activates
% tagging on node texts. It is meant for small pictures containing text in a node
% that should be part of the text flow, e.g. a todo.
% 
% \item[artifact] This marks the picture as an artifact. This is meant for 
% decorations. 
% 
% \item[symbol] This is meant for pictures where the drawing should 
% represent a single symbol. Such pictures should then add an actualtext:
% \begin{verbatim}
% \begin{tikzpicture}[actualtext=A]
% % drawing of a A
% \end{tikzpicture}
% \end{verbatim}
% 
% \end{description}
% \section{Todos}
% \begin{enumerate}
% \item 
% 
% \end{enumerate}
% 
% \begin{implementation}
% \section{Implementation}
%    \begin{macrocode}
%<*package>
%<@@=tag>
%    \end{macrocode}
%    \begin{macrocode}
\ProvidesExplPackage {latex-lab-testphase-tikz} {\ltlabtikzdate} {\ltlabtikzversion}
  {Code related to the tagging of tikz pictures}
%    \end{macrocode}
%
% \subsection{Sockets}
% 
% \begin{socketdecl}{tagsupport/tikzpicture/begin,tagsupport/tikzpicture/end}
% Sockets at the begin and the end of a tikzpicture
%    \begin{macrocode}
\NewSocket{tagsupport/tikzpicture/begin}{0}
\NewSocket{tagsupport/tikzpicture/end}{0}
%    \end{macrocode}
% \end{socketdecl}
% 
% \begin{socketdecl}{tagsupport/tikzpicture/textbegin,tagsupport/tikzpicture/textend}
% Sockets at the end and begin of text parts. 
%    \begin{macrocode}
\NewSocket{tagsupport/tikzpicture/textbegin}{0}
\NewSocket{tagsupport/tikzpicture/textend}{0}
%    \end{macrocode}
% \end{socketdecl}
% 
% \begin{socketdecl}{tagsupport/tikzpicture/alt,tagsupport/tikzpicture/actualtext}
% This socket takes as argument an alt text and will add it to the Figure structure.
%    \begin{macrocode}
\NewSocket{tagsupport/tikzpicture/alt}{1}
\NewSocket{tagsupport/tikzpicture/actualtext}{1}
%    \end{macrocode}
% \end{socketdecl}
% 
% \subsection{Variables}
% 
% \begin{variable}{\l__tag_tikzpicture_usetext_bool}
% We can not rely on tagging sockets to activate tagging in the text parts 
% as they do nothing if tagging is suspended so we use a boolean.
%    \begin{macrocode}
\bool_new:N\l__tag_tikzpicture_usetext_bool
%    \end{macrocode}
% \end{variable}
% 
% \subsection{Plugs}
% 
% \begin{plugdecl}{text (tagsupport/tikzpicture/begin),text (tagsupport/tikzpicture/end)}
% This plug handles the \tikzname\ picture as a text object. So the graphical parts
% are tagged as artifact, but when we encounter a node we activate tagging there.
% There is no Bbox.
%    \begin{macrocode}
\NewSocketPlug{tagsupport/tikzpicture/begin}{text}
 {
   \ifvmode
    {
     \UseTaggingSocket{para/begin}  %check 
    }     
   \fi   
   \tag_mc_end_push: 
   \tagmcbegin{artifact}
   \bool_set_true:N\l__tag_tikzpicture_usetext_bool
%    \end{macrocode}
% We hook into two pgf commands to add the tagging code.
% They are only used for postscript and svg so it should be
% safe inside a tagging socket for now.
% TODO: ask for an interface.
%    \begin{macrocode}
   \def\pgfsys@begin@text
    {
      \bool_if:NT\l__tag_tikzpicture_usetext_bool
        {\ResumeTagging{\tikzpicture}} 
      \tag_socket_use:n{tikzpicture/textbegin}
    }
   \def\pgfsys@end@text
    {
      \tag_socket_use:n{tikzpicture/textend}
      \bool_if:NT\l__tag_tikzpicture_usetext_bool
        {\SuspendTagging{\tikzpicture} } 
    }       
 }
\NewSocketPlug{tagsupport/tikzpicture/end}{text}
 {
   \tagmcend
   \tag_mc_begin_pop:n{}
 } 
%    \end{macrocode}
% \end{plugdecl}
% 
% \begin{plugdecl}{figure (tagsupport/tikzpicture/begin),figure (tagsupport/tikzpicture/end)}
% This plug handles the \tikzname\ picture as a figure. 
% Around the graphic is a \texttt{Figure} environment which will
% use an alt text given in the optional argument and internally tagging is suspended. 
% The Bbox will be set (after the second compilation) to the size of the bounding box.
%    \begin{macrocode}
\NewSocketPlug{tagsupport/tikzpicture/begin}{figure}
 {
   \ifvmode
    {
     \tag_socket_use:n{para/begin}   
    }     
   \fi
   \tag_mc_end_push: 
   \tag_struct_begin:n{tag=Figure,
    alt=a tikz figure%
    }
   \bool_set_false:N\l__tag_tikzpicture_usetext_bool
   \pgfrememberpicturepositiononpagetrue
   \tag_mc_begin:n{tag=Figure}
 } 
 
\NewSocketPlug{tagsupport/tikzpicture/end}{figure}
 {
   \tag_mc_end:   
   \cs_set:Npn\pgfqpoint##1##2
    {         
      \dim_to_decimal_in_bp:n {##1+ \pgf@picminx}
      \c_space_tl
      \dim_to_decimal_in_bp:n {##2+ \pgf@picminy}
      \c_space_tl
      \dim_to_decimal_in_bp:n {##1+ \pgf@picmaxx}
      \c_space_tl
      \dim_to_decimal_in_bp:n {##2+ \pgf@picmaxx}          
    }   
   \cs_if_exist:cT { pgf@sys@pdf@mark@pos@pgfid\the\pgf@picture@serial@count }
    {
      \__tag_prop_gput:cne
        { g__tag_struct_ \g__tag_struct_stack_current_tl _prop }
        { A } 
        {
           << 
             /O /Layout /BBox~
             [
               \use:c 
                { pgf@sys@pdf@mark@pos@pgfid\the\pgf@picture@serial@count }
             ]
           >>   
        }
     }
   \tag_struct_end:
   \tag_mc_begin_pop:n{}
  }  
%    \end{macrocode}
% \end{plugdecl}
% 
% \begin{plugdecl}{symbol (tagsupport/tikzpicture/begin),symbol (tagsupport/tikzpicture/end)}
% This plug handles the \tikzname\ picture as a symbol. 
% It tags the content as a Span and expects and actual text.
% Internally tagging is suspended. 
%    \begin{macrocode}
\NewSocketPlug{tagsupport/tikzpicture/begin}{symbol}
 {
   \ifvmode
    {
     \tag_socket_use:n{para/begin}   
    }     
   \fi
   \tag_mc_end_push: 
   \tag_struct_begin:n{tag=Span}
   \bool_set_false:N\l__tag_tikzpicture_usetext_bool
   \tag_mc_begin:n{}
 } 
 
\NewSocketPlug{tagsupport/tikzpicture/end}{symbol}
 {
   \tag_mc_end:   
   \tag_struct_end:
   \tag_mc_begin_pop:n{}
  }  
%    \end{macrocode}
% \end{plugdecl}
% 
% \begin{plugdecl}{artifact (tagsupport/tikzpicture/begin),artifact (tagsupport/tikzpicture/end)}
% This plug handles the \tikzname\ picture as a artifact, as decoration. 
% So it is surrounded by an artifact MC and internal text do not restart tagging.
%    \begin{macrocode}
\NewSocketPlug{tagsupport/tikzpicture/begin}{artifact}
 {
   \ifvmode
    {
     \tag_socket_use:n{para/begin}   
    }     
   \fi
   \tag_mc_end_push: 
   \bool_set_false:N\l__tag_tikzpicture_usetext_bool
   \tag_mc_begin:n{artifact}
 } 
 
\NewSocketPlug{tagsupport/tikzpicture/end}{artifact}
 {
   \tag_mc_end:
   \tag_mc_begin_pop:n{}
  }  
%    \end{macrocode}
% \end{plugdecl}
%  By default we use the figure plugs
%    \begin{macrocode}
\AssignSocketPlug{tagsupport/tikzpicture/begin}{figure}
\AssignSocketPlug{tagsupport/tikzpicture/end}{figure}
%    \end{macrocode}
%
% We add the sockets to the environment and then suspend tagging.
% TODO: check how to handle the command and if we can use some inner environment
%    \begin{macrocode}
\AddToHook{cmd/pgfpicture/before}
 {
  \tag_socket_use:n{tikzpicture/begin}
  \tag_suspend:n{\tikzpicture}
 }

\AddToHook{cmd/endpgfpicture/after}
 {
  \tag_resume:n{\tikzpicture}
  \tag_socket_use:n{tikzpicture/end}
 }
%    \end{macrocode}
%
% \begin{plugdecl}{text (tagsupport/tikzpicture/textbegin),
%  text (tagsupport/tikzpicture/textend)}
% The text plugs in end the artifact mc and restart if after the text. 
%    \begin{macrocode}
\NewSocketPlug{tagsupport/tikzpicture/textbegin}{text}  
 {
  \tag_mc_end:   
  \tag_mc_begin:n{}       
 }
\NewSocketPlug{tagsupport/tikzpicture/textend}{text}  
 {
  \tag_mc_end:
  \tag_mc_begin:n{artifact}
 } 
\AssignSocketPlug{tagsupport/tikzpicture/textbegin}{text}
\AssignSocketPlug{tagsupport/tikzpicture/textend}{text} 
%    \end{macrocode}
% \end{plugdecl}
%
% \begin{plugdecl}{default (tagsupport/tikzpicture/alt)}
% This setups the alt key.
%    \begin{macrocode}
\NewSocketPlug{tagsupport/tikzpicture/alt}{default}
 { \keys_set:nn { __tag / struct }{alt=#1} }
\AssignSocketPlug{tagsupport/tikzpicture/alt}{default}
%    \end{macrocode}
% \end{plugdecl}
% 
% \begin{plugdecl}{default (tagsupport/tikzpicture/actualtext)}
% This setups the actualtext key.
%    \begin{macrocode}
\NewSocketPlug{tagsupport/tikzpicture/actualtext}{default}
 { \keys_set:nn { __tag / struct }{actualtext=#1} }
\AssignSocketPlug{tagsupport/tikzpicture/actualtext}{default}
%    \end{macrocode}
% \end{plugdecl}
% 
% \subsection{Hooking into \tikzname}
%    \begin{macrocode}
\AddToHook{package/tikz/after}
 {
%    \end{macrocode}
% at first we add an alt and an actual key.
% As keys are processed rather late inside the picture and tagging can be suspended
% we use them always. TODO: check if we can process the key list in the sockets.
%    \begin{macrocode}
   \tikzset{alt/.code={\socket_use:nn{tagsupport/tikzpicture/alt}{#1}}}
   \tikzset{actualtext/.code={\socket_use:nn{tagsupport/tikzpicture/actualtext}{#1}}}
 }
%    \end{macrocode} 
%
% \subsection{tagpdf keys to switch the recipes}
% 
%    \begin{macrocode}
\keys_define:nn { __tag / setup }
 {
   graphic/tagging .choice:,
   graphic/tagging/figure .code:n = 
    {
      \AssignSocketPlug{tagsupport/tikzpicture/begin}{figure}
      \AssignSocketPlug{tagsupport/tikzpicture/end}{figure}
    },
   graphic/tagging/text .code:n = 
    {
      \AssignSocketPlug{tagsupport/tikzpicture/begin}{text}
      \AssignSocketPlug{tagsupport/tikzpicture/end}{text}
    }, 
   graphic/tagging/symbol .code:n = 
    {
      \AssignSocketPlug{tagsupport/tikzpicture/begin}{symbol}
      \AssignSocketPlug{tagsupport/tikzpicture/end}{symbol}
    },     
   graphic/tagging/artifact .code:n = 
    {
      \AssignSocketPlug{tagsupport/tikzpicture/begin}{artifact}
      \AssignSocketPlug{tagsupport/tikzpicture/end}{artifact}
    },     
 }
%    \end{macrocode}
%
% Handle todonotes. TODO This perhaps should go into firstaid instead
%    \begin{macrocode}
\AddToHook{package/todonotes/after}
{
\NewSocket{tagsupport/todonotes/todo}{0}
\NewSocketPlug{tagsupport/todonotes/todo}{default}
 {\tagpdfsetup{graphic/tagging=text}}
\AssignSocketPlug{tagsupport/todonotes/todo}{default}
% 

\renewcommand{\todo}[2][]{%
  % Needed to output any dangling \item of a noskip section (see #36):
  \if@inlabel \leavevmode \fi
  \if@noskipsec \leavevmode \fi
  \if@todonotes@inlinepar
    \ifhmode
      \@bsphack
      \@todonotes@vmodefalse
    \else
      \@savsf\@m
      \@savsk\z@
      \@todonotes@vmodetrue      
    \fi
     {\UseTaggingSocket{todonotes/todo}\@todo[#1]{#2}}%
    \@esphack%
    \if@todonotes@vmode \par \fi
  \else%    
    {\UseTaggingSocket{todonotes/todo}\@todo[#1]{#2}}%
  \fi}
}  
%
%    \end{macrocode}
%    \begin{macrocode}
%</package>
%    \end{macrocode}
%    \begin{macrocode}
%<*latex-lab>
\ProvidesFile{tikz-latex-lab-testphase.ltx}
        [\ltlabtikzdate\space v\ltlabtikzversion\space 
         latex-lab wrapper tikz]

\RequirePackage{latex-lab-testphase-tikz}

%</latex-lab>
%    \end{macrocode}
% \end{implementation}
