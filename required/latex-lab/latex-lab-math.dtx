% \iffalse meta-comment
%
%% File: latex-lab-math.dtx
%
% Copyright (C) 2022-2023 The LaTeX Project
%
% It may be distributed and/or modified under the conditions of the
% LaTeX Project Public License (LPPL), either version 1.3c of this
% license or (at your option) any later version.  The latest version
% of this license is in the file
%
%    https://www.latex-project.org/lppl.txt
%
%
% The development version of the bundle can be found below
%
%    https://github.com/latex3/latex2e/required/latex-lab
%
% for those people who are interested or want to report an issue.
%
%
\def\ltlabmathdate{2023-11-12}
\def\ltlabmathversion{0.5f}
%
%<*driver>
\documentclass{l3doc}
\EnableCrossrefs
\CodelineIndex 

\usepackage{todonotes}

\begin{document}
  \DocInput{latex-lab-math.dtx}
\end{document}
%</driver>
%
% \fi
%
%
% \title{The \texttt{latex-lab-math} code\thanks{}}
% \author{Frank Mittelbach, Joseph Wright, \LaTeX{} Project}
% \date{v\ltlabmathversion\ \ltlabmathdate}
% 
% \maketitle
%
% \newcommand\NEW[1]{\marginpar{\mbox{}\hfill\fbox{New: #1}}}
% \providecommand\class[1]{\texttt{#1.cls}}
% \providecommand\pkg[1]{\texttt{#1}}
%
% \providecommand\hook[1]{\texttt{#1\DescribeHook[noprint]{#1}}}
% \providecommand\socket[1]{\texttt{#1\DescribeSocket[noprint]{#1}}}
% \providecommand\plug[1]{\texttt{#1\DescribePlug[noprint]{#1}}}
%
% \NewDocElement[printtype=\textit{socket},idxtype=socket,idxgroup=Sockets]{Socket}{socketdecl}
% \NewDocElement[printtype=\textit{hook},idxtype=hook,idxgroup=Hooks]{Hook}{hookdecl}
% \NewDocElement[printtype=\textit{plug},idxtype=plug,idxgroup=Plugs]{Plug}{plugdecl}
%
% ^^A \car {...} for marginal comments
% ^^A \car*{...} for longer inline comments
%
% \NewDocumentCommand\car{sO{}m}
%   {\IfBooleanTF{#1}{\todo[inline,color=blue!10,#2]{#3}}^^A
%                    {\todo[color=blue!10,#2]{#3}}}
%
% \NewDocumentCommand\fmi{sO{}m}
%   {\IfBooleanTF{#1}{\todo[inline,#2]{#3}}^^A
%                    {\todo[#2]{#3}}}
%
% \begin{abstract}
%    This is an experimental prototype. It captures math material
%    (basically okay, but the interfaces for packages aren't yet
%    there) and tags the material (which is not yet anywhere near the
%    final state). That part is provided for experimentation and to
%    gather feedback, etc.
% \end{abstract}
%
% \tableofcontents
%
% \section{Introduction}
% \car*{Todo: update all the documentation! Both here and 
%   (what little there is!) in the implementation section.}
%
% Tagging math involves a variety of tasks that require that math is captured before the
% typesetting
% \begin{itemize} 
% \item When typesetting the math MC-tags and structure commands must 
% be inserted at the begin and the end, and perhaps also around lines 
% or other subparts of the equation.   
% \item The source and/or a mathml-representation of the source must be available
% so that it can be (perhaps after some preprocessing) be used in an associated file
% or in an alternate text
% \item It must be possible to measure the math for e.g. a bbox setting. 
% \end{itemize}
% 
% This file implements capture of all math mode material at the outer
% level, i.e., a formula is captured in its entirety with inner text
% blocks (possibly containing further math) absorbed as part of the
% formula. For example,
%\begin{verbatim}
%      \[ a \in A \text{ for all $a<5$}  \]
%\end{verbatim}
% would only result in a single capture of the tokens
% ``\verb*/a \in A \text{ for all $a<5$}/''.
% 
%
% \section{Math capture} \label{sec:mathcapture}
% In the current setup 
%  \begin{itemize}
%   \item |$|, |\(...\)| and |$$| grab (through a command in \cs{everymath}/cs{everydisplay})
%   if the boolean \cs{l_@@_collected_bool} is false.
%   If the boolean is true they behave normally and can for example contain verbatim.
%   
%   \item All (registered) environments grab their body 
%    regardless of the state of the boolean. For
%    |equation|, |equation*| and |math| this is a change as they no longer can 
%    contain verbatim.
%   
%   \item BUG: |\[...\]| grabs if \cs{l_@@_collected_bool} is false. If it is
%   true it falls back to |equation*| and then errors because this can't find the end.   
%  \end{itemize} 
%  
% \subsection{Code level interfaces}
%
% \begin{function}{\math_register_env:n, \math_register_env:nn}
%   \begin{syntax}
%     \cs{math_register_env:n} \Arg{env}
%     \cs{math_register_env:nn} \Arg{env} \Arg{options}
%   \end{syntax}
%   Registers the \meta{env} as a math environment which should be captured
%   and made available. This is necessary for all top-level math mode
%   environments: low-level errors may result if these are not correct
%   set up. One or more key--value \meta{options} may also be given:
%   \begin{itemize}
%     \item[\texttt{arg-spec}] The argument specification taken by the
%       beginning of the environment; this is used to remove non-mathematical
%       material.
%   \end{itemize}
% \end{function}
%
% \begin{function}{\math_processor:n}
%   \begin{syntax}
%     \cs{math_processor:n} \Arg{tokens}
%   \end{syntax}
%   Declares that the captured math content should be passed to the
%   \meta{tokens}, which will receive the environment type as |#1| and
%   the content as |#2|. The processing is done before the typesetting. It is not
%   applied if \cs{ifmeasuring@} is true.
% \end{function}
%
% \subsection{Document level interfaces}
%
% \begin{function}{\RegisterMathEnvironment}
%   \begin{syntax}
%     \cs{RegisterMathEnvironment} \oarg{options} \Arg{env}
%   \end{syntax}
%   Registers the \meta{env} as a math environment which should be captured
%   and made available. This is necessary for all top-level math mode
%   environments: low-level errors may result if these are not correct
%   set up. One or more key--value \meta{options} may also be given:
%   \begin{itemize}
%     \item[\texttt{arg-spec}] The argument specification taken by the
%       beginning of the environment; this is used to remove non-mathematical
%       material.
%   \end{itemize}
% \end{function}
% 
% \section{Math tagging}
% 
% The tagging code has to handle 
% \begin{itemize}
% \item the embedding into the surrounding. This means 
%   \begin{itemize} 
%     \item closing and reopening MC-chunks 
%     \item closing and reopening text/P-structures
%     \item handling interferences of the tagging code with penalties and spacing.
%   \end{itemize}
% \item the actual tagging which means to do some or all of the following tasks:
%   \begin{itemize}
%     \item setup content for an associated source file
%     \item setup content for an associated mathml file 
%     \item setup content for the /Alt key
%     \item setup content for the /ActualText key
%     \item setup attributes
%     \item add associated files     
%     \item add a Formula structure
%     \item surround subparts (e.g. lines) with Formula sub structures
%          (perhaps with their own set of additional content)
%     \item surround elements of the equation with mathml structure elements 
%          (currently only luatex with luamml)
%    \end{itemize}
%  \end{itemize}
%  
%  \subsection{Inline math}
%  
%  The embedding code is added through
%  the sockets 
%   \begin{itemize}
%    \item |tagsupport/math/inline/begin|                
%    \item |tagsupport/math/inline/end|
%   \end{itemize}  
%  The sockets simply push and pop the MC currently. Without
%  tagging they use the noop-plug. 
%  
%  The actual tagging is in done through the sockets
%   \begin{itemize}
%    \item |tagsupport/math/inline/formula/begin|
%     This socket takes the math as argument and its code
%     should output it for typesetting. Without tagging the 
%     socket uses the identity plug.
%     The |default| plug of the socket calls this three internal sockets
%     for the tagging support:
%      \begin{itemize}
%      \item |tagsupport/math/content| This should set up the various
%      content variables (empty variables are ignored by the structure code
%       and so can be used to suppress a setting).
%      \item |tagsupport/math/struct/begin| This calls \cs{tag_struct_begin:n}.
%      It should also write the associated files if needed.
%      \item |tagsupport/math/substruct/begin| this handles subparts.
%      TODO: does it really make sense in inline math to have that?? 
%      \end{itemize}
%    \item |tagsupport/math/inline/formula/end|
%      This socket ends the formula structure(s). The |default|
%      plug calls these internal sockets:
%       \begin{itemize}
%        \item |tagsupport/math/substruct/end|
%        \item |tagsupport/math/struct/end|      
%       \end{itemize}
%   \end{itemize} 
%    
%    
%   
% \section{Known current bugs, etc.}
%     \car*{New Section, now with subsections.\\
%      As indicated, these lists are probably incomplete.\\
%      Some of these have been addressed in a more recent branch.} 
%
% \subsection{Capture/grabbing problems}
%  
% \begin{enumerate}
%   \item Incorrect grabbing of |$|-math when there is also 
%      explicit |$|-math within a \textit{text environment} 
%      that is itself within the math that should all be grabbed.
%      For example,
%       \begin{verbatim}
%        $a\begin{minipage}{1cm}$b$\end{minipage}$
%       \end{verbatim}
%      would only result in the capture of the tokens
%      ``\verb*/a\begin {minipage}{1cm}/''. 
%      This can be avoided by an additional brace group:
%       \begin{verbatim}
%        $a{\begin{minipage}{1cm}$b$\end{minipage}}$
%       \end{verbatim} 
%      
%   \item Similar incorrect grabbing with |$$| also.
%   
%   \item The grabbing, for all the display environments (and |\) \]|), needs
%       to deal with nesting: \pkg{amsmath} contains code for this. 
%        
%   \item The math can't contain verbatim and verbatim-like commands. This is 
%   nothing new for the \pkg{amsmath} environments but changes |$| and |\[\]|
%   and |equation*| (see e.g. tagging-project issue \#30).    
%   
%   \item Begin and end of the math or math environment can not be hidden in commands.
%   For example \verb+>{$}l<{$}+ in a tabular would lead to errors. 
%   Defining |\[| to fall back to |equation*| doesn't work if |equation*| is 
%   a grabbing environment.
%   
%   \item The behaviour of |\[...\]| is faulty. See above.
% \end{enumerate} 
%
% \subsection{Fake math}
% In a number of places in \LaTeX{} math commands (mainly |$|) is used 
% only for technical reason, e.g. to access a math font, to setup a symbol
% or to use \cs{vcenter}. 
% 
% The code identifies such fake math mostly by making use of the \cs{m@th} command
% where two methods are used for the automatic detection:
% 
% \begin{itemize}
%  \item After grabbing math content the code checks if the content contains the token 
%  \cs{m@th} and if yes it doesn't call the processor before reinserting
%  the content and perhaps adding tagging code.
%  This method requires that the math can be grabbed (e.g. that the end dollar is visible)
%  and that the \cs{m@th} is visible. It applies for example in \cs{@iiiparbox} where the
%  code from |$\vcenter| to |\m@th$| is grabbed an put back. It does not work for
%  example for |tabular| where the dollars and the \cs{m@th} token are spread around
%  over three commands. |tabular| needs therefore manual intervention. 
%  
%  A look in the list of usages (in \texttt{usage-of-m@th.md}) justifies this approach.
%  All usages are either no math at all, or related to small elements that probably
%  shouldn't be grabbed and processed on their own. 
%
%  \item \cs{m@th} is redefined so that it sets the boolean \cs{l_@@_collected_bool}
%  to true. If \cs{m@th} is used inside math that has been grabbed
%  this doesn't change much as the boolean is set by the grabbing anyway. For usages
%  outside math the benefit is not so clear: The setting avoids that in \cs{LaTeXe}
%  the epsilon is processed as math, but it also prevents that the content of the amsmath
%  command \cs{boxed} is processed as math. 
%  It means that if one wants to reenable math processing inside some (fake) math
%  one has to do it after \cs{m@th} calls.
% \end{itemize}
% 
%  Open problems
%  \begin{enumerate}
%   \item The grabbing code doesn't pass the info that it detected a \cs{m@th} token.
%   This means that the tagging code has to do the same check (and doesn't do this
%   in all cases yet).
%   
%   \item Commands are missing to locally disable the grabbing and processing, e.g.
%   to handle |tabular|.
%   
%   \item It must be checked if setting the boolean in \cs{m@th} really makes sense
%   or if commands like \cs{LaTeXe} should be handled manually. 
%           
%  \end{enumerate}
%  
% \subsection{Processor}
%  
% The grabbed math is at first passed to the processor. The processor is not called
% in a measuring phase (from the amsmath \cs{ifmeasuring@}) and if the \cs{m@th} 
% token is detected. 
% It is not quite clear what purpose the processor has. As it is a public interface
% it can't be used for internal code. And typesetting happens later and the processor
% can't really change this. Currently it is mostly used for debugging and messages.
% If the \cs{m@th} is found the \cs{l_@@_fakemath_bool} is set, so if the code
% is changed this must be preserved.
%  
% \subsection{Other problems}
% 
% \begin{enumerate} 
%   \item 
%      The presence of \cs{m@th} in association with \cs{ensuremath}
%      does not necessarily indicate fakemath.  This is because  
%      wanting mathsurround to be zero is very reasonable and common, 
%      \emph{even when the math is genuine} (and hence needs to be collected).
%      
%      TODO: this claim needs some examples. 
%      
%   \item User-defined environments can create problems; but this area, of 
%      new, copied and changed environments, has not yet been developed. 
% 
%  \car*{Joseph wrote, inter alia:\\
%      My thinking [regarding] \cs{RegisterMathEnvironment}\\
%    - (New) Math environments should not be created-then-patched, but only
%    generated by a [(future)] dedicated command (\cs{DeclareMathEnvironment}, 
%    presumably)\\
%    - Math environments created with \pkg{ltcmd} [commands] should not be copied, . . .\\
%    - Package authors should be able to manually set up math environments with a public boolean.}
% \end{enumerate} 
%
%
% \subsection{Other ToDos}
%
% \begin{enumerate}
%  \item Add (some of) the math display commands that were \enquote{lifted from 
%    plain}, e.g., \cs{displaylines} \cs{eqalign}(??).
%  \item The breqn packages changes catcodes and that isn't yet covered
%    by our mechanism.
% \end{enumerate}
%
%
% \car*{\cs{MaybeStop} (temporarily) not executed, as it is unknown on Chris' system.}
% \iffalse
%  \MaybeStop{\setlength\IndexMin{200pt}  \PrintIndex  }
% \else
%  \StopEventually{\setlength\IndexMin{200pt}  \PrintIndex  }
% \fi
%
% \section{The Implementation}
%
%    \begin{macrocode}
%<@@=math>
%    \end{macrocode}
%
%    \begin{macrocode}
%<*kernel>
%    \end{macrocode}
%
% \subsection{File declaration}
%
%    \car{Change description here?}
%    \begin{macrocode}
\ProvidesFile{latex-lab-math.ltx}
             [\ltlabmathdate\space
              v\ltlabmathversion\space
              Grab all the math(s) and tag it (experiments)]
%    \end{macrocode}
%
%    Temp loading \ldots
%    \begin{macrocode}
\AddToHook{begindocument/before}{\RequirePackage{latex-lab-testphase-block}}
%    \end{macrocode}
%
%    \begin{macrocode}
\ExplSyntaxOn
%    \end{macrocode}
%
% \subsection{Setup}
%
% Loading \pkg{amsmath} is an absolute requirement: this avoids needing to
% have conditional definitions and deals with how to define \cs{[}/\cs{]}
% neatly.
%    \begin{macrocode}
\AddToHook{begindocument/before}{ \RequirePackage { amsmath } }
%    \end{macrocode} 
%
%
% \subsection{Data structures}
%
% \begin{variable}{\l_@@_collected_bool}
%   Tracks whether math mode material has been collected, which happens inside
%   \pkg{amsmath} environments as well as those handled directly here.
%   If true following math will not grab and/or process. 
%   See \ref{sec:mathcapture} for details.
%    \begin{macrocode}
\bool_new:N \l_@@_collected_bool
%    \end{macrocode}
% \end{variable}
% 
% \begin{variable}{\l_@@_fakemath_bool}
%   Tracks whether math mode material has been identified as fake math during
%   the grabbing phase, which happens currently if the 
%   grabbed contents contains the \cs{m@th} token.
%
%    \begin{macrocode}
\bool_new:N \l_@@_fakemath_bool
%    \end{macrocode}
% \end{variable}
%
%
%  \car{Change first tl name below: `env' $=>$ `info'?\\
%        Or do we need an extra storage tl?}
%
% \begin{variable}{\g_@@_grabbed_env_tl, \g_@@_grabbed_math_tl}
% \cs{g_@@_grabbed_env_tl} contains the name of the math environment
% (\texttt{math} in the case of inline math,
% \cs{g_@@_grabbed_math_tl} the math content.
%    \begin{macrocode}
\tl_new:N \g_@@_grabbed_env_tl
\tl_new:N \g_@@_grabbed_math_tl
%    \end{macrocode}
% \end{variable}
% 
% \begin{variable}{\l_@@_tmpa_tl,\l_@@_tmpa_skip}
% Temporary variables
%    \begin{macrocode}
\tl_new:N \l_@@_tmpa_tl
\skip_new:N \l_@@_tmpa_skip
%    \end{macrocode}
% \end{variable}
%
% \begin{variable}{\l_@@_mathcontent_alt_tl,
%  \l_@@_mathcontent_actual_tl,
%  \l_@@_mathcontent_AF_tl}
% Temporary variables to hold math content that should 
% be used in actual or alt text and stored as AF. 
%    \begin{macrocode}
\tl_new:N \l_@@_mathcontent_alt_tl
\tl_new:N \l_@@_mathcontent_actual_tl
\tl_new:N \l_@@_mathcontent_AF_source_tl
\tl_new:N \l_@@_mathcontent_AF_mathml_tl
%    \end{macrocode}
% \end{variable}
% 
% \subsection{Sockets}
% \subsubsection{Main inline math sockets}
% 
% \begin{socketdecl}
%   {
%     tagsupport/math/inline/begin,
%     tagsupport/math/inline/end,
%     tagsupport/math/inline/formula/begin,
%     tagsupport/math/inline/formula/end,
%   }
%   The first two sockets are meant to embed inline
%   math into the surrounding (so to close/reopen e.g. MC-chunks).
%   The other two implement the actual formula structure.                 
%    \begin{macrocode}
\socket_new:nn {tagsupport/math/inline/begin}{0}
\socket_new:nn {tagsupport/math/inline/end}{0}
\socket_new:nn {tagsupport/math/inline/formula/begin}{1} %
\socket_new:nn {tagsupport/math/inline/formula/end}{0}
%    \end{macrocode}
%\end{socketdecl}
% 
% 
% \begin{plugdecl}{MC}
%    \begin{macrocode}
\socket_new_plug:nnn
  {tagsupport/math/inline/begin}
  {MC}
  {\tag_mc_end_push:}
\socket_new_plug:nnn
  {tagsupport/math/inline/end}
  {MC}
  {\tag_mc_begin_pop:n{}}  
%    \end{macrocode}
% \end{plugdecl}
% 
% We probably will want to test different tagging receipes.
% \begin{plugdecl}{default}
%    \begin{macrocode}
\socket_new_plug:nnn
  {tagsupport/math/inline/formula/begin}
  {default}
  {  
    \socket_use:n{tagsupport/math/content}
    \socket_use:n{tagsupport/math/struct/begin}
    % inner formula if multiple parts (not really implemented yet)
    \socket_use:n{tagsupport/math/substruct/begin}
    #1
  }
\socket_new_plug:nnn
  {tagsupport/math/inline/formula/end}
  {default}
  {  
    \socket_use:n{tagsupport/math/substruct/end}
    \socket_use:n{tagsupport/math/struct/end}
  }  
%    \end{macrocode}
% \end{plugdecl}
% 
% \subsubsection{Main display math sockets}
% 
% \begin{socketdecl}
%   {
%     tagsupport/math/display/begin,
%     tagsupport/math/display/end,
%     tagsupport/math/display/formula/begin,
%     tagsupport/math/display/formula/end,
%   }
%   The first two sockets are meant to embed display
%   math into the surrounding (so to close/reopen e.g. MC-chunks and
%   P-structure).
%   The other two implement the actual formula structure.                 
%    \begin{macrocode}
\socket_new:nn {tagsupport/math/display/begin}{0}
\socket_new:nn {tagsupport/math/display/end}{0}
\socket_new:nn {tagsupport/math/display/formula/begin}{1} %
\socket_new:nn {tagsupport/math/display/formula/end}{0}
%    \end{macrocode}
%\end{socketdecl}

% \begin{plugdecl}{default}
%    \begin{macrocode}
\socket_new_plug:nnn
  {tagsupport/math/display/begin}
  {default}
  { \@kernel@close@P }
\socket_new_plug:nnn
  {tagsupport/math/display/end}
  {default}
  {}  
%    \end{macrocode}
% \end{plugdecl}


% \begin{plugdecl}{default}
%    \begin{macrocode}
\socket_new_plug:nnn
  {tagsupport/math/display/formula/begin}
  {default}
  { 
   \@kernel@math@begin
   #1 
  }
\socket_new_plug:nnn
  {tagsupport/math/display/formula/end}
  {default}
  {  
  }  
%    \end{macrocode}
% \end{plugdecl}
%
% \subsubsection{Internal sockets}
% \begin{socketdecl}{tagsupport/math/content}
% The math content is stored in associated files and used for 
% actual and alternative text. As the exact text is still
% unclear we use a socket to be able to test variants.
% The socket should set all four tl vars above, if needed
% to identical values. It can use the two variables 
% \cs{g_@@_grabbed_env_tl} and \cs{g_@@_grabbed_math_tl}
%    \begin{macrocode}
\socket_new:nn {tagsupport/math/content}{0}
%    \end{macrocode}
% \end{socketdecl}
% 
% \begin{plugdecl}{actual+source}
% TODO: think about some naming convention ...
%    \begin{macrocode}
\socket_new_plug:nnn
  {tagsupport/math/content}
  {actual+source}
  {
   \tl_set:Ne\l_@@_mathcontent_actual_tl
    {
       LaTeX~ formula~ starts~
       \exp_not:N\begin{\g_@@_grabbed_env_tl}
       \c_space_tl
       \exp_not:V\g_@@_grabbed_math_tl
       \c_space_tl
       \exp_not:N\end{\g_@@_grabbed_env_tl}
       \c_space_tl LaTeX~ formula~ ends~
   }   
   \tl_set_eq:NN \l_@@_mathcontent_AF_source_tl \l_@@_mathcontent_actual_tl
   \tl_set:Nn    \l_@@_mathcontent_AF_mathml_tl {}
   \tl_set:Nn    \l_@@_mathcontent_alt_tl    {}
  }
\socket_assign_plug:nn {tagsupport/math/content}{actual+source}  
%    \end{macrocode}
% \end{plugdecl}  
% 
% \begin{socketdecl}{tagsupport/math/struct/begin,
%  tagsupport/math/struct/end}
% For the main structure we use a socket too. 
% This allow e.g. to create a special one for luamml
% which setups additional objects. 
% The begin socket can use the two variables 
% \cs{g_@@_grabbed_env_tl} and \cs{g_@@_grabbed_math_tl}
%    \begin{macrocode}
\socket_new:nn {tagsupport/math/struct/begin}{0}
\socket_new:nn {tagsupport/math/struct/end}{0}
%    \end{macrocode}
% \end{socketdecl}
%
% \begin{plugdecl}{default}
% TODO: think about some naming convention ...
%    \begin{macrocode}
\socket_new_plug:nnn
  {tagsupport/math/struct/begin}
  {default}
  {
   \tag_struct_begin:n
     {
       tag=Formula,
       AFinline-o  = \l_@@_mathcontent_AF_source_tl,
       title-o     = \g_@@_grabbed_env_tl,
       actualtext  = \l_@@_mathcontent_actual_tl,
       alt         = \l_@@_mathcontent_alt_tl
     }
  }
\socket_new_plug:nnn
  {tagsupport/math/struct/end}
  {default}
  { \tag_struct_end: }
  
\socket_assign_plug:nn {tagsupport/math/struct/begin}{default}  
\socket_assign_plug:nn {tagsupport/math/struct/end}{default}   
%    \end{macrocode}  
% \end{plugdecl}   
% 
% \begin{socketdecl}{tagsupport/math/substruct/begin,
%  tagsupport/math/substruct/end}
%  This holds the code to handle subparts of the formula.
%    \begin{macrocode}
\socket_new:nn {tagsupport/math/substruct/begin}{0}
\socket_new:nn {tagsupport/math/substruct/end}{0}
%    \end{macrocode}
% \end{socketdecl}
% 
% \begin{plugdecl}{default}
%    \begin{macrocode}
\socket_new_plug:nnn
  {tagsupport/math/substruct/begin}
  {default}
  { \grabaformulapartandstart }
\socket_new_plug:nnn
  {tagsupport/math/substruct/end}
  {default}
  { 
    \tagmcend
    \if@subformulas
      \tagstructend
    \fi
 } 
\socket_assign_plug:nn {tagsupport/math/substruct/begin}{default}  
\socket_assign_plug:nn {tagsupport/math/substruct/end}{default}   
%    \end{macrocode}  
% \end{plugdecl}  

% 
% 
% 
%  \begin{macro}{\__tag_math_disable:}
%  Similar to the table code we collect the plugs that should be
%  assigned to do nothing if we don't want tagging
%    \begin{macrocode}
\cs_new_protected:Npn \__tag_math_disable: 
  {
    \socket_assign_plug:nn {tagsupport/math/inline/begin}{noop}
    \socket_assign_plug:nn {tagsupport/math/inline/end}{noop}
    \socket_assign_plug:nn {tagsupport/math/inline/formula/begin}{identity}
    \socket_assign_plug:nn {tagsupport/math/inline/formula/end}{noop}
    \socket_assign_plug:nn {tagsupport/math/display/begin}{noop}
    \socket_assign_plug:nn {tagsupport/math/display/end}{noop}
    \socket_assign_plug:nn {tagsupport/math/display/formula/begin}{identity}
    \socket_assign_plug:nn {tagsupport/math/display/formula/end}{noop}   
  } 
%    \end{macrocode}
% \end{macro}
% 
%  \begin{macro}{\__tag_math_enable:}
%  Similar to the table code we collect the default plugs that should be
%  assigned if we want tagging
%    \begin{macrocode}
\cs_new_protected:Npn \__tag_math_enable: 
  {
    \socket_assign_plug:nn {tagsupport/math/inline/begin}{MC}
    \socket_assign_plug:nn {tagsupport/math/inline/end}{MC}
    \socket_assign_plug:nn {tagsupport/math/inline/formula/begin}{default}
    \socket_assign_plug:nn {tagsupport/math/inline/formula/end}{default}
    \socket_assign_plug:nn {tagsupport/math/display/begin}{default}
    \socket_assign_plug:nn {tagsupport/math/display/end}{default}
    \socket_assign_plug:nn {tagsupport/math/display/formula/begin}{default}
    \socket_assign_plug:nn {tagsupport/math/display/formula/end}{default}
  } 
%    \end{macrocode}
% \end{macro}
% At begin document we can activate:
%    \begin{macrocode}
\AtBeginDocument{\tag_if_active:T{\__tag_math_enable: }}
%    \end{macrocode}
%
% \subsection{Interface commands}
%
% \begin{macro}
%   {\@@_process:nn, \@@_process:Vn, \@@_process_auxi:nn, \@@_process_auxii:nn}
%   A no-op place-holder; the internal wrapper means that it does not need to
%   be concerned with internals.
%    \begin{macrocode}
\cs_new_protected:Npn \@@_process:nn #1#2
  {
    \legacy_if:nF { measuring@ }
      {
        \tl_if_in:nnTF {#2} { \m@th }
          { \bool_set_true:N\l_@@_fakemath_bool }
          { \tl_trim_spaces_apply:nN {#2} \@@_process_auxi:nn {#1} }
      }
  }
\cs_generate_variant:Nn \@@_process:nn { V }
\cs_new_protected:Npn \@@_process_auxi:nn #1#2
  {
    \tl_gset:Nn \g_@@_grabbed_env_tl {#2}
    \tl_gset:Nn \g_@@_grabbed_math_tl {#1}
    \@@_process_auxii:nn {#2} {#1}
  }
\cs_new_protected:Npn \@@_process_auxii:nn #1#2 { }
%    \end{macrocode}
% \end{macro}
%
% \begin{macro}{\math_processor:n}
%   A simple installer
%    \begin{macrocode}
\cs_new_protected:Npn \math_processor:n #1
  { \cs_set_protected:Npn \@@_process_auxii:nn ##1##2 {#1} }
%    \end{macrocode}
% \end{macro}
%
% \subsection{Content grabbing}
%
% \begin{macro}{\@@_grab_dollar:w}
%   Grab up to a single |$|, for inline math mode, suppressing
%   any processing if the token is \tn{m@th} found in the content.
%
%    \fmi{what's that test doing?}
%
%    \car{It is some kind of fix, to avoid the remote
%      possibility that the math is empty, making the code
%      produce an unwanted \texttt{\$\$}.}
%
%\car{cf.~the code for this in \cs{@ensuredmath}}
%
%\car{It is harmless but unnecessary in the dollardollar grabbing below.}
%
%    \begin{macrocode}
\cs_new_protected:Npn \@@_grab_dollar:w % $
  #1 $
  {
%    \end{macrocode}
%    \fmi{what's that test doing?}
%    \begin{macrocode}
    \tl_if_blank:nF {#1}
      {
        \@@_process:nn { math } {#1} % $
%    \end{macrocode}
%  We do not want math tagging in fakemath or when measuring,
%    \begin{macrocode}
      \bool_lazy_or:nnTF 
        {\legacy_if_p:n { measuring@ }}
        { \l_@@_fakemath_bool }
        { #1 $ }
        {
            \socket_use:n {tagsupport/math/inline/begin} %end P-MC  
            \socket_use:nn{tagsupport/math/inline/formula/begin}{#1}
             $
            \socket_use:n {tagsupport/math/inline/formula/end}
            \socket_use:n {tagsupport/math/inline/end} % restart P-MC
        }         
      }
  }
%    \end{macrocode}
% \end{macro}
% 
% \begin{macro}{\@@_grab_dollardollar:w}
%   And for the classical \TeX{} display structure.
%    \begin{macrocode}
\cs_new_protected:Npn \@@_grab_dollardollar:w % $$
  #1 $$
  {
    \tl_if_blank:nF {#1}
      {
        \@@_process:nn { equation* } {#1}
        \socket_use:n {tagsupport/math/display/begin}
        \socket_use:nn{tagsupport/math/display/formula/begin}{#1}
        $$
      }
  }
%    \end{macrocode}
% To allow to use the code without tagging we guard. But probably tagpdf should
% provide some tools for such manual para-ends.
%    \begin{macrocode}
\cs_new_protected:Npn \@kernel@close@P {
   \tag_if_active:T
    {
     \tag_mc_end: %end P-chunk, should perhaps be \tag_mc_end_push: ...
        \__tag_gincr_para_end_int:
        \__tag_check_para_end_show:nn{red}{} %debug: show para
        \tag_struct_end:
    }
}

\cs_new_protected:Npn \@@_tag_dollardollar_display_end: {
%  \typeout{== tag dollarldollar display end}
%  \ShowTagging{struct-stack}
 % \tagpdfparaOff
  \para_raw_end:
  \tagpdfparaOn
  \l_@@_tmpa_skip \lastskip
  \@kernel@math@end
  \penalty \postdisplaypenalty
%    \end{macrocode}
% This reinserts the below display skips. It must be doubled to
% get the right amount:
%    \begin{macrocode}
  \skip_vertical:n { -\l_@@_tmpa_skip * 2 }
%
  \@doendpe             % this has no \end{...} to take care of it
}

%    \end{macrocode}
% \end{macro}
%
%
% \begin{macro}{\@@_grab_inline:w}
%   Collect inline math content and deal with the need to move to math mode.
%    \begin{macrocode}
\cs_new_protected:Npn \@@_grab_inline:w % \(
  #1 \)
  {
    \tl_if_blank:nF {#1}
      {
        $ #1 $
      }
    \bool_set_false:N \l_@@_collected_bool
  }
%    \end{macrocode}
% \end{macro}
%
% \begin{macro}{\@@_grab_eqn:w}
%   For the most common use of \cs{[}/\cs{]}: turn into an environment.
%    \begin{macrocode}
\cs_new_protected:Npn \@@_grab_eqn:w % \[
  #1 \]
   {
%     \typeout{collected? = \bool_if:NTF \l_@@_collected_bool {true}{false}}
     \begin { equation* } #1 \end { equation* }
   }
%    \end{macrocode}
% \end{macro}
%
% \subsection{Marking math environments}
%
% A general mechanism for math mode environments that do not grab their
% content (\emph{cf.}~most \pkg{amsmath} environments).
%
% \begin{variable}{\l_@@_env_name_tl}
%  To allow us to carry out \enquote{special effects}
%    \begin{macrocode}
\tl_new:N \l_@@_env_name_tl
%    \end{macrocode}
% \end{variable}
%
% Here we set up specialised handling of environments. The idea for the
% \texttt{arg-spec} key is that if an environment takes arguments, we
% don't worry during the main grabbing. Rather, we remove the arguments
% from the grabbed content and forward only the payload. That is done by
% (ab)using \pkg{ltcmd}.
%    \begin{macrocode}
\keys_define:nn { @@ }
  {
     arg-spec .code:n =
       {
         \ExpandArgs { c } \DeclareDocumentCommand
           { @@_env \l_@@_env_name_tl _aux: }
           {#1}
           { \@@_env_forward:w }
       }
  }
%    \end{macrocode}
%
% \begin{macro}{\math_register_env:nn}
% \begin{macro}{\math_register_env:n}
% \begin{macro}{\RegisterMathEnvironment}
%   Set up to capture environment content and make available.
%
%    \begin{macrocode}
\cs_new_protected:Npn \math_register_env:nn #1#2
  {
    \tl_set:Nn \l_@@_env_name_tl {#1}
    \keys_set:nn { @@ } {#2}
    \cs_gset_eq:cc { @@_env_ #1 _begin: } {#1}
    \cs_gset_eq:cc { @@_env_ #1 _end: } { end #1 }
%
    \ExpandArgs { nne } \RenewDocumentEnvironment {#1} { b }
      {
%        \bool_set_true:N \exp_not:N \l_@@_collected_bool
%        \cs_if_exist:cTF { @@_env #1 _aux: }
%          {
%            \exp_not:c { @@_env #1 _aux: }
%              ##1 \exp_not:N \@@_env_end: {#1}
%          }
%          { \exp_not:N \@@_process:nn {#1} {##1} }
        \exp_not:N \bool_if:NTF \exp_not:N \l_@@_collected_bool
          {
%            \typeout{===>B1}
          }
          {
%            \typeout{===>B2}
            \cs_if_exist:cTF { @@_env #1 _aux: }
              {
                \exp_not:c { @@_env #1 _aux: }
                  ##1 \exp_not:N \@@_env_end: {#1}
              }
              { \exp_not:N \@@_process:nn {#1} {##1} }
            \exp_not:n { \@kernel@math@registered@begin }
            \bool_set_true:N \exp_not:N \l_@@_collected_bool
          }
%        \exp_not:N \tracingall
        \exp_not:c { @@_env_ #1 _begin: }
        ##1
        \exp_not:c { @@_env_ #1 _end: }
%        \exp_not:c { @@_env_ #1 _end: }
%        \exp_not:N \tracingnone
%        \exp_not:n { \@kernel@math@registered@end }
     }
      {
      }
  }

\cs_new_protected:Npn \math_register_halign_env:nn #1#2
  {
    \tl_set:Nn \l_@@_env_name_tl {#1}
    \keys_set:nn { @@ } {#2}
    \cs_gset_eq:cc { @@_env_ #1 _begin: } {#1}
    \cs_gset_eq:cc { @@_env_ #1 _end: } { end #1 }
%
    \ExpandArgs { nnee } \RenewDocumentEnvironment {#1} { b }
      {
%        \bool_set_true:N \exp_not:N \l_@@_collected_bool
%        \cs_if_exist:cTF { @@_env #1 _aux: }
%          {
%            \exp_not:c { @@_env #1 _aux: }
%              ##1 \exp_not:N \@@_env_end: {#1}
%          }
%          { \exp_not:N \@@_process:nn {#1} {##1} }
        \exp_not:N \bool_if:NTF \exp_not:N \l_@@_collected_bool
          {
%            \typeout{===>B1}
          }
          {
%            \typeout{===>B2}
            \cs_if_exist:cTF { @@_env #1 _aux: }
              {
                \exp_not:c { @@_env #1 _aux: }
                  ##1 \exp_not:N \@@_env_end: {#1}
              }
              { \exp_not:N \@@_process:nn {#1} {##1} }
            \exp_not:n { \@kernel@math@registered@begin }
            \bool_set_true:N \exp_not:N \l_@@_collected_bool
          }
%        \exp_not:N \tracingall
        \exp_not:c { @@_env_ #1 _begin: }
        ##1
%        \exp_not:c { @@_env_ #1 _end: }
%        \exp_not:N \tracingnone
     }
      {
        \exp_not:c { @@_env_ #1 _end: }
      }
  }

\cs_new_protected:Npn \math_register_odd_env:nn #1#2
  {
    \tl_set:Nn \l_@@_env_name_tl {#1}
    \keys_set:nn { @@ } {#2}
    \cs_gset_eq:cc { @@_env_ #1 _begin: } {#1}
    \cs_gset_eq:cc { @@_env_ #1 _end: } { end #1 }
%
    \ExpandArgs { nnee } \RenewDocumentEnvironment {#1} { b }
      {
        \exp_not:N \bool_if:NTF \exp_not:N \l_@@_collected_bool
          {
%            \typeout{===>B1}
          }
          {
%            \typeout{===>B2}
            \cs_if_exist:cTF { @@_env #1 _aux: }
              {
                \exp_not:c { @@_env #1 _aux: }
                  ##1 \exp_not:N \@@_env_end: {#1}
              }
              { \exp_not:N \@@_process:nn {#1} {##1} }
            \exp_not:n { \@kernel@math@registered@begin }
            \bool_set_true:N \exp_not:N \l_@@_collected_bool
          }
%        \exp_not:N \tracingall
        \exp_not:c { @@_env_ #1 _begin: }
        ##1
     }
      {
        \exp_not:c { @@_env_ #1 _end: }
% needed if we don't have $$...$$
%        \exp_not:n { \typeout{---> @kernel@math@registered@end }}
        \exp_not:n { \@kernel@math@registered@end }
      }
  }


%  FMi: compare with block change!
%
%  \DeclareRobustCommand*\begin[1]{%
%  \UseHook{env/#1/before}%
%  \@ifundefined{#1}%
%    {\def\reserved@a{\@latex@error{Environment #1 undefined}\@eha}}%
%    {\def\reserved@a{\def\@currenvir{#1}%
%        \edef\@currenvline{\on@line}%
%        \@execute@begin@hook{#1}%
%        \csname #1\endcsname}}%
%  \@ignorefalse
%  \begingroup
%  \@endpefalse  % tmp!!! is it ok to drop this here?
%  \reserved@a}


\cs_new:Npn \@kernel@math@registered@begin {
%  \ShowTagging{struct-stack}
%\typeout{==>A1}\ShowTagging{struct-stack,mc-current}
  \mode_if_vertical:TF
       {
%         \legacy_if:nTF { @endpe }
%           { \legacy_if_set_false:n { @endpe } }
%           { \__block_list_beginpar_vmode: }
%
%         \typeout{==>~ at:~ \g__tag_struct_tag_tl}
%
        \tag_if_active:T
          {
            \exp_args:Noo\str_if_eq:nnF \g__tag_struct_tag_tl { \l__tag_para_main_tag_tl }    % needs correction!
             {
%               \typeout{==>A2}
               \__block_beginpar_vmode:
             }              % needs correction!
          }
       }
       {
%         \typeout{==>A3}
         \@kernel@close@P
%         \tagmcend                      % needs correction!
       }
  \@kernel@math@begin
  \tagpdfparaOff
%  \typeout{==>MC1}\ShowTagging{mc-current}
}

\cs_new:Npn \@kernel@math@registered@end {
%  \typeout{==>MC2}\ShowTagging{mc-current}
  \para_raw_end:
  \tagpdfparaOn
  \@kernel@math@end
%  \typeout{==>MC3}\ShowTagging{mc-current}
  \@endpetrue
}

\cs_new_protected:Npn \math_register_env:n #1
  { \math_register_env:nn {#1} { } }
\NewDocumentCommand \RegisterMathEnvironment { O{} m }
  { \math_register_env:nn {#2} {#1} }
%    \end{macrocode}
% \end{macro}
% \end{macro}
% \end{macro}
%
% \begin{macro}{\@@_env_forward:w}
%    \begin{macrocode}
\cs_new_protected:Npn \@@_env_forward:w #1 \@@_env_end: #2
  { \@@_process:nn {#2} {#1} }
%    \end{macrocode}
% \end{macro}
%
% \subsection{Document commands}
%
% \car*{Add one more here: \texttt{displaymath}, which
%        is equivalent to \cs{[} , \cs{]}\\
%        and hence to the basic \texttt{equation*}.\\
%       Added in more recent branch.}
%
% \begin{macro}
%   {\equation, \@@_equation_begin:, \equation*, \@@_equation_star_begin:}
% \begin{macro}
%   {\endequation, \@@_equation_end:, \endequation*, \@@_equation_star_end:}
%   These environments are not set up by \pkg{amsmath} to collect their body,
%   so we do that here. This has to be done \emph{after} we can be sure
%   \pkg{amsmath} is loaded.
%
% \car*{Note that with \pkg{amsmath} loaded, \texttt{equation*} and \texttt{equation}\\
%        are the two basics: they are used to define the other single-row\\
%        display environments, etc.}
%
%    \begin{macrocode}
\tl_gput_right:Nn \@kernel@before@begindocument
  {
    \math_register_env:n { equation }
    \math_register_env:n { equation* }
% at the moment register_env can only do display math
%    \math_register_env:n { math }
    \RenewDocumentEnvironment{math} {b}{$#1$}{}
% and this one doesn't work either
%    \math_register_env:n { displaymath }
    \RenewDocumentEnvironment{displaymath} {b}{\[#1\]}{}
  }
%    \end{macrocode}
% \end{macro}
% \end{macro}
%
% \begin{macro}{\(, \)}
%  If math mode has not been collected, we need to  do that; otherwise, worry
%  about whether we are in math mode or not. The closing command here can only
%  occur inside a collected math block: otherwise it will be simply used as
%  a delimiter.
%    \begin{macrocode}
\cs_gset_protected:Npn \( % \)
  {
    \bool_if:NTF \l_@@_collected_bool
      {
        \mode_if_math:TF
          { \@badmath }
          { $ }
      }
      {
        \@@_grab_inline:w
      }
  } % \(
\cs_gset_protected:Npn \)
  {
    \mode_if_math:TF
      { $ }
      { \@badmath }
  }
%    \end{macrocode}
% \end{macro}
%
% \begin{macro}{\[, \]}
%   Again, we need to watch for when \pkg{amsmath} is loaded after this code.
%   The flag usage here is to cover the case where \cs{[}/\cs{]} is hidden
%   inside another environment. In this case the grabbing happens on
%    the outer level and should not be repeated.
%    \begin{macrocode}
\tl_gput_right:Nn \@kernel@before@begindocument
  {
    \cs_gset_protected:Npn \[ % \]
       {
        \bool_if:NTF \l_@@_collected_bool
          { \begin { equation* } }
          { \@@_grab_eqn:w }
      } % \[
    \cs_gset_protected:Npn \]
      {
        \bool_if:NTF \l_@@_collected_bool
          { \end{ equation* } }
          { \@badmath }
      }
  }
%    \end{macrocode}
% \end{macro}
%
%
% \begin{macro}{\ensuremath}
%   A bit of nesting fun to make sure we collect only if required.
%   \fmi{why does ensuremath need handling at all?}
%
%   \car{Indeed!  Currently, this is setup to process the math that
%     it has anyways already captured as its argument; thus it is more
%     efficient than leaving the capture to be repeated by the \cs{everymath}}
%
%    \begin{macrocode}
%\cs_gset_protected:Npn \ensuremath #1
%  {
%    \mode_if_math:TF
%      {#1}
%      {
%        \bool_if:NTF \l_@@_collected_bool
%          { \@ensuredmath {#1} }
%          {
%            \bool_set_true:N \l_@@_collected_bool
%            \@@_process:nn { math } {#1}
%            \@ensuredmath {#1}
%            \bool_set_false:N \l_@@_collected_bool
%          }
%      }
%  }
%    \end{macrocode}
% \end{macro}
%
% \subsection{\cs{everymath} and \cs{everydisplay}}
%
% The business end for grabbing inline math and \enquote{raw} \TeX{}
% display. Most display math mode is actually handled elsewhere, as we
% have macro control.
%    \begin{macrocode}

\def\@kernel@math@begin {
%  \typeout{==>~math~begin}
% needs different handling if we support nesting
  \socket_use:n{tagsupport/math/content}
  \socket_use:n{tagsupport/math/struct/begin}
% inner formula if multiple parts (not really implemented yet)
  \socket_use:n{tagsupport/math/substruct/begin}
% the above does:
%    \tagstructbegin{tag=Formula}\tagmcbegin{}
%  or just
% \tagmcbegin{}
}
\def\@kernel@math@end {
%  \typeout{==>~math~end}
%  \ShowTagging{struct-stack}
  \socket_use:n{tagsupport/math/substruct/end}
  \socket_use:n{tagsupport/math/struct/end}
%  \ShowTagging{struct-stack}
}

\exp_args:No \tex_everymath:D
  {
    \tex_the:D \tex_everymath:D
    \bool_if:NF \l_@@_collected_bool
      {
        \bool_set_true:N \l_@@_collected_bool
        \@@_grab_dollar:w
      }
  }

\exp_args:No \tex_everydisplay:D
  {
    \tex_the:D \tex_everydisplay:D
    \iftrue  % this may have to be a settable flag!
%      {
%        \typeout{==>~ in~ everydisplay}
%    \end{macrocode}
% flipping the \cs{belowdisplay} values is done so that we get (assumption)
% a negative skip and not make the page bigger then we take that out,
% then we add the tagging code (in \cs{@@_tag_dollardollar_display_end} ) and
% then we put a real \cs{postdisplaypenalty} in and
% the right skip (of which we don't know if it is short or a
% normal \cs{belowdisplayskip}). This might need some refinement if that skip
% is actually negative from the start
% (not sure it ever is and is worth bothering about)
%    \begin{macrocode}
        \skip_set:Nn \belowdisplayskip      {-\belowdisplayskip}
        \skip_set:Nn \belowdisplayshortskip {-\belowdisplayshortskip}
        \int_set:Nn \postdisplaypenalty {10000}
%
        \group_insert_after:N \@@_tag_dollardollar_display_end:
%      }
    \fi
    \bool_if:NF \l_@@_collected_bool
      {
        \bool_set_true:N \l_@@_collected_bool
        \@@_grab_dollardollar:w
      }
  }
%    \end{macrocode}
%
% \subsection{Modifying kernel environments}
%
%   We need to cover this even though it is, of course, not encouraged.
%    \begin{macrocode}
\math_register_env:n { eqnarray }
\math_register_env:n { eqnarray* }
%    \end{macrocode}
%
% Places where math mode is (ab)used.
%    \begin{macrocode}
\clist_map_inline:nn
  { tabular , tabular* }
  {
    \AddToHook{ env / #1 / begin }
      { \bool_set_true:N \l_@@_collected_bool }
  }
%    \end{macrocode}
%
% \begin{macro}{\@@_m@th:, \m@th}
%   Handle non-math use of math mode. At present nesting isn't supported as
%   \cs{m@th} pops up in a few places that \emph{are} math mode!
%    \begin{macrocode}
\cs_new_eq:NN \@@_m@th: \m@th
\cs_gset_protected:Npn \m@th
  {
    \bool_set_true:N \l_@@_collected_bool
    \@@_m@th:
  }
%    \end{macrocode}
% \end{macro}
%
% \subsection{Modifying \pkg{amsmath}}
%
% \begin{macro}{\@@_amsmath_align@:nn}
% \begin{macro}
%   {
%     \@@_amsmath_gather@:n ,
%     \@@_amsmath_multline@:n
%   }
% \begin{macro}{\align@}
% \begin{macro}{\gather@, \multline@}
%   Mark up all of the display environments as the content is captured anyway.
%   We then use an internal macro in each environment type to insert the
%   processing code. Each of these is slightly different, so we cannot use a
%   simple loop here. The test for \cs{split@tag} is required as the
%   \texttt{split} environment internally uses \texttt{gather} \emph{when not
%   within an \pkg{amsmath}} environment, for example inside \texttt{equation}.
%   Without the precaution, we'd get two copies of the grabbed math, the second
%   of which would start with \cs{split@tag}.
%    \begin{macrocode}



\tl_gput_right:Nn \@kernel@before@begindocument {
%
\renewenvironment{gather*}{%
  \start@gather\st@rredtrue
}
{%
% this redirection doesn't work if we alter "gather"!
  %  \endgather
% so replace it with its real meaning
  \math@cr \black@\totwidth@ \egroup
  $$\ignorespacesafterend
}
%    \end{macrocode}
%
%    \begin{macrocode}
\def\common@align@ending {
  \math@cr \black@\totwidth@
  \egroup
  \ifingather@
    \restorealignstate@
    \egroup
    \nonumber
    \ifnum0=`{\fi\iffalse}\fi
  \else
    $$%
  \fi
  \ignorespacesafterend
}
\renewenvironment{alignat}{%
  \start@align\z@\st@rredfalse
}{%
  \common@align@ending
}
\renewenvironment{alignat*}{%
  \start@align\z@\st@rredtrue
}{%
  \common@align@ending
}
\renewenvironment{xalignat}{%
  \start@align\@ne\st@rredfalse
}{%
  \common@align@ending
}
\renewenvironment{xalignat*}{%
  \start@align\@ne\st@rredtrue
}{%
  \common@align@ending
}
\renewenvironment{xxalignat}{%
  \start@align\tw@\st@rredtrue
}{%
  \common@align@ending
}
\renewenvironment{align}{%
  \start@align\@ne\st@rredfalse\m@ne
}{%
  \common@align@ending
}
\renewenvironment{align*}{%
  \start@align\@ne\st@rredtrue\m@ne
}{%
  \common@align@ending
}
\renewenvironment{flalign}{%
  \start@align\tw@\st@rredfalse\m@ne
}{%
  \common@align@ending
}
\renewenvironment{flalign*}{%
  \start@align\tw@\st@rredtrue\m@ne
}{%
  \common@align@ending
}
%
\renewenvironment{multline*}{\start@multline\st@rredtrue}
{%
  \iftagsleft@ \@xp\lendmultline@ \else \@xp\rendmultline@ \fi
  \ignorespacesafterend
}
%    \end{macrocode}
%    Also for "false?"
%    \begin{macrocode}
\def\measuring@true{\let\ifmeasuring@\iftrue\tag_stop:}
%    \end{macrocode}
%
%    \begin{macrocode}
%
  \math_register_halign_env:nn {align}{}
  \math_register_halign_env:nn {align*}{}
  \math_register_halign_env:nn {flalign}{}
  \math_register_halign_env:nn {flalign*}{}
  \math_register_halign_env:nn {gather}{}
  \math_register_halign_env:nn {gather*}{}
  \math_register_halign_env:nn {multline}{}
  \math_register_halign_env:nn {multline*}{}
  \math_register_halign_env:nn {xalignat}{}
  \math_register_halign_env:nn {xalignat*}{}
  \math_register_halign_env:nn {xxalignat}{}
  %
  \@namedef{maketag @ @ @} #1{%
%    \typeout{--->maketag @ @ @}
    \ifmeasuring@
      \hbox{\m@th\normalfont#1}%
    \else
      \tagmcend \tagstructbegin{tag=Lbl}%
      \tagmcbegin{tag=Lbl}%
      \hbox{\m@th\normalfont#1}%
      \tagmcend \tagstructend \tagmcbegin{}%
    \fi
  }
%    \end{macrocode}
%
%
%    \begin{macrocode}
\@namedef{math@cr @ @ @ gather}{%
    \ifst@rred\nonumber\fi
   &\relax
    \make@display@tag
%
    \maybestartnewformulatag
%
    \ifst@rred\else\global\@eqnswtrue\fi
    \global\advance\row@\@ne
    \cr
}
%    \end{macrocode}
%
%    \begin{macrocode}
\@namedef{math@cr @ @ @ align}{%
  \ifst@rred\nonumber\fi
  \if@eqnsw \global\tag@true \fi
  \global\advance\row@\@ne
  \add@amps\maxfields@
  \omit
  \kern-\alignsep@
  \iftag@
    \setboxz@h{\@lign\strut@{\make@display@tag}}%
    \place@tag
  \fi
%
    \maybestartnewformulatag
%
  \ifst@rred\else\global\@eqnswtrue\fi
  \global\lineht@\z@
  \cr
}
%    \end{macrocode}
%
%    \begin{macrocode}
\def\restore@math@cr{\@namedef{math@cr @ @ @}{
%
    \maybestartnewformulatag
%
    \cr}}
\restore@math@cr
%    \end{macrocode}
%
%    \begin{macrocode}
}
%    \end{macrocode}
% \end{macro}
% \end{macro}
% \end{macro}
% \end{macro}
%
%
%

%  \begin{macro}{\@@_split_at_nl:NN}
%  This splits grabbed math at newlines. 
%    
%    \begin{macrocode}
\cs_new:Npn \@@_split_at_nl:NN #1#2 {
  \tl_set:Nf \l_@@_tmpa_tl {
      \exp_after:wN \@@_split_at_nl_first:w #1 \\ \q_nil \\ \s_stop }
  \exp_after:wN \@@_split_at_nl_aux:nnNN \l_@@_tmpa_tl #1 #2
}
%    \end{macrocode}
% and the auxiliary commands
%    \begin{macrocode}
\cs_new:Npn \@@_split_at_nl_first:w #1 \\ #2 \\ #3 \s_stop
  {
    \quark_if_nil:nTF {#2}
      { {#1} {  } }
      {
        \@@_split_chk_if_begin:ww #1 \begin \q_nil \s_mark
          #2 \\ #3 \s_stop
      }
  }
  
\cs_new_protected:Npn \@@_split_at_nl_aux:nnNN #1 #2 #3 #4
  {
    \tl_gset:Nn #4 {#1}
    \tl_gset:Nn #3 {#2}
  }

\cs_new:Npn \@@_split_chk_if_begin:ww 
   #1 \begin #2 #3 \s_mark #4 \\ \q_nil \\ \s_stop
  {
    \quark_if_nil:nTF {#2}
      { {#1} {#4} }
      {
        \exp_after:wN \@@_split_collect_one_end:w
          \@@_split_cleanup_begin_q_nil:w #1 \begin{#2} #3 \\ #4 \s_stop
            { } { 1 }
      }
  }

\cs_new:Npn \@@_split_cleanup_begin_q_nil:w #1 \begin \q_nil {#1}

\cs_new:Npn \@@_split_collect_one_end:w #1 \end #2 #3 \s_stop #4 #5
  {
    \exp_args:Nf \@@_split_check_count_begins:nnnn
      { \@@_split_count_begins:n { #4 #1 } } {#5}
      { #4 #1 \end{#2} } {#3}
  }
\cs_new:Npn \@@_split_count_begins:n #1
  { \int_eval:n { 0 \@@_split_count_begins:w #1 \begin \q_nil } }

\cs_new:Npn \@@_split_count_begins:w #1 \begin #2
  { \quark_if_nil:nF {#2} { +1 \@@_split_count_begins:w } }

\cs_new:Npn \@@_split_check_count_begins:nnnn #1 #2 #3 #4
  {
    \int_compare:nNnTF {#1} = {#2}
      {
        \exp_last_unbraced:Nf \@@_split_final_cleanup:nn
          { \@@_split:n { \@@_split_guard:n {#3} #4 } }
      }
      {
        \exp_args:No \use_ii_i:nn
          { \exp_after:wN { \int_value:w \int_eval:n { #2 + 1 } } }
          { \@@_split_collect_one_end:w #4 \s_stop {#3} }
      }
  }
\cs_new:Npn \@@_split_final_cleanup:nn #1 #2
  {
      \exp:w \@@_split_final_cleanup:w #1
        \@@_split_guard:n \q_nil \s_mark { }
      {#2}
  }
\cs_new:Npn \@@_split_final_cleanup:w #1 \@@_split_guard:n #2 #3 \s_mark #4
  {
    \quark_if_nil:nTF {#2}
      { \exp_end: { #4 #1 } }
      { \@@_split_final_cleanup:w #3 \s_mark { #4 #1 #2 } }
  }
  
\cs_new:Npn \@@_split:n #1 {
    \@@_split_at_nl_first:w #1 \\ \q_nil \\ \s_stop }
  
% this looks unused.  
%\NewDocumentCommand \splitnl { mm +m }
%  {
%    \tl_set:Nf \l_@@_tmpa_tl { \split:n {#3} }
%    \show \l_@@_tmpa_tl
%    \exp_after:wN \__splitnl_aux:nnNN \l_@@_tmpa_tl #1 #2
%  }
%    \end{macrocode}
%  \end{macro}
%
%
%  \begin{macro}{\maybestartnewformulatag}
%    
%    \begin{macrocode}

\newif\if@subformulas
\tl_new:N \result  
  
\cs_new_protected:Npn\grabaformulapartandstart {
  \@@_split_at_nl:NN  \g_@@_grabbed_math_tl \result
  \typeout{====>first-result=\meaning\result}
  \typeout{====>first-tmpmathcontent=\meaning\g_@@_grabbed_math_tl}
  \tl_if_empty:NTF \g_@@_grabbed_math_tl
     {
       \typeout{====>formula~ has~ no~ subparts}
       \global\@subformulasfalse
     }
     {
       \typeout{====>formula~ has~ subparts}
       \global\@subformulastrue
       \edef\resulttitle{\g_@@_grabbed_env_tl\space (part)}
       \tagstructbegin{tag=Formula,
%    \end{macrocode}
%    For now we don't put anything in /alt or /ActualText on subformulas
%    \begin{macrocode}
%         alt=\result,
         title-o=\resulttitle
       }
    }
    \tagmcbegin{}
}

\cs_new_protected:Npn\grabaformulapartandmayberestart {
  \@@_split_at_nl:NN  \g_@@_grabbed_math_tl \result
  \typeout{====>result=\meaning\result}
  \typeout{====>tmpmathcontent=\meaning\g_@@_grabbed_math_tl}
%  \tl_if_empty:NTF \g_@@_grabbed_math_tl
%     {
%       \typeout{====>tmpmathcontent=empty}
%     }
%     {
%       \typeout{====>tmpmathcontent=not-empty}
       \edef\resulttitle{\g_@@_grabbed_env_tl\space (part)}
       \tagstructbegin{tag=Formula,
         alt=\result,
         title-o=\resulttitle
       }
%    }
    \tagmcbegin{}
}
%    \end{macrocode}
%  \end{macro}
%
%
%
%
%    \begin{macrocode}
\def\maybestartnewformulatag {
\if@subformulas
 \ifmeasuring@\else
%
  \tl_if_empty:NF \g_@@_grabbed_math_tl
     {
       \tagmcend 
       \tagstructend
       \grabaformulapartandmayberestart
     }
 \fi
\fi
}
%    \end{macrocode}
%
%    The breqn packages changes catcodes and that isn't yet covered
%    by our mechanism.
%    \begin{macrocode}
%\AddToHook{package/breqn/after}{
%  \typeout{===>~ in~ hook}
%  \math_register_halign_env:nn {dmath}{}
%  \math_register_halign_env:nn {dgroup*}{}
%}  
%    \end{macrocode}
%    
%    \begin{macrocode}
\ExplSyntaxOff
%    \end{macrocode}
%    
%    \begin{macrocode}
%<@@=>
%    \end{macrocode}
%    
%    \begin{macrocode}
%</kernel>
%    \end{macrocode}
%
% \Finale
%
% 
