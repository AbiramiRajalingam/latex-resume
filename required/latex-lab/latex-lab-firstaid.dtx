% \iffalse meta-comment
%
%% File: latex-lab-firstaid.dtx (C) Copyright 2023-2024 LaTeX Project
%
% It may be distributed and/or modified under the conditions of the
% LaTeX Project Public License (LPPL), either version 1.3c of this
% license or (at your option) any later version.  The latest version
% of this license is in the file
%
%    https://www.latex-project.org/lppl.txt
%
%
% The development version of the bundle can be found below
%
%    https://github.com/latex3/latex2e/required/latex-lab
%
% for those people who are interested or want to report an issue.
%
\def\ltlabfirstaiddate{2024-09-26}
\def\ltlabfirstaidversion{0.85f}
%<*driver>
\documentclass{l3doc}
\EnableCrossrefs
\CodelineIndex
\begin{document}
  \DocInput{latex-lab-firstaid.dtx}
\end{document}
%</driver>
%
% \fi
%
% \title{The \textsf{latex-lab-firstaid} package\\
% Temporary patches to external packages needed for the tagging project}
% \author{\LaTeX{} Project\thanks{Initial implementation done by Ulrike Fischer}}
% \date{v\ltlabfirstaidversion\ \ltlabfirstaiddate}
%
% \maketitle
%
% \newcommand{\xt}[1]{\textsl{\textsf{#1}}}
% \newcommand{\TODO}[1]{\textbf{[TODO:} #1\textbf{]}}
% \newcommand{\docclass}{document class \marginpar{\raggedright document class
% customizations}}
%
% \providecommand\hook[1]{\texttt{#1}}
%
% \begin{abstract}
% \end{abstract}
%
% \section{Introduction}
%
% The followings contains small temporary changes to external packages to avoid
% errors with the new tagging code. 
% 
% Similar to the main firstaid package the goal is to remove the 
% patches once the packages have been updated.

% \section{Implementation}
%
%    \begin{macrocode}
%<*package>
%<@@=tag>
%    \end{macrocode}
%    \begin{macrocode}
\ProvidesPackage {latex-lab-testphase-firstaid} [%
   \ltlabfirstaiddate\space v\ltlabfirstaiddate\space
   Temporary patches to external packages needed for the tagging project]
%    \end{macrocode}
%  \begin{macro}{\FirstAidNeededT}
%    This is a very simple help to ensure that we only apply first aid
%    to an unmodified package or class. It only works in the case the
%    file has already been loaded and the csname \cs{ver@\#1.\#2} got
%    defined (holding the current date, version, and short description
%    info). We then compare its content to a frozen string and make
%    the modification \verb=#3= only if both agree. If they differ we
%    assume that the package/class in question got updated by its
%    maintainer.
%    \begin{macrocode}
\ExplSyntaxOn 
\providecommand\FirstAidNeededT[3]{            
  \exp_args:Ncx\str_if_eq:onF{ver@#1.#2}{#3}
      { \typeout{==>~ First~ Aid~ for~ #1.#2~ no~ longer~ applied!^^J
          \@spaces Expected:^^J
          \@spaces\@spaces #3^^J
          \@spaces but~ found:^^J
          \@spaces\@spaces \use:c{ver@#1.#2}^^J
          \@spaces so~ I'm~ assuming~ it~ got~ fixed.
      } }
  \exp_args:Ncx\str_if_eq:onT{ver@#1.#2}{#3}
}
%    \end{macrocode}
%  \end{macro}

% \subsection{ams classes}
% The amsart, amsbook and amsproc classes do not use \cs{@author} to store the author list
% but a command \cs{authors}. To be able to nevertheless use the authors in the
% xmp-metadata we map \cs{@author} to this new command. 
% 
%    \begin{macrocode}
\AddToHook{class/amsart/after}
 {\def\@author{\authors}}
\AddToHook{class/amsbook/after}
 {\def\@author{\authors}}
\AddToHook{class/amsproc/after}
 {\def\@author{\authors}}  
%    \end{macrocode}
%
% \subsection{ams classes and amsthm}
% The amsart, amsbook and amsproc classes redefine the theorem code
% and this breaks the tagging added by the block code. The following
% reenables tagging. It does \emph{not} give a completly identical output
% (similar to the new theorem code, 
%  see \url{https://github.com/latex3/tagging-project/issues/715}). 
%  The code also does not try to use sockets yet, as the theorem definitions
%  in the block code don't do that yet either.  
%
%    \begin{macrocode}
\AddToHook{class/amsart/after}[latex-lab-testphase-firstaid/amsthm]
 {\tag_if_active:T{\@@_firstaid_amsthm:}}
\AddToHook{class/amsbook/after}[latex-lab-testphase-firstaid/amsthm]
 {\tag_if_active:T{\@@_firstaid_amsthm:}}
\AddToHook{class/amsproc/after}[latex-lab-testphase-firstaid/amsthm]
 {\tag_if_active:T{\@@_firstaid_amsthm:}} 
\AddToHook{package/amsthm/after}[latex-lab-testphase-firstaid/amsthm]
 {\tag_if_active:T{\@@_firstaid_amsthm:}} 
%    \end{macrocode}
%    \begin{macrocode}
\cs_new_protected:Npn \@@_firstaid_amsthm:
  {
%    \end{macrocode}
%  \cs{@endtheorem} must use the endblock code
%    \begin{macrocode}
    \def\@endtheorem{\endblockenv}  
%    \end{macrocode}
% In \cs{@thm} we have to remove the \cs{trivlist}
%    \begin{macrocode}
    \RenewDocumentCommand\@thm{mmmO{}}{%
      \ifhmode\unskip\unskip\par\fi
      \normalfont
      \let\thmheadnl\relax
      \let\thm@swap\@gobble
      \thm@notefont{\fontseries\mddefault\upshape}%
      \thm@headpunct{.}% add period after heading
      \thm@headsep 5\p@ plus\p@ minus\p@\relax
      \thm@space@setup
      ##1% style overrides
      \@topsep \thm@preskip               % used by thm head
      \@topsepadd \thm@postskip           % used by \@endparenv
%    \end{macrocode}
% We store the counter name so that the anchor can make use of it.
%    \begin{macrocode}
      \tl_set:Nn \l__block_thm_current_counter_tl{##2}
      \tl_if_empty:nTF{##2}
       {
        \@begintheorem{##3}{}[##4] 
       }
       {
         \@kernel@refstepcounter{##2}
         \@begintheorem{##3}{\csname the##2\endcsname}[##4]
       }
      }
%    \end{macrocode}
% \cs{@begintheorem} has a larger number of changes
%    \begin{macrocode}
    \def\@begintheorem##1##2[##3]{%
%    \end{macrocode}
% We use the theorem instance.
%    \begin{macrocode}
      \UseInstance{blockenv}{theorem}{}
%    \end{macrocode}
% While create the caption/label we disable para-tagging.
%    \begin{macrocode}
      \tagpdfparaOff
      \mode_leave_vertical:
%    \end{macrocode}
% The anchor for links
%    \begin{macrocode}
      \MakeLinkTarget{\l__block_thm_current_counter_tl}
      \group_begin:
      \normalfont
      \the\thm@headfont \thm@indent
        \@ifempty{##1}
         {\let\thmname\@gobble}
%    \end{macrocode}
% we insert the MC and the Lbl structure into \cs{thmname},
% \cs{thmnumber} and \cs{thmnote}. This will also work
% with new theorem style as long as they use these command.
%    \begin{macrocode}
         {\def\thmname####1{\tag_mc_begin:n {}####1\tag_mc_end:}}%
        \@ifempty{##2}
         {\let\thmnumber\@gobble}
         {\def\thmnumber####1
           {\tag_struct_begin:n{tag=Lbl}\tag_mc_begin:n {}
             ####1
            \tag_mc_end:\tag_struct_end:}}%
        \@ifempty{##3}
         {\let\thmnote\@gobble}
         {\def\thmnote####1{\tag_mc_begin:n{}####1\tag_mc_end:}}%
        \thm@swap\swappedhead\thmhead{##1}{##2}{##3}%      
        \thmheadnl % possibly a newline.
        \hskip\thm@headsep
      \group_end: 
%    \end{macrocode}
% Now we restart para tagging and start a paragraph. The socket
% is currently defined in tagpdf, so the code should only
% be used if tagging is active!
%    \begin{macrocode}
       \tagpdfparaOn
       \UseTaggingSocket{para/begin}  %
      \ignorespaces}  
%    \end{macrocode}
% This redefines the standard styles for the theorem heads.
% \cs{thm@headpunct} has been moved into the head code
% to make tagging more easier. 
%    \begin{macrocode}
    \def\thmhead@plain##1##2##3{%
      \tag_struct_begin:n{tag=Caption}  
      \thmname{##1}
      \thmnumber{
        \@ifnotempty{##1}{~}\@upn{##2}   
       }%   
      \thmnote{\pdffakespace\space{\the\thm@notefont(##3)}}
      %moved for easier tagging
      \tag_mc_begin:n{}\the\thm@headpunct\tag_mc_end:  
      \tag_struct_end:
      }
    \let\thmhead\thmhead@plain
    \def\swappedhead##1##2##3{%
      \tag_struct_begin:n{tag=Caption}
      \thmnumber{##2}
      \thmname{\@ifnotempty{##2}{\nobreakspace}##1}
      \thmnote{\pdffakespace\space{\the\thm@notefont(##3)}}
      \tag_mc_begin:n{}\the\thm@headpunct\tag_mc_end:
      \tag_struct_end:
     }  
    \let\swappedhead@plain=\swappedhead 
%    \end{macrocode}
% At last some adjustments for the proof environment.
% The qed symbols use a drawn box by default. We add an actualtext.
%    \begin{macrocode}
    \renewcommand{\openbox}{\leavevmode
      \hbox to.77778em{\pdf_bdc:nn{Span}{/ActualText<FEFF220E>}%
      \pdffakespace\hfil\vrule
      \vbox to.675em{\hrule width.6em\vfil\hrule}%
      \vrule\hfil\pdf_emc:}}  
%    \end{macrocode}
% And redefine proof to no longer use a trivlist.
%    \begin{macrocode}
    \renewenvironment{proof}[1][\proofname]{\par
      \pushQED{\qed}%
      \UseInstance{blockenv}{theorem}{beginsep=6\p@\@plus6\p@}
      \normalfont
      \tagpdfparaOff 
      \AddToHookNext{para/begin}
       {\tag_struct_begin:n{tag=Caption}
        \tag_mc_begin:n{}%
         \textit{##1\@addpunct{.}}%
        \tag_mc_end:
        \tag_struct_end:
        \tagpdfparaOn
        \UseTaggingSocket{para/begin}
        \pdffakespace\hspace{\labelsep}} 
      \ignorespaces
    }{%
      \popQED\endblockenv\par
     }
  }
\ExplSyntaxOff
%    \end{macrocode}
%
% \subsection{verse}
%
%    The \pkg{verse} package has its own definition of the
%    \env{verse} environment, which would tag correctly, except that
%    it is overwritten by the block code in the hook
%    \texttt{begindocument/before}. So the simplest way to make
%    tagging work is to reinstall the package version afterwards,
%    which is what we are doing here.
%    \begin{macrocode}
\AddToHook{package/verse/after}[latex-lab-firstaid]{%
  \FirstAidNeededT{verse}{sty}{2014/05/10 v2.4b verse typesetting}%
   {%
     \AtBeginDocument{%
       \renewenvironment{verse}[1][\linewidth]{%
         \stepcounter{verse@envctr}%
         \setcounter{poemline}{0}\refstepcounter{poemline}%
         \setcounter{vslineno}{1}%
         \let\\=\@vscentercr
         \list{}{\itemsep \z@
                 \itemindent  -\vindent
                 \listparindent\itemindent
                 \parsep       \stanzaskip
                 \ifdim #1 < \linewidth
                   \rightmargin        \z@
                   \setlength{\leftmargin}{\linewidth}%
                   \addtolength{\leftmargin}{-#1}%
                   \addtolength{\leftmargin}{-0.5\leftmargin}%
                 \else
                   \rightmargin        \leftmargin
                 \fi
                 \addtolength{\leftmargin}{\vindent}}%
         \item[]%
       }%
       {\endlist}%
     }%
   }%
}
%    \end{macrocode}
%    Of course, this means that the
%    optional argument of the environment then only accepts a length
%    value and not any more a key value list for altering the
%    environment settings.
%    
%    A more elabroate version could be something like this that allows
%    key/val and legacy interface. Or one could extend the list
%    template to support a \texttt{list-width} key.
%\begin{verbatim}
% \ExplSyntaxOn
% \cs_new_protected:Npn \ExtractAndDropKey #1#2#3#4#5 {
%   \tl_set_eq:NN #4 \c_novalue_tl      % or empty?
%   \keys_define:nn { #1 } { #2 .code:n = \tl_set:Nn #4{##1} }
%   \keys_set_known:nnN { #1 } { #3 } #5
% }
% \ExplSyntaxOff
% 
% % Change the env definition for verse matching verse.sty
% % This keeps the verse.sty interface as it is and only adjusts the
% % main environment to use the basic list env with the verse.sty
% % specific settings.
% \makeatletter
% 
% \AddToHook{package/verse/after}{%
%   \AtBeginDocument{%
%   \RenewDocumentEnvironment{verse}{={verse-width}!O{\linewidth}}%
%     {% 
%       \stepcounter{verse@envctr}%
%       \setcounter{poemline}{0}\refstepcounter{poemline}%
%       \setcounter{vslineno}{1}%
%       \let\\=\@vscentercr
%   %
%       \ExtractAndDropKey{verse}{verse-width}{#1}\@vswidth\@vsremainingkvlist
%   % If other keys have been specified but not verse-width we have no
%   % default for \@vswidth and need to set it again
%       \ExpandArgs{o}\IfNoValueT \@vswidth
%                         {\def\@vswidth{\linewidth}}%
%   %
%   % This is a bit ugly but we can't stick \cs{@vsremainingkvlist} into
%   % the instance argument as keys are expected to be visible on
%   % top-level not hidden inside a macro.  The alternative is to push
%   % in \verb=#1= but then the key/value \verb/verse-width=.../ is
%   % passed into the instance which is not known there (not harmful as
%   % it will get ignored but noticeably more and unnecessary
%   % processing).
%   %                      
%       \def\next##1{%
%         \UseInstance{blockenv}{list}% 
%           {%
%             item-indent =-\vindent,%
%             parindent   =-\vindent,%
%             par-skip    =\stanzaskip,%
%             item-skip   =0pt,%
%             leftmargin  = (\linewidth-\@vswidth)/2+\vindent,%
%             rightmargin = \ifdim\@vswidth<\linewidth 0pt
%                           \else (\linewidth-\@vswidth)/2\fi,%
%             ##1%
%            }}%
%       \ExpandArgs{o}\next\@vsremainingkvlist
%       \item\relax    
%     }{\endblockenv}%
%   }%
% }  
% \makeatother
%\end{verbatim}
%
%
% \subsection{blindtext}
%
% The blindtext package generates lists with nested \cs{loop} command.
% The inner loop introduces a group around the list which error when lists are tagged
% as \cs{@doendpe} is lost.
%    \begin{macrocode}
\AddToHook{file/blindtext.sty/after}[latex-lab-firstaid]{%
  \FirstAidNeededT{blindtext}{sty}{\blindfiledate\space\blindfileversion\space%
                 blindtext-Package}%
   {%
    \renewcommand{\blind@items}{%
       \setcounter{blind@listcount}{1}%
      \loop
      \ifblind@optional
        \ifblind@long
          \item[\blindtext@countitem] \blindtext@text
        \else
          \item[\blindtext@count] \blindtext@item
        \fi
        \else % \blind@optionalfalse
          \ifblind@long
            \item \blindtext@text
          \else
            \item \blindtext@countitem
          \fi
        \fi % \ifblind@optional
        {%
          \loop
          \ifnum\value{blindlistlevel}>\value{blindlist@level}%
            \stepcounter{blind@levelcount}%
            \blind@list[\value{blindlist}]\relax
            \addtocounter{blind@levelcount}{-1}%
            \setcounter{blind@listcount}{1}%
          \repeat
        \expandafter}\if@endpe\@doendpe\fi % <----------changed
      \ifnum\value{blind@listcount}<\value{blindlist}%
        \stepcounter{blind@listcount}%
      \repeat
    }
   }%
 }
%    \end{macrocode}
%
%
% \subsection{cleveref}
%
% The cleveref package redefines \cs{@makefntext} and this means that the patches in
% the new footnote code fails. We use a hook instead
%    \begin{macrocode}
\AddToHook{package/cleveref/after} 
 {
   \let\@makefntext\cref@old@makefntext
   \AddToHook{cmd/@makefntext/before}{%
    \cref@constructprefix{footnote}{\cref@result}%
    \protected@edef\cref@currentlabel{%
     [footnote][\arabic{footnote}][\cref@result]%
    \p@footnote\@thefnmark}}
%    \end{macrocode}
% The second problem is that it redefines \cs{refstepcounter} to use an argument.
% In itself this isn't a problem but as the new definition doesn't grab the arguments
% directly it fails over the generic hook inserted at the end of the definition
%    \begin{macrocode}
   \RenewDocumentCommand\refstepcounter{om}{%
     \IfNoValueTF{#1}%
      {\refstepcounter@noarg{#2}}%
      {\refstepcounter@optarg[#1]{#2}}%
    }%
 }   
%    \end{macrocode}
%
% \subsection{booktabs}
% In some cases booktabs inserts a 
% \cs{multispan} into the table (through the commands \cs{@cmidruleb} 
% and \cs{@cmidrulea} and this then errors 
% with the tagging code. 
% This affects both tabular and longtable 
% (but longtable more as booktabs handles lines in longtable differently).
% See also issue \url{https://github.com/latex3/tagging-project/issues/69}
% 
% 
\ExplSyntaxOn
\AddToHook{package/booktabs/after} 
 {
  \def\@cmidrulea{
   \multispan\@cmidla
   &\multispan\@cmidlb
   \unskip\hskip\cmrkern@l
  {
   \tag_mc_begin:n{artifact}
   \CT@arc@\leaders\hrule \@height\@thisrulewidth\hfill\kern\z@}
   \hskip\cmrkern@r
   \tag_mc_end: \int_gdecr:N \g__tbl_row_int 
   \cr}

  \def\@cmidruleb{%
    \multispan\@cmidlb
    \unskip\hskip \cmrkern@l%
   {
    \tag_mc_begin:n{artifact}
    \CT@arc@\leaders\hrule \@height\@thisrulewidth\hfill\kern\z@}
    \hskip\cmrkern@r
    \tag_mc_end: \int_gdecr:N \g__tbl_row_int 
    \cr} 
  }
\ExplSyntaxOff 
%
%    \begin{macrocode}
%</package>
%    \end{macrocode}
%
%
%    \begin{macrocode}
%<*latex-lab>
\ProvidesFile{firstaid-latex-lab-testphase.ltx}
        [\ltlabfirstaiddate\space v\ltlabfirstaidversion\space
         latex-lab wrapper firstaid]

\RequirePackage{latex-lab-testphase-firstaid}

%</latex-lab>
%    \end{macrocode}
