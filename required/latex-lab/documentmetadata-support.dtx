% \iffalse meta-comment
%
%% File: documentmetadata-support.dtx (C) 2021 LaTeX Project
%
% It may be distributed and/or modified under the conditions of the
% LaTeX Project Public License (LPPL), either version 1.3c of this
% license or (at your option) any later version.  The latest version
% of this license is in the file
%
%    https://www.latex-project.org/lppl.txt
%
%
% The development version of the bundle can be found below
%
%    https://github.com/FrankMittelbach/
%
% for those people who are interested or want to report an issue.
%
%<*driver>
\documentclass{l3doc}
\EnableCrossrefs
\CodelineIndex
\begin{document}
  \DocInput{documentmetadata-support.dtx}
\end{document}
%</driver>
%
% \fi
%

% \title{The \texttt{documentmetadata-support} code\thanks{}}
% \author{Frank Mittelbach, \LaTeX{} Project}
%
% \maketitle
%
%
% \begin{abstract}
% \end{abstract}
%
% \section{Introduction}
%
%    Currently the following keys are implemented:
%
%    \begin{description}
%       \item[\texttt{backend}] to specify the backend to use; this is
%           usually determined automatically.
%^^A       This will probably be extended to  pass the value also to packages.
%
%       \item[\texttt{pdfversion}] e.g. \texttt{pdfversion=1.7}
%
%       \item[\texttt{uncompress}] no value. Forces an uncompressed pdf.
%
%       \item[\texttt{lang}] to set the Lang entry in the Catalog.
%          E.g. \texttt{lang=de-DE}. The initial value is |en-US|
%
%       \item[\texttt{pdfstandard}] Choice key to set the pdf standard.
%         Currently |A-1b|, |A-2a|, |A-2b|, |A-2u|, |A-3a|, |A-3b| and |A-3u| are accepted as
%         values. The casing is irrelevant, |a-1b| works too.
%         The underlying code to ensure the requirements (as far as they
%         can be ensured) is incomplete, but a color profile is included and the
%         /OutputIntent is set. The |u| variants for example do not force unicode,
%         but they will pass the information to hyperref and hyperxmp. The |a| variants
%         do \emph{not} enforce (or even test) a tagged pdf yet.
%         More information can be found in the documentation
%         of \pkg{l3pdfmeta}.
%
%       \item[\texttt{colorprofiles}] This allows to load icc-colorprofiles. Details
%       are described in the documentation of \pkg{l3pdfmeta}.
%
%       \item[\texttt{pdfmanagement}] Boolean. This activates/deactivates
%         the core management code. By default the value is true.
%
%       \item[\texttt{firstaidoff}] This accepts a comma lists of keywords and disables the patches
%       related to them. More information can be found in the documentation of
%       \pkg{pdfmanagement-firstaid}.
%
%       \item[\texttt{testphase}] This key is used to load testphase code. The values it accepts
%       and their effect will change over time, when testphase packages are added or
%       removed when the code is moved into the kernel. Currently the accepted values are
%       \texttt{tagpdf} (this loads the tagpdf package), \texttt{headings} (this loads
%       code which reimplements heading commands), and \texttt{ptagging}, which loads code
%       to allow paragraph tagging to work with engines other than luatex.
%       \item[\texttt{activate}] This key is used to enable some document wide functions. It is
%       currently in an experimental state. The values and their behaviour are subject to change.
%       Currently the only value is |tagging|,
%       which will do |\tagpdfsetup{activate,paratagging,interwordspace}|. It requires that
%       \pkg{tagpdf} has been loaded first with the |testphase| key.
%       \item[\texttt{debug}] This key activates some debug options. Currently only the
%       keys |para| (with the default and only value |show|),
%       and |log| (with the values of \pkg{tagpdf}) and |uncompress| (which does the same
%       as |uncompress| as main key)  are known.
%    \end{description}
%
%
%
% \StopEventually{\setlength\IndexMin{200pt}  \PrintIndex  }
%
%
% \section{The Implementation}
%
%    \begin{macrocode}
%<*code>
%    \end{macrocode}
%
%    \begin{macrocode}

\RequirePackage{pdfmanagement-testphase}
%    \end{macrocode}
%
%    \begin{macrocode}
\ExplSyntaxOn\makeatletter
%    \end{macrocode}



%  \begin{macro}{\DocumentMetadata}
%    
%    \cs{DocumentMetadata} should not be used after
%    \cs{documentclass} so we error in this case.
%    It can be used more than once
%     but follow-up calls should not do the initialization code.
%    \begin{macrocode}
\cs_set_protected:Npn \DocumentMetadata #1
  {
    \cs_if_eq:NNTF \documentclass \@twoclasseserror
      { \msg_error:nn { meta } { after-class } }
      {
%    \end{macrocode}
%
%    The wanted backend must be detected first, we read the init
%    key and then force the loading of the backend.
%    The backend can contain management commands, so the boolean should
%    be set to true first.
%    \begin{macrocode}
        \bool_gset_true:N \g__pdfmanagement_active_bool
        \keys_set_groups:nnn { document / metadata} {init}{ #1 }
        %if no backend has been loaded force it now:
        \str_if_exist:NF \c_sys_backend_str
          {
            \sys_load_backend:n {}
          }
%    \end{macrocode}
%    Now we load the extra backend code:
%    \begin{macrocode}
        \ExplSyntaxOn\makeatletter
          \file_input:n {l3backend-testphase-\c_sys_backend_str.def}
        \ExplSyntaxOff\makeatother
%    \end{macrocode}
%    Set the default language, process the rest of the keys,
%    and setup the generic driver.
%    \begin{macrocode}
%FMi wouldn't it be cleaner to set the default when declaring the key?
        \keys_set_filter:nnn  { document / metadata } { init } { lang=en-US, #1 }
        \bool_if:NT \g__pdfmanagement_active_bool
          {
            \PassOptionsToPackage{customdriver=hgeneric-testphase}{hyperref}
          }
%    \end{macrocode}
%    \cs{pdfmanagement_add:nnn} has collected values in this hook.
%    \begin{macrocode}
        \hook_use_once:n {pdfmanagement/add}
%    \end{macrocode}
%    Now we redefine \cs{DocumentMetadata} so that it only process the
%    keys on any further calls.
%
%     We need to update the hyperref option if the active status changes.
%    \begin{macrocode}
        \cs_set_protected:Npn \DocumentMetadata ##1
          {
            \keys_set_filter:nnn  { document / metadata } { init } { ##1 }
%FMi
            \str_remove_all:cn {opt@hyperref.sty}{customdriver=hgeneric-testphase}
%FMi this doesn't seem to need the F case if we move the removal to
%    the front
            \bool_if:NT \g__pdfmanagement_active_bool
             {
%FMi               \str_remove_all:cn {opt@hyperref.sty}{customdriver=hgeneric-testphase}
               \PassOptionsToPackage{customdriver=hgeneric-testphase}{hyperref}
             }
%FMi             {
%FMi               \str_remove_all:cn {opt@hyperref.sty}{customdriver=hgeneric-testphase}
%FMi             }
          }
%    \end{macrocode}
% Load more modules, the testphase code and the firstaid code.
%    \begin{macrocode}
%FMi this doesn't look correct -- testphase code is not loaded if only
%    in second call --- see test 001
        \g__pdfmanagement_testphase_tl
        \RequirePackage{pdfmanagement-firstaid}
      }
}
%    \end{macrocode}
%  \end{macro}




%    \begin{macrocode}

%FMi defined elsewhere
%FMi
%FMi \clist_new:N \g__pdfmanagement_firstaidoff_clist
%FMi \tl_new:N \g__pdfmanagement_testphase_tl

\keys_define:nn { document / metadata }
  {
     backend .choices:nn =
       { dvipdfmx , dvips , dvisvgm , luatex , pdftex , pdfmode , xdvipdfmx , xetex }
       {
        \sys_load_backend:n {#1}
      },
    ,backend .groups:n = { init }
    ,pdfversion .code:n =
      {
        \pdf_version_gset:n { #1 }
        \AddToDocumentProperties[document]{pdfversion}{#1}
      }
    ,uncompress .code:n =
      {
        \pdf_uncompress:
      }
    ,uncompress .value_forbidden:n = true
    ,lang .code:n =
      {
        \pdfmanagement_add:nnn {Catalog} {Lang}{(#1)}
        \AddToDocumentProperties[document]{lang}{#1}
      }
    %,xmpmeta .bool_gset:N = \g_pdfmeta_xmp_bool %see pdfmeta unused and undefined for now!
    % this uses internal command from pdfmeta, it should probably move there ...
    ,pdfstandard .code:n =
      {
        \exp_args:Nnx
        \keys_set:nn {document / metadata} {_pdfstandard=\str_uppercase:n{#1}}
      }
    ,_pdfstandard .choices:nn =
      {A-1B,A-2A,A-2B,A-2U,A-3A,A-3B,A-3U}
      {
        \prop_if_exist:cT { g__pdfmeta_standard_pdf/#1_prop }
          {
            \prop_gset_eq:Nc \g__pdfmeta_standard_prop { g__pdfmeta_standard_pdf/#1 _prop }
          }
        \AddToDocumentProperties [document]{pdfstandard}{#1}
      }
    ,_pdfstandard / unknown .code:n =
      {
        \msg_warning:nnn{pdf}{unknown-standard}{#1}
      }
    ,pdfmanagement .bool_gset:N = \g__pdfmanagement_active_bool
    ,firstaidoff .clist_gset:N = \g__pdfmanagement_firstaidoff_clist
    ,testphase .multichoice:
    ,testphase / tagpdf .code:n =
      {
        \tl_gput_right:Nn\g__pdfmanagement_testphase_tl
          {
            \RequirePackage{tagpdf}
            \AddToDocumentProperties [document]{testphase/tagpdf}{loaded}
          }
      }
    ,testphase / headings .code:n =
      {
        \tl_gput_right:Nn\g__pdfmanagement_testphase_tl
          {
            \RequirePackage{headings-testphase}
            \AddToDocumentProperties [document]{testphase/headings}{loaded}
          }
      }
    ,testphase / ptagging .code:n =
      {
        \tl_gput_right:Nn\g__pdfmanagement_testphase_tl
          {
            \AddToHook{class/after}
              {
                \RequirePackage{ptagging-testphase}
                \AddToDocumentProperties [document]{testphase/ptagging}{loaded}
              }
          }
      }
    ,testphase / unknown .code:n =
      {
        \tl_gput_right:Nn\g__pdfmanagement_testphase_tl
           {
             \AddToHook{class/after}
               {
                 \RequirePackage{#1-testphase}
                 \AddToDocumentProperties [document]{testphase/#1}{loaded}
               }
           }
      }
    ,activate .multichoice:
    ,activate / tagging .code:n =
      {
        \AddToHook{package/after/tagpdf}
          {
            \tagpdfsetup{activate,paratagging,interwordspace}
            \AddToDocumentProperties [document]{tagging}{active}
            \AddToDocumentProperties [document]{tagging/para}{active}
            \AddToDocumentProperties [document]{tagging/interwordspace}{active}
          }
      }
    ,debug .code:n =
      {
        \keys_set:nn { document / metadata / debug } {#1}
      }
    ,debug / para .code:n =
      {
        \AddToHook{package/after/tagpdf}
          {
            \tagpdfsetup{paratagging-show}
          }
      }
    ,debug / log .code:n =
      {
        \AddToHook{package/after/tagpdf}
          {
            \tagpdfsetup{log=#1}
          }
      }
    ,debug / uncompress .code:n =
      {
        \pdf_uncompress:
      }
  }
%    \end{macrocode}
%    
%    \begin{macrocode}
\msg_new:nnn  { meta } { after-class }
              {
                \token_to_str:N \DocumentMetadata \c_space_tl
                should~be~used~only~before~\token_to_str:N\documentclass
              }
%    \end{macrocode}
%
%    \begin{macrocode}
\ExplSyntaxOff\makeatother
%    \end{macrocode}
%
%
%    \begin{macrocode}
%<*code>
%    \end{macrocode}
%
% \Finale
%



\endinput

\prop_new:N \g__pdfmanagement_documentproperties_prop %
\NewDocumentCommand\AddToDocumentProperties{O{\@currname}mm}
  {
    \exp_args:NNx
      \prop_gput:Nnn \g__pdfmanagement_documentproperties_prop
        {
          \tl_if_blank:eTF {#1}{top-level/}{#1/} #2
        }
        { #3}
  }
\NewExpandableDocumentCommand\GetDocumentProperties{m}
  {
    \prop_item:Nn \g__pdfmanagement_documentproperties_prop {#1}
  }
\msg_new:nnn  { pdfmanagement } { show-properties }
  {
    The~following~document~properties~have~been~stored:
    #1
  }
\NewDocumentCommand\ShowDocumentProperties {}
  {
    \msg_show:nnx {pdfmanagement}{show-properties}
      {
        \prop_map_function:NN \g__pdfmanagement_documentproperties_prop \msg_show_item:nn
      }
  }

