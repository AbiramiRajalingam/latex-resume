% \iffalse meta-comment
%
%% File: latex-lab-new-or-2.dtx
% Copyright (C) 2022-2023 The LaTeX Project
%
% It may be distributed and/or modified under the conditions of the
% LaTeX Project Public License (LPPL), either version 1.3c of this
% license or (at your option) any later version.  The latest version
% of this license is in the file
%
%    https://www.latex-project.org/lppl.txt
%
%
% The development version of the bundle can be found below
%
%    https://github.com/latex3/latex2e/required/latex-lab
%
% for those people who are interested or want to report an issue.
%
\def\ltlabneworIIdate{2023-07-25}
\def\ltlabneworIIversion{0.85b}

%<*driver>
\documentclass{l3doc}
\EnableCrossrefs
\CodelineIndex
\begin{document}
  \DocInput{latex-lab-new-or-2.dtx}
\end{document}
%</driver>
%
% \fi
%
%
% \title{The \texttt{latex-lab-testphase-new-or-2} code\thanks{}}
% \author{Frank Mittelbach, \LaTeX{} Project}
% % \date{v\ltlabneworIIversion\ \ltlabneworIIdate}
% \maketitle
%
% \newcommand\fmi[1]{\begin{quote} TODO: \itshape #1\end{quote}}
% \newcommand\NEW[1]{\marginpar{\mbox{}\hfill\fbox{New: #1}}}
% \providecommand\pkg[1]{\texttt{#1}}
%
% \providecommand\hook[1]{\texttt{#1\DescribeHook[noprint]{#1}}}
% \providecommand\socket[1]{\texttt{#1\DescribeSocket[noprint]{#1}}}
% \providecommand\plug[1]{\texttt{#1\DescribePlug[noprint]{#1}}}
%
% \NewDocElement[printtype=\textit{socket},idxtype=socket,idxgroup=Sockets]{Socket}{socketdecl}
% \NewDocElement[printtype=\textit{plug},idxtype=plug,idxgroup=Plugs]{Plug}{plugdecl}
% \NewDocElement[printtype=\textit{hook},idxtype=hook,idxgroup=Hooks]{Hook}{hookdecl}
%
%
% \begin{abstract}
%    This code implements changes to the output routine intended to be
%    moved in the \LaTeX{} kernel at some point in the future.
% \end{abstract}
%
%
%
%
% \section{Introduction}
%
%
% \section{Hooks and replaceable code blocks}
%
%
% \subsubsection{Replaceable code blocks (sockets)}
%
%    To cater for different layouts with respect to text, footnotes,
%    and bottom-floats placements there are two sockets for
%    now.
%    \begin{description}
%    \item[\socket{@makecol/outputbox} (0 arguments)]
%
%      In code for this socket the \cs{@outputbox} (holding the
%      galley text for the current column or page) is augmented by
%      attaching floats and footnote areas together with appropriate
%      spacing.
%
%      Prior to calling the socket the output routine has already
%      decided which floats go into which area and which get deferred.
%      Therefore, the assumption is that the code in the socket attaches all
%      areas that contain floats. If this is not done, then the order
%      of floats is likely to be screwed up unless unused floats
%      are moved to the defer list in an appropriate way (for now we
%      don't offer any interface for that scenario). 
%
%      Before the code in the soocket is run, any existing glue at the bottom
%      of the \cs{@outputbox} is removed and stored in a safe
%      place. If needed, it can be reinserted with one of the helper
%      commands.
%
%      To support setting this up the following helper commands are available:
%      \begin{description}
%      \item[\cs{@outputbox@append} (1 argument)]
%
%        \DescribeMacro[noprint]{@outputbox@append}
%
%        This general purpose command alters the \cs{@outputbox} box by
%        appending material to it.
%
%      \item[\cs{@outputbox@appendfootnotes} (0 arguments)]
%
%        \DescribeMacro[noprint]{@outputbox@appendfootnotes}
%
%        This command appends the footnotes to the \cs{@outputbox} (if
%        there are any). If not, then it does nothing.
%
%      \item[\cs{@outputbox@attachfloats} (0 arguments)]
%      \item[\cs{@outputbox@attachtopfloats} (0 arguments)]
%      \item[\cs{@outputbox@attachbottomfloats} (0 arguments)]
%
%        \DescribeMacro[noprint]{@outputbox@attachfloats}
%        \DescribeMacro[noprint]{@outputbox@attachtopfloats}
%        \DescribeMacro[noprint]{@outputbox@attachbottomfloats}
%   
%        Attaching top and bottom floats can usually be done in one
%        go, but for special layouts we might want more control so we
%        provide also separate commands.
%
%      \item[\cs{@outputbox@reinsertbskip} (0 arguments)]
%
%        \DescribeMacro[noprint]{@outputbox@reinsertbskip}
%
%        Reinsert the bottom skip of the \cs{@outputbox} that was
%        saved before.   
%   
%      \item[Testing for existence of material]
%
%        \DescribeMacro[noprint]{@if@footnotes@TF}
%        \DescribeMacro[noprint]{@if@bfloats@TF}
%
%        There are a number of helpers to run conditional code
%        depending on whether or not there are footnotes or bottom
%        floats. They are \cs{@if@footnotes@TF} and
%        \cs{@if@bfloats@TF}
%        (names are likely to change).
%   
%      \end{description}
%
%      \fmi{Decide on names for these helper commands. We could keep
%        them similar to the above (because they are only supposed to be
%        used by a few packages). However, we could alternatively use
%        CamelCase names, under the assumption that clases may also
%        directly define a plug for the socket, instead of loading a
%        support package such as \pkg{footmisc} (or in addition
%        to). --- decide}
%      
%      This socket cannot be empty but needs appropriate code; a set
%      of suitable plugs for it are already given in the kernel. These
%      are
%      \begin{description}
%      \item[\plug{space-footnotes-floats}]
%
%         After the galley text there is a vertical \cs{vfill}
%         followed by any footnotes followed by the bottom floats, if any.
%
%      \item[\plug{floats-footnotes-space}]
%
%         As before but the \cs{vfill} is at the bottom (page is
%         ragged  bottom).
%
%      \item[\plug{footnotes-space-floats}]
%
%         As before but the \cs{vfill} is between footnotes and floats.
%
%      \item[\plug{space-floats-footnotes}]
%
%         Here the footnotes come last.\footnote{There are two more
%         permutations, but neither of them has ever been requested so
%         they aren't set up by default --- doing that in a class
%         would be trivial though.}
%
%      \item[\plug{floats-footnotes}]
%
%         All excess space has to be distributed across the existing
%         glue on the page, e.g., within the text galley, the
%         separation between blocks, etc.
%         The order is text, floats, footnotes.
%
%      \item[\plug{footnotes-floats}]
%
%         As the previous one but floats and footnotes are
%         swapped. This is the \LaTeX{} default, i.e., this plug is
%         assigned to the socket.
%
%      \end{description}
%
%-----------------------
%
%    \item[\socket{@makecol/footnotes} (0 arguments)]
%
%       This socket is used to manipulate the footnote
%        material inside \cs{box}\cs{footins}. It if contains code, it
%        is supposed to do some processing of that box and then write
%        the result back into it (and nothing else!). By default it
%        does nothing, i.e., has the \plug{noop} assigned.
%
%       If (short) footnotes are run as a pargraph this socket gets
%         the plug \plug{para} assigned which is defined elsewhere. 
%
%    \end{description}
%
% \StopEventually{\setlength\IndexMin{200pt}  \PrintIndex  }
%
%
% \section{The Implementation}
%
%    \begin{macrocode}
%<*package>
%    \end{macrocode}
%
% \subsection{File declaration}
%    \begin{macrocode}
\ProvidesPackage{latex-lab-testphase-new-or-2}
        [\ltlabneworIIdate\space v\ltlabneworIIversion\space 
         Changes to the output routine]
%    \end{macrocode}
%        
% \subsection{\cs{@makecol} reimplementation}
%
%    In order for other packages to prepend or append code to
%    \cs{@makecol} the generic command hooks
%    \texttt{cmd/@makecol/before} and \texttt{cmd/@makecol/after} can
%    be used, so for now there is nothing we need to do for this.
%
%
%  \begin{macro}{\@makecol}
%    \cs{@makecol} is shortened a lot, basically all the hardwired
%    code in the middle has moved into a socket.
%    \begin{macrocode}
\def \@makecol {%
%    \end{macrocode}
%    We start with a kernel hook for tagging.
%    \fmi{The name is likely to change and it probably will eventually
%         be replaced with a socket.}
%    \begin{macrocode}
  \@kernel@before@cclv
%    \end{macrocode}
%    Save away box 255 as \cs{@outputbox} to make it available for further adjustments.
%    \begin{macrocode}
  \setbox\@outputbox \box\@cclv
%    \end{macrocode}
%    The only real addition is the next command which either does
%    nothing or removes an infinite glue from the bottom of the
%    \cs{@outputbox}.
%    \begin{macrocode}
  \@outputbox@removebskip
%    \end{macrocode}
%    When this code is run any ``here'' floats in the \cs{@outputbox} are already handled, so we
%    recycle their registers and put them back to the \cs{@freelist}.
%    \begin{macrocode}
  \let\@elt\relax
  \xdef\@freelist{\@freelist\@midlist}%
  \global \let \@midlist \@empty
%    \end{macrocode}
%    Here we have the configurable part. This socket is supposed to add floats,
%    footnotes and stretchable vertical space as appropriate to the \cs{@outputbox}.
%    \begin{macrocode}
  \UseSocket{@makecol/outputbox}%
%    \end{macrocode}
%    Then we deal with any \cs{enlargethispage} or run the normal code
%    to build a column.
%    \begin{macrocode}
  \ifvbox\@kludgeins
     \@make@specialcolbox
  \else
     \@make@normalcolbox
  \fi
  \global \maxdepth \@maxdepth
}
%    \end{macrocode}
%  \end{macro}
%
%
%  \begin{macro}{\@outputbox@depth}
%    We need to know the depth of \cs{@outputbox} once in a
%    while. Rather than using a temp dimen (as it was done in the
%    past), we give it a proper register.
%    \begin{macrocode}
\newdimen\@outputbox@depth
%    \end{macrocode}
%  \end{macro}
%
%  \begin{macro}{\@make@normalcolbox}
%    Taken out of \cs{@makecol} for readability.
%    \begin{macrocode}
\def \@make@normalcolbox {%
   \setbox\@outputbox \vbox to\@colht {%
       \@texttop
       \@outputbox@depth \dp\@outputbox
       \unvbox \@outputbox
       \vskip -\@outputbox@depth
       \@textbottom
      }%
}
%    \end{macrocode}
%  \end{macro}
%
%
%  \begin{macro}{\@make@specialcolbox}
%    Make the colbox when \cs{enlargethispage} was used.
%    \begin{macrocode}
\def \@make@specialcolbox {%
   \@outputbox@append {\vskip-\@outputbox@depth}%
   \@tempdima \@colht
   \ifdim \wd\@kludgeins>\z@
     \advance \@tempdima -\ht\@outputbox
     \advance \@tempdima \pageshrink
     \setbox\@outputbox \vbox to \@colht {%
       \unvbox\@outputbox
       \vskip \@tempdima
       \@textbottom
       }%
   \else
     \advance \@tempdima -\ht\@kludgeins
     \setbox \@outputbox \vbox to \@colht {%
       \vbox to \@tempdima {%
         \unvbox\@outputbox
         \@textbottom}%
       \vss}%
   \fi
   {\setbox \@tempboxa \box \@kludgeins}%
}
%    \end{macrocode}
%  \end{macro}
%
%
%  \begin{macro}{\@outputbox@removebskip}
%
%    This is really a bug fix for the kernel but one we only
%    automatically make in new documents that are using \cs{DocumentMetadata}.
%  \fmi{may make optional for legacy docs}
%    If
%    \cs{raggedbottom} is in force, footnotes get attached to the main
%    galley at a distance of \cs{footskip} on all pages except on
%    those that are ended by \cs{newpage} or \cs{clearpage} where the
%    \cs{vfil} from \cs{newpage} pushes the footnotes to the very bottom.
%
%    This is kind of a weird difference to a page  ending with
%    \cs{pagebreak}---in that case the page is also run
%    short, but the footnotes are not pushed to the bottom.
%
%    In \pkg{footmisc} \cs{@outputbox@removebskip} is only applied when
%    \pkg{footmisc} is called with with an option specifying the
%    footnote placement, i.e., not  in the default case.
%    In new documents we apply it always.
%    \begin{macrocode}
\def\@outputbox@removebskip{%
%    \end{macrocode}
%    We first test if we are in a \cs{raggedbottom} layout. If not we
%    do nothing, but we don't disable the code because
%    \cs{raggedbottom} may get used only for some parts of the
%    document.
%    \begin{macrocode}
  \ifx\@textbottom\relax \else
%    \end{macrocode}
%    We then append some negative glue at the end of \cs{@outputbox}
%    provided it has a glue stretch order of 1 or more (i.e., contains
%    a \texttt{fil} or \texttt{fill} part).
%    \begin{macrocode}
    \@outputbox@append{%
      \@tempskipa\lastskip
      \ifnum \gluestretchorder\@tempskipa>\z@
        \vskip-\@tempskipa
%    \end{macrocode}
%        
%  \begin{macro}{\@outputbox@reinsertbskip}
%    We also record the value so that it can be reinserted
%    elsewhere. As we have to do this globally, we also need to
%    explicitly reset it if we don't find any such glue.
%    \begin{macrocode}
        \xdef\@outputbox@reinsertbskip
            {\noexpand\@outputbox@append{\vskip\the\@tempskipa}}%
      \else
        \global\let\@outputbox@reinsertbskip\relax
      \fi
    }%
 \fi
}
%    \end{macrocode}
%    We need a trivial top-level definition for
%    \cs{@outputbox@reinsertbskip} in case the first page has no
%    bottom glue and the command gets called.
%    \begin{macrocode}
\let\@outputbox@reinsertbskip\relax
%    \end{macrocode}
%  \end{macro}
%  \end{macro}
%
%
%
%  \begin{macro}{\@kernel@before@cclv,
%                \@kernel@before@footins}
%    These two commands are internal kernel hooks intended for tagging
%    support in case that is active. They should not be altered by package code!
%    By default they do nothing (and
%    may have been defined already by \cs{DocumentMetadata}).
%    \fmi{The names might change and perhaps each of them is turned
%    into a socket named something like \socket{tagsupport/before/cclv}
%    or similar.}
%    \begin{macrocode}
\providecommand\@kernel@before@cclv{}
\providecommand\@kernel@before@footins{}
%    \end{macrocode}
%  \end{macro}
%
%
%
%
% \subsection{The output routine configuration components}
%
%    Here we provide the components that are used to define code for the
%    \socket{@makecol/outputbox} socket.
%
%
%  \begin{macro}{\@outputbox@append}
%
%    This general purpose command alters the \cs{@outputbox} box by
%    appending material to it. As this is a box typesetting operation
%    we make sure that the last line of the box reflects the true
%    depth of the last line (in case that is needed later). We also
%    expose the current depth of \cs{@outputbox} as
%    \cs{@outputbox@depth} before unboxing so that its value can be
%    used by \verb=#1= if wanted.
%    \begin{macrocode}
\def\@outputbox@append #1{%
%  \if!\detokenize{#1}!\else
     \setbox\@outputbox \vbox {%
       \boxmaxdepth \@maxdepth
       \@outputbox@depth\dp\@outputbox      % if needed in #1
       \unvbox \@outputbox
       #1%
     }%
%  \fi
}
%    \end{macrocode}
%  \end{macro}
%
%
%
%
%
%  \begin{macro}{\@outputbox@appendfootnotes}
%
%    This command appends the footnotes to the \cs{@outputbox} (if
%    there are any). If not, then it does nothing.
%    \begin{macrocode}
\def\@outputbox@appendfootnotes {%
   \ifvoid\footins \else
%    \end{macrocode}
%    First come two configuration points: what to do if we are in a split
%    footnote situation and a second one that does some manipulation
%    of the \cs{footins} box before it gets appended.
% \fmi{The code for handling split footnotes will get revised as part of socket handling  in the future}
%    \begin{macrocode}
     \@makecol@handlesplitfootnotes
%    \end{macrocode}
%
%    \begin{macrocode}
     \UseSocket{@makecol/footnotes}%
%    \end{macrocode}
%    Then the footnotes are appended:
%    \begin{macrocode}
     \@outputbox@append{%
       \vskip \skip\footins
%    \end{macrocode}
%    This is a kernel hook for tagging.
%    \fmi{The name is likely to change and it probably will eventually
%         be replaced with a socket.}
%    \begin{macrocode}
       \@kernel@before@footins
%    \end{macrocode}
%
%    \begin{macrocode}
       \color@begingroup
         \normalcolor
         \footnoterule
%    \end{macrocode}
%    Support for \pkg{pdfcolfoot}, eventually this can go once color
%    is properly supported. The csname is constructed in case it
%    doesn't exist.
%    \begin{macrocode}
         \csname pdfcolfoot@current\endcsname
         \unvbox \footins
       \color@endgroup
      }%
  \fi
}
%    \end{macrocode}
%  \end{macro}
%
%
%
%  \begin{macro}{\@outputbox@attachfloats}
%  \begin{macro}{\@outputbox@attachtopfloats}
%  \begin{macro}{\@outputbox@attachbottomfloats}
%    Attaching top and bottom floats can usually be done in one go,
%    but for special layouts we might want more control so we provide
%    also separate commands.
%    \begin{macrocode}
\let \@outputbox@attachfloats \@combinefloats
%    \end{macrocode}
%
%    \begin{macrocode}
\def \@outputbox@attachtopfloats {%
  \ifx \@toplist\@empty \else \@cflt \fi
}
\def \@outputbox@attachbottomfloats {%
    \ifx \@botlist\@empty \else \@cflb \fi
}
%    \end{macrocode}
%  \end{macro}
%  \end{macro}
%  \end{macro}
%
%
%
%
%  \begin{macro}{\@makecol@handlesplitfootnotes}
%  \begin{macro}{\@makecol@splitfootnotemessagehook}
%    This is only an early draft and doesn't do much.
%    Contains  incomplete preparation for tagging commented out.
% \fmi{Interfaces and code will change in the future}
%    \begin{macrocode}
\def\@makecol@handlesplitfootnotes {%
%  \ifx\splitfootnote@continuation\@empty \else
%    \setbox\footins\vbox{\splitfootnote@continuation\unvbox\footins}%
%    \global\let\splitfootnote@continuation\@empty
%  \fi
  \ifnum\insertpenalties>\z@
    \@makecol@splitfootnotemessagehook
%    \setbox\footins\vbox{\unvbox\footins --- END at split}%
%    \gdef\splitfootnote@continuation    {--- START after split}%
  \fi
}
%\def\splitfootnote@continuation{}
%    \end{macrocode}
%    This  could issue warning if split footnotes are encountered.
%    \begin{macrocode}
\let \@makecol@splitfootnotemessagehook \@empty
%    \end{macrocode}
%  \end{macro}
%  \end{macro}
%
%
%
%  \begin{socketdecl}{@makecol/footnote}
%
%    The socket allowing the manipulation of \cs{footins} box
%    (result needs to be moved back in there). Used when footnotes are
%    reformatted into a single paragraph by the
%    \texttt{para} option of \pkg{footmisc}. By default it does nothing.
%    \begin{macrocode}
\NewSocket{@makecol/footnotes}{0}
%    \end{macrocode}
%  \end{socketdecl}
%
%
%
% \fmi{Some temp interfaces until configuration points are available.}
%
%  \begin{macro}{\@if@flushbottom@TF}
%    Test for \cs{flushbottom} (currently not used).
%    \begin{macrocode}
\def\@if@flushbottom@TF{%
  \ifx\@textbottom\relax
    \expandafter\@firstoftwo
  \else
    \expandafter\@secondoftwo
  \fi
}
%    \end{macrocode}
%  \end{macro}
%
%
%  \begin{macro}{\@if@footnotes@TF}
%    Test if footnotes are present on the current page.
%    \begin{macrocode}
\def\@if@footnotes@TF{%
  \ifvoid\footins
    \expandafter\@secondoftwo
  \else
    \expandafter\@firstoftwo
  \fi
}
%    \end{macrocode}
%  \end{macro}
%
%
%  \begin{macro}{\@if@bfloats@TF}
%    Test if bottom floats are around.
%    \begin{macrocode}
\def\@if@bfloats@TF{%
  \ifx \@botlist\@empty
    \expandafter\@secondoftwo
  \else
    \expandafter\@firstoftwo
  \fi
}
%    \end{macrocode}
%  \end{macro}
%
%
%
% \subsection{The \cs{@makecol} configuration}
%
%
%  \begin{socketdecl}{@makecol/outputbox}
%    We have one socket that is supposed to augment the \cs{@outputbox}
%    by attaching floats and footnotes with appropriate spacing.
%
%    \begin{macrocode}
\NewSocket{@makecol/outputbox}{0}
%    \end{macrocode}
%  \end{socketdecl}
%
%    The following plugs are available for
%    this socket:
%
%  \begin{plugdecl}{space-footnotes-floats}
%    After the galley text there is a vertical \cs{vfill} followed by
%    any footnotes followed by the bottom floats, if any.
%    \begin{macrocode}
\NewSocketPlug {@makecol/outputbox}{space-footnotes-floats} {%
  \@if@footnotes@TF
      {\@outputbox@append{\vfill}}%
      {\@if@bfloats@TF
        {\@outputbox@append{\vfill}}%
        {\@outputbox@reinsertbskip}%
      }%
  \@outputbox@appendfootnotes
  \@outputbox@attachfloats
}
%    \end{macrocode}
%  \end{plugdecl}
%
%  \begin{plugdecl}{floats-footnotes-space}
%     As before but the \cs{vfill} is at the bottom (page is ragged
%    bottom).
%    \begin{macrocode}
\NewSocketPlug {@makecol/outputbox}{floats-footnotes-space} {%
  \@outputbox@attachfloats
  \@if@footnotes@TF
      {\@outputbox@append{\vfill}}%
      {\@outputbox@reinsertbskip}%
  \@outputbox@appendfootnotes
}
%    \end{macrocode}
%  \end{plugdecl}
%
%  \begin{plugdecl}{footnotes-space-floats}
%    As before but the \cs{vfill} is between footnotes and floats.
%    \begin{macrocode}
\NewSocketPlug {@makecol/outputbox}{footnotes-space-floats} {%
  \@outputbox@appendfootnotes
  \@if@bfloats@TF
      {\@outputbox@append{\vfill}}%
      {\@outputbox@reinsertbskip}%
  \@outputbox@attachfloats
}
%    \end{macrocode}
%  \end{plugdecl}
%
%  \begin{plugdecl}{space-floats-footnotes}
%         Here the footnotes come last.\footnote{There are two more
%         permutations, but neither of them has ever been requested so
%         they aren't set up by default --- doing that in a class
%         would be trivial though.}
%    \begin{macrocode}
\NewSocketPlug {@makecol/outputbox}{space-floats-footnotes} {%
  \@if@footnotes@TF
      {\@outputbox@append{\vfill}}%
      {\@if@bfloats@TF
        {\@outputbox@append{\vfill}}%
        {\@outputbox@reinsertbskip}}%
  \@outputbox@attachfloats
  \@outputbox@appendfootnotes
}
%    \end{macrocode}
%  \end{plugdecl}
%
%
%
%  \begin{plugdecl}{floats-footnotes}
%         All excess space has to be distributed across the existing
%         glue on the page, e.g., within the text galley, the
%         separation between blocks, etc.
%         The order is text, floats, footnotes.
%    \begin{macrocode}
\NewSocketPlug {@makecol/outputbox}{floats-footnotes} {%
  \@outputbox@attachfloats
  \@outputbox@appendfootnotes
%    \end{macrocode}
%    We do reinsert the bottom skip from \cs{newpage} if it
%    was taken out earlier. This is, strictly speaking, not necessary
%    in most cases, but it is a \cs{vfil} while \cs{raggedbottom} is
%    only generating \verb=\vspace{0pt plus .0001fil}=, so if you have
%    several \cs{vfil} on the page before the \cs{newpage} you would
%    alter the space distribution if one is taken out.
%    \begin{macrocode}
  \@outputbox@reinsertbskip
}
%    \end{macrocode}
%
%  \end{plugdecl}
%
%  \begin{plugdecl}{footnotes-floats}
%    \begin{macrocode}
\NewSocketPlug {@makecol/outputbox}{footnotes-floats} {%
  \@outputbox@appendfootnotes
  \@outputbox@attachfloats
  \@outputbox@reinsertbskip
}
%    \end{macrocode}
%    The \texttt{footnote-floats} plug is the default used by
%    \LaTeX{}; it can be overwritten either through \pkg{footmisc} or
%    by assigning any of the other plugs (or by coding yet another
%    plug for the socket).
%    \begin{macrocode}
\AssignSocketPlug {@makecol/outputbox}{footnotes-floats}
%    \end{macrocode}
%  \end{plugdecl}
%
%
%
%
%
% \section  {Replacement for the \pkg{footmisc} package}
%
%    The replacement for \pkg{footmisc}. If the new code is used, we must replace
%    the package if it is loaded by the user:
%    \begin{macrocode}
\declare@file@substitution{footmisc.sty}{latex-lab-footmisc.ltx}
%    \end{macrocode}
%
%
%
% \section {Temp stuff that is related but should go to the kernel}
%
%    This is the code from \texttt{latex-lab-footnotes.dtx}.
%    \begin{macrocode}
% \iffalse meta-comment
%
%% File: latex-lab-footnotes.dtx
% Copyright (C) 2022 The LaTeX Project
%
% It may be distributed and/or modified under the conditions of the
% LaTeX Project Public License (LPPL), either version 1.3c of this
% license or (at your option) any later version.  The latest version
% of this license is in the file
%
%    https://www.latex-project.org/lppl.txt
%
%
% The development version of the bundle can be found below
%
%    https://github.com/latex3/latex2e/required/latex-lab
%
% for those people who are interested or want to report an issue.
%
%<*driver>
\documentclass{l3doc}
\EnableCrossrefs
\CodelineIndex
\begin{document}
  \DocInput{latex-lab-footnotes.dtx}
\end{document}
%</driver>
%
% \fi
%
%
% \title{The \texttt{latex-lab-footnotes} code\thanks{}}
% \author{Frank Mittelbach, \LaTeX{} Project}
%
% \maketitle
%
% \newcommand\fmi[1]{\begin{quote} TODO: \itshape #1\end{quote}}
% \newcommand\NEW[1]{\marginpar{\mbox{}\hfill\fbox{New: #1}}}
% \providecommand\pkg[1]{\texttt{#1}}
%
% \begin{abstract}
% \end{abstract}
%
% \section{Introduction}
%
%    This code implements changes to the output routine. It is loaded by the
%    pdfmanagement.
%
%
%
% \StopEventually{\setlength\IndexMin{200pt}  \PrintIndex  }
%
%
% \section{The Implementation}
%
%    \begin{macrocode}
%<*kernel>
%    \end{macrocode}
%
% \subsection{File declaration}
%    \begin{macrocode}
\ProvidesFile{latex-lab-footnotes.ltx}
        [2022-02-25 v0.1a changes to the footnote interfaces]
%    \end{macrocode}
%
%    \begin{macrocode}

% latex.ltx

        % not looked at yet
        
\long\def\@mpfootnotetext#1{%
  \global\setbox\@mpfootins\vbox{%
    \unvbox\@mpfootins
    \reset@font\footnotesize
    \hsize\columnwidth
    \@parboxrestore
    \def\@currentcounter{mpfootnote}%
    \protected@edef\@currentlabel
         {\csname p@mpfootnote\endcsname\@thefnmark}%
    \color@begingroup
      \@makefntext{%
        \rule\z@\footnotesep\ignorespaces#1\@finalstrut\strutbox}%
    \par
    \color@endgroup}}


\def\@makefnmark{\hbox{\@textsuperscript{\normalfont\@thefnmark}}}



\def\@mpfn{footnote}
\def\thempfn{\thefootnote}



%-------------------------------------


\ExplSyntaxOn


\cs_new:Npn \fn_step_fnmark:nn #1#2 {
  \tl_if_novalue:nTF {#1}
    {
      \stepcounter {#2}
      \protected@xdef \@thefnmark { \cs:w the#2 \cs_end: }
    }
    {
     \group_begin:
        \int_set:cn { c@#2 }{ #1 }
        \unrestored@protected@xdef \@thefnmark { \cs:w the#2 \cs_end: }
     \group_end:
    }           
}

\cs_new:Npn \fn_set_fnmark:nn #1#2 {
  \tl_if_novalue:nTF {#1}
    {
      \protected@xdef \@thefnmark { \cs:w the#2 \cs_end: }
    }
    {
     \group_begin:
        \int_set:cn { c@#2 }{ #1 }
        \unrestored@protected@xdef \@thefnmark { \cs:w the#2 \cs_end: }
     \group_end:
    }           
}

%-------------------------------------

\NewHook{fnmark/before}
\NewHook{fnmark/begin}
\NewHook{fnmark}
\NewHook{fnmark/end}
\NewHook{fnmark/after}

% for hyperref links and perhaps tagging
\tl_new:N \@kernel@after@fnmark
\tl_new:N \@kernel@before@fnmark@end

\cs_new:Npn \__fnote_debug_footnotemark: {
  \ShowCommand\@kernel@after@fnmark
  \ShowCommand\@kernel@before@fnmark@end
  \LogHook{fnmark/before}
  \LogHook{fnmark/begin}
  \LogHook{fnmark}
  \LogHook{fnmark/end}
  \LogHook{fnmark/after}
  \cs_gset_eq:NN \__fnote_debug_footnotemark: \prg_do_nothing:
}


\cs_new:Npn \fnote_footnotemark: {
  \__fnote_debug_footnotemark:
%-------
% bibarts
% chextras  --- actually in the wrong place does an \unskip
  \UseHook{fnmark/before}
%-------
  \leavevmode
  \ifhmode
    \edef\@x@sf{\the\spacefactor}
%-------
% bxjsja-minimal.def   --- what they do could be done at ``bibarts''
%                         (a bit less efficient)
% memhfixc.sty
% footmisc.sty
    \UseHook{fnmark/begin}
%-------
    \nobreak
  \fi
%-------
% hyperref.sty    
  \UseHook{fnmark}
  \@kernel@after@fnmark
%-------
  \@makefnmark
%-------
% hyperref.sty
% memhfixc.sty  --- could move fnmark/after
% scrlttr2.cls  --- could vanish if footmisc uses a hook
% footmisc.sty
  \@kernel@before@fnmark@end
  \UseHook{fnmark/end}
%-------
  \ifhmode
    \spacefactor \@x@sf \relax
  \fi
%    
%-------
  \UseHook{fnmark/after}
%-------
}

% alterations not covered:
%
% ./arabtex/afoot.sty  --- too different (and probably too old)


% Provide the name \LaTeXe{} is used to.

\let \@footnotemark \fnote_footnotemark:



%-------------------------------------

\NewHook{fntext/before}
\NewHook{fntext}
\NewHook{fntext/para}
\NewHook{fntext/after}


\cs_new:Npn \__fnote_debug_footnotetext: {
  \ShowCommand\@footnotetext@process
  \ShowCommand\@footnotetext@processii
  \ShowCommand\@footnotetext@processiii
  \ShowCommand\@footnotetext@processiv
  \LogHook{fntext/before}
  \LogHook{fntext}
  \LogHook{fntext/para}
  \LogHook{fntext/after}
  \cs_gset_eq:NN \__fnote_debug_footnotetext: \prg_do_nothing:
  }

\cs_new:Npn \fnote_footnotetext:n #1 {
  \__fnote_debug_footnotetext:
%-------
% ./linguex/linguex.sty
  \UseHook{fntext/before}
%-------
  \@footnotetext@process {  % config point
%-------
% resetting baselinestretch ... (could be done further down)
% ./uafthesis/uafthesis.cls
% ./setspace/setspace.sty
% ./footmisc/footmisc.sty (normal)
    \UseHook{fntext}
%-------
    \reset@font
    \footnotesize
%-------
% some classes use a different font size, e.g.,
% ./nrc/nrc1.cls  ./nrc/nrc2.cls
% but those could be done in fntext/para instead
%-------
%    \end{macrocode}
%    In case of sidenotes the next settings are pointless, but as they
%    do not hurt (except for the \cs{hsize} setting) and are needed
%    for all other cases we make them here and overwrite them for side notes
%    \begin{macrocode}
    \interlinepenalty\interfootnotelinepenalty
    \splittopskip\footnotesep
    \splitmaxdepth \dp\strutbox
    \floatingpenalty \@MM
    \hsize\columnwidth
    \@parboxrestore
    \def\@currentcounter{footnote}
    \protected@edef \@currentlabel { \p@footnote \@thefnmark }
%-------
% altering para parameters ...
% code for resphilosophica came earlier but it could go here.
% Has the advantage that one can also overwrite \cs{@currentcounter}
% and \cs{@currentlabel} is that is necessary.
%
% ./resphilosophica/resphilosophica.cls
    \UseHook{fntext/para}
%-------
    \color@begingroup
%-------
% fnpara wants to replace \@makefntext{...} (footmisc probably too ...)
% needs handling!
%-------
      \@footnotetext@processii       % config point
        {
%-------
% ./resphilosophica/resphilosophica.cls
%-------
          \@footnotetext@processiii  % config point
          \ignorespaces
%-------
% Maybe those better go inside \@makefntext or maybe not:
% bibarts
% fnbreak.sty
%-------
          #1
%-------
% bibarts
% fnbreak.sty
%-------
          \@footnotetext@processiv   % config point
        }
      \par
    \color@endgroup
  }
%-------
% ./linguex/linguex.sty
  \UseHook{fntext/after}
%-------
}

% default for config point #1 arg
\def\@footnotetext@process { \insert\footins }

% default for config point #1 arg
\def\@footnotetext@processii {  % config point
      \@makefntext
}

% default for config point 0 args
\def\@footnotetext@processiii { \rule\z@\footnotesep }

% default for config point 0 args
\def\@footnotetext@processiv  { \@finalstrut\strutbox }



% Provide the name \LaTeXe{} is used to

\cs_set_eq:NN \@footnotetext \fnote_footnotetext:n


% alterations not covered:
%
% ./revtex4-1/revtex4-1.cls  ./revtex/ltxutil.sty ./revtex/revtex4-2.cls ... (need analysis)
% ./bigfoot/bigfoot.sty



%    \end{macrocode}
%
%
%
%
%
%
% \subsection{Firstaid vor packages and classes}
%
% \subsection{\pkg{setspace}
%
%    It should not overwrite it any longer but use a hook, so for now we
%    do just that here.
%    \begin{macrocode}
\AddToHook{package/setspace/after}
   {\let \@footnotetext \fnote_footnotetext:n 
    \AddToHook{fntext}[setspace]{\let\baselinestretch\setspace@singlespace}}
%    \end{macrocode}
%
%
%
%
% \subsection{\pkg{hyperref}
%
%    Prevent hyperref from redefining footnote stuff --- this is a
%    temp solution.
%    \begin{macrocode}
\AddToHook{package/hyperref/after}{
  \let\H@@footnotetext\fnote_footnotetext:n
  \let\H@@footnotemark\fnote_footnotemark:
  \let \@footnotetext \fnote_footnotetext:n
  \let \@footnotemark \fnote_footnotemark:
}


          



%-------------------------------------

% use of kerns to mark h-mode positions (unit sp)
%
% 1 = CJK
% 2 = CJK
% 3 = multiple footnotes (footmisc, koma, eledmac, tufte, memoir,
%    parnotes, sidenotes)
% 3 = outer kern in letter spacing (letterspace)
% 3 = beginning of list (examdesign.cls)
% 4 = CJK pigin
% 5 = CJK ruby

% 1-4 = polyglossia for korean

%-------------------------------------


\DeclareDocumentCommand\footnotetext {om} {
  \fn_set_fnmark:nn {#1} \@mpfn
  \@footnotetext {#2}
}


%-------------------------------------

\NewHook{fnote/after}

\DeclareDocumentCommand\footnote {om} {
  \fn_step_fnmark:nn {#1} \@mpfn
  \@footnotemark
  \@footnotetext {#2}
%-------
% for multiple flags mainly
  \UseHook{fnote/after}
%-------
}

%-------------------------------------


\DeclareDocumentCommand\footnotemark {o} {
  \fn_step_fnmark:nn {#1} { footnote }
  \@footnotemark
}


%-------------------------------------

\DeclareDocumentCommand\footref {m}{%
  \begingroup
    \unrestored@protected@xdef\@thefnmark{\ref{#1}}%
  \endgroup
  \@footnotemark
}



%-------------------------------------
%  Tagging
%-------------------------------------

% hyperref support only when loaded (improve integration)

\newcounter{absfootnote}
\AddToHook{fnmark}{\stepcounter{absfootnote}}  % too simple (fails with opt args)

\AddToHook{fnmark}
{
  \cs_if_exist:NT \tag_struct_begin:n
     {
       \tag_mc_end_push:
       \exp_args:Nx
       \tag_struct_begin:n{tag=Lbl,ref=fn.\the\c@absfootnote}
       \tag_mc_begin:n{tag=Lbl}
       \IfPackageLoadedTF{hyperref}{ \hyper@linkstart {link}{fn.\the\c@absfootnote} }{}
     }
}

\AddToHook{fnmark/end}
{
  \cs_if_exist:NT \tag_struct_begin:n
    {
      \IfPackageLoadedTF{hyperref}{ \hyper@linkend }{}
      \tag_mc_end:
      \tag_struct_end:
      \tag_mc_begin_pop:n{}
    }
}

\AddToHook{fntext/before}
{
  \cs_if_exist:NT \tag_struct_begin:n
    {
      \bool_gset_eq:NN   \g__tag_saved_in_mc_bool \g__tag_in_mc_bool
      \bool_gset_false:N \g__tag_in_mc_bool
    } 
} 

\AddToHook{fntext/after}
{
  \cs_if_exist:NT \tag_struct_begin:n
    {
      \bool_gset_eq:NN \g__tag_in_mc_bool\g__tag_saved_in_mc_bool
    }
}

% align this with redefs below instead ...

\AtBeginDocument{
  \def\@makefntext #1 {
      \parindent 1em
      \noindent
      \tag@FENote { \hb@xt@1.8em{\hss\@makefnmark} }{ #1 }
  }
}

\def\tag@FENote#1#2{
  \cs_if_exist:NTF \tag_struct_begin:n
    {
      \tag_mc_end_push:
      \IfPackageLoadedTF{hyperref}{ \hypertarget{fn.\the\c@absfootnote}{} }{}
      \tag_struct_begin:n
          {
            tag=FENote
            ,label=fn.\the\c@absfootnote
          }
          \tag_struct_begin:n{tag=Lbl}
          \tag_mc_begin:n{tag=Lbl}
          #1
          \tag_mc_end:
          \tag_struct_end:
          \tag_mc_begin:n{tag=FENote}
          #2
          \tag_mc_end:
          \tag_struct_end:
          \tag_mc_begin_pop:n{}
    }
    { #1 #2 }
}

%-------------------------------------


\ExplSyntaxOff

        
%</kernel>
%    \end{macrocode}
%
%
% \section{Reimplementing the \pkg{foomisc} package}
%
%    \begin{macrocode}
%<*footmisc>
%%
%% Copyright (c) 1995-2011 Robin Fairbairns
%% Copyright (c) 2018-2022 Robin Fairbairns, Frank Mittelbach
%%
%% This file is part of the `latex-lab Bundle'.
%% --------------------------------------------
%%
%% It may be distributed and/or modified under the
%% conditions of the LaTeX Project Public License, either version 1.3c
%% of this license or (at your option) any later version.
%% The latest version of this license is in
%%    https://www.latex-project.org/lppl.txt
%% and version 1.3c or later is part of all distributions of LaTeX
%% version 2008 or later.
%%
%% This work has the LPPL maintenance status 'maintained'.
%%
%%
%% File: footmisc.dtx (C) Copyright 1995-2011 Robin Fairbairns
%%                    (C) Copyright 2018-2022 Frank Mittelbach
\NeedsTeXFormat{LaTeX2e}
\providecommand\DeclareRelease[3]{}
\providecommand\DeclareCurrentRelease[2]{}

\DeclareRelease{v5}{2011-06-06}{footmisc-2011-06-06.sty}
\DeclareCurrentRelease{}{2022-02-14}
\ProvidesPackage{latex-lab-footmisc}%
        [2022/02/14 v6.0b
     a miscellany of footnote facilities -- latex-lab version%
                   ]

\NeedsTeXFormat{LaTeX2e}[2020/10/01]
\newtoks\FN@temptoken
\providecommand\protected@writeaux{%
  \protected@write\@auxout
}
\def\l@advance@macro{\@@dvance@macro\edef}
\def\@@dvance@macro#1#2#3{\expandafter\@tempcnta#2\relax
  \advance\@tempcnta#3\relax
  #1#2{\the\@tempcnta}%
}
\let\@advance@macro\l@advance@macro
\newdimen\footnotemargin
\footnotemargin1.8em\relax
\DeclareOption{symbol}{\renewcommand\thefootnote{\fnsymbol{footnote}}}
\newif\ifFN@robust \FN@robustfalse
\DeclareOption{symbol*}{%
  \renewcommand\thefootnote{\@fnsymbol\c@footnote}%
  \FN@robusttrue
  \AtEndOfPackage{\setfnsymbol{lamport*-robust}}%
}
\newif\ifFN@para  \FN@parafalse
\DeclareOption{para}{%  
%    \end{macrocode}
%    Options are executed in the orderof declaration, thus no point in
%    checking for side option as footmisc did in the past
%    \begin{macrocode}
%    \PackageError{footmisc}{Option "\CurrentOption" incompatible with
%                            option "side"}%
%                 {I shall ignore "\CurrentOption"}%
  \FN@paratrue
}
%    \end{macrocode}
%
%    \begin{macrocode}
\DeclareOption{side}{\ifFN@para
    \PackageError{footmisc}{Option "\CurrentOption" incompatible with
                            option "para"}%
                 {I shall ignore "\CurrentOption"}%
  \else
    \def\@footnotetext@process {\marginpar}
    \AddToHook{fntext/para}{%
      \hsize\marginparwidth     % correct the default \hsize
      \footnotesep\z@           % don't add a default separation
    }
    \def\@footnotetext@processii  {\@makefntext}
    \def\@footnotetext@processiii {}
    \def\@footnotetext@processiv  {}
  \fi
}
\let\footnotelayout\@empty
\DeclareOption{ragged}{%
  \@ifundefined{RaggedRight}%
    {\renewcommand\footnotelayout{\linepenalty50 \raggedright}}%
    {\renewcommand\footnotelayout{\linepenalty50 \RaggedRight}}%
}
\newif\ifFN@perpage
\FN@perpagefalse
\DeclareOption{perpage}{%
  \FN@perpagetrue
}
\newif\ifFN@fixskip      \FN@fixskipfalse

\let\FN@bottomcases\thr@@
\newif\ifFN@abovefloats  \FN@abovefloatstrue
\DeclareOption{bottom}{%
  \let\FN@bottomcases\@ne
  \FN@abovefloatsfalse
  \FN@fixskiptrue
}
\DeclareOption{bottomfloats}{%
  \let\FN@bottomcases\tw@
  \FN@abovefloatstrue \FN@fixskiptrue
}
\DeclareOption{abovefloats}{\FN@abovefloatstrue  \FN@fixskiptrue}
\DeclareOption{belowfloats}{\FN@abovefloatsfalse \FN@fixskiptrue}
\DeclareOption{marginal}{%
  \footnotemargin-0.8em\relax
}
\DeclareOption{flushmargin}{%
  \footnotemargin0pt\relax
}
\newif\ifFN@hangfoot  \FN@hangfootfalse
\DeclareOption{hang}{%
  \FN@hangfoottrue
}
\newcommand*\hangfootparskip{0.5\baselineskip}
\newcommand*\hangfootparindent{0em}%
\DeclareOption{norule}{%
  \renewcommand\footnoterule{}%
  \advance\skip\footins 4\p@\@plus2\p@\relax
}
\DeclareOption{splitrule}{%
  \gdef\split@prev{0}
  \let\pagefootnoterule\footnoterule
  \let\mpfootnoterule\footnoterule
  \def\splitfootnoterule{\kern-3\p@ \hrule \kern2.6\p@}
  \def\footnoterule{\relax
    \ifx \@listdepth\@mplistdepth
      \mpfootnoterule
    \else
      \ifnum\split@prev=\z@
        \pagefootnoterule
      \else
        \splitfootnoterule
      \fi
      \xdef\split@prev{\the\insertpenalties}%
    \fi
  }%
}
\newif\ifFN@stablefootnote  \FN@stablefootnotefalse
\DeclareOption{stable}{\FN@stablefootnotetrue}
\newif\ifFN@multiplefootnote  \FN@multiplefootnotefalse
\DeclareOption{multiple}{\FN@multiplefootnotetrue}
\ProcessOptions
%    \end{macrocode}
%    This version of \pkg{footmisc} can assume that the new OR code is
%    already available, thus nothing needs loading at this
%    point. However, as long as we use this code also in a package
%    version that can be loaded by other package while we are in a
%    transition phase it is not clear whether not the kernel code is
%    already available for other packages.
%    \begin{macrocode}
%\@ifundefined{@kernel@before@cclv}
%  {% \iffalse meta-comment
%
%% File: latex-lab-new-or.dtx
% Copyright (C) 2022 The LaTeX Project
%
% It may be distributed and/or modified under the conditions of the
% LaTeX Project Public License (LPPL), either version 1.3c of this
% license or (at your option) any later version.  The latest version
% of this license is in the file
%
%    https://www.latex-project.org/lppl.txt
%
%
% The development version of the bundle can be found below
%
%    https://github.com/latex3/latex2e/required/latex-lab
%
% for those people who are interested or want to report an issue.
%
%<*driver>
\documentclass{l3doc}
\EnableCrossrefs
\CodelineIndex
\begin{document}
  \DocInput{latex-lab-new-or.dtx}
\end{document}
%</driver>
%
% \fi
%
%
% \title{The \texttt{latex-lab-new-or} code\thanks{}}
% \author{Frank Mittelbach, \LaTeX{} Project}
%
% \maketitle
%
% \newcommand\fmi[1]{\begin{quote} TODO: \itshape #1\end{quote}}
% \newcommand\NEW[1]{\marginpar{\mbox{}\hfill\fbox{New: #1}}}
% \providecommand\pkg[1]{\texttt{#1}}
%
% \begin{abstract}
% \end{abstract}
%
% \section{Introduction}
%
%    This code implements changes to the output routine. It is loaded by the
%    pdfmanagement.
%
%
%
% \StopEventually{\setlength\IndexMin{200pt}  \PrintIndex  }
%
%
% \section{The Implementation}
%
%    \begin{macrocode}
%<*code>
%    \end{macrocode}
%
% \subsection{File declaration}
%    \begin{macrocode}
\ProvidesFile{latex-lab-new-or.ltx}
        [2022-02-06 v0.1 changes to the output routine]
%    \end{macrocode}
% \subsection{\cs{@makecol} reimplementation}
%
%    In order for other packages to prepend or append code to
%    \cs{@makecol}, they can use the generic command hooks
%    \texttt{cmd/@makecol/before} and \texttt{cmd/@makecol/after}, so
%    there is nothing we need to do here.
%
%
%  \begin{macro}{\@makecol}
%    \cs{@makecol} is shortened a lot, basically all the hardwired
%    code in the middle has moved into a configuration point.
%    \begin{macrocode}
\def \@makecol {%
  \@kernel@before@cclv
  \setbox\@outputbox \box\@cclv
%    \end{macrocode}
%    The only real addition is the next command which either does
%    nothing or removes an infinite glue from the bottom of the
%    \cs{@outputbox}.
%    \begin{macrocode}
  \@outputbox@removebskip
%    \end{macrocode}
%    Any ``here'' floats in the \cs{@outputbox} are now handled so we
%    recycle their registers and put them back to the \cs{@freelist}.
%    \begin{macrocode}
  \let\@elt\relax
  \xdef\@freelist{\@freelist\@midlist}%
  \global \let \@midlist \@empty
%    \end{macrocode}
%    Here we have the configurable part.
% \fmi{Interface to configuration points will change in the future}
%    \begin{macrocode}
  \@makecol@appendblocks
%    \end{macrocode}
%    The we deal with any \cs{enlargethispage} or run the normal code
%    to build a column.
%    \begin{macrocode}
  \ifvbox\@kludgeins
     \@makespecialcolbox
  \else
     \@makenormalcolbox
  \fi
  \global \maxdepth \@maxdepth
}
%    \end{macrocode}
%  \end{macro}
%
%
%  \begin{macro}{\@outputbox@depth}
%    We need to know the depth of \cs{@outputbox} once in a
%    while. Rather than using a temp dimen (as it was done in the
%    past), we give it a proper register.
%    \begin{macrocode}
\newdimen\@outputbox@depth
%    \end{macrocode}
%  \end{macro}
%
%  \begin{macro}{\@makenormalcolbox}
%    Taken out of \cs{@makecol} for readability.
%    \begin{macrocode}
\def \@makenormalcolbox {%
   \setbox\@outputbox \vbox to\@colht {%
       \@texttop
       \@outputbox@depth \dp\@outputbox
       \unvbox \@outputbox
       \vskip -\@outputbox@depth
       \@textbottom
      }%
}
%    \end{macrocode}
%  \end{macro}
%
%
%  \begin{macro}{\@makespecialcolbox}
%    Make the colbox when \cs{enlargethispage} was used.
%    \begin{macrocode}
\def \@makespecialcolbox {%
   \@outputbox@append {\vskip-\@outputbox@depth}%
   \@tempdima \@colht
   \ifdim \wd\@kludgeins>\z@
     \advance \@tempdima -\ht\@outputbox
     \advance \@tempdima \pageshrink
     \setbox\@outputbox \vbox to \@colht {%
       \unvbox\@outputbox
       \vskip \@tempdima
       \@textbottom
       }%
   \else
     \advance \@tempdima -\ht\@kludgeins
     \setbox \@outputbox \vbox to \@colht {%
       \vbox to \@tempdima {%
         \unvbox\@outputbox
         \@textbottom}%
       \vss}%
   \fi
   {\setbox \@tempboxa \box \@kludgeins}%
}
%    \end{macrocode}
%  \end{macro}
%
%  \begin{macro}{\@outputbox@removebskip}
%    The real definition for this is in \pkg{footmisc}.
%    \begin{macrocode}
\let\@outputbox@removebskip \relax
\let\@outputbox@reinsertbskip\relax
%    \end{macrocode}
%  \end{macro}


%  \begin{macro}{\@outputbox@removebskip}
%
%    This is really a bug fix for the kernel, but perhaps one has to
%    make it optional because it is in there since day one). If
%    \cs{raggedbottom} is in force, footnotes get attached to the main
%    galley at a distance of \cs{footskip} on all pages except on
%    those that are ended by \cs{newpage} or \cs{clearpage} where the
%    \cs{vfil} from \cs{newpage} pushes the footnotes to the very bottom.
%
%    This is kind of a weird difference to a page  ending with
%    \cs{pagebreak}---in that case the page is also run
%    short, but the footnotes are not pushed to the bottom.
%
%    In \pkg{footmisc} \cs{@outputbox@removebskip} is only applied when
%    \pkg{footmisc} is called with with an option specifying the
%    footnote placement, i.e., not  in the default case.
%    In new documents we apply it always.
%    \begin{macrocode}
  \def\@outputbox@removebskip{%
%    \end{macrocode}
%    We first test if we are in a \cs{raggedbottom} layout. If not we
%    do nothing, but we don't disable the code because
%    \cs{raggedbottom} may get used only for some parts of the
%    document.
%    \begin{macrocode}
    \ifx\@textbottom\relax \else
%    \end{macrocode}
%    We then append some negative glue at the end of \cs{@outputbox}
%    provided it has a glue stretch order of 1 or more (i.e., contains
%    a \texttt{fil} or \texttt{fill} part).
%    \begin{macrocode}
      \@outputbox@append{%
        \@tempskipa\lastskip
        \ifnum \gluestretchorder\@tempskipa>\z@
          \vskip-\@tempskipa
%    \end{macrocode}
%  \begin{macro}{\@outputbox@reinsertbskip}
%    We also record the value so that it can be reinserted
%    elsewhere. As we have to do this globally, we also need to
%    explicitly reset it if we don't find any such glue.
%    \begin{macrocode}
          \xdef\@outputbox@reinsertbskip
              {\noexpand\@outputbox@append{\vskip\the\@tempskipa}}%
        \else
          \global\let\@outputbox@reinsertbskip\relax
        \fi
      }%
   \fi
  }
%    \end{macrocode}
%    We need a trivial top-level definition for
%    \cs{@outputbox@reinsertbskip} in case the first page has no
%    bottom glue and the command gets called.
%    \begin{macrocode}
\let\@outputbox@reinsertbskip\relax
%    \end{macrocode}
%  \end{macro}
%  \end{macro}
%
%
%
%  \begin{macro}{\@kernel@before@cclv}
%  \begin{macro}{\@kernel@before@footins}
%    These two commands are internal kernel hooks intended for tagging
%    support in case that is active. By default they do nothing (and
%    may have been defined already by \cs{DocumentMetadata}).
%    \begin{macrocode}
\providecommand\@kernel@before@cclv{}
\providecommand\@kernel@before@footins{}
%    \end{macrocode}
%  \end{macro}
%  \end{macro}
%
%
%
%
% \subsection{The output routine configuration components}
%
%    Here we provide the components that are used to define
%    \cs{@makecol@appendblocks}.
%
%
%  \begin{macro}{@outputbox@append}
%
%    This general purpose command alters the \cs{@outputbox} box by
%    appending material to it. As this is a box typesetting operation
%    we make sure that the last line of the box reflects the true
%    depth of the last line (in case that is needed later). We also
%    expose the current depth of \cs{@outputbox} as
%    \cs{@outputbox@depth} before unboxing so that its value can be
%    used by \verb=#1= if wanted.
%    \begin{macrocode}
\def\@outputbox@append #1{%
%  \if!\detokenize{#1}!\else
     \setbox\@outputbox \vbox {%
       \boxmaxdepth \@maxdepth
       \@outputbox@depth\dp\@outputbox      % if needed in #1
       \unvbox \@outputbox
       #1%
     }%
%  \fi
}
%    \end{macrocode}
%  \end{macro}
%
%
%
%
%
%  \begin{macro}{\@outputbox@appendfootnotes}
%
%    This command appends the footnotes to the \cs{@outputbox} (if
%    there are any). If not then it does nothing.
%    \begin{macrocode}
\def\@outputbox@appendfootnotes {%
   \ifvoid\footins \else
%    \end{macrocode}
%    First come two configuration points: what to do if we are in a split
%    footnote situation and a second one that does some manipulation
%    of the \cs{footins} box before it gets appended.
% \fmi{this code will get revised as part of CP handling  in the future}
%    \begin{macrocode}
     \@makecol@handlesplitfootnotes
     \@makecol@preparefootinshook
%    \end{macrocode}
%    Then the footnotes are appended:
%    \begin{macrocode}
     \@outputbox@append{%
       \vskip \skip\footins
       \@kernel@before@footins
       \color@begingroup
         \normalcolor
         \footnoterule
%    \end{macrocode}
%    Support for \pkg{pdfcolfoot}, eventually this can go once color
%    is properly supported.
%    \begin{macrocode}
         \csname pdfcolfoot@current\endcsname
         \unvbox \footins
       \color@endgroup
      }%
  \fi
}
%    \end{macrocode}
%  \end{macro}
%
%
%
%  \begin{macro}{\@outputbox@attachfloats}
%  \begin{macro}{\@outputbox@attachtopfloats}
%  \begin{macro}{\@outputbox@attachbottomfloats}
%    Attaching top and bottom floats can usually be done in one go,
%    but for special layouts we might want more control so we provide
%    also separate commands.
%    \begin{macrocode}
\let \@outputbox@attachfloats \@combinefloats
%    \end{macrocode}
%
%    \begin{macrocode}
\def \@outputbox@attachtopfloats {%
  \ifx \@toplist\@empty \else \@cflt \fi
}
\def \@outputbox@attachbottomfloats {%
    \ifx \@botlist\@empty \else \@cflb \fi
}
%    \end{macrocode}
%  \end{macro}
%  \end{macro}
%  \end{macro}
%
%
%
%
%  \begin{macro}{\@makecol@handlesplitfootnotes}
%  \begin{macro}{\@makecol@splitfootnotemessagehook}
%    This is only an early draft and doesn't do much.
%    Contains  incomplete preparation for tagging commented out.
% \fmi{Interfaces and code will change in the future}
%    \begin{macrocode}
\def\@makecol@handlesplitfootnotes {%
%  \ifx\splitfootnote@continuation\@empty \else
%    \setbox\footins\vbox{\splitfootnote@continuation\unvbox\footins}%
%    \global\let\splitfootnote@continuation\@empty
%  \fi
  \ifnum\insertpenalties>\z@
    \@makecol@splitfootnotemessagehook
%    \setbox\footins\vbox{\unvbox\footins --- END at split}%
%    \gdef\splitfootnote@continuation    {--- START after split}%
  \fi
}
%\def\splitfootnote@continuation{}
%    \end{macrocode}
%    This  could issue warning if split footnotes are encountered.
%    \begin{macrocode}
\let \@makecol@splitfootnotemessagehook \@empty
%    \end{macrocode}
%  \end{macro}
%  \end{macro}
%
%
%
%  \begin{macro}{\@makecol@preparefootinshook}
%
%    Configuration point to support manipulation of footins box
%    (result needs to be moved back in there). Used by the
%    \texttt{para} option.
% \fmi{Interface will change in the future}
%    \begin{macrocode}
\let \@makecol@preparefootinshook \@empty
%    \end{macrocode}
%
%  \end{macro}
%
%
%
% \fmi{Some temp interfaces until configuration points are available.}
%
%  \begin{macro}{\@if@flushbottom@TF}
%    Test for \cs{flushbottom} (currently not used).
%    \begin{macrocode}
\def\@if@flushbottom@TF{%
  \ifx\@textbottom\relax
    \expandafter\@firstoftwo
  \else
    \expandafter\@secondoftwo
  \fi
}
%    \end{macrocode}
%  \end{macro}
%
%
%  \begin{macro}{\@if@footnotes@TF}
%    Test if footnotes are present on the current page.
%    \begin{macrocode}
\def\@if@footnotes@TF{%
  \ifvoid\footins
    \expandafter\@secondoftwo
  \else
    \expandafter\@firstoftwo
  \fi
}
%    \end{macrocode}
%  \end{macro}
%
%
%  \begin{macro}{\@if@bfloats@TF}
%    Test if bottom floats are around.
%    \begin{macrocode}
\def\@if@bfloats@TF{%
  \ifx \@botlist\@empty
    \expandafter\@secondoftwo
  \else
    \expandafter\@firstoftwo
  \fi
}
%    \end{macrocode}
%  \end{macro}
%
%
%
% \subsection{The \cs{@makecol} configuration}
%
%
%  \begin{macro}{\@makecol@appendblocks}
%
%    Here is only the configuration for the default case for now,
%    others are provided by \pkg{footmisc}.
%
%    \begin{macrocode}
    \def\@makecol@appendblocks {%
       \@outputbox@appendfootnotes
       \@outputbox@attachfloats
%    \end{macrocode}
%    We do, however, reinsert the bottom skip from \cs{newpage} if it
%    was taken out earlier. This is, strictly speaking, not necessary
%    in most cases, but it is a \cs{vfil} while \cs{raggedbottom} is
%    only generating \verb=\vspace{0pt plus .0001fil}=, so if you have
%    several \cs{vfil} on the page before the \cs{newpage} you would
%    alter the space distribution if one is taken out.
%    \begin{macrocode}
       \@outputbox@reinsertbskip
    }
%    \end{macrocode}
%  \end{macro}
%
%
% \section  {Replacement for the \pkg{footmisc} package}
%
%    The replacement for \pkg{footmisc}. If the new code is used, we must substitute
%    the package:
%    \begin{macrocode}
\declare@file@substitution{footmisc.sty}{latex-lab-footmisc.ltx}
%    \end{macrocode}

%    \begin{macrocode}
%</code>
%    \end{macrocode}
%    \begin{macrocode}
%<*footmisc>
%%
%% Copyright (c) 1995-2011 Robin Fairbairns
%% Copyright (c) 2018-2022 Robin Fairbairns, Frank Mittelbach
%%
%% This file is part of the `latex-lab Bundle'.
%% --------------------------------------------
%%
%% It may be distributed and/or modified under the
%% conditions of the LaTeX Project Public License, either version 1.3c
%% of this license or (at your option) any later version.
%% The latest version of this license is in
%%    https://www.latex-project.org/lppl.txt
%% and version 1.3c or later is part of all distributions of LaTeX
%% version 2008 or later.
%%
%% This work has the LPPL maintenance status 'maintained'.
%%
%%
%% File: footmisc.dtx (C) Copyright 1995-2011 Robin Fairbairns
%%                    (C) Copyright 2018-2022 Frank Mittelbach
\NeedsTeXFormat{LaTeX2e}
\providecommand\DeclareRelease[3]{}
\providecommand\DeclareCurrentRelease[2]{}

\DeclareRelease{v5}{2011-06-06}{footmisc-2011-06-06.sty}
\DeclareCurrentRelease{}{2022-02-14}
\ProvidesPackage{latex-lab-footmisc}%
        [2022/02/14 v6.0b
     a miscellany of footnote facilities%
                   ]

\NeedsTeXFormat{LaTeX2e}[2020/10/01]
\newtoks\FN@temptoken
\providecommand\protected@writeaux{%
  \protected@write\@auxout
}
\def\l@advance@macro{\@@dvance@macro\edef}
\def\@@dvance@macro#1#2#3{\expandafter\@tempcnta#2\relax
  \advance\@tempcnta#3\relax
  #1#2{\the\@tempcnta}%
}
\let\@advance@macro\l@advance@macro
\newdimen\footnotemargin
\footnotemargin1.8em\relax
\DeclareOption{symbol}{\renewcommand\thefootnote{\fnsymbol{footnote}}}
\newif\ifFN@robust \FN@robustfalse
\DeclareOption{symbol*}{%
  \renewcommand\thefootnote{\@fnsymbol\c@footnote}%
  \FN@robusttrue
  \AtEndOfPackage{\setfnsymbol{lamport*-robust}}%
}
\newif\ifFN@para  \FN@parafalse
\DeclareOption{para}{\ifFN@sidefn
    \PackageError{footmisc}{Option "\CurrentOption" incompatible with
      option "side"}%
      {I shall ignore "\CurrentOption"}%
  \else
    \FN@paratrue
  \fi
}
\newif\ifFN@sidefn  \FN@sidefnfalse
\DeclareOption{side}{\ifFN@para
    \PackageError{footmisc}{Option "\CurrentOption" incompatible with
      option "para"}%
      {I shall ignore "\CurrentOption"}%
  \else
    \FN@sidefntrue
  \fi
}
\let\footnotelayout\@empty
\DeclareOption{ragged}{%
  \@ifundefined{RaggedRight}%
    {\renewcommand\footnotelayout{\linepenalty50 \raggedright}}%
    {\renewcommand\footnotelayout{\linepenalty50 \RaggedRight}}%
}
\newif\ifFN@perpage
\FN@perpagefalse
\DeclareOption{perpage}{%
  \FN@perpagetrue
}
\newif\ifFN@fixskip      \FN@fixskipfalse

\let\FN@bottomcases\thr@@
\newif\ifFN@abovefloats  \FN@abovefloatstrue
\DeclareOption{bottom}{%
  \let\FN@bottomcases\@ne
  \FN@abovefloatsfalse
  \FN@fixskiptrue
}
\DeclareOption{bottomfloats}{%
  \let\FN@bottomcases\tw@
  \FN@abovefloatstrue \FN@fixskiptrue
}
\DeclareOption{abovefloats}{\FN@abovefloatstrue  \FN@fixskiptrue}
\DeclareOption{belowfloats}{\FN@abovefloatsfalse \FN@fixskiptrue}
\DeclareOption{marginal}{%
  \footnotemargin-0.8em\relax
}
\DeclareOption{flushmargin}{%
  \footnotemargin0pt\relax
}
\newif\ifFN@hangfoot  \FN@hangfootfalse
\DeclareOption{hang}{%
  \FN@hangfoottrue
}
\newcommand*\hangfootparskip{0.5\baselineskip}
\newcommand*\hangfootparindent{0em}%
\DeclareOption{norule}{%
  \renewcommand\footnoterule{}%
  \advance\skip\footins 4\p@\@plus2\p@\relax
}
\DeclareOption{splitrule}{%
  \gdef\split@prev{0}
  \let\pagefootnoterule\footnoterule
  \let\mpfootnoterule\footnoterule
  \def\splitfootnoterule{\kern-3\p@ \hrule \kern2.6\p@}
  \def\footnoterule{\relax
    \ifx \@listdepth\@mplistdepth
      \mpfootnoterule
    \else
      \ifnum\split@prev=\z@
        \pagefootnoterule
      \else
        \splitfootnoterule
      \fi
      \xdef\split@prev{\the\insertpenalties}%
    \fi
  }%
}
\newif\ifFN@stablefootnote  \FN@stablefootnotefalse
\DeclareOption{stable}{\FN@stablefootnotetrue}
\newif\ifFN@multiplefootnote  \FN@multiplefootnotefalse
\DeclareOption{multiple}{\FN@multiplefootnotetrue}
\ProcessOptions
%    \end{macrocode}
%    Here we insert the new OR code if it hasn't been loaded before
%    and replace what \pkg{footmisc} has. (needs perhaps another test).
%    \begin{macrocode}
\@ifundefined{@kernel@before@cclv}
  {% \iffalse meta-comment
%
%% File: latex-lab-new-or.dtx
% Copyright (C) 2022 The LaTeX Project
%
% It may be distributed and/or modified under the conditions of the
% LaTeX Project Public License (LPPL), either version 1.3c of this
% license or (at your option) any later version.  The latest version
% of this license is in the file
%
%    https://www.latex-project.org/lppl.txt
%
%
% The development version of the bundle can be found below
%
%    https://github.com/latex3/latex2e/required/latex-lab
%
% for those people who are interested or want to report an issue.
%
%<*driver>
\documentclass{l3doc}
\EnableCrossrefs
\CodelineIndex
\begin{document}
  \DocInput{latex-lab-new-or.dtx}
\end{document}
%</driver>
%
% \fi
%
%
% \title{The \texttt{latex-lab-new-or} code\thanks{}}
% \author{Frank Mittelbach, \LaTeX{} Project}
%
% \maketitle
%
% \newcommand\fmi[1]{\begin{quote} TODO: \itshape #1\end{quote}}
% \newcommand\NEW[1]{\marginpar{\mbox{}\hfill\fbox{New: #1}}}
% \providecommand\pkg[1]{\texttt{#1}}
%
% \begin{abstract}
% \end{abstract}
%
% \section{Introduction}
%
%    This code implements changes to the output routine. It is loaded by the
%    pdfmanagement.
%
%
%
% \StopEventually{\setlength\IndexMin{200pt}  \PrintIndex  }
%
%
% \section{The Implementation}
%
%    \begin{macrocode}
%<*code>
%    \end{macrocode}
%
% \subsection{File declaration}
%    \begin{macrocode}
\ProvidesFile{latex-lab-new-or.ltx}
        [2022-02-06 v0.1 changes to the output routine]
%    \end{macrocode}
% \subsection{\cs{@makecol} reimplementation}
%
%    In order for other packages to prepend or append code to
%    \cs{@makecol}, they can use the generic command hooks
%    \texttt{cmd/@makecol/before} and \texttt{cmd/@makecol/after}, so
%    there is nothing we need to do here.
%
%
%  \begin{macro}{\@makecol}
%    \cs{@makecol} is shortened a lot, basically all the hardwired
%    code in the middle has moved into a configuration point.
%    \begin{macrocode}
\def \@makecol {%
  \@kernel@before@cclv
  \setbox\@outputbox \box\@cclv
%    \end{macrocode}
%    The only real addition is the next command which either does
%    nothing or removes an infinite glue from the bottom of the
%    \cs{@outputbox}.
%    \begin{macrocode}
  \@outputbox@removebskip
%    \end{macrocode}
%    Any ``here'' floats in the \cs{@outputbox} are now handled so we
%    recycle their registers and put them back to the \cs{@freelist}.
%    \begin{macrocode}
  \let\@elt\relax
  \xdef\@freelist{\@freelist\@midlist}%
  \global \let \@midlist \@empty
%    \end{macrocode}
%    Here we have the configurable part.
% \fmi{Interface to configuration points will change in the future}
%    \begin{macrocode}
  \@makecol@appendblocks
%    \end{macrocode}
%    The we deal with any \cs{enlargethispage} or run the normal code
%    to build a column.
%    \begin{macrocode}
  \ifvbox\@kludgeins
     \@makespecialcolbox
  \else
     \@makenormalcolbox
  \fi
  \global \maxdepth \@maxdepth
}
%    \end{macrocode}
%  \end{macro}
%
%
%  \begin{macro}{\@outputbox@depth}
%    We need to know the depth of \cs{@outputbox} once in a
%    while. Rather than using a temp dimen (as it was done in the
%    past), we give it a proper register.
%    \begin{macrocode}
\newdimen\@outputbox@depth
%    \end{macrocode}
%  \end{macro}
%
%  \begin{macro}{\@makenormalcolbox}
%    Taken out of \cs{@makecol} for readability.
%    \begin{macrocode}
\def \@makenormalcolbox {%
   \setbox\@outputbox \vbox to\@colht {%
       \@texttop
       \@outputbox@depth \dp\@outputbox
       \unvbox \@outputbox
       \vskip -\@outputbox@depth
       \@textbottom
      }%
}
%    \end{macrocode}
%  \end{macro}
%
%
%  \begin{macro}{\@makespecialcolbox}
%    Make the colbox when \cs{enlargethispage} was used.
%    \begin{macrocode}
\def \@makespecialcolbox {%
   \@outputbox@append {\vskip-\@outputbox@depth}%
   \@tempdima \@colht
   \ifdim \wd\@kludgeins>\z@
     \advance \@tempdima -\ht\@outputbox
     \advance \@tempdima \pageshrink
     \setbox\@outputbox \vbox to \@colht {%
       \unvbox\@outputbox
       \vskip \@tempdima
       \@textbottom
       }%
   \else
     \advance \@tempdima -\ht\@kludgeins
     \setbox \@outputbox \vbox to \@colht {%
       \vbox to \@tempdima {%
         \unvbox\@outputbox
         \@textbottom}%
       \vss}%
   \fi
   {\setbox \@tempboxa \box \@kludgeins}%
}
%    \end{macrocode}
%  \end{macro}
%
%  \begin{macro}{\@outputbox@removebskip}
%    The real definition for this is in \pkg{footmisc}.
%    \begin{macrocode}
\let\@outputbox@removebskip \relax
\let\@outputbox@reinsertbskip\relax
%    \end{macrocode}
%  \end{macro}


%  \begin{macro}{\@outputbox@removebskip}
%
%    This is really a bug fix for the kernel, but perhaps one has to
%    make it optional because it is in there since day one). If
%    \cs{raggedbottom} is in force, footnotes get attached to the main
%    galley at a distance of \cs{footskip} on all pages except on
%    those that are ended by \cs{newpage} or \cs{clearpage} where the
%    \cs{vfil} from \cs{newpage} pushes the footnotes to the very bottom.
%
%    This is kind of a weird difference to a page  ending with
%    \cs{pagebreak}---in that case the page is also run
%    short, but the footnotes are not pushed to the bottom.
%
%    In \pkg{footmisc} \cs{@outputbox@removebskip} is only applied when
%    \pkg{footmisc} is called with with an option specifying the
%    footnote placement, i.e., not  in the default case.
%    In new documents we apply it always.
%    \begin{macrocode}
  \def\@outputbox@removebskip{%
%    \end{macrocode}
%    We first test if we are in a \cs{raggedbottom} layout. If not we
%    do nothing, but we don't disable the code because
%    \cs{raggedbottom} may get used only for some parts of the
%    document.
%    \begin{macrocode}
    \ifx\@textbottom\relax \else
%    \end{macrocode}
%    We then append some negative glue at the end of \cs{@outputbox}
%    provided it has a glue stretch order of 1 or more (i.e., contains
%    a \texttt{fil} or \texttt{fill} part).
%    \begin{macrocode}
      \@outputbox@append{%
        \@tempskipa\lastskip
        \ifnum \gluestretchorder\@tempskipa>\z@
          \vskip-\@tempskipa
%    \end{macrocode}
%  \begin{macro}{\@outputbox@reinsertbskip}
%    We also record the value so that it can be reinserted
%    elsewhere. As we have to do this globally, we also need to
%    explicitly reset it if we don't find any such glue.
%    \begin{macrocode}
          \xdef\@outputbox@reinsertbskip
              {\noexpand\@outputbox@append{\vskip\the\@tempskipa}}%
        \else
          \global\let\@outputbox@reinsertbskip\relax
        \fi
      }%
   \fi
  }
%    \end{macrocode}
%    We need a trivial top-level definition for
%    \cs{@outputbox@reinsertbskip} in case the first page has no
%    bottom glue and the command gets called.
%    \begin{macrocode}
\let\@outputbox@reinsertbskip\relax
%    \end{macrocode}
%  \end{macro}
%  \end{macro}
%
%
%
%  \begin{macro}{\@kernel@before@cclv}
%  \begin{macro}{\@kernel@before@footins}
%    These two commands are internal kernel hooks intended for tagging
%    support in case that is active. By default they do nothing (and
%    may have been defined already by \cs{DocumentMetadata}).
%    \begin{macrocode}
\providecommand\@kernel@before@cclv{}
\providecommand\@kernel@before@footins{}
%    \end{macrocode}
%  \end{macro}
%  \end{macro}
%
%
%
%
% \subsection{The output routine configuration components}
%
%    Here we provide the components that are used to define
%    \cs{@makecol@appendblocks}.
%
%
%  \begin{macro}{@outputbox@append}
%
%    This general purpose command alters the \cs{@outputbox} box by
%    appending material to it. As this is a box typesetting operation
%    we make sure that the last line of the box reflects the true
%    depth of the last line (in case that is needed later). We also
%    expose the current depth of \cs{@outputbox} as
%    \cs{@outputbox@depth} before unboxing so that its value can be
%    used by \verb=#1= if wanted.
%    \begin{macrocode}
\def\@outputbox@append #1{%
%  \if!\detokenize{#1}!\else
     \setbox\@outputbox \vbox {%
       \boxmaxdepth \@maxdepth
       \@outputbox@depth\dp\@outputbox      % if needed in #1
       \unvbox \@outputbox
       #1%
     }%
%  \fi
}
%    \end{macrocode}
%  \end{macro}
%
%
%
%
%
%  \begin{macro}{\@outputbox@appendfootnotes}
%
%    This command appends the footnotes to the \cs{@outputbox} (if
%    there are any). If not then it does nothing.
%    \begin{macrocode}
\def\@outputbox@appendfootnotes {%
   \ifvoid\footins \else
%    \end{macrocode}
%    First come two configuration points: what to do if we are in a split
%    footnote situation and a second one that does some manipulation
%    of the \cs{footins} box before it gets appended.
% \fmi{this code will get revised as part of CP handling  in the future}
%    \begin{macrocode}
     \@makecol@handlesplitfootnotes
     \@makecol@preparefootinshook
%    \end{macrocode}
%    Then the footnotes are appended:
%    \begin{macrocode}
     \@outputbox@append{%
       \vskip \skip\footins
       \@kernel@before@footins
       \color@begingroup
         \normalcolor
         \footnoterule
%    \end{macrocode}
%    Support for \pkg{pdfcolfoot}, eventually this can go once color
%    is properly supported.
%    \begin{macrocode}
         \csname pdfcolfoot@current\endcsname
         \unvbox \footins
       \color@endgroup
      }%
  \fi
}
%    \end{macrocode}
%  \end{macro}
%
%
%
%  \begin{macro}{\@outputbox@attachfloats}
%  \begin{macro}{\@outputbox@attachtopfloats}
%  \begin{macro}{\@outputbox@attachbottomfloats}
%    Attaching top and bottom floats can usually be done in one go,
%    but for special layouts we might want more control so we provide
%    also separate commands.
%    \begin{macrocode}
\let \@outputbox@attachfloats \@combinefloats
%    \end{macrocode}
%
%    \begin{macrocode}
\def \@outputbox@attachtopfloats {%
  \ifx \@toplist\@empty \else \@cflt \fi
}
\def \@outputbox@attachbottomfloats {%
    \ifx \@botlist\@empty \else \@cflb \fi
}
%    \end{macrocode}
%  \end{macro}
%  \end{macro}
%  \end{macro}
%
%
%
%
%  \begin{macro}{\@makecol@handlesplitfootnotes}
%  \begin{macro}{\@makecol@splitfootnotemessagehook}
%    This is only an early draft and doesn't do much.
%    Contains  incomplete preparation for tagging commented out.
% \fmi{Interfaces and code will change in the future}
%    \begin{macrocode}
\def\@makecol@handlesplitfootnotes {%
%  \ifx\splitfootnote@continuation\@empty \else
%    \setbox\footins\vbox{\splitfootnote@continuation\unvbox\footins}%
%    \global\let\splitfootnote@continuation\@empty
%  \fi
  \ifnum\insertpenalties>\z@
    \@makecol@splitfootnotemessagehook
%    \setbox\footins\vbox{\unvbox\footins --- END at split}%
%    \gdef\splitfootnote@continuation    {--- START after split}%
  \fi
}
%\def\splitfootnote@continuation{}
%    \end{macrocode}
%    This  could issue warning if split footnotes are encountered.
%    \begin{macrocode}
\let \@makecol@splitfootnotemessagehook \@empty
%    \end{macrocode}
%  \end{macro}
%  \end{macro}
%
%
%
%  \begin{macro}{\@makecol@preparefootinshook}
%
%    Configuration point to support manipulation of footins box
%    (result needs to be moved back in there). Used by the
%    \texttt{para} option.
% \fmi{Interface will change in the future}
%    \begin{macrocode}
\let \@makecol@preparefootinshook \@empty
%    \end{macrocode}
%
%  \end{macro}
%
%
%
% \fmi{Some temp interfaces until configuration points are available.}
%
%  \begin{macro}{\@if@flushbottom@TF}
%    Test for \cs{flushbottom} (currently not used).
%    \begin{macrocode}
\def\@if@flushbottom@TF{%
  \ifx\@textbottom\relax
    \expandafter\@firstoftwo
  \else
    \expandafter\@secondoftwo
  \fi
}
%    \end{macrocode}
%  \end{macro}
%
%
%  \begin{macro}{\@if@footnotes@TF}
%    Test if footnotes are present on the current page.
%    \begin{macrocode}
\def\@if@footnotes@TF{%
  \ifvoid\footins
    \expandafter\@secondoftwo
  \else
    \expandafter\@firstoftwo
  \fi
}
%    \end{macrocode}
%  \end{macro}
%
%
%  \begin{macro}{\@if@bfloats@TF}
%    Test if bottom floats are around.
%    \begin{macrocode}
\def\@if@bfloats@TF{%
  \ifx \@botlist\@empty
    \expandafter\@secondoftwo
  \else
    \expandafter\@firstoftwo
  \fi
}
%    \end{macrocode}
%  \end{macro}
%
%
%
% \subsection{The \cs{@makecol} configuration}
%
%
%  \begin{macro}{\@makecol@appendblocks}
%
%    Here is only the configuration for the default case for now,
%    others are provided by \pkg{footmisc}.
%
%    \begin{macrocode}
    \def\@makecol@appendblocks {%
       \@outputbox@appendfootnotes
       \@outputbox@attachfloats
%    \end{macrocode}
%    We do, however, reinsert the bottom skip from \cs{newpage} if it
%    was taken out earlier. This is, strictly speaking, not necessary
%    in most cases, but it is a \cs{vfil} while \cs{raggedbottom} is
%    only generating \verb=\vspace{0pt plus .0001fil}=, so if you have
%    several \cs{vfil} on the page before the \cs{newpage} you would
%    alter the space distribution if one is taken out.
%    \begin{macrocode}
       \@outputbox@reinsertbskip
    }
%    \end{macrocode}
%  \end{macro}
%
%
% \section  {Replacement for the \pkg{footmisc} package}
%
%    The replacement for \pkg{footmisc}. If the new code is used, we must substitute
%    the package:
%    \begin{macrocode}
\declare@file@substitution{footmisc.sty}{latex-lab-footmisc.ltx}
%    \end{macrocode}

%    \begin{macrocode}
%</code>
%    \end{macrocode}
%    \begin{macrocode}
%<*footmisc>
%%
%% Copyright (c) 1995-2011 Robin Fairbairns
%% Copyright (c) 2018-2022 Robin Fairbairns, Frank Mittelbach
%%
%% This file is part of the `latex-lab Bundle'.
%% --------------------------------------------
%%
%% It may be distributed and/or modified under the
%% conditions of the LaTeX Project Public License, either version 1.3c
%% of this license or (at your option) any later version.
%% The latest version of this license is in
%%    https://www.latex-project.org/lppl.txt
%% and version 1.3c or later is part of all distributions of LaTeX
%% version 2008 or later.
%%
%% This work has the LPPL maintenance status 'maintained'.
%%
%%
%% File: footmisc.dtx (C) Copyright 1995-2011 Robin Fairbairns
%%                    (C) Copyright 2018-2022 Frank Mittelbach
\NeedsTeXFormat{LaTeX2e}
\providecommand\DeclareRelease[3]{}
\providecommand\DeclareCurrentRelease[2]{}

\DeclareRelease{v5}{2011-06-06}{footmisc-2011-06-06.sty}
\DeclareCurrentRelease{}{2022-02-14}
\ProvidesPackage{latex-lab-footmisc}%
        [2022/02/14 v6.0b
     a miscellany of footnote facilities%
                   ]

\NeedsTeXFormat{LaTeX2e}[2020/10/01]
\newtoks\FN@temptoken
\providecommand\protected@writeaux{%
  \protected@write\@auxout
}
\def\l@advance@macro{\@@dvance@macro\edef}
\def\@@dvance@macro#1#2#3{\expandafter\@tempcnta#2\relax
  \advance\@tempcnta#3\relax
  #1#2{\the\@tempcnta}%
}
\let\@advance@macro\l@advance@macro
\newdimen\footnotemargin
\footnotemargin1.8em\relax
\DeclareOption{symbol}{\renewcommand\thefootnote{\fnsymbol{footnote}}}
\newif\ifFN@robust \FN@robustfalse
\DeclareOption{symbol*}{%
  \renewcommand\thefootnote{\@fnsymbol\c@footnote}%
  \FN@robusttrue
  \AtEndOfPackage{\setfnsymbol{lamport*-robust}}%
}
\newif\ifFN@para  \FN@parafalse
\DeclareOption{para}{\ifFN@sidefn
    \PackageError{footmisc}{Option "\CurrentOption" incompatible with
      option "side"}%
      {I shall ignore "\CurrentOption"}%
  \else
    \FN@paratrue
  \fi
}
\newif\ifFN@sidefn  \FN@sidefnfalse
\DeclareOption{side}{\ifFN@para
    \PackageError{footmisc}{Option "\CurrentOption" incompatible with
      option "para"}%
      {I shall ignore "\CurrentOption"}%
  \else
    \FN@sidefntrue
  \fi
}
\let\footnotelayout\@empty
\DeclareOption{ragged}{%
  \@ifundefined{RaggedRight}%
    {\renewcommand\footnotelayout{\linepenalty50 \raggedright}}%
    {\renewcommand\footnotelayout{\linepenalty50 \RaggedRight}}%
}
\newif\ifFN@perpage
\FN@perpagefalse
\DeclareOption{perpage}{%
  \FN@perpagetrue
}
\newif\ifFN@fixskip      \FN@fixskipfalse

\let\FN@bottomcases\thr@@
\newif\ifFN@abovefloats  \FN@abovefloatstrue
\DeclareOption{bottom}{%
  \let\FN@bottomcases\@ne
  \FN@abovefloatsfalse
  \FN@fixskiptrue
}
\DeclareOption{bottomfloats}{%
  \let\FN@bottomcases\tw@
  \FN@abovefloatstrue \FN@fixskiptrue
}
\DeclareOption{abovefloats}{\FN@abovefloatstrue  \FN@fixskiptrue}
\DeclareOption{belowfloats}{\FN@abovefloatsfalse \FN@fixskiptrue}
\DeclareOption{marginal}{%
  \footnotemargin-0.8em\relax
}
\DeclareOption{flushmargin}{%
  \footnotemargin0pt\relax
}
\newif\ifFN@hangfoot  \FN@hangfootfalse
\DeclareOption{hang}{%
  \FN@hangfoottrue
}
\newcommand*\hangfootparskip{0.5\baselineskip}
\newcommand*\hangfootparindent{0em}%
\DeclareOption{norule}{%
  \renewcommand\footnoterule{}%
  \advance\skip\footins 4\p@\@plus2\p@\relax
}
\DeclareOption{splitrule}{%
  \gdef\split@prev{0}
  \let\pagefootnoterule\footnoterule
  \let\mpfootnoterule\footnoterule
  \def\splitfootnoterule{\kern-3\p@ \hrule \kern2.6\p@}
  \def\footnoterule{\relax
    \ifx \@listdepth\@mplistdepth
      \mpfootnoterule
    \else
      \ifnum\split@prev=\z@
        \pagefootnoterule
      \else
        \splitfootnoterule
      \fi
      \xdef\split@prev{\the\insertpenalties}%
    \fi
  }%
}
\newif\ifFN@stablefootnote  \FN@stablefootnotefalse
\DeclareOption{stable}{\FN@stablefootnotetrue}
\newif\ifFN@multiplefootnote  \FN@multiplefootnotefalse
\DeclareOption{multiple}{\FN@multiplefootnotetrue}
\ProcessOptions
%    \end{macrocode}
%    Here we insert the new OR code if it hasn't been loaded before
%    and replace what \pkg{footmisc} has. (needs perhaps another test).
%    \begin{macrocode}
\@ifundefined{@kernel@before@cclv}
  {\input{latex-lab-new-or.ltx}}{}
%    \end{macrocode}
%
%    Footnote box layout for para footnotes;
%    this would also be the hook to support dblfootnotes (from the
%    \texttt{dblfnote} package if we integrate that).
%    \begin{macrocode}
\ifFN@para
  \def\@makecol@preparefootinshook {%
     \global\setbox\footins\vbox{\FN@makefootnoteparagraph}%
    }
\fi
%    \end{macrocode}
%
%    \begin{macrocode}
\ifFN@fixskip
  \def\@outputbox@removebskip{%
    \ifx\@textbottom\relax \else
      \@outputbox@append{%
        \@tempskipa\lastskip
        \ifnum \gluestretchorder\@tempskipa>\z@
          \vskip-\@tempskipa
          \xdef\@outputbox@reinsertbskip
              {\noexpand\@outputbox@append{\vskip\the\@tempskipa}}%
        \else
          \global\let\@outputbox@reinsertbskip\relax
        \fi
      }%
   \fi
  }
\let\@outputbox@reinsertbskip\relax
\else
  \let\@outputbox@removebskip \relax
  \let\@outputbox@reinsertbskip\relax
\fi
%    \end{macrocode}
%
%
%
%    \begin{macrocode}
\ifcase \FN@bottomcases\relax
\ERROR
\or
  \ifFN@abovefloats
    \def\@makecol@appendblocks {%
       \@if@footnotes@TF
          {\@outputbox@append{\vfill}}%
          {\@if@bfloats@TF{\@outputbox@append{\vfill}}%
                          {\@outputbox@reinsertbskip}}%
       \@outputbox@appendfootnotes
       \@outputbox@attachfloats
      }
  \else
    \def\@makecol@appendblocks {%
       \@outputbox@attachfloats
       \@if@footnotes@TF
          {\@outputbox@append{\vfill}}%
          {\@outputbox@reinsertbskip}%
       \@outputbox@appendfootnotes
    }
  \fi
\or
  \ifFN@abovefloats
     \def\@makecol@appendblocks {%
        \@outputbox@appendfootnotes
        \@if@bfloats@TF
            {\@outputbox@append{\vfill}}%
            {\@outputbox@reinsertbskip}%
        \@outputbox@attachfloats
     }
  \else
     \def\@makecol@appendblocks {%
       \@if@footnotes@TF
          {\@outputbox@append{\vfill}}%
          {\@if@bfloats@TF{\@outputbox@append{\vfill}}%
                          {\@outputbox@reinsertbskip}}%
        \@outputbox@attachfloats
        \@outputbox@appendfootnotes
     }
  \fi
\or
  \ifFN@abovefloats
    \def\@makecol@appendblocks {%
       \@outputbox@appendfootnotes
       \@outputbox@attachfloats
       \@outputbox@reinsertbskip
    }
  \else
    \def\@makecol@appendblocks {%
       \@outputbox@attachfloats
       \@outputbox@appendfootnotes
       \@outputbox@reinsertbskip
}
  \fi
\else
\ERROR
\fi

\newif\ifFN@setspace
\@ifpackageloaded{setspace}{%
  \FN@setspacetrue
  \@ifclassloaded{memoir}{%
    \let\FN@baselinestretch\m@m@singlespace
  }{%
    \let\FN@baselinestretch\setspace@singlespace
  }%
}{%
  \FN@setspacefalse
}
\ifFN@para
  \long\def\FN@footnotetext#1{%
    \insert\footins{%
      \ifFN@setspace
        \let\baselinestretch\FN@baselinestretch
      \fi
      \reset@font\footnotesize
      \interlinepenalty\interfootnotelinepenalty
      \splittopskip\footnotesep
      \splitmaxdepth \dp\strutbox
      \floatingpenalty\@MM
      \hsize\columnwidth
      \@parboxrestore
      \protected@edef\@currentlabel{\csname p@footnote\endcsname\@thefnmark}%
      \color@begingroup
        \setbox\FN@tempboxa\hbox{%
          \@makefntext{\ignorespaces#1\strut
            \penalty-10\relax
            \hskip\footglue
          }% end of \@makefntext parameter
        }% end of \hbox
        \dp\FN@tempboxa\z@
        \ht\FN@tempboxa\dimexpr\wd\FN@tempboxa *%
                        \footnotebaselineskip / \columnwidth\relax
        \box\FN@tempboxa
      \color@endgroup
    }%
    \FN@mf@prepare
  }
\else
  \ifFN@sidefn
    \long\def\FN@footnotetext#1{%
      \marginpar{%
        \ifFN@setspace
          \let\baselinestretch\FN@baselinestretch
        \fi
        \reset@font\footnotesize
        \protected@edef\@currentlabel{%
          \csname p@footnote\endcsname\@thefnmark
        }%
        \color@begingroup
          \@makefntext{%
            \ignorespaces#1%
          }%
        \color@endgroup
      }%
      \FN@mf@prepare
    }%
  \else
    \long\def\FN@footnotetext#1{%
      \insert\footins{%
        \ifFN@setspace
          \let\baselinestretch\FN@baselinestretch
        \fi
        \reset@font\footnotesize
        \interlinepenalty\interfootnotelinepenalty
        \splittopskip\footnotesep
        \splitmaxdepth \dp\strutbox
        \floatingpenalty\@MM
        \hsize\columnwidth
        \@parboxrestore
        \protected@edef\@currentlabel{%
          \csname p@footnote\endcsname\@thefnmark
        }%
        \color@begingroup
          \@makefntext{%
            \rule\z@\footnotesep
            \ignorespaces#1\@finalstrut\strutbox
          }%
        \color@endgroup
      }%
      \FN@mf@prepare
    }%
  \fi
\fi
\ifFN@para
  \let\FN@tempboxa\@tempboxa
  \newbox\FN@tempboxb
  \newbox\FN@tempboxc
  \newskip\footglue \footglue=1em plus.3em minus.3em
  \long\def\@makefntext#1{\leavevmode
    \@makefnmark\nobreak
    \hskip.5em\relax#1%
  }
%%%%%%%%%%%%%%%%%%%%%%%%%%%%%%%%%%%%%%%%%%%%%%%%%%%%%%%%%%%%%%%%%%%%%%%%%%%%%
  \newdimen\footnotebaselineskip
  {%
    \footnotesize
    \global
      \footnotebaselineskip=\normalbaselineskip
  }

  \long\def\FN@makefootnoteparagraph{\unvbox\footins \FN@makehboxofhboxes
    \setbox\FN@tempboxa=\hbox{\unhbox\FN@tempboxa \FN@removehboxes}%
    \FN@setfootnoteparawidth
    \@parboxrestore
    \baselineskip=\footnotebaselineskip
    \noindent
    \rule{\z@}{\footnotesep}%
    \unhbox\FN@tempboxa\par
  }
  \def\FN@makehboxofhboxes{\setbox\FN@tempboxa=\hbox{}%
    \loop
      \setbox\FN@tempboxb=\lastbox
      \ifhbox\FN@tempboxb
      \setbox\FN@tempboxa=\hbox{\box\FN@tempboxb\unhbox\FN@tempboxa}%
    \repeat
  }
  \def\FN@removehboxes{\setbox\FN@tempboxa=\lastbox
    \ifhbox
      \FN@tempboxa{\FN@removehboxes}%
      \unhbox\FN@tempboxa
    \fi
  }
\fi
\@ifpackageloaded{multicol}
  {\def\FN@setfootnoteparawidth
    {\hsize\ifnum\doublecol@number>\@ne
                  \textwidth
            \else \columnwidth \fi}}
  {\def\FN@setfootnoteparawidth{\hsize\columnwidth}}
\ifFN@perpage
  \RequirePackage{perpage}
  \MakePerPage{footnote}
\fi
\ifFN@para
\else
  \long\def\@makefntext#1{%
    \ifFN@hangfoot
      \bgroup
      \setbox\@tempboxa\hbox{%
        \ifdim\footnotemargin>0pt
          \hb@xt@\footnotemargin{\@makefnmark\hss}%
        \else
          \@makefnmark
        \fi
      }%
      \leftmargin\wd\@tempboxa
      \rightmargin\z@
      \linewidth \columnwidth
      \advance \linewidth -\leftmargin
      \parshape \@ne \leftmargin \linewidth
      \footnotesize
      \@setpar{{\@@par}}%
      \leavevmode
      \llap{\box\@tempboxa}%
      \parskip\hangfootparskip\relax
      \parindent\hangfootparindent\relax
    \else
      \parindent1em
      \noindent
      \ifdim\footnotemargin>\z@
        \hb@xt@ \footnotemargin{\hss\@makefnmark}%
      \else
        \ifdim\footnotemargin=\z@
          \llap{\@makefnmark}%
        \else
          \llap{\hb@xt@ -\footnotemargin{\@makefnmark\hss}}%
        \fi
      \fi
    \fi
    \footnotelayout#1%
    \ifFN@hangfoot
      \par\egroup
    \fi
  }
\fi
\ifFN@multiplefootnote
  \providecommand*{\multiplefootnotemarker}{3sp}
  \providecommand*{\multfootsep}{,}
  \newcommand*\FN@footnotemark{%
    \leavevmode
    \ifhmode
      \edef\@x@sf{\the\spacefactor}%
      \FN@mf@check
      \nobreak
    \fi
    \@makefnmark
    \FN@mf@prepare
    \ifhmode\spacefactor\@x@sf\fi
    \relax
  }
  \def\FN@mf@prepare{%
    \kern-\multiplefootnotemarker
    \kern\multiplefootnotemarker\relax
  }
  \def\FN@mf@check{%
    \ifdim\lastkern=\multiplefootnotemarker\relax
      \edef\@x@sf{\the\spacefactor}%
      \unkern
      \textsuperscript{\multfootsep}%
      \spacefactor\@x@sf\relax
    \fi
  }
\else
  \let\FN@mf@prepare\relax
  \let\FN@footnotemark\@footnotemark
\fi
\ifFN@stablefootnote
\let\FN@sf@@footnote\footnote
\def\footnote{\ifx\protect\@typeset@protect
    \expandafter\FN@sf@@footnote
  \else
    \expandafter\FN@sf@gobble@opt
  \fi
}
\edef\FN@sf@gobble@opt{\noexpand\protect
  \expandafter\noexpand\csname FN@sf@gobble@opt \endcsname}
\expandafter\def\csname FN@sf@gobble@opt \endcsname{%
  \@ifnextchar[%]
    \FN@sf@gobble@twobracket
    \@gobble
}
\def\FN@sf@gobble@twobracket[#1]#2{}
\let\FN@sf@@footnotemark\footnotemark
\def\footnotemark{\ifx\protect\@typeset@protect
    \expandafter\FN@sf@@footnotemark
  \else
    \expandafter\FN@sf@gobble@optonly
  \fi
}
\edef\FN@sf@gobble@optonly{\noexpand\protect
  \expandafter\noexpand\csname FN@sf@gobble@optonly \endcsname}
\expandafter\def\csname FN@sf@gobble@optonly \endcsname{%
  \@ifnextchar[%]
    \FN@sf@gobble@bracket
    {}%
}
\def\FN@sf@gobble@bracket[#1]{}
\fi
\newcommand\setfnsymbol[1]{%
  \@bsphack
  \@ifundefined{FN@fnsymbol@#1}%
  {%
    \PackageError{footmisc}{Symbol style "#1" not known}%
    \@eha
  }{%
    \expandafter\let\expandafter\@fnsymbol\csname
                        FN@fnsymbol@#1\endcsname
  }%
  \@esphack
}
\let\FN@fnsymbol@lamport\@fnsymbol
\newif\if@tempswb
\DeclareDocumentCommand\DefineFNsymbols {smO{text}m}{%
  \expandafter\ifx\csname FN@fnsymbol@#2\endcsname\relax
    \PackageInfo{footmisc}{Declaring symbol style #2}%
  \else
    \PackageWarning{footmisc}{Redeclaring symbol style #2}%
  \fi
  \toks@{}%
  \def\@tempb{\end}%
  \FN@build@symboldef#4\end
  \def\@tempc{math}%
  \def\@tempd{#3}%
  \expandafter\xdef\csname FN@fnsymbol@#2\endcsname##1{%
    \ifx\@tempc\@tempd
      \noexpand\ensuremath
    \else
      \noexpand\nfss@text
    \fi
    {%
      \noexpand\ifcase##1%
      \the\toks@
      \noexpand\else
      \IfBooleanTF#1{\noexpand\@ctrerr}%
        {\noexpand\FN@orange##1}%
      \noexpand\fi
    }%
  }%
}
\def\FN@build@symboldef#1{%
  \def\@tempa{#1}%
  \ifx\@tempa\@tempb
  \else
    \toks@\expandafter{\the\toks@\or#1}%
    \expandafter\FN@build@symboldef
  \fi
}
\DeclareDocumentCommand\DefineFNsymbolsTM {smm}{%
  \expandafter\ifx\csname FN@fnsymbol@#2\endcsname\relax
    \PackageInfo{footmisc}{Declaring symbol style #2}%
  \else
    \PackageWarning{footmisc}{Redeclaring symbol style #2}%
  \fi
  \toks@{}%
  \def\@tempb{\end}%
  \FN@build@symboldefTM#3\end\@null
  \expandafter\xdef\csname FN@fnsymbol@#2\endcsname##1{%
    \noexpand\ifcase##1%
      \the\toks@
    \noexpand\else
      \IfBooleanTF#1{\noexpand\@ctrerr}%
        {\noexpand\FN@orange##1}%
      \noexpand\fi
  }%
}
\def\FN@build@symboldefTM#1#2{%
  \def\@tempa{#1}%
  \ifx\@tempa\@tempb
  \else
    \toks@\expandafter{\the\toks@\or\TextOrMath{#1}{#2}}%
    \expandafter\FN@build@symboldefTM
  \fi
}
\def\FN@orange#1{%
  \ifFN@robust
    \@arabic#1%
    \@bsphack
    \PackageInfo{footmisc}{Footnote number \number#1 out of range}%
    \protect\@fnsymbol@orange
    \@esphack
  \else \@ctrerr \fi
}
\global\let\@diagnose@fnsymbol@orange\relax
\AtEndDocument{\@diagnose@fnsymbol@orange}
\def\@fnsymbol@orange{%
  \gdef\@diagnose@fnsymbol@orange{%
    \PackageWarningNoLine{footmisc}{Some footnote number(s)
      were out of range
      \MessageBreak
      see log for details%
    }%
  }%
}
\DefineFNsymbolsTM{bringhurst}{%
  \textasteriskcentered *%
  \textdagger    \dagger
  \textdaggerdbl \ddagger
  \textsection   \mathsection
  \textbardbl    \|%
  \textparagraph \mathparagraph
}%
\DefineFNsymbolsTM{chicago}{%
  \textasteriskcentered *%
  \textdagger    \dagger
  \textdaggerdbl \ddagger
  \textsection   \mathsection
  \textbardbl    \|%
  \#\#%
}%
\DefineFNsymbolsTM{wiley}{%
  \textasteriskcentered *%
  {\textasteriskcentered\textasteriskcentered}{**}%
  \textdagger    \dagger
  \textdaggerdbl \ddagger
  \textsection   \mathsection
  \textparagraph \mathparagraph
  \textbardbl    \|%
}%
\DefineFNsymbolsTM{lamport-robust}{%
  \textasteriskcentered *%
  \textdagger    \dagger
  \textdaggerdbl \ddagger
  \textsection   \mathsection
  \textparagraph \mathparagraph
  \textbardbl    \|%
  {\textasteriskcentered\textasteriskcentered}{**}%
  {\textdagger\textdagger}{\dagger\dagger}%
  {\textdaggerdbl\textdaggerdbl}{\ddagger\ddagger}%
}
\DefineFNsymbolsTM*{lamport*}{%
  \textasteriskcentered *%
  \textdagger    \dagger
  \textdaggerdbl \ddagger
  \textsection   \mathsection
  \textparagraph \mathparagraph
  \textbardbl    \|%
  {\textasteriskcentered\textasteriskcentered}{**}%
  {\textdagger\textdagger}{\dagger\dagger}%
  {\textdaggerdbl\textdaggerdbl}{\ddagger\ddagger}%
  {\textsection\textsection}{\mathsection\mathsection}%
  {\textparagraph\textparagraph}{\mathparagraph\mathparagraph}%
  {\textasteriskcentered\textasteriskcentered\textasteriskcentered}{***}%
  {\textdagger\textdagger\textdagger}{\dagger\dagger\dagger}%
  {\textdaggerdbl\textdaggerdbl\textdaggerdbl}{\ddagger\ddagger\ddagger}%
  {\textsection\textsection\textsection}%%
    {\mathsection\mathsection\mathsection}%
  {\textparagraph\textparagraph\textparagraph}%%
    {\mathparagraph\mathparagraph\mathparagraph}%
}
\setfnsymbol{lamport*}
\DefineFNsymbolsTM{lamport*-robust}{%
  \textasteriskcentered *%
  \textdagger    \dagger
  \textdaggerdbl \ddagger
  \textsection   \mathsection
  \textparagraph \mathparagraph
  \textbardbl    \|%
  {\textasteriskcentered\textasteriskcentered}{**}%
  {\textdagger\textdagger}{\dagger\dagger}%
  {\textdaggerdbl\textdaggerdbl}{\ddagger\ddagger}%
  {\textsection\textsection}{\mathsection\mathsection}%
  {\textparagraph\textparagraph}{\mathparagraph\mathparagraph}%
  {\textasteriskcentered\textasteriskcentered\textasteriskcentered}{***}%
  {\textdagger\textdagger\textdagger}{\dagger\dagger\dagger}%
  {\textdaggerdbl\textdaggerdbl\textdaggerdbl}{\ddagger\ddagger\ddagger}%
  {\textsection\textsection\textsection}%%
    {\mathsection\mathsection\mathsection}%
  {\textparagraph\textparagraph\textparagraph}%%
    {\mathparagraph\mathparagraph\mathparagraph}%
}
\newcommand\mpfootnotemark{%
  \@ifnextchar[%
    \@xmpfootnotemark
    {%
      \stepcounter\@mpfn
      \protected@xdef\@thefnmark{\thempfn}%
      \@footnotemark
    }%
}
\def\@xmpfootnotemark[#1]{%
  \begingroup
    \csname c@\@mpfn\endcsname #1\relax
    \unrestored@protected@xdef\@thefnmark{\thempfn}%
  \endgroup
  \@footnotemark
}
\@ifpackageloaded{hyperref}{%
  \let\H@@footnotetext\FN@footnotetext
  \let\H@@footnotemark\FN@footnotemark
}{%
  \let \@footnotetext \FN@footnotetext
  \let\@footnotemark  \FN@footnotemark
}
\endinput
%</footmisc>
%    \end{macrocode}
% \Finale
%
}{}
%    \end{macrocode}
%
%    Footnote box layout for para footnotes;
%    this would also be the hook to support dblfootnotes (from the
%    \texttt{dblfnote} package if we integrate that).
%    \begin{macrocode}
\ifFN@para
  \def\@makecol@preparefootinshook {%
     \global\setbox\footins\vbox{\FN@makefootnoteparagraph}%
    }
\fi
%    \end{macrocode}
%
%    \begin{macrocode}
\ifFN@fixskip
  \def\@outputbox@removebskip{%
    \ifx\@textbottom\relax \else
      \@outputbox@append{%
        \@tempskipa\lastskip
        \ifnum \gluestretchorder\@tempskipa>\z@
          \vskip-\@tempskipa
          \xdef\@outputbox@reinsertbskip
              {\noexpand\@outputbox@append{\vskip\the\@tempskipa}}%
        \else
          \global\let\@outputbox@reinsertbskip\relax
        \fi
      }%
   \fi
  }
\let\@outputbox@reinsertbskip\relax
\else
  \let\@outputbox@removebskip \relax
  \let\@outputbox@reinsertbskip\relax
\fi
%    \end{macrocode}
%
%
%
%    \begin{macrocode}
\ifcase \FN@bottomcases\relax
\ERROR
\or
  \ifFN@abovefloats
    \def\@makecol@appendblocks {%
       \@if@footnotes@TF
          {\@outputbox@append{\vfill}}%
          {\@if@bfloats@TF{\@outputbox@append{\vfill}}%
                          {\@outputbox@reinsertbskip}}%
       \@outputbox@appendfootnotes
       \@outputbox@attachfloats
      }
  \else
    \def\@makecol@appendblocks {%
       \@outputbox@attachfloats
       \@if@footnotes@TF
          {\@outputbox@append{\vfill}}%
          {\@outputbox@reinsertbskip}%
       \@outputbox@appendfootnotes
    }
  \fi
\or
  \ifFN@abovefloats
     \def\@makecol@appendblocks {%
        \@outputbox@appendfootnotes
        \@if@bfloats@TF
            {\@outputbox@append{\vfill}}%
            {\@outputbox@reinsertbskip}%
        \@outputbox@attachfloats
     }
  \else
     \def\@makecol@appendblocks {%
       \@if@footnotes@TF
          {\@outputbox@append{\vfill}}%
          {\@if@bfloats@TF{\@outputbox@append{\vfill}}%
                          {\@outputbox@reinsertbskip}}%
        \@outputbox@attachfloats
        \@outputbox@appendfootnotes
     }
  \fi
\or
  \ifFN@abovefloats
    \def\@makecol@appendblocks {%
       \@outputbox@appendfootnotes
       \@outputbox@attachfloats
       \@outputbox@reinsertbskip
    }
  \else
    \def\@makecol@appendblocks {%
       \@outputbox@attachfloats
       \@outputbox@appendfootnotes
       \@outputbox@reinsertbskip
}
  \fi
\else
\ERROR
\fi

\newif\ifFN@setspace
\@ifpackageloaded{setspace}{%
  \FN@setspacetrue
  \@ifclassloaded{memoir}{%
    \let\FN@baselinestretch\m@m@singlespace
  }{%
    \let\FN@baselinestretch\setspace@singlespace
  }%
}{%
  \FN@setspacefalse
}
\ifFN@para
  \long\def\FN@footnotetext#1{%
    \insert\footins{%
      \ifFN@setspace
        \let\baselinestretch\FN@baselinestretch
      \fi
      \reset@font\footnotesize
      \interlinepenalty\interfootnotelinepenalty
      \splittopskip\footnotesep
      \splitmaxdepth \dp\strutbox
      \floatingpenalty\@MM
      \hsize\columnwidth
      \@parboxrestore
      \protected@edef\@currentlabel{\csname p@footnote\endcsname\@thefnmark}%
      \color@begingroup
        \setbox\FN@tempboxa\hbox{%
          \@makefntext{\ignorespaces#1\strut
            \penalty-10\relax
            \hskip\footglue
          }% end of \@makefntext parameter
        }% end of \hbox
        \dp\FN@tempboxa\z@
        \ht\FN@tempboxa\dimexpr\wd\FN@tempboxa *%
                        \footnotebaselineskip / \columnwidth\relax
        \box\FN@tempboxa
      \color@endgroup
    }%
    \FN@mf@prepare
  }
\else
  \ifFN@sidefn
    \long\def\FN@footnotetext#1{%
      \marginpar{%
        \ifFN@setspace
          \let\baselinestretch\FN@baselinestretch
        \fi
        \reset@font\footnotesize
        \protected@edef\@currentlabel{%
          \csname p@footnote\endcsname\@thefnmark
        }%
        \color@begingroup
          \@makefntext{%
            \ignorespaces#1%
          }%
        \color@endgroup
      }%
      \FN@mf@prepare
    }%
  \else
    \long\def\FN@footnotetext#1{%
      \insert\footins{%
        \ifFN@setspace
          \let\baselinestretch\FN@baselinestretch
        \fi
        \reset@font\footnotesize
        \interlinepenalty\interfootnotelinepenalty
        \splittopskip\footnotesep
        \splitmaxdepth \dp\strutbox
        \floatingpenalty\@MM
        \hsize\columnwidth
        \@parboxrestore
        \protected@edef\@currentlabel{%
          \csname p@footnote\endcsname\@thefnmark
        }%
        \color@begingroup
          \@makefntext{%
            \rule\z@\footnotesep
            \ignorespaces#1\@finalstrut\strutbox
          }%
        \color@endgroup
      }%
      \FN@mf@prepare
    }%
  \fi
\fi
\ifFN@para
  \let\FN@tempboxa\@tempboxa
  \newbox\FN@tempboxb
  \newbox\FN@tempboxc
  \newskip\footglue \footglue=1em plus.3em minus.3em
  \long\def\@makefntext#1{\leavevmode
    \@makefnmark\nobreak
    \hskip.5em\relax#1%
  }
%%%%%%%%%%%%%%%%%%%%%%%%%%%%%%%%%%%%%%%%%%%%%%%%%%%%%%%%%%%%%%%%%%%%%%%%%%%%%
  \newdimen\footnotebaselineskip
  {%
    \footnotesize
    \global
      \footnotebaselineskip=\normalbaselineskip
  }

  \long\def\FN@makefootnoteparagraph{\unvbox\footins \FN@makehboxofhboxes
    \setbox\FN@tempboxa=\hbox{\unhbox\FN@tempboxa \FN@removehboxes}%
    \FN@setfootnoteparawidth
    \@parboxrestore
    \baselineskip=\footnotebaselineskip
    \noindent
    \rule{\z@}{\footnotesep}%
    \unhbox\FN@tempboxa\par
  }
  \def\FN@makehboxofhboxes{\setbox\FN@tempboxa=\hbox{}%
    \loop
      \setbox\FN@tempboxb=\lastbox
      \ifhbox\FN@tempboxb
      \setbox\FN@tempboxa=\hbox{\box\FN@tempboxb\unhbox\FN@tempboxa}%
    \repeat
  }
  \def\FN@removehboxes{\setbox\FN@tempboxa=\lastbox
    \ifhbox
      \FN@tempboxa{\FN@removehboxes}%
      \unhbox\FN@tempboxa
    \fi
  }
\fi
\@ifpackageloaded{multicol}
  {\def\FN@setfootnoteparawidth
    {\hsize\ifnum\doublecol@number>\@ne
                  \textwidth
            \else \columnwidth \fi}}
  {\def\FN@setfootnoteparawidth{\hsize\columnwidth}}
\ifFN@perpage
  \RequirePackage{perpage}
  \MakePerPage{footnote}
\fi
\ifFN@para
\else
  \long\def\@makefntext#1{%
    \ifFN@hangfoot
      \bgroup
      \setbox\@tempboxa\hbox{%
        \ifdim\footnotemargin>0pt
          \hb@xt@\footnotemargin{\@makefnmark\hss}%
        \else
          \@makefnmark
        \fi
      }%
      \leftmargin\wd\@tempboxa
      \rightmargin\z@
      \linewidth \columnwidth
      \advance \linewidth -\leftmargin
      \parshape \@ne \leftmargin \linewidth
      \footnotesize
      \@setpar{{\@@par}}%
      \leavevmode
      \llap{\box\@tempboxa}%
      \parskip\hangfootparskip\relax
      \parindent\hangfootparindent\relax
    \else
      \parindent1em
      \noindent
      \ifdim\footnotemargin>\z@
        \hb@xt@ \footnotemargin{\hss\@makefnmark}%
      \else
        \ifdim\footnotemargin=\z@
          \llap{\@makefnmark}%
        \else
          \llap{\hb@xt@ -\footnotemargin{\@makefnmark\hss}}%
        \fi
      \fi
    \fi
    \footnotelayout#1%
    \ifFN@hangfoot
      \par\egroup
    \fi
  }
\fi
\ifFN@multiplefootnote
  \providecommand*{\multiplefootnotemarker}{3sp}
  \providecommand*{\multfootsep}{,}
  \newcommand*\FN@footnotemark{%
    \leavevmode
    \ifhmode
      \edef\@x@sf{\the\spacefactor}%
      \FN@mf@check
      \nobreak
    \fi
    \@makefnmark
    \FN@mf@prepare
    \ifhmode\spacefactor\@x@sf\fi
    \relax
  }
  \def\FN@mf@prepare{%
    \kern-\multiplefootnotemarker
    \kern\multiplefootnotemarker\relax
  }
  \def\FN@mf@check{%
    \ifdim\lastkern=\multiplefootnotemarker\relax
      \edef\@x@sf{\the\spacefactor}%
      \unkern
      \textsuperscript{\multfootsep}%
      \spacefactor\@x@sf\relax
    \fi
  }
\else
  \let\FN@mf@prepare\relax
  \let\FN@footnotemark\@footnotemark
\fi
\ifFN@stablefootnote
\let\FN@sf@@footnote\footnote
\def\footnote{\ifx\protect\@typeset@protect
    \expandafter\FN@sf@@footnote
  \else
    \expandafter\FN@sf@gobble@opt
  \fi
}
\edef\FN@sf@gobble@opt{\noexpand\protect
  \expandafter\noexpand\csname FN@sf@gobble@opt \endcsname}
\expandafter\def\csname FN@sf@gobble@opt \endcsname{%
  \@ifnextchar[%]
    \FN@sf@gobble@twobracket
    \@gobble
}
\def\FN@sf@gobble@twobracket[#1]#2{}
\let\FN@sf@@footnotemark\footnotemark
\def\footnotemark{\ifx\protect\@typeset@protect
    \expandafter\FN@sf@@footnotemark
  \else
    \expandafter\FN@sf@gobble@optonly
  \fi
}
\edef\FN@sf@gobble@optonly{\noexpand\protect
  \expandafter\noexpand\csname FN@sf@gobble@optonly \endcsname}
\expandafter\def\csname FN@sf@gobble@optonly \endcsname{%
  \@ifnextchar[%]
    \FN@sf@gobble@bracket
    {}%
}
\def\FN@sf@gobble@bracket[#1]{}
\fi
\newcommand\setfnsymbol[1]{%
  \@bsphack
  \@ifundefined{FN@fnsymbol@#1}%
  {%
    \PackageError{footmisc}{Symbol style "#1" not known}%
    \@eha
  }{%
    \expandafter\let\expandafter\@fnsymbol\csname
                        FN@fnsymbol@#1\endcsname
  }%
  \@esphack
}
\let\FN@fnsymbol@lamport\@fnsymbol
\newif\if@tempswb
\DeclareDocumentCommand\DefineFNsymbols {smO{text}m}{%
  \expandafter\ifx\csname FN@fnsymbol@#2\endcsname\relax
    \PackageInfo{footmisc}{Declaring symbol style #2}%
  \else
    \PackageWarning{footmisc}{Redeclaring symbol style #2}%
  \fi
  \toks@{}%
  \def\@tempb{\end}%
  \FN@build@symboldef#4\end
  \def\@tempc{math}%
  \def\@tempd{#3}%
  \expandafter\xdef\csname FN@fnsymbol@#2\endcsname##1{%
    \ifx\@tempc\@tempd
      \noexpand\ensuremath
    \else
      \noexpand\nfss@text
    \fi
    {%
      \noexpand\ifcase##1%
      \the\toks@
      \noexpand\else
      \IfBooleanTF#1{\noexpand\@ctrerr}%
        {\noexpand\FN@orange##1}%
      \noexpand\fi
    }%
  }%
}
\def\FN@build@symboldef#1{%
  \def\@tempa{#1}%
  \ifx\@tempa\@tempb
  \else
    \toks@\expandafter{\the\toks@\or#1}%
    \expandafter\FN@build@symboldef
  \fi
}
\DeclareDocumentCommand\DefineFNsymbolsTM {smm}{%
  \expandafter\ifx\csname FN@fnsymbol@#2\endcsname\relax
    \PackageInfo{footmisc}{Declaring symbol style #2}%
  \else
    \PackageWarning{footmisc}{Redeclaring symbol style #2}%
  \fi
  \toks@{}%
  \def\@tempb{\end}%
  \FN@build@symboldefTM#3\end\@null
  \expandafter\xdef\csname FN@fnsymbol@#2\endcsname##1{%
    \noexpand\ifcase##1%
      \the\toks@
    \noexpand\else
      \IfBooleanTF#1{\noexpand\@ctrerr}%
        {\noexpand\FN@orange##1}%
      \noexpand\fi
  }%
}
\def\FN@build@symboldefTM#1#2{%
  \def\@tempa{#1}%
  \ifx\@tempa\@tempb
  \else
    \toks@\expandafter{\the\toks@\or\TextOrMath{#1}{#2}}%
    \expandafter\FN@build@symboldefTM
  \fi
}
\def\FN@orange#1{%
  \ifFN@robust
    \@arabic#1%
    \@bsphack
    \PackageInfo{footmisc}{Footnote number \number#1 out of range}%
    \protect\@fnsymbol@orange
    \@esphack
  \else \@ctrerr \fi
}
\global\let\@diagnose@fnsymbol@orange\relax
\AtEndDocument{\@diagnose@fnsymbol@orange}
\def\@fnsymbol@orange{%
  \gdef\@diagnose@fnsymbol@orange{%
    \PackageWarningNoLine{footmisc}{Some footnote number(s)
      were out of range
      \MessageBreak
      see log for details%
    }%
  }%
}
\DefineFNsymbolsTM{bringhurst}{%
  \textasteriskcentered *%
  \textdagger    \dagger
  \textdaggerdbl \ddagger
  \textsection   \mathsection
  \textbardbl    \|%
  \textparagraph \mathparagraph
}%
\DefineFNsymbolsTM{chicago}{%
  \textasteriskcentered *%
  \textdagger    \dagger
  \textdaggerdbl \ddagger
  \textsection   \mathsection
  \textbardbl    \|%
  \#\#%
}%
\DefineFNsymbolsTM{wiley}{%
  \textasteriskcentered *%
  {\textasteriskcentered\textasteriskcentered}{**}%
  \textdagger    \dagger
  \textdaggerdbl \ddagger
  \textsection   \mathsection
  \textparagraph \mathparagraph
  \textbardbl    \|%
}%
\DefineFNsymbolsTM{lamport-robust}{%
  \textasteriskcentered *%
  \textdagger    \dagger
  \textdaggerdbl \ddagger
  \textsection   \mathsection
  \textparagraph \mathparagraph
  \textbardbl    \|%
  {\textasteriskcentered\textasteriskcentered}{**}%
  {\textdagger\textdagger}{\dagger\dagger}%
  {\textdaggerdbl\textdaggerdbl}{\ddagger\ddagger}%
}
\DefineFNsymbolsTM*{lamport*}{%
  \textasteriskcentered *%
  \textdagger    \dagger
  \textdaggerdbl \ddagger
  \textsection   \mathsection
  \textparagraph \mathparagraph
  \textbardbl    \|%
  {\textasteriskcentered\textasteriskcentered}{**}%
  {\textdagger\textdagger}{\dagger\dagger}%
  {\textdaggerdbl\textdaggerdbl}{\ddagger\ddagger}%
  {\textsection\textsection}{\mathsection\mathsection}%
  {\textparagraph\textparagraph}{\mathparagraph\mathparagraph}%
  {\textasteriskcentered\textasteriskcentered\textasteriskcentered}{***}%
  {\textdagger\textdagger\textdagger}{\dagger\dagger\dagger}%
  {\textdaggerdbl\textdaggerdbl\textdaggerdbl}{\ddagger\ddagger\ddagger}%
  {\textsection\textsection\textsection}%%
    {\mathsection\mathsection\mathsection}%
  {\textparagraph\textparagraph\textparagraph}%%
    {\mathparagraph\mathparagraph\mathparagraph}%
}
\setfnsymbol{lamport*}
\DefineFNsymbolsTM{lamport*-robust}{%
  \textasteriskcentered *%
  \textdagger    \dagger
  \textdaggerdbl \ddagger
  \textsection   \mathsection
  \textparagraph \mathparagraph
  \textbardbl    \|%
  {\textasteriskcentered\textasteriskcentered}{**}%
  {\textdagger\textdagger}{\dagger\dagger}%
  {\textdaggerdbl\textdaggerdbl}{\ddagger\ddagger}%
  {\textsection\textsection}{\mathsection\mathsection}%
  {\textparagraph\textparagraph}{\mathparagraph\mathparagraph}%
  {\textasteriskcentered\textasteriskcentered\textasteriskcentered}{***}%
  {\textdagger\textdagger\textdagger}{\dagger\dagger\dagger}%
  {\textdaggerdbl\textdaggerdbl\textdaggerdbl}{\ddagger\ddagger\ddagger}%
  {\textsection\textsection\textsection}%%
    {\mathsection\mathsection\mathsection}%
  {\textparagraph\textparagraph\textparagraph}%%
    {\mathparagraph\mathparagraph\mathparagraph}%
}
\newcommand\mpfootnotemark{%
  \@ifnextchar[%
    \@xmpfootnotemark
    {%
      \stepcounter\@mpfn
      \protected@xdef\@thefnmark{\thempfn}%
      \@footnotemark
    }%
}
\def\@xmpfootnotemark[#1]{%
  \begingroup
    \csname c@\@mpfn\endcsname #1\relax
    \unrestored@protected@xdef\@thefnmark{\thempfn}%
  \endgroup
  \@footnotemark
}
\@ifpackageloaded{hyperref}{%
  \let\H@@footnotetext\FN@footnotetext
  \let\H@@footnotemark\FN@footnotemark
}{%
  \let \@footnotetext \FN@footnotetext
  \let\@footnotemark  \FN@footnotemark
}
\endinput
%</footmisc>
%    \end{macrocode}
% \Finale
%
}{}
%    \end{macrocode}
%
%    Footnote box layout for para footnotes;
%    this would also be the hook to support dblfootnotes (from the
%    \texttt{dblfnote} package if we integrate that).
%    \begin{macrocode}
\ifFN@para
  \def\@makecol@preparefootinshook {%
     \global\setbox\footins\vbox{\FN@makefootnoteparagraph}%
    }
\fi
%    \end{macrocode}
%
%    \begin{macrocode}
\ifFN@fixskip
  \def\@outputbox@removebskip{%
    \ifx\@textbottom\relax \else
      \@outputbox@append{%
        \@tempskipa\lastskip
        \ifnum \gluestretchorder\@tempskipa>\z@
          \vskip-\@tempskipa
          \xdef\@outputbox@reinsertbskip
              {\noexpand\@outputbox@append{\vskip\the\@tempskipa}}%
        \else
          \global\let\@outputbox@reinsertbskip\relax
        \fi
      }%
   \fi
  }
\let\@outputbox@reinsertbskip\relax
\else
  \let\@outputbox@removebskip \relax
  \let\@outputbox@reinsertbskip\relax
\fi
%    \end{macrocode}
%
%
%
%    \begin{macrocode}
\ifcase \FN@bottomcases\relax
\ERROR
\or
  \ifFN@abovefloats
    \def\@makecol@appendblocks {%
       \@if@footnotes@TF
          {\@outputbox@append{\vfill}}%
          {\@if@bfloats@TF{\@outputbox@append{\vfill}}%
                          {\@outputbox@reinsertbskip}}%
       \@outputbox@appendfootnotes
       \@outputbox@attachfloats
      }
  \else
    \def\@makecol@appendblocks {%
       \@outputbox@attachfloats
       \@if@footnotes@TF
          {\@outputbox@append{\vfill}}%
          {\@outputbox@reinsertbskip}%
       \@outputbox@appendfootnotes
    }
  \fi
\or
  \ifFN@abovefloats
     \def\@makecol@appendblocks {%
        \@outputbox@appendfootnotes
        \@if@bfloats@TF
            {\@outputbox@append{\vfill}}%
            {\@outputbox@reinsertbskip}%
        \@outputbox@attachfloats
     }
  \else
     \def\@makecol@appendblocks {%
       \@if@footnotes@TF
          {\@outputbox@append{\vfill}}%
          {\@if@bfloats@TF{\@outputbox@append{\vfill}}%
                          {\@outputbox@reinsertbskip}}%
        \@outputbox@attachfloats
        \@outputbox@appendfootnotes
     }
  \fi
\or
  \ifFN@abovefloats
    \def\@makecol@appendblocks {%
       \@outputbox@appendfootnotes
       \@outputbox@attachfloats
       \@outputbox@reinsertbskip
    }
  \else
    \def\@makecol@appendblocks {%
       \@outputbox@attachfloats
       \@outputbox@appendfootnotes
       \@outputbox@reinsertbskip
}
  \fi
\else
\ERROR
\fi

% next can be dropped when cleaned up
\newif\ifFN@setspace
\@ifpackageloaded{setspace}%
 {%
   \FN@setspacetrue
   \@ifclassloaded{memoir}%
     {%
       \AddToHook{fntext}{\let\baselinestretch\m@m@singlespace}%
       \let\FN@baselinestretch\m@m@singlespace
     }%
     {%
%       \AddToHook{fntext}{\let\baselinestretch\setspace@singlespace}%
       \let\FN@baselinestretch\setspace@singlespace
     }%
 }%
 {%
   \FN@setspacefalse
 }



\ifFN@para
  \def\@footnotetext@process {\insert\footins}
  \long\def\@footnotetext@processii #1{%
    \setbox\FN@tempboxa\hbox{\@makefntext{#1}}%
    \dp\FN@tempboxa\z@
    \ht\FN@tempboxa
      \dimexpr\wd\FN@tempboxa *%
              \footnotebaselineskip /\columnwidth\relax
    \box\FN@tempboxa
  }


  \def\@footnotetext@processiii {}
  \def\@footnotetext@processiv {% config point
           \strut
           \penalty-10\relax
           \hskip\footglue
  }
\fi



\ifFN@para
  \let\FN@tempboxa\@tempboxa
  \newbox\FN@tempboxb
  \newbox\FN@tempboxc
  \newskip\footglue \footglue=1em plus.3em minus.3em
  \long\def\@makefntext#1{\leavevmode
    \@makefnmark\nobreak
    \hskip.5em\relax#1%
  }
%%%%%%%%%%%%%%%%%%%%%%%%%%%%%%%%%%%%%%%%%%%%%%%%%%%%%%%%%%%%%%%%%%%%%%%%%%%%%
  \newdimen\footnotebaselineskip
  {%
    \footnotesize
    \global
      \footnotebaselineskip=\normalbaselineskip
  }

  \long\def\FN@makefootnoteparagraph{\unvbox\footins \FN@makehboxofhboxes
    \setbox\FN@tempboxa=\hbox{\unhbox\FN@tempboxa \FN@removehboxes}%
    \FN@setfootnoteparawidth
    \@parboxrestore
    \baselineskip=\footnotebaselineskip
    \noindent
    \rule{\z@}{\footnotesep}%
    \unhbox\FN@tempboxa\par
  }
  \def\FN@makehboxofhboxes{\setbox\FN@tempboxa=\hbox{}%
    \loop
      \setbox\FN@tempboxb=\lastbox
      \ifhbox\FN@tempboxb
      \setbox\FN@tempboxa=\hbox{\box\FN@tempboxb\unhbox\FN@tempboxa}%
    \repeat
  }
  \def\FN@removehboxes{\setbox\FN@tempboxa=\lastbox
    \ifhbox
      \FN@tempboxa{\FN@removehboxes}%
      \unhbox\FN@tempboxa
    \fi
  }
\fi
\@ifpackageloaded{multicol}
  {\def\FN@setfootnoteparawidth
    {\hsize\ifnum\doublecol@number>\@ne
                  \textwidth
            \else \columnwidth \fi}}
  {\def\FN@setfootnoteparawidth{\hsize\columnwidth}}
\ifFN@perpage
  \RequirePackage{perpage}
  \MakePerPage{footnote}
\fi
\ifFN@para
\else
  \long\def\@makefntext#1{%
    \ifFN@hangfoot
      \bgroup
      \setbox\@tempboxa\hbox{%
        \ifdim\footnotemargin>0pt
          \hb@xt@\footnotemargin{\@makefnmark\hss}%
        \else
          \@makefnmark
        \fi
      }%
      \leftmargin\wd\@tempboxa
      \rightmargin\z@
      \linewidth \columnwidth
      \advance \linewidth -\leftmargin
      \parshape \@ne \leftmargin \linewidth
      \footnotesize
      \@setpar{{\@@par}}%
      \leavevmode
      \llap{\box\@tempboxa}%
      \parskip\hangfootparskip\relax
      \parindent\hangfootparindent\relax
    \else
      \parindent1em
      \noindent
      \ifdim\footnotemargin>\z@
        \hb@xt@ \footnotemargin{\hss\@makefnmark}%
      \else
        \ifdim\footnotemargin=\z@
          \llap{\@makefnmark}%
        \else
          \llap{\hb@xt@ -\footnotemargin{\@makefnmark\hss}}%
        \fi
      \fi
    \fi
    \footnotelayout#1%
    \ifFN@hangfoot
      \par\egroup
    \fi
  }
\fi
\ifFN@multiplefootnote
  \providecommand*{\multiplefootnotemarker}{3sp}
  \providecommand*{\multfootsep}{,}
  \AddToHook{fnmark/begin}{\FN@mf@check}
  \AddToHook{fnmark/end}  {\FN@mf@prepare}
  \AddToHook{fnote/after} {\FN@mf@prepare}
%
  \def\FN@mf@prepare{%
    \kern-\multiplefootnotemarker
    \kern\multiplefootnotemarker\relax
  }
  \def\FN@mf@check{%
    \ifdim\lastkern=\multiplefootnotemarker\relax
%?? is that necessary or even correct ??    
      \edef\@x@sf{\the\spacefactor}%
%?? shouldn't that be 2 unkerns ?? (none would also be ok)
      \unkern  % new
      \unkern
      \textsuperscript{\multfootsep}%
      \spacefactor\@x@sf\relax
    \fi
  }
\else
  \let\FN@mf@prepare\relax
\fi
\ifFN@stablefootnote
\let\FN@sf@@footnote\footnote
\def\footnote{\ifx\protect\@typeset@protect
    \expandafter\FN@sf@@footnote
  \else
    \expandafter\FN@sf@gobble@opt
  \fi
}
\edef\FN@sf@gobble@opt{\noexpand\protect
  \expandafter\noexpand\csname FN@sf@gobble@opt \endcsname}
\expandafter\def\csname FN@sf@gobble@opt \endcsname{%
  \@ifnextchar[%]
    \FN@sf@gobble@twobracket
    \@gobble
}
\def\FN@sf@gobble@twobracket[#1]#2{}
\let\FN@sf@@footnotemark\footnotemark
\def\footnotemark{\ifx\protect\@typeset@protect
    \expandafter\FN@sf@@footnotemark
  \else
    \expandafter\FN@sf@gobble@optonly
  \fi
}
\edef\FN@sf@gobble@optonly{\noexpand\protect
  \expandafter\noexpand\csname FN@sf@gobble@optonly \endcsname}
\expandafter\def\csname FN@sf@gobble@optonly \endcsname{%
  \@ifnextchar[%]
    \FN@sf@gobble@bracket
    {}%
}
\def\FN@sf@gobble@bracket[#1]{}
\fi
\newcommand\setfnsymbol[1]{%
  \@bsphack
  \@ifundefined{FN@fnsymbol@#1}%
  {%
    \PackageError{footmisc}{Symbol style "#1" not known}%
    \@eha
  }{%
    \expandafter\let\expandafter\@fnsymbol\csname
                        FN@fnsymbol@#1\endcsname
  }%
  \@esphack
}
\let\FN@fnsymbol@lamport\@fnsymbol
\newif\if@tempswb
\DeclareDocumentCommand\DefineFNsymbols {smO{text}m}{%
  \expandafter\ifx\csname FN@fnsymbol@#2\endcsname\relax
    \PackageInfo{footmisc}{Declaring symbol style #2}%
  \else
    \PackageWarning{footmisc}{Redeclaring symbol style #2}%
  \fi
  \toks@{}%
  \def\@tempb{\end}%
  \FN@build@symboldef#4\end
  \def\@tempc{math}%
  \def\@tempd{#3}%
  \expandafter\xdef\csname FN@fnsymbol@#2\endcsname##1{%
    \ifx\@tempc\@tempd
      \noexpand\ensuremath
    \else
      \noexpand\nfss@text
    \fi
    {%
      \noexpand\ifcase##1%
      \the\toks@
      \noexpand\else
      \IfBooleanTF#1{\noexpand\@ctrerr}%
        {\noexpand\FN@orange##1}%
      \noexpand\fi
    }%
  }%
}
\def\FN@build@symboldef#1{%
  \def\@tempa{#1}%
  \ifx\@tempa\@tempb
  \else
    \toks@\expandafter{\the\toks@\or#1}%
    \expandafter\FN@build@symboldef
  \fi
}
\DeclareDocumentCommand\DefineFNsymbolsTM {smm}{%
  \expandafter\ifx\csname FN@fnsymbol@#2\endcsname\relax
    \PackageInfo{footmisc}{Declaring symbol style #2}%
  \else
    \PackageWarning{footmisc}{Redeclaring symbol style #2}%
  \fi
  \toks@{}%
  \def\@tempb{\end}%
  \FN@build@symboldefTM#3\end\@null
  \expandafter\xdef\csname FN@fnsymbol@#2\endcsname##1{%
    \noexpand\ifcase##1%
      \the\toks@
    \noexpand\else
      \IfBooleanTF#1{\noexpand\@ctrerr}%
        {\noexpand\FN@orange##1}%
      \noexpand\fi
  }%
}
\def\FN@build@symboldefTM#1#2{%
  \def\@tempa{#1}%
  \ifx\@tempa\@tempb
  \else
    \toks@\expandafter{\the\toks@\or\TextOrMath{#1}{#2}}%
    \expandafter\FN@build@symboldefTM
  \fi
}
\def\FN@orange#1{%
  \ifFN@robust
    \@arabic#1%
    \@bsphack
    \PackageInfo{footmisc}{Footnote number \number#1 out of range}%
    \protect\@fnsymbol@orange
    \@esphack
  \else \@ctrerr \fi
}
\global\let\@diagnose@fnsymbol@orange\relax
\AtEndDocument{\@diagnose@fnsymbol@orange}
\def\@fnsymbol@orange{%
  \gdef\@diagnose@fnsymbol@orange{%
    \PackageWarningNoLine{footmisc}{Some footnote number(s)
      were out of range
      \MessageBreak
      see log for details%
    }%
  }%
}
\DefineFNsymbolsTM{bringhurst}{%
  \textasteriskcentered *%
  \textdagger    \dagger
  \textdaggerdbl \ddagger
  \textsection   \mathsection
  \textbardbl    \|%
  \textparagraph \mathparagraph
}%
\DefineFNsymbolsTM{chicago}{%
  \textasteriskcentered *%
  \textdagger    \dagger
  \textdaggerdbl \ddagger
  \textsection   \mathsection
  \textbardbl    \|%
  \#\#%
}%
\DefineFNsymbolsTM{wiley}{%
  \textasteriskcentered *%
  {\textasteriskcentered\textasteriskcentered}{**}%
  \textdagger    \dagger
  \textdaggerdbl \ddagger
  \textsection   \mathsection
  \textparagraph \mathparagraph
  \textbardbl    \|%
}%
\DefineFNsymbolsTM{lamport-robust}{%
  \textasteriskcentered *%
  \textdagger    \dagger
  \textdaggerdbl \ddagger
  \textsection   \mathsection
  \textparagraph \mathparagraph
  \textbardbl    \|%
  {\textasteriskcentered\textasteriskcentered}{**}%
  {\textdagger\textdagger}{\dagger\dagger}%
  {\textdaggerdbl\textdaggerdbl}{\ddagger\ddagger}%
}
\DefineFNsymbolsTM*{lamport*}{%
  \textasteriskcentered *%
  \textdagger    \dagger
  \textdaggerdbl \ddagger
  \textsection   \mathsection
  \textparagraph \mathparagraph
  \textbardbl    \|%
  {\textasteriskcentered\textasteriskcentered}{**}%
  {\textdagger\textdagger}{\dagger\dagger}%
  {\textdaggerdbl\textdaggerdbl}{\ddagger\ddagger}%
  {\textsection\textsection}{\mathsection\mathsection}%
  {\textparagraph\textparagraph}{\mathparagraph\mathparagraph}%
  {\textasteriskcentered\textasteriskcentered\textasteriskcentered}{***}%
  {\textdagger\textdagger\textdagger}{\dagger\dagger\dagger}%
  {\textdaggerdbl\textdaggerdbl\textdaggerdbl}{\ddagger\ddagger\ddagger}%
  {\textsection\textsection\textsection}%%
    {\mathsection\mathsection\mathsection}%
  {\textparagraph\textparagraph\textparagraph}%%
    {\mathparagraph\mathparagraph\mathparagraph}%
}
\setfnsymbol{lamport*}
\DefineFNsymbolsTM{lamport*-robust}{%
  \textasteriskcentered *%
  \textdagger    \dagger
  \textdaggerdbl \ddagger
  \textsection   \mathsection
  \textparagraph \mathparagraph
  \textbardbl    \|%
  {\textasteriskcentered\textasteriskcentered}{**}%
  {\textdagger\textdagger}{\dagger\dagger}%
  {\textdaggerdbl\textdaggerdbl}{\ddagger\ddagger}%
  {\textsection\textsection}{\mathsection\mathsection}%
  {\textparagraph\textparagraph}{\mathparagraph\mathparagraph}%
  {\textasteriskcentered\textasteriskcentered\textasteriskcentered}{***}%
  {\textdagger\textdagger\textdagger}{\dagger\dagger\dagger}%
  {\textdaggerdbl\textdaggerdbl\textdaggerdbl}{\ddagger\ddagger\ddagger}%
  {\textsection\textsection\textsection}%%
    {\mathsection\mathsection\mathsection}%
  {\textparagraph\textparagraph\textparagraph}%%
    {\mathparagraph\mathparagraph\mathparagraph}%
}
\newcommand\mpfootnotemark{%
  \@ifnextchar[%
    \@xmpfootnotemark
    {%
      \stepcounter\@mpfn
      \protected@xdef\@thefnmark{\thempfn}%
      \@footnotemark
    }%
}
\def\@xmpfootnotemark[#1]{%
  \begingroup
    \csname c@\@mpfn\endcsname #1\relax
    \unrestored@protected@xdef\@thefnmark{\thempfn}%
  \endgroup
  \@footnotemark
}
%    \end{macrocode}
%    TEMP PATCHES FOR TESTING
%    \begin{macrocode}

\endinput



\ExplSyntaxOn
\AddToHook{cmd/footnote/before}{\tag_mc_end_push:\bgroup\tagpdfparaOff\tagstructbegin{tag=Note}}
\AddToHook{cmd/footnote/after}{\tagstructend\egroup\tag_mc_begin_pop:n{}}
\ExplSyntaxOff



\endinput
%</footmisc>
%    \end{macrocode}
% \Finale
%

%    \end{macrocode}
%
%
%    \begin{macrocode}
%</package>
%    \end{macrocode}
%
% \Finale
%
