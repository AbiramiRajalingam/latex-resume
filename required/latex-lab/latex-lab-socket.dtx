% \iffalse meta-comment
%
%% File: latex-lab-socket.dtx
% Copyright (C) 2023 The LaTeX Project
%
% It may be distributed and/or modified under the conditions of the
% LaTeX Project Public License (LPPL), either version 1.3c of this
% license or (at your option) any later version.  The latest version
% of this license is in the file
%
%    https://www.latex-project.org/lppl.txt
%
%
% The development version of the bundle can be found below
%
%    https://github.com/latex3/latex2e/required/latex-lab
%
% for those people who are interested or want to report an issue.
%
%<*driver>
\documentclass{l3doc}
\EnableCrossrefs
\CodelineIndex
\begin{document}
  \DocInput{latex-lab-socket.dtx}
\end{document}
%</driver>
%
% \fi
%
%
% \title{The \texttt{latex-lab-testphase-socket} code}
% \author{Frank Mittelbach, \LaTeX{} Project}
%
% \maketitle
%
% \newcommand\fmi[1]{\begin{quote} TODO: \itshape #1\end{quote}}
% \newcommand\NEW[1]{\marginpar{\mbox{}\hfill\fbox{New: #1}}}
% \providecommand\pkg[1]{\texttt{#1}}
% \providecommand\hook[1]{\texttt{#1}}
% \providecommand\env[1]{\texttt{#1}}
% \providecommand\plug[1]{\texttt{#1}}
% \providecommand\socket[1]{\texttt{#1}}
%
%
% \begin{abstract}
%    This code implements sockets which are places in the code into
%    which predeclared chunks of code (plugs) can be placed. Both the sockets
%    and the plugs are \enquote{named} and each socket is
%    assigned exactly one plug at any given time.
% \end{abstract}
%
% \tableofcontents
%
% \section{Introduction}
%
% A \LaTeX{} source file is transformed into a typeset document by
% executing code for each command or environment in the document
% source. Through various steps this code transforms the input and
% eventually generates typeset output appearing in a \enquote{galley}
% from which individual pages are cut off in an asyncronous way. This
% page generating process is normally not directly associated with
% commands in the input\footnote{Excepts for directives such as
%   \cs{newpage}.} but is triggered whenever the galley has received
% enough material to form another page (giving current settings).
% 
% As part of this transformation input data may get stored in some form
% and later reused, for example, as part of the output routine
% processing.
% 
% \section{Configuration of the transformation process}
% 
% There are three different major methods offered by \LaTeX{} to
% configure the transformation process:
% \begin{itemize}
% \item through the template mechanism,
% \item through the hook mechanism, or
% \item through sockets.
% \end{itemize}
% They offer different possibilities (with different features and
% limitations) and are intended for specific use cases, though it is
% possible to combine them.
% 
% \subsection{The template mechanism}
% 
% The template mechanism is intended for more complex document-level
% elements (e.g., headings such as \verb=\section= or environments like
% \env{itemize}). The template code implements the overall processing
% logic for such an element and offers a set of parameters to influence
% the final result.
% 
% The document element is then implemented by a) selecting a suitable
% template (there may be more than one available for the kind of
% document element) and b) by setting its parameters to desired
% values. This then forms a so-called instance which is executed when
% the document element is found in the source.
% 
% By altering the parameter values (in a document class or in the
% document preamble) or, if more drastic layout changes are desired, by
% selecting a different template and then adjusting its parameters, a
% wide variety of layouts can be realized through simple configuration
% setups without the need to develop new code.
% 
% The target audience of this method are therefore document class
% developers or users who wish to alter an existing layout (implemented
% by a document class) in certain (minor) ways.
% 
% The template mechanism is currently documented as part of the
% \pkg{xtemplate} package and one more elaborate implementation can be
% found as part of the \texttt{latex-lab} code for lists (to be
% documented further).
% 
% \subsection{The hook mechanism}
% 
% Hooks are places in the kernel code (or in packages) that offer
% packages the possibility to inject additional code at specific
% points in the processing in a controlled way without the need to
% replace the existing code block (and thereby overwriting
% modifications/extensions made by other packages). The target
% audience is therefore mainly package developers, even though some
% hooks can be useful for document authors.
% 
% Obviously, what can reasonably be added into a hook depends on the
% individual hook (hopefully documented as part of the hook
% documentation), but in general the idea behind hooks is that more
% than one package could add code into the hook at the same
% time. Perhaps the most famous hook (that \LaTeX{} had for a very
% long time) is \hook{begindocument} into which many packages add code
% to through \cs{AtBeginDocument}\marg{code} (which is nowadays
% implemented as a shorthand for
% \cs{AddToHook}\texttt{\{\hook{begindocument}\}}\marg{code}). To
% resolve possible conflicts between injections by different packages
% there is a rule mechanism by which code chunks in a hook can be
% ordered in a certain way and by which incompatible packages can be
% detected if a resolution is impossible.
% 
% In contrast to template code, there is no standard configuration
% method through parameters for hooks, i.e., the code added to a hook
% \enquote{is} the configuration. If it wants to provide for
% configuration through parameters it has to also provide its own
% method to set such parameters in some way. However, in that case it
% is likely that using a hook is not the right approach and the
% developer better calls a template instance instead which then offers
% configuration through a key/value interface.
% 
% In most cases, hooks do not take any arguments as input. Instead, the data
% that they can (and are allowed to) access depends on the surrounding
% context.
% 
% For example, the various hooks available during the page shipout
% process in \LaTeX's output routine can (and have to) access the
% accumulated page material stored in a box named
% \verb=\ShipoutBox=. This way, code added to, say, the
% \hook{shipout/before} hook could access the page content, alter it,
% and then write it back into \verb=\ShipoutBox= and any other code
% added to this hook could then operate on the modified content.  Of
% course, for such a scheme to work the code prior to executing the hook
% would need to setup up data in appropriate places and the hook
% documentation would need to document what kind of storage can be
% accessed (and possibly altered) by the hook.
% 
% There are also hooks that take arguments (typically portions of
% document data) and in that case the hook code can access these
% arguments through \verb=#1=, \verb=#2=, etc.
% 
% The hook mechanism is documented in \texttt{lthooks-doc.pdf}.
%
%
%
% \subsection{The socket mechanism}
% 
% In some cases there is code that implements a certain programming
% logic (for example, combining footnotes, floats, and the text for the
% current page to be shipped out) and if this logic should change (e.g.,
% footnotes to be placed above bottom floats instead of below) then this
% whole code block needs to be replaced with different code.
% 
% In theory, this could be implemented with templates, i.e., the code
% simply calls some instance that implements the logic and that instance
% is altered by selecting a different templates and/or adjusting their
% parameters. However, in many cases customization through parameters is
% overkill in such a case (or otherwise awkward, because paramerization
% is better done on a higher level instead of individually for small
% blocks of code) and using the template mechanism just to replace one
% block of code with a different one results in a fairly high
% performance hit. It is therefore usually not a good choice.
% 
% In theory, it would also be possible to use a hook, but again that is
% basically a misuse of the concept, because in this use case there should
% never be more that one block of code inside the hook; thus, to alter
% the processing logic one would need to set up rules that replace code
% rather than (as intended) execute all code added to the hook.
% 
% 
% 
% For this reason \LaTeX{} now offers a third mechanism:
% \enquote{sockets} into which one can place exactly one code block
%   --- a \enquote{plug}.
%
% In a nutshell: instead of having a fixed code block somewhere as part
% of the code, implementing a certain programming logic there is a
% reference to a named socket at this point.
% 
% This is done by first declaring the named socket with:
% \begin{quote}
% \cs{NewSocket}\marg{socket-name}\marg{number-of-arguments}
% \end{quote}
% This is then referenced at the point where the replaceable code block
% should be executed with:
% \begin{quote}
%  \cs{UseSocket}\marg{socket-name}
% \end{quote}
% or, if the socket should take a number of arguments with
% \begin{quote}
%   \cs{UseSocket}\marg{socket-name}\marg{arg\textsubscript{1}}\ldots
%   \marg{arg\textsubscript{number-of-arguments}}
% \end{quote}
% 
% In addition, several code blocks (a.k.a.\ plugs) implementing different logic for this
% socket are set up, each with a declaration of the form:
% \begin{quote}
%   \cs{NewSocketPlug}\marg{socket-name}\marg{socket-plug-name}\marg{code})
% \end{quote}
% Finally,
% one of them is assigned to the socket:
% \begin{quote}
% \cs{AssignSocketPlug}\marg{socket-name}\marg{socket-plug-name}
% \end{quote}
% If the programming logic should change, then all that is necessary is
% to make a new assignment with \cs{AssignSocketPlug} to a different
% \marg{socket-plug-name}.
% 
% If the socket takes arguments, then those need to be provided to
% \cs{UseSocket} and in that case they can be referenced in the \meta{code}
% argument of \cs{NewSocketPlug} with \verb=#1=, \verb=#2=, etc.
% 
% In most cases a named socket is used only in a single place, but there
% is, of course, nothing wrong wth using it in several places, as long
% as the code in all places is supposed to change in the same way.
% 
% 
%
% \subsubsection{Syntax}
%
% We give both the \LaTeXe{} and the L3 programming layer command names
% 
% \begin{function}{\NewSocket,\socket_new:nn}
% \begin{syntax}
% \cs{NewSocket}    \Arg{socket-name}\Arg{number-of-arguments}
% \cs{socket_new:nn}\Arg{socket-name}\Arg{number-of-arguments}
% \end{syntax}
%   Declares a new socket with name \meta{socket-name} having
%   \meta{number-of-arguments} arguments. By default it does nothing
%   other than gobbling the arguments,
%   i.e., executes the plug \plug{noop}, which is automatically
%   defined for the new socket as part of the declaration.
%   \fmi{For some sockets a noop hook is useless so perhaps we
%   should not make that default declaration. On the other hand for
%   many sockets it is useful -- decide }
%
%   Its documentation should describe its purpose, its inputs and the
%   expected results as discussed above.
%
%   The declaration is only allowed at top-level, i.e., not inside a group.
% \end{function}
%
% 
% \begin{function}{\NewSocketPlug,\socket_new_plug:nnn,\socket_set_plug:nnn}
% \begin{syntax}
% \cs{NewSocketPlug}      \Arg{socket-name}\Arg{socket-plug-name}\Arg{code}
% \cs{socket_new_plug:nnn}\Arg{socket-name}\Arg{socket-plug-name}\Arg{code}
% \cs{socket_set_plug:nnn}\Arg{socket-name}\Arg{socket-plug-name}\Arg{code}
% \end{syntax}
%   Declares a new plug for socket \meta{socket-name} that runs
%   \meta{code} when executing. It complains if the plug was already
%   declared previously.
%
%   The form \cs{socket_set_plug:nnn} changes an existing plug. As
%   this should normally not be necessary we currently have only an L3
%   layer name for that.
%
%   The declarations can be made inside a group and obey scope.
% \end{function}
%
% 
% \begin{function}{\AssignSocketPlug,\socket_assign_plug:nn}
% \begin{syntax}
% \cs{AssignSocketPlug}     \Arg{socket-name}\Arg{socket-plug-name}
% \cs{socket_assign_plug:nn}\Arg{socket-name}\Arg{socket-plug-name}
% \end{syntax}
%   Assigns the plug \meta{socket-plug-name} to the socket
%   \meta{socket-name}. It errors if either socket or plug is not
%   defined.
%
%   The assignment is local, i.e., it obeys scope.
% \end{function}
%
% 
% \begin{function}{\UseSocket,\socket_use:n}
% \begin{syntax}
% \cs{UseSocket}   \Arg{socket-name}
% \cs{socket_use:n}\Arg{socket-name}
% \end{syntax}
%   Executes the socket \meta{socket-name} by retrieving the
%   \meta{code} of the current plug assigned to the socket. This is
%   the only command that would appear inside macro code in packages.
% \end{function}
%
% 
% \begin{function}{\ShowSocket,\socket_show:n,\LogSocket,\socket_log:n}
% \begin{syntax}
% \cs{ShowSocket}   \Arg{socket-name}
% \cs{socket_show:n}\Arg{socket-name}
% \end{syntax}
%   Displays information about the socket \meta{socket-name} and its
%   state then stops and waits for further instructions --- at the
%   moment some what rudimentary.
% \end{function}
%
% 
% \begin{function}{\DebugSocketsOn,\DebugSocketsOff}
% \begin{syntax}
% \cs{DebugSocketsOn} \ldots\ \cs{DebugSocketsOff}
% \end{syntax}
%   Turns debugging of sockets on or off.\fmi{implement}
% \end{function}
%
% 
% 
%
% \subsubsection{Details and semantics}
% 
% In this section we collect some normative statements.
% 
% \begin{itemize}
% 
% \item
%  From a functional point of view sockets are like simple \TeX{} macros,
%   i.e., they expect 0 to 9 mandatory arguments and get replaced by
%   their \enquote{expansion}
% 
% \item
%   A socket is \enquote{named} and the name consists of ASCII letters
%   \texttt{[a-z]},
%   \texttt{[A-Z]}, \texttt{[0-9]}, \texttt{[-/@]} only
% 
% \item
%  Best practice naming conventions are \ldots\ \emph{to be documented}
% 
% \item
%    A socket has documented inputs which are
% 
%     \begin{itemize}
%     \item
%       the positional arguments (if any) with a description of what
%       they contain when used
%     
%     \item
%       implicit data (registers and other 2e/expl3 data stores) that
%       the socket is allowed to make use of, with a documented description
%       what they contain (if relevant for the task at hand---no need to
%       describe the whole \LaTeX{} universe)
% 
%     \item
%        information about the state of the \TeX{} engine (again when
%       relevant), e.g. is called in mmode or vmode or in the output routine or \ldots
%       
%     \item
%       \ldots\ \empty{anything missing?}
%     \end{itemize}
% 
%       
% \item
%     A socket has documented results/outputs which can be
% 
%     \begin{itemize}
%     \item
%        what kind of data it should write to the current list (if that
%        is part of its task)
% 
%     \item
%        what kind of registers and other 2e/expl3 data stores it should
%        modify and in what way
% 
%     \item
%        what kind of state changes it should do (if any)
% 
%     \item
%        \emph{\ldots\ anything else?}
%     \end{itemize}
% 
% \item
%
%   At any time a socket has one block of code (a plug :-)\,)
%   associated with it. Such code is itself named and the association
%   is done by linking the socket name to the code name (putting a
%   plug into the socket).
% 
% \item
%
%   The name of a plug consists of ASCII letters \texttt{[a-z]},
%   \texttt{[A-Z]}, \texttt{[0-9]}, \texttt{[-/@]} only.
%
% \item
%
%   When declaring a plug it is stated for which socket it is meant
%   (i.e., its code can only be used with that socket). This means
%   that the same plug name can be used with different sockets
%   referring to different code in each case.
% 
% \item
%   Configuration of  a socket can only be done by
%   linking different code to it. Nevertheless the code linked to it can
%   provide its own means of configuration (but this is outside of the
%   spec).
% 
% \item
%   Technically execution of a socket (\cs{UseSocket}) involves
% 
%     \begin{itemize}
%     \item
%        doing any house keeping (like writing debugging info, \ldots);
% 
%     \item
%        looking up the current code association (what plug is in the socket);
%     
%     \item
%       executing this code which will pick up the mandatory arguments
%       (happens at this point, not
%        before), i.e., it is like calling a csname defined with
% \begin{verbatim}      
%   \def\foo#1#2...{...#1...#2...}
% \end{verbatim}
% 
%     \item
%       do some further house keeping (if needed).
%     \end{itemize}
% 
% \item
%   A socket is typically only used in one place in code, but this is not
%   a requirement, i.e., if the same operation with the same inputs need
%   to be carried out in several places the same named socket can be used.
% 
% \end{itemize}
% 
% 
%
%
%
%
% \StopEventually{\setlength\IndexMin{200pt}  \PrintIndex  }
%
%
% \section{The Implementation}
%
%    The implementation of the socket mechanism should be redone and we
%    should probably store the different code chunks in a property
%    list so that we can have a decent \cs{ShowSocket} command the shows
%    the available alternatives.
%
%    \begin{macrocode}
%<*code>
%<@@=socket>
%    \end{macrocode}
%
%
%    \begin{macrocode}
\ProvidesExplPackage {latex-lab-testphase-socket} {2023-07-20} {0.8a}
  {sockets (or replaceable code blocks)}
%    \end{macrocode}
%
% \subsection{The L3 layer commands}
%
%  \begin{macro}{\socket_new:nn}
%    
%    Declaring a socket creates a str to hold the name (a pointer) to the
%    code that should be used when the socket is executed, and an integer to
%    hold the number of arguments of that socket.  Initially, an
%    ``empty'' code chunk is created and assigned so the socket
%    does nothing by default other than swallowing its arguments (if any).
%    \begin{macrocode}
\cs_new_protected:Npn \socket_new:nn #1 #2 {
  \str_if_exist:cTF { l_@@_#1_plug_str }
      {
        \errmessage { Socket~ '#1'~ already~ declared! }
      }
      {
        \int_compare:nNnTF  \tex_currentgrouplevel:D  = 0
          {
            \str_new:c { l_@@_#1_plug_str }
            \seq_new:c { l_@@_#1_plugs_seq }
            \int_const:cn { c_@@_#1_args_int } {#2}
            \socket_new_plug:nnn {#1} { noop } { }
            \socket_assign_plug:nn {#1} { noop }
          }
          {
            \errmessage { Sockets~ can~ only~ be~ declared~ at~ top-level! }
          }
      }
}
%    \end{macrocode}
%  \end{macro}
%  
%  
%  \begin{macro}{\socket_show:n,\socket_log:n}
%    Show the current state of the socket --- for now this is just a
%    quick draft and should be redone and extended.
%    \begin{macrocode}
\cs_new_protected:Npn \socket_show:n #1 {
  \str_if_exist:cTF { l_@@_#1_plug_str }
      {
        \typeout{ Socket~ #1:}
        \typeout{ \@spaces number~ of~ arguments~ =~ \int_use:c { c_@@_#1_args_int } }
        \typeout{ \@spaces available~plugs~ =~
                     \seq_use:cnnn { l_@@_#1_plugs_seq }{,~}{,~}{,~} }
        \typeout{ \@spaces current~ plug~ =~ \str_use:c { l_@@_#1_plug_str } }
        \typeout{ \@spaces definition~ =~
          \exp_args:Nc \cs_meaning:N 
                       { _@@_#1_code_ \str_use:c { l_@@_#1_plug_str } :w } }
        \typeout{}
      }
      {
        \errmessage { Socket~ '#1'~ not~ declared! }
      }
}
\cs_new_eq:NN \socket_log:n \socket_show:n
%    \end{macrocode}
%  \end{macro}
%  
%  
%  
%  
%  
%  \begin{macro}{\socket_new_plug:nnn}
%    
%    Declaring a code for a socket is just making a definition, taking
% the number of arguments from the saved int.
%    \begin{macrocode}
\cs_new_protected:Npn \socket_new_plug:nnn {
  \@@_define_code:Nnnnn \cs_new_protected:Npn \seq_put_right:cn 
}
\cs_new_protected:Npn \socket_set_code:nnn {
  \@@_define_code:Nnnnn \cs_set_protected:Npn  \use_none:nn 
}

\cs_new_protected:Npn \@@_define_code:Nnnnn #1 #2 #3 #4 #5 {
  \str_if_exist:cTF { l_@@_#3_plug_str }
      {
        \cs_generate_from_arg_count:cNnn
           { @@_#3_code_#4:w }
           #1
           { \int_use:c { c_@@_#3_args_int } }
           {#5}
        #2 { l_@@_#3_plugs_seq } {#4}
      }
      {
        \errmessage { Socket~ '#3'~ not~ declared! }
      }
}
%    \end{macrocode}
%  \end{macro}
%  
%  
%  
%  \begin{macro}{\socket_assign_plug:nn}
%    
%    Assigning stored code to a socket just changes the name in
%    the socket string. The assignment is local to the current group.
%    \begin{macrocode}
\cs_new_protected:Npn \socket_assign_plug:nn #1 #2 {
  \str_if_exist:cTF { l_@@_#1_plug_str }
      {
        \cs_if_exist:cTF { @@_#1_code_#2:w }
          {
            \str_set:cn { l_@@_#1_plug_str } {#2}
          }
          {
            \errmessage { Socket~ instance~ '#2'~ for~ the~ socket~ '#1'~ not~ declared! }
          }
      }
      {
        \errmessage { Socket~ '#1'~ not~ declared! }
      }
}
%    \end{macrocode}
%  \end{macro}
%  
%  
%  \begin{macro}{\socket_use:n}
%    
%    And using it is more or less a \cs{use:c} so very lightweight. We do not add a
%    runtime check for speed reasons!
%    \begin{macrocode}
\cs_new_protected:Npn \socket_use:n #1 {
  \use:c { _@@_#1_code_ \str_use:c { l_@@_#1_plug_str } :w }
}
%    \end{macrocode}
%  \end{macro}
%  
% \subsection{The \LaTeXe{} interface commands}
%
%  \begin{macro}{\NewSocket,\NewSocketPlug,
%                \ShowSocket,\LogSocket,
%                \AssignSocketPlug,\UseSocket,
%                \DebugSocketsOn,\DebugSocketsOff}
%    As we expect that there are existing \LaTeXe{} packages that may
%   want to make use of the socket mechanism, we provide 2e names for
%   most of the commands.
%    \begin{macrocode}
\cs_new_eq:NN \NewSocket         \socket_new:nn 
\cs_new_eq:NN \ShowSocket        \socket_show:n 
\cs_new_eq:NN \LogSocket         \socket_log:n 
%    \end{macrocode}
%
%    \begin{macrocode}
\cs_new_eq:NN \NewSocketPlug     \socket_new_plug:nnn
\cs_new_eq:NN \AssignSocketPlug  \socket_assign_plug:nn
\cs_new_eq:NN \UseSocket         \socket_use:n
%    \end{macrocode}
%    Not yet implemented:
%    \begin{macrocode}
\cs_new_eq:NN \DebugSocketsOn    \prg_do_nothing:
\cs_new_eq:NN \DebugSocketsOff   \prg_do_nothing:
%    \end{macrocode}
%  \end{macro}
%  
%
%    \begin{macrocode}
%<@@=>
%</code>
%    \end{macrocode}
%
% \Finale
%
