% \iffalse meta-comment
%
%% File: bib-kernel-functions.dtx (C) Copyright 2022-2023 LaTeX Project
%
% It may be distributed and/or modified under the conditions of the
% LaTeX Project Public License (LPPL), either version 1.3c of this
% license or (at your option) any later version.  The latest version
% of this license is in the file
%
%    https://www.latex-project.org/lppl.txt
%
%
% The development version of the bundle can be found below
%
%    https://github.com/latex3/latex2e/required/latex-lab
%
% for those people who are interested or want to report an issue.
%
%<*driver>
\documentclass{l3doc}
\EnableCrossrefs
\CodelineIndex
\begin{document}
  \DocInput{latex-lab-bib-kernel-changes.dtx}
\end{document}
%</driver>
%
% \fi
%
% \title{The \textsf{bib-kernel-changes} package\\
% Changes and additions to the kernel related to tagging and links in citations and 
% bibliography entries}
% \author{\LaTeX{} Project\thanks{Initial implementation done by Ulrike Fischer}}
% \date{Version 0.80 2023-05-10}
%
% \maketitle
%
% \newcommand{\xt}[1]{\textsl{\textsf{#1}}}
% \newcommand{\TODO}[1]{\textbf{[TODO:} #1\textbf{]}}
% \newcommand{\docclass}{document class \marginpar{\raggedright document class
% customizations}}
%
% \providecommand\hook[1]{\texttt{#1}}
%
% \begin{abstract}
% \end{abstract}
%
% \section{Introduction}
%
% The followings contains small changes to improve tagging of
% bibliography entries and citations. 
% 
% The tagging of the bibliography is actually quite straightforward:
% A bibliography is typically a list with a heading and the sec and block tagging
% code handles that. 
% 
% There are here only two problems: 
% 
% \begin{itemize}
% \item The structure number of the \texttt{LI} element
% create by a \cs{bibitem} must be recorded somehow to allow to reference it in 
% a \cs{cite}. 
% \item \pkg{hyperref} redefines the item command and so breaks the list structure
% see \url{https://github.com/latex3/latex2e/discussions/1010#discussioncomment-5565418}
% \end{itemize}
% 
% Both problems are rather easy to resolve, but it must be checked if other packages 
% interfere again by redefining the commands.
% 
% More difficult is the tagging of citation commands. Citations should be inside 
% a Reference structure and contain a /Ref entry pointing to the relevant 
% item in the bibliography. For simple citations like 
% \enquote{[1]} or \enquote{Doody (2023)} this is easy, but it is not obvious how
% to handle combined citations like \enquote{Doody (2003,2018)} (or even compressed citations
% like \enquote{[1-3]}). The implementation follows here the links: whatever hyperref would
% link is set as the reference.
% 
% An additional problem are the various packages trying to improve citation commands
% which all should be check. Currently only natbib has been tried. 
% 
% The biblatex package isn't handled yet.  
% 
% 
% \section{Implementation}
%    \begin{macrocode}
%<*package>
%<@@=tag>
%    \end{macrocode}
%    \begin{macrocode}
\ProvidesExplPackage {latex-lab-testphase-bib} {2023-05-18} {0.8}
  {Code related to the tagging of bibliography and cite command}
%    \end{macrocode}
% We need at least the block tagging code.
%    \begin{macrocode}
\RequirePackage{latex-lab-testphase-block-tagging}  
%    \end{macrocode}
%
% At first we suppress the patches from hyperref. This will only work with the next
% hyperref!
%    \begin{macrocode}
\def\hyper@nopatch@bib{}
%    \end{macrocode}

% \begin{macro}{\@extra@binfo,\@extra@b@citeb}
% These are taken from hyperref, they are for chapterbib compability (and also
% signal to chapterbib not to change the citation commands)
%    \begin{macrocode}
\providecommand*\@extra@binfo{}%
\providecommand\@extra@b@citeb{}
%    \end{macrocode}
% \end{macro}
% 
% \begin{variable}{\l_@@_bib_target_tl}
% Items should add a target, to avoid that user code
% interferes we don't rely on \cs{@currentHref}
%    \begin{macrocode}
\tl_new:N\l_@@_bib_target_tl
%    \end{macrocode}
% \end{variable}

%\subsection{Handling the bibliography}
% \begin{macro}{\lbibitem}
% The item command if an optional argument is use. 
% 
% We only prepend some code. 
% If we had already generic hooks with arguments 
% we could probably use them \ldots
%    \begin{macrocode}
\def\@lbibitem[#1]#2{%
%    \end{macrocode}
% we store the target name for the following code.
%    \begin{macrocode}
  \tl_set:Nx\l_@@_bib_target_tl{cite.#2\@extra@b@citeb}
%    \end{macrocode}
% The target is added at the begin of the paragraph.
%    \begin{macrocode}
  \AddToHookNext{para/begin}{\makebox[0pt][r]{\MakeLinkTarget*{\l_@@_bib_target_tl}\hspace{\leftmargin}}}
  \item[\@biblabel{#1}\hfill]\if@filesw
      {\let\protect\noexpand
       \immediate
       \write\@auxout{\string\bibcite{#2}{#1}}}\fi\ignorespaces}
%    \end{macrocode}
% \end{macro}
% 
% \begin{macro}{\@bibitem}
% Similar for bibitem. 
% TODO: If hyperref is loaded we will get a second target from the refstepcounter,
% but this is ignored for now. 
%    \begin{macrocode}
\def\@bibitem#1{
  \tl_set:Nx\l_@@_bib_target_tl{cite.#1\@extra@b@citeb}%
  \AddToHookNext{para/begin}{\makebox[0pt][r]{\MakeLinkTarget*{\l_@@_bib_target_tl}\hspace{\leftmargin}}}
  \item\if@filesw \immediate\write\@auxout
       {\string\bibcite{#1}{\the\value{\@listctr}}}\fi\ignorespaces}
%    \end{macrocode}
% \end{macro}
% 
% TODO The LI-structure should set a label, we redefine the internal command locally for 
% now, but perhaps this need a receipe? 
% 
%    \begin{macrocode}
\AddToHook{env/thebibliography/begin}
  {\cs_set:Npn \__block_list_item_begin:
    { \tagstructbegin{tag=\LItag,label=\l_@@_bib_target_tl}}
  }
%    \end{macrocode}
%
% \subsection{Handling citation commands}
% We redefine similar to hyperref the \cs{bibcite} command to inject link and
% structure. Even if it looks a bit odd it is now used for many years and so 
% hopefully compatible with various packages. But differently to hyperref we use
% the new hooks with arguments.
% TODO: consider name. Perhaps use the generic names?
%    \begin{macrocode}
\NewMirroredHookPairWithArguments{bibcite/before}{bibcite/after}{2}  
\def\bibcite#1#2{%
   \@newl@bel{b}{#1\@extra@binfo}{%
      \UseHookWithArguments{bibcite/before}{2}{#1}{#2}
       #2
      \UseHookWithArguments{bibcite/after}{2}{#1}{#2} 
       }%
     }% 
%    \end{macrocode}
% Now we add the tagging structure.
% TODO: the ref key should expand its argument directly!
%    \begin{macrocode}
\AddToHookWithArguments{bibcite/before}
  {
    \tag_mc_end_push:
    \exp_args:Nx\tagstructbegin{tag=Reference,ref=cite.#1\@extra@b@citeb}
    \tagmcbegin{}
  }
\AddToHookWithArguments{bibcite/after}[tag]
  {
    \tag_mc_end:
    \tagstructend
    \tag_mc_begin_pop:n{}
  }
%    \end{macrocode}
% At last the code for hyperref, the link will be inside the reference, but
% this can be changed with a rule.
%    \begin{macrocode}
\AddToHook{package/hyperref/after}
 {
  \AddToHookWithArguments{bibcite/before}{\hyper@linkstart{cite}{cite.#1\@extra@b@citeb}}
  \AddToHookWithArguments{bibcite/after}{\hyper@linkend}
 }
%    \end{macrocode}
%
% \subsection{Natbib support}
%  natbib offers various hooks that can be used. The main problem is
%  to coordinate with the hyperref use of the same hooks.
%  We also have to add something at the begin of \cs{@lbibitem}. 
%  As generic hooks with arguments aren't available yet, we have to copy the definition
%  
%    \begin{macrocode}
\AddToHook{package/natbib/after}
  {
    \def\hyper@natanchorstart#1{\MakeLinkTarget*{#1}}
    \def\@lbibitem[#1]#2{%
      \tl_set:Nx\l__tag_bib_target_tl{cite.#2\@extra@b@citeb}
      \if\relax\@extra@b@citeb\relax\else
        \@ifundefined{br@#2\@extra@b@citeb}{}{%
         \@namedef{br@#2}{\@nameuse{br@#2\@extra@b@citeb}}%
        }%
      \fi
      \@ifundefined{b@#2\@extra@b@citeb}{%
       \def\NAT@num{}%
      }{%
       \NAT@parse{#2}%
      }%
      \def\NAT@tmp{#1}%
      \expandafter\let\expandafter\bibitemOpen\csname NAT@b@open@#2\endcsname
      \expandafter\let\expandafter\bibitemShut\csname NAT@b@shut@#2\endcsname
      \@ifnum{\NAT@merge>\@ne}{%
       \NAT@bibitem@first@sw{%
        \@firstoftwo
       }{%
        \@ifundefined{NAT@b*@#2}{%
         \@firstoftwo
        }{%
         \expandafter\def\expandafter\NAT@num\expandafter{\the\c@NAT@ctr}%
         \@secondoftwo
        }%
       }%
      }{%
       \@firstoftwo
      }%
      {%
       \global\advance\c@NAT@ctr\@ne
       \@ifx{\NAT@tmp\@empty}{\@firstoftwo}{%
        \@secondoftwo
       }%
       {%
        \expandafter\def\expandafter\NAT@num\expandafter{\the\c@NAT@ctr}%
        \global\NAT@stdbsttrue
       }{}%
       \bibitem@fin
       \item[\hfil\NAT@anchor{#2}{\NAT@num}]%
       \global\let\NAT@bibitem@first@sw\@secondoftwo
       \NAT@bibitem@init
      }%
      {%
       \NAT@anchor{#2}{}%
       \NAT@bibitem@cont
       \bibitem@fin
      }%
      \@ifx{\NAT@tmp\@empty}{%
        \NAT@wrout{\the\c@NAT@ctr}{}{}{}{#2}%
      }{%
        \expandafter\NAT@ifcmd\NAT@tmp(@)(@)\@nil{#2}%
      }%
    }%
%    \end{macrocode}
% we redefine the hook to use latex hooks.
%    \begin{macrocode}
    \NewMirroredHookPairWithArguments{natbib/linkstart}{natbib/linkend}{1}  
    \renewcommand\hyper@natlinkstart[1]{\UseHookWithArguments{natbib/linkstart}{1}{#1}}
    \renewcommand\hyper@natlinkend{\UseHookWithArguments{natbib/linkend}{1}{}}
    \AddToHookWithArguments{natbib/linkstart}
      {
         \leavevmode
         \tag_mc_end_push:
         \exp_args:Nx\tagstructbegin{tag=Reference,ref=cite.#1\@extra@b@citeb}
         \tagmcbegin{}
      } 
    \AddToHook{natbib/linkend}
      {  
       \tag_mc_end:
       \tagstructend
       \tag_mc_begin_pop:n{}
      }
  }     
%    \end{macrocode}
% if hyperref is loaded we have to repeat the definition
% 
%    \begin{macrocode}
\AddToHook{package/hyperref/after}
 {
    \renewcommand\hyper@natlinkstart[1]{\UseHookWithArguments{natbib/linkstart}{1}{#1}}
    \renewcommand\hyper@natlinkend{\UseHookWithArguments{natbib/linkend}{1}{}}
    \AddToHookWithArguments{natbib/linkstart}
      {
         \Hy@backout{#1}%
         \hyper@linkstart{cite}{cite.#1}%
         \def\hyper@nat@current{#1}
      } 
    \AddToHook{natbib/linkend}
      {  
       \hyper@linkend
      }
 }
%    \end{macrocode}
%    \begin{macrocode}
%</package>
%    \end{macrocode}
%    \begin{macrocode}
%<*latex-lab>
\ProvidesFile{bib-latex-lab-testphase.ltx}
        [2023-05-18 v0.8 code related to the tagging of bib and citations]

\RequirePackage{latex-lab-testphase-bib}

%</latex-lab>
%    \end{macrocode}
