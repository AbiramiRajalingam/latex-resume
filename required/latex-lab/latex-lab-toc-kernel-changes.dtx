% \iffalse meta-comment
%
%% File: toc-kernel-functions.dtx (C) Copyright 2022-2023 LaTeX Project
%
% It may be distributed and/or modified under the conditions of the
% LaTeX Project Public License (LPPL), either version 1.3c of this
% license or (at your option) any later version.  The latest version
% of this license is in the file
%
%    https://www.latex-project.org/lppl.txt
%
%
% The development version of the bundle can be found below
%
%    https://github.com/latex3/latex2e/required/latex-lab
%
% for those people who are interested or want to report an issue.
%
%<*driver>
\documentclass{l3doc}
\EnableCrossrefs
\CodelineIndex
\begin{document}
  \DocInput{latex-lab-toc-kernel-changes.dtx}
\end{document}
%</driver>
%
% \fi
%
% \title{The \textsf{toc-kernel-changes} package\\
% Changes and additions to the kernel related to tagging and links in toc entries}
% \author{\LaTeX{} Project\thanks{Initial implementation done by Ulrike Fischer}}
% \date{Version 0.81 2023-02-10}
%
% \maketitle
%
% \newcommand{\xt}[1]{\textsl{\textsf{#1}}}
% \newcommand{\TODO}[1]{\textbf{[TODO:} #1\textbf{]}}
% \newcommand{\docclass}{document class \marginpar{\raggedright document class
% customizations}}
%
% \providecommand\hook[1]{\texttt{#1}}
%
% \begin{abstract}
% \end{abstract}
%
% \section{Introduction}
%
% The followings contains various changes to kernel commands
% needed for the tagging of table of contents and similar lists.
% \section{General kernel and class changes}
% 
% Some of the changes can probably could go into the kernel. 
% This are marked with [kernel?],
% the other are marked with [latex-lab]. A few changes must be done in the classes.
% They are marked as [class]. They are inserted with hooks here: this will break
% if a non-standard class is used.
%
% \subsection{Providing the counter representation \cs{theHxx} generally}
% [kernel?]
%
% The \cs{theHxx} representation allows to create a unique representation of a counter
% that is for example used to create destination names. It will also be needed
% to add |/Ref| keys to various tagging structures. It makes therefore sense
% to provide it by default as soon as a counter is created. The |\@addtoreset| change
% (taken also from hyperref) tries to ensure the counter stays unique if it is reset.
%
% At first we suppress the patches from hyperref:
%    \begin{macrocode}
%<*kernelchange>
\def\hyper@nopatch@counter{}
%    \end{macrocode}
%
%    \begin{macrocode}
\def\@definecounter#1{\expandafter\newcount\csname c@#1\endcsname
     \setcounter{#1}\z@
     \global\expandafter\let\csname cl@#1\endcsname\@empty
     \@addtoreset{#1}{@ckpt}%
     \global\expandafter\let\csname p@#1\endcsname\@empty
     \expandafter\gdef\csname theH#1\endcsname{\the\value{#1}}%new
     \expandafter
     \gdef\csname the#1\expandafter\endcsname\expandafter
          {\expandafter\@arabic\csname c@#1\endcsname}}
\def\@addtoreset#1#2{\expandafter\@cons\csname cl@#2\endcsname {{#1}}%
  \expandafter\gdef\csname theH#1\endcsname{\csname theH#2\endcsname.\the\value{#1}}%
}
%    \end{macrocode}
% The following counters are defined in the kernel
%    \begin{macrocode}
\gdef\theHenumi{\the\value{enumi}}
\gdef\theHenumii{\the\value{enumii}}
\gdef\theHenumiii{\the\value{enumiii}}
\gdef\theHenumiv{\the\value{enumiv}}
%    \end{macrocode}
%
% \subsection{Providing and updating \cs{@currentHref}}
% [kernel?]
%
%  \cs{@currentHref} contains the current unique representation of a counter.
%  It is useful also without hyperref as it allows to connect with a |/Ref|
%  key a toc entry with the heading it refers too. So \cs{refstepcounter}
%  is extended to update the command. This is done at the begin so that the
%  hyperref code can still adapt it later (using the cmd/refstepcounter/after hook
%  would be too late, and could also break with cleverref.)
%
%    \begin{macrocode}
\def\@currentHref{}
\def\refstepcounter#1{\stepcounter{#1}%
    \edef\@currentcounter{#1}%
    \xdef\@currentHref {#1.\csname theH#1\endcsname}%
    \protected@edef\@currentlabel
       {\csname p@#1\expandafter\endcsname\csname the#1\endcsname}%
}
%    \end{macrocode}
%
% \subsection{Assigning the headings level numbers}
% [kernel?]
%
% The code for the toc must be able to compare the level of |subsection| against
% |section|. So some numbers are needed. It uses for now the commands used by
% |hyperref| for the bookmark levels. While headings are more a class thing, it
% could make sense to provide a standard set already in the kernel.
% The numbers are a bit different to the one in hyperref, but imho sensible.
% 
% Note for the future: 
% once that heading stuff is turned into template 
% instances the template should set that number.
%    \begin{macrocode}
\def\toclevel@part{-1}
\def\toclevel@chapter{0}
\def\toclevel@section{1}
\def\toclevel@subsection{2}
\def\toclevel@subsubsection{3}
\def\toclevel@paragraph{4}
\def\toclevel@subparagraph{5}
\def\toclevel@figure{1}
\def\toclevel@table{1}
%</kernelchange>
%    \end{macrocode}
% \subsection{load kernel changes}
%    \begin{macrocode}
%<*package>
\RequirePackage{latex-lab-kernel-changes}
%    \end{macrocode}
%
% \subsection{\cs{contentsline} extension}
% [latex-lab]
% The tagging code must add code at the begin and end of \cs{contentsline}.
% This code must have access to the arguments. We set it up as config point
% for now.
%    \begin{macrocode}
\def\@contentsline@cfgpoint@before#1#2#3#4{}
\def\@contentsline@cfgpoint@after#1#2#3#4{}
\def\@contentsline@arg@one{}%level
\def\@contentsline@arg@two{}%text
\def\@contentsline@arg@three{}%page
\def\@contentsline@arg@four{}%destination  
\def\contentsline#1#2#3#4%
  {%
    \def\@contentsline@arg@one{#1}%level
    \def\@contentsline@arg@two{#2}%text
    \def\@contentsline@arg@three{#3}%page
    \def\@contentsline@arg@four{#4}%destination  
    \@contentsline@cfgpoint@before{#1}{#2}{#3}{#4}%
    %\gdef\@contentsline@destination{#4}%
    \csname l@#1\endcsname{#2}{#3}%
    \@contentsline@cfgpoint@after{#1}{#2}{#3}{#4}%
  }
%    \end{macrocode}
%
% \subsection{\cs{addcontentsline} changes}
% [latex-lab]
%
% \cs{addcontentsline} should always write the unique counter representation
% \cs{@currentHref} so that we can create the |/Ref| key. At the same time
% we need a hook, so that hyperref can add its bookmarks code. The generic hook
% would be ok, but we need the arguments, so we store them first.
%
%    \begin{macrocode}
\NewHook{cmd/addcontentsline/before}
\def\addcontentsline#1#2#3{% toc extension, type, tag
    \def\addcontentsline@arg@one{#1}% name??
    \def\addcontentsline@arg@two{#2}%
    \def\addcontentsline@arg@three{#3}%
    \UseHook{cmd/addcontentsline/before}%
    \addtocontents{#1}{%
      \protect\contentsline{#2}{#3}{\thepage}{\@currentHref}\protected@file@percent
    }}
%    \end{macrocode}
%
% \subsection{\cs{@starttoc}}
% [kernel?]
%
% We add a configuration point before and after reading the file
% for the tagging.
%    \begin{macrocode}
\def\@starttoc@cfgpoint@before#1{}
\def\@starttoc@cfgpoint@after#1{}
\def\@starttoc#1{%
  \begingroup
    \makeatletter
    \@starttoc@cfgpoint@before{#1}%
    \@input{\jobname.#1}%
    \@starttoc@cfgpoint@after{#1}
    \if@filesw
      \expandafter\newwrite\csname tf@#1\endcsname
      \immediate\openout \csname tf@#1\endcsname \jobname.#1\relax
    \fi
    \@nobreakfalse
  \endgroup}
%    \end{macrocode}
%
% \subsection{Formatting of the toc entries}
%
% \subsubsection{Kernel commands}
% [latex-lab] or [kernel?]?
%
% We need hooks before and after the content and before and after the page number.
% The number needs some thing so that it can be tagged as label.
% The text and the page
% hooks are used in manual \cs{l@xx} commands and in \cs{l@xx} defined by
% \cs{@dottedtocline} or by other means.
%    \begin{macrocode}
\NewMirroredHookPair{contentsline/number/before}{contentsline/number/after}%
\NewMirroredHookPair{contentsline/text/before}{contentsline/text/after}%
\NewMirroredHookPair{contentsline/page/before}{contentsline/page/after}%
%    \end{macrocode}
%
% The dot in the leader must be set as artifact.
% But luatex requires that the mc are set outside the leaders.
% So we need a config point.
%
% \cs{@dottedtocline} needs hooks and formatting commands.
%    \begin{macrocode}
\def\@dottedtocline@cfgpoint@leaders#1{#1}

\def\@dottedtocline#1#2#3#4#5{%
  \ifnum #1>\c@tocdepth \else
    \vskip \z@ \@plus.2\p@
    {\leftskip #2\relax \rightskip \@tocrmarg \parfillskip -\rightskip
     \parindent #2\relax\@afterindenttrue
     \interlinepenalty\@M
     \leavevmode
     \@tempdima #3\relax
     \advance\leftskip \@tempdima \null\nobreak\hskip -\leftskip
     \UseHook{contentsline/text/before}%
     \csname contentsline@text@#1@format\endcsname{#4}%
     \UseHook{contentsline/text/after}%
     \nobreak
     \@dottedtocline@cfgpoint@leaders{%
       \leaders\hbox{$\m@th
         \mkern \@dotsep mu\hbox{.}\mkern \@dotsep
          mu$}\hfill}%
     \nobreak
     \hb@xt@\@pnumwidth{\hfil\normalfont \normalcolor
     \UseHook{contentsline/page/before}%
     #5%
     \UseHook{contentsline/page/after}%
                        \kern-\p@\kern\p@}%
     \par}%
  \fi}
%    \end{macrocode}
%
% \subsubsection{\cs{numberline}}
% [kernel?]
% Inside or outside the box?
%    \begin{macrocode}
\def\numberline#1{\hb@xt@\@tempdima{%
  \UseHook{contentsline/number/before}%
  #1\hfil
  \UseHook{contentsline/number/after}%
  }}
%    \end{macrocode}
%
% \subsubsection{\cs{l@xx} in the classes}
% Depending on the class we have to redefine also more commands
% For other classes it should be documented what is needed.
%    \begin{macrocode}
\AddToHook{class/article/after}
 {
  \renewcommand*\l@part[2]{%
  \ifnum \c@tocdepth >-2\relax
    \addpenalty\@secpenalty
    \addvspace{2.25em \@plus\p@}%
    \setlength\@tempdima{3em}%
    \begingroup
      \parindent \z@ \rightskip \@pnumwidth
      \parfillskip -\@pnumwidth
      {\leavevmode
       \large \bfseries
       \UseHook{contentsline/text/before}%
       \csname contentsline@text@-1@format\endcsname{#1}%
       \UseHook{contentsline/text/after}%
       \hfil
       \hb@xt@\@pnumwidth{\hss
       \UseHook{contentsline/page/before}%
       #2%
       \UseHook{contentsline/page/after}%
                          \kern-\p@\kern\p@}}\par
       \nobreak
    \endgroup
  \fi}
\renewcommand*\l@section[2]{%
  \ifnum \c@tocdepth >\z@
    \addpenalty\@secpenalty
    \addvspace{1.0em \@plus\p@}%
    \setlength\@tempdima{1.5em}%
    \begingroup
      \parindent \z@ \rightskip \@pnumwidth
      \parfillskip -\@pnumwidth
      \leavevmode \bfseries
      \advance\leftskip\@tempdima
      \hskip -\leftskip
       \UseHook{contentsline/text/before}%
       \csname contentsline@text@1@format\endcsname{#1}%
       \UseHook{contentsline/text/after}%
       \nobreak\hfil
      \nobreak\hb@xt@\@pnumwidth{\hss
      \UseHook{contentsline/page/before}%
       #2%
      \UseHook{contentsline/page/after}%
      \kern-\p@\kern\p@}\par
    \endgroup
  \fi}
 }
\AddToHook{class/report/after}
 {
    \renewcommand*\l@part[2]{%
     \ifnum \c@tocdepth >-2\relax
       \addpenalty{-\@highpenalty}%
       \addvspace{2.25em \@plus\p@}%
       \setlength\@tempdima{3em}%
       \begingroup
         \parindent \z@ \rightskip \@pnumwidth
         \parfillskip -\@pnumwidth
         {\leavevmode
          \large \bfseries
          \UseHook{contentsline/text/before}%
          \csname contentsline@text@-1@format\endcsname{#1}%
          \UseHook{contentsline/text/after}%
          \hfil
          \hb@xt@\@pnumwidth{\hss
          \UseHook{contentsline/page/before}%
          #2%
          \UseHook{contentsline/page/after}%
                             \kern-\p@\kern\p@}}\par
          \nobreak
       \endgroup
     \fi}
   \renewcommand*\l@chapter[2]{%
     \ifnum \c@tocdepth >\m@ne
       \addpenalty{-\@highpenalty}%
       \vskip 1.0em \@plus\p@
       \setlength\@tempdima{1.5em}%
       \begingroup
         \parindent \z@ \rightskip \@pnumwidth
         \parfillskip -\@pnumwidth
         \leavevmode \bfseries
         \advance\leftskip\@tempdima
         \hskip -\leftskip
         \UseHook{contentsline/text/before}%
         \csname contentsline@text@0@format\endcsname
           {#1}%
         \UseHook{contentsline/text/after}%
          \nobreak\hfil
         \nobreak\hb@xt@\@pnumwidth{\hss
          \UseHook{contentsline/page/before}%
          #2%
          \UseHook{contentsline/page/after}%
                                    \kern-\p@\kern\p@}\par
         \penalty\@highpenalty
       \endgroup
     \fi}
 }
\AddToHook{class/book/after}
 {
    \renewcommand*\l@part[2]{%
     \ifnum \c@tocdepth >-2\relax
       \addpenalty{-\@highpenalty}%
       \addvspace{2.25em \@plus\p@}%
       \setlength\@tempdima{3em}%
       \begingroup
         \parindent \z@ \rightskip \@pnumwidth
         \parfillskip -\@pnumwidth
         {\leavevmode
          \large \bfseries
          \UseHook{contentsline/text/before}%
          \csname contentsline@text@-1@format\endcsname{#1}%
          \UseHook{contentsline/text/after}%
          \hfil
          \hb@xt@\@pnumwidth{\hss
          \UseHook{contentsline/page/before}%
          #2%
          \UseHook{contentsline/page/after}%
                             \kern-\p@\kern\p@}}\par
          \nobreak
       \endgroup
     \fi}
   \renewcommand*\l@chapter[2]{%
     \ifnum \c@tocdepth >\m@ne
       \addpenalty{-\@highpenalty}%
       \vskip 1.0em \@plus\p@
       \setlength\@tempdima{1.5em}%
       \begingroup
         \parindent \z@ \rightskip \@pnumwidth
         \parfillskip -\@pnumwidth
         \leavevmode \bfseries
         \advance\leftskip\@tempdima
         \hskip -\leftskip
         \UseHook{contentsline/text/before}%
         \csname contentsline@text@0@format\endcsname
           {#1}%
         \UseHook{contentsline/text/after}%
          \nobreak\hfil
         \nobreak\hb@xt@\@pnumwidth{\hss
          \UseHook{contentsline/page/before}%
          #2%
          \UseHook{contentsline/page/after}%
                                    \kern-\p@\kern\p@}\par
         \penalty\@highpenalty
       \endgroup
     \fi}
 }
%    \end{macrocode}
%    \begin{macrocode}
%</package>
%    \end{macrocode}

%    \begin{macrocode}
%<*latex-lab>
\ProvidesFile{toc-tagging-latex-lab-testphase.ltx}
        [2022-10-26 v0.8 code related to the tagging of toc and similar lists]

\RequirePackage{latex-lab-testphase-toc-tagging}

%</latex-lab>
%    \end{macrocode}
