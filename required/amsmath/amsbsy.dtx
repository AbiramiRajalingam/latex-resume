%
% \iffalse meta-comment
%
% Copyright (C) 1995, 1999 American Mathematical Society.
% Copyright (C) 2016-2021 LaTeX3 Project and American Mathematical Society.
%
% This work may be distributed and/or modified under the
% conditions of the LaTeX Project Public License, either version 1.3c
% of this license or (at your option) any later version.
% The latest version of this license is in
%   https://www.latex-project.org/lppl.txt
% and version 1.3c or later is part of all distributions of LaTeX
% version 2005/12/01 or later.
%
% This work has the LPPL maintenance status `maintained'.
%
% The Current Maintainer of this work is the LaTeX3 Project.
%
% \fi
%
%\iffalse
%<*driver>
\documentclass{amsdtx}
\def\MaintainedByLaTeXTeam#1{%
\begin{center}%
\fbox{\fbox{\begin{tabular}{@{}l@{}}%
This file is maintained by the \LaTeX{} Project team.\\%
Bug reports can be opened (category \texttt{#1}) at\\%
\url{https://latex-project.org/bugs/}.\end{tabular}}}\end{center}}
\usepackage{amsbsy}
\GetFileInfo{amsbsy.sty}
\begin{document}
\title{The \pkg{amsbsy} package}
\author{Frank Mittelbach\and Rainer Sch\"opf\and Michael Downes}
\date{Version \fileversion, \filedate}
\DocInput{amsbsy.dtx}
\end{document}
%</driver>
%\fi
%
% \maketitle
% \MaintainedByLaTeXTeam{amslatex}
%
% \MakeShortVerb\|
%
% \section{Introduction}
%
%    The package \texttt{amsbsy}, first written in 1989, implements a
%    few commands for producing \textbf{bold} characters in the `normal'
%    \emph{math version}.
% \begin{quote}\em
%    Note: It is recommended nowadays to use the \pkg{bm} package, which
%    became available in 1997.
% \end{quote}
%
%    If we have bold fonts which contain the character in question then
%    we will use these fonts to produce the wanted character. But
%    sometimes math fonts are only available in a certain weight
%    (e.g.\ the AMS symbol fonts).
% \DescribeMacro\pmb
%    For these cases we provide a command which is called \verb+\pmb+ (an
%    acronym for \textbf{p}oor \textbf{m}an's \textbf{b}old) with one argument.
%    The bolder weight is achieved by copying the argument three times in
%    slightly different positions which is better than nothing but no
%    match for a real bold font.
%
% \DescribeMacro\boldsymbol
%    There also exists the \verb+\boldsymbol+ command which is better in
%    all cases where bold fonts exists. This command will internally
%    switch to the corresponding `bold' \emph{math version} typeset its
%    argument in this version.
%
%    Both commands will preserve the nature of their arguments, i.e.\ if
%    they get a relational atom their result will again be a relation as
%    far as \tex/'s mathspacing is concerned.
%
%    Since it is good policy to make at least a small test we try to
%    typeset the infinity sign ($\infty$) first with \verb+\pmb+ and then
%    with \verb+\boldsymbol+.
% \[
%           \pmb{\infty} = \boldsymbol{\infty} \quad ?
% \]
% \StopEventually{}
%
%    Standard package info.
%    \begin{macrocode}
\NeedsTeXFormat{LaTeX2e}% LaTeX 2.09 can't be used (nor non-LaTeX)
[1994/12/01]% LaTeX date must December 1994 or later
\ProvidesPackage{amsbsy}[1999/11/29 v1.2d Bold Symbols]
%    \end{macrocode}
%
% \section{The implementation}
%
%    We need some functions from the \pkg{amsgen} package.
%    \begin{macrocode}
\RequirePackage{amsgen}
%    \end{macrocode}
%
%\begin{macro}{\boldsymbol}
%    In implementing boldsymbol, we must take into account \tex/'s
%    limitation of only 16 mathgroups (math families, in Knuth's
%    terminology). If we wanted to maintain mathgroups for both the bold
%    and non-bold version of each math font, it would not take long to
%    run out of mathgroups. Therefore what we do instead for a bold
%    symbol is embed it in an \verb|\hbox|; inside that \verb|\hbox|,
%    when we start another math formula, we can change all the
%    mathgroups to their bold equivalents.
%
%    However, to get the correct math style inside the hbox (display,
%    text, script or scriptscript) we have to use \verb|\mathchoice|.
%    Since \verb|\mathversion{bold}| has a lot of overhead, and
%    \verb|\mathchoice| typesets the argument text four times, we would
%    rather not put the \verb|\mathversion| command inside each
%    \verb|\hbox| in the \verb|\mathchoice|; on the other hand,
%    \verb|\mathversion| gives an error message if it's used in math
%    mode. Therefore if we want to execute \verb|\mathversion{bold}|
%    before starting the \verb|\mathchoice| we have to temporarily
%    disable the \verb|\@nomath| error. (The error message is intended
%    to keep people from accidentally emboldening a preceding part of a
%    math formula, since only the mathgroups defined at the end of a
%    math formula will determine the fonts used in that formula; but we
%    are going to typeset our bold symbol not in the current formula but
%    in an embedded formula, so that this danger doesn't apply here.)
%    \begin{macrocode}
\DeclareRobustCommand{\boldsymbol}[1]{%
%    \end{macrocode}
%    Start a group to localize the change of \verb|\@nomath|:
%    \begin{macrocode}
  \begingroup
%    \end{macrocode}
%    Disable \verb|\@nomath| so that we don't have to leave math
%    mode before executing \verb|\mathversion|:
%    \begin{macrocode}
  \let\@nomath\@gobble \mathversion{bold}%
%    \end{macrocode}
%    \cs{math@atom} is a test macro which looks at its argument and
%    produces a math atom of the proper class.
%    \begin{macrocode}
  \math@atom{#1}{%
%    \end{macrocode}
%    Although it is tempting to use \verb|\text| here, to save some
%    main memory, that caused a bug in the past due to some internal
%    interactions with \verb|\mathversion|.
%    \begin{macrocode}
  \mathchoice%
    {\hbox{$\m@th\displaystyle#1$}}%
    {\hbox{$\m@th\textstyle#1$}}%
    {\hbox{$\m@th\scriptstyle#1$}}%
    {\hbox{$\m@th\scriptscriptstyle#1$}}}%
%    \end{macrocode}
%    End the group we started earlier.
%    \begin{macrocode}
  \endgroup}
%    \end{macrocode}
% \end{macro}
%
% \begin{macro}{\math@atom}
%    The macro \verb+\math@atom+ looks at its argument and produce a
%    correct math atom, i.e.\ a primitive like \verb+\mathopen+.
%    Until the day we have a real implementation for all cases we use
%    the \verb+\binrel@+ command from \amstex/ which can distinguish
%    between binary, relation and ord atoms.
%    \begin{macrocode}
\def\math@atom#1#2{%
   \binrel@{#1}\binrel@@{#2}}
%    \end{macrocode}
% \end{macro}
%
% \begin{macro}{\pmb}
%    Poor man's bold command, works by typesetting multiple copies of
%    the given argument with small offsets.
%    \begin{macrocode}
\DeclareRobustCommand{\pmb}{%
  \ifmmode\else \expandafter\pmb@@\fi\mathpalette\pmb@}
%    \end{macrocode}
%
%    \cs{pmb@@} is called by \cn{pmb} in the non-math-mode case.
%    Discard the first two arguments which are for the math-mode case.
%    \begin{macrocode}
\def\pmb@@#1#2#3{\leavevmode\setboxz@h{#3}%
   \dimen@-\wdz@
   \kern-.5\ex@\copy\z@
   \kern\dimen@\kern.25\ex@\raise.4\ex@\copy\z@
   \kern\dimen@\kern.25\ex@\box\z@
}
%    \end{macrocode}
%
%    \begin{macrocode}
\newdimen\pmbraise@
%    \end{macrocode}
%    Note: because of the use of \cs{mathpalette}, if \cs{pmb@} is applied to a
%    single math italic character (or a single character from some other
%    slanted math font), the italic correction will be added. This will
%    cause subscripts to fall too far away from the character in some
%    cases, e.g., $\pmb{T}_1$ or $\pmb{\mathcal{T}}_1$.
%    \begin{macrocode}
\def\pmb@#1#2{\setbox8\hbox{$\m@th#1{#2}$}%
  \setboxz@h{$\m@th#1\mkern.5mu$}\pmbraise@\wdz@
  \binrel@{#2}%
  \dimen@-\wd8 %
  \binrel@@{%
    \mkern-.8mu\copy8 %
    \kern\dimen@\mkern.4mu\raise\pmbraise@\copy8 %
    \kern\dimen@\mkern.4mu\box8 }%
}
%    \end{macrocode}
% \end{macro}
%
%    \begin{macrocode}
\def\binrel@#1{\begingroup
  \setboxz@h{\thinmuskip0mu
    \medmuskip\m@ne mu\thickmuskip\@ne mu
    \setbox\tw@\hbox{$#1\m@th$}\kern-\wd\tw@
    ${}#1{}\m@th$}%
%    \end{macrocode}
%    The \cn{noexpand} here should be unnecessary, but just in case
%    \ldots
%    \begin{macrocode}
  \edef\@tempa{\endgroup\let\noexpand\binrel@@
    \ifdim\wdz@<\z@ \mathbin
    \else\ifdim\wdz@>\z@ \mathrel
    \else \relax\fi\fi}%
  \@tempa
}
%    \end{macrocode}
%    For completeness, assign a default value for \cs{binrel@@}.
%    \begin{macrocode}
\let\binrel@@\relax
%    \end{macrocode}
%
%    The usual \cs{endinput} to ensure that random garbage at the end of
%    the file doesn't get copied by \fn{docstrip}.
%    \begin{macrocode}
\endinput
%    \end{macrocode}
%
% \CheckSum{131}
% \Finale
