%
% \iffalse meta-comment
%
% Copyright (C) 1995, 1999 American Mathematical Society.
% Copyright (C) 2016-2019 LaTeX3 Project and American Mathematical Society.
%
% This work may be distributed and/or modified under the
% conditions of the LaTeX Project Public License, either version 1.3c
% of this license or (at your option) any later version.
% The latest version of this license is in
%   https://www.latex-project.org/lppl.txt
% and version 1.3c or later is part of all distributions of LaTeX
% version 2005/12/01 or later.
% 
% This work has the LPPL maintenance status `maintained'.
% 
% The Current Maintainer of this work is the LaTeX3 Project.
%
% \fi
%
% \iffalse
%<*driver>
\documentclass{amsdtx}
\def\MaintainedByLaTeXTeam#1{%
\begin{center}%
\fbox{\fbox{\begin{tabular}{@{}l@{}}%
This file is maintained by the \LaTeX{} Project team.\\%
Bug reports can be opened (category \texttt{#1}) at\\%
\url{https://latex-project.org/bugs/}.\end{tabular}}}\end{center}}
\usepackage{amstext}
\GetFileInfo{amstext.sty}
\begin{document}
\title{The \pkg{amstext} package}
\author{Frank Mittelbach \and Rainer Sch\"opf}
\date{Version \fileversion, \filedate}
\DocInput{amstext.dtx}
\end{document}
%</driver>
% \fi
%
% \maketitle
% \MaintainedByLaTeXTeam{amslatex}
%
% \MakeShortVerb\|
%
% \section{Introduction}
%
%    This style file implements the \AmSTeX{} macro \cn{text} for use
%    with the new font selection scheme.  The \cn{text} macro is a
%    sophisticated command which allows the user to insert ``normal text''
%    into math formulas without worrying about correct sizes in sub- or
%    superscripts. It can also be used in ordinary text; there it
%    produces an unbreakable unit similar to \cn{mbox}.
%
%    Here is an example demonstrating some of its features:
% \[
%^^A mathbin spacing is nonscript glue, disappears in superscript
%    x^{2\,\times\,\text{size of $y$}}
%    \leq
%    z_{i_{\text{upper bound of the array}}}
% \]
%    This was produced by
% \begin{verbatim}
% \[
%    x^{2\,\times\,\text{size of $y$}}
%    \leq
%    z_{i_{\text{upper bound of the array}}}
% \]
% \end{verbatim}
%
%    Additionally this style file redefines an internal \fn{plain.tex}
%    macro called \cs{mathhexbox} so that commands like \cn{dag}
%    or \cn{P} will change sizes if used in math subscripts.
%
% \StopEventually{}
%    Package information.
%    \begin{macrocode}
\NeedsTeXFormat{LaTeX2e}% LaTeX 2.09 can't be used (nor non-LaTeX)
[1994/12/01]% LaTeX date must be December 1994 or later
\ProvidesPackage{amstext}[2000/06/29 v2.01 AMS text]
%    \end{macrocode}
%
% \section{The implementation}
%    We need a few tools from \fn{amsgen.sty}.
%    \begin{macrocode}
\RequirePackage{amsgen}
%    \end{macrocode}
%
% \begin{macro}{\text}
%    Now we come to the \cn{text} macro which is used to place
%    ordinary text inside of math formulas. If it is used outside
%    math it will produce an unbreakable unit of text.
%    \begin{macrocode}
\DeclareRobustCommand{\text}{%
  \ifmmode\expandafter\text@\else\expandafter\mbox\fi}
%    \end{macrocode}
%    At the present time (late 1994) the \latex/ internal function
%    |\nfss@text| is used in |\ref|, in font commands like |\textbf|,
%    and in a few text symbol definitions like |\$| and |\pounds|. By
%    equating |\nfss@text| to |\text| we give it the ability of
%    |\text| to change sizes properly if used in a subscript.
%    \begin{macrocode}
\let\nfss@text\text
%    \end{macrocode}
% \end{macro}
%
% \begin{macro}{\text@}
%    If \cn{text} is encountered inside math mode the macro
%    \cs{text@} is called.  It has one mandatory argument, the text
%    which should be produced.  Since we do not know in which math
%    style we are currently in we call \cs{mathchoice} to tyeset
%    our text in all four possible styles.
%    \begin{macrocode}
\def\text@#1{{\mathchoice
%    \end{macrocode}
%    To save token space we call a macro \cs{textdef@} which takes
%    three arguments: the current math style, the corresponding size
%    macro and the text to typeset possibly with some additional
%    information for typesetting.
%    \begin{macrocode}
  {\textdef@\displaystyle\f@size{#1}}%
%    \end{macrocode}
%    The other three cases are similar except for the
%    \cs{iffirstchoice} switch which we set to false. This is
%    done to prevent \LaTeX{} macros like \cn{ref} or \cn{index}
%    from writing their arguments more than once.
%    \begin{macrocode}
  {\textdef@\textstyle\f@size{\firstchoice@false #1}}%
  {\textdef@\textstyle\sf@size{\firstchoice@false #1}}%
  {\textdef@\textstyle \ssf@size{\firstchoice@false #1}}%
%    \end{macrocode}
%    Here we need to check whether a math size-change occurred inside
%    the argument of \cn{text}. If so, restore
%    \begin{macrocode}
  \check@mathfonts
  }%
}
%    \end{macrocode}
%    The macros \cs{f@size}, \cs{sf@size} and \cs{ssf@size} hold the
%    sizes which should be used when we are loading a new font for use
%    in \cs{textfont}, \cs{scriptfont} and \cs{scriptscriptfont}. There
%    is some question whether we should use use \cs{tf@size} or
%    \cs{f@size} for the main size, but since the primary purpose of the
%    \cn{text} macro is to switch back to text within a display, it
%    seems that \cs{f@size} is the better choice. (Indeed it could be
%    said that the \cn{text} actually provides two different functions:
%    one for escaping out of math mode in a display to print some words,
%    and the other for handling math objects that are named by a
%    fragment of text, when \cn{operatorname} isn't the right choice.
%    For the latter \cs{tf@size} might be more correct but for the
%    former \cs{f@size} is clearly better.)
% \end{macro}
%
% \begin{macro}{\textdef@}
%    To typeset the argument of \cn{text} correctly we have to make
%    several actions. We start by placing everything inside an
%    \cs{hbox}.  But this is not enough: we need one extra level of
%    grouping.  These extra braces are necessary because of the new
%    font selection scheme which might produce an \cs{aftergroup}
%    to globally restore some font values after the current group. To
%    prevent any damage by this mechanism we add the braces thereby
%    bringing the token inserted by \cs{aftergroup} inside the
%    \cs{hbox}.\footnote{The mechanism will not produce a second
%    \cs{aftergroup}. For more details see the technical documentation
%    for NFSS2.}
%    \begin{macrocode}
\def\textdef@#1#2#3{\hbox{{%
%    \end{macrocode}
%    Since text typeset inside an \cs{hbox} always stays in the size
%    of the text surrounding the formula we have to adjust this for
%    script and scriptscript sizes.  For any math formula inside this
%    argument this will be achieved by setting \cs{everymath} to
%    the first argument of \cs{textdef@} since this argument
%    contains the math size in the current typeset case of
%    \cs{mathchoice}.
%    Since \latex/ also knows about \cn{parbox}es and the
%    \env{minipage} environment it might be neccessary to adjust
%    \cs{everydisplay} too but this has to be tested further.
%    \begin{macrocode}
                    \everymath{#1}%
%    \end{macrocode}
%    The next line of code changes locally (i.e.\ inside the current
%    \cs{hbox}) the value of \cs{f@size}. This macro holds the
%    size for typesetting ordinary text (e.g.\ loading or selecting a
%    new font via \cs{selectfont}). By changing it to a smaller
%    value a following \cs{selectfont} will switch to the wanted
%    size.
%    \begin{macrocode}
                    \let\f@size#2\selectfont
%    \end{macrocode}
%    Now we simply call the third argument and close all open groups.
%    \begin{macrocode}
                    #3}}}
%    \end{macrocode}
% \end{macro}
%
% \begin{macro}{\iffirstchoice@}
%    Here is the switch that we use to decide if \cn{ref} etc.\
%    should print its warnings.  The default is true since
%    normally these warnings shouldn't be suppressed.
%    \begin{macrocode}
\newif\iffirstchoice@
\firstchoice@true
%    \end{macrocode}
% \end{macro}
%
% \subsection{Re-definition of \latex/ macros to work with \cn{text}}
%
%    If a counter-changing command occurs inside the argument of
%    \cn{text}, we don't want the counter to be changed four times
%    because \cn{stepcounter} and \cn{addtocounter} have global effect.
%    So we add the \cs{iffirstchoice@} test to make the counter
%    operations execute only once.
%
% \begin{macro}{\stepcounter}
%    Use \cs{def} rather than \cn{renewcommand*} because the star-form
%    (for non-\cs{long} definitions) doesn't work with the June 1994
%    release of \latex/.
%    \begin{macrocode}
\def\stepcounter#1{%
  \iffirstchoice@
     \addtocounter{#1}\@ne
     \begingroup \let\@elt\@stpelt \csname cl@#1\endcsname \endgroup
  \fi
}
%    \end{macrocode}
% \end{macro}
%
% \begin{macro}{\addtocounter}
%    \begin{macrocode}
\def\addtocounter#1#2{%
  \iffirstchoice@
  \@ifundefined {c@#1}{\@nocounterr {#1}}%
    {\global \advance \csname c@#1\endcsname #2\relax}%
  \fi}
%    \end{macrocode}
% \end{macro}
%
%    For \cn{ref}, \cn{pageref}, and indeed anything else that issues a
%    warning or error, \cn{text} will produce four copies of the
%    warning/error message. To suppress the last three copies, we change
%    \cs{GenericInfo}, \cs{GenericWarning}, \cs{GenericError}.
%    \begin{macrocode}
\let\m@gobble\@empty
\@xp\let\csname m@gobble4\endcsname\@gobblefour
\long\@xp\def\csname m@gobble6\endcsname#1#2#3#4#5#6{}
%    \end{macrocode}
%
%    \begin{macrocode}
\toks@{%
  \csname m@gobble\iffirstchoice@\else 4\fi\endcsname
  \protect}
\edef\GenericInfo{\the\toks@
  \@xp\@nx\csname GenericInfo \endcsname}
%
\edef\GenericWarning{\the\toks@
  \@xp\@nx\csname GenericWarning \endcsname}
%
\toks@{%
  \csname m@gobble\iffirstchoice@\else 6\fi\endcsname
  \protect}
\edef\GenericError{\the\toks@
  \@xp\@nx\csname GenericError \endcsname}
%    \end{macrocode}
%
%    At one time \cn{label}, \cs{@wrindex} and \cs{@wrglossary} were
%    changed here too to use the \cs{iffirstchoice@} test but it seems
%    that was a mistake because those are non-immediate writes.
%    Something like
% \begin{verbatim}
% \text{something \index{foo}}
% \end{verbatim}
%    within a math formula would therefore \emph{lose the index term} if
%    the surrounding context was not displaystyle. (Unlikely in
%    practice, but not impossible.) [mjd,1994/12/09]
%
%\subsection{Applications of \cn{text}}
%
% \begin{macro}{\mathhexbox}
%    We start with an re-definition of the \fn{plain.tex} macro
%    \cs{mathhexbox}. (Although M. Spivak in \amstex/ uses the name
%    \cs{mathhexbox@} for this purpose, I [FMi] don't see any reason
%    to use a new name since the new definition is superior, has the
%    same syntax and is used for the same purpose.)
%    \begin{macrocode}
\begingroup \catcode`\"=12
\gdef\mathhexbox#1#2#3{\text{$\m@th\mathchar"#1#2#3$}}
\endgroup
%    \end{macrocode}
%    This redefinition means that now symbols like \S, \P, \dag, \ldots,
%    which are defined via \cs{mathhexbox} in \fn{plain.tex} or
%    elsewhere now correctly change sizes if they are used in math mode.
% \end{macro}
%
%    The usual \cs{endinput} to ensure that random garbage at the end of
%    the file doesn't get copied by \fn{docstrip}.
%    \begin{macrocode}
\endinput
%    \end{macrocode}
%
% \CheckSum{129}
% \Finale
