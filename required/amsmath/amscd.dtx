%
% \iffalse meta-comment
%
% Copyright (C) 1995, 1999 American Mathematical Society.
% Copyright (C) 2016-2021 LaTeX Project and American Mathematical Society.
%
% This work may be distributed and/or modified under the
% conditions of the LaTeX Project Public License, either version 1.3c
% of this license or (at your option) any later version.
% The latest version of this license is in
%   https://www.latex-project.org/lppl.txt
% and version 1.3c or later is part of all distributions of LaTeX
% version 2005/12/01 or later.
%
% This work has the LPPL maintenance status `maintained'.
%
% The Current Maintainer of this work is the LaTeX Project.
%
% \fi
%
%\iffalse
%<*driver>
\documentclass{amsdtx}
\def\MaintainedByLaTeXTeam#1{%
\begin{center}%
\fbox{\fbox{\begin{tabular}{@{}l@{}}%
This file is maintained by the \LaTeX{} Project team.\\%
Bug reports can be opened (category \texttt{#1}) at\\%
\url{https://latex-project.org/bugs/}.\end{tabular}}}\end{center}}
\usepackage{amscd}
\GetFileInfo{amscd.sty}
%    Dummy citation to avoid error in an example.
\bibcite{fre:cichon}{C}
\begin{document}
\title{The \pkg{amscd} package}
\author{Frank Mittelbach\and Rainer Sch\"opf\and Michael Downes}
\date{Version \fileversion, \filedate}
\DocInput{amscd.dtx}
\end{document}
%</driver>
%\fi
%
% \maketitle
% \MaintainedByLaTeXTeam{amslatex}
%
% \MakeShortVerb\|
% \newcommand{\operatorname}[1]{\mathop{\mathrm{#1}}}
% \newcommand{\per}{\operatorname{per}}
% \newcommand{\cov}{\operatorname{cov}}
% \newcommand{\non}{\operatorname{non}}
% \newcommand{\cf}{\operatorname{cf}}
% \newcommand{\add}{\operatorname{add}}
% \newcommand{\End}{\operatorname{End}}
%
% \section{Introduction}
%
%    The \pkg{amscd} package provides a \env{CD} environment that
%    emulates the commutative diagram capabilities of \amstex/ version
%    2.x. This means that only simple rectangular diagrams are
%    supported, with no diagonal arrows or more exotic features. Many
%    users will be better served by one of the more powerful diagram
%    packages such as \pkg{diagram}, \pkg{xypic}, or \pkg{kuvio}.
%
%    Example:
%    \begin{equation}\begin{CD}
%    S^{{\mathcal{W}}_\Lambda}\otimes T   @>j>>   T\\
%    @VVV                                    @VV{\End P}V\\
%    (S\otimes T)/I                  @=      (Z\otimes T)/J
%    \end{CD}\end{equation}
% \begin{verbatim}
% \begin{equation}\begin{CD}
% S^{{\mathcal{W}}_\Lambda}\otimes T   @>j>>   T\\
% @VVV                                    @VV{\End P}V\\
% (S\otimes T)/I                  @=      (Z\otimes T)/J
% \end{CD}\end{equation}
% \end{verbatim}
%    (assuming \cn{End} is defined as an `operator name'.
%
%    Another example:
%
%    We will make liberal use of Cichon's Diagram \cite{fre:cichon}:
%    \begin{equation}
%    \begin{CD}
%    \cov(\mathcal{L}) @>>> \non(\mathcal{K}) @>>> \cf(\mathcal{K}) @>>>
%    \cf(\mathcal{L})\\
%    @VVV @AAA @AAA @VVV\\
%    \add(\mathcal{L}) @>>> \add(\mathcal{K}) @>>> \cov(\mathcal{K}) @>>>
%    \non(\mathcal{L})
%    \end{CD}\end{equation}
% \begin{verbatim}
% \begin{equation}\begin{CD}
% \cov(\mathcal{L}) @>>> \non(\mathcal{K}) @>>> \cf(\mathcal{K}) @>>>
% \cf(\mathcal{L})\\
% @VVV @AAA @AAA @VVV\\
% \add(\mathcal{L}) @>>> \add(\mathcal{K}) @>>> \cov(\mathcal{K}) @>>>
% \non(\mathcal{L})
% \end{CD}\end{equation}
% \end{verbatim}
%
% \StopEventually{}
%
%    Standard package info.
%    \begin{macrocode}
\NeedsTeXFormat{LaTeX2e}% LaTeX 2.09 can't be used (nor non-LaTeX)
[1994/12/01]% LaTeX date must December 1994 or later
\ProvidesPackage{amscd}[2017/04/14 v2.1 AMS Commutative Diagrams]
%    \end{macrocode}
%    \begin{macrocode}
\RequirePackage{amsgen}
%    \end{macrocode}
%      Better not to redefine \verb|\math@cr| if it is already defined,
%      because for CD use only we want to omit the part of the code
%      related to \verb|\dspbrk@lvl| (see \fn{amsmath.sty}).
%
%    [mjd,1999/11/04] These definitions have gone somewhat obsolete; but
%    we had probably better leave them as they are for backward
%    compatibility.
%    \begin{macrocode}
\@ifundefined{math@cr}{%
  \def\math@cr{{\ifnum0=`}\fi
    \@ifstar{\global\@eqpen\@M\math@cr@}%
          {\global\@eqpen\interdisplaylinepenalty \math@cr@}}
%    \end{macrocode}
%    The following section merely duplicates some code from the
%    \pkg{amsmath} package, in case the \pkg{amscd} package is used by
%    itself. For documentation of the code refer to \fn{amsmath.dtx}.
%    \begin{macrocode}
  \def\math@cr@{\new@ifnextchar[\math@cr@@{\math@cr@@[\z@]}}
  \def\math@cr@@[#1]{\ifnum0=`{\fi}\math@cr@@@
    \noalign{\vskip#1\relax}}
  \def\restore@math@cr{\def\math@cr@@@{\cr}}
}{}
\restore@math@cr
%    \end{macrocode}
%    [mjd,1999/11/04] These definitions too are somewhat obsolete;
%    but we had probably better leave them as they are for backward
%    compatibility.
%    \begin{macrocode}
\@ifundefined{rightarrowfill@}{
  \def\rightarrowfill@#1{\m@th\setboxz@h{$#1-$}\ht\z@\z@
    $#1\copy\z@\mkern-6mu\cleaders
    \hbox{$#1\mkern-2mu\box\z@\mkern-2mu$}\hfill
    \mkern-6mu\mathord\rightarrow$}
  \def\leftarrowfill@#1{\m@th\setboxz@h{$#1-$}\ht\z@\z@
    $#1\mathord\leftarrow\mkern-6mu\cleaders
    \hbox{$#1\mkern-2mu\copy\z@\mkern-2mu$}\hfill
    \mkern-6mu\box\z@$}
  \def\leftrightarrowfill@#1{\m@th\setboxz@h{$#1-$}\ht\z@\z@
    $#1\mathord\leftarrow\mkern-6mu\cleaders
    \hbox{$#1\mkern-2mu\box\z@\mkern-2mu$}\hfill
    \mkern-6mu\mathord\rightarrow$}
}{}
%    \end{macrocode}
%
%    \begin{macrocode}
\def\atdef@#1{\expandafter\def\csname\space @\string#1\endcsname}
\@ifundefined{Iat}{%
  \DeclareRobustCommand{\Iat}{\FN@\at@}
}{}
\begingroup \catcode`\@=\active
%    \end{macrocode}
%    Define math \qc{\@} to replicate its mathcode-dictated behavior.
%    This is in case \qc{\@} occurs outside of \env{CD}.
%    \begin{macrocode}
\csname if\string @compatibility\endcsname
\else \fam=\mathcode`\@
  \xdef @{\mathchar\number\fam\space }
\fi
\gdef\CDat{\let @=\Iat}
\endgroup
\mathcode`\@="8000 % make @ pseudo-active in math
\def\at@{\let\next@\at@@
 \ifcat\noexpand\next a\else
 \ifcat\noexpand\next0\else
 \ifcat\noexpand\next\relax\else
 \let\next@\at@@@\fi\fi\fi\next@}
\def\at@@#1{\expandafter
  \ifx\csname\space @\string#1\endcsname\relax
    \DN@{\at@@@#1}%
  \else
    \DN@{\csname\space @\string#1\endcsname}%
  \fi\next@}%
%    \end{macrocode}
%    The following items should be defined only if they are not
%    already defined, either to leave the package name untouched (in
%    the case of \cs{PackageError}) or to avoid redundant allocation
%    of token or dimen registers.
%    \begin{macrocode}
\@ifundefined{default@tag}{%
  \def\default@tag{%
    \def\tag{\PackageError{amscd}{\protect\tag\space not allowed
      here}\@eha}}%
}{}%
\@ifundefined{at@@@}{%
  \def\at@@@{\PackageError{amscd}{\Invalid@@ @}{\the\athelp@}\char64\relax}
}{}
\@ifundefined{athelp@}{\csname newhelp\endcsname\athelp@
{Only certain combinations beginning with @ make sense to me.^^J%
I'll assume you wanted @@ for a printed @.}}{}
\@ifundefined{minaw@}{\newdimen\minaw@}{}
\@ifundefined{bigaw@}{\newdimen\bigaw@}{}
%    \end{macrocode}
%
%      Assignment of a couple of dimensions, and initialization of
%      \verb=\ampersand@=. We check to see if we need to define
%      \verb=\minaw@= and \verb=\bigaw@=.
%    \begin{macrocode}
\minaw@11.111pt
\newdimen\minCDarrowwidth
\minCDarrowwidth2.5pc
\newif\ifCD@
\let\ampersand@\relax
%    \end{macrocode}
%
%      Added \verb|\restore@math@cr\default@tag| to fix line numbering
%      problems, 7-JUN-1991. Suggested by I.~Zakharevich.
%    \begin{macrocode}
\newenvironment{CD}{%
  \CDat
  \bgroup\relax\iffalse{\fi\let\ampersand@&\iffalse}\fi
  \CD@true\vcenter\bgroup\let\\\math@cr\restore@math@cr\default@tag
  \tabskip\z@skip\baselineskip20\ex@
  \lineskip3\ex@\lineskiplimit3\ex@\halign\bgroup
  &\hfill$\m@th##$\hfill\crcr
}{%
  \crcr\egroup\egroup\egroup
}
%    \end{macrocode}
%
% \begin{macro}{\CD@check}
%    This check is used by all macros that must not appear outside the
%    \env{CD} environment. The first argument is the symbol to be used
%    after \verb+@+, the second one the action to be performed.
%
%    Once again we use the trick of defining a temporary control
%    sequence \verb+\next@+ and calling it after the final \verb+\fi+.
%    This is not absolutely necessary, but it ensures that the
%    conditional text is processed in one and the same column
%    of the enclosing alignment.
%    \begin{macrocode}
\def\CD@check#1#2{\ifCD@\DN@{#2}\else
  \DN@{\PackageError{amscd}{@\protect#1 not
    allowed outside of the CD environment}\@eha}%
  \fi\next@}
%    \end{macrocode}
% \end{macro}
%
%    \begin{macrocode}
\atdef@>#1>#2>{\ampersand@
  \ifCD@ \global\bigaw@\minCDarrowwidth \else \global\bigaw@\minaw@ \fi
  \setboxz@h{$\m@th\scriptstyle\;{#1}\;\;$}%
  \ifdim\wdz@>\bigaw@\global\bigaw@\wdz@\fi
%    \end{macrocode}
%      If \verb|#2| is empty we can save some work.
%    \begin{macrocode}
  \@ifnotempty{#2}{\setbox\@ne\hbox{$\m@th\scriptstyle\;{#2}\;\;$}%
    \ifdim\wd\@ne>\bigaw@\global\bigaw@\wd\@ne\fi}%
 \ifCD@\enskip\fi
   \mathrel{\mathop{\hbox to\bigaw@{\rightarrowfill@\displaystyle}}%
     \limits^{#1}\@ifnotempty{#2}{_{#2}}}%
 \ifCD@\enskip\fi \ampersand@}
%
\atdef@<#1<#2<{\ampersand@
  \ifCD@ \global\bigaw@\minCDarrowwidth \else \global\bigaw@\minaw@ \fi
  \setboxz@h{$\m@th\scriptstyle\;\;{#1}\;$}%
  \ifdim\wdz@>\bigaw@ \global\bigaw@\wdz@ \fi
  \@ifnotempty{#2}{\setbox\@ne\hbox{$\m@th\scriptstyle\;\;{#2}\;$}%
    \ifdim\wd\@ne>\bigaw@ \global\bigaw@\wd\@ne \fi}%
  \ifCD@\enskip\fi
    \mathrel{\mathop{\hbox to\bigaw@{\leftarrowfill@\displaystyle}}%
      \limits^{#1}\@ifnotempty{#2}{_{#2}}}%
  \ifCD@\enskip\fi \ampersand@}
%    \end{macrocode}
%
%      Variants of the above two arrows, using \verb|(| and \verb|)|
%      characters instead of \verb|<| and \verb|>| characters, are
%      provided for those whose keyboards don't have the latter.
%    \begin{macrocode}
\begingroup \catcode`\~=\active \lccode`\~=`\@
\lowercase{%
  \global\atdef@)#1)#2){~>#1>#2>}
  \global\atdef@(#1(#2({~<#1<#2<}
}% end lowercase
\endgroup
%    \end{macrocode}
%
%    \begin{macrocode}
\atdef@ A#1A#2A{\CD@check{A..A..A}{\llap{$\m@th\vcenter{\hbox
  {$\scriptstyle#1$}}$}\Big\uparrow
  \rlap{$\m@th\vcenter{\hbox{$\scriptstyle#2$}}$}&&}}
%
\atdef@ V#1V#2V{\CD@check{V..V..V}{\llap{$\m@th\vcenter{\hbox
  {$\scriptstyle#1$}}$}\Big\downarrow
  \rlap{$\m@th\vcenter{\hbox{$\scriptstyle#2$}}$}&&}}
%
\atdef@={\CD@check={&\enskip\mathrel
  {\vbox{\hrule\@width\minCDarrowwidth\vskip2\ex@\hrule\@width
  \minCDarrowwidth}}\enskip&}}
%
\atdef@|{\CD@check|{\Big\Vert&&}}
%
\atdef@\vert{\CD@check\vert{\Big\Vert&&}}
%
\atdef@.{\CD@check.{&&}}
%    \end{macrocode}
%
%    The \cn{pretend} command has weird syntax that doesn't fit well
%    with standard \latex/ syntax so we leave it undone, at least for
%    now. [mjd,1994/10/27]
%    \begin{macrocode}
%\def\pretend#1\haswidth#2{\setboxz@h{$\m@th\scriptstyle{#2}$}\hbox
% to\wdz@{\hfill$\m@th\scriptstyle{#1}$\hfill}}
%    \end{macrocode}
%
%    The usual \cs{endinput} to ensure that random garbage at the end of
%    the file doesn't get copied by \fn{docstrip}.
%    \begin{macrocode}
\endinput
%    \end{macrocode}
%
% \CheckSum{459}
% \Finale
