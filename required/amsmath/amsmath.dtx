%
% \iffalse meta-comment
%
% Copyright (C) 1995, 2000, 2013 American Mathematical Society.
% Copyright (C) 2016-2021 LaTeX Project and American Mathematical Society.
%
% This work may be distributed and/or modified under the
% conditions of the LaTeX Project Public License, either version 1.3c
% of this license or (at your option) any later version.
% The latest version of this license is in
%   https://www.latex-project.org/lppl.txt
% and version 1.3c or later is part of all distributions of LaTeX
% version 2005/12/01 or later.
%
% This work has the LPPL maintenance status `maintained'.
%
% The Current Maintainer of this work is the LaTeX Project.
%
% \fi
%
%\iffalse
%<*driver>
\documentclass{amsdtx}
\def\MaintainedByLaTeXTeam#1{%
\begin{center}%
\fbox{\fbox{\begin{tabular}{@{}l@{}}%
This file is maintained by the \LaTeX{} Project team.\\%
Bug reports can be opened (category \texttt{#1}) at\\%
\url{https://latex-project.org/bugs/}.\end{tabular}}}\end{center}}
\raggedbottom
\let\savedarg\arg
\usepackage{amsmath}
\let\arg\savedarg
\GetFileInfo{amsmath.sty}
\begin{document}
\title{The \pkg{amsmath} package}
\author{Frank Mittelbach\and Rainer Sch\"opf\and Michael Downes\and
  David M. Jones\and David Carlisle}
\date{Version \fileversion, \filedate}
\providecommand{\histnote}{}
\renewenvironment{histnote}{%
  \trivlist\item[\hspace{\labelsep}\bfseries Historical Note:]%
}{%
  \endtrivlist
}
\DocInput{amsmath.dtx}
\end{document}
%</driver>
%\fi
%
% \maketitle
% \MaintainedByLaTeXTeam{amslatex}
%
% \MakeShortVerb\|
%
% \section{Introduction}
%
%    A \latex/ package named \pkg{amstex} was created in 1988--1989 by
%    adapting \fn{amstex.tex} for use within \latex/. The \pkg{amsmath}
%    package is the successor of the \pkg{amstex} package. It was
%    substantially overhauled to integrate it with \latex/2e, which
%    arrived on the scene in 1994. It provides more or less the same
%    features, but there are quite a few organizational differences as
%    well as some new features and bug fixes. For example, the
%    \pkg{amstex} package automatically loaded the \pkg{amsfonts}
%    package, but the \pkg{amsmath} package does not. At the present
%    time (November 1999) user-level documentation of the commands
%    provided here is found in the AMSmath Users' Guide,
%    \fn{amsldoc.tex}.
%
% \StopEventually{}
%
%    Standard file identification.
%    \begin{macrocode}
\NeedsTeXFormat{LaTeX2e}% LaTeX 2.09 can't be used (nor non-LaTeX)
[1994/12/01]% LaTeX date must be December 1994 or later
%    \end{macrocode}
%
%    Providing a rollback to earlier version(s)
%    \begin{macrocode}
\providecommand\DeclareRelease[3]{}
\providecommand\DeclareCurrentRelease[2]{}
%
\DeclareRelease{}{2018-12-01}{amsmath-2018-12-01.sty}
\DeclareCurrentRelease{}{2019-04-01}
%    \end{macrocode}
%
%    \begin{macrocode}
\ProvidesPackage{amsmath}[2020/09/23 v2.17i AMS math features]
%    \end{macrocode}
%
% \section{Catcode defenses}
%
%    Some packages change the catcode of characters that are essential
%    in low-level \tex/ syntax. Any package that does so does not
%    qualify as a PWWO package (\qq{Plays Well With Others}) because it
%    can cause other packages to fail if they are loaded later. \LaTeX{}
%    is partly to blame for this because it fails to provide adequate
%    built-in safeguards in the package loading mechanisms. In the
%    absence of such safeguards, we will provide them here.
%    \begin{macrocode}
\edef\@temp{\catcode 96=\number\catcode 96 }
\catcode\string `\`=12
\def\do#1{\catcode\number`#1=\number\catcode`#1}
\edef\@temp{%
  \noexpand\AtEndOfPackage{%
    \@temp
    \do\"\do\'\do\(\do\)\do\*\do\+\do\,\do\-\do\.%
    \do\/\do\<\do\=\do\>\do\[\do\]\do\^\do\_\relax
  }%
}
\@temp
\def\do#1{\catcode\number`#1=12 }
\do\"\do\'\do\(\do\)\do\*\do\+\do\,\do\-\do\.
\do\/\do\<\do\=\do\>\do\[\do\]
\catcode`\^=7 \catcode`\_=8
%    \end{macrocode}
%
% %%%%%%%%%%%%%%%%%%%%%%%%%%%%%%%%%%%%%%%%%%%%%%%%%%%%%%%%%%%%%%%%%%%%%%
% \section{Declare some options}
%
%    Handling of limits on integrals, sums, operatornames.
%    \begin{macrocode}
\DeclareOption{intlimits}{\let\ilimits@\displaylimits}
\DeclareOption{nointlimits}{\let\ilimits@\nolimits}
\DeclareOption{sumlimits}{\let\slimits@\displaylimits}
\DeclareOption{nosumlimits}{\let\slimits@\nolimits}
\DeclareOption{namelimits}{\PassOptionsToPackage{namelimits}{amsopn}}
\DeclareOption{nonamelimits}{%
  \PassOptionsToPackage{nonamelimits}{amsopn}}
%    \end{macrocode}
%
%    The following two switches might have been defined already by the
%    documentclass, but it doesn't hurt to re-execute the \cs{newif}'s.
%    \begin{macrocode}
\newif\ifctagsplit@
\newif\iftagsleft@
%    \end{macrocode}
%    Right or left placement of equation numbers.
%    \begin{macrocode}
\DeclareOption{leqno}{\tagsleft@true}
\DeclareOption{reqno}{\tagsleft@false}
\DeclareOption{centertags}{\ctagsplit@true}
\DeclareOption{tbtags}{\ctagsplit@false}
%    \end{macrocode}
%
%    The \opt{cmex10} option is an escape hatch for people who don't
%    happen to have sizes 7--9 of the \fn{cmex} fonts available to them
%    yet. (Strictly speaking they are considered part of a minimum
%    \latex/ distribution now, i.e., all \LaTeXe{} users should have
%    them, without needing to get the AMSFonts distrib.)
%    \begin{macrocode}
\DeclareOption{cmex10}{%
  \ifnum\cmex@opt=\@ne \def\cmex@opt{0}%
  \else \def\cmex@opt{10}\fi
}
%    \end{macrocode}
%    To help things work out better with various package loading orders
%    of \pkg{amsmath} and \pkg{amsfonts}, we establish a variable to
%    communicate the status of the cmex font definition. If the
%    \pkg{amsfonts} package was loaded first this variable might be
%    already defined, in which case we want to preserve its value.
%    \begin{macrocode}
\@ifundefined{cmex@opt}{\def\cmex@opt{7}}{}
%    \end{macrocode}
%
%%%%%%%%%%%%%%%%%%%%%%%%%%%%%%%%%%%%%%%%%%%%%%%%%%%%%%%%%%%%%%%%%%%%%%%%
% \section{Flush-left equations [DMJ]}
%
%    The left margin of math enviroments is controlled by
%    \cs{@mathmargin}.  This can be set to \cs{@centering} to
%    implement the default behaviour, i.e., centered equations, and to
%    something else to implement the flushleft style.
%
%    In theory, all that's needed to activate the flushleft mode
%    in the AMS document classes is something like this:
% \begin{verbatim}
% \DeclareOption{fleqn}{%
%   \AtBeginDocument{\@mathmargin30pt\relax}%
% }
% \end{verbatim}
%    (In fact, unless the document class wants to specify the
%    \cs{@mathmargin}, it doesn't need to do anything with the
%    \opt{fleqn} option.)
%    \begin{macrocode}
\newif\if@fleqn
%
\newskip\@mathmargin
\@mathmargin\@centering
%
\DeclareOption{fleqn}{%
    \@fleqntrue
    \@mathmargin = -1sp
    \let\mathindent=\@mathmargin
    \AtBeginDocument{%
        \ifdim\@mathmargin= -1sp
            \@mathmargin\leftmargini minus\leftmargini
        \fi
    }%
}
%    \end{macrocode}
%
%    DMJ: This ensures that \cs{@mathmargin} is given some sort of
%    sensible default if the class doesn't specify one, while still
%    allowing a user to override the default value in their document
%    preamble.  (Incidentally, I'm initializing \cs{@mathmargin} to
%    \cs{leftmargini} for compatibility with \fn{fleqn.clo}, but I'm
%    not at all convinced that's the right thing to do.)
%
%    The next question is what happens when amsmath is used with
%    one of the standard classes.  Unfortunately, \latex/ implements
%    \opt{fleqn} somewhat clumsily; instead of paramaterizing the
%    definitions of the math structures (as I've attempted to do
%    here), \fn{fleqn.clo} declares a dimen \cn{mathindent} that is
%    much like my \cs{@mathmargin} and then redefines \cn\[, \cn\],
%    \cn{equation}, and \cn{eqnarray}.  This means that things could
%    get rather messy in 2.09 compatibility mode, since \fn{fleqn.clo}
%    might be loaded after \fn{amsmath.sty}, which could cause a real
%    mess.
%
%    [mjd,1999/07/07]: Let \cs{mathindent} = \cs{@mathmargin} as
%    envisioned by DMJ. Compatibility-mode documents will all use the
%    \pkg{amstex} package, not \pkg{amsmath}. There is a remote chance
%    of a problem if someone makes an assignment to \cs{mathindent} in a
%    way that implicitly assumes it is a dimen register (inasmuch as it
%    has now become a skip register), and the string ``plus'' follows in
%    the input stream, but if someone's document croaks in that way, I
%    think they will just have to bite the bullet and fix it. The
%    alternative is to penalize a lot of other users with a known
%    handicap.
%
% \section{Spacing around \cn{aligned} and \cn{gathered}}
%
%    [dpc, 2016] Option to control the space to the left of aligned and gathered.
%
%   Previously \cn{aligned} and \cn{gathered} inserted a thin space on
%   their left but not their right, there is no good reason for this
%   that anyone can remember, it has just always been that way
%   inherited from amstex. The usual advice to authors has been to use
%   |\!\begin{aligned}| to get better spacing.
%
% Here introduce:
%
% \opt{alignedleftspaceyes} to keep the behaviour of adding this space.
%
% \opt{alignedleftspaceno} to disable adding this space.
%
% \opt{alignedleftspaceyesifneg} the new default behaviour, do not add
% the space unless the environment is preceded by a negative skip or
% kern, so that |\!\begin{aligned}| works as before.
%
% \changes{v2.16a}{2016/11/05}{New options to control aligned spacing}
%    \begin{macrocode}
\DeclareOption{alignedleftspaceyes}{\def\alignedspace@left{\null\,}}
\DeclareOption{alignedleftspaceno}{\def\alignedspace@left{\null}}
\DeclareOption{alignedleftspaceyesifneg}{%
\def\alignedspace@left{%
  \edef\@tempa{\expandafter\@car\the\lastskip\@nil}%
  \if-\@tempa\null\,%
  \else
    \edef\@tempa{\expandafter\@car\the\lastkern\@nil}%
    \if-\@tempa\null\,%
    \else\null
    \fi
  \fi}%
}
%    \end{macrocode}
%
%
%    \begin{macrocode}
\DeclareOption{?}{}
%    \end{macrocode}
%
%    \begin{macrocode}
\ExecuteOptions{%
  nointlimits,sumlimits,namelimits,centertags,alignedleftspaceyesifneg}
%    \end{macrocode}
%    The \cs{par} after \cs{ProcessOptions} is to ensure the correct
%    line number on screen if an error occurs during option processing;
%    otherwise the lookahead for a \qc{\*} option would result in \tex/
%    showing the following line instead.
%    \begin{macrocode}
\ProcessOptions\par
%    \end{macrocode}
%
%    \begin{macrocode}
\@ifpackagewith{amsmath}{?}{%
  \typeout{^^J%
Documentation for the amsmath package is found in amsldoc.dvi^^J%
(or .pdf or .tex).^^J%
^^J%
See also https://www.ams.org/tex/amslatex.html.^^J%
^^J%
Note: Using the first edition of The LaTeX Companion (1994) without^^J%
errata as a guide for amsmath use is not recommended.^^J%
  }%
}{%
  \typeout{%
For additional information on amsmath, use the \lq ?\rq\space option.%
  }%
}
%    \end{macrocode}
%
%    Processing to handle the \opt{cmex10} option is a little tricky
%    because of different possible loading orders for \pkg{amsmath}
%    and \pkg{amsfonts}. The package \pkg{amsmath} sets the
%    \cs{cmex@opt} flag to 0, 7 or 10, and in the past the package
%    \pkg{amsfonts} did set the flag to 1 or 0. These days it always
%    sets it to 10. The situation is a bit unsettled but we don't own
%    \pkg{amsfonts}.
% \changes{v2.17d}{2019/12/01}{docu update, no code change (gh/200)}
%    \begin{macrocode}
\ifnum\cmex@opt=7 \relax
  \DeclareFontShape{OMX}{cmex}{m}{n}{%
    <-8>cmex7<8>cmex8<9>cmex9%
    <10><10.95><12><14.4><17.28><20.74><24.88>cmex10%
  }{}%
  \expandafter\let\csname OMX/cmex/m/n/10\endcsname\relax
\else
  \ifnum\cmex@opt=\z@ % need to override cmex7 fontdef from amsfonts
%    \end{macrocode}
%    Force reloading of the OMX/cmex font definition file.
%    \begin{macrocode}
    \begingroup
    \fontencoding{OMX}\fontfamily{cmex}%
    \expandafter\let\csname OMX+cmex\endcsname\relax
    \try@load@fontshape
    \endgroup
%    \end{macrocode}
%    The \fn{cmex10} font gets special preload handling in the building
%    of the \latex/ format file, need an extra bit here to work around
%    that.
%    \begin{macrocode}
    \expandafter\let\csname OMX/cmex/m/n/10\endcsname\relax
    \def\cmex@opt{10}%
  \fi
\fi
%    \end{macrocode}
%
% %%%%%%%%%%%%%%%%%%%%%%%%%%%%%%%%%%%%%%%%%%%%%%%%%%%%%%%%%%%%%%%%%%%%%%
% \section{Call some other packages}
%
%    The \pkg{amstext} package provides the \cn{text} command. The
%    \pkg{amsbsy} package provides \cn{boldsymbol} and \cn{pmb}. (Since
%    1997 it is usually better to use the \pkg{bm} package instead; but
%    I think we have to keep \pkg{amsbsy} here for backward
%    compatibility [mjd,1999/11/19].) The \pkg{amsopn} package provides
%    \cn{DeclareMathOperator}.
%    \begin{macrocode}
\RequirePackage{amstext}[1995/01/25]
\RequirePackage{amsbsy}[1995/01/20]
\RequirePackage{amsopn}[1995/01/20]
%    \end{macrocode}
%
% %%%%%%%%%%%%%%%%%%%%%%%%%%%%%%%%%%%%%%%%%%%%%%%%%%%%%%%%%%%%%%%%%%%%%%
% \section{Miscellaneous}
%
%  \begin{macro}{\ams@newcommand}
%    Where \pkg{stix} and \pkg{amsmath} define the same control
%    sequences, we want to avoid inadvertently overriding \pkg{stix}'s
%    definitions.  If \pkg{stix} is loaded before \pkg{amsmath}, the
%    following conditional takes care of the problem.  There is
%    similar code in the \pkg{stix} package in case \pkg{amsmath} is
%    loaded first.
%    \begin{macrocode}
\@ifpackageloaded{stix}{%
    \let\ams@newcommand\providecommand
    \let\ams@renewcommand\providecommand
    \let\ams@def\providecommand
    \let\ams@DeclareRobustCommand\providecommand
}{%
    \let\ams@newcommand\newcommand
    \let\ams@renewcommand\renewcommand
    \let\ams@def\def
    \let\ams@DeclareRobustCommand\DeclareRobustCommand
}
%    \end{macrocode}
%  \end{macro}
%
%  \begin{macro}{\@amsmath@err}
%    Defining this error function saves main mem.
%    \begin{macrocode}
\def\@amsmath@err{\PackageError{amsmath}}
%    \end{macrocode}
%  \end{macro}
%
% \begin{macro}{\AmS}
%    The \cs{AmS} prefix can be used to construct the combination
%    |\AmS-\LaTeX|.
%    \begin{macrocode}
\providecommand{\AmS}{{\protect\AmSfont
  A\kern-.1667em\lower.5ex\hbox{M}\kern-.125emS}}
%    \end{macrocode}
%    In \cn{AmSfont} we call cmsy directly in lieu of trying to access
%    it through the math fonts setup (e.g. |\the\textfont2|) because
%    math fonts can't be relied on to be properly set up if we are not
%    inside a math formula. This means that if this command is used in a
%    document where CM fonts are not wanted, then a font substitution
%    will need to be declared, e.g.:
% \begin{verbatim}
% \DeclareFontShape{OMS}{cmsy}{m}{n}{ <-> sub * xxx/m/n }{}
% \end{verbatim}
%    where |xxx| is some alternate font family.
%    Taking the first letter of \cs{f@series} will produce |b| or |m|
%    for the most common values (|b,bx,m|). It may produce nonsense for
%    more unusual values of \cs{f@series}, so for safety's sake we have
%    an additional \cs{if} test. We want to avoid setting the series to
%    |bx| because in a standard \latex/ installation the combination
%    |cmsy/bx/n| does not have a font definition, and the user
%    would get a font substitution warning on screen.
%    \begin{macrocode}
\newcommand{\AmSfont}{%
  \usefont{OMS}{cmsy}{\if\@xp\@car\f@series\@nil bb\else m\fi}{n}}
%    \end{macrocode}
% \end{macro}
%
%  \begin{macro}{\@mathmeasure}
%    The function |\@mathmeasure| takes three arguments; the third arg
%    is typeset as a math formula in an hbox, using arg |#2| as the
%    mathstyle, and the result is left in the box named by the first
%    arg. It is assumed that we are already in math mode, so we can turn
%    off |\everymath| (in particular, |\check@mathfonts|).
%
%    As of 2018/12 release we don't turn off |\evermath| as this
%    optimization can be harmful.
% \changes{v2.17b}{2018/12/01}{Don't drop \cs{frozen@everymath}}
%    \begin{macrocode}
\ifx\leavevmode@ifvmode\@undefined  % kernel is < 2018/12
\def\@mathmeasure#1#2#3{\setbox#1\hbox{\frozen@everymath\@emptytoks
    \m@th$#2#3$}}
\else
\def\@mathmeasure#1#2#3{\setbox#1\hbox{%
  \m@th$#2#3$}}
\fi
%    \end{macrocode}
%  \end{macro}
%
%    The \cs{inf@bad} constant is for testing overfull boxes.
%    \begin{macrocode}
\@ifundefined{inf@bad}{%
  \newcount\inf@bad \inf@bad=1000000 \relax
}{}
%    \end{macrocode}
%
%\subsection{Math spacing commands}
%
%    \begin{macro}{\tmspace}
%    \begin{macro}{\,}
%    \begin{macro}{\thinspace}
%    \begin{macro}{\!}
%    \begin{macro}{\negthinspace}
%    \begin{macro}{\:}
%    \begin{macro}{\medspace}
%    \begin{macro}{\negmedspace}
%    \begin{macro}{\;}
%    \begin{macro}{\thickspace}
%    \begin{macro}{\negthickspace}
%    Here we fill in some gaps in the set of spacing commands, and make them
%    all work equally well in or out of math.
%    We want all these commands to be robust but declaring them all with
%    \cs{DeclareRobustCommand} uses up an control sequence name per
%    command; to avoid this, we define a common command \cs{tmspace}
%    (text-or-math space) which carries the robustness burden for all of
%    them. The standard \cs{relax} before the \cs{ifmmode} is not
%    necessary because of the \cs{protect} added by
%    \cs{DeclareRobustCommand}.
% \changes{v2.17b}{2018/12/01}{Start LR-mode for \cs{thinspace} and
%   friends if necessary (github/49)}
%
%    We start by undefining a number of commands (which in a current
%    \LaTeX{} kernel will be defined, so that the
%    \cs{DeclareRobustCommand} declarations below do not add a
%    ``Command redefined'' info into the log.
%    \begin{macrocode}
\let\tmspace\@undefined
\let\,\@undefined
\let\!\@undefined
\let\:\@undefined
\let\negmedspace\@undefined
\let\negthickspace\@undefined
%    \end{macrocode}
%
%
% \changes{v2.17g}{2020/03/07}{Math/text spacing commands are now in
%    the \LaTeX{} kernel and are made robust (gh/303)}
%    \begin{macrocode}
\ifx\leavevmode@ifvmode\@undefined
\DeclareRobustCommand\tmspace[3]{%
  \ifmmode\mskip#1#2\else\kern#1#3\fi\relax}
\else
\DeclareRobustCommand\tmspace[3]{%
  \ifmmode\mskip#1#2\else\leavevmode@ifvmode\kern#1#3\fi\relax}
\fi
\DeclareRobustCommand\,{\tmspace+\thinmuskip{.1667em}}
\let\thinspace\,
\DeclareRobustCommand\!{\tmspace-\thinmuskip{.1667em}}
\let\negthinspace\!
\DeclareRobustCommand\:{\tmspace+\medmuskip{.2222em}}
\let\medspace\:
\DeclareRobustCommand\negmedspace{\tmspace-\medmuskip{.2222em}}
\renewcommand\;{\tmspace+\thickmuskip{.2777em}}
\let\thickspace\;
\DeclareRobustCommand\negthickspace{\tmspace-\thickmuskip{.2777em}}
%    \end{macrocode}
%    \end{macro}
%    \end{macro}
%    \end{macro}
%    \end{macro}
%    \end{macro}
%    \end{macro}
%    \end{macro}
%    \end{macro}
%    \end{macro}
%    \end{macro}
%    \end{macro}
%
%  \begin{macro}{\mspace}
%    And while we're at it, why don't we provide an equivalent of
%    \cn{hspace} for math mode use. This allows use of |mu| units in
%    (for example) constructing compound math symbols.
%    \begin{macrocode}
\newcommand{\mspace}[1]{\mskip#1\relax}
%    \end{macrocode}
%  \end{macro}
%
% \subsection{Vertical bar symbols}
%    \begin{macro}{\lvert}
%    \begin{macro}{\rvert}
%    \begin{macro}{\lVert}
%    \begin{macro}{\rVert}
%    Add left/right specific versions of \cn{vert}, \cn{Vert}. Don't
%    assume the delimiter codes are the CM defaults.
%    \begin{macrocode}
\def\@tempa#1#2\@nil{%
  \ifx\delimiter#1\@tempcnta#2\relax\else\@tempcnta\z@\fi
}
\@xp\@tempa\vert\@empty\@nil
\ifnum\@tempcnta>\z@
  \advance\@tempcnta "4000000
%    \end{macrocode}
%    Use \cs{protected} on the new delimiters.
% \changes{v2.17e}{2020/01/20}{Make delimiters robust (gh/251))}
%    \begin{macrocode}
  \protected\xdef\lvert{\delimiter\number\@tempcnta\space }
  \advance\@tempcnta "1000000
  \protected\xdef\rvert{\delimiter\number\@tempcnta\space }
\else
  \ifx\@@undefined\lvert
    % Fall back to cmex encoding since we don't know what else to do.
    \DeclareMathDelimiter{\lvert}
      {\mathopen}{symbols}{"6A}{largesymbols}{"0C}
    \DeclareMathDelimiter{\rvert}
      {\mathclose}{symbols}{"6A}{largesymbols}{"0C}
  \fi
\fi
\@xp\@tempa\Vert\@empty\@nil
\ifnum\@tempcnta>\z@
  \advance\@tempcnta "4000000
  \protected\xdef\lVert{\delimiter\number\@tempcnta\space }
  \advance\@tempcnta "1000000
  \protected\xdef\rVert{\delimiter\number\@tempcnta\space }
\else
  \ifx\@@undefined\lVert
    \DeclareMathDelimiter{\lVert}
      {\mathopen}{symbols}{"6B}{largesymbols}{"0D}
    \DeclareMathDelimiter{\rVert}
      {\mathclose}{symbols}{"6B}{largesymbols}{"0D}
  \fi
\fi
%    \end{macrocode}
%    \end{macro}
%    \end{macro}
%    \end{macro}
%    \end{macro}
%
% \subsection{Fractions}
%    Bury the generalized fraction primitives \cs{over}, \cs{atop},
%    etc., because of their bizarre syntax, which is decidedly out of
%    place in a \latex/ document.
%    \begin{macrocode}
\@saveprimitive\over\@@over
\@saveprimitive\atop\@@atop
\@saveprimitive\above\@@above
\@saveprimitive\overwithdelims\@@overwithdelims
\@saveprimitive\atopwithdelims\@@atopwithdelims
\@saveprimitive\abovewithdelims\@@abovewithdelims
%    \end{macrocode}
%
%    \begin{macro}{\primfrac}
%    If someone insists on using \cs{over}, give a warning the first
%    time and then resurrect the old definition. Laissez-faire policy.
%    \begin{macrocode}
\DeclareRobustCommand{\primfrac}[1]{%
  \PackageWarning{amsmath}{%
Foreign command \@backslashchar#1;\MessageBreak
\protect\frac\space or \protect\genfrac\space should be used instead%
\MessageBreak
  }
  \global\@xp\let\csname#1\@xp\endcsname\csname @@#1\endcsname
  \csname#1\endcsname
}
%    \end{macrocode}
%    \end{macro}
%    \begin{macrocode}
\renewcommand{\over}{\primfrac{over}}
\renewcommand{\atop}{\primfrac{atop}}
\renewcommand{\above}{\primfrac{above}}
\renewcommand{\overwithdelims}{\primfrac{overwithdelims}}
\renewcommand{\atopwithdelims}{\primfrac{atopwithdelims}}
\renewcommand{\abovewithdelims}{\primfrac{abovewithdelims}}
%    \end{macrocode}
%
%    \cn{frac} calls \cn{@@over} directly instead of via \cn{genfrac}, for
%    better speed because it is so common. \cn{tfrac} and \cn{dfrac} are
%    abbreviations for some commonly needed mathstyle overrides. To
%    conserve csnames we avoid making \cn{dfrac} and \cn{tfrac} robust
%    (\cn{genfrac} is itself robust).
%    \changes{v2.17i}{2020/09/23}{added \cs{Ustack} for luatex (moved patch from lualatex-math)}
%    \begin{macrocode}
%
\ifx\directlua\@undefined
\DeclareRobustCommand{\frac}[2]{{\begingroup#1\endgroup\@@over#2}}
\else
\DeclareRobustCommand{\frac}[2]{{\Ustack{\begingroup#1\endgroup\@@over#2}}}
\fi
\newcommand{\dfrac}{\genfrac{}{}{}0}
\newcommand{\tfrac}{\genfrac{}{}{}1}
%    \end{macrocode}
%    The \cn{binom} command for binomial notation works like \cn{frac}
%    and has similar variants. Note that we do not use \cs{z@} in
%    \cn{dbinom} and \cn{tbinom} because they are not top-level robust
%    like \cn{binom}, and so the \cs{z@} with the potentially
%    problematic \qc{\@} character would become visible when writing one
%    of those commands to a \fn{.toc} file.
%    \begin{macrocode}
\DeclareRobustCommand{\binom}{\genfrac()\z@{}}
\newcommand{\dbinom}{\genfrac(){0pt}0}
\newcommand{\tbinom}{\genfrac(){0pt}1}
%    \end{macrocode}
%
%  \begin{macro}{\genfrac}
% \changes{v2.16a}{2016/11/05}{New genfrac implementation for extended
% TeXs}
%    This command provides access to \tex/'s generalized fraction
%    primitives. Args: \arg{1} left delim, \arg{2} right delim, \arg{3}
%    line thickness, \arg{4} mathstyle override, \arg{5} numerator,
%    \arg{6} denominator. But we only read the first four args at first,
%    in order to give us a moment to select the proper generalized
%    fraction primitive. Any of those four args could be empty, and when
%    empty the obvious defaults are selected (no delimiters, default
%    line thickness (normally .4pt), and no mathstyle override).
%
%    the |withdelims| primitives do not work in xetex with OpenType
%    fonts, and the relevant font dimen parameters are often not set
%    in luatex as theer are no matching values in the OpenType Math
%    table, so here we use variants that use the font parameters if
%    they are set, but scale using |\left\right| rather than the
%    |withdelims| primitives.
%
%    \begin{macrocode}
\ifx\directlua\@undefined
%    \end{macrocode}
%
%    \begin{macrocode}
\ifx\XeTeXcharclass\@undefined
%    \end{macrocode}
% Classic version
%    \begin{macrocode}
\DeclareRobustCommand{\genfrac}[4]{%
  \def\@tempa{#1#2}%
  \edef\@tempb{\@nx\@genfrac\@mathstyle{#4}%
    \csname @@\ifx @#3@over\else above\fi
    \ifx\@tempa\@empty \else withdelims\fi\endcsname}
  \@tempb{#1#2#3}}
%    \end{macrocode}
%
%    \begin{macrocode}
\else
%    \end{macrocode}
%     XeTeX version
%    \begin{macrocode}
\def\genfrac@rule#1#2#3#4{%
\hbox{$\left#1\vcenter{\hrule \@width\z@
                       \@height
                       \ifdim\fontdimen#2#3\tw@=\z@
                       #4\fontdimen6#3\tw@
                       \else
                       \fontdimen#2#3\tw@
                       \fi
                      }\right.$}}
%    \end{macrocode}
%
%    \begin{macrocode}
\def\genfrac@choice#1#2{%
\ifx @#2@\else
%    \end{macrocode}
% \changes{v2.17a}{2017/09/02}{move \cs{nulldelimiterspace} correction}
%    \begin{macrocode}
\ifx c#1\kern-\nulldelimiterspace\fi
{\delimitershortfall\z@\delimiterfactor\@m
 \mathsurround\z@\nulldelimiterspace\z@
\mathchoice
{\genfrac@rule{#2}{20}\textfont{2.39}}%
{\genfrac@rule{#2}{21}\textfont{1}}%
{\genfrac@rule{#2}{21}\scriptfont{1.45}}%
{\genfrac@rule{#2}{21}\scriptscriptfont{1.35}}%
}%
\ifx o#1\kern-\nulldelimiterspace\fi
\fi
}
%    \end{macrocode}
%
%    \begin{macrocode}
\DeclareRobustCommand{\genfrac}[6]{{%
\@mathstyle{#4}%
\genfrac@choice o{#1}%
{\begingroup#5\endgroup\ifx @#3@\@@over\else\@@above\fi#3\relax#6}%
\genfrac@choice c{#2}%
}}
%    \end{macrocode}
%
%    \begin{macrocode}
\fi
\else
%    \end{macrocode}
% LuaTeX version
%    \begin{macrocode}
\def\genfrac@rule#1#2#3{%
\hbox{$\left#1\vcenter{\hrule \@width\z@
                       \@height
                       \ifdim\Umathfractiondelsize#2=\z@
                       #3\fontdimen6#3\tw@
                       \else
                       \Umathfractiondelsize#2%
                       \fi
                      }\right.$}}
%    \end{macrocode}
%
%    \begin{macrocode}
\def\genfrac@choice#1#2{%
\ifx @#2@\else
%    \end{macrocode}
% \changes{v2.17a}{2017/09/02}{move \cs{nulldelimiterspace} correction}
%    \begin{macrocode}
\ifx c#1\kern-\nulldelimiterspace\fi
{\delimitershortfall\z@\delimiterfactor\@m
 \mathsurround\z@\nulldelimiterspace\z@
\mathchoice
{\genfrac@rule{#2}\displaystyle{2.39}}%
{\genfrac@rule{#2}\textstyle{1}}%
{\genfrac@rule{#2}\scriptstyle{1.45}}%
{\genfrac@rule{#2}\scriptscriptstyle{1.35}}%
}%
\ifx o#1\kern-\nulldelimiterspace\fi
\fi
}
%    \end{macrocode}
% \changes{v2.17i}{2020/09/23}{added \cs{Ustack} (moved patch from lualatex-math)}
%    \begin{macrocode}
\DeclareRobustCommand{\genfrac}[6]{{%
\@mathstyle{#4}%
\genfrac@choice o{#1}%
{\Ustack {\begingroup#5\endgroup\ifx @#3@\@@over\else\@@above\fi#3\relax#6}}%
\genfrac@choice c{#2}%
}}
%    \end{macrocode}
%
%    \begin{macrocode}
\fi
%    \end{macrocode}
%
% End of test for Lua\TeX/Xe\TeX.
%
%    \cs{@genfrac} takes the preceding arguments and reads the
%    numerator and denominator. Note that there's no convenient way to
%    make the numerator and denominator \emph{contents}
%    displaystyle, through this interface.
%
%    Args: \arg{1} mathstyle, \arg{2} fraction primitive,
%    \arg{3} delimiters and rule thickness,
%    \arg{4} numerator, \arg{5} denominator.
%    \begin{macrocode}
\def\@genfrac#1#2#3#4#5{{#1{\begingroup#4\endgroup#2#3\relax#5}}}
%    \end{macrocode}
%  \end{macro}
%
%    Empty mathstyle arg: no change; 0 = displaystyle, 1 = textstyle, 2
%    = scriptstyle, 3 = scriptscriptstyle.
%    \begin{macrocode}
\def\@mathstyle#1{%
  \ifx\@empty#1\@empty\relax
  \else\ifcase#1\displaystyle % case 0
    \or\textstyle\or\scriptstyle\else\scriptscriptstyle\fi\fi}
%    \end{macrocode}
%
% \subsection{Sums and Integrals}
%    Default value for sum limits is \cs{displaylimits}, see option
%    `nosumlimits'.
%
%    We redefine all the cumulative operator symbols to use
%    \cs{slimits@} so that switching between \cs{displaylimits} and
%    \cs{nolimits} can be controlled by package options. Also add
%    \cs{DOTSB} for the benefit of the dots lookahead. But we'd better
%    make sure \cn{coprod} and the others are simple mathchars; if not,
%    the attempted changes will probably fail miserably.
%
% \changes{v2.15}{2016/02/20}{Accept \cs{Umathchar}}
%    \begin{macrocode}
\begingroup
%    \end{macrocode}
%    \begin{macrocode}
\edef\@tempa{\string\mathchar"}
\edef\@tempd{\string\Umathchar"}
\def\@tempb#1"#2\@nil{#1"}
\edef\@tempc{\expandafter\@tempb\meaning\coprod "\@nil}
\ifx\@tempc\@tempd\let\@tempc\@tempa\fi
%    \end{macrocode}
%    \begin{macrocode}
\ifx\@tempa\@tempc
  \global\let\coprod@\coprod
  \gdef\coprod{\DOTSB\coprod@\slimits@}
  \global\let\bigvee@\bigvee
  \gdef\bigvee{\DOTSB\bigvee@\slimits@}
  \global\let\bigwedge@\bigwedge
  \gdef\bigwedge{\DOTSB\bigwedge@\slimits@}
  \global\let\biguplus@\biguplus
  \gdef\biguplus{\DOTSB\biguplus@\slimits@}
  \global\let\bigcap@\bigcap
  \gdef\bigcap{\DOTSB\bigcap@\slimits@}
  \global\let\bigcup@\bigcup
  \gdef\bigcup{\DOTSB\bigcup@\slimits@}
  \global\let\prod@\prod
  \gdef\prod{\DOTSB\prod@\slimits@}
  \global\let\sum@\sum
  \gdef\sum{\DOTSB\sum@\slimits@}
  \global\let\bigotimes@\bigotimes
  \gdef\bigotimes{\DOTSB\bigotimes@\slimits@}
  \global\let\bigoplus@\bigoplus
  \gdef\bigoplus{\DOTSB\bigoplus@\slimits@}
  \global\let\bigodot@\bigodot
  \gdef\bigodot{\DOTSB\bigodot@\slimits@}
  \global\let\bigsqcup@\bigsqcup
  \gdef\bigsqcup{\DOTSB\bigsqcup@\slimits@}
\fi
\endgroup
%    \end{macrocode}
%
% \subsection{Roots and radicals}
%
%    \begin{macro}{\root}
%    This root stuff needs syntax work and implementation work. Surely
%    something more compact can be done?? [mjd, 1994/09/05]
%    \begin{macrocode}
\newcommand{\leftroot}{\@amsmath@err{\Invalid@@\leftroot}\@eha}
\newcommand{\uproot}{\@amsmath@err{\Invalid@@\uproot}\@eha}
\newcount\uproot@
\newcount\leftroot@
\renewcommand{\root}{\relaxnext@
  \DN@{\ifx\@let@token\uproot\let\next@\nextii@\else
   \ifx\@let@token\leftroot\let\next@\nextiii@\else
   \let\next@\plainroot@\fi\fi\next@}%
  \def\nextii@\uproot##1{\uproot@##1\relax\FN@\nextiv@}%
  \def\nextiv@{\ifx\@let@token\@sptoken\DN@. {\FN@\nextv@}\else
   \DN@.{\FN@\nextv@}\fi\next@.}%
  \def\nextv@{\ifx\@let@token\leftroot\let\next@\nextvi@\else
   \let\next@\plainroot@\fi\next@}%
  \def\nextvi@\leftroot##1{\leftroot@##1\relax\plainroot@}%
   \def\nextiii@\leftroot##1{\leftroot@##1\relax\FN@\nextvii@}%
  \def\nextvii@{\ifx\@let@token\@sptoken
   \DN@. {\FN@\nextviii@}\else
   \DN@.{\FN@\nextviii@}\fi\next@.}%
  \def\nextviii@{\ifx\@let@token\uproot\let\next@\nextix@\else
   \let\next@\plainroot@\fi\next@}%
  \def\nextix@\uproot##1{\uproot@##1\relax\plainroot@}%
  \bgroup\uproot@\z@\leftroot@\z@\FN@\next@}
\def\plainroot@#1\of#2{\setbox\rootbox\hbox{%
 $\m@th\scriptscriptstyle{#1}$}%
 \mathchoice{\r@@t\displaystyle{#2}}{\r@@t\textstyle{#2}}
 {\r@@t\scriptstyle{#2}}{\r@@t\scriptscriptstyle{#2}}\egroup}
%    \end{macrocode}
%    \end{macro}
%
% \changes{v2.0}{1999/06/17}{Normalize @@sqrt to sqrtsign}

%    Name change from \cs{@@sqrt} to \cs{sqrtsign} happened in the
%    1995/12/01 release of \latex/. If we were to assume that
%    \cs{sqrtsign} is defined then someone with the 1995/06/01 release
%    of \latex/ would have trouble using this package.
%    \begin{macrocode}
\@ifundefined{sqrtsign}{\let\sqrtsign\@@sqrt}{}
\def\r@@t#1#2{\setboxz@h{$\m@th#1\sqrtsign{#2}$}%
 \dimen@\ht\z@\advance\dimen@-\dp\z@
 \setbox\@ne\hbox{$\m@th#1\mskip\uproot@ mu$}%
 \advance\dimen@ by1.667\wd\@ne
 \mkern-\leftroot@ mu\mkern5mu\raise.6\dimen@\copy\rootbox
 \mkern-10mu\mkern\leftroot@ mu\boxz@}
%    \end{macrocode}
%
% \subsection{Et cetera}
%
% \changes{v2.0}{1999/06/18}{Leave Greek cap letters unaltered}
%
%    Specific names for the variant italic cap Greek letters are not
%    defined by \latex/. If no preceding package defined these, we will
%    define them now.
%    \begin{macrocode}
\@ifundefined{varGamma}{%
  \DeclareMathSymbol{\varGamma}{\mathord}{letters}{"00}
  \DeclareMathSymbol{\varDelta}{\mathord}{letters}{"01}
  \DeclareMathSymbol{\varTheta}{\mathord}{letters}{"02}
  \DeclareMathSymbol{\varLambda}{\mathord}{letters}{"03}
  \DeclareMathSymbol{\varXi}{\mathord}{letters}{"04}
  \DeclareMathSymbol{\varPi}{\mathord}{letters}{"05}
  \DeclareMathSymbol{\varSigma}{\mathord}{letters}{"06}
  \DeclareMathSymbol{\varUpsilon}{\mathord}{letters}{"07}
  \DeclareMathSymbol{\varPhi}{\mathord}{letters}{"08}
  \DeclareMathSymbol{\varPsi}{\mathord}{letters}{"09}
  \DeclareMathSymbol{\varOmega}{\mathord}{letters}{"0A}
}{}
%    \end{macrocode}
%
%    \begin{macro}{\overline}
%    \amstex/ redefines \cn{overline} as shown here, for reasons that
%    are probably less important in \latex/: Make it read its argument
%    as a macro argument rather than a ``math field'' (\emph{The
%    \tex/book}, Chapter 26), to avoid problems when something that is
%    apparently a single symbol is actually a non-simple macro (e.g.,
%    \cn{dag}) \emph{and} is given as a single-token argument without
%    enclosing braces.
%    \begin{macrocode}
\@saveprimitive\overline\@@overline
\DeclareRobustCommand{\overline}[1]{\@@overline{#1}}
%    \end{macrocode}
%    \end{macro}
%
%    \begin{macro}{\boxed}
%    The \cs{boxed} command is specifically intended to put a box around
%    an equation or piece of an equation. (Not including the equation
%    number.) This isn't trivial for end-users to do it properly
%    with \cs{fbox} so we provide a command for them.
%    \begin{macrocode}
\newcommand{\boxed}[1]{\fbox{\m@th$\displaystyle#1$}}
%    \end{macrocode}
%    \end{macro}
%
%    \begin{macro}{\implies}
%    \begin{macro}{\impliedby}
%    \begin{macrocode}
\newcommand{\implies}{\DOTSB\;\Longrightarrow\;}
\newcommand{\impliedby}{\DOTSB\;\Longleftarrow\;}
%    \end{macrocode}
%    \end{macro}
%    \end{macro}
%
%    \begin{macro}{\And}
%    \begin{macrocode}
\def\And{\DOTSB\;\mathchar"3026 \;}
%    \end{macrocode}
%    \end{macro}
%
%    \begin{macro}{\nobreakdash}
%    The command \cn{nobreakdash} is designed only for use before a
%    hyphen or dash (|-|, |--|, or |---|).
%    Setting the hyphen in a box and then unboxing it means that the
%    normal penalty will not be added after it---and if the penalty is
%    not there a break will not be taken (unless an explicit penalty
%    or glue follows, thus the final \verb=\nobreak=).
%    \begin{macrocode}
\newcommand{\nobreakdash}{\leavevmode
  \toks@\@emptytoks \def\@tempa##1{\toks@\@xp{\the\toks@-}\FN@\next@}%
  \DN@{\ifx\@let@token-\@xp\@tempa
       \else\setboxz@h{\the\toks@\nobreak}\unhbox\z@\fi}%
  \FN@\next@
}
%    \end{macrocode}
%    \end{macro}
%
%    \begin{macro}{\colon}
%    \cs{colon} is for a colon in math that resembles a text colon:
%    small space on the left, larger space on the right. The \qc{\:}
%    character by itself is treated as a \cs{mathrel} i.e. large, equal
%    spacing on both sides.
%    \begin{macrocode}
\renewcommand{\colon}{\nobreak\mskip2mu\mathpunct{}\nonscript
  \mkern-\thinmuskip{:}\mskip6muplus1mu\relax}
%    \end{macrocode}
%    \end{macro}
%
% %%%%%%%%%%%%%%%%%%%%%%%%%%%%%%%%%%%%%%%%%%%%%%%%%%%%%%%%%%%%%%%%%%%%%%
% \section{Ellipsis dots}
%
%    We can't use \cs{newif} for \cs{ifgtest@} because we want
%    to include \cs{global} in the definitions of
%    \cs{gtest@true} and \cs{gtest@false}.
%    \begin{macrocode}
\let\ifgtest@\iffalse                              % initial value
\def\gtest@true{\global\let\ifgtest@\iftrue}
\def\gtest@false{\global\let\ifgtest@\iffalse}
\let\DOTSI\relax
\let\DOTSB\relax
\let\DOTSX\relax
{\uccode`7=`\\ \uccode`8=`m \uccode`9=`a \uccode`0=`t \uccode`!=`h
 \uppercase{%
  \gdef\math@#1#2#3#4#5#6\math@{\gtest@false\ifx 7#1\ifx 8#2%
  \ifx 9#3\ifx 0#4\ifx !#5\xdef\meaning@{#6}\gtest@true
  \fi\fi\fi\fi\fi}}}
{\uccode`7=`c \uccode`8=`h \uccode`9=`\"
 \uppercase{\gdef\mathch@#1#2#3#4#5#6\mathch@{\gtest@false
  \ifx 7#1\ifx 8#2\ifx 9#5\gtest@true\xdef\meaning@{9#6}\fi\fi\fi}}}
%    \end{macrocode}
% \changes{v2.15}{2016/02/20}{Accept \cs{Umathchar}}
%    \begin{macrocode}
{\uccode`(=`U \uccode`)=`m
 \uppercase{\gdef\Umathch@#1#2#3#4"#5"#6\Umathch@{\gtest@false
  \ifx(#2\ifx)#3\gtest@true
  \ifcase"#5 \or\or\gdef\thedots@{\dotsb@}\or\gdef\thedots@{\dotsb@}\fi
  \fi\fi
  }}}
%    \end{macrocode}
% \changes{v2.15}{2016/02/20}{Decode the mathcode of character tokens for Unicode TeX}
% For Unicode TeXs, if the next token is a character token look up the (U)mathcode. Do not do this for classic TeX for compatibility reasons.
%    \begin{macrocode}
\ifx\Umathcharnumdef\@undefined
\gdef\thecharacter@#1\thecharacter@{}
\else
{\uccode`(=`t \uccode`)=`c
 \uppercase{\gdef\thecharacter@#1#2#3#4#5\thecharacter@{%
  \ifx(#1\ifx)#4%
    \@xp\getmathcode@\meaning@\getmathcode@
  \fi\fi
}}}
\def\getmathcode@#1 #2 #3#4\getmathcode@{%
  \Umathcharnumdef\@tempa\Umathcodenum`#3\relax
  \edef\meaning@{\meaning\@tempa}%
  \@xp\Umathch@\meaning@\Umathch@
}
\fi
%    \end{macrocode}
%    \begin{macrocode}
\newcount\classnum@
\def\getmathch@#1.#2\getmathch@{\classnum@#1 \divide\classnum@4096
 \ifcase\number\classnum@\or\or\gdef\thedots@{\dotsb@}\or
 \gdef\thedots@{\dotsb@}\fi}
{\uccode`4=`b \uccode`5=`i \uccode`6=`n
 \uppercase{\gdef\mathbin@#1#2#3{\relaxnext@
  \def\nextii@##1\mathbin@{\ifx\@sptoken\@let@token\gtest@true\fi}%
  \gtest@false\DN@##1\mathbin@{}%
 \ifx 4#1\ifx 5#2\ifx 6#3\DN@{\FN@\nextii@}\fi\fi\fi\next@}}}
{\uccode`4=`r \uccode`5=`e \uccode`6=`l
 \uppercase{\gdef\mathrel@#1#2#3{\relaxnext@
  \def\nextii@##1\mathrel@{\ifx\@sptoken\@let@token\gtest@true\fi}%
 \gtest@false\DN@##1\mathrel@{}%
 \ifx 4#1\ifx 5#2\ifx 6#3\DN@{\FN@\nextii@}\fi\fi\fi\next@}}}
{\uccode`5=`m \uccode`6=`a \uccode`7=`c
 \uppercase{\gdef\macro@#1#2#3#4\macro@{\gtest@false
  \ifx 5#1\ifx 6#2\ifx 7#3\gtest@true
  \xdef\meaning@{\macro@@#4\macro@@}\fi\fi\fi}}}
\def\macro@@#1->#2\macro@@{#2}
\newcount\DOTSCASE@
{\uccode`6=`\\ \uccode`7=`D \uccode`8=`O \uccode`9=`T \uccode`0=`S
 \uppercase{\gdef\DOTS@#1#2#3#4#5{\gtest@false\DN@##1\DOTS@{}%
  \ifx 6#1\ifx 7#2\ifx 8#3\ifx 9#4\ifx 0#5\let\next@\DOTS@@
  \fi\fi\fi\fi\fi
  \next@}}}
{\uccode`3=`B \uccode`4=`I \uccode`5=`X
 \uppercase{\gdef\DOTS@@#1{\relaxnext@
  \def\nextii@##1\DOTS@{\ifx\@sptoken\@let@token\gtest@true\fi}%
  \DN@{\FN@\nextii@}%
  \ifx 3#1\global\DOTSCASE@\z@\else
  \ifx 4#1\global\DOTSCASE@\@ne\else
  \ifx 5#1\global\DOTSCASE@\tw@\else\DN@##1\DOTS@{}%
  \fi\fi\fi\next@}}}
{\uccode`5=`\\ \uccode`6=`n \uccode`7=`o \uccode`8=`t
 \uppercase{\gdef\not@#1#2#3#4{\relaxnext@
  \def\nextii@##1\not@{\ifx\@sptoken\@let@token\gtest@true\fi}%
 \gtest@false\DN@##1\not@{}%
 \ifx 5#1\ifx 6#2\ifx 7#3\ifx 8#4\DN@{\FN@\nextii@}\fi\fi\fi
 \fi\next@}}}
%    \end{macrocode}
% \changes{v2.15}{2016/02/20}{macro added to strip \cs{long} during tests}
%    \begin{macrocode}
{\uccode`9=`\l %
 \uppercase{\gdef\striplong@#1#2#3\relax{%
  \ifx9#2 \@xp\@xp\@xp\zap@to@space\fi}}}
\def\zap@to@space#1 {}
%    \end{macrocode}
%    \begin{macrocode}
\def\keybin@{\gtest@true
 \ifx\@let@token+\else\ifx\@let@token=\else
 \ifx\@let@token<\else\ifx\@let@token>\else
 \ifx\@let@token-\else\ifx\@let@token*\else\ifx\@let@token:\else
   \gtest@false\fi\fi\fi\fi\fi\fi\fi}
%    \end{macrocode}
%    Patch to ensure \cs{@ldots} is defined. (Name changed to
%    \cn{mathellipsis} in Dec 94 release of \latex/.)
%    \begin{macrocode}
\@ifundefined{@ldots}{\def\@ldots{\mathellipsis}}{}
%    \end{macrocode}
%    \begin{macro}{\ldots}
%    \begin{macro}{\dots}
%    Reiterate the standard definition of \cs{ldots} to keep it from
%    being clobbered by the redefinition of \cs{dots}.
%    \begin{macrocode}
\DeclareRobustCommand{\ldots}{%
  \ifmmode \mathellipsis \else \textellipsis \fi
}
\DeclareRobustCommand{\dots}{%
  \ifmmode \@xp\mdots@\else \@xp\textellipsis \fi
}
%    \end{macrocode}
%    \end{macro}
%    \end{macro}
%    \begin{macrocode}
\def\tdots@{\leavevmode\unskip\relaxnext@
 \DN@{$\m@th\@ldots\,
   \ifx\@let@token,\,$\else\ifx\@let@token.\,$\else
   \ifx\@let@token;\,$\else\ifx\@let@token:\,$\else
   \ifx\@let@token?\,$\else\ifx\@let@token!\,$\else
     $ \fi\fi\fi\fi\fi\fi}%
  \ \FN@\next@}
\def\mdots@{\FN@\mdots@@}
%    \end{macrocode}
% \changes{v2.15}{2016/02/20}{Indent \cs{mdots@@} for readability and add additional tests}
%    \begin{macrocode}
\def\mdots@@{\gdef\thedots@{\dotso@}%
 \ifx\@let@token\boldsymbol
   \gdef\thedots@\boldsymbol{\boldsymboldots@}%
 \else
   \ifx,\@let@token \gdef\thedots@{\dotsc}%
   \else
     \ifx\not\@let@token
       \gdef\thedots@{\dotsb@}%
     \else
       \keybin@
       \ifgtest@ % if \keybin@ test
         \gdef\thedots@{\dotsb@}%
       \else
%    \end{macrocode}
% \changes{v2.15d}{2016/06/28}{Add space token to prevent runaway argument error}
%    \begin{macrocode}
         \xdef\meaning@{\meaning\@let@token. .........}%
%    \end{macrocode}
% In previous versions \verb|\long| macros were not seen by the lokkahead.
% That was bad as this file uses \verb|\(re)newcommand| for \verb|\implies| etc.
%    \begin{macrocode}
         \xdef\meaning@@{\@xp\striplong@\meaning@\relax\meaning@}%
%    \end{macrocode}
%    \begin{macrocode}
         \@xp\math@\meaning@\math@
         \ifgtest@ % if \mathxxx test
           \@xp\mathch@\meaning@\mathch@
           \ifgtest@ % if \mathchar
             \@xp\getmathch@\meaning@\getmathch@
           \fi % end if \mathchar
         \else  % \not \mathxxx
%    \end{macrocode}
% Test for \verb|\Umathchar| added.
%    \begin{macrocode}
             \@xp\Umathch@\meaning@"0"\Umathch@
             \ifgtest@ % if \Umathchar
             \else % else not \Umathcar
%    \end{macrocode}
%    \begin{macrocode}
           \@xp\macro@\meaning@@\macro@
           \ifgtest@ % if macro test
             \@xp\not@\meaning@\not@
             \ifgtest@ % if macro starts \not test
               \gdef\thedots@{\dotsb@}%
             \else% else not \not
               \@xp\DOTS@\meaning@\DOTS@
               \ifgtest@ % \if DOTS
                 \ifcase\number\DOTSCASE@ %ifcase dots
                   \gdef\thedots@{\dotsb@}%
                 \or\gdef\thedots@{\dotsi}\else
                 \fi % endifcase dots
               \else % not macro starts \DOTS
                 \@xp\math@\meaning@\math@
                 \ifgtest@ % \if macro starts \mathxxxx
                   \@xp\mathbin@\meaning@\mathbin@
                   \ifgtest@ % if macro starts \mathbin
                     \gdef\thedots@{\dotsb@}%
                   \else % not macro starting \mathbin
                     \@xp\mathrel@\meaning@\mathrel@
                     \ifgtest@ % if macro starts \mathrel
                       \gdef\thedots@{\dotsb@}%
                     \fi % endif macro starts \mathrel (no else)
                   \fi % endif macro starts \mathbin
                 \fi % endif macro starts with \mathxxx (no else)
               \fi % endif macro starts \DOTS else
             \fi % end macro  starting \not \ifgtest@ test (no else)
%    \end{macrocode}
% Additional test for a catcode 12 character.
%    \begin{macrocode}
             \else
               \@xp\thecharacter@\meaning@\thecharacter@
%    \end{macrocode}
%    \begin{macrocode}
             \fi % end macro \ifgtest@ test (no else)
           \fi % end if \Umathchar test
         \fi % end \math@   \ifgtest@
       \fi % end \keybin@ \ifgtest@ test (no else)
     \fi % end if \not (no else)
   \fi % end if comma (no else)
 \fi % end if boldsymbol (no else)
 \thedots@}
%    \end{macrocode}
%
%    The \qc{\=} character is necessary in the two \cs{let} assignments
%    in \cs{boldsymboldots@}, because the symbol we are making
%    bold might be an \qc{\=} sign.
%    \begin{macrocode}
\def\boldsymboldots@#1{%
  \bold@true\let\@let@token=#1\let\delayed@=#1\mdots@@
  \boldsymbol#1\bold@false}
%    \end{macrocode}
%
%    The definition of \cs{@cdots} is merely the \fn{plain.tex}
%    definition of \cs{cdots}.
%    \begin{macrocode}
\ams@def\@cdots{\mathinner{\cdotp\cdotp\cdotp}}
\newcommand{\dotsi}{\!\@cdots}
\let\dotsb@\@cdots
%    \end{macrocode}
%
%    If any new right delimiters are defined, they would need to be
%    added to the definition of \cs{rightdelim@} in order for \cn{dots}
%    to work properly in all cases.
%    \begin{macrocode}
\def\rightdelim@{\gtest@true
 \ifx\@let@token)\else
 \ifx\@let@token]\else
 \ifx\@let@token\rbrack\else
 \ifx\@let@token\}\else
 \ifx\@let@token\rbrace\else
 \ifx\@let@token\rangle\else
 \ifx\@let@token\rceil\else
 \ifx\@let@token\rfloor\else
 \ifx\@let@token\rgroup\else
 \ifx\@let@token\rmoustache\else
 \ifx\@let@token\right\else
 \ifx\@let@token\bigr\else
 \ifx\@let@token\biggr\else
 \ifx\@let@token\Bigr\else
 \ifx\@let@token\Biggr\else\gtest@false
 \fi\fi\fi\fi\fi\fi\fi\fi\fi\fi\fi\fi\fi\fi\fi}
\def\extra@{%
 \rightdelim@\ifgtest@
 \else\ifx\@let@token$\gtest@true
 \else\xdef\meaning@{\meaning\@let@token..........}%
 \@xp\macro@\meaning@\macro@\ifgtest@
 \@xp\DOTS@\meaning@\DOTS@
 \ifgtest@
 \ifnum\DOTSCASE@=\tw@\gtest@true\else\gtest@false
 \fi\fi\fi\fi\fi}
\newif\ifbold@
\def\dotso@{\relaxnext@
 \ifbold@
  \let\@let@token\delayed@
  \def\nextii@{\extra@\@ldots\ifgtest@\,\fi}%
 \else
  \def\nextii@{\DN@{\extra@\@ldots\ifgtest@\,\fi}\FN@\next@}%
 \fi
 \nextii@}
%    \end{macrocode}
%    Why not save some tokens? (space vs. time).
%    \begin{macrocode}
\def\extrap@#1{%
 \DN@{#1\,}%
 \ifx\@let@token,\else
 \ifx\@let@token;\else
 \ifx\@let@token.\else\extra@
 \ifgtest@\else
 \let\next@#1\fi\fi\fi\fi\next@}
%    \end{macrocode}
%
%    \begin{macro}{\cdots}
%    \begin{macro}{\dotsb}
%    \begin{macro}{\dotsm}
%    \begin{macro}{\dotso}
%    \begin{macro}{\dotsc}
%    The \cn{cdots} command.
%    \begin{macrocode}
\ams@DeclareRobustCommand{\cdots}{\DN@{\extrap@\@cdots}\FN@\next@}
\let\dotsb\cdots
\let\dotsm\cdots
\DeclareRobustCommand{\dotso}{\relax
  \ifmmode \DN@{\extrap@\@ldots}%
  \else \let\next@\tdots@\fi
  \FN@\next@}
\DeclareRobustCommand{\dotsc}{%
  \DN@{\ifx\@let@token;\@ldots\,%
       \else \ifx\@let@token.\@ldots\,%
       \else \extra@\@ldots \ifgtest@\,\fi
       \fi\fi}%
  \FN@\next@}
%    \end{macrocode}
%    \end{macro}
%    \end{macro}
%    \end{macro}
%    \end{macro}
%    \end{macro}
%    \begin{macro}{\longrightarrow}
%    \begin{macro}{\Longrightarrow}
%    \begin{macro}{\longleftarrow}
%    \begin{macro}{\Longleftarrow}
%    \begin{macro}{\longleftrightarrow}
%    \begin{macro}{\Longleftrightarrow}
%    \begin{macro}{\mapsto}
%    \begin{macro}{\longmapsto}
%    \begin{macro}{\hookrightarrow}
%    \begin{macro}{\hookleftarrow}
%    \begin{macro}{\iff}
%    Various arrows.
%    \begin{macrocode}
\renewcommand{\longrightarrow}{%
  \DOTSB\protect\relbar\protect\joinrel\rightarrow}
\renewcommand{\Longrightarrow}{%
  \DOTSB\protect\Relbar\protect\joinrel\Rightarrow}
\renewcommand{\longleftarrow}{%
  \DOTSB\leftarrow\protect\joinrel\protect\relbar}
\renewcommand{\Longleftarrow}{%
  \DOTSB\Leftarrow\protect\joinrel\protect\Relbar}
\renewcommand{\longleftrightarrow}{\DOTSB\leftarrow\joinrel\rightarrow}
\renewcommand{\Longleftrightarrow}{\DOTSB\Leftarrow\joinrel\Rightarrow}
\renewcommand{\mapsto}{\DOTSB\mapstochar\rightarrow}
\renewcommand{\longmapsto}{\DOTSB\mapstochar\longrightarrow}
\renewcommand{\hookrightarrow}{\DOTSB\lhook\joinrel\rightarrow}
\renewcommand{\hookleftarrow}{\DOTSB\leftarrow\joinrel\rhook}
\renewcommand{\iff}{\DOTSB\;\Longleftrightarrow\;}
%    \end{macrocode}
%    \end{macro}
%    \end{macro}
%    \end{macro}
%    \end{macro}
%    \end{macro}
%    \end{macro}
%    \end{macro}
%    \end{macro}
%    \end{macro}
%    \end{macro}
%    \end{macro}
%    \begin{macro}{\doteq}
%    The \cn{doteq} command formerly used \cs{buildrel}; we avoid that
%    because it requires `\cn{over}' as part of its syntax. Use 0pt
%    instead of \cs{z@} for robustitude.
%    \begin{macrocode}
\renewcommand{\doteq}{%
  \DOTSB\mathrel{\mathop{\kern0pt =}\limits^{\textstyle.}}}
%    \end{macrocode}
%    \end{macro}
%
% %%%%%%%%%%%%%%%%%%%%%%%%%%%%%%%%%%%%%%%%%%%%%%%%%%%%%%%%%%%%%%%%%%%%%%
% \section{Integral signs}
%
%    \begin{macro}{\if@display}
%    The straightforward \cs{ifinner} test to see if the current math
%    context is non-display, fails if, for instance, we are typesetting
%    a multiline display within an \cs{halign}, with the pieces going
%    into constructions like
%    \begin{verbatim}
%    $\displaystyle...$
%    \end{verbatim}
%    So we need a better test to find out if we are `in a display'. We
%    therefore create \cs{if@display}.
%
%    \begin{macrocode}
\newif\if@display
\everydisplay\@xp{\the\everydisplay \@displaytrue}
%    \end{macrocode}
%    \end{macro}
%
%    \begin{macro}{\int}
%    \begin{macro}{\oint}
%    \begin{macro}{\iint}
%    \begin{macro}{\iiint}
%    \begin{macro}{\iiiint}
%    \begin{macro}{\idotsint}
%    Default value for integral limits is \cs{nolimits}, see the
%    definition of the `nointlimits' option.
%    \begin{macrocode}
\renewcommand{\int}{\DOTSI\intop\ilimits@}
\renewcommand{\oint}{\DOTSI\ointop\ilimits@}
\def\intkern@{\mkern-6mu\mathchoice{\mkern-3mu}{}{}{}}
\def\intdots@{\mathchoice{\@cdots}%
 {{\cdotp}\mkern1.5mu{\cdotp}\mkern1.5mu{\cdotp}}%
 {{\cdotp}\mkern1mu{\cdotp}\mkern1mu{\cdotp}}%
 {{\cdotp}\mkern1mu{\cdotp}\mkern1mu{\cdotp}}}
%
\ams@newcommand{\iint}{\DOTSI\protect\MultiIntegral{2}}
\ams@newcommand{\iiint}{\DOTSI\protect\MultiIntegral{3}}
\ams@newcommand{\iiiint}{\DOTSI\protect\MultiIntegral{4}}
\newcommand{\idotsint}{\DOTSI\protect\MultiIntegral{0}}
%    \end{macrocode}
%    \end{macro}
%    \end{macro}
%    \end{macro}
%    \end{macro}
%    \end{macro}
%    \end{macro}
%
%    \begin{macro}{\MultiIntegral}
%    If the \cs{limits} option is applied, use \cs{mathop} and fudge
%    the left-hand space a bit to make the subscript visually centered.
%
%    \verb'#1' is the multiplicity.
%    \begin{macrocode}
\newcommand{\MultiIntegral}[1]{%
  \edef\ints@c{\noexpand\intop
    \ifnum#1=\z@\noexpand\intdots@\else\noexpand\intkern@\fi
    \ifnum#1>\tw@\noexpand\intop\noexpand\intkern@\fi
    \ifnum#1>\thr@@\noexpand\intop\noexpand\intkern@\fi
    \noexpand\intop
    \noexpand\ilimits@
  }%
  \futurelet\@let@token\ints@a
}
%    \end{macrocode}
%    \end{macro}
%
%    \begin{macrocode}
\def\ints@a{%
  \ifx\limits\@let@token \ints@b
  \else \ifx\displaylimits\@let@token \ints@b
  \else\ifx\ilimits@\displaylimits \ints@b
  \fi\fi\fi
  \ints@c
}
%    \end{macrocode}
%
%    \begin{macrocode}
\def\ints@b{%
  \mkern-7mu\mathchoice{\mkern-2mu}{}{}{}%
  \mathop\bgroup
    \mkern7mu\mathchoice{\mkern2mu}{}{}{}%
    \let\ilimits@\egroup
}%
%    \end{macrocode}
%
% %%%%%%%%%%%%%%%%%%%%%%%%%%%%%%%%%%%%%%%%%%%%%%%%%%%%%%%%%%%%%%%%%%%%%%
% \section{Size dependent definitions}
%
%    We now define all stuff which has to change whenever a new math
%    size is to be activated. \latex/ provides a hook called
%    |\every@math@size| to support such a need. All assignments in the
%    |\every@math@size| hook that need to take outside effect should be
%    global.
%
% \subsection{Struts for math}
%
%    The various kinds of struts could use some analysis and perhaps
%    consolidation.
%
%    For example perhaps the \cn{bBigg} delimiters could use
% \begin{verbatim}
% 1.2\ht\strutbox (1.8, 2.4, 3.0)
% \end{verbatim}
%    instead of
% \begin{verbatim}
% 1.0\big@size (1.5, 2.0, 2.5)
% \end{verbatim}
%    since \cs{strut} is reset with every size change [mjd, 1994/10/07].
%    But this change would introduce the possibility of changed line
%    and page breaks in existing documents, so would need to be
%    handled with care.
%
% \begin{macro}{\Mathstrut@}
% \begin{macro}{\Mathstrutbox@}
% \begin{macro}{\resetMathstrut@}
%    Here comes the code for Spivak's |\Mathstrut@|.
%    \begin{macrocode}
\newbox\Mathstrutbox@
\setbox\Mathstrutbox@=\hbox{}
\def\Mathstrut@{\copy\Mathstrutbox@}
%    \end{macrocode}
%    The setting of the height and depth of the |\Mathstrutbox@| is done
%    in the |\every@math@size| hook since it depends on the height of a
%    paren. As \cs{every@math@size} is triggered by |$| after a font
%    size change, we want to avoid using another math formula |$...$| to
%    measure the math paren height; instead we go through the mathcode
%    of the \qc{\(} character. We assume that the mathcode has a leading
%    hex digit 4 indicating `open delimiter'; this allows us to make a
%    relatively simple function to get the correct font and character
%    position.
%
% \changes{v2.15}{2016/02/20}{Modify \cs{resetMathstrut@} for Unicode \TeX}
% \changes{v2.15b}{2016/03/10}{Modify \cs{resetMathstrut@} for classic \TeX\ to preserve box 0}
% Original code assuming |\mathcode| is kept for 8bit \TeX. Unicode \TeX{}
% uses |\Umathcharnumdef| which works for xetex and luatex, which use
% different forms for |\mathchardef|. (New luatex always reports
% definitions using |\Umathchardef| syntax even if |\mathchardef| used.)
%
% The unicode version uses e-tex |\fontcharht| to avoid boxing which
% could also be done for pdftex, but not done here.
%    \begin{macrocode}
\ifx\Umathcharnumdef\@undefined
%    \end{macrocode}
% Original code
%    \begin{macrocode}
\def\resetMathstrut@{%
  \begingroup
  \setbox\z@\hbox{%
    \mathchardef\@tempa\mathcode`\(\relax
    \def\@tempb##1"##2##3{\the\textfont"##3\char"}%
    \expandafter\@tempb\meaning\@tempa \relax
  }%
 \edef\@tempa{%
     \ht\Mathstrutbox@\the\ht\z@\relax
     \dp\Mathstrutbox@\the\dp\z@\relax}%
  \expandafter\endgroup\@tempa
}
\else
%    \end{macrocode}
% xetex/luatex version
% \changes{v2.15a}{2016/03/03}{missing percent added}
%    \begin{macrocode}
\def\resetMathstrut@{%
    \begingroup
    \Umathcharnumdef\@tempa\Umathcodenum`\(\relax
    \def\@tempb##1"##2"##3"##4\relax{%
      \endgroup
      \ht\Mathstrutbox@=\fontcharht\textfont"##3 "##4\relax
      \dp\Mathstrutbox@=\fontchardp\textfont"##3 "##4\relax}%
    \expandafter\@tempb\meaning\@tempa \relax
}
%    \end{macrocode}
%    \begin{macrocode}
\fi
%    \end{macrocode}
%    These height and depth assignments are implicitly global.
%    \begin{macrocode}
\addto@hook\every@math@size{\resetMathstrut@}
%    \end{macrocode}
% \end{macro}
% \end{macro}
% \end{macro}
%
% \begin{macro}{\strut@}
% \begin{macro}{\strutbox@}
%    Next follows a special internal strut which is supposed to match
%    the height and the depth of a normal |\strut| minus
%    |\normallineskiplimit| according to M. Spivak.
%
%    This should really go into the definition of \cs{size@update}, and
%    then the box reset could be local; but \cs{size@update} doesn't
%    have any hook and is handled in such a way that it cannot even be
%    changed except by changing \cs{set@fontsize}. So instead we put
%    \cs{reset@strutbox@} into \cs{every@math@size} and make it global.
%    Then because of some complications in the way \cs{glb@settings} and
%    \cs{check@mathfonts} work, we have to re-invoke it at the beginning
%    of every environment that might use \cs{strut@}. Fortunately this
%    can be achieved (more or less) through the \cs{spread@equation}
%    hook. [mjd,2000/03/10]
%    \begin{macrocode}
\newbox\strutbox@
\def\strut@{\copy\strutbox@}
\def\reset@strutbox@{%
  \global\setbox\strutbox@\hbox{%
    \lower.5\normallineskiplimit
       \vbox{\kern-\normallineskiplimit\copy\strutbox}}}
\addto@hook\every@math@size{\reset@strutbox@}
\AtBeginDocument{\reset@strutbox@}
%    \end{macrocode}
% \end{macro}
% \end{macro}
%
% \subsection{Big delimiters}
%
%    We are now going to redefine the plain \tex/ commands \cn{big},
%    \cn{bigl}, etc., to produce different results in different sizes.
%    Actually we only have to define \cn{big}, \cn{Big}, etc., since
%    they are used to construct the directional versions \cn{bigl},
%    \cn{bigr}, and the rest.
%
% \begin{macro}{\big}
% \begin{macro}{\Big}
% \begin{macro}{\bigg}
% \begin{macro}{\Bigg}
%     To save token space we put everything into the common macro
%    |\bBigg@|. The macros are now simply a call to |\bBigg@| with a
%    factor to determine the correct height of the delimiter as an
%    argument. This code should better go into a future version of
%    the \latex/ kernel; the macro |\n@space| is then superfluous (since
%    it is only used once) and should be removed to avoid wasting hash
%    table space unnecessarily.
%    \begin{macrocode}
\renewcommand{\big}{\bBigg@\@ne}
\renewcommand{\Big}{\bBigg@{1.5}}
\renewcommand{\bigg}{\bBigg@\tw@}
\renewcommand{\Bigg}{\bBigg@{2.5}}
%    \end{macrocode}
% \end{macro}
% \end{macro}
% \end{macro}
% \end{macro}
%
% \begin{macro}{\bBigg@}
%    Now we tackle the macro which has to do the real work. It
%    actually has two arguments, the factor and the wanted delimiter.
% \changes{v2.17b}{2018/12/01}{Start LR-mode for \cs{bigl} and
%   friends if necessary (github/49)}
%    \begin{macrocode}
\ifx\leavevmode@ifvmode\@undefined
\def\bBigg@#1#2{%
%    \end{macrocode}
%    We start with an extra set of braces because we want
%    constructions like |\def\bigl{\mathopen\big}| to work without the
%    overhead of extra arguments.
%    \begin{macrocode}
  {\@mathmeasure\z@{\nulldelimiterspace\z@}%
     {\left#2\vcenter to#1\big@size{}\right.}%
   \box\z@}}
\else
\def\bBigg@#1#2{\leavevmode@ifvmode
  {\@mathmeasure\z@{\nulldelimiterspace\z@}%
     {\left#2\vcenter to#1\big@size{}\right.}%
   \box\z@}}
\fi
%    \end{macrocode}
% \end{macro}
%
% \begin{macro}{\big@size}
%    |\big@size| needs to be set to 1.2 times the height of a math
%    paren. This height is already recorded in |\Mathstrutbox@|.
%    \begin{macrocode}
\addto@hook\every@math@size{%
  \global\big@size 1.2\ht\Mathstrutbox@
  \global\advance\big@size 1.2\dp\Mathstrutbox@ }
\newdimen\big@size
%    \end{macrocode}
% \end{macro}
%
%
% %%%%%%%%%%%%%%%%%%%%%%%%%%%%%%%%%%%%%%%%%%%%%%%%%%%%%%%%%%%%%%%%%%%%%%
% \section{Math accents}
%
%    We want to change the leading digit of math accents to be
%    \cs{accentclass@} so that it can vary according to certain internal
%    purposes.
%    \begin{macrocode}
\def\accentclass@{7}
\def\noaccents@{\def\accentclass@{0}}
%    \end{macrocode}
%
%    There are a few \meta{math alphabet}s in the standard fonts where
%    we have to change the extra macros because the standard definitions
%    don't account for these accent problems. The first is for the
%    \cs{mathit} command.
%    \begin{macrocode}
\DeclareFontEncoding{OML}{}{\noaccents@}
%    \end{macrocode}
%    The next one corrects the \cs{cal} alphabet.
%    \begin{macrocode}
\DeclareFontEncoding{OMS}{}{\noaccents@}
%    \end{macrocode}
%
%    \begin{macro}{\dddot}
%    \begin{macro}{\ddddot}
%    Triple and quadruple dot accents.
% \changes{v2.17f}{2020/02/20}{Add a kern so that a single char is not
%       vertically shifted and move the dots slightly to the right (gh/126)}
%    \begin{macrocode}
\ams@newcommand{\dddot}[1]{%
  {\mathop{\kern\z@#1}\limits^{\vbox to-1.4\ex@{\kern-\tw@\ex@
   \hbox{\,\normalfont...}\vss}}}}
\ams@newcommand{\ddddot}[1]{%
  {\mathop{\kern\z@#1}\limits^{\vbox to-1.4\ex@{\kern-\tw@\ex@
   \hbox{\,\normalfont....}\vss}}}}
%    \end{macrocode}
%    \end{macro}
%    \end{macro}
%
%    The following code deals with support for compound accents.
%    By redefining \cs{set@mathaccent} we ensure that
%    \cn{DeclareMathAccent} will define accent commands to run our
%    \cs{mathaccentV} function instead of the primitive \cs{mathaccent}.
% \changes{v2.17d}{2019/11/16}{Added \cs{MakeRobust} to match kernel
%                              change (gh/216)}
%    \begin{macrocode}
\def\set@mathaccent#1#2#3#4{%
%    \end{macrocode}
%    Now that the redefinitions done inside amsmath of the basic accents
%    are all robust  we can drop the \cs{protect} here.
%    \begin{macrocode}
  \xdef#2{%
%    \@nx\protect
    \@nx\mathaccentV
    {\@xp\@gobble\string#2}\hexnumber@#1#4}%
  \MakeRobust#2%
}
%    \end{macrocode}
%
%    \begin{macro}{\hat}
%    \begin{macro}{\check}
%    \begin{macro}{\tilde}
%    \begin{macro}{\acute}
%    \begin{macro}{\grave}
%    \begin{macro}{\dot}
%    \begin{macro}{\ddot}
%    \begin{macro}{\breve}
%    \begin{macro}{\bar}
%    \begin{macro}{\vec}
%    \begin{macro}{\mathring}
%    We redefine the standard math accent commands to
%    call \cs{mathaccentV}, using the mathgroup/encoding-number
%    information embedded in their previous definitions. If the
%    definition of an accent command does not have the expected form, we
%    leave the accent command alone, but give a warning. For widehat and
%    widetilde, we need to avoid clobbering the definitions done by the
%    \pkg{amsfonts} package. Arbitrating the contention between
%    \pkg{amsmath} and \pkg{amsfonts} to allow doubling a widetilde
%    accent looks tricky, so for the time being [mjd,1999/07/19] we just
%    leave \cn{widehat} and \cn{widetilde} alone. As a result, if the
%    \pkg{amsmath} package is loaded on top of a vanilla \latex/
%    documentclass, everything runs through with no warnings. If a
%    Lucida Math or other math fonts package is loaded in addition to
%    \pkg{amsmath}, there are greater difficulties, but those are
%    addressed elsewhere.
%
%
% \changes{v2.15}{2016/02/20}{Detect \cs{Umathaccent} on package load}
% Adjust the test made at package load to recognise
% |\Umathaccent|. Although currently it is just used to give a
% modified warning that the accents will not be redefined.
%
% Note that the engines behave quite differently here, luatex
% even without these definitions using the OpenType accent set up by
% unicode-math  stacks |\hat{hat{f}}| correctly but xetex acts like
% classic tex and needs this adjustment. This difference is not
% addressed here at all.
%
% This test is just at package loading and has no affect on the
% definitions used in 8bit TeX.
%    \begin{macrocode}
%\def\@tempa#1{\@xp\@tempb\meaning#1\@nil#1}
%    \end{macrocode}
%    The extended definition below tests if the accent is already
%    robust (as newer \LaTeX{} kernels do this by default) and if so
%    picks up the robust definition. However, as of
%    now it still redefines it to be non-robust.
%    \begin{macrocode}
\def\@tempa#1{%
  \@ifundefined{\@xp\@gobble\string#1\space}%
     {\@xp\@tempb\meaning#1\@nil#1}%
     {\@xp\@xp\@xp\@tempb\@xp\meaning
       \csname\@xp\@gobble\string#1\space\endcsname\@nil#1}%
}
\def\@tempb#1>#2#3 #4\@nil#5{%
  \@xp\ifx\csname#3\endcsname\mathaccent
    \@tempc#4?"7777\@nil#5%
  \else
  \@xp\ifx\csname#3\endcsname\Umathaccent
    \@tempd#4\@nil#5%
  \else
    \PackageWarningNoLine{amsmath}{%
      Unable to redefine math accent \string#5}%
  \fi\fi}
\def\@tempc#1"#2#3#4#5#6\@nil#7{%
%    \end{macrocode}
%    Drop the inner part of the robust accent so that it can be recreated
%    without a warning.
% \changes{v2.17d}{2019/11/16}{Make temporarily fragile to match kernel
%                              change (gh/216)}
%    \begin{macrocode}
  \@xp\let\csname\@xp\@gobble\string#7\space\endcsname\@undefined
  \chardef\@tempd="#3\relax\set@mathaccent\@tempd{#7}{#2}{#4#5}}
\def\@tempd#1\@nil#2{%
    \PackageWarningNoLine{amsmath}{%
      Unable to redefine \string\Umathaccent\space\string#2}%
}
%    \end{macrocode}
%    \begin{macrocode}
\@tempa{\hat}
\@tempa{\check}
\@tempa{\tilde}
\@tempa{\acute}
\@tempa{\grave}
\@tempa{\dot}
\@tempa{\ddot}
\@tempa{\breve}
\@tempa{\bar}
\@tempa{\vec}
\@ifundefined{mathring}{%
  \DeclareMathAccent{\mathring}{\mathalpha}{operators}{"17}
}{%
  \@tempa{\mathring}
}
%%\@tempa\widetilde
%%\@tempa\widehat
%    \end{macrocode}
%    \end{macro}
%    \end{macro}
%    \end{macro}
%    \end{macro}
%    \end{macro}
%    \end{macro}
%    \end{macro}
%    \end{macro}
%    \end{macro}
%    \end{macro}
%    \end{macro}
%
%    Regression testing of amsmath 2.0 showed that in some documents
%    there occurred fragments of the form
% \begin{verbatim}
% \hat\mathcal{G}
% \end{verbatim}
%    This is not at all correct syntax for the argument of a \latex/
%    command but it produced the intended result anyway because of the
%    internal syntax of the \cs{mathaccent} primitive. With
%    \cs{mathaccentV}, it will yield an error message. We therefore do a
%    special check for such syntax problems in order to make the error
%    message more informative. (dmj: ??????)
%    \begin{macrocode}
\newcommand{\acc@check}{}
\newcommand{\acc@error}{}
\def\acc@check{\@ifnextchar\@empty\relax\acc@error}
%    \end{macrocode}
%    We put most of the tokens in a separate macro so they do not get
%    scanned unless they are actually needed.
%    \begin{macrocode}
\def\acc@error{%
  \@amsmath@err{%
    Improper argument for math accent:\MessageBreak
    Extra braces must be added to prevent wrong output%
  }\@ehc
}
%    \end{macrocode}
%
%    For \cs{mathaccentV} part of the processing is dependent on the
%    depth of nesting of math accent commands. We introduce a dedicated
%    counter for this instead of using chardef because we want to
%    increment/decrement it during processing, and incrementing a
%    chardef integer is more work.
%    \begin{macrocode}
\newcount\macc@depth
%    \end{macrocode}
%
%    Provide this function in case it is not already available.
%    \begin{macrocode}
\long\def\@gobblethree#1#2#3{}
%    \end{macrocode}
%
%    The \cs{mathaccentV} function first counts the number of nested
%    math accents by setting the argument in a throw-away box. (This is
%    not as risky as such an operation would normally be because the
%    argument is generally either a simple math symbol or a nested math
%    accent call with a simple math symbol at the bottom of the
%    nesting.)
%
%    There are two benefits from counting the nesting levels first
%    before doing anything else: (1) we can fall back to a simple
%    \cs{mathaccent} call if the nesting depth is 1, and (2) if the
%    nesting depth is greater than 1, we would like to be able to tell
%    when we have reached the lowest level, because at that point we
%    want to save the argument for later use and place an accent on top
%    of a phantom copy.
%
%    When we have multiple accents, they will be placed on top of the
%    invisible box, followed by some suitable kerns, then a visible copy
%    of the nucleus. To see why, let us look at what goes wrong with a
%    double application of the \cs{mathaccent} primitive. The standard
%    definition of \cs{hat} is \verb'\mathaccent"705E', so
%    \verb'\hat{\hat{F}}' expands to
%\begin{verbatim}
%\mathaccent"705E{\mathaccent"705E{F}}
%\end{verbatim}
%    The result of this operation is
%\begin{verbatim}
%\vbox(12.11111+0.0)x7.81946
%.\hbox(6.94444+0.0)x0.0, shifted 1.40973
%..\OT1/cmr/m/n/10 ^
%.\kern-4.30554
%.\vbox(9.47221+0.0)x7.81946
%..\hbox(6.94444+0.0)x0.0, shifted 2.24309
%...\OT1/cmr/m/n/10 ^
%..\kern-4.30554
%..\hbox(6.83331+0.0)x7.81946
%...\OML/cmm/m/it/10 F
%\end{verbatim}
%    \tex/ starts by constructing a vbox with the hat character on top
%    of the F. Then it puts another hat character on top of the vbox;
%    but without skew information, because that is only applied by
%    \cs{mathaccent} when the base object is a simple symbol. So the
%    first accent is skewed to the correct position but all later
%    accents are not. By the way, the actual width of the F in the above
%    example is less than 7.81946; the box in which it is packed was
%    automatically lengthened by the width of the F's italic correction
%    (without actually putting in a kern for it).
%
%    To get the second accent shifted farther to the right we
%    artificially increase the width of the innermost box and add
%    a compensating kern afterward. Furthermore, to get proper placement
%    of a following subscript or superscript, we take the base symbol
%    out, leaving a phantom in its place, and print it by itself
%    following the kern. We then need to increase the kern amount to
%    move the base character backward under the accents again.
%    Here is what the results look like:
%\begin{verbatim}
%\vbox(12.11111+0.0)x9.48618
%.\hbox(6.94444+0.0)x0.0, shifted 2.24309
%..\OT1/cmr/m/n/10 ^
%.\kern-4.30554
%.\vbox(9.47221+0.0)x9.48618
%..\hbox(6.94444+0.0)x0.0, shifted 2.24309
%...\OT1/cmr/m/n/10 ^
%..\kern-4.30554
%..\hbox(6.83331+0.0)x9.48618
%...\hbox(6.83331+0.0)x7.81946
%...\kern 1.66672
%\kern -9.48618
%\OML/cmm/m/it/10 F
%\end{verbatim}
%
%    Much of this implementation is based on code from the \pkg{accents}
%    package of Javier Bezos. I added the test to revert to a simple
%    \cs{mathaccent} when accents are not nested, and some other
%    refinements to reduce the number of kerns used (to conserve box
%    memory) and the number of cycles through \cs{mathchoice} (to make
%    things run a little faster). It was all rather difficult and my
%    first two attempts had serious bugs but I hope and believe that
%    this version will do better. [mjd,2000/03/15]
%
%    The \qq{V} in \cs{mathaccentV} is just an indication that it takes
%    five arguments. It is important that the name includes
%    \texttt{mathaccent}, otherwise \cs{DeclareMathAccent} will balk at
%    redefining one of our accent commands, for example when an
%    alternative math font package is loaded.
%    \begin{macrocode}
\def\mathaccentV#1#2#3#4#5{%
  \ifmmode
    \gdef\macc@tmp{\macc@depth\@ne}%
    \setbox\z@\hbox{%
      \let\mathaccentV\macc@test
      \let\use@mathgroup\@gobbletwo \let\select@group\@gobblethree
      \frozen@everymath{}$#5$%
    }%
    \macc@tmp
    \ifnum\macc@depth=\@ne
      \global\let\macc@nucleus\@empty
      \mathaccent"\accentclass@
    \else
      \@xp\macc@nested
    \fi
    #2#3#4{#5}%
    \macc@nucleus
  \else
    \@xp\nonmatherr@\csname#1\endcsname
  \fi
}
%    \end{macrocode}
%
%    \begin{macrocode}
\def\macc@test#1#2#3#4{\xdef\macc@tmp{\macc@tmp\advance\macc@depth\@ne}}
%    \end{macrocode}
%
%    \begin{macrocode}
\def\macc@group{-1}
%    \end{macrocode}
%
%    \begin{macrocode}
\def\macc@nested#1#2#3#4{%
  \begingroup
  \let\math@bgroup\@empty \let\math@egroup\macc@set@skewchar
  \mathsurround\z@ \frozen@everymath{\mathgroup\macc@group\relax}%
  \macc@set@skewchar\relax
  \let\mathaccentV\macc@nested@a
  \macc@nested@a\relax#1#2#3{#4}%
  \endgroup
}
%    \end{macrocode}
%
%    \begin{macrocode}
\let\macc@palette\mathpalette
%    \end{macrocode}
%
%    \begin{macrocode}
\def\macc@nested@a#1#2#3#4#5{%
%    \end{macrocode}
%    This test saves some work that would otherwise be always repeated
%    fourfold thanks to \cs{mathchoice}.
%    \begin{macrocode}
  \ifnum\macc@group=\mathgroup
  \else \macc@set@skewchar\relax \edef\macc@group{\the\mathgroup}%
  \fi
  \mathchardef\macc@code "\accentclass@ #2#3#4\relax
  \macc@palette\macc@a{#5}%
}
%    \end{macrocode}
%
%    The reason that \cs{macc@set@skewchar} takes an argument is so that
%    it can serve as a direct substitute for \cs{math@egroup}, in
%    addition to being used separately.
%
%    Setting a skewchar with this method works for symbols of variable
%    mathgroup (class 7, letters and numbers) but not necessarily for
%    special symbols like \cn{partial} or \cs{xi} whose mathgroup
%    doesn't change; fortunately the most commonly used ones come from
%    mathgroup one, which is the fall-back mathgroup for skewchar.
%    \begin{macrocode}
\def\macc@set@skewchar#1{%
  \begingroup
  \ifnum\mathgroup=\m@ne \let\@tempa\@ne
  \else
    \ifnum\skewchar\textfont\mathgroup=\m@ne \let\@tempa\@ne
    \else \let\@tempa\mathgroup
    \fi
  \fi
  \count@=\skewchar\textfont\@tempa
  \advance\count@"7100
  \edef\@tempa{\endgroup
    \mathchardef\noexpand\macc@skewchar=\number\count@\relax}%
  \@tempa
  #1%
}
%    \end{macrocode}
%
%    Arg1 is math-style, arg2 is accent base object. We assume that math
%    style doesn't change within the nested group of accents; this means
%    we can set \cs{macc@style} only once and redefine \cs{macc@palette}
%    to use it, in order to run \cs{mathchoice} only once instead of
%    multiplying the calls exponentially as the nesting level increases.
%    \begin{macrocode}
\def\macc@a#1#2{%
  \begingroup
  \let\macc@style#1\relax
  \def\macc@palette##1{##1\macc@style}%
  \advance\macc@depth\m@ne
  \ifnum\macc@depth=\z@
    \gdef\macc@nucleus{#2}%
%    \end{macrocode}
%    Extra \cs{@empty} tokens are to prevent low-level \tex/ errors from
%    the potential syntactic error that \cs{acc@check} checks for.
%    \begin{macrocode}
    \setbox\z@\hbox{$#1#2\@empty{}\macc@skewchar$}%
    \setbox\tw@\hbox{$#1#2\@empty\macc@skewchar$}%
    \dimen@\tw@\wd\tw@ \advance\dimen@-\tw@\wd\z@
    \xdef\macc@kerna{\the\dimen@\relax}%
    \setbox4\hbox{$#1#2\acc@check\@empty$}%
    \global\setbox\@ne\hbox to\wd4{}%
    \ht\@ne\ht4 \dp\@ne\dp4
    \xdef\macc@kernb{\the\wd4\relax}%
    \mathaccent\macc@code{\box\@ne\kern\macc@kerna}%
  \else
    \mathaccent\macc@code{\let\macc@adjust\@empty #1#2\@empty}%
    \macc@adjust
  \fi
  \endgroup
}
%    \end{macrocode}
%
%    \begin{macrocode}
\def\macc@adjust{%
  \dimen@\macc@kerna\advance\dimen@\macc@kernb
  \kern-\dimen@
}
%    \end{macrocode}
%
%    The commands \cs{Hat}, \cs{Tilde}, \ldots, are supported as
%    synonyms of \cs{hat}, \cs{tilde}, \ldots, for backward
%    compatibility.
%    \begin{macrocode}
\def\Hat{\hat}
\def\Check{\check}
\def\Tilde{\tilde}
\def\Acute{\acute}
\def\Grave{\grave}
\def\Dot{\dot}
\def\Ddot{\ddot}
\def\Breve{\breve}
\def\Bar{\bar}
\def\Vec{\vec}
%    \end{macrocode}
%
%    This error message about math mode is used several times so we make
%    an abbreviation for it.
%    \begin{macrocode}
\def\nonmatherr@#1{\@amsmath@err{\protect
  #1 allowed only in math mode}\@ehd}
%    \end{macrocode}
%
% %%%%%%%%%%%%%%%%%%%%%%%%%%%%%%%%%%%%%%%%%%%%%%%%%%%%%%%%%%%%%%%%%%%%%%
% \section{Mods, continued fractions, etc.}
%
%    \begin{macro}{\bmod}
%    \begin{macro}{\pmod}
%    \begin{macro}{\pod}
%    \begin{macro}{\mod}
%    The commands \cn{bmod}, \cn{pmod}, \cn{pod}, \cn{mod} aren't
%    currently robust. [mjd, 1994/09/05]
%    \begin{macrocode}
\renewcommand{\bmod}{\nonscript\mskip-\medmuskip\mkern5mu\mathbin
  {\operator@font mod}\penalty900
  \mkern5mu\nonscript\mskip-\medmuskip}
\newcommand{\pod}[1]{\allowbreak
  \if@display\mkern18mu\else\mkern8mu\fi(#1)}
\renewcommand{\pmod}[1]{\pod{{\operator@font mod}\mkern6mu#1}}
\newcommand{\mod}[1]{\allowbreak\if@display\mkern18mu
  \else\mkern12mu\fi{\operator@font mod}\,\,#1}
%    \end{macrocode}
%    \end{macro}
%    \end{macro}
%    \end{macro}
%    \end{macro}
%
%    \begin{macro}{\cfrac}
%    Continued fractions. The optional arg l or r controls horizontal
%    placement of the numerators. The |\kern-\nulldelimiterspace|
%    is needed in the definition if we want the right-hand sides of the
%    fraction rules to line up. The \cs{strut} keeps the numerator of
%    a subsidiary cfrac from coming too close to the fraction rule above
%    it.
%    \begin{macrocode}
\newcommand{\cfrac}[3][c]{{\displaystyle\frac{%
  \strut\ifx r#1\hfill\fi#2\ifx l#1\hfill\fi}{#3}}%
  \kern-\nulldelimiterspace}
%    \end{macrocode}
%    \end{macro}
%
%    \begin{macro}{\overset}
%    \begin{macro}{\underset}
%    \cn{overset} and \cn{underset} put symbols above, respectively
%    below, a symbol that is not a \cs{mathop} and therefore does not
%    naturally accept limits. \cs{binrel@@} uses information collected
%    by \cs{binrel@} to make the resulting construction be of type
%    mathrel or mathbin if the base symbol is either of those types.
%    \begin{macrocode}
\newcommand{\overset}[2]{\binrel@{#2}%
  \binrel@@{\mathop{\kern\z@#2}\limits^{#1}}}
%    \end{macrocode}
%
%    \begin{macrocode}
\newcommand{\underset}[2]{\binrel@{#2}%
  \binrel@@{\mathop{\kern\z@#2}\limits_{#1}}}
%    \end{macrocode}
%    \end{macro}
%    \end{macro}
%
%
%  \begin{macro}{\overunderset}
%    This is the combination of the previous two commands which is
%    something that is sometimes needed.
% \changes{v2.17c}{2019/04/01}{New command \cs{overunderset}}
%    \begin{macrocode}
\newcommand{\overunderset}[3]{\binrel@{#3}%
  \binrel@@{\mathop{\kern\z@#3}\limits^{#1}_{#2}}}
%    \end{macrocode}
%  \end{macro}
%
%    \begin{macro}{\sideset}
%    \cn{sideset} allows placing `adscript' symbols at the four
%    corners of a \cs{mathop}, \emph{in addition to} limits. Left-side
%    adscripts go into arg \arg{1}, in the form |_{...}^{...}|, and
%    right-side adscripts go into arg \arg{2}.
%
%    As currently written [mjd, 1995/01/21] this is pretty haphazard.
%    In order to really make it work properly in full generality we'd
%    have to read and measure the top and bottom limits and use
%    mathchoice to always get the right mathstyle for each piece,
%    etc., etc.
%    \begin{macrocode}
\newcommand{\sideset}[3]{%
  \@mathmeasure\z@\displaystyle{#3}%
%    \end{macrocode}
%    Use a global box assignment here since the depth override is
%    implicitly global. Then move the constructed box to a local box
%    register (2) to ensure it won't get destroyed during the next two
%    mathmeasure statements. This precaution may be more extreme than
%    necessary in practice.
%    \begin{macrocode}
  \global\setbox\@ne\vbox to\ht\z@{}\dp\@ne\dp\z@
  \setbox\tw@\box\@ne
  \@mathmeasure4\displaystyle{\copy\tw@#1}%
  \@mathmeasure6\displaystyle{#3\nolimits#2}%
  \dimen@-\wd6 \advance\dimen@\wd4 \advance\dimen@\wd\z@
  \hbox to\dimen@{}\mathop{\kern-\dimen@\box4\box6}%
}
%    \end{macrocode}
%    \end{macro}
%
% \begin{macro}{\smash}
%    We add to the \cn{smash} command an optional argument
%    denoting the part of the formula to be smashed.
% \changes{v2.17b}{2018/12/01}{Start LR-mode for \cs{smash}
%   if necessary (github/49)}
%    \begin{macrocode}
\ifx\leavevmode@ifvmode\@undefined
\renewcommand{\smash}[1][tb]{%
  \def\mb@t{\ht}\def\mb@b{\dp}\def\mb@tb{\ht\z@\z@\dp}%
  \edef\finsm@sh{\csname mb@#1\endcsname\z@\z@\box\z@}%
  \ifmmode \@xp\mathpalette\@xp\mathsm@sh
  \else \@xp\makesm@sh
  \fi
}
\else
\renewcommand{\smash}[1][tb]{%
  \def\mb@t{\ht}\def\mb@b{\dp}\def\mb@tb{\ht\z@\z@\dp}%
  \edef\finsm@sh{\csname mb@#1\endcsname\z@\z@ \leavevmode@ifvmode\box\z@}%
  \ifmmode \@xp\mathpalette\@xp\mathsm@sh
  \else \@xp\makesm@sh
  \fi
}
\fi
%    \end{macrocode}
% \end{macro}
%
% %%%%%%%%%%%%%%%%%%%%%%%%%%%%%%%%%%%%%%%%%%%%%%%%%%%%%%%%%%%%%%%%%%%%%%
% \section{Extensible arrows}
%
%    The minus sign used in constructing these arrow fills is smashed so
%    that superscripts above the arrows won't be too high. This
%    primarily affects the \cn{xleftarrow} and \cn{xrightarrow} arrows.
% \changes{v2.17h}{2020/08/24}{detect unicode engines and use their commands}
%    \begin{macrocode}
\@ifundefined{Umathcode}
  {%
    \mathchardef\std@minus\mathcode`\-\relax
    \mathchardef\std@equal\mathcode`\=\relax
  }
  {%
   \Umathcharnumdef\std@minus\Umathcodenum`\-\relax
   \Umathcharnumdef\std@equal\Umathcodenum`\=\relax
  }
%    \end{macrocode}
%    In case some alternative math fonts are loaded later:
%    \begin{macrocode}
\@ifundefined{Umathcode}
  {%
    \AtBeginDocument{%
      \mathchardef\std@minus\mathcode`\-\relax
      \mathchardef\std@equal\mathcode`\=\relax
      }%
  }
  {%
    \AtBeginDocument{%
      \Umathcharnumdef\std@minus\Umathcodenum`\-\relax
      \Umathcharnumdef\std@equal\Umathcodenum`\=\relax
    }%
  }
%    \end{macrocode}
%
%    \begin{macro}{\relbar}
%    \begin{macro}{\Relbar}
%    \begin{macrocode}
\ams@def\relbar{\mathrel{\mathpalette\mathsm@sh\std@minus}}
\ams@def\Relbar{\mathrel\std@equal}
%    \end{macrocode}
%    \end{macro}
%    \end{macro}
%
%    \begin{macrocode}
\def\arrowfill@#1#2#3#4{%
  $\m@th\thickmuskip0mu\medmuskip\thickmuskip\thinmuskip\thickmuskip
   \relax#4#1\mkern-7mu%
   \cleaders\hbox{$#4\mkern-2mu#2\mkern-2mu$}\hfill
   \mkern-7mu#3$%
}
\def\leftarrowfill@{\arrowfill@\leftarrow\relbar\relbar}
\def\rightarrowfill@{\arrowfill@\relbar\relbar\rightarrow}
\def\leftrightarrowfill@{\arrowfill@\leftarrow\relbar\rightarrow}
\def\Leftarrowfill@{\arrowfill@\Leftarrow\Relbar\Relbar}
\def\Rightarrowfill@{\arrowfill@\Relbar\Relbar\Rightarrow}
\def\Leftrightarrowfill@{\arrowfill@\Leftarrow\Relbar\Rightarrow}
%    \end{macrocode}
%
%    \begin{macrocode}
\def\overarrow@#1#2#3{\vbox{\ialign{##\crcr#1#2\crcr
 \noalign{\nointerlineskip}$\m@th\hfil#2#3\hfil$\crcr}}}
\ams@renewcommand{\overrightarrow}{%
  \mathpalette{\overarrow@\rightarrowfill@}}
\ams@renewcommand{\overleftarrow}{%
  \mathpalette{\overarrow@\leftarrowfill@}}
\ams@newcommand{\overleftrightarrow}{%
  \mathpalette{\overarrow@\leftrightarrowfill@}}
%    \end{macrocode}
%
%    \begin{macrocode}
\def\underarrow@#1#2#3{%
 \vtop{\ialign{##\crcr$\m@th\hfil#2#3\hfil$\crcr
 \noalign{\nointerlineskip\kern1.3\ex@}#1#2\crcr}}}
\ams@newcommand{\underrightarrow}{%
  \mathpalette{\underarrow@\rightarrowfill@}}
\ams@newcommand{\underleftarrow}{%
  \mathpalette{\underarrow@\leftarrowfill@}}
\ams@newcommand{\underleftrightarrow}{%
  \mathpalette{\underarrow@\leftrightarrowfill@}}
%    \end{macrocode}
%
%    \begin{macrocode}
%\newcommand{\xrightarrow}[2][]{\ext@arrow 0359\rightarrowfill@{#1}{#2}}
\def\ext@arrow#1#2#3#4#5#6#7{%
  \mathrel{\mathop{%
%    \end{macrocode}
%    Measure the superscript and subscript.
%    \begin{macrocode}
    \setbox\z@\hbox{#5\displaystyle}%
    \setbox\tw@\vbox{\m@th
      \hbox{$\scriptstyle\mkern#3mu{#6}\mkern#4mu$}%
      \hbox{$\scriptstyle\mkern#3mu{#7}\mkern#4mu$}%
      \copy\z@
    }%
    \hbox to\wd\tw@{\unhbox\z@}}%
%    \end{macrocode}
%    We don't want to place an empty subscript since that will produce
%    too much blank space below the arrow.
%    \begin{macrocode}
  \limits
    \@ifnotempty{#7}{^{\if0#1\else\mkern#1mu\fi
                       #7\if0#2\else\mkern#2mu\fi}}%
    \@ifnotempty{#6}{_{\if0#1\else\mkern#1mu\fi
                       #6\if0#2\else\mkern#2mu\fi}}}%
}
%    \end{macrocode}
%
%    Some extensible arrows to serve as mathrels and taking
%    sub/superscripts. These commands are robust because they take an
%    optional argument.
%    \begin{macrocode}
\newcommand{\xrightarrow}[2][]{\ext@arrow 0359\rightarrowfill@{#1}{#2}}
\newcommand{\xleftarrow}[2][]{\ext@arrow 3095\leftarrowfill@{#1}{#2}}
%    \end{macrocode}
%
% %%%%%%%%%%%%%%%%%%%%%%%%%%%%%%%%%%%%%%%%%%%%%%%%%%%%%%%%%%%%%%%%%%%%%%
% \section{Array-related environments}
% \subsection{Remarks}
%
%    Because these environments can be nested within the equation
%    structures that allow \cn{tag}, there is some cross-influence in
%    the internal workings of the \cn{\\} command.
%
% \subsection{The \env{subarray} environment and \cn{substack} command}
%
%    The \cn{substack} command can be used to set subscripts
%    and superscripts that consist of several lines. Usage:
% \begin{verbatim}
% X_{\substack{a=1\\b=2}}
% \end{verbatim}
%
% \changes{v2.0}{1999/06/17}{Removed environment definitions: Sb, Sp}
%
% \begin{environment}{subarray}
%    The \env{subarray} environment makes a small-size array suitable
%    for use in a subscript or superscript. At the moment the supported
%    arguments are not the full possibilities of \env{array} but only
%    |c| or |l| for centered or left-aligned. And only one column.
%    \changes{v2.17i}{2020/09/23}{moved to version from lualatex-math for luatex}
%    \begin{macrocode}
\ifx\directlua\@undefined
\newenvironment{subarray}[1]{%
%    \end{macrocode}
%    Note: The predecessors of \env{subarray} (\env{Sb} and \env{Sp},
%    inherited from \amstex/) used \cs{vbox} instead of \cs{vcenter}.
%    But when a multiline subscript is placed in \cs{limits} position
%    \cs{vcenter} is no worse than \cs{vbox}, and when it is placed
%    in the \cs{nolimits} position (e.g., for an integral), \cs{vcenter}
%    provides clearly better positioning than \cs{vbox}.
%    \begin{macrocode}
  \vcenter\bgroup
%    \end{macrocode}
%    Use \cs{Let@} to set the proper meaning of the \cn{\\} and \cn{\\*}
%    commands. And restore the meaning of \cs{math@cr@@@} to \cs{cr}
%    (see above) in case \env{subarray} is used inside one of the more
%    complicated alignment macros where the meaning of \cs{math@cr@@@}
%    is different. Similarly, call \cs{default@tag} to ensure that a
%    line break here doesn't get an equation number!
%    \begin{macrocode}
  \Let@ \restore@math@cr \default@tag
%    \end{macrocode}
%    Set the line spacing to be the same as \cs{atop} (when \cs{atop}
%    occurs in \cs{textstyle} or smaller), cf \textit{The \tex/book},
%    Appendix G.
%    \begin{macrocode}
  \baselineskip\fontdimen10 \scriptfont\tw@
  \advance\baselineskip\fontdimen12 \scriptfont\tw@
  \lineskip\thr@@\fontdimen8 \scriptfont\thr@@
  \lineskiplimit\lineskip
%    \end{macrocode}
%    Start the \cs{vbox} \cs{halign} structure that encloses the
%    contents. Notice that we never get \cs{scriptscriptstyle}. That
%    would require a \cs{mathchoice} (ugh).
%    \begin{macrocode}
  \ialign\bgroup\ifx c#1\hfil\fi
    $\m@th\scriptstyle##$\hfil\crcr
}{%
  \crcr\egroup\egroup
}
\else
\newenvironment{subarray}[1]{%
  \vcenter\bgroup
  \Let@ \restore@math@cr \default@tag
  \baselineskip \Umathstacknumup \scriptstyle
  \advance\baselineskip \Umathstackdenomdown \scriptstyle
  \lineskip \Umathstackvgap \scriptstyle
  \lineskiplimit \lineskip
  \ialign\bgroup\ifx c#1\hfil\fi
  \Ustartmath
    \m@th\scriptstyle##
  \Ustopmath
  \hfil\crcr
}{%
  \crcr\egroup\egroup
}
\fi
%    \end{macrocode}
% \end{environment}
%
% \begin{macro}{\substack}
%    The \cn{substack} command is just an abbreviation for the
%    most common use of \env{subarray}.
%    \begin{macrocode}
\newcommand{\substack}[1]{\subarray{c}#1\endsubarray}
%    \end{macrocode}
% \end{macro}
%
% \subsection{Matrices}
%
% \begin{environment}{smallmatrix}
%    \env{smallmatrix} is again an alignment, this time in a centered
%    box. The opening incantations are basically the same as those in
%    \cs{multilimits@}, followed by the alignment itself. A remark:
%    the baselineskip (|9\ex@|) used in \amstex/ is too large for
%    use in text with the usual baselineskip of $12$ or $13$ points; we
%    change it here to |6\ex@| and also adjust the \cs{lineskip}
%    and \cs{lineskiplimit} slightly to compensate. (MJD)
%    \begin{macrocode}
\newenvironment{smallmatrix}{\null\,\vcenter\bgroup
  \Let@\restore@math@cr\default@tag
  \baselineskip6\ex@ \lineskip1.5\ex@ \lineskiplimit\lineskip
  \ialign\bgroup\hfil$\m@th\scriptstyle##$\hfil&&\thickspace\hfil
  $\m@th\scriptstyle##$\hfil\crcr
}{%
  \crcr\egroup\egroup\,%
}
%    \end{macrocode}
% \end{environment}
%
% \begin{environment}{matrix}
%    The \env{matrix} environment is just an \env{array} that provides
%    up to ten centered columns, so that users don't have to give the
%    col-spec argument explicitly---unless they want some of the columns
%    noncentered, that is. The maximum number of columns is actually not
%    fixed at ten but given by the counter |MatrixCols|, and can
%    therefore be increased by changing that counter.
%
%    The extra space of \cn{arraycolsep} that \env{array} adds on each
%    side is a waste so we remove it here (perhaps we should instead
%    remove it from \env{array} in general, but that's a harder task).
%
%    TODO: Think about re-implementing \cn{matrix} to get rid of the
%    \cs{c@MatrixCols} limit and have hard-wired preamble that doesn't
%    have to be rebuilt each time.
%
%    We must use \cn{renewenvironment} for \env{matrix} and
%    \env{pmatrix} because \latex/ doesn't kill the definitions found in
%    \fn{plain.tex}, even though it probably should because of their
%    foreign syntax.
%    \begin{macrocode}
\renewenvironment{matrix}{%
  \matrix@check\matrix\env@matrix
}{%
  \endarray \hskip -\arraycolsep
}
%    \end{macrocode}
% \end{environment}
%
%  \begin{macro}{\env@matrix}
%
%    \begin{macrocode}
\def\env@matrix{\hskip -\arraycolsep
  \let\@ifnextchar\new@ifnextchar
  \array{*\c@MaxMatrixCols c}}
%    \end{macrocode}
%  \end{macro}
%
% \begin{macro}{\c@MaxMatrixCols}
%    \begin{macrocode}
\newcount\c@MaxMatrixCols \c@MaxMatrixCols=10
%    \end{macrocode}
% \end{macro}
%
%  \begin{macro}{\matrix@check}
%    For various reasons, authors sometimes use the Plain \TeX{} form of
%    \cn{matrix} or \cn{pmatrix} in \LaTeX{} documents. If they later
%    add an invocation of the \pkg{amsmath} package to their document,
%    the Plain \TeX{} syntax would lead to rather unintelligible error
%    messages. The \cs{matrix@check} function does some checking to
%    forestall that problem.
%    \begin{macrocode}
\def\matrix@check#1{%
  \@xp\ifx\csname\@currenvir\endcsname#1%
  \else\matrix@error#1%
%    \end{macrocode}
%    This error recovery is not that good but is better than the
%    infinite loop that can result from calling \cs{array} without a
%    matching \cs{endarray}. (The array setup leaves \cs{par} empty.)
%    \begin{macrocode}
    \@xp\@gobble
  \fi
}
%    \end{macrocode}
%  \end{macro}
%
%  \begin{macro}{\matrix@error}
%
%    \begin{macrocode}
\def\matrix@error#1{%
  \@amsmath@err{%
Old form `\string#1' should be \string\begin{\@xp\@gobble\string#1}%
  }{%
`\string#1{...}' is old Plain-TeX syntax whose use is
ill-advised in LaTeX.%
  }%
}
%    \end{macrocode}
%  \end{macro}
%
%    \begin{macrocode}
\renewenvironment{pmatrix}{%
  \left(%
  \matrix@check\pmatrix\env@matrix
}{
  \endmatrix\right)%
}
\newenvironment{bmatrix}{\left[\env@matrix}{\endmatrix\right]}
\newenvironment{Bmatrix}{%
  \left\lbrace\env@matrix
}{%
  \endmatrix\right\rbrace
}
\newenvironment{vmatrix}{\left\lvert\env@matrix}{\endmatrix\right\rvert}
\newenvironment{Vmatrix}{\left\lVert\env@matrix}{\endmatrix\right\rVert}
%    \end{macrocode}
%
%    \begin{macrocode}
\let\hdots\@ldots
%    \end{macrocode}
%
%    \begin{macrocode}
\newcommand{\hdotsfor}[1]{%
  \ifx[#1\@xp\shdots@for\else\hdots@for\@ne{#1}\fi}
\newmuskip\dotsspace@
\def\shdots@for#1]{\hdots@for{#1}}
\def\hdots@for#1#2{\multicolumn{#2}c%
  {\m@th\dotsspace@1.5mu\mkern-#1\dotsspace@
   \xleaders\hbox{$\m@th\mkern#1\dotsspace@.\mkern#1\dotsspace@$}%
           \hfill
   \mkern-#1\dotsspace@}%
   }
%    \end{macrocode}
%
% \begin{environment}{cases}
%    The easiest way to produce the \env{cases} environment is to base
%    it on the \env{array} environment. We must use
%    \cn{renewenvironment} to override the definition of \cn{cases} that
%    \latex/ (unwisely) leaves in place from \fn{plain.tex}.
%    \begin{macrocode}
\renewenvironment{cases}{%
  \matrix@check\cases\env@cases
}{%
  \endarray\right.%
}
\def\env@cases{%
  \let\@ifnextchar\new@ifnextchar
  \left\lbrace
  \def\arraystretch{1.2}%
  \array{@{}l@{\quad}l@{}}%
}
%    \end{macrocode}
% \end{environment}
%
%
% %%%%%%%%%%%%%%%%%%%%%%%%%%%%%%%%%%%%%%%%%%%%%%%%%%%%%%%%%%%%%%%%%%%%%%
% \section{Equation sub-numbering}
%
%    \begin{macrocode}
\newcounter{parentequation}% Counter for ``parent equation''.
%    \end{macrocode}
%
%    We can't assume \cs{ignorespacesafterend} is defined since it was
%    not there in the earliest releases of \latex/ 2e. And we need to
%    include the \cs{global} for the same reason.
%    \begin{macrocode}
\@ifundefined{ignorespacesafterend}{%
  \def\ignorespacesafterend{\global\@ignoretrue}%
}{}
%    \end{macrocode}
%
%    \begin{environment}{subequations}
%    \begin{macrocode}
\newenvironment{subequations}{%
%    \end{macrocode}
%    Before sending down the `equation' counter to the subordinate
%    level, add 1 using standard \cn{refstepcounter}.
%    \begin{macrocode}
  \refstepcounter{equation}%
%    \end{macrocode}
%    Define \cn{theparentequation} equivalent to current
%    \cn{theequation}. \cn{edef} is necessary to expand the current
%    value of the equation counter. This might in rare cases cause
%    something to blow up, in which case the user needs to add
%    \cn{protect}.
%    \begin{macrocode}
  \protected@edef\theparentequation{\theequation}%
  \setcounter{parentequation}{\value{equation}}%
%    \end{macrocode}
%    And set the equation counter to 0, so that the normal incrementing
%    processes in the various equation environments will produce the
%    desired results.
%    \begin{macrocode}
  \setcounter{equation}{0}%
  \def\theequation{\theparentequation\alph{equation}}%
  \ignorespaces
}{%
  \setcounter{equation}{\value{parentequation}}%
  \ignorespacesafterend
}
%    \end{macrocode}
%    \end{environment}
%
% %%%%%%%%%%%%%%%%%%%%%%%%%%%%%%%%%%%%%%%%%%%%%%%%%%%%%%%%%%%%%%%%%%%%%%
% \section{Equation numbering}
%
%    In the multiline equation environments provided here, the task
%    of equation numbering is linked to the task of line breaking
%    in the sense that it is the \cn{\\} command that marks where an
%    equation number for the current line will be processed and added to
%    the page.
%
%    \begin{macro}{\numberwithin}
%    Provide a convenient way to specify that equations should be
%    numbered within sections.
%    \begin{macrocode}
\newcommand{\numberwithin}[3][\arabic]{%
  \@ifundefined{c@#2}{\@nocounterr{#2}}{%
    \@ifundefined{c@#3}{\@nocnterr{#3}}{%
      \@addtoreset{#2}{#3}%
      \@xp\xdef\csname the#2\endcsname{%
        \@xp\@nx\csname the#3\endcsname .\@nx#1{#2}}}}%
}
%    \end{macrocode}
%    \end{macro}
%
%    \begin{macro}{\eqref}
%    To make references to equation numbers easier, we provide
%    \cn{eqref}.  We almost don't need \cn{textup}, except that
%    \cs{tagform@} doesn't supply the italic correction.
%    \begin{macrocode}
\newcommand{\eqref}[1]{\textup{\tagform@{\ref{#1}}}}
%    \end{macrocode}
%    \end{macro}
%
% \subsection{Preliminary macros}
%
%    The following macros implement the \latex/ syntax for the
%    \cn{\\} command, i.e. the possibility to add an asterisk to
%    inhibit a page break, or an optional argument to denote additional
%    vertical space. They are modelled more or less after the
%    corresponding macros for \latex/'s \env{eqnarray} and \env{array}
%    environments.
%
%    [We can perhaps use the eqnarray mechanism if we change it so that
%    it also uses \cs{openup}.]
%
% \begin{macro}{\dspbrk@lvl}
%    We begin by defining the \cs{dspbrk@lvl} counter. This counter
%    records the desirability of a break after the current row, as a
%    number between $0$ and $4$. Its default value is $-1$ meaning that
%    no explicit \cn{displaybreak} command was given, and the default
%    \cs{interdisplaylinepenalty} is to be used.
%    \begin{macrocode}
\newcount\dspbrk@lvl
\dspbrk@lvl=-1
%    \end{macrocode}
% \end{macro}
%
% \begin{macro}{\interdisplaylinepenalty}
%    We set the \cs{interdisplaylinepenalty} to $10000$.
%    \begin{macrocode}
\interdisplaylinepenalty\@M
%    \end{macrocode}
% \end{macro}
%
% \begin{macro}{\allowdisplaybreaks}
%    The \cn{allowdisplaybreaks} command. Since this is intended for use
%    outside displayed formulas (typically in the preamble), it does not
%    need to use \cs{new@ifnextchar}.
%    \begin{macrocode}
\newcommand{\allowdisplaybreaks}[1][4]{%
  \interdisplaylinepenalty\getdsp@pen{#1}\relax
}
%    \end{macrocode}
% \end{macro}
%
% \begin{macro}{\getdsp@pen}
%    Modelled after \latex/'s \cs{@getpen}.  We use higher numbers
%    than would normally be provided by \cs{@lowpenalty},
%    \cs{@medpenalty}, and \cs{@highpenalty}, since display
%    breaks are almost always less desirable.
%    \begin{macrocode}
\def\getdsp@pen#1{%
  \ifcase #1\@M \or 9999 \or 6999 \or 2999 \or \z@\fi
}
%    \end{macrocode}
% \end{macro}
%
% \begin{macro}{\displaybreak}
% \begin{macro}{\dspbrk@}
% \begin{macro}{\dspbrk@context}
% \begin{macro}{\nogood@displaybreak}
%    For breaks in a certain row of a alignment.
%    \begin{macrocode}
\newcommand{\displaybreak}{\new@ifnextchar[\dspbrk@{\dspbrk@[4]}}
\chardef\dspbrk@context=\sixt@@n
%    \end{macrocode}
%
%    \begin{macrocode}
\def\dspbrk@[#1]{%
  \ifmeasuring@
  \else
    \ifcase\dspbrk@context % case 0 --- OK
      \global\dspbrk@lvl #1\relax
    \or                    % case 1 --- inside a box
      \nogood@displaybreak
    \else                  % other cases --- outside of a display
      \@amsmath@err{\Invalid@@\displaybreak}\@eha
    \fi
  \fi
}
%    \end{macrocode}
%
%    This is the value of \cn{displaybreak} when it occurs inside some
%    structure where it will not work.
%    \begin{macrocode}
\def\nogood@displaybreak{%
  \@amsmath@err{\protect
\displaybreak\space cannot be applied here}%
{One of the enclosing environments creates an
  unbreakable box\MessageBreak
(e.g., split, aligned, gathered, ...).}%
}
%    \end{macrocode}
% \end{macro}
% \end{macro}
% \end{macro}
% \end{macro}
%
% \begin{macro}{\math@cr}
%    The macro \cs{math@cr} ends a row inside one of the equation
%    environments, i.e., this is the internal name of the \cn{\\}
%    commands in these environments. As usual for this kind of macro
%    inside of alignments we insert a special brace into \tex/'s input
%    stream. The initial \cs{relax} is needed to trigger entry into the
%    \textit{u} template of the current column if the author ended the
%    current row with an empty column (i.e., the mathcr was immediately
%    preceded by an ampersand).
%    \begin{macrocode}
\def\math@cr{\relax\iffalse{\fi\ifnum0=`}\fi
%    \end{macrocode}
%    The first step is now to check whether an asterisk follows.
%    \cs{@eqpen} is used to hold the penalty value to be put on
%    the vertical list.
%    Then we call up \cs{math@cr@} which performs the next step.
%    If an asterisk is read page breaking is inhibited.
%    \begin{macrocode}
  \@ifstar{\global\@eqpen\@M\math@cr@}%
%    \end{macrocode}
%    Otherwise we have to check the \cs{dspbrk@lvl} value.
%    \begin{macrocode}
          {\global\@eqpen
             \ifnum\dspbrk@lvl <\z@ \interdisplaylinepenalty
              \else -\@getpen\dspbrk@lvl \fi
           \math@cr@}}
%    \end{macrocode}
% \end{macro}
%
% \begin{macro}{\math@cr@}
%    The purpose of \cs{math@cr@} is to check whether an optional
%    argument follows.  If not it provides \cs{z@} as default
%    value.
%    \begin{macrocode}
\def\math@cr@{\new@ifnextchar[\math@cr@@{\math@cr@@[\z@]}}
%    \end{macrocode}
% \end{macro}
%
% \begin{macro}{\math@cr@@}
%    \cs{math@cr@@} closes the special brace opened in
%    \cs{math@cr}, and calls \cs{math@cr@@@} which is supposed
%    the `real' row ending command.  The meaning of this macro depends
%    on the environment in which it is used.
%    \begin{macrocode}
\def\math@cr@@[#1]{\ifnum0=`{\fi \iffalse}\fi\math@cr@@@
%    \end{macrocode}
%    Finally we put the additional space onto the vertical list.
%    \begin{macrocode}
  \noalign{\vskip#1\relax}}
%    \end{macrocode}
% \end{macro}
%
% \begin{macro}{\Let@}
%    \cs{Let@} is called by all environments where \cn{\\}
%    ends a row of an alignment.
%    \begin{macrocode}
\def\Let@{\let\\\math@cr}
%    \end{macrocode}
% \end{macro}
%
% \begin{macro}{\restore@math@cr}
%    We mentioned already that the exact meaning of \cs{math@cr@@@}
%    depends on the current environment.  Since it is often a simple
%    \cs{cr} we provide \cs{restore@math@cr} to reset it.
%    \begin{macrocode}
\def\restore@math@cr{\def\math@cr@@@{\cr}}
%    \end{macrocode}
%    This is also the default case.
%    \begin{macrocode}
\restore@math@cr
%    \end{macrocode}
% \end{macro}
%
% \begin{macro}{\intertext}
% \begin{macro}{\intertext@}
%    The \cn{intertext} command is used for inserting text between the
%    rows of an alignment. It might better be done as an environment,
%    but the \cs{begingroup} from \cn{begin} would cause the
%    \cs{noalign} to fail.
%    \begin{macrocode}
\newcommand{\intertext}{\@amsmath@err{\Invalid@@\intertext}\@eha}
%    \end{macrocode}
%    \cs{intertext@} is called by all environments that allow the use of
%    the \cn{intertext} command.
%    \begin{macrocode}
\def\intertext@{%
  \def\intertext##1{%
%    \end{macrocode}
%    If current mode is not vmode, the most likely reason is that the
%    writer forgot the \cn{\\} that is supposed to precede
%    \cn{intertext}. All right, then, let's try adding it our ownself.
%    But, to be slightly careful: \cn{\\} does a futurelet, and it's
%    slightly dangerous to allow a letted token to barge around loose in
%    our internal code when it has been let to a conditional token
%    like \cs{fi}. So let's interpose something in front of the \cs{fi}
%    for the futurelet to take instead. (And careful again: it has to be
%    something evanescent, not (e.g.) \cs{relax} which would cause the
%    next halign cell to fire up and keep \cs{noalign} from working.)
%    \begin{macrocode}
    \ifvmode\else\\\@empty\fi
    \noalign{%
      \penalty\postdisplaypenalty\vskip\belowdisplayskip
      \vbox{\normalbaselines
%    \end{macrocode}
%    We need to do something extra if the outside environment is a list
%    environment. I don't see offhand an elegant way to test ``are we
%    inside any list environment'' that is both easy and reliable (for
%    example, checking for zero \cs{@totalleftmargin} wouldn't catch the
%    case where \cs{@totalleftmargin} is zero but \cs{linewidth} is less
%    than \cs{columnwidth}), so it seems to me checking \cs{linewidth}
%    is the best practical solution.
% \changes{v2.15c}{2016/05/23}{\cs{ignorespaces} at the start of the argument}
%    \begin{macrocode}
        \ifdim\linewidth=\columnwidth
        \else \parshape\@ne \@totalleftmargin \linewidth
        \fi
        \noindent\ignorespaces##1\par}%
      \penalty\predisplaypenalty\vskip\abovedisplayskip%
    }%
}}
%    \end{macrocode}
% \end{macro}
% \end{macro}
%
%
% \subsection{Implementing tags and labels}
%
%    In this section we describe some of the macros needed to make the
%    \cn{tag} command work in various places. We start by defining a
%    help text to be used when a \cn{tag} command is used somewhere
%    it should not appear.
%
% \begin{macro}{\tag@help}
%    This is the default error help text provided when \cn{tag}
%    generates an error message.
%    Note that \cs{newhelp} generates a control sequence name
%    from the string given as its argument so that a leading
%    backslash is provided automatically.
%    \begin{macrocode}
\newhelp\tag@help
  {tag cannot be used at this point.\space
   If you don't understand why^^Jyou should consult
   the documentation.^^JBut don't worry: just continue, and I'll
   forget what happened.}
%    \end{macrocode}
% \end{macro}
%
% \begin{macro}{\gobble@tag}
%    This macro is to be used when \cn{tag} should silently
%    skip its argument.
%    It is made to handle the \qc{\*}-form of \cn{tag} as well.
%    \begin{macrocode}
\def\gobble@tag{\@ifstar\@gobble\@gobble}
%    \end{macrocode}
% \end{macro}
%
% \begin{macro}{\invalid@tag}
%    \cs{invalid@tag} is a macro that should be used whenever
%    \cn{tag} appears in an illegal place.
%    It sets up \cs{tag@help} (as defined above) as help message,
%    prints its argument as error message, and skips \cn{tag}'s
%    argument.
%    \begin{macrocode}
\def\invalid@tag#1{\@amsmath@err{#1}{\the\tag@help}\gobble@tag}
%    \end{macrocode}
% \end{macro}
%
% \begin{macro}{\dft@tag}
% \begin{macro}{\default@tag}
%    \cs{dft@tag} provides a convenient way to disallow the
%    use of \cn{tag} at certain points.
%    One simply has to write
%    \begin{verbatim}
%\let\tag\dft@tag
%    \end{verbatim}
%    and the \cn{tag} command will produce an error message,
%    with a suitable error help text, and discard its argument.
%    \begin{macrocode}
\def\dft@tag{\invalid@tag{\string\tag\space not allowed here}}
%    \end{macrocode}
%    Since this is used several times we provide an abbreviation for
%    it.
%    \begin{macrocode}
\def\default@tag{\let\tag\dft@tag}
%    \end{macrocode}
%    Since this is also the default, i.e.\ the \cn{tag} command
%    should not be used except in special places, we issue a
%    \cs{default@tag} command.
%    \begin{macrocode}
\default@tag
%    \end{macrocode}
% \end{macro}
% \end{macro}
%
%    Now that we have taken care of the case that \cn{tag} is not
%    allowed we will provide some macros to process tags appropriately.
%    As the user documentation states, a \cn{tag} command (without
%    the asterisk typesets its argument according to the document
%    styles' conventions, whereas a \cn{tag*} command typesets its
%    argument exactly as given. We define therefore the following
%    interface:
%
% \begin{macro}{\maketag@@}
% \begin{macro}{\maketag@@@}
% \begin{macro}{\tagform@}
%    \cn{tag} is supposed to call \cs{maketag@@} which checks
%    whether an asterisk follows. If this is the case it calls up
%    \cs{maketag@@@} which sets its argument `as is'. Otherwise
%    \cs{tagform@} is called to do the job. (This macro is to be
%    defined appropriately by the document style.)
%    \begin{macrocode}
\def\maketag@@{\@ifstar\maketag@@@\tagform@}
%    \end{macrocode}
%    We define \cs{maketag@@@} to use the normal font of the document
%    text (since this is the usual practice for numbering of document
%    elements) and to put a box around the tag. Furthermore we use
%    \cs{m@th} for exceptional cases where the tag involves a
%    superscript or some such math. (Probably from an explicit use of
%    \cs{tag*} rather than from the automatic numbering.)
%    \begin{macrocode}
\def\maketag@@@#1{\hbox{\m@th\normalfont#1}}
%    \end{macrocode}
%    We use the following default definition for \cs{tagform@}
%    that puts only parentheses around the tag.
%    \begin{macrocode}
\def\tagform@#1{\maketag@@@{(\ignorespaces#1\unskip\@@italiccorr)}}
%    \end{macrocode}
% \end{macro}
% \end{macro}
% \end{macro}
%
%    We need to insinuate \cs{tagform@} into \cs{@eqnnum} in case
%    \env{eqnarray} is used (probably in a document that was originally
%    written without use of the \pkg{amsmath} package).
%    \begin{macrocode}
\iftagsleft@
  \def\@eqnnum{\hbox to1sp{}\rlap{\normalfont\normalcolor
    \hskip -\displaywidth\tagform@\theequation}}
\else
  \def\@eqnnum{{\normalfont\normalcolor \tagform@\theequation}}
\fi
%    \end{macrocode}
%
% \begin{macro}{\thetag}
%    Sometimes one needs to set a literal tag according to the rules of
%    the document style. To achieve this we provide the \cn{thetag}
%    command. It typesets its argument by calling \cs{tagform@} on
%    it.
%    \begin{macrocode}
\newcommand{\thetag}{\leavevmode\tagform@}
%    \end{macrocode}
% \end{macro}
%
% \begin{macro}{\df@tag}
% \begin{macro}{\make@df@tag}
% \begin{macro}{\make@df@tag@@}
% \begin{macro}{\make@df@tag@@@}
%    Sometimes it is necessary for a \cn{tag} command to store a tag
%    in a safe place and to process it later, e.g., for a tag in a row
%    of an alignment where the tag can only be typeset when the
%    \cn{\\} at the end of the row was seen. Such a tag is stored in
%    the macro \cs{df@tag} (for `deferred tag'). For this purpose we
%    provide the \cs{make@df@tag} macro. It is built very similar to
%    the \cs{maketag@@} macro above.
%    \begin{macrocode}
\let\df@tag\@empty
\def\make@df@tag{\@ifstar\make@df@tag@@\make@df@tag@@@}
%    \end{macrocode}
%    \cs{make@df@tag} sets \cs{@currentlabel} and defines
%    \cs{df@tag} appropriately.
%
%    To simplify the task of tracking \cs{tag} and \cs{label}
%    commands inside math display environments, we defer \cs{label}
%    commands until the tag is typeset, similar to the way that
%    \cs{tag}s themselves are deferred.  This allows arbitrary
%    placement of \cs{label} and \cs{tag} commands and also means we
%    only increment the \cs{equation} counter when we really need to,
%    thus avoiding the \cs{setb@ck} nonsense that used to be required.
%
%    \begin{macrocode}
\def\make@df@tag@@#1{%
  \gdef\df@tag{\maketag@@@{#1}\def\@currentlabel{#1}}}
%    \end{macrocode}
%    Autogenerated number:
%    \begin{macrocode}
\def\make@df@tag@@@#1{\gdef\df@tag{\tagform@{#1}%
  \toks@\@xp{\p@equation{#1}}\edef\@currentlabel{\the\toks@}}}
%    \end{macrocode}
%    \end{macro}
%    \end{macro}
%    \end{macro}
%    \end{macro}
%
%    \begin{macro}{\ltx@label}
%    \begin{macro}{\label@in@display}
%    \begin{macro}{\df@label}
%    Next, we store the default definition of \cs{label} in
%    \cs{ltx@label} and then define a new version of \cs{label} for use
%    in math display environments. \cs{label@in@display} merely issues a
%    warning message if there is already a pending label (which will be
%    discarded) and then stores the label in \cs{df@label}.
%    \begin{macrocode}
\let\ltx@label\label
%
\def\label@in@display{%
    \ifx\df@label\@empty\else
        \@amsmath@err{Multiple \string\label's:
            label '\df@label' will be lost}\@eha
    \fi
    \gdef\df@label
}
%    \end{macrocode}
%    In case there is an enumerate inside a minipage inside an equation,
%    we need to reset \cn{label} to its normal value:
%    \begin{macrocode}
\toks@\@xp{\@arrayparboxrestore \let\label\ltx@label}%
\edef\@arrayboxrestore{\the\toks@}
%    \end{macrocode}
%
%    \begin{macrocode}
\let\df@label\@empty
%    \end{macrocode}
%    \end{macro}
%    \end{macro}
%    \end{macro}
%
%    \begin{macro}{\make@display@tag}
%    Now we define a macro to process \cs{tag} and \cs{label} commands
%    in various display environments.  If the |@eqnsw| switch is set,
%    then we should supply an equation number; otherwise, if the
%    |@tag| switch is set, we should use the tag stored in
%    \cs{df@tag}.  Finally, we process any pending \cs{label}s.
%
%    TODO: Arguably, \cs{make@display@tag} should issue a warning
%    message if there is a \cs{label} but neither a tag nor an
%    equation number.  Also, it would probably be worthwhile to
%    explore whether \cs{iftag@} could be done away with and replaced
%    by checks to see if \cs{df@tag} is empty or not.
%    \begin{macrocode}
\def\make@display@tag{%
  \if@eqnsw \incr@eqnum \print@eqnum
  \else \iftag@ \df@tag \global\let\df@tag\@empty \fi
  \fi
%    \end{macrocode}
%    Need to check the \cs{ifmeasuring@} flag otherwise the \cs{write}
%    node from \cn{label} might be discarded in a temp box and clearing
%    \cs{df@label} will keep it from being reiterated on the real
%    typesetting pass.
%    \begin{macrocode}
  \ifmeasuring@
  \else
    \ifx\df@label\@empty
    \else
      \@xp\ltx@label\@xp{\df@label}%
      \global\let\df@label\@empty
    \fi
  \fi
}
%    \end{macrocode}
%    \end{macro}
%
%    Now we define the special versions of \cn{tag} used within the
%    \env{align} environments.
%
% \begin{macro}{\tag@in@align}
%    The \cn{tag} command may only appear once in a row of
%    an alignment.  Therefore we first check the switch |tag@|
%    that is set to false at the begin of every row.
%    If this switch is true a \cn{tag} was already given in this
%    row and we define \cs{next@} to expand to a call to
%    \cs{invalid@tag}.
%    \begin{macrocode}
\def\tag@in@align{%
    \relax
    \iftag@
        \DN@{\invalid@tag{Multiple \string\tag}}%
    \else
%    \end{macrocode}
%    Otherwise we set the |tag@| switch.  But there is more to
%    be done: we must also prevent the automatic generation of a
%    tag.  Therefore we also reset the |@eqnsw|.
%    \begin{macrocode}
    \global\tag@true
%    \end{macrocode}
%    Changed to \cs{nonumber}, since that seems to be all that's
%    required.---dmj, 1994/12/21
%    \begin{macrocode}
    \nonumber
%    \end{macrocode}
%    Within a row of an \env{align} environment the \cn{tag}
%    command must not typeset the tag immediately since its
%    position can be determined only later.
%    Therefore we use the \cs{make@df@tag} macro defined
%    earlier.
%    Finally we call \cs{next@} to process the argument
%    that follows.
%    \begin{macrocode}
        \let\next@\make@df@tag
    \fi
    \next@
}
%    \end{macrocode}
% \end{macro}
%
% \begin{macro}{\raisetag}
%    Usage: \cn{raisetag} \meta{dimen}
%
%    This will modify the vertical placement of the tag of the current
%    equation by \meta{dimen}. Note that according to the current uses
%    of \cs{raise@tag} in e.g., \cs{place@tag@gather}, no adjustment
%    occurs if the tag falls in its normal position; i.e., \cn{raisetag}
%    has no effect unless the tag has already been shifted off-line.
%
%    \begin{macrocode}
\newcommand{\raisetag}[1]{\skip@#1\relax
  \xdef\raise@tag{\vskip\iftagsleft@\else-\fi\the\skip@\relax}%
}
%    \end{macrocode}
%    \cn{raise@tag} will be reemptied at the beginning of each equation,
%    which might occur at a |\begin{xxx}| or \cn{\\}.
%    \begin{macrocode}
\let\raise@tag\@empty
%    \end{macrocode}
%    \end{macro}
%
% \begin{macro}{\notag}
%    For consistency we provide \cn{notag}, equivalent to
%    \cn{nonumber}. The alternative would have been to rename
%    \cn{tag} as \cn{number} to go along with \cn{nonumber},
%    but of course \cs{number} is a \tex/ primitive that should not
%    be redefined.
%    \begin{macrocode}
\newcommand{\notag}{\nonumber}
%    \end{macrocode}
% \end{macro}
%
%  \begin{macro}{\nonumber}
%    Need to add some additional code to \cn{nonumber} to deal with some
%    complications related to nested environments.
%    \begin{macrocode}
\renewcommand{\nonumber}{%
  \if@eqnsw
    \ifx\incr@eqnum\@empty \addtocounter{equation}\m@ne \fi
  \fi
  \let\print@eqnum\@empty \let\incr@eqnum\@empty
  \global\@eqnswfalse
}
%    \end{macrocode}
%
%    \begin{macrocode}
\def\print@eqnum{\tagform@\theequation}
\def\incr@eqnum{\refstepcounter{equation}\let\incr@eqnum\@empty}
%    \end{macrocode}
%  \end{macro}
%
% %%%%%%%%%%%%%%%%%%%%%%%%%%%%%%%%%%%%%%%%%%%%%%%%%%%%%%%%%%%%%%%%%%%%%%
% \section{Multiline equation environments}
%
% \subsection{Remarks}
%
%    In late 1994 David M. Jones did a thorough overhaul of these
%    environments so that the number placement and a few other aspects
%    are substantially improved over the original versions that were
%    ported essentially unchanged from \fn{amstex.tex} in 1989. Most of
%    the commentary in this section is DMJ's, and comments of any
%    significance that I added are marked by my initials and date
%    [mjd, 1995/01/11].
%
% \subsection{Preliminaries}
%
% \begin{macro}{\ifinalign@}
% \begin{macro}{\ifingather@}
%    We define two switches that are set to true in certain
%    alignments: |inalign@| and |ingather@| inside of
%    the \env{align} and \env{gather} environments.
%    These switches are needed to control certain actions that
%    depend on the surrounding conditions, more specifically:
%    on the setting already done by the surrounding environments.
%    \begin{macrocode}
\newif\ifinalign@
\newif\ifingather@
%    \end{macrocode}
% \begin{histnote}
% Removed the \cs{ifinany@} test [mjd,1999/06/28] since it was mainly
% used for the purpose now handled by \cs{spread@equation}.
% \changes{v1.2g}{1999/06/28}{Removed ifinany@}
% \end{histnote}
% \end{macro}
% \end{macro}
%
% \begin{macro}{\@arrayparboxrestore}
%    Here we must reset a few additional parameters.
%    \begin{macrocode}
\@xp\def\@xp\@arrayparboxrestore\@xp{\@arrayparboxrestore
  \ingather@false\inalign@false \default@tag
  \let\spread@equation\@spread@equation
  \let\reset@equation\@empty
  \def\print@eqnum{\tagform@\theequation}%
  \def\incr@eqnum{\refstepcounter{equation}\let\incr@eqnum\@empty}%
}
%    \end{macrocode}
% \end{macro}
%
% \begin{macro}{\iftag@}
%    The switch |tag@| is set to false at the beginning of every
%    row and set to true by a \cn{tag} command.
%    This allows us to check whether there is more than one tag on
%    a row.
%    \begin{macrocode}
\newif\iftag@
%    \end{macrocode}
% \end{macro}
%
% \begin{macro}{\ifst@rred}
%   The switch |st@rred| is set to true by all starred environments
%   and set to false by the unstarred versions.
%   One exception is the \env{xxalignat} environment where this is set
%   to true.
%    \begin{macrocode}
\newif\ifst@rred
%    \end{macrocode}
% \end{macro}
%
% \begin{macro}{\ifmeasuring@}
%    All display environments get typeset twice---once during a
%    ``measuring'' phase and then again during a ``production'' phase;
%    \cs{ifmeasuring@} will be used to determine which case we're in,
%    so we can take appropriate action.
%
%    \begin{macrocode}
\newif\ifmeasuring@
%    \end{macrocode}
% \end{macro}
%
%    \begin{macro}{\ifshifttag@}
%    \cs{ifshifttag@} is used by \env{gather} to communicate between
%    \cs{calc@shift@gather} and \cs{place@tag@gather} whether an
%    equation tag should be shifted to a separate line.  It's also
%    used by \env{multline}.
%    \begin{macrocode}
\newif\ifshifttag@
%    \end{macrocode}
%    \end{macro}
%
%    \begin{macro}{\row@}
%    \begin{macrocode}
\newcount\row@
%    \end{macrocode}
%    \end{macro}
%
% \begin{macro}{\column@}
%    The counter \cs{column@} is used by the alignment macros to
%    keep track of the current column.
%
%    \begin{macrocode}
\newcount\column@
%    \end{macrocode}
% \end{macro}
%
%  \begin{macro}{\column@plus}
%    \cs{\column@plus} is a useful abbreviation.
%    \begin{macrocode}
\def\column@plus{%
    \global\advance\column@\@ne
}
%    \end{macrocode}
%  \end{macro}
%
%    \begin{macro}{\maxfields@}
%    \begin{macrocode}
\newcount\maxfields@
%    \end{macrocode}
%    \end{macro}
%
%  \begin{macro}{\add@amp}
%  \begin{macro}{\add@amps}
%    \begin{macrocode}
\def\add@amp#1{\if m#1&\@xp\add@amp\fi}
\def\add@amps#1{%
    \begingroup
    \count@#1\advance\count@-\column@
    \edef\@tempa{\endgroup
      \@xp\add@amp\romannumeral\number\count@ 000q}%
    \@tempa
}
%    \end{macrocode}
%  \end{macro}
%  \end{macro}
%
% \begin{macro}{\andhelp@}
%    The help text stored in \cs{andhelp@} is used for errors
%    generated by too many \qc{\&} characters in a row.
%    \begin{macrocode}
\newhelp\andhelp@
{An extra & here is so disastrous that you should probably exit^^J
and fix things up.}
%    \end{macrocode}
% \end{macro}
%
%    \begin{macro}{\eqnshift@}
%    \cs{eqnshift@} is used by \env{align} and \env{gather} as the
%    indentation of the lines of the environment from the left margin.
%    \begin{macrocode}
\newdimen\eqnshift@
%    \end{macrocode}
%    \end{macro}
%
%    \begin{macro}{\alignsep@}
%    \begin{macrocode}
\newdimen\alignsep@
%    \end{macrocode}
%    \end{macro}
%
%    \begin{macro}{\tagshift@}
%    \begin{macrocode}
\newdimen\tagshift@
%    \end{macrocode}
%    \end{macro}
%
%    \begin{macro}{\mintagsep}
%    \cs{mintagsep} is the minimum allowable separation between an
%    equation and its tag.  We set it to half a quad in
%    \cs{textfont}2, which is \tex/'s built-in value.
%    \begin{macrocode}
\newcommand{\mintagsep}{.5\fontdimen6\textfont\tw@}
%    \end{macrocode}
%    \end{macro}
%
%    \begin{macro}{\minalignsep}
%    This should probably be a skip register [mjd,1999/06/18]
%    \begin{macrocode}
\newcommand{\minalignsep}{10pt}
%    \end{macrocode}
%    \end{macro}
%
%    \begin{macro}{\tagwidth@}
%    \begin{macrocode}
\newdimen\tagwidth@
%    \end{macrocode}
%    \end{macro}
%
%    \begin{macro}{\totwidth@}
%    \begin{macrocode}
\newdimen\totwidth@
%    \end{macrocode}
% \end{macro}
%
% \begin{macro}{\lineht@}
%    The dimen register \cs{lineht@} is used to keep track of the
%    height (or depth, if tags are on the right) of a row in an
%    alignment.
%    \begin{macrocode}
\newdimen\lineht@
%    \end{macrocode}
% \end{macro}
%
%    \begin{macro}{\tag@width}
%    \begin{macro}{\savetaglength@}
%    \begin{macro}{\shift@tag}
%    \begin{macro}{\tag@shifts}
%    \begin{macrocode}
\def\tag@width#1{%
    \ifcase\@xp#1\tag@lengths\fi
}

\def\savetaglength@{%
    \begingroup
        \let\or\relax
        \xdef\tag@lengths{\tag@lengths\or \the\wdz@}%
    \endgroup
}

\def\shift@tag#1{%
    \ifcase\@xp#1\tag@shifts\fi\relax
}

\let\tag@shifts\@empty
%    \end{macrocode}
%    \end{macro}
%    \end{macro}
%    \end{macro}
%    \end{macro}
%
%    \begin{macro}{\saveshift@}
%    \begin{macrocode}
\def\saveshift@#1{%
    \begingroup
        \let\or\relax
        \xdef\tag@shifts{\or#1\tag@shifts}%
    \endgroup
}
%    \end{macrocode}
%    \end{macro}
%
%  \begin{macro}{\spread@equation}
%    This does the line-spacing adjustment that is normally wanted for
%    displayed equations. We also call \cs{reset@strutbox@} here because
%    otherwise a preceding font size change might leave \cs{strutbox@}
%    with wrong contents. This is a less-than-ideal solution but
%    probably good enough for now, until the situation can be
%    overhauled.
%    \begin{macrocode}
\def\spread@equation{\reset@strutbox@
  \openup\jot \let\spread@equation\@empty}
\let\@spread@equation\spread@equation
%    \end{macrocode}
%  \end{macro}
%
% \begin{macro}{\displ@y}
% \begin{macro}{\displ@y@}
% \begin{macro}{\@display@init}
%    \cs{displ@y} is from \fn{plain.tex}, with
%    \cs{interdisplaylinepenalty} changed to \cs{@eqpen}. Also we
%    transplanted most of its internal organs to \cs{@display@init} to
%    support \cs{displ@y@} and other possibilities. Don't try to make
%    sense of these naming conventions! They are a narrowly calculated
%    mishmash of Knuth/Spivak/Lamport/Mittelbach precedents. The reason
%    for not cleaning them up and forcing all names to a consistent
%    scheme is that then in principle we'd have to do it everywhere else
%    too. And we programmers are paranoid about the side effects of name
%    changes.
%    \begin{macrocode}
\def\displ@y{\@display@init{}}
\def\@display@init#1{%
    \global\dt@ptrue \spread@equation
    \everycr{%
        \noalign{%
            #1%
            \ifdt@p
                \global\dt@pfalse
                \vskip-\lineskiplimit
                \vskip\normallineskiplimit
            \else
                \penalty\@eqpen \global\dspbrk@lvl\m@ne
            \fi
        }%
    }%
}
%    \end{macrocode}
%    \cs{displ@y@} is nearly the same; it additionally sets the |tag@|
%    switch and the \cs{column@} and \cs{dspbrk@lvl} counters to their
%    default values. The argument is normally a bit of code to empty out
%    \cs{raise@tag}, but in \env{multline} we don't want that to happen
%    in \cs{everycr}.
%    \begin{macrocode}
\def\displ@y@{\@display@init{%
  \global\column@\z@ \global\dspbrk@lvl\m@ne
  \global\tag@false \global\let\raise@tag\@empty
}}
%    \end{macrocode}
% \end{macro}
% \end{macro}
% \end{macro}
%
% \begin{macro}{\black@}
% \changes{v2.17a}{2017/09/02}{Add fixed width box so overfull warning
%       generated in centred environments in lists}
%    This macro is made to produce an overfull box message and
%    possibly (depending on the value of \cs{overfullrule})
%    a rule in the margin if the total width of an alignment
%    is larger than the value of \cs{displaywidth}.
%    \begin{macrocode}
\def\black@#1{%
    \noalign{%
        \ifdim#1>\displaywidth
            \dimen@\prevdepth
            \nointerlineskip
            \vskip-\ht\strutbox@
            \vskip-\dp\strutbox@
            \vbox{\noindent\hbox to\displaywidth{%
                           \hbox to#1{\strut@\hfill}}}%
            \prevdepth\dimen@
        \fi
    }%
}
%    \end{macrocode}
% \end{macro}
%
%    \begin{macro}{\savecounters@}
%    \begin{macro}{\restorecounters@}
%    These are used during the measuring phase of the various display
%    math environments to save and restore the values of all \latex/
%    counters. We make these local to a group, so nested environments
%    works.
%
%    Changed \cn{stepcounter} to |\csname c@...\endcsname| to avoid
%    overhead of ifundefined test [mjd, 1995/01/20].
%    \begin{macrocode}
\def\savecounters@{%
    \begingroup
        \def\@elt##1{%
          \global\csname c@##1\endcsname\the\csname c@##1\endcsname}%
        \xdef\@gtempa{%
            \cl@@ckpt
            \let\@nx\restorecounters@\@nx\@empty
        }%
    \endgroup
    \let\restorecounters@\@gtempa
}
%
\let\restorecounters@\@empty
%    \end{macrocode}
%    \end{macro}
%    \end{macro}
%
%    \begin{macro}{\savealignstate@}
%    \begin{macro}{\restorealignstate@}
%    These are used to save the values of various parameters that are
%    shared by \env{align} and \env{gather} when the former is used
%    inside the latter.
%    \begin{macrocode}
\def\savealignstate@{%
    \begingroup
        \let\or\relax
        \xdef\@gtempa{%
            \global\totwidth@\the\totwidth@
            \global\row@\the\row@
            \gdef\@nx\tag@lengths{\tag@lengths}%
            \let\@nx\restorealignstate@\@nx\@empty
        }%
    \endgroup
    \let\restorealignstate@\@gtempa
}

\let\restorealignstate@\@empty
%    \end{macrocode}
%    \end{macro}
%    \end{macro}
%
%  \begin{macro}{\savecolumn@}
%  \begin{macro}{\restorecolumn@}
%
%    \begin{macrocode}
\def\savecolumn@{%
  \edef\restorecolumn@{%
    \global\column@\number\column@
    \let\@nx\restorecolumn@\@nx\@empty
  }%
}
\let\restorecolumn@\@empty
%    \end{macrocode}
%  \end{macro}
%  \end{macro}
%
% \subsection{Scanning the environment's body}
%
%    Several of the math alignment macros must scan their body twice:
%    once to determine how wide the columns are and then to actually
%    typeset them. This means that we must collect all text in this body
%    before calling the environment macros.
%
%    \begin{macro}{\@envbody}
%    We start by defining a token register to contain the body.
%    \begin{macrocode}
\newtoks\@envbody
%    \end{macrocode}
% \end{macro}
%
%    \begin{macro}{\addto@envbody}
%    Then we define a macro to add something (i.e.\ its argument) to the
%    token register \cs{@envbody}.
%    \begin{macrocode}
\def\addto@envbody#1{\global\@envbody\@xp{\the\@envbody#1}}
%    \end{macrocode}
%    \end{macro}
%
%    \begin{macro}{\collect@body}
%    The macro \cs{collect@body} starts the scan for the |\end{...}|
%    command of the current environment. It takes a macro name as
%    argument. This macro is supposed to take the whole body of the
%    environment as its argument. For example, |\begin{align}| would
%    call |\collect@body\@align| if |@align#1{...}| is the macro that
%    sets the alignment with body \arg{1}.
%
%    \begin{macrocode}
\def\collect@body#1{%
  \@envbody{\@xp#1\@xp{\the\@envbody}}%
  \edef\process@envbody{\the\@envbody\@nx\end{\@currenvir}}%
  \@envbody\@emptytoks \def\begin@stack{b}%
%    \end{macrocode}
%
%    If we simply called \cs{collect@@body} directly,
%    the error message for a \cn{par} token (usually from a blank line)
%    would be
% \begin{verbatim}
% ! Paragraph ended before \collect@@body was complete.
% \end{verbatim}
%    But we use a little finesse to get a more intelligible error
%    message:
% \begin{verbatim}
% ! Paragraph ended before \multline* was complete.
% \end{verbatim}
%    In order to avoid using up csnames unnecessarily we use the actual
%    environment name as the name of the temporary function that is
%    \cs{let} to \cs{collect@@body}; but then in order to preserve the
%    theoretical possibility of nesting for environments that use
%    \cs{collect@body} (not currently required by any \pkg{amsmath}
%    environment [mjd,1999/06/23]), we do the \cs{let} inside a group.
%    \begin{macrocode}
  \begingroup
  \@xp\let\csname\@currenvir\endcsname\collect@@body
%    \end{macrocode}
%    This small twist eliminates the need for \cs{expandafter}'s in
%    \cs{collect@@body}.
%    \begin{macrocode}
  \edef\process@envbody{\@xp\@nx\csname\@currenvir\endcsname}%
  \process@envbody
}
%    \end{macrocode}
%  \end{macro}
%
%  \begin{macro}{\push@begins}
%    When adding a piece of the current environment's contents to
%    \cs{@envbody}, we scan it to check for additional \cn{begin}
%    tokens, and add a `b' to the stack for any that we find.
%    \begin{macrocode}
\def\push@begins#1\begin#2{%
  \ifx\end#2\else b\@xp\push@begins\fi
}
%    \end{macrocode}
% \end{macro}
%
% \begin{macro}{\collect@@body}
%    \cs{collect@@body} takes two arguments: the first will consist of
%    all text up to the next \cn{end} command, the second will be the
%    \cn{end} command's argument. If there are any extra \cn{begin}
%    commands in the body text, a marker is pushed onto a stack by the
%    \cs{push@begins} function. Empty state for this stack means that we
%    have reached the \cn{end} that matches our original \cn{begin}.
%    Otherwise we need to include the \cn{end} and its argument in the
%    material that we are adding to our environment body accumulator.
%
% \begin{histnote}
%    In a former implementation, the error messages resulting from a
%    typo in the environment name were unsatisfactory, because it was
%    matching of the environment name that was used to determine the end
%    of our environment body, instead of counting begin-end pairs.
%    Thanks to Lars Hellstr\"{o}m for a suggestion that led to this
%    improvement. [mjd,1999/06/23]
% \end{histnote}
%    \begin{macrocode}
\def\collect@@body#1\end#2{%
  \edef\begin@stack{\push@begins#1\begin\end \@xp\@gobble\begin@stack}%
  \ifx\@empty\begin@stack
    \endgroup
    \@checkend{#2}%
    \addto@envbody{#1}%
  \else
    \addto@envbody{#1\end{#2}}%
  \fi
  \process@envbody % A little tricky! Note the grouping
}
%    \end{macrocode}
% \end{macro}
%
%
% \subsection{Simple aligning environments}
%
%  \begin{macro}{\math@cr@@@aligned}
%    From tabskip we get an extra space of minalignsep after every
%    second column; but when this falls at the right edge of the whole
%    aligned structure, we need to cancel that space.
%    \begin{macrocode}
\def\math@cr@@@aligned{%
  \ifodd\column@ \let\next@\@empty
  \else \def\next@{&\kern-\alignsep@}%
  \fi
  \next@ \cr
}
%    \end{macrocode}
%  \end{macro}
%
%
%
%  \begin{macro}{\ams@start@box}
%    This macro tests the optional  positioning argument (in
%    \texttt{gathered} or \texttt{aligned}. It explicitly tests for the
%    value \texttt{b}, \texttt{c} and \texttt{t} and if the value is
%    different, then we assume that it is a bracket group that belongs
%    to the formula instead of being an misspelled optional argument.
%    (In earlier versions of the code anything other than \texttt{b}
%    or \texttt{t} was interpreted as \texttt{c} and the data was
%    otherwise dropped.)
% \changes{v2.17g}{2020/03/10}{Explicity test for b/t/c and return
%    optional argument is different (gh/5)}
%    \begin{macrocode}
\def\ams@start@box#1{%
%    \end{macrocode}
%    As we may pick up an arbitrary part of the formula by mistake, we
%    need to be very careful with the testing to avoid low-level
%    errors. This is why we use \cs{detokenize}. But we also need to
%    expand the argument (if possible) in case the position value is
%    hidden inside a macro. We therefore apply the \cs{romannumeral}
%    trick (known as f-expansion in \texttt{expl3}) in its old form.
%    The code assumes that
%    the default is correctly set up (which in this case is \texttt{c}).
%    \begin{macrocode}
  \edef\reserved@a{\csname ams@pos@\expandafter\detokenize
    \expandafter{\romannumeral-`\0#1}\endcsname}%
  \expandafter\ifx\reserved@a\relax
%    \end{macrocode}
%    If the argument is neither \texttt{b}, \texttt{c} or \texttt{t}
%    we save it in \cs{ams@return@opt@arg}, so it can later be
%    returned as part of the environment body. We could at this point
%    also issue a warning that bracket group was found at the start of
%    the formula and that it is safer to  add a \cs{relax} before it.
%    \begin{macrocode}
     \PackageWarning{amsmath}{%
       Bracket group \detokenize{[#1]} at formula start!\MessageBreak
       It could be a misspelled positional argument.\MessageBreak
       If it belongs to the formula add a \relax in\MessageBreak
        front to hide it}%
     \def\ams@return@opt@arg{[#1]}\vcenter
%    \end{macrocode}
%    If the argument was identified then we clear
%    \cs{ams@return@opt@arg} (just in case somebod ever nests these
%    environment.
%    \begin{macrocode}
  \else
     \let\ams@return@opt@arg\@empty\reserved@a
  \fi
}
%    \end{macrocode}
%
%  \begin{macro}{\ams@pos@t}
%  \begin{macro}{\ams@pos@b}
%  \begin{macro}{\ams@pos@c}
%
%    \begin{macrocode}
\def\ams@pos@t{\vtop}
\def\ams@pos@b{\vbox}
\def\ams@pos@c{\vcenter}
%    \end{macrocode}
%    And we accept an empty argument as a way to get the default (as
%    that was the case before as well, albeit by mistake in some sense).
%    \begin{macrocode}
\let\ams@pos@\ams@pos@c
%    \end{macrocode}
%  \end{macro}
%  \end{macro}
%  \end{macro}
%  \end{macro}
%
%    \begin{macro}{\start@aligned}
%    The \env{aligned} and \env{alignedat} environments are identical
%    except that the latter takes a mandatory argument to specify the
%    number of align structures, while the former allows any number of
%    align structures automatically (the use of \env{alignedat} is
%    deprecated).  So, they will be defined in terms of
%    \cs{start@aligned}, which will take two arguments.  The first
%    argument specifies the placement of the environments; it is
%    either |c|, |t|, or |b|.  The second is the number of align
%    structures; a value of~$-1$ means that an arbitrary number are
%    allowed.
%    \begin{macrocode}
\newcommand{\start@aligned}[2]{%
    \RIfM@\else
        \nonmatherr@{\begin{\@currenvir}}%
    \fi
    \savecolumn@ % Assumption: called inside a group
%    \end{macrocode}
%    The \cs{null} here is to keep the \cs{,} glue from causing the
%    invocation of the clause in \tex/'s built-in tag placement
%    algorithm that can cause an equation to be shifted all the way over
%    to the margin.
%    \begin{macrocode}
    \alignedspace@left
%    \end{macrocode}
%    Select the right kind of box based on the optional argument \verb=#1=.
%    \begin{macrocode}
      \ams@start@box{#1}\bgroup
        \maxfields@#2\relax
        \ifnum\maxfields@>\m@ne
            \multiply\maxfields@\tw@
%    \end{macrocode}
%    Introduced new \cs{math@cr@@@} so we can provide standard error
%    message for too many \qc{\&}'s in \env{alignedat}.
%    \begin{macrocode}
            \let\math@cr@@@\math@cr@@@alignedat
            \alignsep@\z@skip
        \else
            \let\math@cr@@@\math@cr@@@aligned
            \alignsep@\minalignsep
        \fi
%    \end{macrocode}
%    Reset the meaning of \cn{\\}.
%    \begin{macrocode}
        \Let@ \chardef\dspbrk@context\@ne
%    \end{macrocode}
%    Restore the default definition of \cn{tag} (error message), in
%    case \env{aligned} is used inside, e.g., a \env{gather}
%    environment that accepts \cn{tag}.
%    \begin{macrocode}
        \default@tag
        \spread@equation % no-op if already called
%    \end{macrocode}
%    Finally we start the alignment itself. For \env{aligned} we add
%    \cs{minalignsep} after every second column to mimic the
%    behavior of \env{align}. For \env{alignedat} the user has to
%    specify interalign space explicitly.
%    \begin{macrocode}
        \global\column@\z@
        \ialign\bgroup
           &\column@plus
            \hfil
            \strut@
            $\m@th\displaystyle{##}$%
            \tabskip\z@skip
           &\column@plus
            $\m@th\displaystyle{{}##}$%
            \hfil
            \tabskip\alignsep@
            \crcr
%    \end{macrocode}
%    If we picked up a bracket group by mistake here is the place to
%    return it for processing.
% \changes{v2.17g}{2020/03/10}{Explicity test for b/t/c and return
%    optional argument is different (gh/5)}
%    \begin{macrocode}
          \ams@return@opt@arg
}
%    \end{macrocode}
%    \end{macro}
%
%    \begin{macro}{\math@cr@@@alignedat}
%    \cs{math@cr@@@alignedat} checks to make sure the user hasn't put in
%    too many \qc{\&}s in \env{alignedat}.  Since \env{alignedat}
%    doesn't use \cs{displ@y@}, we also reset \cs{column@} here.  Note
%    than in \env{aligned}, \cs{column@} will increase without bound,
%    since it never gets reset, but this is harmless.
%    \begin{macrocode}
\def\math@cr@@@alignedat{%
    \ifnum\column@>\maxfields@
        \begingroup
          \measuring@false
          \@amsmath@err{Extra & on this line}%
            {\the\andhelp@}% "An extra & here is disastrous"
        \endgroup
    \fi
    \global\column@\z@
    \cr
}
%    \end{macrocode}
%    \end{macro}
%
%  \begin{macro}{\alignsafe@testopt}
%    Testing for an optional argument can be really, really tricky in
%    certain complicated contexts. This we discovered by getting some
%    bug reports for uses of \env{aligned}. So here is a safer
%    form of \latex/'s \cs{@testopt} function.
%    \begin{macrocode}
\def\alignsafe@testopt#1#2{%
  \relax\iffalse{\fi\ifnum`}=0\fi
  \@ifnextchar[%
    {\let\@let@token\relax \ifnum`{=\z@\fi\iffalse}\fi#1}%
    {\let\@let@token\relax \ifnum`{=\z@\fi\iffalse}\fi#1[#2]}%
}
%    \end{macrocode}
%  \end{macro}
%
% \begin{environment}{aligned}
%    The \env{aligned} environment takes an optional argument that
%    indicates its vertical position in relation to surrounding
%    material: |t|, |c|, or |b| for top, center, or bottom.
%    \begin{macrocode}
\newenvironment{aligned}{%
  \let\@testopt\alignsafe@testopt
  \aligned@a
}{%
  \crcr\egroup
  \restorecolumn@
  \egroup
}
\newcommand{\aligned@a}[1][c]{\start@aligned{#1}\m@ne}
%    \end{macrocode}
% \end{environment}
%
% \begin{environment}{alignedat}
%    To get a top or bottom positioned \env{alignedat} structure, you
%    would write something like
% \begin{verbatim}
% \begin{alignedat}[t]{3}
% \end{verbatim}
%
%    \begin{macrocode}
\newenvironment{alignedat}{%
  \let\@testopt\alignsafe@testopt
  \alignedat@a
}{%
  \endaligned
}
\newcommand{\alignedat@a}[1][c]{\start@aligned{#1}}
%    \end{macrocode}
% \end{environment}
%
% \begin{environment}{gathered}
%    The \env{gathered} environment is for several lines that are
%    centered independently.
%    \begin{macrocode}
\newenvironment{gathered}[1][c]{%
    \RIfM@\else
        \nonmatherr@{\begin{gathered}}%
    \fi
    \alignedspace@left
%    \end{macrocode}
%    Select the right kind of box based on the optional argument \verb=#1=.
%    \begin{macrocode}
    \ams@start@box{#1}\bgroup
        \Let@ \chardef\dspbrk@context\@ne \restore@math@cr
        \spread@equation
        \ialign\bgroup
            \hfil\strut@$\m@th\displaystyle##$\hfil
            \crcr
%    \end{macrocode}
%    And put a mistaking picked up bracket group back:
% \changes{v2.17g}{2020/03/10}{Explicity test for b/t/c and return
%    optional argument is different (gh/5)}
%    \begin{macrocode}
    \ams@return@opt@arg
}{%
  \endaligned
}
%    \end{macrocode}
% \end{environment}
%
%
%    \subsection{The \env{gather} environment}
%
% \begin{macro}{\start@gather}
%    \begin{macrocode}
\def\start@gather#1{%
    \RIfM@
        \nomath@env
        \DN@{\@namedef{end\@currenvir}{}\@gobble}%
    \else
        $$%
        #1%
        \ifst@rred \else \global\@eqnswtrue \fi
        \let\next@\gather@
    \fi
    \collect@body\next@
}
%    \end{macrocode}
% \end{macro}
%
%    \begin{environment}{gather}
%    \begin{environment}{gather*}
%    \begin{macrocode}
\newenvironment{gather}{%
  \start@gather\st@rredfalse
}{%
  \math@cr \black@\totwidth@ \egroup
  $$\ignorespacesafterend
}

\newenvironment{gather*}{%
  \start@gather\st@rredtrue
}{%
  \endgather
}
%    \end{macrocode}
%    \end{environment}
%    \end{environment}
%
%    \begin{macro}{\gather@}
%    \begin{macrocode}
\def\gather@#1{%
    \ingather@true \let\split\insplit@
    \let\tag\tag@in@align \let\label\label@in@display
    \chardef\dspbrk@context\z@
    \intertext@ \displ@y@ \Let@
    \let\math@cr@@@\math@cr@@@gather
    \gmeasure@{#1}%
    \global\shifttag@false
    \tabskip\z@skip
    \global\row@\@ne
    \halign to\displaywidth\bgroup
        \strut@
        \setboxz@h{$\m@th\displaystyle{##}$}%
        \calc@shift@gather
        \set@gather@field
        \tabskip\@centering
       &\setboxz@h{\strut@{##}}%
        \place@tag@gather
        \tabskip \iftagsleft@ \gdisplaywidth@ \else \z@skip \span\fi
        \crcr
        #1%
}
%    \end{macrocode}
%    \end{macro}
%
%    \begin{macro}{\gmeasure@}
%    \begin{macrocode}
\def\gmeasure@#1{%
    \begingroup
        \measuring@true
        \totwidth@\z@
        \global\let\tag@lengths\@empty
        \savecounters@
        \setbox\@ne\vbox{%
            \everycr{\noalign{\global\tag@false
              \global\let\raise@tag\@empty \global\column@\z@}}%
            \let\label\@gobble
            \halign{%
                \setboxz@h{$\m@th\displaystyle{##}$}%
                \ifdim\wdz@>\totwidth@
                    \global\totwidth@\wdz@
                \fi
               &\setboxz@h{\strut@{##}}%
                \savetaglength@
                \crcr
                #1%
                \math@cr@@@
            }%
        }%
        \restorecounters@
        \if@fleqn
            \global\advance\totwidth@\@mathmargin
        \fi
        \iftagsleft@
            \ifdim\totwidth@>\displaywidth
                \global\let\gdisplaywidth@\totwidth@
            \else
                \global\let\gdisplaywidth@\displaywidth
            \fi
        \fi
    \endgroup
}
%    \end{macrocode}
%    \end{macro}
%
%    \begin{macro}{\math@cr@@@gather}
%    Modified \cs{math@cr@@@gather} so that it always puts in the
%    final field, which needs to be done under the new method for
%    determining tag placement.  This is probably more efficient
%    anyway.
%
%    \begin{macrocode}
\def\math@cr@@@gather{%
    \ifst@rred\nonumber\fi
   &\relax
    \make@display@tag
    \ifst@rred\else\global\@eqnswtrue\fi
%    \end{macrocode}
%    We advance \cs{row@} here, rather than at the beginning of the
%    preamble, because otherwise the \env{split} environment will
%    cause \cs{row@} to be advanced twice instead of once.
%    \begin{macrocode}
    \global\advance\row@\@ne
    \cr
}
%    \end{macrocode}
%    \end{macro}
%
%    \begin{macro}{\calc@shift@gather}
%    \cs{calc@shift@gather} has must make two decisions: (1) whether the
%    equation tag for the current line should be put on a separate
%    line and (2) what the distance between the equation and the
%    equation tag should be.  We implement \tex/'s built-in
%    tag-placement as well as possible, with one improvement: the
%    minimum separation between tag and equation is now a
%    user-settable parameter.
%
%    [1995/01/17] Added a check to make sure that the width of the tag
%    on the current line is $>0$ before testing to see if tagwidth +
%    linewidth + mintagsep $>$ displaywidth. Since an imbedded align
%    shows up as line with width \cn{displaywidth}, and even lines
%    without a tag get processed as if an empty tag were present, the
%    result was that the empty tag assigned to the line containing the
%    align was being shifted downwards, creating extra space after the
%    align.
%    \begin{macrocode}
\def\calc@shift@gather{%
    \dimen@\mintagsep\relax
    \tagwidth@\tag@width\row@\relax
%    \end{macrocode}
%    If we're in \opt{fleqn} mode, there is no flexibility about
%    placement of the equation, so all we can do is see if there's
%    room for the tag in the given margin.
%    \begin{macrocode}
    \if@fleqn
        \global\eqnshift@\@mathmargin
        \ifdim\tagwidth@>\z@
            \advance\dimen@\tagwidth@
            \iftagsleft@
                \ifdim\dimen@>\@mathmargin
                    \global\shifttag@true
                \fi
            \else
                \advance\dimen@\@mathmargin
                \advance\dimen@\wdz@
                \ifdim\dimen@>\displaywidth
                   \global\shifttag@true
                \fi
            \fi
        \fi
    \else
        \global\eqnshift@\displaywidth
        \global\advance\eqnshift@-\wdz@
        \ifdim\tagwidth@>\z@
            \multiply\dimen@\tw@
            \advance\dimen@\wdz@
            \advance\dimen@\tagwidth@
            \ifdim\dimen@>\displaywidth
                \global\shifttag@true
            \else
                \ifdim\eqnshift@<4\tagwidth@
                    \global\advance\eqnshift@-\tagwidth@
                \fi
            \fi
        \fi
        \global\divide\eqnshift@\tw@
        \iftagsleft@
            \global\eqnshift@-\eqnshift@
            \global\advance\eqnshift@\displaywidth
            \global\advance\eqnshift@-\wdz@
        \fi
        \ifdim\eqnshift@<\z@
            \global\eqnshift@\z@
        \fi
    \fi
}
%    \end{macrocode}
%    \end{macro}
%
%    \begin{macro}{\place@tag@gather}
%    \begin{macro}{\set@gather@field}
%    \begin{macrocode}
\def\place@tag@gather{%
    \iftagsleft@
        \kern-\gdisplaywidth@
        \ifshifttag@
            \rlap{\vbox{%
                \normalbaselines
                \boxz@
                \vbox to\lineht@{}%
                \raise@tag
            }}%
            \global\shifttag@false
        \else
            \rlap{\boxz@}%
        \fi
    \else
        \ifdim\totwidth@>\displaywidth
            \dimen@\totwidth@
            \advance\dimen@-\displaywidth
            \kern-\dimen@
        \fi
        \ifshifttag@
            \llap{\vtop{%
                \raise@tag
                \normalbaselines
                \setbox\@ne\null
                \dp\@ne\lineht@
                \box\@ne
                \boxz@
            }}%
            \global\shifttag@false
        \else
            \llap{\boxz@}%
        \fi
    \fi
}
%
\def\set@gather@field{%
    \iftagsleft@
        \global\lineht@\ht\z@
    \else
        \global\lineht@\dp\z@
    \fi
    \kern\eqnshift@
    \boxz@
    \hfil
}
%    \end{macrocode}
%    \end{macro}
%    \end{macro}
%
%
% \subsection{The \env{align} family of environments}
%
%    The \env{align}, \env{flalign}, \env{alignat}, \env{xalignat},
%    and \env{xxalignat} environments are virtually
%    identical, and thus will share much code.  We'll refer to the
%    environments generically as ``\env{align}'' and will
%    distinguish between them explicitly
%    only when necessary.
%
% \begin{macro}{\ifxxat@}
% \begin{macro}{\ifcheckat@}
% \begin{macro}{\xatlevel@}
%    The \cs{xatlevel@} macro will be used, informally speaking, to
%    distinguish between the \env{alignat} and \env{xalignat}, and
%    \env{xxalignat} environments.
%
%    \begin{macrocode}
\newif\ifxxat@

\newif\ifcheckat@

\let\xatlevel@\@empty
%    \end{macrocode}
%    \end{macro}
%    \end{macro}
%    \end{macro}
%
% \begin{macro}{\start@align}
%    \cs{start@align} will be called by all of the \env{align}-like
%    environments.  The first argument will be the \cs{xatlevel@},
%    i.e., 0, 1, or~2; the second argument will be either
%    \cs{st@rredtrue} or \cs{st@rredfalse}.  The third argument will
%    be the number of aligned
%    structures in the environment (either as supplied by the user, or
%    $-1$ to indicate that checking shouldn't be done).  After
%    performing the appropriate error detection and initialization,
%    \cs{start@align} calls \cs{align@}.
%
%    Note that the \cs{equation} counter is no longer stepped at the
%    beginning of these environments.
%
%    TODO: Implement \cs{shoveleft} and \cs{shoveright} for
%    \env{align}.
%    \begin{macrocode}
\def\start@align#1#2#3{%
    \let\xatlevel@#1% always \z@, \@ne, or \tw@
    \maxfields@#3\relax
    \ifnum\maxfields@>\m@ne
        \checkat@true
        \ifnum\xatlevel@=\tw@
            \xxat@true
        \fi
        \multiply\maxfields@\tw@
    \else
        \checkat@false
    \fi
    \ifingather@
        \iffalse{\fi\ifnum0=`}\fi
        \DN@{\vcenter\bgroup\savealignstate@\align@#2}%
    \else
        \ifmmode
          \if@display
             \DN@{\align@recover}%
          \else
            \nomath@env
            \DN@{\@namedef{end\@currenvir}{}\@gobble}%
          \fi
        \else
            $$%
            \let\split\insplit@
            \DN@{\align@#2}%
        \fi
    \fi
    \collect@body\next@
}
%    \end{macrocode}
%
%    With version 1.2 of \pkg{amsmath}, it was possible to use
%    \env{align*} and relatives in certain wrong contexts without
%    getting an error, e.g.
% \begin{verbatim}
% \begin{equation*}
% \begin{align*}
% ...
% \end{align*}
% \end{equation*}
% \end{verbatim}
%
%    For backward compatibility we therefore give only a warning for
%    this condition instead of a full error, and try to recover using
%    the \env{aligned} environment. The alignment of the material may be
%    adversely affected but it will at least remain readable.
%    \begin{macrocode}
\def\align@recover#1#2#3{%
  \endgroup
  \@amsmath@err{%
Erroneous nesting of equation structures;\MessageBreak
trying to recover with `aligned'%
  }\@ehc
  \begin{aligned}\relax#1\end{aligned}%
}
%    \end{macrocode}
% \end{macro}
%
%    \begin{environment}{align}
%    \begin{environment}{align*}
%    \begin{environment}{flalign}
%    \begin{environment}{flalign*}
%    \begin{environment}{alignat}
%    \begin{environment}{alignat*}
%    \begin{environment}{xalignat}
%    \begin{environment}{xalignat*}
%    \begin{environment}{xxalignat}
%    The definitions of the various \env{align} environments are quite
%    straight-forward.
%
%    \begin{macrocode}
\newenvironment{alignat}{%
  \start@align\z@\st@rredfalse
}{%
  \endalign
}
\newenvironment{alignat*}{%
  \start@align\z@\st@rredtrue
}{%
  \endalign
}
\newenvironment{xalignat}{%
  \start@align\@ne\st@rredfalse
}{%
  \endalign
}
\newenvironment{xalignat*}{%
  \start@align\@ne\st@rredtrue
}{%
  \endalign
}
\newenvironment{xxalignat}{%
  \start@align\tw@\st@rredtrue
}{%
  \endalign
}
\newenvironment{align}{%
  \start@align\@ne\st@rredfalse\m@ne
}{%
  \math@cr \black@\totwidth@
  \egroup
  \ifingather@
    \restorealignstate@
    \egroup
    \nonumber
    \ifnum0=`{\fi\iffalse}\fi
  \else
    $$%
  \fi
  \ignorespacesafterend
}
\newenvironment{align*}{%
  \start@align\@ne\st@rredtrue\m@ne
}{%
  \endalign
}
\newenvironment{flalign}{%
  \start@align\tw@\st@rredfalse\m@ne
}{%
  \endalign
}
\newenvironment{flalign*}{%
  \start@align\tw@\st@rredtrue\m@ne
}{%
  \endalign
}
%    \end{macrocode}
%    \end{environment}
%    \end{environment}
%    \end{environment}
%    \end{environment}
%    \end{environment}
%    \end{environment}
%    \end{environment}
%    \end{environment}
%    \end{environment}
%
%
%  \begin{macro}{\align@}
%    TODO: Some of these sets of initializations show up in multiple
%    places.  It might be worth making an abbreviation for them.
%
%    \begin{macrocode}
\def\align@#1#2{%
    \inalign@true \intertext@ \Let@ \chardef\dspbrk@context\z@
    \ifingather@\else\displ@y@\fi
    \let\math@cr@@@\math@cr@@@align
    \ifxxat@\else \let\tag\tag@in@align \fi
    \let\label\label@in@display
    #1% set st@r
    \ifst@rred\else \global\@eqnswtrue \fi
    \measure@{#2}%
    \global\row@\z@
    \tabskip\eqnshift@
    \halign\bgroup
        \span\align@preamble\crcr
        #2%
}
%    \end{macrocode}
%    \end{macro}
%
%    \begin{macro}{\math@cr@@@align}
%    \begin{macrocode}
\def\math@cr@@@align{%
  \ifst@rred\nonumber\fi
  \if@eqnsw \global\tag@true \fi
  \global\advance\row@\@ne
  \add@amps\maxfields@
  \omit
  \kern-\alignsep@
  \iftag@
    \setboxz@h{\@lign\strut@{\make@display@tag}}%
    \place@tag
  \fi
  \ifst@rred\else\global\@eqnswtrue\fi
  \global\lineht@\z@
  \cr
}
%    \end{macrocode}
%    \end{macro}
%
%    \begin{macro}{\math@cr@@@align@measure}
%    \begin{macrocode}
\def\math@cr@@@align@measure{%
   &\omit
    \global\advance\row@\@ne
    \ifst@rred\nonumber\fi
    \if@eqnsw \global\tag@true \fi
    \ifnum\column@>\maxfields@
        \ifcheckat@
            \begingroup
              \measuring@false
              \@amsmath@err{Extra & on this line}%
                {\the\andhelp@}% "An extra & here is disastrous"
            \endgroup
        \else
            \global\maxfields@\column@
        \fi
    \fi
    \setboxz@h{\@lign\strut@{%
        \if@eqnsw
            \stepcounter{equation}%
            \tagform@\theequation
        \else
            \iftag@\df@tag\fi
        \fi
    }}%
    \savetaglength@
    \ifst@rred\else\global\@eqnswtrue\fi
    \cr
}
%    \end{macrocode}
% \end{macro}
%
%    \begin{macro}{\field@lengths}
%    \begin{macro}{\savefieldlength@}
%    \begin{macro}{\fieldlengths@}
%    \begin{macrocode}
\let\field@lengths\@empty

\def\savefieldlength@{%
    \begingroup
        \let\or\relax
        \xdef\field@lengths{%
            \field@lengths
            \ifnum\column@=0
                \or
            \else
                ,%
            \fi
            \the\wdz@
        }%
    \endgroup
}

\def\fieldlengths@#1{%
    \ifcase\@xp#1\field@lengths\fi
}
%    \end{macrocode}
%    \end{macro}
%    \end{macro}
%    \end{macro}
%
%  \begin{macro}{\maxcolumn@widths}
%    \cs{maxcolumn@widths} will be used to hold the widths of the
%    fields of the \env{alignat} environment.  The widths will be
%    separated by the token \cn{or}, making it easy to extract a given
%    width using \cn{ifcase}.
%    \begin{macrocode}
\let\maxcolumn@widths\@empty
%    \end{macrocode}
%  \end{macro}
%
%  \begin{macro}{\maxcol@width}
% \cs{maxcol@width} $n$ = maximum width of $n$th column of the current
%    \env{alignat} (i.e., the $n$th field of \cs{maxcolumn@widths}.)
%    It expands to a \<dimen>, so it can be used as the right-hand
%    side of a \<variable assignment> or \<arithmetic> statement.
%    It's argument can be any \<number>, \<integer variable> or macro
%    that expands to one of these.  [Check to make sure this is true.]
%
%    This is subtler than it looks.
%    \begin{macrocode}
\def\maxcol@width#1{%
    \ifcase\@xp#1\maxcolumn@widths\fi\relax
}
%    \end{macrocode}
%  \end{macro}
%
%    Now comes the real fun.  A typical \env{align} environments looks
%    something like this, where the vertical bars mark the edges of
%    the fields of the underlying \cs{halign}:
%    \[
%    \makeatletter\tabskip\@centering\offinterlineskip
%    \halign to\displaywidth{%^^A
%       &$\strut$\vrule\hfil$\m@th\displaystyle{\@lign#}$\vrule
%      \tabskip1pt&\vrule$\m@th\displaystyle{\@lign#}$\hfil\vrule
%      \tabskip\@centering\cr
%    \omit\small\hfil 1\hfil &\omit\small\hfil 2 &\omit\small\hfil
%    3\hfil &\omit\small\hfil 4\hfil & \omit\small\hfil 5\hfil
%    &\omit\small\hfil 6\hfil\cr
%    \noalign{\vskip8pt\relax}
%    V_i + q_i v_j & =v_i , &  X_i & = x_i - q_i x_j,
%     &  U_i & = u_i,\qquad
%      \hbox{for $i\ne j$;} %&\omit\hfill \llap{(3)}
%    \cr
%    V_j & = v_j, &  X_j & = x_j,
%      &  U_j & = u_j + \sum_{i\ne j} q_i u_i. %&\omit\hfill \llap{(4)}
%    \cr}
%    \]
%    Note that each align structure consists of two fields, with no
%    space between them (a small space has been added here to
%    highlight the boundaries).  Furthermore, the text inside the
%    odd-numbered fields is flushright, while the text inside the
%    even-numbered fields is flushleft.  The equation tags (shown on
%    the right here) can be on either the right or the left.  If there
%    is not room (in a sense to be defined shortly) for the tag on the
%    same line as the equation, the tag will be shifted to a separate
%    line.
%
%    Each environment also has a certain number of ``flexible
%    spaces,'' meaning spaces whose width we are allowed to adjust to
%    take up the amount of ``free space'' in the line, meaning the
%    space not taken up by the equation tag and the fields of the
%    underlying \cs{halign}.
%
%    The flexible spaces come in two flavors: interalign spaces and
%    margin spaces.  If there are $n$ align structures ($n=3$ in the
%    illustration above), there are $n-1$ interalign spaces, unless we
%    are in an \env{alignat} environment, in which case there are no
%    flexible interalign spaces.
%
%    The number of margin spaces is a little more complicated:
%    Normally, there are two, but if we're in \opt{fleqn} mode, there
%    is only one.  Furthermore, if we're in an \env{xxalignat} or
%    \env{flalign} environment (corresponding to $\cs{xatlevel@} = 2$,
%    then there are no flexible margin spaces.
%
%    Calculating the interalign and margin spaces is done in two
%    stages.
%
%    First, the total amount of free space is divided uniformly among
%    all the flexible spaces, without regard for the lengths of the tags
%    on the various lines. For the non-\opt{fleqn} case, this
%    corresponds to centering the align structures between the margins.
%    Note that in \opt{fleqn} mode, the right margin is still allowed to
%    be larger than \cs{@mathmargin}. This introduces an element of
%    asymmetry into the appearance of the environment, but it has the
%    advantage of leaving more space for equation tags in the right
%    margin. If the right margin were constrained to be equal to the
%    left margin in this case, tags would need to be shifted to a
%    separate line more often than would be desirable.
%
%    Ordinarily, all flexible spaces will be given the same width.
%    However, this is not invariably true, since the interalign spaces
%    are constrained to be at least \cs{minalignsep} wide, while---in
%    the absence of equation tags, at least---the margin spaces are
%    allowed to shrink to zero.  As we shall see in a minute, if there
%    are tags in the environment, then the margins are also bounded
%    below by \cs{mintagsep}.
%
%    Next, we examine each line of the environment that has a tag to
%    see if there is a gap of at least \cs{mintagsep} between the
%    equation and its tag.  If there isn't, we attempt to center the
%    equation between the tag and the opposite margin, leaving a gap
%    of at least \cs{mintagsep} on either side, in order to preserve
%    some symmetry, i.e., we want the equation to \emph{look} like
%    it's centered between the margin and the tag, so we don't want
%    the margin space to be less than the gap between the tag and the
%    equation.  (Arguably, it would be better to allow the margin
%    space to shrink to zero in this case in order to avoid shifting
%    the tag to a separate line at any cost, but that would require
%    all of our calculations to be a little more complicated and hence
%    a little slower.)  Finally, if no values of the interalign spaces
%    and the margins (with the constraints outlined above) will
%    produce an acceptable distance between the equation and its tag,
%    then the tag will be shifted to a separate line.
%
%  \begin{macro}{\measure@}
%    \cs{measure@} collects the various bits of information that we'll
%    need to perform the calculations outlined above, namely, the
%    number of align structures in the environment, the natural
%    lengths of the fields on each row, the maximum widths of each
%    column, and the widths of the equation tags on each line.  It
%    also calculates the number of flexible interalign and margin
%    spaces and computes the initial values of the parameters
%    \cs{eqnshift@} and \cs{alignsep@}, which correspond to the widths
%    of the margins and the interalign spaces, respectively.
%    \begin{macrocode}
\def\measure@#1{%
    \begingroup
        \measuring@true
        \global\eqnshift@\z@
        \global\alignsep@\z@
        \global\let\tag@lengths\@empty
        \global\let\field@lengths\@empty
        \savecounters@
        \global\setbox0\vbox{%
            \let\math@cr@@@\math@cr@@@align@measure
            \everycr{\noalign{\global\tag@false
              \global\let\raise@tag\@empty \global\column@\z@}}%
            \let\label\@gobble
            \global\row@\z@
            \tabskip\z@
            \halign{\span\align@preamble\crcr
                #1%
                \math@cr@@@
                \global\column@\z@
                \add@amps\maxfields@\cr
            }%
        }%
        \restorecounters@
%    \end{macrocode}
%    It's convenient to have \cs{maxfields@} rounded up to the nearest
%    even number, so that \cs{maxfields@} is precisely twice the
%    number of align structures.
%    \begin{macrocode}
        \ifodd\maxfields@
            \global\advance\maxfields@\@ne
        \fi
%    \end{macrocode}
%    It doesn't make sense to have a single align structure in either
%    \env{flalign} or \env{xxalignat}.  So, we check for that case now
%    and, if necessary, switch to an \env{align} or \env{alignat}.
%    Arguably, we should issue a warning message, but why bother?
%    \begin{macrocode}
        \ifnum\xatlevel@=\tw@
            \ifnum\maxfields@<\thr@@
                \let\xatlevel@\z@
            \fi
        \fi
%    \end{macrocode}
%    |\box0| now contains the lines of the \cs{halign}.  After the
%    following maneuver, |\box1| will contain the last line of the
%    \cs{halign}, which is what we're interested in.  (Incidentally,
%    the penalty we're removing is the \cs{@eqpen} inserted by
%    \cs{math@cr}.  Normally, this is \cs{interdisplaylinepenalty},
%    unless the user has overridden that with a \cs{displaybreak}
%    command.)
%    \begin{macrocode}
        \setbox\z@\vbox{%
          \unvbox\z@ \unpenalty \global\setbox\@ne\lastbox
        }%
%    \end{macrocode}
%    |\box1| begins with \cs{tabskip} glue and contains alternating
%    \cs{hbox}es (the fields whose widths we're trying to get) and
%    \cs{tabskip} glue [need better diagram]:
%    \begin{verbatim}
%    \hbox{\tabskip\hbox\tabskip...\hbox\tabskip}\end{verbatim}
%    In fact, all the \cs{tabskip} glue will be 0pt, because all the
%    \cs{tabskip}s in an \env{alignat} environment have a natural
%    width of 0pt, and the \cs{halign} has been set in its natural
%    width.
%
%    One nice result of this is that we can read \cs{totwidth@} off
%    immediately, since it is just the width of |\box1|, plus
%    \cs{@mathmargin} if we're in \opt{fleqn} mode.  (Actually, we
%    also have to take \cs{minalignsep} into account, but we'll do
%    that later):
%    \begin{macrocode}
        \global\totwidth@\wd\@ne
        \if@fleqn \global\advance\totwidth@\@mathmargin \fi
%    \end{macrocode}
%    Now we initialize \cs{align@lengths} and start peeling the boxes
%    off, one by one, and adding their widths to \cs{align@lengths}.
%    We stop when we run out of boxes, i.e., when \cs{lastbox} returns
%    a void box.  We're going to build a list using \cs{or} as a
%    delimiter, so we want to disable it temporarily.
%    \begin{macrocode}
        \global\let\maxcolumn@widths\@empty
        \begingroup
          \let\or\relax
          \loop
            \global\setbox\@ne\hbox{%
              \unhbox\@ne \unskip \global\setbox\thr@@\lastbox
            }%
          \ifhbox\thr@@
           \xdef\maxcolumn@widths{ \or \the\wd\thr@@ \maxcolumn@widths}%
          \repeat
        \endgroup
%    \end{macrocode}
%    Now we calculate the number of flexible spaces and the initial
%    values of \cs{eqnshift@} and \cs{alignsep@}.
%    We start by calculating $\cs{displaywidth}-\cs{totwidth@}$,
%    which gives us the total amount of ``free space'' in a row.
%    \begin{macrocode}
        \dimen@\displaywidth
        \advance\dimen@-\totwidth@
%    \end{macrocode}
%    Next we calculate the number of columns of flexible spaces in the
%    display, which depends on whether we're in \opt{fleqn} mode and
%    in which particular environment we are in.
%
%    We use \cs{@tempcnta} to store the total number of flexible spaces
%    in the align and \cs{@tempcntb} for the number of interalign
%    spaces.
%    \begin{macrocode}
        \ifcase\xatlevel@
%    \end{macrocode}
%    In \env{alignat}, the interalign spaces are under user control,
%    not ours.  So, we set \cs{alignsep@} and \cs{minalignsep} both
%    equal to 0pt.  Later, when calculating a new value for
%    \cs{alignsep@}, we will only save the new value if it is less
%    than the current value of \cs{alignsep@} (i.e., \cs{alignsep@}
%    will never increase).  Since the values we calculate will never
%    be negative, this will ensure that \cs{alignsep@} remains zero in
%    \env{alignat}.
%    \begin{macrocode}
            \global\alignsep@\z@
            \let\minalignsep\z@
            \@tempcntb\z@
%    \end{macrocode}
%    In \opt{fleqn} mode, the left margin---and hence the right margin
%    in this case---is fixed.  Otherwise, we divide the free space
%    equally between the two margins.
%    \begin{macrocode}
            \if@fleqn
                \@tempcnta\@ne
                \global\eqnshift@\@mathmargin
            \else
                \@tempcnta\tw@
                \global\eqnshift@\dimen@
                \global\divide\eqnshift@\@tempcnta
            \fi
        \or
%    \end{macrocode}
%    In an \env{align} or \env{xalignat} environment with $n$ aligned
%    structures, there are $n-1$ interalign spaces and either 1 or~2
%    flexible margins, depending on whether we're in \opt{fleqn} mode
%    or not.
%    \begin{macrocode}
            \@tempcntb\maxfields@
            \divide\@tempcntb\tw@
            \@tempcnta\@tempcntb
            \advance\@tempcntb\m@ne
%    \end{macrocode}
%    If we are in \opt{fleqn} mode, we fix the left margin and divide
%    the free space equally among the interalign spaces and the right
%    margin.
%    \begin{macrocode}
            \if@fleqn
                \global\eqnshift@\@mathmargin
                \global\alignsep@\dimen@
                \global\divide\alignsep@\@tempcnta
            \else
%    \end{macrocode}
%    Otherwise, we divide the free space equally among the interalign
%    spaces and both margins.
%    \begin{macrocode}
                \global\advance\@tempcnta\@ne
                \global\eqnshift@\dimen@
                \global\divide\eqnshift@\@tempcnta
                \global\alignsep@\eqnshift@
            \fi
        \or
%    \end{macrocode}
%    Finally, if we're in an \env{flalign} or \env{xxalignat}
%    environment, there are no flexible margins and $n-1$ flexible
%    interalign spaces.
%    \begin{macrocode}
            \@tempcntb\maxfields@
            \divide\@tempcntb\tw@
            \global\advance\@tempcntb\m@ne
            \global\@tempcnta\@tempcntb
            \global\eqnshift@\z@
            \global\alignsep@\dimen@
%    \end{macrocode}
%    If we're in \opt{fleqn} mode, we need to add back the
%    \cs{@mathmargin} that was removed when \cs{dimen@} was originally
%    calculated above.
%    \begin{macrocode}
            \if@fleqn
                \global\advance\alignsep@\@mathmargin\relax
            \fi
            \global\divide\alignsep@\@tempcntb
        \fi
%    \end{macrocode}
%    Now we make sure \cs{alignsep@} isn't too small.
%    \begin{macrocode}
        \ifdim\alignsep@<\minalignsep\relax
            \global\alignsep@\minalignsep\relax
            \ifdim\eqnshift@>\z@
                \if@fleqn\else
                    \global\eqnshift@\displaywidth
                    \global\advance\eqnshift@-\totwidth@
                    \global\advance\eqnshift@-\@tempcntb\alignsep@
                    \global\divide\eqnshift@\tw@
                \fi
            \fi
        \fi
        \ifdim\eqnshift@<\z@
            \global\eqnshift@\z@
        \fi
        \calc@shift@align
%    \end{macrocode}
%    Next, we calculate the value of \cs{tagshift@}.  This is the glue
%    that will be inserted in front of the equation tag to make sure
%    it lines up flush against the appropriate margin.
%    \begin{macrocode}
        \global\tagshift@\totwidth@
        \global\advance\tagshift@\@tempcntb\alignsep@
        \if@fleqn
            \ifnum\xatlevel@=\tw@
                \global\advance\tagshift@-\@mathmargin\relax
            \fi
        \else
            \global\advance\tagshift@\eqnshift@
        \fi
        \iftagsleft@ \else
            \global\advance\tagshift@-\displaywidth
        \fi
%    \end{macrocode}
%    Finally, we increase \cs{totwidth@} by an appropriate multiple of
%    \cs{minalignsep}.  If the result is greater than
%    \cs{displaywidth}, it means that at least one line in the
%    \env{align} is overfull and we will issue an appropriate warning
%    message (via \cs{bl@ck}) at the end of the environment.
%    \begin{macrocode}
        \dimen@\minalignsep\relax
        \global\advance\totwidth@\@tempcntb\dimen@
        \ifdim\totwidth@>\displaywidth
            \global\let\displaywidth@\totwidth@
        \else
            \global\let\displaywidth@\displaywidth
        \fi
    \endgroup
}
%    \end{macrocode}
%  \end{macro}
%
%    The code for calculating the appropriate placement of equation
%    tags in the \env{align} environments is quite complicated and
%    varies wildly depending on the settings of the |tagsleft@| and
%    |@fleqn| switches.  To minimize memory and hash space usage, we
%    only define the variant appropriate for the current setting of
%    those switches.
%
%    It would be worthwhile to examine this code more closely someday
%    and see if it could be optimized any.
%
%    \paragraph{Tag placement when \cs{tagsleft@true},
%    \cs{@fleqntrue}.}
%
%    We begin with the version of \cs{calc@shift@align} appropriate
%    for flush-left equations with tags on the left.
%
%    \begin{macro}{\calc@shift@align}
%    This is the simplest case.  Since the left margin is fixed, in
%    general the only thing to do is check whether there is room for
%    the tag in the left margin.  The only exception is that if
%    $\cs{eqnshift@} = 0\,\mathrm{pt}$---meaning that we're in a
%    \env{flalign} environment and this is the first line with a tag
%    that we've encountered---then we set $\cs{eqnshift@} =
%    \cs{@mathmargin}$ and recalculate \cs{alignsep@}.  This is done
%    by \cs{x@calc@shift@lf}.
%    \begin{macrocode}
\iftagsleft@\if@fleqn
    \def\calc@shift@align{%
        \global\let\tag@shifts\@empty
        \begingroup
%    \end{macrocode}
%    \cs{@tempdima} is initialized to $\cs{@mathmargin} -
%    \cs{mintagsep}$, which yields the maximum size of a tag that will
%    not be shifted to another line.
%    \begin{macrocode}
            \@tempdima\@mathmargin\relax
            \advance\@tempdima-\mintagsep\relax
%    \end{macrocode}
%    Now we examine each row in turn.  If the width of the tag on the
%    line is non-positive---meaning either that there is no tag or
%    else that the user has forced it to have zero width---we mark the
%    tag to remain unshifted.  Otherwise, we call \cs{x@calc@shift@lf}
%    to determine whether any adjustments need to be made to
%    \cs{eqnshift@} and \cs{alignsep@}.  Note the difference in
%    treatment of zero-width tags between this code and \tex/'s
%    built-in algorithm: here, a width of zero prohibits the tag from
%    being shifted, while in \tex/'s built-in algorithm, a width of
%    zero forces the tag to be shifted.
%    \begin{macrocode}
            \loop
                \ifnum\row@>0
                    \ifdim\tag@width\row@>\z@
                        \x@calc@shift@lf
                    \else
                        \saveshift@0%
                    \fi
                    \advance\row@\m@ne
            \repeat
        \endgroup
    }
%    \end{macrocode}
%    \end{macro}
%
%    \begin{macro}{\x@calc@shift@lf}
%    As mentioned above, \cs{x@calc@shift@lf} first checks to see if
%    the current left margin is set to 0 and, if so, resets it to
%    \cs{@mathmargin} and recalculates \cs{alignsep@}.  Next, it
%    checks whether the length of the current tag exceeds the
%    previously calculated limit and, if so, marks the tag to be
%    shifted to a separate line.
%    \begin{macrocode}
    \def\x@calc@shift@lf{%
        \ifdim\eqnshift@=\z@
            \global\eqnshift@\@mathmargin\relax
            \alignsep@\displaywidth
            \advance\alignsep@-\totwidth@
            \global\divide\alignsep@\@tempcntb
            \ifdim\alignsep@<\minalignsep\relax
                \global\alignsep@\minalignsep\relax
            \fi
        \fi
        \ifdim\tag@width\row@>\@tempdima
            \saveshift@1%
        \else
            \saveshift@0%
        \fi
    }
\fi\fi
%    \end{macrocode}
%    \end{macro}
%
%    \paragraph{Tag placement when \cs{tagsleft@false},
%    \cs{@fleqntrue}.}
%
%    Next we consider the case when equations are flush-left, but tags
%    are on the right.  This case is somewhat more complicated than
%    the previous one, since we can adjust the right margin by varying
%    the inter-align separatin. Thus, when a tag is found to be too
%    close to its equation, we first attempt to decrease
%    \cs{alignsep@} enough to move the equation off to an acceptable
%    distance.  Only if that would require a value of \cs{alignsep@}
%    less than \cs{minalignsep} do we move the tag to a separate line.
%
%    \begin{macro}{\calc@shift@align}
%    This version of \cs{calc@shift@align} differs from the previous
%    version only in calling \cs{x@calc@shift@rf} rather than
%    \cs{x@calc@shift@lf}.
%    \begin{macrocode}
\iftagsleft@\else\if@fleqn
    \def\calc@shift@align{%
        \global\let\tag@shifts\@empty
        \begingroup
            \loop
                \ifnum\row@>0
                    \ifdim\tag@width\row@>\z@
                        \x@calc@shift@rf
                    \else
                        \saveshift@0%
                    \fi
                    \advance\row@\m@ne
            \repeat
        \endgroup
    }
%    \end{macrocode}
%    \end{macro}
%
%    \begin{macro}{\x@calc@shift@rf}
%    To start, we need to know two quantities: the number of align
%    structures in the current row and the ``effective length'' of the
%    row, defined as the distance from the left margin to the
%    right edge of the text assuming that \cs{eqnshift@} and
%    \cs{alignsep@} are both~0.  To get the number of align
%    structures, we first count the number of columns by counting the
%    number of entries in the \cs{fieldlengths@} for the current row.
%    The effective length is calcuated by \cs{x@rcalc@width} and put
%    in the temporary register \cs{@tempdimc}, using \cs{@tempdimb} as
%    an auxiliary variable.
%    \begin{macrocode}
    \def\x@calc@shift@rf{%
        \column@\z@
        \@tempdimb\z@
        \@tempdimc\z@
        \edef\@tempb{\fieldlengths@\row@}%
        \@for\@tempa:=\@tempb\do{%
            \advance\column@\@ne
            \x@rcalc@width
        }%
        \begingroup
%    \end{macrocode}
%    If there are $n$ columns in the current row, then there are
%    $\lfloor (n+1)/2 \rfloor$ align structures and $\lfloor (n-1)/2
%    \rfloor$ interalign spaces.
%    \begin{macrocode}
            \advance\column@\m@ne
            \divide\column@\tw@
%    \end{macrocode}
%    If this is smaller than the maximum number of interalign spaces
%    in the environment, then we need to reduce \cs{@tempcnta} (the
%    total number of flexible spaces in the current line) by
%    $\cs{@tempcntb} - \cs{column@}$ and reset \cs{@tempcntb} to
%    \cs{column@}.
%    \begin{macrocode}
            \ifnum\@tempcntb>\column@
                \advance\@tempcnta-\@tempcntb
                \advance\@tempcnta\column@
                \@tempcntb\column@
            \fi
%    \end{macrocode}
%    Next, we add the width of the tag and the (fixed) left margin to
%    the effective length calculated above.  This can be used to
%    calculate how much ``free space'' there is in the current line
%    and thus how much leeway we have to increase the amount of space
%    between the tag and the equation.
%    \begin{macrocode}
            \tagwidth@\tag@width\row@\relax
            \@tempdima\eqnshift@
            \advance\@tempdima\@tempdimc\relax
            \advance\@tempdima\tagwidth@
%    \end{macrocode}
%    The first thing to check is whether the tag should be shifted to
%    a separate line.  To do this, we add the minimum interalign
%    separation and the \cs{mintagsep} to the value of \cs{@tempdima}
%    just calculated.  This yields the minimum acceptable length of
%    the current line.  If that is greater than \cs{displaywidth}, we
%    mark the tag to be calculated.  Otherwise, we mark the tag to be
%    kept on the same line and then check to see if the \cs{alignsep@}
%    needs to be reduced to make room for the tag.
%    \begin{macrocode}
            \dimen@\minalignsep\relax
            \multiply\dimen@\@tempcntb
            \advance\dimen@\mintagsep\relax
            \advance\dimen@\@tempdima
            \ifdim\dimen@>\displaywidth
                \saveshift@1%
            \else
                \saveshift@0%
%    \end{macrocode}
%    Now we perform essentially the same calculation, but using the
%    current value of \cs{alignsep@} instead of \cs{minalignsep}.
%    This gives the current length of the line.  If this is greater
%    than \cs{displaywidth}, we recalculate \cs{alignsep@} to make
%    room for the tag.
%    \begin{macrocode}
                \dimen@\alignsep@\relax
                \multiply\dimen@\@tempcntb
                \advance\dimen@\@tempdima
                \advance\dimen@\tagwidth@
                \ifdim\dimen@>\displaywidth
                    \dimen@\displaywidth
                    \advance\dimen@-\@tempdima
                    \ifnum\xatlevel@=\tw@
                        \advance\dimen@-\mintagsep\relax
                    \fi
                    \divide\dimen@\@tempcnta
                    \ifdim\dimen@<\minalignsep\relax
                        \global\alignsep@\minalignsep\relax
                    \else
                        \global\alignsep@\dimen@
                    \fi
                \fi
            \fi
        \endgroup
    }
\fi\fi
%    \end{macrocode}
%    \end{macro}
%
%    \paragraph{Tag placement when \cs{tagsleft@false},
%    \cs{@fleqnfalse}.}
%
%    This is similar to the previous case, except for the added
%    complication that both \cs{alignsep@} and \cs{eqnshift@} can
%    vary, which makes the computations correspondingly more
%    complicated.
%
%    \begin{macro}{\calc@shift@align}
%    \begin{macrocode}
\iftagsleft@\else\if@fleqn\else
    \def\calc@shift@align{%
        \global\let\tag@shifts\@empty
        \begingroup
            \loop
                \ifnum\row@>0
                    \ifdim\tag@width\row@>\z@
                        \x@calc@shift@rc
                    \else
                        \saveshift@0%
                    \fi
                    \advance\row@\m@ne
            \repeat
        \endgroup
    }
%    \end{macrocode}
%    \end{macro}
%
%    \begin{macro}{\x@calc@shift@rc}
%    \begin{macrocode}
    \def\x@calc@shift@rc{%
        \column@\z@
        \@tempdimb\z@
        \@tempdimc\z@
        \edef\@tempb{\fieldlengths@\row@}%
        \@for\@tempa:=\@tempb\do{%
            \advance\column@\@ne
            \x@rcalc@width
        }%
        \begingroup
            \advance\column@\m@ne
            \divide\column@\tw@
            \ifnum\@tempcntb>\column@
                \advance\@tempcnta-\@tempcntb
                \advance\@tempcnta\column@
                \@tempcntb\column@
            \fi
            \tagwidth@\tag@width\row@\relax
            \@tempdima\@tempdimc
            \advance\@tempdima\tagwidth@
            \dimen@\minalignsep\relax
            \multiply\dimen@\@tempcntb
            \advance\dimen@\mintagsep\relax
            \ifnum\xatlevel@=\tw@ \else
                \advance\dimen@\mintagsep\relax
            \fi
            \advance\dimen@\@tempdima
            \ifdim\dimen@>\displaywidth
                \saveshift@1%
            \else
                \saveshift@0%
                \dimen@\eqnshift@
                \advance\dimen@\@tempdima
                \advance\dimen@\@tempcntb\alignsep@
                \advance\dimen@\tagwidth@
                \ifdim\dimen@>\displaywidth
                    \dimen@\displaywidth
                    \advance\dimen@-\@tempdima
                    \ifnum\xatlevel@=\tw@
                        \advance\dimen@-\mintagsep\relax
                    \fi
                    \divide\dimen@\@tempcnta
                    \ifdim\dimen@<\minalignsep\relax
                        \global\alignsep@\minalignsep\relax
                        \eqnshift@\displaywidth
                        \advance\eqnshift@-\@tempdima
                        \advance\eqnshift@-\@tempcntb\alignsep@
                        \global\divide\eqnshift@\tw@
                    \else
                        \ifdim\dimen@<\eqnshift@
                            \ifdim\dimen@<\z@
                                \global\eqnshift@\z@
                            \else
                                \global\eqnshift@\dimen@
                            \fi
                        \fi
                        \ifdim\dimen@<\alignsep@
                            \global\alignsep@\dimen@
                        \fi
                    \fi
                \fi
            \fi
        \endgroup
    }
\fi\fi
%    \end{macrocode}
%    \end{macro}
%
%    \begin{macro}{\x@rcalc@width}
%    \begin{macrocode}
\iftagsleft@\else
    \def\x@rcalc@width{%
        \ifdim\@tempa > \z@
            \advance\@tempdimc\@tempdimb
            \ifodd\column@
                \advance\@tempdimc\maxcol@width\column@
                \@tempdimb\z@
            \else
                \advance\@tempdimc\@tempa\relax
                \@tempdimb\maxcol@width\column@
                \advance\@tempdimb-\@tempa\relax
            \fi
        \else
            \advance\@tempdimb\maxcol@width\column@\relax
        \fi
    }
\fi
%    \end{macrocode}
%    \end{macro}
%
%    \paragraph{Tag placement when \cs{tagsleft@true},
%    \cs{@fleqnfalse}.}
%
%    \begin{macro}{\calc@shift@align}
%    \begin{macrocode}
\iftagsleft@\if@fleqn\else
    \def\calc@shift@align{%
        \global\let\tag@shifts\@empty
        \begingroup
            \loop
                \ifnum\row@>\z@
                    \ifdim\tag@width\row@>\z@
                        \x@calc@shift@lc
                    \else
                        \saveshift@0%
                    \fi
                    \advance\row@\m@ne
            \repeat
        \endgroup
    }
%    \end{macrocode}
%    \end{macro}
%
%    \begin{macro}{\x@calc@shift@lc}
%    \begin{macrocode}
    \def\x@calc@shift@lc{%
        \column@\z@
%    \end{macrocode}
%    \cs{@tempdima} will (eventually) be set to the effective width of
%    the current row, defined as the distance from the leftmost point
%    of the current line to the end of the last field of the
%    \cs{halign}, ignoring any intervening \cs{tabskip}s, plus the
%    width of the current tag.  That is, it will be the width of the
%    first non-empty field plus the sum of the maximum widths of all
%    following fields, plus the tag width.
%
%    \cs{@tempdimb} will be the ``indentation'' of leftmost end of
%    text, ignoring the \cs{tabskip} glue, i.e., it will be the sum of
%    the maximum widths of any fields to the left of the first
%    non-empty field, plus whatever empty space there is at the
%    beginning of the first non-empty field.
%    \begin{macrocode}
        \@tempdima\z@ % ``width of equation''
        \@tempdimb\z@ % ``indent of equation''
        \edef\@tempb{\fieldlengths@\row@}%
        \@for\@tempa:=\@tempb\do{%
            \advance\column@\@ne
            \x@lcalc@width
        }%
        \begingroup
            \tagwidth@\tag@width\row@\relax
%    \end{macrocode}
%    \cs{@tempdima} is now easy to calculate, since it is just
%    $\cs{totwidth@} - \cs{@tempdimb} + \cs{tagwidth@}$.
%    \begin{macrocode}
            \@tempdima\totwidth@
            \advance\@tempdima-\@tempdimb
            \advance\@tempdima\tagwidth@
%    \end{macrocode}
%    Next, we check to see whether there is room for both the equation
%    and the tag on the same line, by calculating the minimum
%    acceptable length of the current row and comparing that to
%    \cs{displaywidth}.  Note that here we use \cs{@tempcntb}, i.e.,
%    the number of interalign spaces after the first non-empty align
%    structure.
%    \begin{macrocode}
            \dimen@\minalignsep\relax
            \multiply\dimen@\@tempcntb
            \advance\dimen@\mintagsep\relax
            \ifnum\xatlevel@=\tw@ \else
                \advance\dimen@\mintagsep\relax
            \fi
            \advance\dimen@\@tempdima
%    \end{macrocode}
%    If the minimum acceptable width of the current line is greater
%    than \cs{displaywidth}, we mark the current tag to be shifted to
%    a separate line.
%    \begin{macrocode}
            \ifdim\dimen@>\displaywidth
                \saveshift@1%
            \else
%    \end{macrocode}
%    Otherwise, the tag can stay on the same line as the equation, but
%    we need to check whether it is too close to the equation.  So, we
%    calculate the distance between the left margin and the left side
%    of the equation, using the current values of \cs{eqnshift@} and
%    \cs{alignsep@}.  Note that we use \cs{count@} here, not
%    \cs{@tempcntb}, as above.
%    \begin{macrocode}
                \saveshift@0%
                \dimen@\alignsep@
                \multiply\dimen@\count@
                \advance\dimen@\eqnshift@
                \advance\dimen@\@tempdimb
%    \end{macrocode}
%    If the left margin is less than twice the tag width, we calculate
%    new values of \cs{eqnshift@} and \cs{alignsep@} to move the
%    equation further away from the tag.  In particular, we center the
%    current line between its tag and the right margin.  Note that
%    although we later will need to transform \cs{dimen@} into a value
%    suitable for use as \cs{eqnshift@}, for the time being it is more
%    useful to think of it as the space separating the tag from the
%    equation.
%    \begin{macrocode}
                \ifdim\dimen@<2\tagwidth@
                    \dimen@\displaywidth
                    \advance\dimen@-\@tempdima
                    \ifnum\xatlevel@=\tw@
                        \advance\dimen@-\mintagsep\relax
                    \fi
%    \end{macrocode}
%    In certain circumstances we will get a divide-by-zero error here
%    unless we guard against it. Use of \cs{@tempcnta} is complicated,
%    sometimes it is assigned globally, sometimes locally. Need to sort
%    it out one of these days [mjd,2000/06/02].
%    \begin{macrocode}
                    \ifnum\@tempcnta>\z@
                      \divide\dimen@\@tempcnta
                    \else \dimen@\z@
                    \fi
%    \end{macrocode}
%    As usual, we check to make sure we don't set \cs{alignsep@}
%    smaller than \cs{minalignsep} and, in any case, that we don't
%    replace \cs{alignsep@} by a larger value.
%    \begin{macrocode}
                    \ifdim\dimen@<\minalignsep\relax
                        \global\alignsep@\minalignsep\relax
                        \dimen@\displaywidth
                        \advance\dimen@-\@tempdima
                        \advance\dimen@-\@tempcntb\alignsep@
                        \global\divide\dimen@\tw@
                    \else
                        \ifdim\dimen@<\alignsep@
                            \global\alignsep@\dimen@
                        \fi
                    \fi
%    \end{macrocode}
%    Next, we calculate an appropriate value of \cs{eqnshift@},
%    assuming that \cs{dimen@} is the desired separation between the
%    tag and equation of the current line.  This means that we first
%    need to adjust \cs{dimen@} if we're in an \env{flalign}
%    environment.
%    \begin{macrocode}
                    \ifnum\xatlevel@=\tw@
                        \dimen@\mintagsep\relax
                    \fi
%    \end{macrocode}
%    Now we calculate the value of \cs{eqnshift@} needed to produce a
%    separation of \cs{dimen@} between the equation tag and the
%    beginning of the equation.  To do this, we need the following
%    equation to hold:
%    \[
%       \cs{eqnshift@} + n\cs{alignsep@} + \cs{@tempdimb}
%           = \cs{tagwidth@} + \cs{dimen@}
%    \]
%    where $n = \cs{count@}$ is the number of interalign spaces before
%    the first non-empty field of the current line.
%    \begin{macrocode}
                    \advance\dimen@\tagwidth@
                    \advance\dimen@-\@tempdimb
                    \advance\dimen@-\count@\alignsep@
%    \end{macrocode}
%    The value of \cs{eqnshift@} just calculated is the minimum
%    acceptable value; thus, we save it only if it is larger than the
%    current value.
%    \begin{macrocode}
                    \ifdim\dimen@>\eqnshift@
                        \global\eqnshift@\dimen@
                    \fi
                \fi
            \fi
        \endgroup
    }
%    \end{macrocode}
%    \end{macro}
%
%    \begin{macro}{\x@lcalc@width}
%    This macro calculates the ``indentation'' of the current row, as
%    defined above under the description of \cs{x@calc@shift@lc}.
%    This macro is called for each field of the current line, with
%    \cs{@tempa} set to the width of the current field.  Ideally, the
%    loop enclosing \cs{x@lcalc@width} would terminate as soon as
%    \cs{@tempa} is non-zero, but that would be a bit tricky to
%    arrange.  Instead, we use \cs{@tempdima} as a flag to signal when
%    we've encountered the first non-empty field.
%
%    \begin{macrocode}
    \def\x@lcalc@width{%
        \ifdim\@tempdima = \z@
%    \end{macrocode}
%    If the current field is empty (i.e., $\cs{@tempa} =
%    \mathrm{0\,pt}$, then we increment \cs{@tempdimb} by the width of
%    the current field).  Otherwise, we set $\cs{@tempdima} =
%    \mathrm{1\,pt}$ as a signal value and increment \cs{@tempdimb} by
%    the width of whatever empty space there might be at the left of
%    the current field.
%    \begin{macrocode}
            \ifdim\@tempa > \z@
                \@tempdima\p@
                \ifodd\column@
                    \advance\@tempdimb \maxcol@width\column@
                    \advance\@tempdimb-\@tempa
                \fi
%    \end{macrocode}
%    In addition, we need to adjust the values of \cs{@tempcnta} and
%    \cs{@tempcntb} to account for any empty align structures that
%    might occur at the beginning of the current line.  More
%    specifically, we first set \cs{count@} equal to the number of
%    interalign spaces preceding the current field (namely, $\lfloor
%    (\cs{\column@}-1)/2 \rfloor$), and then subtract \cs{count@} from
%    both \cs{@tempcnta} and \cs{@tempcntb}.  The rationale is that
%    for the purposes of adjusting the spacing between the tag and the
%    equation, the only flexible interalign spaces are those after
%    the first non-empty align structure, so we need to treat those
%    different from the ones before the first non-empty align
%    structure.
%    \begin{macrocode}
                \count@\column@
                \advance\count@\m@ne
                \divide\count@\tw@
                \advance\@tempcnta-\count@
                \advance\@tempcntb-\count@
            \else
                \advance\@tempdimb \maxcol@width\column@\relax
            \fi
        \fi
    }
\fi\fi
%    \end{macrocode}
%    \end{macro}
%
%  \begin{macro}{\place@tag}
%    \cs{place@tag} takes care of the placment of tags in the
%    \env{align} environments.
%    \begin{macrocode}
\def\place@tag{%
    \iftagsleft@
        \kern-\tagshift@
        \if1\shift@tag\row@\relax
            \rlap{\vbox{%
                \normalbaselines
                \boxz@
                \vbox to\lineht@{}%
                \raise@tag
            }}%
        \else
            \rlap{\boxz@}%
        \fi
        \kern\displaywidth@
    \else
        \kern-\tagshift@
        \if1\shift@tag\row@\relax
%    \end{macrocode}
%    Added depth to correct vertical spacing of shifted
%    equation tags.---dmj, 1994/12/29
%    \begin{macrocode}
            \llap{\vtop{%
                \raise@tag
                \normalbaselines
                \setbox\@ne\null
                \dp\@ne\lineht@
                \box\@ne
                \boxz@
            }}%
        \else
            \llap{\boxz@}%
        \fi
    \fi
}
%    \end{macrocode}
%  \end{macro}
%
% \begin{macro}{\align@preamble}
%    \begin{macrocode}
\def\align@preamble{%
   &\hfil
    \strut@
    \setboxz@h{\@lign$\m@th\displaystyle{##}$}%
    \ifmeasuring@\savefieldlength@\fi
    \set@field
    \tabskip\z@skip
   &\setboxz@h{\@lign$\m@th\displaystyle{{}##}$}%
    \ifmeasuring@\savefieldlength@\fi
    \set@field
    \hfil
    \tabskip\alignsep@
}
%    \end{macrocode}
%  \end{macro}
%
%  \begin{macro}{\set@field}
%    \cs{set@field} increments the column counter, tracks the value of
%    \cs{lineht@} and finally inserts the box containing the contents
%    of the current field.
%    \begin{macrocode}
\def\set@field{%
    \column@plus
    \iftagsleft@
        \ifdim\ht\z@>\lineht@
            \global\lineht@\ht\z@
        \fi
    \else
        \ifdim\dp\z@>\lineht@
            \global\lineht@\dp\z@
        \fi
    \fi
    \boxz@
}
%    \end{macrocode}
%    \end{macro}
%
%
% \subsection {The \env{split} environment}
%
%    \begin{macro}{\split@err}
%    A special error function for \env{split} to conserve main mem (at a
%    cost of string pool/hash size.
%    \begin{macrocode}
\edef\split@err#1{%
    \@nx\@amsmath@err{%
        \string\begin{split} won't work here%
    }{%
        \@xp\@nx\csname
  Did you forget a preceding \string\begin{equation}?^^J%
  If not, perhaps the `aligned' environment is what
  you want.\endcsname}%
}
%    \end{macrocode}
%    \end{macro}
%
%    \begin{environment}{split}
%    If the \env{split} environment occurs inside \env{align} or
%    \env{gather}, it can make use of the enclosing halign; if it is
%    called inside a simple equation, we add an implicit `gather'
%    container.
%
%    \begin{macrocode}
\newenvironment{split}{%
  \if@display
    \ifinner
      \@xp\@xp\@xp\split@aligned
    \else
      \ifst@rred \else \global\@eqnswtrue \fi
    \fi
  \else \let\endsplit\@empty \@xp\collect@body\@xp\split@err
  \fi
  \collect@body\gather@split
}{%
      \crcr
    \egroup
  \egroup
  \iftagsleft@ \@xp\lendsplit@ \else \@xp\rendsplit@ \fi
}
%    \end{macrocode}
%
%    \begin{macrocode}
\let\split@tag\relax % init
%    \end{macrocode}
%
%    \begin{macrocode}
\def\gather@split#1#2#3{%
  \@xp\endgroup \reset@equation % math@cr will handle equation numbering
  \iftag@
     \toks@\@xp{\df@tag}%
     \edef\split@tag{%
       \gdef\@nx\df@tag{\the\toks@}%
       \global\@nx\tag@true \@nx\nonumber
     }%
  \else \let\split@tag\@empty
  \fi
  \spread@equation
%    \end{macrocode}
%    The extra vcenter wrapper here is not really a good thing but
%    without it there are compatibility problems with old documents that
%    throw in some extra material between \verb'\begin{equation}' and
%    \verb'\begin{split}' (for example, \verb'\hspace{-1pc}' or
%    \verb'\left\{'). [mjd,1999/09/20]
%    \begin{macrocode}
  \vcenter\bgroup
    \gather@{\split@tag  \begin{split}#1\end{split}}%
    \def\endmathdisplay@a{%
      \math@cr \black@ \totwidth@ \egroup
      \egroup
    }%
}
%    \end{macrocode}
%
%    \end{environment}
%
% \begin{macro}{\insplit@}
%    \begin{macrocode}
\def\insplit@{%
  \global\setbox\z@\vbox\bgroup
    \Let@ \chardef\dspbrk@context\@ne \restore@math@cr
    \default@tag % disallow use of \tag here
    \ialign\bgroup
      \hfil
      \strut@
      $\m@th\displaystyle{##}$%
     &$\m@th\displaystyle{{}##}$%
      \hfill % Why not \hfil?---dmj, 1994/12/28
      \crcr
}
%    \end{macrocode}
% \end{macro}
%
% \begin{macro}{\rendsplit@}
%    Moved the box maneuvers inside the \cs{ifinalign@}, since that is
%    the only place they are needed.---dmj, 1994/12/28
%
%    TODO: Explore interaction of tag-placement algorithm with
%    \env{split}.  Is there any way for \env{split} to pass the
%    relevant information out to the enclosing \env{gather} or
%    \env{align}?
%    \begin{macrocode}
\def\rendsplit@{%
    \ifinalign@
%    \end{macrocode}
%    Changed |\box9| into a \cs{vtop} here for better spacing.
%    \begin{macrocode}
        \global\setbox9 \vtop{%
            \unvcopy\z@
            \global\setbox8 \lastbox
            \unskip
        }%
        \setbox\@ne\hbox{%
            \unhcopy8
            \unskip
            \global\setbox\tw@\lastbox
            \unskip
            \global\setbox\thr@@\lastbox
        }%
        \ifctagsplit@
            \gdef\split@{%
                \hbox to\wd\thr@@{}%
               &\vcenter{\vbox{\moveleft\wd\thr@@\boxz@}}%
            }%
        \else
            \global\setbox7 \hbox{\unhbox\tw@\unskip}%
%    \end{macrocode}
%    Added \cs{add@amps} to make sure we put the last line of the
%    \env{split} into the proper column of an \env{align} environment
%    with multiple align structures.---dmj, 1994/12/28
%
%    Special care has to be taken in this case because the \env{split}
%    turns into two lines of the \env{align} instead of just one.  So,
%    we have to make sure that the first line produced by the
%    \env{split} doesn't upset our bookkeeping, hence we call
%    \cs{savetaglength@} to insert 0\,pt as the tag for this
%    pseudo-line, and we advance the \cs{row@} counter and reset
%    \cs{lineht@} afterwards.  It would be nice if we could just
%    replace the \cs{crcr} by \cs{math@cr@@@}, but that would cause
%    problems with the tag processing.
%    \begin{macrocode}
            \gdef\split@{%
                \global\@tempcnta\column@
               &\setboxz@h{}%
                \savetaglength@
                \global\advance\row@\@ne
                \vbox{\moveleft\wd\thr@@\box9}%
                \crcr
                \noalign{\global\lineht@\z@}%
                \add@amps\@tempcnta
                \box\thr@@
               &\box7
            }%
        \fi
    \else
        \ifctagsplit@
            \gdef\split@{\vcenter{\boxz@}}%
        \else
%    \end{macrocode}
%    Changed to just \cs{boxz@}, otherwise last line gets centered
%    rather than aligned properly with respect to the rest of the
%    lines.  But this means that we can't see inside of the last line
%    to decide whether the tag needs to be moved.  Will have to think
%    about this.---dmj, 1994/12/28
%    \begin{macrocode}
            \gdef\split@{%
                \boxz@
%                \box9
%                \crcr
%                \hbox{\box\thr@@\box7}%
            }%
        \fi
    \fi
    \aftergroup\split@
}
%    \end{macrocode}
% \end{macro}
%
% \begin{macro}{\lendsplit@}
%    \begin{macrocode}
\def\lendsplit@{%
    \global\setbox9\vtop{\unvcopy\z@}%
    \ifinalign@
%    \end{macrocode}
%    Moved following two boxes inside the \cs{ifinalign@}, since they
%    are only used in that case.  In fact, if we just kept track of
%    the width of the first column, we could dispense with this
%    entirely.  Surely that would be more efficient than all these box
%    copies.---dmj, 1994/12/28
%    \begin{macrocode}
        \setbox\@ne\vbox{%
            \unvcopy\z@
            \global\setbox8\lastbox
        }%
        \setbox\@ne\hbox{%
            \unhcopy8%
            \unskip
            \setbox\tw@\lastbox
            \unskip
            \global\setbox\thr@@\lastbox
        }%
        \ifctagsplit@
            \gdef\split@{%
                \hbox to\wd\thr@@{}%
               &\vcenter{\vbox{\moveleft\wd\thr@@\box9}}%
            }%
        \else
            \gdef\split@{%
                \hbox to\wd\thr@@{}%
               &\vbox{\moveleft\wd\thr@@\box9}%
            }%
        \fi
    \else
        \ifctagsplit@
            \gdef\split@{\vcenter{\box9}}%
        \else
            \gdef\split@{\box9}%
        \fi
    \fi
    \aftergroup\split@
}
%    \end{macrocode}
% \end{macro}
%
%    With \pkg{amsmath} 1.2 it was possible to put things like
%    \verb'\left\{' between \verb'\begin{equation}' and
%    \verb'\begin{split}' without getting any error message. For
%    backward compatibility we try to avoid a fatal error in this case
%    and instead attempt recovery with \env{aligned}.
%    \begin{macrocode}
\def\split@aligned#1#2{%
   \iffalse{\fi\ifnum0=`}\fi
   \collect@body\split@al@a}
%    \end{macrocode}
%
%    \begin{macrocode}
\def\split@al@a#1#2#3{%
  \split@warning
  \endgroup
%    \end{macrocode}
%    If the \opt{fleqn} and \opt{tbtags} options are both in effect then
%    we will need to add an optional argument on the \env{aligned}
%    environment.
%    \begin{macrocode}
  \toks@{\begin{aligned}}%
  \if@fleqn \split@al@tagcheck \fi
%    \end{macrocode}
%    The \cs{relax} here is to prevent \cs{@let@token} from being left
%    equal to an ampersand if that happens to be the first thing in the body.
%    \begin{macrocode}
  \the\toks@\relax#1\end{aligned}%
  \ifnum0=`{\fi\iffalse}\fi
}
%    \end{macrocode}
%
%    \begin{macrocode}
\def\split@al@tagcheck{%
   \ifctagsplit@
   \else
     \iftagsleft@ \toks@\@xp{\the\toks@ [t]}%
     \else \toks@\@xp{\the\toks@ [b]}%
     \fi
   \fi
}
%    \end{macrocode}
%
%    \begin{macrocode}
\def\split@warning{%
  \PackageWarning{amsmath}{%
Cannot use `split' here;\MessageBreak trying to recover with `aligned'}%
}
%    \end{macrocode}
%
% \subsection{The \env{multline} environment}
%
%    In the original \amstex/, \cn{multlinegap} is a macro with an
%    argument that resets an internal dimension (one with an \qc{\@}
%    character in its name). Here, to save control sequence names, we
%    define \cn{multlinegap} to be the dimension itself and the
%    documentation instructs users to use \cn{setlength} if they
%    need to change it.
% \begin{macro}{\multlinegap}
% \begin{macro}{\multlinetaggap}
%    Changed \cs{multlinegap} and \cs{multlinetaggap} to skip
%    registers.  Also changed name to \cs{multlinetaggap} from
%    \cs{multlinetaggap@}.
%    \begin{macrocode}
\newskip\multlinegap
\multlinegap10pt
\newskip\multlinetaggap
\multlinetaggap10pt
%    \end{macrocode}
% \end{macro}
% \end{macro}
%
%    \begin{macro}{\start@multline}
%    \begin{macrocode}
\def\start@multline#1{%
    \RIfM@
        \nomath@env
        \DN@{\@namedef{end\@currenvir}{}\@gobble}%
    \else
        $$%
        #1%
        \ifst@rred
            \nonumber
        \else
            \global\@eqnswtrue
        \fi
        \let\next@\multline@
    \fi
    \collect@body\next@
}
%    \end{macrocode}
%    \end{macro}
%
% \begin{environment}{multline}
% \begin{environment}{multline*}
%    \begin{macrocode}
\newenvironment{multline}{%
  \start@multline\st@rredfalse
}{%
  \iftagsleft@ \@xp\lendmultline@ \else \@xp\rendmultline@ \fi
  \ignorespacesafterend
}
%    \end{macrocode}
%
%    \begin{macrocode}
\newenvironment{multline*}{\start@multline\st@rredtrue}{\endmultline}
%    \end{macrocode}
% \end{environment}
% \end{environment}
%
% \begin{macro}{\multline@}
%    \begin{macrocode}
\def\multline@#1{%
    \Let@
%    \end{macrocode}
%
%    For multline neither \cs{displ@y} no \cs{displ@y@} is quite right;
%    we want to advance the row number and (I suppose?) the
%    display-pagebreak level, but we only want to do tag-related stuff
%    once before the first line, not repeat it for every line. (Recall
%    that the arg of \cs{@display@init} goes into \cs{everycr}.)
%    \begin{macrocode}
    \@display@init{\global\advance\row@\@ne \global\dspbrk@lvl\m@ne}%
    \chardef\dspbrk@context\z@
    \restore@math@cr
%    \end{macrocode}
%    The \env{multline} environment is somewhat unusual, in that
%    \cs{tag} and \cs{label} are enabled only during the measuring
%    phase and disabled during the production phase.
%    Here we disable \cs{tag} and \cs{label}; \cs{mmeasure@} will
%    re-enable them temporarily.
%    \begin{macrocode}
    \let\tag\tag@in@align
    \global\tag@false \global\let\raise@tag\@empty
    \mmeasure@{#1}%
    \let\tag\gobble@tag \let\label\@gobble
    \tabskip \if@fleqn \@mathmargin \else \z@skip \fi
    \totwidth@\displaywidth
    \if@fleqn
        \advance\totwidth@-\@mathmargin
    \fi
    \halign\bgroup
        \hbox to\totwidth@{%
%    \end{macrocode}
%    In order to get the spacing of the last line right in fleqn
%    mode, we need to play a little game here.  Normally the
%    stretchability of the \cs{hskip} here will be suppressed by the
%    \cs{hfil} at the end of the template, except inside the last line,
%    when that \cs{hfil} will be removed by the \cs{hfilneg} in
%    \cs{lendmultline@}.
%    \begin{macrocode}
            \if@fleqn
                \hskip \@centering \relax
            \else
                \hfil
            \fi
            \strut@
            $\m@th\displaystyle{}##\endmultline@math
            \hfil
        }%
        \crcr
%    \end{macrocode}
%    In \opt{fleqn} mode, it's the \cs{tabskip} of \cs{@mathmargin}
%    that needs to be removed in the first line, not the \cs{hfil} at
%    the beginning of the template.
%    \begin{macrocode}
        \if@fleqn
            \hskip-\@mathmargin
            \def\multline@indent{\hskip\@mathmargin}% put it back
        \else
            \hfilneg
            \def\multline@indent{\hskip\multlinegap}%
        \fi
        \iftagsleft@
            \iftag@
                \begingroup
                    \ifshifttag@
                        \rlap{\vbox{%
                                \normalbaselines
                                \hbox{%
                                    \strut@
                                    \make@display@tag
                                }%
                                \vbox to\lineht@{}%
                                \raise@tag
                        }}%
%    \end{macrocode}
%    If the equation tag doesn't fit on the same line with the first
%    line of the display, we'll indent the first line by
%    \cn{multlinegap}.  This is a change from \pkg{amstex}, where the
%    first line would have been flush against the left margin in this
%    case.  A corresponding change will be made in \cs{rendmultline@}.
%    \begin{macrocode}
                        \multline@indent
                    \else
                        \setbox\z@\hbox{\make@display@tag}%
                        \dimen@\@mathmargin \advance\dimen@-\wd\z@
                        \ifdim\dimen@<\multlinetaggap
                          \dimen@\multlinetaggap
                        \fi
                        \box\z@ \hskip\dimen@\relax
                    \fi
                \endgroup
            \else
                \multline@indent
            \fi
        \else
            \multline@indent
        \fi
    #1%
}
%    \end{macrocode}
%
%    An extra level of indirection for the closing \verb'$' in multline
%    allows us to avoid getting an extra thinmuskip from a final
%    mathpunct in the equation contents, when equation numbers are on
%    the right. If we did not use this workaround, the sequence of
%    elements for a final comma would be, e.g.,
% \begin{verbatim}
% ... ,<hskip><box containing equation number>
% \end{verbatim}
%    which is equivalent to a sequence \verb'<mathpunct><mathord>' as
%    far as the automatic math spacing is concerned.
%    \begin{macrocode}
\def\endmultline@math{$}
%    \end{macrocode}
% \end{macro}
%
% \begin{macro}{\lendmultline@}
%    Bug fix: changed \cs{crcr} to \cs{math@cr} so that \cs{@eqpen}
%    gets reset properly if \cs{displaybreak} is used on the
%    penultimate line of an \env{align}.
%    \begin{macrocode}
\def\lendmultline@{%
        \hfilneg
        \hskip\multlinegap
        \math@cr
    \egroup
    $$%
}
%    \end{macrocode}
% \end{macro}
%
% \begin{macro}{\rendmultline@}
%    \begin{macrocode}
\def\rendmultline@{%
    \iftag@
        $\let\endmultline@math\relax
            \ifshifttag@
                \hskip\multlinegap
%    \end{macrocode}
%    Added depth to correct vertical spacing of shifted equation
%    tags.---dmj, 1994/12/29
%    \begin{macrocode}
                \llap{\vtop{%
                    \raise@tag
                    \normalbaselines
                    \setbox\@ne\null
                    \dp\@ne\lineht@
                    \box\@ne
                    \hbox{\strut@\make@display@tag}%
                }}%
            \else
                \hskip\multlinetaggap
                \make@display@tag
            \fi
    \else
        \hskip\multlinegap
    \fi
    \hfilneg
%    \end{macrocode}
%    Use \cs{math@cr} rather than just \cs{crcr} so that \cs{@eqpen}
%    gets reset properly if \cs{displaybreak} is used.
%    \begin{macrocode}
        \math@cr
    \egroup$$%
}
%    \end{macrocode}
% \end{macro}
%
% \begin{macro}{\mmeasure@}
%    \begin{macrocode}
\def\mmeasure@#1{%
    \begingroup
        \measuring@true
%    \end{macrocode}
%    We use \cs{begin/endgroup} rather than |{}| in this definition of
%    \cn{label} because the latter would create an extra (wasteful of
%    main mem) null box in the current math list. [mjd, 1995/01/17]
%    \begin{macrocode}
        \def\label##1{%
          \begingroup\measuring@false\label@in@display{##1}\endgroup}%
        \def\math@cr@@@{\cr}%
        \let\shoveleft\@iden \let\shoveright\@iden
        \savecounters@
        \global\row@\z@
        \setbox\@ne\vbox{%
            \global\let\df@tag\@empty
            \halign{%
                \setboxz@h{\@lign$\m@th\displaystyle{}##$}%
                \iftagsleft@
                    \ifnum\row@=\@ne
                        \global\totwidth@\wdz@
                        \global\lineht@\ht\z@
                    \fi
                \else
                    \global\totwidth@\wdz@
                    \global\lineht@\dp\z@
                \fi
                \crcr
                #1%
                \crcr
            }%
        }%
        \ifx\df@tag\@empty\else\global\tag@true\fi
        \if@eqnsw\global\tag@true\fi
        \iftag@
            \setboxz@h{%
                \if@eqnsw
                    \stepcounter{equation}%
                    \tagform@\theequation
                \else
                    \df@tag
                \fi
            }%
            \global\tagwidth@\wdz@
            \dimen@\totwidth@
            \advance\dimen@\tagwidth@
            \advance\dimen@\multlinetaggap
            \iftagsleft@\else
                \if@fleqn
                    \advance\dimen@\@mathmargin
                \fi
            \fi
            \ifdim\dimen@>\displaywidth
                \global\shifttag@true
            \else
                \global\shifttag@false
            \fi
        \fi
        \restorecounters@
    \endgroup
}
%    \end{macrocode}
% \end{macro}
%
%    \begin{macro}{\shoveleft}
%    \begin{macro}{\shoveright}
%    \cs{shoveleft} and \cs{shoveright} need to do slightly different
%    things depending on whether tags are on the left or the right and
%    whether we're in \opt{fleqn} mode.  For compactness of code, we
%    make the appropriate decisions at ``compile'' time rather than at
%    load time.
%
%    TODO: Investigate making \cs{shoveright} behave ``properly''(?) if
%    used on the first line of a \env{multline} and make \cs{shoveleft}
%    behave properly if used on the last line of a \env{multline}. But
%    in his \fn{amstex.doc} Spivak indicates those commands should never
%    be used on a first or last line. Perhaps better to leave the
%    question open unless/until real-life examples turn up.
%    \begin{macrocode}
\iftagsleft@
    \def\shoveright#1{%
        #1%
        \hfilneg
        \hskip\multlinegap
    }
\else
    \def\shoveright#1{%
        #1%
        \hfilneg
        \iftag@
            \ifshifttag@
                \hskip\multlinegap
            \else
                \hskip\tagwidth@
                \hskip\multlinetaggap
            \fi
        \else
            \hskip\multlinegap
        \fi
    }
\fi

\if@fleqn
    \def\shoveleft#1{#1}%
\else
    \iftagsleft@
        \def\shoveleft#1{%
            \setboxz@h{$\m@th\displaystyle{}#1$}%
            \setbox\@ne\hbox{$\m@th\displaystyle#1$}%
            \hfilneg
            \iftag@
                \ifshifttag@
                    \hskip\multlinegap
                \else
                    \hskip\tagwidth@
                    \hskip\multlinetaggap
                \fi
            \else
                \hskip\multlinegap
            \fi
            \hskip.5\wd\@ne
            \hskip-.5\wdz@
            #1%
        }
    \else
        \def\shoveleft#1{%
            \setboxz@h{$\m@th\displaystyle{}#1$}%
            \setbox\@ne\hbox{$\m@th\displaystyle#1$}%
            \hfilneg
            \hskip\multlinegap
            \hskip.5\wd\@ne
            \hskip-.5\wdz@
            #1%
        }
    \fi
\fi
%    \end{macrocode}
%    \end{macro}
%    \end{macro}
%
%    \subsection{The \env{equation} environment}
%
%    Rewritten from the ground up for version 2.0 to fix no-shrink and
%    no-shortskips bugs [mjd,2000/01/06].
%
%    Standard \latex/ provides three environments for one-line equations:
%    \cn{[}\cn{]}, \env{equation}, and \env{displaymath}. We add
%    \env{equation*} as a synonym for \env{displaymath}.
%    \begin{macrocode}
\@saveprimitive\leqno\@@leqno
\@saveprimitive\eqno\@@eqno
\def\eqno{\@@eqno\let\eqno\relax\let\leqno\relax}
\def\leqno{\@@leqno\let\leqno\relax\let\eqno\relax}
%
\let\veqno=\@@eqno
\iftagsleft@ \let\veqno=\@@leqno \fi
%    \end{macrocode}
%
%    Support for the \pkg{showkeys} package: provide no-op definitions
%    for a couple of SK functions, if they are not already defined. Then
%    we can just call them directly in our code without any extra fuss.
%    If the \pkg{showkeys} package is loaded later, our trivial
%    definitions will get overridden and everything works fine.
%    \begin{macrocode}
\@ifundefined{SK@@label}{%
  \let\SK@@label\relax \let\SK@equationtrue\relax
}{}
%    \end{macrocode}
%
%    \begin{macrocode}
\let\reset@equation\@empty
%    \end{macrocode}
%
%    Cf \cs{tag@in@align}. This is a bit of a mess though. Could use
%    some work. [mjd,1999/12/21]
%    \begin{macrocode}
\let\alt@tag\@empty
\def\tag@in@display#1#{\relax\tag@in@display@a{#1}}
\def\tag@in@display@a#1#2{%
  \iftag@
    \invalid@tag{Multiple \string\tag}\relax
  \else
    \global\tag@true \nonumber \reset@equation \st@rredtrue
    \if *\string#1%
      \gdef\alt@tag{\def\SK@tagform@{#2\@gobble}%
        \ifx\SK@@label\relax \let\tagform@\SK@tagform@ \fi
      }%
      \make@df@tag@@{#2}%
    \else
      \make@df@tag@@@{#2}%
    \fi
  \fi
}
%    \end{macrocode}
%
%    \begin{macrocode}
\let\restore@hfuzz\@empty
%    \end{macrocode}
%
%    \begin{macrocode}
\def\mathdisplay#1{%
  \ifmmode \@badmath
  \else
    $$\def\@currenvir{#1}%
%    \end{macrocode}
%    Allow use of \cn{displaybreak}.
%    \begin{macrocode}
    \let\dspbrk@context\z@
%    \end{macrocode}
%    Although in some cases simpler label handling would seem to be
%    sufficient, always using \cs{label@in@display} makes it easier to
%    support the \pkg{showkeys} package.
%    \begin{macrocode}
    \let\tag\tag@in@display \let\label\label@in@display \SK@equationtrue
    \global\let\df@label\@empty \global\let\df@tag\@empty
    \global\tag@false
    \let\mathdisplay@push\mathdisplay@@push
    \let\mathdisplay@pop\mathdisplay@@pop
    \if@fleqn
%    \end{macrocode}
%    Turn off overfull box messages temporarily\mdash otherwise there
%    would be unwanted extra ones emitted during our measuring
%    operations.
%    \begin{macrocode}
      \edef\restore@hfuzz{\hfuzz\the\hfuzz\relax}%
      \hfuzz\maxdimen
%    \end{macrocode}
%    Initially set the equation body in a box of displaywidth. Then if
%    the box is not overfull, as we find by checking \cs{badness}, we
%    have acquired useful information for the subsequent processing.
%    \begin{macrocode}
      \setbox\z@\hbox to\displaywidth\bgroup
        \let\split@warning\relax \restore@hfuzz
        \everymath\@emptytoks \m@th $\displaystyle
    \fi
  \fi
}
%    \end{macrocode}
%
%    Arg 1 is not currently used. I thought it might come in handy for
%    error messages.
%    \begin{macrocode}
\def\endmathdisplay#1{%
  \ifmmode \else \@badmath \fi
  \endmathdisplay@a
  $$%
%    \end{macrocode}
%    I guess the following code means this structure is non-reentrant.
%    But there is plenty of scope for tricky bugs here; suppressing them
%    by brute force at least makes it possible to get things working
%    correctly for normal use. [mjd,2000/01/06]
%    \begin{macrocode}
  \global\let\df@label\@empty \global\let\df@tag\@empty
  \global\tag@false \global\let\alt@tag\@empty
  \global\@eqnswfalse
}
%    \end{macrocode}
%
%    \begin{macrocode}
\def\endmathdisplay@a{%
  \if@eqnsw \gdef\df@tag{\tagform@\theequation}\fi
  \if@fleqn \@xp\endmathdisplay@fleqn
  \else \ifx\df@tag\@empty \else \veqno \alt@tag \df@tag \fi
    \ifx\df@label\@empty \else \@xp\ltx@label\@xp{\df@label}\fi
  \fi
  \ifnum\dspbrk@lvl>\m@ne
    \postdisplaypenalty -\@getpen\dspbrk@lvl
    \global\dspbrk@lvl\m@ne
  \fi
}
%    \end{macrocode}
%
%    A boolean variable: Was that last box overfull or not? A value of 0
%    means yes, it was overfull.
%    \begin{macrocode}
\let\too@wide\@ne
%    \end{macrocode}
%
%    Special handling is needed for flush-left equations. We need to
%    measure the equation body (found in box 0 after we close it with
%    the \cs{egroup}). Then after a fairly normal test to see if it fits
%    within the available space, we need to consider overlapping into
%    the displayindent area if displayindent is nonzero (as in an
%    indented list). If there is an equation number we may have to shift
%    it by hand to a separate line when there is not enough room;
%    we can no longer take advantage of the automatic shifting provided
%    by the \cn{leqno}, \cn{eqno} primitives.
%
%    We initially add \cs{@mathmargin} glue at the end of box 0 to get
%    an accurate overfull test. If \cs{@mathmargin} contains any shrink
%    then we cannot reliably tell whether the box will be overfull or
%    not simply by doing hand calculations from the actual width of the
%    equation body. We have to actually set the box and find out what
%    happens.
%
%    On the other hand if we put the \cs{@mathmargin} glue at the
%    beginning of the box it's awkward to remove it afterwards. So we
%    first put it in at the end and later we will move it to the
%    beginning as needed.
%
%    \begin{macrocode}
\def\endmathdisplay@fleqn{%
  $\hfil\hskip\@mathmargin\egroup
%    \end{macrocode}
%    We need to save the information about whether box 0 was overfull in
%    a variable, otherwise it will disappear in the next setbox
%    operation. And we couldn't set the equation number box earlier than
%    now, because the body of the equation might have contained a
%    \cs{tag} command (well, it could have been done, but this way
%    we can reuse the tag-handling code from elsewhere).
%    \begin{macrocode}
  \ifnum\badness<\inf@bad \let\too@wide\@ne \else \let\too@wide\z@ \fi
  \ifx\@empty\df@tag
  \else
    \setbox4\hbox{\df@tag
      \ifx\df@label\@empty \else \@xp\ltx@label\@xp{\df@label}\fi
    }%
  \fi
  \csname emdf@%
    \ifx\df@tag\@empty U\else \iftagsleft@ L\else R\fi\fi
  \endcsname
}
%    \end{macrocode}
%
%    For an unnumbered flush-left equation we hope first that the
%    the contents fit within displaywidth. If not we need to fall back
%    on a more complicated reboxing operation.
%    \begin{macrocode}
\def\emdf@U{%
  \restore@hfuzz
  \ifodd\too@wide % not too wide: just need to swap the glue around
    \hbox to\displaywidth{\hskip\@mathmargin\unhbox\z@\unskip}%
  \else % M+B > displaywidth
    \emdf@Ua
  \fi
}
%    \end{macrocode}
%
%    Some notation: $M$ \cs{@mathmargin}, $B$ the width of the equation
%    body, $I$ \cs{displayindent}, $D$ \cs{displaywidth}, $N$ the width
%    of the equation number (aka the tag), $S$ \cs{mintagsep}, $C$
%    \cs{columnwidth}. If $M+B > \mbox{displaywidth}$, and if we assume
%    $M$ contains shrink, then the only solution left is to encroach
%    into the displayindent space.
%    \begin{macrocode}
\def\emdf@Ua{%
  \hbox to\columnwidth{%
    \ifdim\displayindent>\z@
      \hskip\displayindent minus\displayindent
    \fi
    \hskip\@mathmargin \unhbox\z@ \unskip
  }%
  \displayindent\z@ \displaywidth\columnwidth
}
%    \end{macrocode}
%
%    Find out first if the tag fits in ideal position. If so we can just
%    plunk down box 2. Otherwise we need to do something more complicated.
%    \begin{macrocode}
\def\emdf@R{%
  \setbox\tw@\hbox to\displaywidth{%
    \hskip\@mathmargin \unhcopy\z@\unskip\hfil\hskip\mintagsep\copy4
  }%
  \restore@hfuzz
  \ifnum\badness<\inf@bad \box\tw@ \else \emdf@Ra \fi
}
%    \end{macrocode}
%
%    We shift the equation number to line 2 if it does not fit within
%    \cs{displaywidth}. Note that we do not first attempt to let the
%    equation body shift leftward into the \cs{displayindent} space. If
%    that is desired it will have to be done by hand by adding negative
%    space at the beginning of the equation body. I don't expect this to
%    arise very often in practice since most of the time
%    \cs{displayindent} is zero anyway.
%    \begin{macrocode}
\def\emdf@Ra{%
  \skip@\displayindent minus\displayindent
  \displayindent\z@ \displaywidth\columnwidth
  \spread@equation \everycr{}\tabskip\z@skip
  \halign{\hbox to\displaywidth{##}\cr
    \relax
    \ifdim\skip@>\z@ \hskip\skip@ \fi
    \hskip\@mathmargin\unhbox\z@\unskip\hfil\cr
    \noalign{\raise@tag}%
    \hfil\box4 \cr}%
}
%    \end{macrocode}
%
%    Find out first if the tag fits in ideal position. If so we can just
%    plunk down box 2. Otherwise we need to do something more
%    complicated.
%    \begin{macrocode}
\def\emdf@L{%
%    \end{macrocode}
%    Calculate the difference between $M$ and $N+S$. If the latter is
%    greater, we don't want to add any extra glue between the number and
%    the equation body. Otherwise the amount that we want to add is
%    \verb'x minus x' where $x=M-(N+S)$. I.e., the distribution of
%    spaces across the line is $N,S,x minus x,B,hfil$.
%    \begin{macrocode}
  \@tempdima\@mathmargin
  \advance\@tempdima-\wd4 \advance\@tempdima-\mintagsep
  \skip@\@tempdima minus\@tempdima
  \setbox\tw@\hbox to\displaywidth{%
    \copy4\hskip\mintagsep
    \ifdim\skip@>\z@ \hskip\skip@\fi
    \unhcopy\z@\unskip
  }%
  \restore@hfuzz
  \ifnum\badness<\inf@bad \box\tw@ \else \emdf@La \fi
}
%    \end{macrocode}
%
%    If the equation body and equation number will not fit on the same
%    line, we put the number on line 1 and the body on line 2, with the
%    body positioned as for an unnumbered equation.
%    \begin{macrocode}
\def\emdf@La{%
  \spread@equation \everycr{}\tabskip\z@skip
  \halign{\hbox to\displaywidth{##}\cr
    \box4 \hfil \cr
    \noalign{\raise@tag}%
    \hskip\@mathmargin\unhbox\z@\unskip\hfil\cr}%
}
%    \end{macrocode}
%
%    If someone has \verb'\[ \]' nested inside a minipage environment
%    nested inside a numbered equation, the mathdisplay variables that
%    are global will get out of whack unless we take extra care. So we
%    make a stack and push all the variables before entering mathdisplay
%    and pop them afterwards. But we can save a little work by not doing
%    this at the top level, only at inner levels.
%    \begin{macrocode}
\newtoks\mathdisplay@stack
\let\mathdisplay@push\@empty
\def\mathdisplay@@push{%
  \begingroup
  \toks@\@xp{\df@label}\@temptokena\@xp{\df@tag}%
  \toks8\@xp{\alt@tag}%
  \edef\@tempa{%
    \global\if@eqnsw\@nx\@eqnswtrue\else\@nx\@eqnswfalse\fi
    \global\iftag@\@nx\tag@false\else\@nx\tag@true\fi
    \gdef\@nx\df@label{\the\toks@}\gdef\@nx\df@tag{\the\@temptokena}%
    \gdef\@nx\alt@tag{\the\toks8}%
    \global\mathdisplay@stack{\the\mathdisplay@stack}%
  }%
  \global\mathdisplay@stack\@xp{\@tempa}
  \endgroup
}
%    \end{macrocode}
%
%    \begin{macrocode}
\let\mathdisplay@pop\@empty
\def\mathdisplay@@pop{\the\mathdisplay@stack}
%    \end{macrocode}
%
%    \begin{macrocode}
\renewenvironment{equation}{%
  \incr@eqnum
  \mathdisplay@push
  \st@rredfalse \global\@eqnswtrue
  \mathdisplay{equation}%
}{%
  \endmathdisplay{equation}%
  \mathdisplay@pop
  \ignorespacesafterend
}
%    \end{macrocode}
%
%    \begin{macrocode}
\newenvironment{equation*}{%
  \mathdisplay@push
  \st@rredtrue \global\@eqnswfalse
  \mathdisplay{equation*}%
}{%
  \endmathdisplay{equation*}%
  \mathdisplay@pop
  \ignorespacesafterend
}
%    \end{macrocode}
%
%    Note: \latex/ defines the \env{displaymath} environment in
%    terms of \cn{[} and \cn{]}.
%    \begin{macrocode}
\DeclareRobustCommand{\[}{\begin{equation*}}
\DeclareRobustCommand{\]}{\end{equation*}}
%    \end{macrocode}
%
%    The usual \cs{endinput} to ensure that random garbage at the end of
%    the file doesn't get copied by \fn{docstrip}.
%    \begin{macrocode}
\endinput
%    \end{macrocode}
%
% \section{Credits}
%
%    Much of the code for the \pkg{amsmath} package had its orgin in
%    \fn{amstex.tex}, written by Michael Spivak. The initial work of
%    porting \fn{amstex.tex} to \fn{amstex.sty} was done in 1988--1989
%    by Frank Mittelbach and Rainer Sch\"opf. In 1994 David M. Jones
%    added the support for the \opt{fleqn} option and did extensive
%    improvements to the \env{align[at]} family of environments and to
%    the equation number handling in general. Michael Downes at the AMS
%    served as coordinator for the efforts of Mittelbach, Sch\"opf, and
%    Jones, and has contributed various bug fixes and additional
%    refinements over time.
%
%    Versions 1.0 and 1.1 of the package carried the name \pkg{amstex}
%    instead of \pkg{amsmath}, to indicate its origins; the name was
%    changed in 1994 to make it user-oriented rather than
%    history-oriented.
%
\endinput
