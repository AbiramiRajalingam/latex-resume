% \iffalse meta-comment
%
% Copyright (C) 1993-2019
%
% The LaTeX3 Project and any individual authors listed elsewhere
% in this file.
%
% This file is part of the Standard LaTeX `Tools Bundle'.
% -------------------------------------------------------
%
% It may be distributed and/or modified under the
% conditions of the LaTeX Project Public License, either version 1.3c
% of this license or (at your option) any later version.
% The latest version of this license is in
%    https://www.latex-project.org/lppl.txt
% and version 1.3c or later is part of all distributions of LaTeX
% version 2005/12/01 or later.
%
% The list of all files belonging to the LaTeX `Tools Bundle' is
% given in the file `manifest.txt'.
%
% \fi
% \iffalse
%%
%% (C) Copyright (C) 1999-2019 Frank Mittelbach
%% All rights reserved.
%%
%<*dtx>
          \ProvidesFile{trace.dtx}
%</dtx>
%<package>\NeedsTeXFormat{LaTeX2e}
%<package>\ProvidesPackage{trace}
%<driver>\ProvidesFile{trace.drv}
% \fi
%         \ProvidesFile{trace.dtx}
          [2018/10/13 v1.1e trace LaTeX code]
%
% \iffalse
%<*driver>
\documentclass{ltxdoc}
\usepackage{doc}
\setcounter{StandardModuleDepth}{1}

\usepackage{shortvrb}
\MakeShortVerb\|

\DeclareRobustCommand\eTeX{\ensuremath{\varepsilon}-\kern-.125em\TeX}

\begin{document}
\DocInput{trace.dtx}
\end{document}
%</driver>
% \fi
%
%
% \GetFileInfo{trace.dtx}
%
% \title{The \textsf{trace} package\thanks{This file
%         has version number \fileversion, last
%         revised \filedate.}}
% \author{Frank Mittelbach}
% \date{\filedate}
% \MaintainedByLaTeXTeam{tools}
% \maketitle
%
%
%
% \section{Introduction}
%
% When writing new macros one often finds that they do not work as
% expected (at least I do :-). If this happens and one can't
% immediately figure out why there is a problem one has to start doing
% some serious debugging. \TeX{} offers a lot of bells and whistles to
% control what is being traced but often enough I find myself applying
% the crude command |\tracingall| which essentially means ``give me
% whatever tracing information is available''.
%
% In fact I normally use \eTeX{} in such a case, since that \TeX{}
% extension offers me a number of additional tracing possibilities
% which I find extremely helpful. The most important ones are
% |\tracingassigns|, which will show you changes to register values
% and changes to control sequences when they happen, and
% |\tracinggroups|, which will tell you what groups are entered
% or left (very useful if your grouping got out of sync).
%
% So what I really write is
%\begin{verbatim}
%  \tracingassigns=1\tracinggroups=1\tracingall
%\end{verbatim}
% That in itself is already a nuisance (since it is a mouthful) but
% there is a worse catch: when using |\tracingall| you do get a awful
% lot of information and some of it is really useless.
%
% For example, if \LaTeX{} has to load a new font it enters some
% internal routines of NFSS which scan font definition tables etc.
% And 99.9\% of the time you are not at all interested in that part of the
% processing but in the two lines before and the five lines
% after. However, you have to scan through a few hundred lines of
% output to find the lines you need.
%
% Another example is the \texttt{calc} package. A simple statement
% like |\setlength| |\linewidth| |{1cm}| inside your macro will result
% in
%
% \renewcommand\MacroFont{\fontencoding\encodingdefault
%                   \fontfamily\ttdefault
%                   \fontseries\mddefault
%                   \fontshape\updefault
%                   \footnotesize}%
%\begin{verbatim}
%  \setlength ->\protect \setlength
%  {\relax}
%
%  \setlength  ->\calc@assign@skip
%
%  \calc@assign@skip ->\calc@assign@generic \calc@Askip \calc@Bskip
%
%  \calc@assign@generic #1#2#3#4->\let \calc@A #1\let \calc@B #2\expandafter \calc
%  @open \expandafter (#4!\global \calc@A \calc@B \endgroup #3\calc@B
%  #1<-\calc@Askip
%  #2<-\calc@Bskip
%  #3<-\linewidth
%  #4<-1cm
%  {\let}
%  {\let}
%  {\expandafter}
%  {\expandafter}
%
%  \calc@open (->\begingroup \aftergroup \calc@initB \begingroup \aftergroup \calc
%  @initB \calc@pre@scan
%  {\begingroup}
%  {\aftergroup}
%  {\begingroup}
%  {\aftergroup}
%
%  \calc@pre@scan #1->\ifx (#1\expandafter \calc@open \else \ifx \widthof #1\expan
%  dafter \expandafter \expandafter \calc@textsize \else \calc@numeric \fi \fi #1
%  #1<-1
%  {\ifx}
%  {false}
%  {\ifx}
%  {false}
%
%  \calc@numeric ->\afterassignment \calc@post@scan \global \calc@A
%  {\afterassignment}
%  {\global}
%  {\fi}
%  {\fi}
%
%  \calc@post@scan #1->\ifx #1!\let \calc@next \endgroup \else \ifx #1+\let \calc@
%  next \calc@add \else \ifx #1-\let \calc@next \calc@subtract \else \ifx #1*\let
%  \calc@next \calc@multiplyx \else \ifx #1/\let \calc@next \calc@dividex \else \i
%  fx #1)\let \calc@next \calc@close \else \calc@error #1\fi \fi \fi \fi \fi \fi \
%  calc@next
%  #1<-!
%  {\ifx}
%  {true}
%  {\let}
%  {\else}
%  {\endgroup}
%  {restoring \calc@next=undefined}
%
%  \calc@initB ->\calc@B \calc@A
%  {\skip44}
%  {\global}
%  {\endgroup}
%  {restoring \skip44=0.0pt}
%
%  \calc@initB ->\calc@B \calc@A
%  {\skip44}
%  {\dimen27}
%\end{verbatim}
%
% \noindent Do you still remember what I was talking about?
%
% No? We're trying to find a problem in macro code without having to scan
% too many uninteresting lines. To make this possible we have to
% redefine a number of key commands to turn tracing off temporarily in
% the hope that this will reduce the amount of noise during the
% trace. For example, if we change one of the \texttt{calc} internals
% slightly, the above tracing output can be reduced to:
%
%\begin{verbatim}
%  \setlength ->\protect \setlength
%  {\relax}
%
%  \setlength  ->\calc@assign@skip
%
%  \calc@assign@skip ->\calc@assign@generic \calc@Askip \calc@Bskip
%
%  \calc@assign@generic #1#2#3#4->\let \calc@A #1\let \calc@B #2\expandafter \calc
%  @open \expandafter (#4!\global \calc@A \calc@B \endgroup #3\calc@B
%  #1<-\calc@Askip
%  #2<-\calc@Bskip
%  #3<-\linewidth
%  #4<-1cm
%  {\let}
%  {\let}
%  {\expandafter}
%  {\expandafter}
%
%  \calc@open (->\begingroup \conditionally@traceoff \aftergroup \calc@initB \begi
%  ngroup \aftergroup \calc@initB \calc@pre@scan
%  {\begingroup}
%
%  \conditionally@traceoff ->\tracingrestores \z@ \tracingcommands \z@ \tracingpag
%  es \z@ \tracingmacros \z@ \tracingparagraphs \z@
%  {\tracingrestores}
%  {\tracingcommands}
%  {restoring \tracingrestores=1}
%
%  \calc@initB ->\calc@B \calc@A
%  {\skip44}
%  {\dimen27}
%\end{verbatim}
%
% \noindent\label{conttraceoffexample}
% Still a lot of noise but definitely preferable to the
% original case.
%
% I redefined those internals that I found most annoyingly
% noisy. There are probably many others that could be treated in a
% similar fashion, so if you think you found one worth adding please
% drop me a short note.
%
% \[ * \quad * \quad * \]
%
% \DescribeMacro\traceon
% \DescribeMacro\traceoff
% The package defines the two macros |\traceon| and |\traceoff| to
% unconditionally turn tracing on or off, respectively. |\traceon| is
% like |\tracingall| but additionally adds |\tracingassigns| and
% |\tracinggroups| if the \eTeX{} program (in extended mode) is
% used. And |\traceoff| will turn tracing off again, a command which
% is already badly missing in plain \TeX{}, since it is often not
% desirable to restrict the tracing using extra groups in the
% document.
%
% \DescribeMacro\conditionally@traceon
% \DescribeMacro\conditionally@traceoff
% There are also two internal macros that turn tracing on and off, but
% only if the user requested tracing in the first place. These are the
% ones that are used internally within the code below.
%
% Since the package overwrites some internals of other packages you
% should load it as the last package in your preamble using
% |\usepackage{trace}|.
%
% The package offers the option \texttt{logonly} that suppresses
% terminal output during tracing (unless |\tracingall| is used). This
% is useful if the \TeX{} implementation used gets rather slow when
% writing a lot of information to the terminal.
%
% It also offers the option \texttt{full} in which case |\traceon|
% will trace all parts of the code, i.e., essentially work like
% |\tracingall|.
%
% \section{A sample file}
%
% The following small test file shows the benefits of the
% \texttt{trace} package. If one uncomments the line loading the
% package, the amount of tracing data will be drastically reduced.
% Without the \texttt{trace} package we get 6594 lines in the log
% file; adding the package will reduce this to 1618 lines.
%
%\begin{verbatim}
%  \documentclass{article}
%  \usepackage{calc}
%  %\usepackage{trace} % uncomment to see difference
%
%  \begin{document}
%  \ifx\traceon\undefined \tracingall \else \traceon \fi
%
%  \setlength\linewidth{1cm}
%
%  $foo=\bar a$
%
%  \small \texttt{\$}  \stop
%\end{verbatim}
%
% \StopEventually{}
%
% \section{Implementation}
%
% \renewcommand\MacroFont{\fontencoding\encodingdefault
%                   \fontfamily\ttdefault
%                   \fontseries\mddefault
%                   \fontshape\updefault
%                   \small}
%
% This package is for use with \LaTeX{} (though something similar
% could be produced for other formats).
%    \begin{macrocode}
%<*package>
\NeedsTeXFormat{LaTeX2e}[1998/12/01]
%    \end{macrocode}
%
%    The package has a option that suppresses tracing on the
%    terminal, i.e., if used will not set |\tracingonline| to
%    one. This has been added in version 1.1a since some \TeX{}
%    implementations get rather slow when outputting to a terminal.
%    \begin{macrocode}
\DeclareOption{logonly}
   {\let\tracingonline@p\z@}
%    \end{macrocode}
%    The default is showing the tracing information on the terminal.
%    \begin{macrocode}
\let\tracingonline@p\@ne
%    \end{macrocode}
%
%    If the option |full| is selected then all code should be traced,
%    i.e., the commands |\conditionally@traceoff| and
%    |\conditionally@traceon| should do nothing. We set them to
%    |\@empty| not |\relax| since the latter might produce a math ord
%    in certain circumstances.  We also have to make sure that
%    |\traceon| (as defined further down) is not redefining
%    |\conditionally@traceoff| again. To make this all work these
%    redefinitions have to wait until the end of the package.
%  \changes{v1.1c}{2003/04/30}{Option ``full'' added}
%    \begin{macrocode}
\DeclareOption{full}
    {\AtEndOfPackage{\let\conditionally@traceoff\@empty
                     \let\conditionally@traceon\@empty
                     \let\traceon\tr@ce@n
    }}
%    \end{macrocode}
%
%    \begin{macrocode}
\ProcessOptions\relax
%    \end{macrocode}
%
%
%
% \begin{macro}{\if@tracing}
%    We need a switch to determine if we want any tracing at
%    all. Otherwise, if we use |\traceoff|\ldots|\traceon|
%    internally, we would unconditionally turn on tracing even when no
%    tracing was asked for in the first place.
%    \begin{macrocode}
\newif\if@tracing
%    \end{macrocode}
% \end{macro}
%
% \begin{macro}{\traceon}
%    This macro ensures that |\conditionally@traceoff| is actually
%    turning off switches (since |\tracinall| might have disabled it)
%    and then calls |\tr@ce@n| to setup tracing.
%    \begin{macrocode}
\def\traceon{\let\conditionally@traceoff\unconditionally@traceoff
             \tr@ce@n}
%    \end{macrocode}
% \end{macro}

% \begin{macro}{\tr@ce@n}
%  \changes{v1.1c}{2003/04/30}{Macro added}
% \begin{macro}{\conditionally@traceoff}
%    As stated in the introduction, the amount of tracing being done
%    should depend on the formatter we use. Initially first test if we are
%    running with \eTeX{} in extended mode. In the latter case the command
%    |\tracinggroups| is defined.
%    But for a number of years now \LaTeX{} only works with \eTeX{} so
%    we drop that part of the code. For now I leave it in the file
%    together with its documentation, but commented out.
%    \begin{macrocode}
%\ifx\tracinggroups\undefined
%    \end{macrocode}
%
%    If we are using standard \TeX{} then |\tr@ce@n| is more or less
%    another name for |\tracingall|. The only differences are that we
%    set the above |@tracing| switch to true and reorder the
%    assignments within it somewhat so that it will output no tracing
%    information about itself. In contrast, |\tracingall| itself produces
%
%\begingroup
%    \renewcommand\MacroFont{\fontencoding\encodingdefault
%                   \fontfamily\ttdefault
%                   \fontseries\mddefault
%                   \fontshape\updefault
%                   \footnotesize}%
%\begin{verbatim}
%  {vertical mode: \tracingstats}
%  {\tracingpages}
%  {\tracinglostchars}
%  {\tracingmacros}
%  {\tracingparagraphs}
%  {\tracingrestores}
%  {\errorcontextlines}
%
%  \showoutput ->\tracingoutput \@ne \showboxbreadth \maxdimen \showboxdepth \maxd
%  imen \errorstopmode \showoverfull
%  {\tracingoutput}
%  {\showboxbreadth}
%  {\showboxdepth}
%  {\errorstopmode}
%
%  \showoverfull ->\tracingonline \@ne
%  {\tracingonline}
%\end{verbatim}
%\endgroup
%
%    \noindent Which is quite a lot given that none of it is of any
%    help to the task at hand. In contrast |\tr@ce@n| will produce
%    nothing whatsoever since the noise generating switches are set at
%    the very end.
%    \begin{macrocode}
%  \def\tr@ce@n{%
%    \end{macrocode}
%    We start by setting the |@tracing| switch to signal that tracing
%    is asked for. This is then followed by setting the various
%    tracing primitives of \TeX.
%    \begin{macrocode}
%    \@tracingtrue
%    \tracingstats\tw@
%    \tracingpages\@ne
%    \tracinglostchars\@ne
%    \tracingparagraphs\@ne
%    \errorcontextlines\maxdimen
%    \tracingoutput\@ne
%    \showboxbreadth\maxdimen
%    \showboxdepth\maxdimen
%    \errorstopmode
%    \tracingmacros\tw@
%    \tracingrestores\@ne
%    \tracingcommands\tw@
%    \end{macrocode}
%    The setting of |\tracingonline| depends on the option
%    \texttt{logonly}:
%    \begin{macrocode}
%    \tracingonline\tracingonline@p
%  }
%    \end{macrocode}
%
%    Now what should |\conditionally@traceoff| do in this case? Should
%    it revert all settings changed by |\tr@ce@n|? It should not, since
%    our goal is to shorten the trace output, thus setting all of the
%    uninteresting values back makes the output unnecessarily
%    longer. Therefore we restrict ourselves to those |\tracing...|
%    internals that really contribute to listings like the above.
%
%    And one additional point is worth mentioning. The order in which
%    we turn the tracing internals off has effects on the output we
%    see. So what needs to be turned off first? Either
%    |\tracingrestores| or |\tracingcommands|; it makes no difference
%    which, as long as they both come first. This is because those two
%    are the only tracing switches that produce output while tracing
%    the command |\conditionally@traceoff| itself (see example on
%    page~\pageref{conttraceoffexample}).
%
%    In principle we would need to test the |@tracing| switch to see
%    if there is anything to turn off; after all, this is the
%    conditional trace off. However this would lead to
%    extra output if we are currently tracing so we skip the test and
%    instead accept that in case we are not doing any tracing we
%    unnecessarily set the tracing primitives back to zero (i.e., the
%    value they already have).
%  \changes{v1.1c}{2003/04/30}{Turn off \cs{tracingoutput}}
%    \begin{macrocode}
%  \def\conditionally@traceoff{%
%    \tracingrestores\z@
%    \tracingcommands\z@
%    \tracingpages\z@
%    \tracingmacros\z@
%    \tracingparagraphs\z@
%    \tracingoutput\z@
%    \showboxbreadth\m@ne
%    \showboxdepth\m@ne
%    \end{macrocode}
%
%    As remarked above there are more tracing switches set by
%    |\tr@ce@n|, however there is no point in resetting
%    |\tracinglostchars| so we leave it alone.
%  \changes{v1.1c}{2003/04/30}{Reset \cs{tracingstats} to one}
%    \begin{macrocode}
%   \tracingstats\@ne
%% \tracinglostchars\z@
%    \end{macrocode}
%    Since this is the command that only conditionally turns off
%    tracing we do not touch the |@tracing| switch. This way a
%    |\conditionally@traceon| will be able to turn the tracing on
%    again.
%    \begin{macrocode}
%  }
%    \end{macrocode}
%
%    That covers the case for the standard \TeX{} program. If
%    |\tracingsgroups| was defined we assume that we are running with
%    \eTeX{} in extended mode.
%    \begin{macrocode}
%\else
%    \end{macrocode}
%
%    In that case |\tr@ce@n| does more than |\tracingall|: it also
%    turns on tracing of assignments and tracing of
%    grouping.\footnote{These are my personal preference settings;
%    \eTeX{} does in fact offer some more tracing switches and perhaps
%    one or or more of them should be added here as well.}
%    To keep tracing at a minimum |\tracingassigns| should be turned
%    on last (in fact like before we disassemble |\tracingall|
%    and reorder it partially).
%    \begin{macrocode}
  \def\tr@ce@n{%
    \@tracingtrue
    \tracingstats\tw@
    \tracingpages\@ne
    \tracinglostchars\@ne
    \tracingparagraphs\@ne
    \errorcontextlines\maxdimen
%    \end{macrocode}
%    We only change |\tracingoutput| if it hasn't already been enabled by
%    |\showoutput|. If that's not the case, we set it to 2 so that we
%    can distingush the two cases.    
%  \changes{v1.1e}{2018/10/13}{Only reset \cs{tracingoutput} if not
%    set by \cs{showoutput} earlier}
%    \begin{macrocode}
    \ifnum\tracingoutput=\@ne
    \else
      \tracingoutput\tw@
      \showboxbreadth\maxdimen
      \showboxdepth\maxdimen
    \fi
%    \end{macrocode}
%    
%    \begin{macrocode}
    \errorstopmode
    \tracingmacros\tw@
    \tracinggroups\@ne
    \tracingrestores\@ne
    \tracingcommands\tw@
    \tracingassigns\@ne
    \tracingonline\tracingonline@p
  }
%    \end{macrocode}
%
%    When turning tracing off again we now also have to turn off those
%    additional tracing switches. But what to turn off in what order?
%    Since |\tracingassigns| is quite noisy (two lines of output per
%    assignment) and the whole command expansion consists of
%    assignments, we had best start with this switch and follow it again
%    by |\tracingrestores| and |\tracingcommands|. The rest can be in
%    any order, it doesn't make a difference.
%
%    With the same reasoning as before we omit testing for the
%    |@tracing| switch and always set the primitives back to zero.
%  \changes{v1.1c}{2003/04/30}{Turn off \cs{tracingoutput}}
%  \changes{v1.1c}{2003/04/30}{Reset \cs{tracingstats} to one}
%    \begin{macrocode}
  \def\conditionally@traceoff{%
    \tracingassigns\z@
    \tracingrestores\z@
    \tracingcommands\z@
    \tracingpages\z@
    \tracingmacros\z@
%    \end{macrocode}
%    If |\tracingoutput| is 2 it was set above, if it is 1 it was set
%    by |\showoutput| and we leave it alone and if it is 0 there is
%    nothing to do as well.
%  \changes{v1.1e}{2018/10/13}{Only reset \cs{tracingoutput} if not
%    set by \cs{showoutput} earlier}
%    \begin{macrocode}
    \ifnum\tracingoutput=\tw@
      \tracingoutput\z@
      \showboxbreadth\m@ne
      \showboxdepth\m@ne
    \fi
%    \end{macrocode}
%    
%    \begin{macrocode}
    \tracingstats\@ne
    \tracingparagraphs\z@
    \tracinggroups\z@
  }
%    \end{macrocode}
%
%    This concludes the part that depends on the formatter being
%    used.
%    \begin{macrocode}
%\fi
%    \end{macrocode}
% \end{macro}
% \end{macro}
%
%
% \begin{macro}{\unconditionally@traceoff}
%  \changes{v1.1c}{2003/04/30}{Macro added}
%    A saved version of whatever |\conditionally@traceoff| was defined
%    to be. We need this since the latter might get disabled by
%    |\tracingall| or by the \texttt{full} option.
%    \begin{macrocode}
\let\unconditionally@traceoff\conditionally@traceoff
%    \end{macrocode}
% \end{macro}
%
% \begin{macro}{\tracingall}
%  \changes{v1.1c}{2003/04/30}{Macro added}
%    We redefine |\tracingall| to trace the same stuff than |\tr@ce@n| (i.e.,
%    more when \eTeX{} is being used) and ensure that everything gets
%    traced by disabling |\conditionally@traceoff|. And, of course,
%    |\tracingall| should always report on the terminal.
%    \begin{macrocode}
\def\tracingall{\let\conditionally@traceoff\@empty
  \let\tracingonline@p\@ne
  \tr@ce@n
}
%    \end{macrocode}
% \end{macro}
%
%
% \begin{macro}{\traceoff}
% \begin{macro}{\conditionally@traceon}
%    Above we have defined |\conditionally@traceoff| and |\traceon| so
%    now we have to define their counterparts.
%
%    To stop tracing unconditionally we call
%    |\unconditionally@traceoff| and then reset the |@tracing| switch
%    to false.
%    \begin{macrocode}
\def\traceoff{\unconditionally@traceoff \@tracingfalse}
%    \end{macrocode}
%
%    Now the |\conditionally@traceon| command will look at the
%    |@tracing| switch and if it is true it will call |\traceon| to
%    restart tracing (note that the latter command unnecessarily sets
%    the switch to true as well). The reason for the |\expandafter| is
%    to get rid of the |\fi| primitive which would otherwise show up in
%    the tracing output (and perhaps puzzle somebody).
%    \begin{macrocode}
\def\conditionally@traceon{\if@tracing \expandafter \traceon \fi}
%    \end{macrocode}
% \end{macro}
% \end{macro}
%
%
% The rest of the package now consists of redefinitions of certain
% commands to make use of |\conditionally@traceoff|.
%
% \subsection{Taming \texttt{calc}}
%
% \begin{macro}{\calc@open}
%    Near the start of parsing a calc expression the macro |\calc@open|
%    is called. Since it already involves a group it is perfectly
%    suitable for our task---we don't even have to restart the tracing as
%    this is done automatically for us.
%    \begin{macrocode}
\def\calc@open({\begingroup
   \conditionally@traceoff
   \aftergroup\calc@initB
   \begingroup\aftergroup\calc@initB
   \calc@pre@scan}
%    \end{macrocode}
% \end{macro}
%
%
% \subsection{Making NFSS less noisy}
%
% \begin{macro}{\define@newfont}
%    Whenever NFSS determines that the font currently asked for is not
%    already loaded, it will start looking through font definition
%    files and then load the font. This results in a very large number
%    of tracing lines which are  not normally of interest (unless there
%    is a bug in that area---something we hope should have been found
%    by now). Again the code already contains its own group so we only
%    have to turn the tracing off.
%    \begin{macrocode}
\def\define@newfont{%
  \begingroup
    \conditionally@traceoff
    \let\typeout\@font@info
    \escapechar\m@ne
    \expandafter\expandafter\expandafter
       \split@name\expandafter\string\font@name\@nil
      \try@load@fontshape % try always
    \expandafter\ifx
       \csname\curr@fontshape\endcsname \relax
      \wrong@fontshape\else
      \extract@font\fi
  \endgroup}
%    \end{macrocode}
% \end{macro}
%
% \begin{macro}{\frozen@everymath}
% \begin{macro}{\frozen@everydisplay}
%    At the beginning of every math formula NFSS will check whether or
%    not the math fonts are properly set up and if not will load
%    whatever is needed. So we surround that part of the code
%    with |\conditionally@traceoff| and |\conditionally@traceon|
%    thereby avoiding all this uninteresting output.
%    \begin{macrocode}
\frozen@everymath =
   {\conditionally@traceoff \check@mathfonts \conditionally@traceon
    \the\everymath}
\frozen@everydisplay =
   {\conditionally@traceoff \check@mathfonts \conditionally@traceon
    \the\everydisplay}
%    \end{macrocode}
% \end{macro}
% \end{macro}
%
%
%
% \section{Checking for italic corrections}
%
% \begin{macro}{\maybe@ic@}
%    When executing |\textit| or its friends, \LaTeX{} looks ahead to
%    determine whether or not to add an italic correction at the
%    end. This involves looping through the |\nocorrlist| which
%    outputs a lot of tracing lines we are normally not interested
%    in. So we disable tracing for this part of the processing.
%  \changes{v1.1d}{2014/04/21}{Use \cs{ifmaybe@ic} not
%           \cs{if@tempswa} as the kernel does (pr/4200)}
%    \begin{macrocode}
\def \maybe@ic@ {%
  \ifdim \fontdimen\@ne\font>\z@
  \else
    \conditionally@traceoff
    \maybe@ictrue
    \expandafter\@tfor\expandafter\reserved@a\expandafter:\expandafter=%
        \nocorrlist
    \do \t@st@ic
    \ifmaybe@ic \sw@slant \fi
    \conditionally@traceon
  \fi
}
%    \end{macrocode}
% \end{macro}
%
%    \begin{macrocode}
%</package>
%    \end{macrocode}
%
%
% \Finale
%
% \endinput
