% \iffalse meta-comment
%
% Copyright (C) 1993-2021
%
% The LaTeX Project and any individual authors listed elsewhere
% in this file.
%
% This file is part of the Standard LaTeX `Tools Bundle'.
% -------------------------------------------------------
%
% It may be distributed and/or modified under the
% conditions of the LaTeX Project Public License, either version 1.3c
% of this license or (at your option) any later version.
% The latest version of this license is in
%    https://www.latex-project.org/lppl.txt
% and version 1.3c or later is part of all distributions of LaTeX
% version 2005/12/01 or later.
%
% The list of all files belonging to the LaTeX `Tools Bundle' is
% given in the file `manifest.txt'.
%
% \fi
%\iffalse
%    This file is built for \LaTeXe, so we make sure an error is
%    generated when it is used with another format
%<*dtx>
\ProvidesFile{layout.dtx}
%</dtx>
%<+package>\NeedsTeXFormat{LaTeX2e}
%<+package>\ProvidesPackage{layout}
%<+driver>\ProvidesFile{layout.drv}
%\ProvidesFile{layout.dtx}
                [2020-07-25 v1.2d Show layout parameters]
%
%    A short driver is provided that can be extracted if necessary by
%    the \textsf{DocStrip} program provided with \LaTeXe.
%    \begin{macrocode}
%<*driver>
\documentclass{ltxdoc}

\usepackage{layout}

\newcommand\Lopt[1]{\textsf{#1}}
\pagestyle{myheadings}
\DisableCrossrefs
\begin{document}
\DocInput{layout.dtx}
\end{document}
%</driver>
%    \end{macrocode}
%\fi
% \changes{v1.1c}{1994/07/13}{Moved Identification code to the front
%    and removed the permanent use of \cs{filedate} and friends}
% \changes{v1.1d}{1994/09/08}{Stored texts in control sequences to
%    allow other languages}
%
% \makeatletter
% \def\allowtofu{^^A
%\def\UTFviii@undefined@err##1{^^A
%  \PackageWarning{inputenc}{Unicode character \expandafter
%                          \UTFviii@splitcsname\string##1\relax
%                          \MessageBreak
%                          not set up for use with LaTeX}^^A
%  \raisebox{.8pt}{\fboxsep1pt\kern.1pt\fbox{$\cdot$}\kern.1pt}^^A
%   }}
% \makeatother
%
% \GetFileInfo{layout.dtx}
% \title{Displaying page layout variables}
% \author{Kent McPherson a.o.\thanks{Converted for \LaTeXe\ by
%        Johannes Braams and modified by Hideo Umeki}}
% \date{\filedate}
%
% \markboth{layout package version \fileversion\space as of \filedate}
%          {layout package version \fileversion\space as of \filedate}
%
% \MaintainedByLaTeXTeam{tools}
% \maketitle
%
% \section{Introduction}
%
%    This \LaTeXe\ package is a reimplementation of
%    \texttt{layout.sty} by Kent McPherson. It defines the command
%    |\layout| which produces an overview of the layout of the current
%    document. The command |\layout*| recomputes the values it uses to
%    produce the overview.
%
%    The figure on the next page shows the output of the |\layout|
%    command for this document.
%
% \begin{figure}[p]
% \layout
% \end{figure}
%
% \StopEventually{}
%
% \section{The implementation}
%
% \changes{v1.1b}{1994/03/23}{removed the \cs{wlog} as
%    \cs{ProvidesPackage} does that now}
% \changes{v1.1d}{1994/09/08}{Added language options dutch and
%    english}
%
%    This package prints a figure to illustrate the layout that is
%    implemented by the document class. In the figure several words
%    appear. They are stored in control sequences to be able to select
%    a different language.
% \changes{v1.1e}{1994/10/29}{The dutch texts should be one word.}
% \changes{v1.1f}{1995/03/14}{Added \cs{notshown}}
% \changes{v1.1f}{1995/03/14}{Added option french}
% \changes{v1.1j}{1995/10/29}{Added the options spanish, brazilian and
%    portuguese}
% \changes{v1.1k}{1995/11/23}{Documentation fixes}
% \changes{v1.2}{1998/04/13}{Redesign of the output by Hideo Umeki}
% \changes{v1.2c}{2000/09/21}{Added option german}
% \changes{v1.2c}{2000/09/25}{Added option for italian by Claudio Beccari}
%    \begin{macrocode}
%<*package>
\DeclareOption{dutch}{%
  \def\Headertext{Kopregel}
  \def\Bodytext{Broodtekst}
  \def\Footertext{Voetregel}
  \def\MarginNotestext{Marge\\Notities}
  \def\oneinchtext{een inch}
  \def\notshown{niet getoond}
  }
\DeclareOption{german}{%
  \def\Headertext{Kopfzeile}
  \def\Bodytext{Haupttext}
  \def\Footertext{Fu{\ss}zeile}
  \def\MarginNotestext{Rand-\\ notizen}
  \def\oneinchtext{ein Zoll}
  \def\notshown{ohne Abbildung}
  }
\DeclareOption{ngerman}{\ExecuteOptions{german}}
\DeclareOption{english}{%
  \def\Headertext{Header}
  \def\Bodytext{Body}
  \def\Footertext{Footer}
  \def\MarginNotestext{Margin\\Notes}
  \def\oneinchtext{one inch}
  \def\notshown{not shown}
  }
\DeclareOption{french}{%
  \def\Headertext{Ent\^{e}te}
  \def\Bodytext{Corps}
  \def\Footertext{Pied de page}
  \def\MarginNotestext{Marge\\Notes}
  \def\oneinchtext{un pouce}
  \def\notshown{non affich\'{e}}
  }
\DeclareOption{francais}{\ExecuteOptions{french}}
\DeclareOption{spanish}{%
  \def\Headertext{Encabezamiento}
  \def\Bodytext{Cuerpo}
  \def\Footertext{Pie de p\'agina}
  \def\MarginNotestext{Notas\\ Marginales}
  \def\oneinchtext{una pulgada}
  \def\notshown{no mostradas}
  }
\DeclareOption{portuguese}{%
  \def\Headertext{Cabe\c{c}alho}
  \def\Bodytext{Corpo}
  \def\Footertext{Rodap\'e}
  \def\MarginNotestext{Notas\\ Marginais}
  \def\oneinchtext{uma polegada}
  \def\notshown{n\~ao mostradas}
  }
\DeclareOption{brazilian}{%
  \def\Headertext{Cabe\c{c}alho}
  \def\Bodytext{Corpo}
  \def\Footertext{Rodap\'e}
  \def\MarginNotestext{Notas\\ Marginais}
  \def\oneinchtext{uma polegada}
  \def\notshown{n\~ao mostradas}
  }
\DeclareOption{italian}{%
  \def\Headertext{Testatina}
  \def\Bodytext{Corpo}
  \def\Footertext{Piedino}
  \def\MarginNotestext{Note\\ Marginali}
  \def\oneinchtext{un pollice}
  \def\notshown{non mostrato}
  }
%    \end{macrocode}
%    
%    \changes{v1.2d}{2020-07-25}{Added option for Japanese (gh353)}
% \begin{allowtofu}
%    \begin{macrocode}
\DeclareOption{japanese}{%
  \def\Headertext{天}
  \def\Bodytext{基本版面}
  \def\Footertext{地}
  \def\MarginNotestext{傍\\注}
  \def\oneinchtext{1インチ}
  \def\notshown{非表示}
  }
%    \end{macrocode}
% \end{allowtofu}
%
%    This package has an option \Lopt{verbose}. Using it will make the
%    command |\layout| type some of the parameters on the terminal.
%    \begin{macrocode}
\DeclareOption{verbose}{\let\LayOuttype\typeout}
\DeclareOption{silent}{\let\LayOuttype\@gobble}
%    \end{macrocode}
%
% \changes{v1.1h}{1995/04/07}{Added the options \Lopt{integers}
%    (default) and \Lopt{reals}}
%
%    The normal behaviour of this package when showing the values of
%    the parameters is to truncate them. However, if you want to see
%    the real parameter values you can use the option \Lopt{reals} to
%    get that effect.
% \changes{v1.1i}{1995/06/25}{\LaTeX\ release 1995/06/01 no longer
%    needs double hashmarks in the argument of \cs{DeclareOption}}
%    \begin{macrocode}
\def\lay@value{}
\DeclareOption{integers}{%
  \renewcommand*{\lay@value}[2]{%
    \expandafter\number\csname #1@#2\endcsname pt}}
\DeclareOption{reals}{%
  \renewcommand*{\lay@value}[2]{\the\csname #2\endcsname}}
%    \end{macrocode}
%
%    The default language is English, the default mode is
%    \Lopt{silent} and the default way of showing parameter values is
%    to use integers.
%    \begin{macrocode}
\ExecuteOptions{english,silent,integers}
\ProcessOptions
%    \end{macrocode}
%
% \begin{macro}{\LayOutbs}
%    Define |\LayOutbs| to produce a backslash. We use a definition
%    which also works with OT1 fonts.
% \changes{v1.2b}{1998/06/21}{Renamed \cs{bs} to avoid possible conflicts
%   with other packages}
%    \begin{macrocode}
\newcommand\LayOutbs{}
\chardef\LayOutbs`\\
%    \end{macrocode}
% \end{macro}
%
% \begin{macro}{\ConvertToCount}
%    This macro stores the value of a \emph{length} register in a
%    \emph{count} register.
%    \begin{macrocode}
\def\ConvertToCount#1#2{%
%    \end{macrocode}
%    First copy the value
%    \begin{macrocode}
  #1=#2
%    \end{macrocode}
%    Then divide it by 65536.
%    \begin{macrocode}
  \divide #1 by 65536}
%    \end{macrocode}
%    The result of this is that the \emph{count} register holds the
%    value of the \emph{length} register in points.
% \end{macro}
%
% \begin{macro}{\SetToHalf}
% \begin{macro}{\SetToQuart}
%    Small macros used in computing positions.
%    \begin{macrocode}
\def\SetToHalf#1#2{#1=#2\relax\divide#1by\tw@}
\def\SetToQuart#1#2{#1=#2\relax\divide#1by4}
%    \end{macrocode}
% \end{macro}
%
% \begin{macro}{\Identify}
%    A small macro used in identifying dimensions.
%    \begin{macrocode}
\def\Identify#1{%
  \put(\PositionX,\PositionY){\circle{20}}
  \put(\PositionX,\PositionY){\makebox(0,0){\tiny #1}}
}
%    \end{macrocode}
% \end{macro}
% \end{macro}
%
% \begin{macro}{\InsideHArrow}
%    This macro is used to produce two horizontal arrows inside a box.
%    The argument gives the width of the box.
% \changes{v1.1c}{1994/07/14}{Added check for a negative arrow length}
%    \begin{macrocode}
\def\InsideHArrow#1{{%
  \ArrowLength = #1
  \divide\ArrowLength by \tw@
  \advance\ArrowLength by -10
  \advance\PositionX by -10
  \ifnum\ArrowLength<\z@
    \put(\PositionX,\PositionY){\vector(1,0){-\ArrowLength}}
    \advance\PositionX by 20
    \put(\PositionX,\PositionY){\vector(-1,0){-\ArrowLength}}
  \else
    \put(\PositionX,\PositionY){\vector(-1,0){\ArrowLength}}
    \advance\PositionX by 20
    \put(\PositionX,\PositionY){\vector(+1,0){\ArrowLength}}
  \fi
}}
%    \end{macrocode}
% \end{macro}
%
% \begin{macro}{\InsideVArrow}
%    This macro is used to produce two vertical arrows inside a box.
%    The argument gives the height of the box.
%    \begin{macrocode}
\def\InsideVArrow#1{{%
  \ArrowLength = #1
  \divide\ArrowLength by \tw@
  \advance\ArrowLength by -10
  \advance\PositionY by -10
  \put(\PositionX,\PositionY){\vector(0,-1){\ArrowLength}}
  \advance\PositionY by 20
  \put(\PositionX,\PositionY){\vector(0,+1){\ArrowLength}}
}}
%    \end{macrocode}
% \end{macro}
%
% \begin{macro}{\OutsideHArrow}
%    This macro is used to produce two horizontal arrows to delimit a
%    length. The first argument is the position for the right arrow,
%    the second argument gives the length and the third specifies the
%    length of the arrows.
%    \begin{macrocode}
\def\OutsideHArrow#1#2#3{{%
  \PositionX = #1
  \advance\PositionX by #3
  \put(\PositionX,\PositionY){\vector(-1,0){#3}}
  \PositionX = #1 \advance\PositionX-#2
  \advance\PositionX by -#3
  \put(\PositionX,\PositionY){\vector(+1,0){#3}}
}}
%    \end{macrocode}
% \end{macro}
%
% \begin{macro}{\OutsideVArrow}
%    This macro is used to produce two vertical arrows to delimit a
%    length. The first argument is the position for the lower arrow,
%    the second argument gives the length and the third and fourth
%    specify the lenghts of the lower and upper arrow.
%    \begin{macrocode}
\def\OutsideVArrow#1#2#3#4{{%
  \PositionY = #1
  \advance\PositionY by -#3
  \put(\PositionX,\PositionY){\vector(0,+1){#3}}
  \PositionY = #1
  \advance\PositionY#2
  \advance\PositionY#4
  \put(\PositionX,\PositionY){\vector(0,-1){#4}}
}}
%    \end{macrocode}
% \end{macro}
%
% \begin{macro}{\Show}
% \changes{v1.1g}{1995/04/03}{Simplified the definition, now show
%    complete value}
% \changes{v1.1h}{1995/04/07}{Use \cs{lay@value} to show the value
%    because of the option which decides which of two ways of doing it
%    should be used}
%
%    Macro used in the table that shows the setting of the parameters.
%    \begin{macrocode}
\def\Show#1#2{\LayOutbs #2 = \lay@value{#1}{#2}}
%    \end{macrocode}
% \end{macro}
%
% \begin{macro}{\Type}
% \changes{v1.1g}{1995/04/03}{Simplified the definition, now show
%    complete value}
% \changes{v1.1h}{1995/04/07}{Use \cs{lay@value} to show the value
%    because of the option which decides which of two ways of doing it
%    should be used}
% \changes{v1.2b}{1998/06/21}{Renamed \cs{type} to \cs{LayOuttype}
%   to avoid possible conflicts with other packages}
%
%    Macro used to show a setting of a parameter on the terminal.
%    \begin{macrocode}
\def\Type#1#2{%
  \LayOuttype{#2 = \lay@value{#1}{#2}}}
%    \end{macrocode}
% \end{macro}
%
% \begin{macro}{\oneinch}
%    A constant, giving the length of an inch in points (approximately)
%    \begin{macrocode}
\newcount\oneinch
\oneinch=72
%    \end{macrocode}
% \end{macro}
%
%    Because the overview of the layout is produced in a figure
%    environment we need to allocate a number of counters that are
%    used to store the values of various dimensions.
%
% \begin{macro}{\cnt@paperwidth}
% \begin{macro}{\cnt@paperheight}
%    The dimensions of the paper
%    \begin{macrocode}
\newcount\cnt@paperwidth
\newcount\cnt@paperheight
\ConvertToCount\cnt@paperwidth\paperwidth
\ConvertToCount\cnt@paperheight\paperheight
%    \end{macrocode}
% \end{macro}
% \end{macro}
%
% \begin{macro}{\cnt@hoffset}
% \begin{macro}{\cnt@voffset}
%    the offsets,
%    \begin{macrocode}
\newcount\cnt@hoffset
\newcount\cnt@voffset
\ConvertToCount\cnt@hoffset\hoffset
\ConvertToCount\cnt@voffset\voffset
%    \end{macrocode}
% \end{macro}
% \end{macro}
%
% \begin{macro}{\cnt@textheight}
% \begin{macro}{\cnt@textwidth}
%    dimensions of the text area,
%    \begin{macrocode}
\newcount\cnt@textheight
\newcount\cnt@textwidth
%    \end{macrocode}
% \end{macro}
% \end{macro}
%
% \begin{macro}{\cnt@topmargin}
% \begin{macro}{\cnt@oddsidemargin}
% \begin{macro}{\cnt@evensidemargin}
%    margins,
%    \begin{macrocode}
\newcount\cnt@topmargin
\newcount\cnt@oddsidemargin
\newcount\cnt@evensidemargin
%    \end{macrocode}
% \end{macro}
% \end{macro}
% \end{macro}
%
% \begin{macro}{\cnt@headheight}
% \begin{macro}{\cnt@headsep}
%    dimensions of the running heads,
%    \begin{macrocode}
\newcount\cnt@headheight
\newcount\cnt@headsep
%    \end{macrocode}
% \end{macro}
% \end{macro}
%
% \begin{macro}{\cnt@marginparsep}
% \begin{macro}{\cnt@marginparwidth}
% \begin{macro}{\cnt@marginparpush}
%    marginal paragraphs,
%    \begin{macrocode}
\newcount\cnt@marginparsep
\newcount\cnt@marginparwidth
\newcount\cnt@marginparpush
%    \end{macrocode}
% \end{macro}
% \end{macro}
% \end{macro}
%
% \begin{macro}{\cnt@footskip}
%    the distance between the running footers and the text,
%    \begin{macrocode}
\newcount\cnt@footskip
%    \end{macrocode}
% \end{macro}
%    and the height of the footers, which is needed here to display a
%    box, but which isn't used by \LaTeX.
% \begin{macro}{\fheight}
%    \begin{macrocode}
\newcount\fheight
\fheight=12
%    \end{macrocode}
% \end{macro}
%
%    Apart from integer representations of the page layout parameters
%    we also need registers to store reference values in.
%
% \begin{macro}{\ref@top}
%    The position of the top of the `printable area' is
%    one inch below the top of the paper by default. The value of
%    |\ref@top| is relative to the lower left corner of the picture
%    environment that will be used.
%
%    \begin{macrocode}
\newcount\ref@top
\ref@top=\cnt@paperheight \advance\ref@top by -\oneinch
%    \end{macrocode}
% \end{macro}
%
% \begin{macro}{\ref@hoffset}
% \begin{macro}{\ref@voffset}
%    For the offsets,
%    \begin{macrocode}
\newcount\ref@hoffset
\newcount\ref@voffset
%    \end{macrocode}
%    The |\hoffset| and |\voffset| values are added to the default
%    offset of one inch.
%    \begin{macrocode}
\ref@hoffset=\cnt@hoffset  \advance\cnt@hoffset by \oneinch
\ref@voffset=\cnt@voffset
%    \end{macrocode}
%
%    |\cnt@voffset| is converted to be relative to the origin of the
%    picture.
%    \begin{macrocode}
\cnt@voffset=\ref@top
\advance\cnt@voffset by -\ref@voffset
%    \end{macrocode}
% \end{macro}
% \end{macro}
%
% \begin{macro}{\ref@head}
%    and the text areas, running heads,
%    \begin{macrocode}
\newcount\ref@head
%    \end{macrocode}
% \end{macro}
%
% \begin{macro}{\ref@body}
%    body of the text
%    \begin{macrocode}
\newcount\ref@body
%    \end{macrocode}
% \end{macro}
%
% \begin{macro}{\ref@foot}
%    and running footers.
%    \begin{macrocode}
\newcount\ref@foot
%    \end{macrocode}
% \end{macro}
%
% \begin{macro}{\ref@margin}
% \begin{macro}{\ref@marginwidth}
% \begin{macro}{\ref@marginpar}
%    These are different for even and odd pages, so they are computed
%    by |\layout|.
%    \begin{macrocode}
\newcount\ref@margin
\newcount\ref@marginwidth
\newcount\ref@marginpar
%    \end{macrocode}
% \end{macro}
% \end{macro}
% \end{macro}
%
%    The following are a number of scratch registers, used in the
%    positioning of the various pices of the picture.
%    \begin{macrocode}
\newcount\Interval
\newcount\ExtraYPos
\newcount\PositionX
\newcount\PositionY
\newcount\ArrowLength
%    \end{macrocode}
%
%  \begin{macro}{\lay@getvalues}
% \changes{v1.1j}{1995/10/30}{Added macro to compute values at later
%    time}
%    All values that might change during the document are computed by
%    calling the macro |\lay@getvalues|. By default this macro is
%    executed at |\begin{document}|.
%    \begin{macrocode}
\def\lay@getvalues{%
  \ConvertToCount\cnt@textheight\textheight
  \ConvertToCount\cnt@textwidth\textwidth
  \ConvertToCount\cnt@topmargin\topmargin
  \ConvertToCount\cnt@oddsidemargin\oddsidemargin
  \ConvertToCount\cnt@evensidemargin\evensidemargin
  \ConvertToCount\cnt@headheight\headheight
  \ConvertToCount\cnt@headsep\headsep
  \ConvertToCount\cnt@marginparsep\marginparsep
  \ConvertToCount\cnt@marginparwidth\marginparwidth
  \ConvertToCount\cnt@marginparpush\marginparpush
  \ConvertToCount\cnt@footskip\footskip
  \ref@head=\ref@top
    \advance\ref@head by -\ref@voffset
    \advance\ref@head by -\cnt@topmargin
    \advance\ref@head by -\cnt@headheight
  \ref@body=\ref@head
    \advance\ref@body by -\cnt@headsep
    \advance\ref@body by -\cnt@textheight
  \ref@foot=\ref@body
    \advance\ref@foot by -\cnt@footskip
  }
\AtBeginDocument{\lay@getvalues}
%    \end{macrocode}
%  \end{macro}
%  \begin{macro}{\computevalues}
%
% \begin{macro}{\layout}
% \begin{macro}{\layout*}
%    The command |\layout| makes the picture and table that display
%    the current settings of the layout parameters.
%
% \changes{v1.1b}{1994/03/23}{Produce two pages in twoside mode}
% \changes{v1.1j}{1995/10/30}{Added \cs{layout*}}
%    \begin{macrocode}
\newcommand\layout{%
  \@ifstar{\lay@getvalues\lay@xlayout}{\lay@xlayout}}
\def\lay@xlayout{%
  \lay@layout
  \if@twoside
    \lay@layout
  \fi}
%    \end{macrocode}
%
% \begin{macro}{\lay@layout}
%    The internal macro |\lay@layout| does all the dirty work.
%    \begin{macrocode}
\newcommand\lay@layout{%
  \thispagestyle{empty}
%    \end{macrocode}
%
%    The actions of |\layout| depend on the pagestyle.
%    \begin{macrocode}
  \if@twoside
    \ifodd\count\z@
%    \end{macrocode}
%
%    Here we deal with an odd page in the twosided case.
%
%    \begin{macrocode}
      \typeout{Two-sided document style, odd page.}
%    \end{macrocode}
%
%    So we compute |\ref@marginwidth|, |\ref@marginpar| and
%    |\ref@margin|.
% \changes{v1.1}{1994/02/23}{Added check for reversemargin}
%    \begin{macrocode}
      \ref@marginwidth=\cnt@oddsidemargin
      \ref@marginpar=\oneinch
      \advance\ref@marginpar by \ref@hoffset
      \advance\ref@marginpar by \cnt@oddsidemargin
      \ref@margin\ref@marginpar
      \if@reversemargin
        \advance\ref@marginpar by -\cnt@marginparsep
        \advance\ref@marginpar by -\cnt@marginparwidth
      \else
        \advance\ref@marginpar by \cnt@textwidth
        \advance\ref@marginpar by \cnt@marginparsep
      \fi
%    \end{macrocode}
%
%    \begin{macrocode}
    \else
%    \end{macrocode}
%
%    Here we deal with an even page in the twosided case.
%
%    \begin{macrocode}
  \typeout{Two-sided document style, even page.}
%    \end{macrocode}
%
%    So we compute |\ref@marginwidth|, |\ref@marginpar| and
%    |\ref@margin|.
% \changes{v1.1}{1994/02/23}{Added check for reversemargin}
%    \begin{macrocode}
      \ref@marginwidth=\cnt@evensidemargin
      \ref@marginpar=\oneinch
      \advance\ref@marginpar by \ref@hoffset
      \advance\ref@marginpar by \cnt@evensidemargin
      \ref@margin\ref@marginpar
      \if@reversemargin
        \advance\ref@marginpar by \cnt@textwidth
        \advance\ref@marginpar by \cnt@marginparsep
      \else
        \advance\ref@marginpar by -\cnt@marginparsep
        \advance\ref@marginpar by -\cnt@marginparwidth
      \fi
%    \end{macrocode}
%
%    \begin{macrocode}
    \fi
  \else
%    \end{macrocode}
%
%    Finally we the case for single sided printing.
%
% \changes{v1.1}{1994/02/23}{Added check for reversemargin}
%    \begin{macrocode}
    \typeout{One-sided document style.}
    \ref@marginwidth=\cnt@oddsidemargin
    \ref@marginpar=\oneinch
    \advance\ref@marginpar by \ref@hoffset
    \advance\ref@marginpar by \cnt@oddsidemargin
    \ref@margin\ref@marginpar
    \if@reversemargin
      \advance\ref@marginpar by -\cnt@marginparsep
      \advance\ref@marginpar by -\cnt@marginparwidth
    \else
      \advance\ref@marginpar by \cnt@textwidth
      \advance\ref@marginpar by \cnt@marginparsep
    \fi
  \fi
%    \end{macrocode}
%
%
%  Now we begin the picture environment; dividing all the lengths by
%  two is done by setting |\unitlength| to \texttt{0.5pt}
%    \begin{macrocode}
  \setlength{\unitlength}{.5pt}
  \begin{picture}(\cnt@paperwidth,\cnt@paperheight)
    \centering
    \thicklines
%    \end{macrocode}
%
%    First we have the pagebox and reference lines,
%
%    \begin{macrocode}
    \put(0,0){\framebox(\cnt@paperwidth,\cnt@paperheight){\mbox{}}}
    \put(0,\cnt@voffset){\dashbox{10}(\cnt@paperwidth,0){\mbox{}}}
    \put(\cnt@hoffset,0){\dashbox{10}(0,\cnt@paperheight){\mbox{}}}
%    \end{macrocode}
%
%    then the header,
%
%    \begin{macrocode}
    \put(\ref@margin,\ref@head){%
      \framebox(\cnt@textwidth,\cnt@headheight)%
        {\footnotesize\Headertext}}
%    \end{macrocode}
%
%    the body of the text area,
%
%    \begin{macrocode}
    \put(\ref@margin,\ref@body){%
      \framebox(\cnt@textwidth,\cnt@textheight){\Bodytext}}
%    \end{macrocode}
%
%     the footer
%
%    \begin{macrocode}
    \put(\ref@margin,\ref@foot){%
      \framebox(\cnt@textwidth,\fheight){\footnotesize\Footertext}}
%    \end{macrocode}
%
%    and the space for marginal notes.
%
%    \begin{macrocode}
    \put(\ref@marginpar,\ref@body){%
      \framebox(\cnt@marginparwidth,\cnt@textheight)%
               {\footnotesize\shortstack{\MarginNotestext}}}
%    \end{macrocode}
%
%    Then we start putting in `arrows' to mark the various parameters.
%    From here we use |\thinlines|.
%    \begin{macrocode}
    \thinlines
%    \end{macrocode}
%
%    |\PositionX| and |\PositionY| will be the coordinates of the center of
%    the arrow displaying |\textwidth|.
%    \begin{macrocode}
    \SetToHalf\PositionX\cnt@textwidth
    \advance\PositionX by \ref@margin
%    \end{macrocode}
%    The arrow should be a bit above the bottom of the `body box'.
%    \begin{macrocode}
    \PositionY = \ref@body
    \advance\PositionY by 50
%    \end{macrocode}
%    An identifying number is put here, in a circle.
%    \begin{macrocode}
    \Identify{8}
%    \end{macrocode}
%    Then the arrow is drawn.
%    \begin{macrocode}
    \InsideHArrow\cnt@textwidth
%    \end{macrocode}
%
%    Now the |\textheight|
%    \begin{macrocode}
    \SetToHalf\PositionY\cnt@textheight
    \advance\PositionY by \ref@body
%    \end{macrocode}
%
%    The x-position of the arrow is at $4/5$ of the width of the `body
%    box'.
%    \begin{macrocode}
    \PositionX = \cnt@textwidth
    \divide\PositionX by 5
    \multiply \PositionX by 4
    \advance\PositionX by \ref@margin
%    \end{macrocode}
%
%    An identifying number is put here, in a circle.
%    \begin{macrocode}
    \Identify{7}
    \InsideVArrow\cnt@textheight
%    \end{macrocode}
%
%
%    The |\hoffset|,
% \changes{v1.2}{1998/04/13}{\cs{PositionY} for label 1 is
%    fixed at 50}
%    \begin{macrocode}
    \PositionY = 50
    \SetToHalf\PositionX\cnt@hoffset
    \Identify{1}
    \InsideHArrow\cnt@hoffset
%    \end{macrocode}
%
%
%    The width of the margin.
% \changes{v1.2}{1998/04/13}{The direction of the arrows should be
%    switched by the sign of \cs{oddsidemargin}}
%    \begin{macrocode}
    \SetToQuart\PositionY\cnt@textheight
    \advance\PositionY by \ref@body
    \ifnum\ref@marginwidth > 0
      \OutsideHArrow\ref@margin\ref@marginwidth{20}
      \PositionX = \cnt@hoffset
    \else
      \OutsideHArrow\cnt@hoffset{-\ref@marginwidth}{20}
      \PositionX = \ref@margin
    \fi
    \advance\PositionX by -30
    \Identify{3}
%    \end{macrocode}
%
%    the |\marginparwidth|,
%    \begin{macrocode}
    \SetToQuart\PositionY\cnt@textheight
    \advance\PositionY by \ref@body
%    \end{macrocode}
%    This arrow has to be bit below the one for the |\oddsidemargin|
%    or\\ |\evensidemargin|.
%    \begin{macrocode}
    \advance\PositionY by 30
    \SetToHalf\PositionX\cnt@marginparwidth
    \advance\PositionX by \ref@marginpar
    \Identify{10}
    \InsideHArrow\cnt@marginparwidth
%    \end{macrocode}
%
%
%    The |\marginparsep|, this depends on single or double sided
%    printing.
%    \begin{macrocode}
    \advance\PositionY by 30
    \if@twoside
%    \end{macrocode}
%
%    Twosided mode, reversemargin;
% \changes{v1.1b}{1994/03/23}{\cs{OutSideHArrow} should be
%    \cs{OutsideHArrow}}
% \changes{v1.2}{1998/04/13}{Added check for reversemargin}
%    \begin{macrocode}
      \if@reversemargin
        \ifodd\count\z@
          \OutsideHArrow\ref@margin\cnt@marginparsep{20}
          \PositionX = \ref@margin
        \else
          \OutsideHArrow\ref@marginpar\cnt@marginparsep{20}
          \PositionX = \ref@marginpar
        \fi
      \else
%    \end{macrocode}
%    Not reversemargin;
%    \begin{macrocode}
        \ifodd\count\z@
          \OutsideHArrow\ref@marginpar\cnt@marginparsep{20}
          \PositionX = \ref@marginpar
        \else
          \OutsideHArrow\ref@margin\cnt@marginparsep{20}
          \PositionX = \ref@margin
        \fi
      \fi
    \else
%    \end{macrocode}
%
%    Single sided mode.
% \changes{v1.2}{1998/04/13}{Added check for reversemargin}
%    \begin{macrocode}
      \if@reversemargin
        \OutsideHArrow\ref@margin\cnt@marginparsep{20}
        \PositionX = \ref@margin
      \else
        \OutsideHArrow\ref@marginpar\cnt@marginparsep{20}
        \PositionX = \ref@marginpar
      \fi
    \fi
%    \end{macrocode}
% \changes{v1.2}{1998/04/13}{The \cs{PositionX} of the label 9 is
%    changed to the left side of the arrows}
%    \begin{macrocode}
    \advance\PositionX by -\cnt@marginparsep
    \advance\PositionX by -30
    \Identify{9}
%    \end{macrocode}
%
%    Identify the |\footskip|. The arrow will be located on $1/8$th of
%    the |\textwidth|.
% \changes{v1.2}{1998/04/13}{The \cs{PositionY} of the label 11 is
%    changed to the upper side of the arrows}
%    \begin{macrocode}
    \PositionX = \cnt@textwidth
    \divide\PositionX by 8
    \advance\PositionX by \ref@margin
    \OutsideVArrow\ref@foot\cnt@footskip{20}{20}
    \PositionY = \ref@foot
    \advance\PositionY by \cnt@footskip
    \advance\PositionY by 30
    \Identify{11}
%    \end{macrocode}
%
%    Identify the |\voffset|. The arrow will be located a bit to the
%    left of the edge of the paper.
%    \begin{macrocode}
    \PositionX = \cnt@paperwidth
    \advance\PositionX by -50
    \PositionY = \cnt@paperheight
    \ExtraYPos = \PositionY
    \advance\ExtraYPos by -\cnt@voffset
    \advance\PositionY by \cnt@voffset
    \divide\PositionY by \tw@
    \Identify{2}
    \InsideVArrow\ExtraYPos
%    \end{macrocode}
%
%    Identify |\topmargin|, |\headheight| and |\headsep|.
%
%    The arrows will be located on $1/8$th of the |\textwidth|, with
%    intervals of the same size, stored in |\Interval|.
%    \begin{macrocode}
    \Interval = \cnt@textwidth
    \divide\Interval by 8
    \PositionX = \ref@margin
    \advance\PositionX by \Interval
%    \end{macrocode}
%    First the |\topmargin|. If |\topmargin| has a positive value, the
%    arrow is upward. Otherwise, it is downward. The number label is
%    always placed at the base of the arrow.
% \changes{v1.2}{1998/04/13}{The direction of the arrows should be
%    switched by the sign of \cs{topmargin}}
%    \begin{macrocode}
    \ifnum\cnt@topmargin > \z@
      \ExtraYPos = \ref@head
      \advance\ExtraYPos\cnt@headheight
      \OutsideVArrow\ExtraYPos\cnt@topmargin{20}{20}
      \PositionY = \ExtraYPos
      \advance\PositionY by \cnt@topmargin
    \else
      \ExtraYPos = \cnt@voffset
      \OutsideVArrow\ExtraYPos{-\cnt@topmargin}{20}{20}
      \PositionY = \ExtraYPos
      \advance\PositionY by -\cnt@topmargin
    \fi
    \advance\PositionY by 30
    \Identify{4}
    \advance\PositionX by \Interval
%    \end{macrocode}
%    Then the |\headheight|
% \changes{v1.2}{1998/04/13}{The \cs{PositionY} of the label 5 is
%    fixed}
%    \begin{macrocode}
    \OutsideVArrow\ref@head\cnt@headheight{20}{20}
    \PositionY = \ref@head
    \advance\PositionY by \cnt@headheight
    \advance\PositionY by 30
    \Identify{5}
    \advance\PositionX by \Interval
%    \end{macrocode}
%    and finally the |\headsep|
% \changes{v1.2}{1998/04/13}{The \cs{PositionY} of the label 6 is
%    fixed}
%    \begin{macrocode}
    \ExtraYPos=\ref@body
    \advance\ExtraYPos\cnt@textheight
    \OutsideVArrow\ExtraYPos\cnt@headsep{20}{20}
    \PositionY = \ref@body
    \advance\PositionY by \cnt@textheight
    \advance\PositionY by -30
    \Identify{6}
%    \end{macrocode}
%
%    Here we can end the picture environment and insert a little
%    space.
%    \begin{macrocode}
  \end{picture}

  \medskip
%    \end{macrocode}
%
%    Below the picture we put a table to show the actual values of the
%    parameters.  Note that fractional points are truncated, i.e.,
%    \texttt{72.27pt} is displayed as \texttt{72pt}
%
%    The table is typeset inside a box with a depth of 0 to always
%    keep it on the same page as the picture.
% \changes{v1.1b}{1994/03/23}{Showing oddside and evenside margins was
%    defective}
% \changes{v1.1c}{1994/07/14}{in compatibility mode \cs{footnotesize}
%    calls \cs{normalfont}; therefore we need to switch to a tt font
%    later}
% \changes{v1.1f}{1995/03/14}{introduced \cs{notshown}}
%    \begin{macrocode}
  \vtop to 0pt{%
    \@minipagerestore\footnotesize\ttfamily
    \begin{tabular}{@{}rl@{\hspace{20pt}}rl}
      1 & \oneinchtext\ + \LayOutbs\texttt{hoffset}
        & 2 & \oneinchtext\ + \LayOutbs\texttt{voffset} \\
      3 & \if@twoside
            \ifodd\count\z@ \Show{cnt}{oddsidemargin}
            \else \Show{cnt}{evensidemargin}
            \fi
          \else
            \Show{cnt}{oddsidemargin}
          \fi                    & 4 & \Show{cnt}{topmargin} \\
      5 & \Show{cnt}{headheight} & 6 & \Show{cnt}{headsep} \\
      7 & \Show{cnt}{textheight} & 8 & \Show{cnt}{textwidth} \\
      9 & \Show{cnt}{marginparsep}&10& \Show{cnt}{marginparwidth} \\
      11& \Show{cnt}{footskip}   &   & \Show{cnt}{marginparpush}
       \rlap{(\notshown)}\\
        & \Show{ref}{hoffset}    &   & \Show{ref}{voffset} \\
        & \Show{cnt}{paperwidth} &   & \Show{cnt}{paperheight} \\

  \end{tabular}\vss}
%    \end{macrocode}
%    When the option \Lopt{verbose} was used the following lines will
%    show dimensions on the terminal.
%    \begin{macrocode}
  \Type{ref}{hoffset}
  \Type{ref}{voffset}
  \Type{cnt}{textheight}
  \Type{cnt}{textwidth}
%    \end{macrocode}
%    Finally we start a new page.
%    \begin{macrocode}
  \newpage
}
%</package>
%    \end{macrocode}
% \end{macro}
% \end{macro}
% \end{macro}
% \end{macro}
%
% \Finale
\endinput
